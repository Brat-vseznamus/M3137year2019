\documentclass[12pt, a4paper]{article}

\usepackage{lastpage}
\usepackage{mathtools}
\usepackage{xltxtra}
\usepackage{libertine}
\usepackage{amsmath}
\usepackage{amsthm}
\usepackage{amsfonts}
\usepackage{amssymb}
\usepackage{enumitem}
\usepackage{xcolor}
\usepackage[left=1.5cm, right=1.5cm, top=2cm, bottom=2cm, bindingoffset=0cm, headheight=15pt]{geometry}
\usepackage{fancyhdr}
\usepackage[russian]{babel}
% \usepackage[utf8]{inputenc}
\usepackage{catchfilebetweentags}
\usepackage{accents}
\usepackage{calc}
\usepackage{etoolbox}
\usepackage{mathrsfs}
\usepackage{wrapfig}

\providetoggle{useproofs}
\settoggle{useproofs}{false}

\pagestyle{fancy}
\lfoot{M3137y2019}
\rhead{\thepage\ из \pageref{LastPage}}

\newcommand{\R}{\mathbb{R}}
\newcommand{\Q}{\mathbb{Q}}
\newcommand{\C}{\mathbb{C}}
\newcommand{\Z}{\mathbb{Z}}
\newcommand{\B}{\mathbb{B}}
\newcommand{\N}{\mathbb{N}}

\newcommand{\const}{\text{const}}

\newcommand{\teormin}{\textcolor{red}{!}\ }

\DeclareMathOperator*{\xor}{\oplus}
\DeclareMathOperator*{\equ}{\sim}
\DeclareMathOperator{\Ln}{\text{Ln}}
\DeclareMathOperator{\sign}{\text{sign}}
\DeclareMathOperator{\Sym}{\text{Sym}}
\DeclareMathOperator{\Asym}{\text{Asym}}
% \DeclareMathOperator{\sh}{\text{sh}}
% \DeclareMathOperator{\tg}{\text{tg}}
% \DeclareMathOperator{\arctg}{\text{arctg}}
% \DeclareMathOperator{\ch}{\text{ch}}

\DeclarePairedDelimiter{\ceil}{\lceil}{\rceil}
\DeclarePairedDelimiter{\abs}{\left\lvert}{\right\rvert}

\setmainfont{Linux Libertine}

\theoremstyle{plain}
\newtheorem{axiom}{Аксиома}
\newtheorem{lemma}{Лемма}

\theoremstyle{remark}
\newtheorem*{remark}{Примечание}
\newtheorem*{exercise}{Упражнение}
\newtheorem*{consequence}{Следствие}
\newtheorem*{example}{Пример}
\newtheorem*{observation}{Наблюдение}

\theoremstyle{definition}
\newtheorem{theorem}{Теорема}
\newtheorem*{definition}{Определение}
\newtheorem*{obozn}{Обозначение}

\setlength{\parindent}{0pt}

\newcommand{\dbltilde}[1]{\accentset{\approx}{#1}}
\newcommand{\intt}{\int\!}

% magical thing that fixes paragraphs
\makeatletter
\patchcmd{\CatchFBT@Fin@l}{\endlinechar\m@ne}{}
  {}{\typeout{Unsuccessful patch!}}
\makeatother

\newcommand{\get}[2]{
    \ExecuteMetaData[#1]{#2}
}

\newcommand{\getproof}[2]{
    \iftoggle{useproofs}{\ExecuteMetaData[#1]{#2proof}}{}
}

\newcommand{\getwithproof}[2]{
    \get{#1}{#2}
    \getproof{#1}{#2}
}

\newcommand{\import}[3]{
    \subsection{#1}
    \getwithproof{#2}{#3}
}

\newcommand{\given}[1]{
    Дано выше. (\ref{#1}, стр. \pageref{#1})
}

\renewcommand{\ker}{\text{Ker }}
\newcommand{\im}{\text{Im }}
\newcommand{\grad}{\text{grad}}

\lhead{Матстат \textit{(практика)}}
\cfoot{}
\rfoot{15.9.2021}

\begin{document}

\section*{Метод моментов}

\begin{exercise}
    Из теории известно, что величина \(X\) имеет показательное распределение (\(X \in E_\alpha\)), \(\overline{X} = 2.54\). Дать оценку параметра \(\alpha\) распределения. Будет ли эта оценка смещённой? Если да, то в какую сторону.
\end{exercise}
\begin{solution}
    \(\E X = \alpha^{ -1}\). \((\alpha^*)^{ - 1} = \overline{X} = 2.54 \Rightarrow \alpha^* \approx 0.394\).

    \(\overline{X}\) --- несмещённая оценка. Тогда \(\frac{1}{\overline{X}}\) очевидно смещённая:
    \[\E \alpha^* = \E \frac{1}{\overline{X}} \symref{йенсена}{\geq} \frac{1}{\E \overline{X}} = \frac{1}{\E X} = \alpha\]
    \blfootnote{\eqref{йенсена}: Неравенство Йенсена, \(f\) выпуклая вниз.}

    Таким образом, систематическая ошибка есть, смещение вверх.
\end{solution}

\begin{exercise}
    Из теории известно, что \(X \in \Gamma_{\alpha, \lambda}\), по статданным \(\overline{X} = 5.4, \overline{X^2} = 32.25\). Найти оценки параметров \(\alpha\) и \(\lambda\).
\end{exercise}
\begin{solution}
    \[\begin{cases}
            \E \overline{X} = \frac{\lambda}{\alpha} = \overline{X} \\
            \D \overline{X} = \frac{\lambda}{\alpha^2} = \overline{X^2} - (\overline{X})^2
        \end{cases}\]
    \[\begin{cases}
            \alpha^* = \frac{\overline{X}}{\overline{X^2} - (\overline{X})^2} \approx 1.75 \\
            \lambda^* = \overline{X} \cdot \alpha^* \approx 9.44
        \end{cases}\]
\end{solution}

\end{document}
