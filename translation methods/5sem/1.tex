\chapter{2 сентября}

\section{Введение}

Этот курс --- про парсеры. Рассмотрим их работу в общем случае.

\begin{enumerate}
    \item На вход подается строка.
    \item Строка разбивается на неделимые блоки \textit{(лексемы или токены)} лексическим анализом.
    \item Последовательность токенов с учетом синтаксиса языка переводится в дерево разбора путем синтаксического разбора \textit{(парсинга)}.
    \item Дерево разбора не есть самоцель, дерево переводится с учетом семантики языка в искомый результат.
\end{enumerate}

Адепты architecture-driven подхода могут захотеть разделить семантику и синтаксис, однако это проблематично. Рассмотрим арифметические выражения как пример.

Токены арифметических выражений это \( + , \cdot, (, ), n\), где \(n\) --- число. Синтаксис задается следующей контекстно-свободной грамматикой:
\begin{itemize}
    \item \(E \to n\)
    \item \(E \to (E)\)
    \item \(E \to E + E\)
    \item \(E \to E \cdot E\)
\end{itemize}

Однако, эта грамматика не однозначна, и выражение \(2 + 2 \cdot 2\) можно разобрать по-разному, из-за чего невозможно навесить семантику. Таким образом, синтаксис нужно задавать с учетом семантики:
\begin{itemize}
    \item \(E \to T\)
    \item \(E \to T + E\)
    \item \(T \to F\)
    \item \(T \to F \cdot T\)
    \item \(F \to n\)
    \item \(F \to (E)\)
\end{itemize}

Но с такой грамматикой операции правоассоциативные и семантику не получится навесить с добавлением вычитания. В правильной грамматике нужно переставить местами аргументы второго правила.

Рассмотрим, как мы будем писать калькулятор арифметических выражений по дереву разбора. Наивный подход --- обойти дерево DFS-ом и рассматривать детей вершины, в которой мы находимся. Однако, таким образом информация о синтаксисе описывается в двух сущностях --- в парсере и в калькуляторе. Это неудобно, поэтому часто парсинг и вычисления комбинируются в один шаг без построения дерева разбора. На примере арифметических выражений:
\begin{itemize}
    \item \(E_0.val = T.val\)
    \item \(E_0.val = E_1.val + T.val\)
    \item \(E_0.val = E_1.val - T.val\)
    \item \(\vdots\)
\end{itemize}

Такой подход называется \textbf{синтаксически управляемая трансляция}.

Итого существуют четыре подхода дизайну систем парсинга в зависимости от сложности задачи:
\begin{enumerate}
    \item Ad hoc: без теории, наивно.
    \item Parser + walker: Парсер производит дерево разбора и walker его обходит.
    \item Синтаксически управляемая трансляция.
    \item Декомпозиция задач.
\end{enumerate}

Этот курс рассматривает второй и третий подходы.

Рассмотрим пример калькулятора арифметических выражений:
\begin{minted}{python}
int expr():
    r = term()
    nexttoken()
    while token == '+':
        nexttoken()
        t = term()
        r += t

int term():
    r = factor()
    nexttoken()
    while token == '*':
        nexttoken()
        f = factor()
        r += f

int factor():
    if token == '('
        nexttoken()
        r = expr()
        assert token == ')'
        nexttoken()
    else # token = 'n'
        r = tokenval()
        nexttoken()
\end{minted}

Какая связь между этим кодом и грамматикой арифметических выражений? Оказывается, весьма близкая и код можно получить из нее.

\section{\(LL(k), \mathrm{FIRST}, \mathrm{FOLLOW}\)}

\begin{definition}[контекстно-свободная грамматика]\itemfix
    \begin{itemize}
        \item Алфавит \(\Sigma\) --- множество токенов
        \item Нетерминалы \(N\)
        \item Стартовый нетерминал \(S \in N\)
        \item Правила \(P \subset N \times (N \cup \sum)^*\)
    \end{itemize}
\end{definition}
\begin{definition}
    \(\ev{A, \alpha} \in P \Leftrightarrow A \to \alpha\)
\end{definition}
\begin{definition}
    \(\alpha \Rightarrow \beta\) --- из \(\alpha\) выводится за один шаг \(\beta\), если:
    \begin{itemize}
        \item \(\alpha = \alpha_1 A \alpha_2\)
        \item \(\beta = \alpha_1 \xi \alpha_2\)
        \item \(A \to \xi \in P\)
    \end{itemize}
\end{definition}
\begin{definition}[язык грамматики]\itemfix
    \(L(\Gamma) = \{x \mid S \Rightarrow^* x\}, x \in \Sigma^*\), где \( \Rightarrow^*\) есть замыкание отношения \( \Rightarrow \).
\end{definition}

\begin{definition}
    Грамматика \textbf{однозначна}, если для любого слова из языка есть только одно дерево разбора и \textbf{неоднозначна} иначе.
\end{definition}

\begin{remark}
    Здесь и далее буквы из конца латинского алфавита обозначают нетерминалы, а буквы греческого алфавита --- строки из терминалов и/или нетерминалов.
\end{remark}
\begin{definition}
    \(\Gamma \in LL(1)\), если из выполнения следующих двух условий:
    \begin{itemize}
        \item \(S \Rightarrow^* xA\alpha \Rightarrow x\xi\alpha \Rightarrow^* xcy\)
        \item \(S \Rightarrow^* xA\beta \Rightarrow x\eta\beta \Rightarrow^* xcz\)
    \end{itemize}
    следует \(c \in \Sigma\), или \(c = \varepsilon\), или \(y = \varepsilon\), или \(z = \varepsilon\), тогда \(\xi = \eta\).
\end{definition}

\begin{definition}
    \(\Gamma \in LL(k)\), если из выполнения следующих двух условий:
    \begin{itemize}
        \item \(S \Rightarrow^* xA\alpha \Rightarrow x\xi\alpha \Rightarrow^* xcy\)
        \item \(S \Rightarrow^* xA\beta \Rightarrow x\eta\beta \Rightarrow^* xcz\)
    \end{itemize}
    следует \(c \in \Sigma^k\), или \(c \in \Sigma^{ \leq k}\), или \(y = \varepsilon\), или \(z = \varepsilon\), тогда \(\xi = \eta\).
\end{definition}

В частности, \(LL(0)\) --- линейные программы.

\(LL(1)\) грамматики есть класс всех грамматик, которые можно разобрать рекурсивным спуском.

Определение \(LL(1)\) грамматик не конструктивно, т.к. проверка определения может длиться бесконечно \textit{(по количеству всех выводов)}. Определим конструктивный критерий принадлежности \(LL(1)\), для этого мы рассмотрим две вспомогательные функции:

\begin{itemize}
    \item FIRST: \((N \cup \Sigma)^* \to 2^{\Sigma \cup \{\varepsilon\}}\)
    \item FOLLOW: \(N \to 2^{\Sigma \cup \{\$\}}\)
\end{itemize}
\[\mathrm{FIRST}(\alpha) \coloneqq \{c \mid \alpha \Rightarrow^* c \beta\} \cup \{\varepsilon \mid \alpha \Rightarrow^* \varepsilon\}\]
\[\mathrm{FOLLOW}(A) \coloneqq \{c \mid S \Rightarrow^* \alpha Ac\beta\} \cup \{\$ \mid S \Rightarrow^* \alpha A\}\]

\begin{remark}
    Мы считаем, что в грамматике нет нетерминалов, из которых нельзя вывести строку из терминалов. Это допущение не теряет общности, т.к. существует алгоритм удаления ``бесполезных'' нетерминалов, см. курс дискретной математики.
\end{remark}

\begin{theorem}
    \(\Gamma \in LL(1) \Leftrightarrow \forall A \to \alpha, A \to \beta\):
    \begin{enumerate}
        \item \(\mathrm{FIRST}(\alpha) \cap \mathrm{FIRST}(\beta) = \emptyset\)
        \item \(\varepsilon \in \mathrm{FIRST}(\alpha) \Rightarrow \mathrm{FIRST}(\beta) \cap \mathrm{FOLLOW}(A) = \emptyset\)
    \end{enumerate}
\end{theorem}
