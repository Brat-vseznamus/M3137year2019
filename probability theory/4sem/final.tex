\documentclass[12pt, a4paper]{article}

\usepackage{lastpage}
\usepackage{mathtools}
\usepackage{xltxtra}
\usepackage{libertine}
\usepackage{amsmath}
\usepackage{amsthm}
\usepackage{amsfonts}
\usepackage{amssymb}
\usepackage{enumitem}
\usepackage{xcolor}
\usepackage[left=1.5cm, right=1.5cm, top=2cm, bottom=2cm, bindingoffset=0cm, headheight=15pt]{geometry}
\usepackage{fancyhdr}
\usepackage[russian]{babel}
% \usepackage[utf8]{inputenc}
\usepackage{catchfilebetweentags}
\usepackage{accents}
\usepackage{calc}
\usepackage{etoolbox}
\usepackage{mathrsfs}
\usepackage{wrapfig}

\providetoggle{useproofs}
\settoggle{useproofs}{false}

\pagestyle{fancy}
\lfoot{M3137y2019}
\rhead{\thepage\ из \pageref{LastPage}}

\newcommand{\R}{\mathbb{R}}
\newcommand{\Q}{\mathbb{Q}}
\newcommand{\C}{\mathbb{C}}
\newcommand{\Z}{\mathbb{Z}}
\newcommand{\B}{\mathbb{B}}
\newcommand{\N}{\mathbb{N}}

\newcommand{\const}{\text{const}}

\newcommand{\teormin}{\textcolor{red}{!}\ }

\DeclareMathOperator*{\xor}{\oplus}
\DeclareMathOperator*{\equ}{\sim}
\DeclareMathOperator{\Ln}{\text{Ln}}
\DeclareMathOperator{\sign}{\text{sign}}
\DeclareMathOperator{\Sym}{\text{Sym}}
\DeclareMathOperator{\Asym}{\text{Asym}}
% \DeclareMathOperator{\sh}{\text{sh}}
% \DeclareMathOperator{\tg}{\text{tg}}
% \DeclareMathOperator{\arctg}{\text{arctg}}
% \DeclareMathOperator{\ch}{\text{ch}}

\DeclarePairedDelimiter{\ceil}{\lceil}{\rceil}
\DeclarePairedDelimiter{\abs}{\left\lvert}{\right\rvert}

\setmainfont{Linux Libertine}

\theoremstyle{plain}
\newtheorem{axiom}{Аксиома}
\newtheorem{lemma}{Лемма}

\theoremstyle{remark}
\newtheorem*{remark}{Примечание}
\newtheorem*{exercise}{Упражнение}
\newtheorem*{consequence}{Следствие}
\newtheorem*{example}{Пример}
\newtheorem*{observation}{Наблюдение}

\theoremstyle{definition}
\newtheorem{theorem}{Теорема}
\newtheorem*{definition}{Определение}
\newtheorem*{obozn}{Обозначение}

\setlength{\parindent}{0pt}

\newcommand{\dbltilde}[1]{\accentset{\approx}{#1}}
\newcommand{\intt}{\int\!}

% magical thing that fixes paragraphs
\makeatletter
\patchcmd{\CatchFBT@Fin@l}{\endlinechar\m@ne}{}
  {}{\typeout{Unsuccessful patch!}}
\makeatother

\newcommand{\get}[2]{
    \ExecuteMetaData[#1]{#2}
}

\newcommand{\getproof}[2]{
    \iftoggle{useproofs}{\ExecuteMetaData[#1]{#2proof}}{}
}

\newcommand{\getwithproof}[2]{
    \get{#1}{#2}
    \getproof{#1}{#2}
}

\newcommand{\import}[3]{
    \subsection{#1}
    \getwithproof{#2}{#3}
}

\newcommand{\given}[1]{
    Дано выше. (\ref{#1}, стр. \pageref{#1})
}

\renewcommand{\ker}{\text{Ker }}
\newcommand{\im}{\text{Im }}
\newcommand{\grad}{\text{grad}}

\usepackage{sectsty}

\allsectionsfont{\raggedright}
\sectionfont{\fontsize{14}{15}\selectfont}

\lhead{Билеты}
\rfoot{}

\settoggle{useproofs}{true}

\renewcommand{\import}[3]{
    \section{#1}
    \getwithproof{#2}{#3}
}

\begin{document}

\import{Пространство элементарных исходов. Случайные события. Операции над  событиями.}{1}{1}

\import{Статистическое определение вероятности. Классическое определение вероятности.}{1}{2.1}
\get{1}{2.2}

\import{Геометрическое определение вероятности. Задача Бюффона об игле.}{1}{3}

\import{Аксиоматическое определение вероятности. Вероятностное пространство. Свойства вероятности.}{2}{4}

\import{Аксиома непрерывности. Ее смысл и вывод.}{2}{5}

\import{Свойства операций сложения и умножения. Формула сложения вероятностей.}{2}{6}

\import{Независимость событий. Независимые события в совокупности и попарно. Пример Бернштейна.}{2}{7}

\import{Условная вероятность. Формула умножения событий.}{3}{8}

\import{Полная группа событий. Формула полной вероятности. Формула Байеса.}{3}{9}

\import{Последовательность независимых испытаний. Формула Бернулли. Наиболее вероятное число успехов в схеме Бернулли.}{4}{10}

\import{Локальная и интегральная формулы Муавра-Лапласа (без док-ва).}{4}{11}

\import{Вероятность отклонения относительной частоты от вероятности события. Закон больших чисел Бернулли.}{4}{12}

\import{Схемы испытаний: Бернулли, до первого успеха. Биномиальное и геометрическое распределения. Свойство отсутствия последействия.}{5}{}

\subsection{Схема Бернулли}

См. билет 10.

\get{5}{13.1}
\get{5}{13.2}
\get{5}{13.3}
\get{5}{13.4}

\import{Урновая схема с возвратом и без возврата. Гипергеометрическое распределение. Теорема об его асимптотическом приближении к биномиальному.}{5}{14}

\import{Схема Пуассона. Формула Пуассона. Оценка погрешности в формуле Пуассона.}{5}{15}

Схема Пуассона: см. формулу Пуассона, т.е. \(n \to +\infty, p \to 0\) в схеме Бернулли.

\import{Случайные величины, определение. Измеримость функции, ее смысл. Вероятностное пространство (R, B, P). Распределение случайной величины.}{1}{}

\import{Дискретные случайные величины. Определение, закон распределения, числовые характеристики.}{1}{}

\import{Свойства математического ожидания и дисперсии дискретной случайной величины.}{1}{}

\import{Стандартные дискретные распределения и их числовые характеристики (Бернулли, биномиальное, геометрическое, Пуассона).}{1}{}

\import{Функция распределения и ее свойства (в свойствах 4, 5, 6 достаточно привести одно из доказательств).}{1}{}

% \import{Абсолютно непрерывные случайные величины. Плотность и ее свойства.}{1}{}

% \import{Числовые характеристики абсолютно непрерывной случайной величины, их свойства.}{1}{}

% \import{Равномерное распределение. }{1}{}

% \import{Показательное распределение. Свойство нестарения.}{1}{}

% \import{Нормальное распределение. Стандартное нормальное распределение, его числовые характеристики.}{1}{}

% \import{Связь между стандартным нормальным и нормальным распределениями. Следствия.}{1}{}

% \import{Гамма-функция и гамма-распределение, его свойства.}{1}{}

% \import{Сингулярные распределения. Теорема Лебега (без док-ва).}{1}{}

% \import{Преобразования случайных величин. Борелевские функции. Стандартизация случайной величины. }{1}{}

% \import{Линейное преобразование случайной величины. Теорема о монотонном преобразовании (без док-ва).}{1}{}

% \import{Квантильное преобразование. Моделирование случайной величины с помощью датчика случайных чисел.}{1}{}

% \import{Виды сходимостей случайных величин, связь между ними. Теорема об эквивалентности сходимостей к константе (все без док-ва).}{1}{}

% \import{Математическое ожидание преобразованной случайной величины. Свойства моментов.}{1}{}

% \import{Неравенство Йенсена, следствие.}{1}{}

% \import{Неравенства Маркова, Чебышева, правило трех сигм.}{1}{}

% \import{Среднее арифметическое одинаковых независимых случайных величин. Закон больших чисел Чебышева.}{1}{}

% \import{Вывод закона больших чисел Бернулли из закона больших чисел Чебышева. Законы больших чисел Хинчина и Колмогорова (только формулировки), закон больших чисел Маркова (с док-м).}{1}{}

% \import{Совместные распределения случайных величин. Функция совместного распределения, ее свойства. Независимость случайных величин.}{1}{}

% \import{Дискретная система двух случайных величин. Закон совместного распределения. Маргинальные распределения.}{1}{}

% \import{Абсолютно непрерывная система двух случайных величин. Плотность совместного распределения, ее свойства.}{1}{}

% \import{Функции от двух случайных величин. Теорема о функции распределения. Формула свертки.}{1}{}

% \import{Суммы стандартных распределений, устойчивость по суммированию (биномиальное, Пуассона, стандартное нормальное).}{1}{}

% \import{Условные распределения и условные математические ожидания. Случаи дискретной и абсолютно непрерывной систем двух случайных величин.}{1}{}

% \import{Пространство случайных величин. Скалярное произведение, неравенство Коши-Буняковского-Шварца. }{1}{}

% \import{Условное математическое ожидание как случайная величина, его свойства. Обобщенная формула полной вероятности.}{1}{}

% \import{Числовые характеристики зависимости случайных величин. Ковариация, ее свойства. Коэффициент корреляции, его свойства. Корреляция случайных величин.}{1}{}

% \import{Характеристическая функция случайной величины, ее свойства. Теорема о непрерывном соответствии (формулировка).}{1}{}

% \import{Характеристические функции стандартных распределений (Бернулли. биномиальное, Пуассона, нормальное). Следствия.}{1}{}

% \import{Доказательство закона больших чисел Хинчина.}{1}{}

% \import{Центральная предельная теорема. Вывод из нее предельной теоремы Муавра-Лапласа. Неравенство Берри-Ессеена (формулировка). }{1}{}

\unfinished

\end{document}
