\documentclass[12pt, a4paper, oneside]{book}

\usepackage{catchfilebetweentags}
\ExecuteMetaData[../../preamble.sty]{preamble}

\addto\captionsrussian{\renewcommand{\chaptername}{Лекция}}

\let\MakeUppercase\relax

\lfoot{}
\cfoot{}
\rfoot{}
\lhead{\leftmark}

\usepackage{tocloft}
\advance\cftsecnumwidth -1em\relax
\advance\cftsubsecindent -1em\relax
\advance\cftsubsecnumwidth -1em\relax

\usepackage{titletoc}
\titlecontents*{chapter}% <section-type>
  [0pt]% <left>
  {}% <above-code>
  {\bfseries\chaptername\ \thecontentslabel\quad}% <numbered-entry-format>
  {}% <numberless-entry-format>
  {\bfseries\hfill\contentspage}% <filler-page-format>

\let\endtitlepage\relax

\usepackage{remreset}
\makeatletter
  \@removefromreset{section}{chapter}
\makeatother

\patchcmd{\chapter}{\thispagestyle{plain}}{\thispagestyle{fancy}}{}{}

\renewcommand{\thesection}{\arabic{section}}

\begin{document}

\title{Теория вероятности}
\author{Михайлов Максим}

\maketitle

\tableofcontents

\chapter{13 февраля}

\section{События}

\subsection{Статистическое определение вероятности}

\begin{definition}
    Пусть проводится \(n\) реальных экспериментов, событие \(A\) произошло в \(n_A\) экспериментах. Отношение \(\cfrac{n_A}{n}\) называется \textbf{частотой события} \(A\). Эксперименты показывают, что при увеличении числа \(n\) эта частота ``стабилизируется'' около некоторого числа, под которым понимаем \textbf{статистическую вероятность}.
    \[P(A)\approx \frac{n_A}{n}\]
\end{definition}

Очевидно это определение не формально, поэтому мы им пользоваться не будем.

\subsection{Пространство элементарных исходов. Случайные события.}

\begin{definition}\itemfix
    \begin{itemize}
        \item \textbf{Пространством элементарных исходов} \(\Omega\) называется множество, содержащее все возможные результаты данного эксперимента, из которых при испытании происходит ровно один.
        \item Элементы данного множества называются \textbf{элементарными исходами} и обозначаются \(w\in \Omega\).
        \item \textbf{Случайными событиями} называются подмножества \(A\subset \Omega\).
        \item Событие \(A\) \textbf{наступило}, если в ходе эксперимента произошёл один из элементарных исходов, входящих в \(A\).
        \item Такие исходы называются \textbf{благоприятными} к \(A\).
    \end{itemize}
\end{definition}

\begin{example}\itemfix
    \begin{enumerate}
        \item Бросают монетку. \(\Omega = \{\text{Г}, \text{Р}\} \) \textit{(герб, решка)}.
        \item Бросают кубик. \(\Omega = \{1,2,3,4,5,6\} \). \(A\) --- выпало четное число очков. Тогда \(A = \{2,4,6\} \).
        \item Монета бросается дважды:
              \begin{enumerate}
                  \item Учитываем порядок: \(\Omega = \{\text{ГГ, РР, ГР, РГ}\} \)
                  \item Не учитываем порядок: \(\Omega = \{\text{ГГ, РР, ГР}\} \)
              \end{enumerate}
        \item Бросается дважды кубик, порядок учитывается. \(A\) --- разность очков делится на 3, т.е. \(A = \{(1,4), (4,1), (3,3), (5,2), (2,5), (3, 6), (6, 3), (1, 1), (2, 2), (4, 4), (5, 5), (6, 6)\} \)
        \item Монета бросается до выпадения герба. \(\Omega = \{\text{Г, РГ, РРГ, \dots}\} \) --- счётное число исходов.
        \item Монета бросается на плоскость. \(\Omega = \{(x, y) \ | \ x, y\in\R\} \) --- несчётное число исходов.
    \end{enumerate}
\end{example}

\subsection{Операции над событиями}

\(\Omega\) --- универсальное \textit{(достоверное)} событие, т.к. содержит все элементарные исходы.

\(\emptyset\) --- невозможное событие.

\begin{definition}
    \(A + B\) это \(A\cup B\)

    \begin{figure}[h]
        \centering
        \includesvg[scale=0.5]{images/intersection.svg}
    \end{figure}
\end{definition}

\begin{definition}
    \(A\cdot B\) это \(A\cap B\)

    \begin{figure}[h]
        \centering
        \includesvg[scale=0.5]{images/union.svg}
    \end{figure}
\end{definition}

\begin{definition}
    \textbf{Противоположным} к \(A\) называется событие \(\overline A\), соответствующее тому, что \(A\) не произошло, т.е. \(\Omega\setminus A\)
\end{definition}

\begin{definition}
    Дополнение \(A\setminus B\) это \(A\cdot \overline B\)
\end{definition}

\begin{definition}
    События \(A\) и \(B\) называются \textbf{несовместными}, если \(A\cdot B = \emptyset\)
\end{definition}

\begin{definition}
    Событие \(A\) влечет событие \(B\), если \(A\subset B\).
\end{definition}

\section{Вероятность}

\begin{definition}
    \(0 \leq P(A) \leq 1\) --- вероятность наступления события \(A\).
\end{definition}

\subsection{Классическое определение вероятности}

Пусть \(\Omega\) содержит конечное число исходов, причем их можно считать равновозможными. Тогда применимо классическое определение вероятности.

\(P(A) = \cfrac{|A|}{|\Omega|} = \cfrac{m}{n}\), где \(n\) --- число всех возможных элементарных исходов, \(m\) --- число элементарных исходов, благоприятных \(A\).

В частности, если \(|\Omega|= n\), а \(A\) --- элементарный исход, то \(P(A) = \frac{1}{n}\).

\begin{prop}\itemfix
    \begin{enumerate}
        \item \(0 \leq P(A) \leq 1\)
        \item \(P(\emptyset) = 0\)
        \item \(P(\Omega) = 1\)
        \item \? % TODO #65
    \end{enumerate}
\end{prop}

Если \(A\) и \(B\) несовместны, то \(P(A + B) = P(A) + P(B)\)

\begin{proof}
    \(|A| : = m_1, |B| : = m_2, |A\cup B|= m_1 + m_2\)

    \[P(A + B) = \frac{m_1 + m_2}{n} = \frac{m_1}{n} + \frac{m_2}{n} = P(A) + P(B)\]
\end{proof}

\begin{example}
    Найти вероятность, что при бросании кости выпадет чётное число очков.

    \[n = 6, m = 3, \frac{m}{n} = \frac{1}{2}\]
\end{example}

\begin{example}
    В ящике лежат 3 белых и 2 чёрных шара. Вынули 3 шара. Найти вероятность того, что из них две белых и один чёрный.

    \[n = \binom{5}{3} = 10\]
    \[m = \binom{3}{2} \binom{2}{1} = 12\]
    \[P(A) = \frac{6}{10}\]
\end{example}

Однако, это определение редко применимо.

\subsection{Геометрическое определение вероятности}

\begin{definition}\itemfix
    \begin{itemize}
        \item \(\Omega \subset \R^n\) --- замкнутая ограниченная область.
        \item \(\mu\) --- конечная мера множества \(\Omega\), например мера Лебега
    \end{itemize}

    Пусть выбирают точку наугад, т.е. вероятность попадания точки в область \(A\) зависит от меры \(A\), но не от её положения.

    Тогда
    \(P(A) = \cfrac{\mu(A)}{\mu(\Omega)}\)
\end{definition}

\begin{remark}
    По этому определению мера точки равна \(0\) и вероятность попадания в конкретную точку тоже равна \(0\).
\end{remark}

\begin{example}
    Монета диаметром 6 сантиметров бросается на пол, вымощенный квадратной плиткой со стороной 20 сантиметров. Найти вероятность того, что монета целиком окажется на одной плитке.

    Без ущерба для общности можно рассматривать, что монета бросается на одну плитку и положение монеты определяется положением её центра.

    Чтобы монета лежала полностью на одной плитке, необходимо, чтобы её центр лежал на расстоянии \( \geq 3\) сантиметра от каждой стороны:

    \begin{figure}[h]
        \centering
        \includesvg[scale=0.8]{images/задача-1.svg}
    \end{figure}

    \[S(\Omega) = 20^2 = 400\]
    \[S(A) = 14^2 = 196\]
    \[P(A) = \frac{196}{400} = 0.49\]
\end{example}

\begin{example}
    \? % TODO #64

    \[A : X \leq l \sin \varphi\]
    \[S(\Omega) = \pi l\]
    \[S(A) = \int_0^\pi l \sin \varphi d\varphi = - l \cos \varphi \Big|_0^\pi = - l(\cos \pi - \cos 0) = 2l\]
    \[P(A) = \frac{S(A)}{S(\Omega)} = \frac{2}{\pi}\]
\end{example}

Это определение кажется хорошим --- оно согласовано с классическим. Но и это определение редко применимо на практике, т.к. обычно вероятность зависит от положения в пространстве или множество исходов несчётно.

\chapter{20 февраля}

\subsection{Аксиоматическое определение вероятности}

Пусть \(\Omega\) --- пространство элементарных исходов.

\begin{definition}
    Систему \(\mathcal{F}\) подмножеств \(\Omega\) называют \(\sigma\)-алгеброй событий, если:
    \begin{enumerate}
        \item \(\Omega\in\mathcal{F}\)
        \item \(A\in \mathcal{F} \Rightarrow \overline A \in \mathcal{F}\)
        \item \(A_1 \dots A_n \dots \in \mathcal{F} \Rightarrow \bigcup\limits_{i = 1}^{+\infty} A_i \in \mathcal{F}\)
    \end{enumerate}
\end{definition}

\begin{remark}
    Из 2 и 3 следует 1.
\end{remark}

\begin{prop}\itemfix
    \begin{enumerate}
        \item \(\emptyset \in \mathcal{F}\), т.к. \(\overline \Omega = \emptyset\)
        \item \(A_1 \dots \in \mathcal{F} \Rightarrow \bigcup A_i \in \mathcal{F}\)

              \begin{proof}
                  \(A_1 \dots \in \mathcal{F} \Rightarrow \overline A_1 \dots \in \mathcal{F} \Rightarrow \bigcup \overline A_i \in \mathcal{F} \Rightarrow \overline{\bigcup \overline A_i} = \bigcap A_i \in \mathcal{F}\)
              \end{proof}

        \item \(A, B \in F \Rightarrow A\setminus B \in \mathcal{F}\)
    \end{enumerate}
\end{prop}

\? % TODO

\begin{definition}
    Пусть \(\Omega\) --- множество элементарных исходов, \(\mathcal{F}\) --- \(\sigma\)-алгебра над ним.

    \textbf{Вероятностью} на \((\Omega, \mathcal{F})\) называется функция \(P(A) : \mathcal{F} \to \R\) со свойствами:
    \begin{enumerate}
        \item \(P(A) \geq 0\) --- свойство неотрицательности
        \item Если события \(A_1 \dots A_n \dots \) --- равновероятные, т.е. \(A_i \cap A_j = \emptyset\), то \(P(\bigcap A) = \sum P(A_i)\) --- свойство счётной аддитивности
        \item \(P(\Omega) = 1\) --- свойство нормированности
    \end{enumerate}
\end{definition}

\begin{remark}
    Вероятность есть нормированная мера.
\end{remark}

\begin{definition}
    Тройка \((\Omega, \mathcal{F}, P)\) называется вероятностным пространством.
\end{definition}

\begin{prop}\itemfix
    \begin{enumerate}
        \item \(P(\emptyset) = 0\)

              \begin{proof}
                  \(\underbrace{P(\emptyset + \Omega)}_{1} = P(\emptyset) + \underbrace{P(\Omega)}_1 \Rightarrow P(\emptyset) = 0\)
              \end{proof}

        \item Формула обратной вероятности: \(P(A) = 1 - P(\overline A)\)

              \begin{proof}
                  \(A\) и \(\overline A\) --- несовместны, \(A + \overline A = \Omega\).

                  \(P(A + \overline A) = P(A) + P(\overline A) = 1 \Rightarrow P(A) = 1 - P(\overline A)\)
              \end{proof}

        \item \(0 \leq P(A) \leq 1\)

              \begin{proof}
                  \begin{enumerate}
                      \item \(P(A) \geq 0\)
                      \item \(P(A) = 1 - P(\overline A) \leq 1\)
                  \end{enumerate}
              \end{proof}
    \end{enumerate}
\end{prop}

\subsection{Аксиома непрерывности}

Пусть имеется убывающая цепочка событий \(A_1 \supset A_2 \supset \dots \) и \(\bigcap A_i = \emptyset\). Тогда \(P(A_n) \xrightarrow{n \to +\infty} 0\)

\? % TODO
% соответствующая вероятность также должна изменяться непрерывно.

\begin{theorem}
    Эта аксиома следует из второй аксиомы.
\end{theorem}
\begin{proof}
    \(A_n = \sum_{i = n}^{+\infty} A_i \overline A_{i+1} \cup \bigcap_{i = n}^{+\infty} A_i\)

    \? % TODO

    Т.к. по условию \(P(\bigcap_{i = 1}^{+\infty} A_i) = \emptyset\) и \(\bigcap_{i = n}^{+\infty} A_i = \bigcap_{i = 1}^{+\infty} A_i\), то \(P(\bigcap_{i = n}^{+\infty} A_i) = 0\).

    Таким образом, \(P(A_n) = \sum_{i = n}^{+\infty} P(A_i \overline A_{i + 1})\) --- остаточный член сходящейся последовательности \( \Rightarrow P(A_n) \to 0\)
\end{proof}

\begin{remark}
    Аксиома счётной аддитивности следует из аксиомы непрерывности и свойства конечной аддитивности.
\end{remark}

\subsection{Формула сложения}

Если \(A\) и \(B\) несовместны, то \(P(A + B) = P(A) + P(B)\)

\begin{theorem}
    \(P(A + B) = P(A) + P(B) - P(A B)\)
\end{theorem}
\begin{proof}
    \[A + B = A\overline B + AB + \overline A B\]
    \begin{align*}
        P(A + B) & = P(A\overline B) + P(AB) + P(\overline A B)                     \\
                 & = (P(A\overline B) + P(AB)) + (P(\overline A B) + P(AB)) - P(AB) \\
                 & = P(A) + P(B) - P(AB)
    \end{align*}
\end{proof}

Аналогично можно доказать формулу включения-исключения:
\[P\left(\sum A_i\right) = \sum P(A_i) - \sum_{i < j} P(A_i A_j) + \sum P(A_i A_j A_k) + \dots + ( - 1)^{n - 1} P(A_1 A_2 \dots A_n)\]

\begin{example}
    Опущен.
\end{example}

\section{Независимые события}

\begin{definition}
    События \(A\) и \(B\) называются \textbf{независимыми}, если \(P(AB) = P(A)P(B)\)
\end{definition}

\begin{proof}
    Если \(A\) и \(B\) независимы, то \(A\) и \(\overline{B}\) --- независимы.
\end{proof}
\begin{proof}
    \[P(A) = P(A(B + \overline B)) = P(AB + A\overline B) = P(AB) + P(A\overline B)\]
    \[P(A \overline B) = P(A) - P(AB) = P(A) - P(A)P(B) = P(A)(1 - P(B)) = P(AB)\]
    Таким образом, \(A\) и \(\overline{B}\) --- независимы.
\end{proof}

\begin{definition}
    События \(A_1 \dots A_n\) называются \textbf{независимыми в совокупности}, если для любого набора \(1 \leq i_1 \dots i_n \leq n\) \(P(A_{i_1}, A_{i_2}, \dots A_{i_n}) = P(A_{i_1})P(A_{i_2}) \dots P(A_{i_n})\)
\end{definition}

\begin{example}[Бернштейн]
    Три грани правильного тетраедра выкрашены в красный, синий, зеленый цвета, а четвертая грань --- во все эти три цвета.

    Бросаем тетраедр и смотрим на грань, на которую он упал. События:
    \begin{itemize}
        \item \(A\) --- красный цвет
        \item \(B\) --- синий цвет
        \item \(C\) --- зеленый цвет
    \end{itemize}

    \[P(A) = P(B) = P(C) = \frac{1}{2}\]
    \[P(AB) = P(AC) = P(BC) = \frac{1}{4}\]
    Таким образом, все события попарно независимы.
    \[P(ABС) = \frac{1}{8}\]
\end{example}

\begin{remark}
    Если в условии есть ``хотя бы'', т.е. требуется найти вероятность суммы совместных независимых событий, то применима формула обратной вероятности.
\end{remark}

\begin{example}
    Найти вероятность того, что при четырёх бросаниях кости хотя бы один раз выпадет шестерка.

    \(A_i\) --- при \(i\)-том броске хотя бы один раз выпала шестерка.

    \[P(\overline A_1) = P(\overline A_2) = P(\overline A_3) = P(\overline 4) = \frac{5}{6}\]

    \[\overline A = \overline A_1 \dots \overline A_4\]
    \[P(\overline A) = \left( \frac{5}{6} \right)^4\]
    \[P(A) = 1 - \left( \frac{5}{6} \right)^4\]
\end{example}

\begin{example}
    Два стрелка стреляют по мишени. Вероятность попадания первого стрелка \(0.6\), второго --- \(0.8\). Найти вероятность того, что попадет ровно один стрелок.

    \(A_1\) --- первый стрелок попал, \(A_2\) --- второй стрелок попал, \(A\) --- ровно один попал.

    \[P(A_1) = 0.8, P(\overline A_1) = 0.2, P(A_2) = 0.6, P(\overline A_2) = 0.4\]

    \[A = A_1 \overline A_2 + \overline A_1 A_2\]
    \[P(A) = P(A_1) P(\overline A_2) + P(\overline A_1) P(A_2) = 0.8 \cdot 0.4 + 0.6 \cdot 0.2 = 0.44\]
\end{example}

\chapter{27 февраля}

\section{Условная вероятность}

\begin{obozn}
    \(P(A|B)\) --- вероятность наступления события \(A\), вычисленная в предположении, что событие \(B\) уже произошло.
\end{obozn}

\begin{example}
    Кубик подбрасывается один раз. Известно, что выпало больше \(3\) очков. Какова вероятность, что выпало чётное число очков?

    Пусть \(A\) --- чётное число очков, \(B\) --- больше \(3\) очков.

    \[n = 3 \quad (4, 5, 6) \quad m = 2 \quad (4, 6)\]
    \[P(A | B) = \frac{m}{n} = \frac{2}{3} = \frac{\frac{2}{6}}{\frac{3}{6}} = \frac{P(A \cdot B)}{P(B)}  \]
\end{example}

\begin{definition}
    \textbf{Условной вероятностью} события \(A\) при условии, что имело место событие \(B\), называется величина \(P(A|B) = \frac{P(AB)}{P(B)}\)
\end{definition}

\begin{theorem}
    \(P(A_1 \dots A_n) = P(A_1)\cdot P(A_2 | A_1)\cdot P(A_3| A_2A_1) \dots P(A_n | A_{n - 1}\dots A_1)\)
\end{theorem}
\begin{proof}
    По индукии.
    \begin{itemize}
        \item [\textbf{База}] \(n = 2\) --- по определению полной вероятности.
        \item [\textbf{Переход}] \[P(A_1 \dots A_n) = P(A_1\dots A_{n - 1}) P(A_n | A_1 \dots A_{n - 1}) = P(A_1)\cdot P(A_2 | A_1) \dots P(A_n | A_{n - 1}\dots A_1)\]
    \end{itemize}
\end{proof}

\begin{definition}
    События \(A\) и \(B\) независимы, если \(P(A|B) = P(A)\), что равносильно \(P(AB) = P(A)P(B)\) --- прошлому определению.
\end{definition}
\begin{proof}
    \[P(A|B) = \frac{P(AB)}{P(B)} = P(A) \Leftrightarrow P(AB) = P(A)P(B)\]
\end{proof}

\subsection{Полная группа событий}

\begin{definition}
    События \(H_1\dots H_n\dots \) образуют \textbf{полную группу событий}, если они попарно несовместны и содержат все элементарные исходы.
\end{definition}

\begin{theorem}
    Пусть \(H_1 \dots H_n\) --- полная группа событий. Тогда \(P(A) = \sum_{k = 1}^{+\infty} P(H_i)P(A | H_i)\)
\end{theorem}
\begin{proof}
    \begin{align*}
        P(A) & = P(\Omega A)                                                              \\
             & = P((H_1 + \dots H_n + \dots ) A)                                          \\
             & = P\left(\sum_{k = 1}^{+\infty} H_k A\right)                               \\
             & \symrefeq{по счетной аддитивности} \sum_{k = 1}^{+\infty} P(H_i)P(A | H_i) \\
    \end{align*}

    \ref{по счетной аддитивности}: По лемме о счётной аддитивности и т.к. \(H_iA\) и \(H_jA\) несовместны.
\end{proof}

\subsection{Формула Байеса}

Эта формула также называется формулой проверки гипотезы.

\begin{theorem}
    Пусть \(H_1 \dots H_n\) --- полная группа событий и известно, что \(A\) произошло. Тогда \[P(H_i | A) = \frac{P(H_i)P(A | H_i)}{\sum_{k = 1}^{+\infty} P(H_k)P(A|H_k)} \]
\end{theorem}
\begin{proof}
    \[P(H_i | A) = \frac{P(H_i A)}{P(A)} = \frac{P(H_i)P(A | H_i)}{\sum_{k = 1}^{+\infty} P(H_k)P(A|H_k)}\]
\end{proof}

\end{document}

