\chapter{15 мая}

\section{Комплексная случайная величина}
\[ \xi + i\eta \]
\[ \mathbb{E}(\xi + i \eta) = \mathbb{E}\xi + iE\eta \]
%<*47>
\begin{definition}
    Характеристической функцией случайной величины \(\xi\) называется функция \[\varphi_{\xi}(t) = \mathbb{E} e^{it\xi}\quad t \in \R\]
\end{definition}

\begin{prop}\itemfix
    \begin{enumerate}
        \item
              Характеристическая функция существует для любой случайной величины, при чем
              \[ |\varphi_\xi(t)| \le 1 \]

              \begin{proof}
                  \[ \mathbb{D}\eta  = \mathbb{E}\eta^2 - (\mathbb{E}\eta)^2 \ge 0 \implies (\mathbb{E}\eta)^2 \le \mathbb{E}\eta^2 \]
                  \[ |\varphi_\xi(t)|^2 = |\mathbb{E} e^{it\xi}|^2 = |\mathbb{E}\cos t\xi + iE\sin t\xi|^2 = (\mathbb{E}\cos t\xi)^2 + (\mathbb{E}\sin t\xi)^2 \le   \]
                  \[ \le \mathbb{E}\cos^2t\xi + \mathbb{E}\sin^2 t\xi = \mathbb{E}(\cos^2t\xi + \sin^2t\xi) = E1 = 1 \]
              \end{proof}
        \item
              Пусть \(\varphi_\xi(t)\) --- характеристическая функция случайной величины \(\xi\), тогда характеристическая функция случайной величины \(\eta = a + b\xi\):
              \[ \varphi_{a + b\xi}(t) = e^{ita}\cdot\varphi_\xi(bt)\]

              \begin{proof}
                  \[ \varphi_{a + b\xi}(t) = Ee^{it(a + b\xi)} = Ee^{ita}\cdot e^{itb\xi} = e^{ita}\cdot Ee^{i(tb)\xi} = e^{ita} \cdot \varphi_\xi(bt) \]
              \end{proof}
        \item
              Пусть случайная величины \(\xi\) и \(\eta\) независимы, тогда
              \[ \varphi_{\xi + \eta}(t) = \varphi_\xi(t) \cdot \varphi_\eta(t) \]

              \begin{proof}
                  \[ \varphi_{\xi + \eta}(t)  = Ee^{it(\xi + \eta)} = Ee^{it\xi}\cdot e^{it\eta} = Ee^{it\xi}\cdot Ee^{it\eta} = \varphi_\xi(t)\cdot \varphi_\eta(t) \]
              \end{proof}
        \item
              Пусть существует \(k\)-тый момент случайной величины \(\xi\), тогда характеристическая функция \(\varphi_\xi(t)\) непрерывно дифференцируема \(k\) раз и \[\varphi_\xi^{(k)}(0) = i^k\cdot \mathbb{E}\xi^k\]

              \begin{proof}
                  Доказательство существования непрерывности опустим
                  \[ \varphi_\xi^{(k)}(0) = \left(\frac{\partial^k}{\partial t^k} Ee^{it\xi}\right) = \mathbb{E}\left(\frac{\partial^k}{\partial t^k} e^{it\xi}\right) = \mathbb{E}(i^k\xi^ke^{it\xi}) \xlongequal[t = 0]{} \mathbb{E}(i^k \xi^k e^0) = i^k\cdot \mathbb{E}\xi^k \]
              \end{proof}
        \item
              Пусть \(\mathbb{E}|\xi|^k < \infty\). Тогда
              \[ \varphi_\xi(t) = \varphi_\xi(0) + it \mathbb{E}\xi - \frac{t^2}{2}\mathbb{E}\xi^2 + \dots + \frac{i^kt^k}{k!}\mathbb{E}\xi^2 + o(|t|^k) \]

              \begin{proof}
                  По формуле Тейлора в точке \(t = 0\)
                  \[ \varphi_\xi(t) = \varphi_\xi(0) + \frac{\varphi_\xi'(0)}{1!}t + \frac{\varphi_\xi''(0)}{2!}t^2 + \dots + \frac{\varphi_\xi^{(k)}}{k!}t^k + o(|t|^2) = \]
                  \[ = \varphi_\xi(0) + itE\xi - \frac{t^2}{2}\mathbb{E}\xi^2 + \dots + \frac{i^k t^k}{k!} \mathbb{E}\xi^k + o(|t|^2) \]
              \end{proof}
        \item
              \label{свойство 6 эксп СВ}
              Распределение случайной величины восстанавливается по характеристической функции, т.е. существует взаимно однозначное соответствие между распределениями и характеристическими функциями. В частности если \(\xi\) --- абсолютно непрерывная случайная величина, то плотность:
              \[ f_\xi(x) = \frac{1}{\sqrt{2\pi}}\int_{-\infty}^\infty e^{-itx}\cdot \varphi_\xi(t)\,dt \]

    \end{enumerate}
\end{prop}
\begin{theorem}[о непрерывном соответсвии]
    Последовательность случайных величин \(\xi_n\) слабо сходится к случайной величине \(\xi\), тогда и только тогда, когда соответствующая последовательность характеристических функций поточечно сходится к характеристической функции \(\varphi_\xi(t)\)
    \[ \xi_n \rightrightarrows \xi \Leftrightarrow \forall t \in \R\ \varphi_{\xi_n}(t) \to \varphi_\xi(t)  \]
\end{theorem}
%</47>

\subsection{Характеристические функции стандартных распределений}

%<*48.1>
\subsubsection{Распределение Бернулли}

\(\xi \in B_p\)

\begin{tabular}{C|C|C}
    \xi  & 0     & 1 \\
    \eta & 1 - p & p
\end{tabular}

\[ \varphi_\xi(t) = \mathbb{E} e^{it\xi} = e^{it\cdot 0} p(\xi = 0)  + e^{it1}\cdot p(\xi = 1) = 1 - p + pe^{it}\]
\subsubsection{Биномиальное распределение}
\(\xi \in B_{n,p}\) --- число успехов при \(n\) независимых испытаниях
\[ \xi = \xi_1 + \dots + \xi_n \]
, где \(\xi_i = B_{p}\) --- число успехов при одном испытании
\[ \varphi_\xi(t) = (1 - p + pe^{it})^n \]
\subsubsection{Распределение Пуассона}
\(\xi \in \Pi_\lambda\)
\[ p(\xi = k) = \frac{\lambda^k}{k!}e^{-\lambda}\]
\[ \varphi_\xi(t) = Ee^{it\xi} = \sum_{n = 0}^\infty e^{itk} \cdot p(\xi = k) = \sum_{n = 0}^\infty e^{itk} \frac{\lambda^k}{k!}e^{-\lambda} =  \]
\[ = e^{-\lambda} \sum_{n = 0}^\infty \frac{(\lambda e^{it})^k}{k!} = e^{-\lambda}e^{\lambda e^{it}} = \exp(\lambda(e^{it} - 1))\]

\begin{corollary}
    Если \(\xi \in \Pi_\lambda\), \(\eta \in \Pi_\mu\) --- независимые случайные величины, то \(\xi + \eta = \Pi_{\lambda + \mu}\)
\end{corollary}
%</48.1>

\begin{proof}
    \[ \varphi_{\xi + \eta}(t) = \varphi_\xi(t)\cdot \varphi_\eta(t) = \exp(\lambda(it - 1))\cot \exp(\mu(it - 1)) = \exp((\lambda + \mu)(it - 1)) \]
    По свойству \ref{свойство 6 эксп СВ}
\end{proof}

\subsubsection{Гамма распределение}
\(\xi \in \Gamma_{\alpha, \lambda}\)
\[ f_\xi(x) = \frac{\alpha^\lambda}{\Gamma(\lambda)}x^{\lambda - 1}e^{-\alpha x}\quad x > 0 \]
\begin{align*}
    \varphi_\xi(t) & = \mathbb{E} e^{it\xi}                                                                                          \\
                   & = \int_{-\infty}^{+\infty} e^{ - itx} f_\xi(x) dx                                                               \\
                   & = \int_{-\infty}^{+\infty} e^{ - itx} \frac{\alpha^\lambda}{\Gamma(\lambda)} x^{\lambda - 1} e^{ - \alpha x} dx \\
                   & = \frac{\alpha^\lambda}{\Gamma(\lambda)} \int_{-\infty}^{+\infty} x^{\lambda - 1} e^{ - x(it + \alpha)} dx      \\
                   & = \?                                                                                                            \\
                   & = \left(\frac{\alpha}{\alpha - it}\right)^\lambda                                                               \\
\end{align*}

\subsubsection{Экспоненциальное распределение}
\(\xi \in E_\alpha = \Gamma_{\alpha, 1}\)
\[ \varphi_\xi(t) = \frac{\alpha}{\alpha - it} \]
\begin{corollary}
    Пусть \(\xi \in \Gamma_{\alpha, \lambda_1}\), \(\eta \in \Gamma_{\alpha, \lambda_2}\) --- независимые случайные величины. Тогда \(\xi + \eta \in \Gamma_{\alpha, \lambda_1 + \lambda_2}\)
\end{corollary}

%<*48.2>
\subsubsection{Стандартное нормальное распределение}
\(\xi \in N_{0,1}\)
\[ f_\xi(x) = \frac{1}{\sqrt{2\pi}}e^{-\frac{x^2}{2}} \]
\begin{align*}
    \varphi_\xi(t)
     & = \?                 \\
     & = e^{-\frac{t^2}{2}} \\
\end{align*}

\subsubsection{Нормальное распределение}
\(\xi \in N_{a,\sigma^2}\)
\[ \eta = \frac{\xi - a}{\sigma} \in N(0, 1)\quad \xi = a + \sigma\eta \]
\[ \varphi_\xi(t) = \varphi_{a + \sigma\eta}(t) = e^{ita}\cdot \varphi_\eta(\sigma t) = e^{ita}\cdot e^{-\frac{(\sigma t)^2}{2}} \]
\begin{corollary}
    \(\xi \in N(a_1, \sigma_1^2)\), \(\eta \in N(a_2, \sigma_2^2)\) --- независимые случайные величин. Тогда
    \[ \xi + \eta \in N(a_1 + a_2, \sigma_1^2 + \sigma_2^2) \]
\end{corollary}
%</48.2>

\subsection{Доказательство основных теорем}
\begin{lemma}
    \[ \left(1 + \frac{x}{n} + o\left(\frac{1}{n}\right)\right)^n \xrightarrow[n \to \infty]{} e^x \]
\end{lemma}
%<*49>
\begin{theorem}
    Пусть \(\xi_1, \xi_2, \dots\) --- последовательность независимых одинаково распределенных случайных величин с конечным первым моментом \(a\).
    Тогда
    \[ \frac{S_n}{n} = \frac{\xi_1 + \dots + \xi_n}{n} \xrightarrow{p} a \]
\end{theorem}
\begin{proof}
    Докажем, что \(\frac{S_n}{n} \rightrightarrows a\), тогда по теореме об эквивалентности сходимости к константе искомое будет верно.
    \[\frac{S_n}{n} \rightrightarrows a \Leftrightarrow \varphi_{\frac{S_n}{n}}(t) \to \varphi_a(t) = \mathbb{E} e^{ita} = e^{ita} \quad \forall t \in \R\]
    \[\varphi_{\frac{S_n}{n}}(t) = \varphi_{S_n}\left( \frac{t}{n} \right) = \left(\varphi_{\xi_1}\left( \frac{t}{n} \right)\right)^n\]
    По свойству 5, т.к. первый момент существует,
    \[\varphi_{\xi_1}(t) = 1 + it \mathbb{E}\xi_1 + \smallO(t) = 1 + ita + \smallO(t)\]
    \[\varphi_{\frac{S_n}{n}}(t) = \left(\varphi_{\xi_1}\left( \frac{t}{n} \right)\right)^n = \left(\varphi_{\xi_1}\left( 1 + \frac{ita}{n} + \smallO\left( \frac{t}{n} \right) \right)\right)^n \to e^{ita}\]
\end{proof}
%</49>

\begin{theorem}
    Пусть \(\xi_1, \xi_2, \dots\) --- последовательность независимых одинаково распределенных случайных величин с конечной дисперсией. \(a = \mathbb{E}\xi_1\), \(\sigma^2 = \mathbb{D}\xi_1\). \\
    Тогда
    \[ \frac{S_n - na}{\sigma\sqrt{n}} \rightrightarrows N_{0, 1} \]
\end{theorem}
\begin{proof}
    \unfinished
\end{proof}

\subsubsection{Предельная теорема Муавра-Лапласа}

\begin{theorem}\itemfix
    \begin{itemize}
        \item \(\nu_n(A)\) --- число появлений события \(A\) при \(n\) испытаниях
        \item \(p\) --- вероятность одного испытания
        \item \(q = 1 - p\)
    \end{itemize}
    Тогда
    \[ \frac{\nu_n(A) - np}{\sqrt{npq}}\rightrightarrows N_{0,1} \]
\end{theorem}
\begin{corollary}[формула Муавра-Лапласа]
    \[p(x_1 \le \nu_n \le x_2) = p\left(\frac{x_1 - np}{\sqrt{npq}} \le \underbrace{\frac{\nu_n - np}{\sqrt{npq}}}_\eta \le \frac{x_2 - np}{\sqrt{npq}} \right) = \]
    \[ = F_\eta\left(\frac{x_1 - np}{\sqrt{npq}}\right)\cdot F_\eta\left(\frac{x_2 - np}{\sqrt{npq}}\right) \xrightarrow[n \to \infty]{} \Phi_0\left(\frac{x_1 - np}{\sqrt{npq}}\right) \cdot \Phi_0\left(\frac{x_2 - np}{\sqrt{npq}}\right)\]
\end{corollary}
\begin{remark}
    Аналогичным образом ЦПТ(центральная предельная теорема) применяется при приближенных оценках вероятностей связанных с суммами большого числа независимых случайных величин. Какова погрешность?
\end{remark}
\begin{theorem}
    В условиях ЦПТ, \? с конечным третьим моментом
    \[ \left| p\left( \frac{S_n - \mathbb{E}\xi_1}{\sqrt{nD\xi_1}} \right) - \Phi_0(x) \right| \le C \frac{\mathbb{E}|\xi_1 - \mathbb{E}\xi_1|^2}{\sqrt{n}(\sqrt{\mathbb{D}\xi_1})^3} \quad \forall x \in \R \]
\end{theorem}
\begin{remark}
    Можно взять \(C = 0.4\)
\end{remark}
