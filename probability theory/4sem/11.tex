\chapter{24 апреля}

\section{Совместные распределения случайных величин}

Предположим, что мы выполняем некоторый эксперимент. У нас есть некоторые случайные величины и нас интересует, как они связаны, а также мы хотим построить модель, которая по значениям одних случайных величин достаточно точно предсказывает значения других величин.

%<*38>
\begin{definition}
    \textbf{Случайным вектором} \((\xi_1, \xi_2 \dots \xi_n)\) называется упорядоченный набор случайных величин, заданных на одном вероятностном пространстве \((\Omega, \mathfrak{F}, P)\). Случайный вектор задает отображение \((\xi_1 \dots \xi_n)(\omega) : \Omega \to \R^n\). Поэтому случайный вектор также называют \textbf{многомерной случайной величиной}, а соответствующее распределение \(P(\prod) = P(\omega \in \Omega)(\xi_1 \dots \xi_n)(\omega) \in B\), где \(B \in \mathfrak{B}(\R^n)\) называется \textbf{многомерным распределением}.
\end{definition}

Таким образом, мы получаем новое вероятностное пространство \((\R^n, \mathfrak{B}(\R^n), P(B))\).

\begin{remark}
    \(\mathfrak{B}(\R^n)\) --- борелевская \(\sigma\)-алгебра --- минимальная\footnote{т.е. полученная римановским продолжением} \(\sigma\)-алгебра, порожденная \(n\)-мерными прямоугольниками.
\end{remark}

\subsection{Функция распределения}

\begin{definition}
    \textbf{Функцией совместного распределения} случайных величин \(\xi_1 \dots \xi_n\) называется функция \(F_{\xi_1 \dots \xi_n}(x_1 \dots x_n) = P(\xi_1 < x_1) \cdots P(\xi_n < x_n)\), то есть вероятность попадания в бесконечный параллелепипед.
\end{definition}

Т.к. любой параллелепипед можно представить в виде пересечения таких параллелепипедов, то можно найти меру любого борелевского множества. Таким образом, по \(F\) можно найти \(P(B) \ \ \forall B \in \mathfrak{B}(\R^n)\)

\begin{remark}
    В дальнейшем мы будем изучать только системы из двух случайных величин \(\xi\) и \(\eta\). \(F_{\xi, \eta}(x, y) = P(\xi < x, \eta < y)\). Геометрически \(F\) есть вероятность попадания в третью четверть относительно точки \((x, y)\).
\end{remark}

\begin{prop}\itemfix
    \begin{enumerate}
        \item \(0 \leq F_{\xi, \eta} (x, y) \leq 1\)
        \item \(F_{\xi, \eta} (x, y)\) --- неубывающая по каждому из аргументов.
        \item \(\lim\limits_{x \to -\infty} F(x, y) = 0, \lim\limits_{y \to -\infty} F(x, y) = 0, \lim\limits_{\substack{x \to +\infty \\ y \to +\infty}} F(x, y) = 1\). Интуитивно это следует из геометрического определения \(F\).
        \item \(F(x, y)\) непрерывна слева по каждому аргументу.
        \item Можно восстановить частное \textit{(маргинальное)} распределение:
              \[\lim_{x \to +\infty} F_{\xi, \eta}(x, y) = F_\eta(y) \quad \lim_{y \to +\infty} F_{\xi, \eta}(x, y) = F_\xi(x)\]
        \item Можно вычислить вероятность попадания в любой прямоугольник:
              \[P(x_1 \leq \xi < x_1, y_1 \leq \eta < y_2) = F(x_2, y_2) - F(x_1, y_2) - F(x_2, y_1) + F(x_1, y_1)\]
              \begin{proof}
                  Рассмотрим событие
                  \[\{\xi < x_2, \eta < y_2\} = \{x_1 \leq \xi < x_1, y_1 \leq \eta < y_2\} + \{\xi < x_1, \eta < y_2\} + \{\xi < x_2, \eta < y_2\}\]
                  Первое событие из правой части не совместно с вторым и третьим. Возьмём вероятность:
                  \[P(\xi < x_2, \eta < y_2) = P(x_1 \leq \xi < x_1, y_1 \leq \eta < y_2) + P(\{\xi < x_1, \eta < y_2\} + \{\xi < x_2, \eta < y_2\})\]
                  \begin{multline}
                      F(x_2, y_2) = P(x_1 \leq \xi < x_1, y_1 \leq \eta < y_2) + P(\xi < x_1, \eta < y_2) \\
                      + P(\xi < x_2, \eta < y_1) - P(\xi < x_1, \eta < y_2)
                  \end{multline}
                  \[P(x_1 \leq \xi < x_2, y_1 \leq \eta < y_2) = F(x_2, y_2) - F(x_1, y_2) - F(x_2, y_1) + F(x_1, y_1)\]
              \end{proof}

              Также можно привести тривиальное геометрическое доказательство.
    \end{enumerate}
\end{prop}

\subsection{Независимость случайных величин}

\begin{definition}
    Случайные величины \(\xi_1 \dots \xi_n\) \textbf{независимы в совокупности}, если для любого набора борелевских множеств \(B_1 \dots B_n \in \mathfrak{B}(\R)\) \(P(\xi_1 \in B_1 \dots \xi_n \in B_n) = P(\xi_1 \in B_1) \cdots P(\xi_n \in B_n)\)
\end{definition}

\begin{definition}
    Случайные величины \(\xi_1 \dots \xi_n\) \textbf{попарно независимы}, если независимы любые две из них.
\end{definition}

\begin{remark}
    Независимость в совокупности более сильная, чем попарная независимость, т.е. из независимости в совокупности следует попарная независимость. Обратное утверждение неверно.
\end{remark}

\begin{remark}
    В дальнейшем под независимыми понимаем независимые в совокупности.
\end{remark}
%</38>

\begin{remark}
    В дальнейшем мы изучаем только:
    \begin{enumerate}
        \item Дискретную систему двух случайных величин.
        \item Абсолютно непрерывную систему двух случайных величин.
    \end{enumerate}
\end{remark}

\subsection{Дискретная система двух случайных величин}

%<*39>
\begin{definition}
    Случайные величины \(\xi\) и \(\eta\) имеют дискретное совместное распределение, если случайный вектор \((\xi, \eta)\) принимает не более чем счётное число значений.
\end{definition}

То есть существует конечный или счётный набор пар чисел \((x_i, y_j)\), такой что:
\begin{enumerate}
    \item \(P(\xi = x_i, \eta = y_j) > 0\)
    \item \(\sum_{i, j} P(\xi = x_i, \eta = y_j) = 1\)
\end{enumerate}

Таким образом, двумерная случайная дискретного величина задается законом распределения --- таблицей вероятностей \(p_{ij} = P(\xi = x_i, \eta = y_j)\). Зная общий \textit{(совместный)} закон распределения, можно найти частный \textit{(маргинальный)} закон распределения случайных величин \(\xi\) и \(\eta\) по формулам:
\begin{itemize}
    \item \(\xi : p_i = \sum_{j = 1}^n p_{ij}\)
    \item \(\eta : q_j = \sum_{i_ = 1}^n p_{ij}\)
\end{itemize}

\begin{definition}
    Дискретные случайные величины \(\xi_1 \dots \xi_n\) \textbf{независимы}, если \(P(\xi_1 = x_1 \dots \xi_n = x_n) = P(\xi_1 = x_1) \cdots P(\xi_n = x_n)\).
\end{definition}

\begin{definition}
    Случайные величины \(\xi\) и \(\eta\) \textbf{независимы}, если во всех клетках распределения \(p_{ij} = p_i q_j\)
\end{definition}
%</39>

\begin{example}
    \unfinished
\end{example}

\subsection{Абсолютно непрерывная система двух случайных величин}

%<*40>
\begin{definition}
    Случайные величины \(\xi\) и \(\eta\) имеют \textbf{абсолютно непрерывное совместное распределение}, если существует функция \(f_{\xi, \eta} \geq 0\), такая что
    \[P((\xi, \eta) \in B) = \iint_B f_{\xi, \eta} dx dy \quad \forall B \in \mathfrak{B}(\R^n)\]
    При этом \(f_{\xi, \eta}\) называется \textbf{плотностью} совместного распределения.
\end{definition}

Геометрический смысл плотности: если \(B\) --- замкнутая ограниченная область, то \(P((\xi, \eta) \in B)\) есть объем цилиндрического тела, ограниченного снизу \(B\), а сверху \(f_{\xi, \eta}\).

\begin{prop}\itemfix
    \begin{enumerate}
        \item \(f_{\xi, \eta}(x, y) \geq 0\)
        \item Нормировка: \(\iint_{\R^2} f_{\xi, \eta}(x, y) dxdy = 1\)
        \item \(F_{\xi, \eta}(x, y) = \int_{-\infty}^x \int_{-\infty}^y f_{\xi, \eta}(x, y) dx dy\)
        \item \(f_{\xi, \eta}(x, y) = \frac{\partial^2 F_{\xi, \eta}(x, y)}{\partial x \partial y}\)
        \item Если случайные величины \(\xi, \eta\) имеют абсолютной непрерывное распределение \(f(x, y)\), то маргинальное распределение \(\xi\) и \(\eta\) также абсолютно непрерывные с плотностями \(f_\xi(x) = \int_{ -\infty}^{+\infty} f(x, y) dy, f_\eta(y) = \int_{ - \infty}^{+\infty} f(x, y) dx\)
              \begin{proof}
                  \[F_\xi(x) = \lim_{y \to +\infty} F_{\xi, \eta}(x, y) = \int_{ - \infty}^x \int_{ -\infty}^{+\infty} f(x, y) dy dx = \int_{ - \infty}^x f_\xi(x) dx = F_\xi(x) = \int_{ - \infty}^{+\infty} f(x, y) dy\]
                  \(f_\eta(y)\) --- аналогично.
              \end{proof}
        \item Абсолютно непрерывные случайные величины \(\xi_1 \dots \xi_n\) независимы\footnote{в совокупности} тогда и только тогда, когда плотность их совместного распределения существует и равна произведению их плотностей:
              \[f_{\xi_1 \dots \xi_n}(x_1 \dots x_n) = f_{\xi_1}(x_1) \cdots f_{\xi_n}(x_n)\]
              \begin{proof}
                  Докажем для случая \(n = 2\), общий случай аналогичен.

                  Случайные величины \(\xi\) и \(\eta\) независимые \(\Leftrightarrow\)
                  \begin{align*}
                      F_{\xi, \eta}(x, y) & =  F_\xi(x) F_\eta(y)                                                 \\
                                          & = \int_{ - \infty}^x f_\xi(x) dx \cdot \int_{ - \infty}^y f_\eta(y)dy \\
                                          & =  \int_{ - \infty}^x \int_{ - \infty}^y f_\xi(x) f_\eta(y) dx dy     \\
                                          & =  \int_{ - \infty}^x \int_{ - \infty}^y f_{\xi, \eta}(x, y) dx dy    \\
                  \end{align*}

                  Таким образом, \(f_{\xi, \eta}(x, y) = f_\xi(x) \cdot f_\eta(y)\)
              \end{proof}
    \end{enumerate}
\end{prop}

\begin{remark}
    Совместное распределение двух абсолютно непрерывных случайных величин не обязано быть абсолютно непрерывным --- оно может быть сингулярным.
\end{remark}

\begin{example}
    Бросаем точку на отрезок на плоскости. Первая координата --- \(\xi\), вторая координата --- \(\eta\). Мера отрезка на плоскости есть 0, при этом число точек несчётно. Таким образом, распределение \((\xi, \eta)\) сингулярное.
\end{example}
%</40>

\subsection{Многомерное равномерное распределение}

\begin{definition}
    Пусть \(D \in \R^n\) --- борелевское множество конечной меры Лебега \(\lambda D > 0\). Случайный вектор \(\xi_1 \dots \xi_n\) имеет \textbf{равномерное распределение} в области \(D\), если плотность в этой области постоянна, а вне её равна \(0\).
\end{definition}
