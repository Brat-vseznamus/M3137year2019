\chapter{1 мая}

%<*41>
Пусть \(\xi_1\) и \(\xi_2\) --- случайные величины с плотностью \(f(x, y)\) и \(g : \R^2 \to \R\). Что мы можем сказать про случайную величину \(g(\xi_1, \xi_2)\)?

\begin{theorem}
    Пусть \(z \in \R, D_z = \{(x, y)\ |\ g(x, y) < z\}\). Тогда случайная величина \(\eta = g(x, y)\) имеет функцию распределения:
    \[F_\eta(z) = \iint_{D_z} f(x, y) dx dy\]
\end{theorem}
\begin{proof}
    \[F_\eta = P(\eta < z) = P(g(\xi_1, \xi_2) < z) = P((\xi_1, \xi_2) \in D_z) = \iint_{D_z} f(x, y) dx dy\]
\end{proof}

\subsection{Формула свертки}

\begin{theorem}
    Пусть \(\xi_1, \xi_2\) --- независимые, абсолютно непрерывные случайные величины с плотностями \(f_{\xi_1}(x)\) и \(f_{\xi_2}(x)\) \\
    Тогда \(\xi_1 + \xi_2\) имеет абсолютно непрерывное распределение с плотностью
    \[ f_{\xi_1 + \xi_2}(t) = \int_{-\infty}^\infty f_{\xi_1}(t)\cdot f_{\xi_2}(t - x)\,dx \]
\end{theorem}
\begin{proof}
    Т.к. случайные величины \(\xi_1\) и \(\xi_2\) независимы, то плотность совместного распределения равна произведению плотностей: \(f_{\xi_1\xi_2}(x, y) = f_{\xi_1}(x)f_{\xi_2}(y)\). Применим предыдущую теорему для \(\eta = g(\xi_1, \xi_2) = \xi_1 + \xi_2\). Тогда \(D_z = \{(x, y) \in \R^2 \big| x + y < z\}\)
    \[ f_{\xi_1 + \xi_2}(z) = \iint_{D_z} f_{\xi_1\xi_2}(x, y)\,dx\,dy = \int_{-\infty}^\infty \,dx \int_{-\infty}^{t - x} f_{\xi_1}(x) f_{\xi_2}(y)\,dy = \]
    \[ \left[\begin{matrix}
                y = t - x            & t = y + x    & dy = dy \\
                y(-\infty) = -\infty & t(z - x) = z &
            \end{matrix}\right] = \int_{\-\infty}^\infty\,dx\int_{-\infty}^z f_{\xi_1}(x)f_{\xi_2}(t - x)\,dt = \]
    \[ = \int_{-\infty}^z \underbrace{\int_{-\infty}^\infty f_{\xi_1}(x)f_{\xi_2}(t - x)\,dx}_{f_{\xi_1 + \xi_2}(t)} \,dt \implies f_{\xi_1 + \xi_2}(t) = \int_{-\infty}^\infty f_{\xi_1}(x)f_{\xi_2}(t - x)\,dx \]
\end{proof}
%</41>

\section{Сумма стандартных распределений}

%<*42>
\begin{definition}
    Если сумма двух независимых случайных величин одного типа распределений также будет этого типа, то говорят что это распределение \textbf{устойчиво} относительно суммирования.
\end{definition}
\begin{example}
    Независимые случайные величины:
    \begin{itemize}
        \item \(\xi_1 \in B_{n, p}\)
        \item \(\xi_2 \in B_{m, p}\)
    \end{itemize}
    Тогда \(\xi_1 + \xi_2 \in B_{n + m, p}\)
\end{example}
\begin{proof}
    \(\xi_1 + \xi_2\) --- число успехов в серии из \(m + n\) испытаний, где \(p\) --- вероятность успеха при одном испытании. \(\xi_1 + \xi_2 \in B_{n + m, p}\)
\end{proof}
\begin{example}
    Независимые случайные величины:
    \begin{itemize}
        \item \(\xi_1 \in \Pi_\lambda\)
        \item \(\xi_2 \in \Pi_\mu\)
    \end{itemize}
    Тогда \(\xi_1 + \xi_2 \in \Pi_{\lambda + \mu}\)
\end{example}
\begin{proof}
    \[ P(\xi_1 + \xi_2 = k) = \sum_{i = 0}^k P(\xi_1 = i, \xi_2 = k - i) = \sum_{i = 0}^k p(\xi_1 = i)\cdot p(\xi_2 = k - i) = \]
    \[ = \sum_{i = 0}^k \frac{\lambda^i}{i!} e^{ - \lambda} \frac{\mu^{k - i}}{(k - i)!} e^{ - \mu} = e^{ - \lambda - \mu} \sum_{i = 0}^k \frac{\lambda^i}{i!} \frac{\mu^{k - i}}{(k - i)!} = \frac{(\lambda + \mu)^k}{k!} e^{ - (\lambda + \mu)} \]
\end{proof}
\begin{example}
    \(\xi_1, \xi_2 \in N(0, 1)\) --- независимые случайные величины. Тогда \(\xi_1 + \xi_2 \in N(0, 2)\)
\end{example}
\begin{proof}
    \begin{align*}
        f_{\xi_1 + \xi_2}(t) & = \int_{ -\infty}^{+\infty} \frac{1}{\sqrt{2\pi}} e^{ - \frac{x^2}{2}} \frac{1}{\sqrt{2\pi}} e^{ - \frac{(t - x)^2}{2}}               \\
                             & = \frac{1}{2\pi} \int_{ -\infty}^{+\infty} e^{ -(x^2 - tx + \frac{t^2}{2})} dx                                                        \\
                             & = \frac{1}{2\pi} e^{ - \frac{t^2}{4}} \int_{ -\infty}^{+\infty} e^{ - \left(x - \frac{t}{2}\right)^2} d\left( x - \frac{t}{2} \right) \\
                             & = \frac{1}{2\pi} e^{ - \frac{t^2}{4}} \sqrt{\pi}
    \end{align*}
\end{proof}
%</42>
\begin{example}\itemfix
    \begin{itemize}
        \item \(\xi_1 \in N(a_1, \sigma_1^2)\)
        \item \(\xi_2 \in N(a_2, \sigma_2^2)\)
    \end{itemize}
    Тогда \(\xi_1 + \xi_2 \in N(a_1 + a_2, \sigma_1^2 + \sigma_2^2)\)
\end{example}
\begin{example}
    \(\xi_1, \dots, \xi_n \in E_\alpha\) --- независимые случайные величины. Тогда \(\xi_1 + \dots + \xi_n \in \Gamma_{\alpha, n}\)
\end{example}
\begin{proof}
    По индукции:
    \begin{itemize}
        \item База: \(E_\alpha = \Gamma_{\alpha, 1}\)
        \item Переход: Пусть \(\xi_1 + \dots + \xi_{k - 1} = \Gamma_{\alpha, k - 1}\), тогда:
              \[ f_{\xi_{k - 1}}(x) = \begin{cases}
                      0                                                     & x \le 0 \\
                      \frac{\alpha^{k - 1}}{(x - 2)!}x^{k - 1}e^{-\alpha x} & x \ge 0
                  \end{cases} \]
              По формуле свертки
              \[ \unfinished \]
    \end{itemize}
\end{proof}
\begin{example}\itemfix
    \begin{itemize}
        \item \(\xi_1 \in \Gamma_{\alpha, \lambda_1}\)
        \item \(\xi_2 \in \Gamma_{\alpha, \lambda_2}\)
    \end{itemize}
    Тогда \(\xi_1 + \xi_2 \in \Gamma_{\alpha, \lambda_1 + \lambda_2}\)
\end{example}
\begin{proof}
    \unfinished
\end{proof}
\begin{example}
    \(\xi_1, \xi_2 \in U(0, 1)\)
\end{example}

\section{Условное распределение}

%<*43>
\begin{definition}
    \textbf{Условным распределением} случайной величины из системы случайных величин \((\xi, \eta)\) называется ее распределение найденное при условии, что другая случайная величина приняла определенное значение
\end{definition}
\begin{obozn}
    \(\xi | \eta = y\) --- \(\xi\) при условии что \(\eta\) приняла значение \(y\)
\end{obozn}
\begin{definition}
    Условным математическим ожиданием \(E(\xi | \eta = y)\) называется математическое ожидание случайной величины \(\xi\) при соответствующем условном распределении:
    \begin{enumerate}
        \item Условное распределение в дискретной системе двух случайных величин

              \[P(\xi = x_i|\eta = y_j) = \frac{P(\xi = x_i, \eta = y_j)}{P(\eta = y_j)} \]
        \item Условное распределение в непрерывной системе двух случайных величин

              Пусть двумерная абсолютно непрерывная случайная величина \((\xi, \eta)\) задана плотностью \(f_{\xi, \eta}(x, y)\). Тогда плотность условного распределения \(\xi | \eta = y\) будет равна:
              \[ f(x|y) = \frac{f_{\xi, \eta}(x, y)}{f_{\eta}(y)} \]
    \end{enumerate}
\end{definition}
\begin{definition}
    Функция \(f(x|y)\) называется \textbf{условной плотностью}. Аналогично \(f(y | x) = \frac{f_{\xi, \eta}(x, y)}{f_{\xi}(x)}\)
\end{definition}
\begin{lemma}
    Условное математическое ожидание вычисляется по формуле:
    \[ E(\xi | \eta = y) = \int_{-\infty}^\infty x\cdot f(x |y)\,dx \]
    Аналогично
    \[ E(\eta | \xi = x) = \int_{-\infty}^\infty y\cdot f(y | x)\,dy \]
\end{lemma}
%</43>
\begin{remark}
    При фиксированном значении переменной \(x\) \(f(y | x)\) будет функцией зависящей только от \(y\), а условное математическое ожидание будет числом. Если считать \(x\) переменной, то условное математическое ожидание является функцией зависящей от \(x\) и называется функцией \textbf{регрессии} \(\eta\) на \(\xi\). Т.к. \(\eta\) --- случайная величина, то \(E(\xi | \eta)\) можно рассматривать как случайную величину.
\end{remark}
