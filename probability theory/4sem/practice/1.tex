\documentclass[12pt, a4paper]{article}

\usepackage{lastpage}
\usepackage{mathtools}
\usepackage{xltxtra}
\usepackage{libertine}
\usepackage{amsmath}
\usepackage{amsthm}
\usepackage{amsfonts}
\usepackage{amssymb}
\usepackage{enumitem}
\usepackage{xcolor}
\usepackage[left=1.5cm, right=1.5cm, top=2cm, bottom=2cm, bindingoffset=0cm, headheight=15pt]{geometry}
\usepackage{fancyhdr}
\usepackage[russian]{babel}
% \usepackage[utf8]{inputenc}
\usepackage{catchfilebetweentags}
\usepackage{accents}
\usepackage{calc}
\usepackage{etoolbox}
\usepackage{mathrsfs}
\usepackage{wrapfig}

\providetoggle{useproofs}
\settoggle{useproofs}{false}

\pagestyle{fancy}
\lfoot{M3137y2019}
\rhead{\thepage\ из \pageref{LastPage}}

\newcommand{\R}{\mathbb{R}}
\newcommand{\Q}{\mathbb{Q}}
\newcommand{\C}{\mathbb{C}}
\newcommand{\Z}{\mathbb{Z}}
\newcommand{\B}{\mathbb{B}}
\newcommand{\N}{\mathbb{N}}

\newcommand{\const}{\text{const}}

\newcommand{\teormin}{\textcolor{red}{!}\ }

\DeclareMathOperator*{\xor}{\oplus}
\DeclareMathOperator*{\equ}{\sim}
\DeclareMathOperator{\Ln}{\text{Ln}}
\DeclareMathOperator{\sign}{\text{sign}}
\DeclareMathOperator{\Sym}{\text{Sym}}
\DeclareMathOperator{\Asym}{\text{Asym}}
% \DeclareMathOperator{\sh}{\text{sh}}
% \DeclareMathOperator{\tg}{\text{tg}}
% \DeclareMathOperator{\arctg}{\text{arctg}}
% \DeclareMathOperator{\ch}{\text{ch}}

\DeclarePairedDelimiter{\ceil}{\lceil}{\rceil}
\DeclarePairedDelimiter{\abs}{\left\lvert}{\right\rvert}

\setmainfont{Linux Libertine}

\theoremstyle{plain}
\newtheorem{axiom}{Аксиома}
\newtheorem{lemma}{Лемма}

\theoremstyle{remark}
\newtheorem*{remark}{Примечание}
\newtheorem*{exercise}{Упражнение}
\newtheorem*{consequence}{Следствие}
\newtheorem*{example}{Пример}
\newtheorem*{observation}{Наблюдение}

\theoremstyle{definition}
\newtheorem{theorem}{Теорема}
\newtheorem*{definition}{Определение}
\newtheorem*{obozn}{Обозначение}

\setlength{\parindent}{0pt}

\newcommand{\dbltilde}[1]{\accentset{\approx}{#1}}
\newcommand{\intt}{\int\!}

% magical thing that fixes paragraphs
\makeatletter
\patchcmd{\CatchFBT@Fin@l}{\endlinechar\m@ne}{}
  {}{\typeout{Unsuccessful patch!}}
\makeatother

\newcommand{\get}[2]{
    \ExecuteMetaData[#1]{#2}
}

\newcommand{\getproof}[2]{
    \iftoggle{useproofs}{\ExecuteMetaData[#1]{#2proof}}{}
}

\newcommand{\getwithproof}[2]{
    \get{#1}{#2}
    \getproof{#1}{#2}
}

\newcommand{\import}[3]{
    \subsection{#1}
    \getwithproof{#2}{#3}
}

\newcommand{\given}[1]{
    Дано выше. (\ref{#1}, стр. \pageref{#1})
}

\renewcommand{\ker}{\text{Ker }}
\newcommand{\im}{\text{Im }}
\newcommand{\grad}{\text{grad}}

\lhead{Теория вероятности \textit{(практика)}}
\cfoot{}
\rfoot{9.2.2021}

\begin{document}

\subsection*{Разбалловка}

Будут две \textit{(возможно три)} контрольные по 30 баллов, 10 баллов за посещения и 10 баллов за домашние задания. Будет возможность набрать дополнительные баллы. Максимальный автомат без экзамена --- 4.

\section*{Базовая теория вероятности}

Случайные события обозначаются буквами \(A, B, C\), каждому событию соответствует числовая харакетристика \(P(A)\) --- вероятность наступления события \(A\), при этом \(0 \leq P(A) \leq 1\).

Формула вероятности \(P(A) = \frac{m}{n}\), где \(n\) --- число всех возможных исходов, \(m\) --- число исходов, соответствующих событию \(A\).

\begin{exercise}
    Бросается кубик. Какова вероятность, что выпадет чётное число?

    \textbf{Ответ}: \(\frac{3}{6} = \frac{1}{2}\)
\end{exercise}

\begin{exercise}
    В коробке 4 красных и 3 синих карандаша. Вынули 3 из них. Найти вероятность того, что из них будут два красных, а один --- синий.

    Число всех возможных элементарных исходов:
    \[n = \binom{7}{3} = \frac{7!}{3!4!} = 35\]
    Число искомых исходов:
    \[m = \binom{4}{2} \binom{3}{1} = \frac{4!}{2!2!} \cdot \frac{3!}{1!2!} = 18\]

    \textbf{Ответ}: \(\frac{18}{35}\)
\end{exercise}

\begin{exercise}
    В ящике 5 чёрных шаров, 3 белых и 2 красных. Вынули половину из них. Найти вероятность того, что из них будут два белых и два чёрных.

    Число всех возможных элементарных исходов:
    \[n = \binom{10}{5} = \frac{10!}{5!5!} = 252\]
    Число искомых исходов:
    \[m = \binom{3}{2} \binom{5}{2} \binom{2}{1} = \frac{3!}{2!1!} \cdot \frac{5!}{3!2!} \cdot \frac{2!}{1!1!} = 60\]

    \textbf{Ответ}: \(\frac{60}{252} = \frac{10}{43} \)
\end{exercise}

\begin{exercise}
    На первом этаже шестиэтажного дома в лифт зашли трое человек. Какова вероятность того, что они выйдут на разных этажах? \textit{(Выйти на первом этаже нельзя)}

    Первый человек выходит где угодно, второй --- на любом из оставшихся четырех этажей, третий --- на любом из трех.

    \[1 \cdot \frac{4}{5} \cdot \frac{3}{5} = \frac{12}{25}\]

    Формальное решение:
    \[n = 5^3 = 125\]
    \[m = A_5^3 = 5\cdot 4\cdot 3 = 60\]
    \[P(A) = \frac{60}{125} = \frac{12}{25}\]

    \textbf{Ответ}: \(\frac{12}{25}\)
\end{exercise}

\begin{exercise}
    На полке расставляется 8 книг. Найти вероятность того, что 3 конкретные книги будут стоять рядом.

    \[n = 8! = 40320\]
    \[m = 6 \cdot 3! \cdot 5! = 4320\]
    \[P(A) = \frac{4320}{40320} = \frac{3}{28}\]

    \textbf{Ответ}: \(\frac{3}{28}\)
\end{exercise}

\begin{exercise}
    За круглым столом сидят 5 человек. Найти вероятность того, что два конкретных человека окажутся рядом.

    \[n = 5! = 120\]
    \[m = 5\cdot 2! \cdot 3! = 60\]
    \[P(A) = \frac{1}{2}\]

    \textbf{Ответ}: \(\frac{1}{2}\)
\end{exercise}

\begin{exercise}
    На доске поставили белую и черную ладьи. Найти вероятность того, что они не будут бить друг друга.

    \[n = 64 \cdot 63\]
    \[m = 64 \cdot 49\]
    \[P(A) = \frac{49}{63} = \frac{7}{9}\]

    \textbf{Ответ}: \(\frac{7}{9}\)
\end{exercise}

\begin{exercise}
    8 команд разбивают на две группы. Найти вероятность того, что две сильнейших команды окажутся в разных группах.

    \textbf{Ответ}: \(\frac{3}{7}\)
\end{exercise}

\begin{exercise}
    Бросаются два кубика. Найти вероятность того, что в сумме выпало не менее девяти очков.

    \[n = 36\]

    Искомых случаев \(5\cdot 2\): \((6, 6), (6, 5), (6, 4), (6, 3), (5, 4)\) и их перестановки.

    \[m = 5\cdot 2\]
    \[P(A) = \frac{10}{36} = \frac{5}{18}\]

    \textbf{Ответ}: \(\frac{5}{18}\)
\end{exercise}

\begin{exercise}
    Имеются карточки с буквами ``П'', ``Л'', ``И'', ``А''. Карточки выкладываются в ряд. Найти вероятность того, что получится осмысленное слово.

    \[n = 4! = 12\]
    Слов два: ``пила'' и ``липа''.

    \[m = 2\]

    \textbf{Ответ}: \(\frac{2}{12} = \frac{1}{6}\)
\end{exercise}

\subsection*{Домашнее задание}

\begin{exercise}
    В коробке 4 красных, 3 синих и 2 жёлтых карандаша. Из коробки вынули 6 карандашей. Найти вероятность того, что среди них будет поровну каждого цвета.
\end{exercise}

\begin{exercise}
    Известно, что у двух человек четыре козыря. Найти вероятность того, что они у них распределились в соотношении:
    \begin{enumerate}
        \item 2-2
        \item 3-1 или 1-3
        \item 4-0 или 0-4
    \end{enumerate}
\end{exercise}

\end{document}