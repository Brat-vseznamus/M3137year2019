\chapter{11 сентября}

План курса:
\begin{itemize}
    \item Полугруппа
    \item Группа
          \begin{itemize}
              \item Гомоморфизм
              \item Фактор-группа
              \item Теорема о ядре
              \item Произведение групп
          \end{itemize}
    \item Кольцо
          \begin{itemize}
              \item \(\mathbb{Z}\)
              \item Остатки
              \item Китайская теорема об остатках
              \item Алгоритм Евклида
              \item Кольцо многочленов
              \item Алгебра многочленов
          \end{itemize}
    \item Поле
          \begin{itemize}
              \item Поля Галуа
              \item Расширения Галуа
              \item Алгебраические кривые
              \item Диофантовы уравнения
          \end{itemize}
\end{itemize}

Начиная с групп мы будем использовать формализм теории категорий.

\section{Алгебраические структуры}

\subsection{Структуры с одним законом композиции}

Пусть \(M\) --- множество с законом композиции \(T : \forall x, y \in M \ \ \exists xTy \in M\).

\begin{remark}
    Такой закон называется \textbf{внутренним}, т.к. оба его аргумента \(\in M\).
\end{remark}

\begin{obozn}
    \(x \cdot y, x \circ y, x + y, x^y, x * y\)
\end{obozn}

Закон задает структуру на множестве.

\begin{definition}
    \(e_L \in M : \forall x \in M \ \ e_L \cdot x = x\) --- \textbf{левый нейтральный} элемент

    \(e_R \in M : \forall x \in M \ \ x \cdot e_R = x\) --- \textbf{правый нейтральный} элемент
\end{definition}

\begin{lemma}
    \(\exists e_L, e_R \in M \Rightarrow e_L = e_R \defeq e\)
\end{lemma}
\begin{proof}
    \(e_L = e_L \cdot e_R = e_R\)
\end{proof}

\begin{lemma}
    \(e, e'\) --- нейтральные элементы \( \Rightarrow e = e'\).
\end{lemma}
\begin{proof}
    \(e = e \cdot e' = e'\)
\end{proof}

\begin{definition}
    \(p \in M : p \cdot p = p\) --- \textbf{идемпотент}
\end{definition}

\begin{definition}
    \(z \in M : z \cdot x = z \cdot y \Rightarrow x = y\) --- \textbf{регулярный} элемент \textit{(левый)}
\end{definition}

\begin{definition}
    \(x \in M, \exists e \in M\). Элемент \(z \in M : z \cdot x = e\) --- \textbf{левый обратный} элемент к \(x\).

    \(y \in M : x \cdot y = e\) --- \textbf{правый обратный} элемент к \(x\).
\end{definition}

\begin{lemma}
    Если \(\exists y, z\), то \(y = z \defeq x^{-1}\) --- \textbf{обратный} элемент.
\end{lemma}
\begin{proof}
    \(z = z \cdot e = z \cdot (x \cdot y) = (z \cdot x) \cdot y = e \cdot y = y\). Здесь мы воспользовались \textbf{ассоциативностью} закона композиции.
\end{proof}

\begin{definition}
    \(\Theta_L : \forall x \in M \ \ \Theta_L \cdot x = \Theta_L\) --- \textbf{поглощающий} \textit{(слева)} элемент

    \(\Theta_R : \forall x \in M \ \ x \cdot \Theta_R = \Theta_R\) --- \textbf{поглощающий} \textit{(справа)} элемент
\end{definition}

\begin{lemma}
    \(\exists \Theta_L, \Theta_R \Rightarrow \Theta_L = \Theta_R \defeq \Theta\)
\end{lemma}
\begin{proof}
    \(\Theta_L = \Theta_L \cdot \Theta_R = \Theta_R\)
\end{proof}

\(\sphericalangle x,y,z \in M, x \cdot y \cdot z = (x \cdot y) \cdot z\) или \(x \cdot (y \cdot z)\). Какое выбрать? Без ассоциативности непонятно. Поэтому мы требуем ассоциативность в рамках этого курса.

То же самое можно сказать для семейства элементов.

\begin{theorem}[об ассоциативном законе]
    \(1 \leq k \leq n \Rightarrow T_{i = 1}^n x_i = \left( T_{i = 1}^k x_i \right) T \left( T_{i = k + 1}^n x_i \right)\)
\end{theorem}

\begin{definition}
    \(\sphericalangle \forall x, y \in M \ \ xTy = yTx\). Тогда \(T\) называется \textbf{коммутативным}.
\end{definition}

\begin{definition}
    \(\exists x, y \in M : xTy = yTx\). Тогда \(x, y\) называются \textbf{перестановочными} относительно закона.
\end{definition}

\begin{theorem}[об ассоциативном, коммутативном законе]
    Аргументы ассоциативного, коммутативного закона можно переставлять как угодно.
\end{theorem}

\subsection{Структуры с двумя законами композиции}

Пусть \(M\) --- множество с законами композиции \(*, \circ\). Нас интересует случай, когда эти два закона взаимосвязаны.

Как воспринимать \(x * y \circ z\)? Может иметь место \textbf{дистрибутивность} \(*\) относительно \(\circ\) \textit{(слева)}: \(x * (y \circ z) = (x * y) \circ (x * z)\)

\(\sphericalangle e\) --- нейтральный элемент по \(\circ\). \(\sphericalangle x * y = x * (e \circ y) = (x * e) \circ (x * y) \Rightarrow x * e = e\). Поэтому из поля нельзя убрать ноль.

\subsection{Основные алгебраические структуры}

\begin{itemize}
    \item \textbf{Полугруппа} --- множество с ассоциативным законом
    \item \textbf{Моноид} --- полугруппа с единицей
    \item \textbf{Группа} --- моноид с обратным элементом для любого
    \item \textbf{Абелева группа} --- группа с коммутативным законом
    \item \textbf{Кольцо} --- два закона, по первому --- абелева группа, по второму --- полугруппа
    \item \textbf{Поле} --- по двум законам группа
\end{itemize}
