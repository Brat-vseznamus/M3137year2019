\chapter{4 сентября}

\section{Вводная лекция}

Хотя этот курс формально называется ``теория чисел'', мы не будем рассматривать только теорию чисел. Теория чисел, разумеется, про числа, делители, простоту, алгоритм Евклида и т.д.. Однако, её можно обобщить на произвольные полугруппы, группы, кольца и поля. Поэтому мы будем рассматривать теорию чисел через призму общей алгебры.

Например, в кольце целых чисел есть понятие ``простое число''. А в каких ещё кольцах есть ``простые'' элементы и каким условиям эти кольца удовлетворяют? Оказывается, кольцо многочленов содержит простые элементы и поэтому там применим алгоритм Евклида.

Мы также затронем теорию категорий \textit{(терминальные объекты)}, алгебраическую геометрию \textit{(криптографию на эллиптических кривых)}.
