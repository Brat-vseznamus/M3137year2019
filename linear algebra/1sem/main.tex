\documentclass[12pt, a4paper]{article}

\usepackage{lastpage}
\usepackage{mathtools}
\usepackage{xltxtra}
\usepackage{libertine}
\usepackage{amsmath}
\usepackage{amsthm}
\usepackage{amsfonts}
\usepackage{amssymb}
\usepackage{enumitem}
\usepackage{xcolor}
\usepackage[left=1.5cm, right=1.5cm, top=2cm, bottom=2cm, bindingoffset=0cm, headheight=15pt]{geometry}
\usepackage{fancyhdr}
\usepackage[russian]{babel}
% \usepackage[utf8]{inputenc}
\usepackage{catchfilebetweentags}
\usepackage{accents}
\usepackage{calc}
\usepackage{etoolbox}
\usepackage{mathrsfs}
\usepackage{wrapfig}

\providetoggle{useproofs}
\settoggle{useproofs}{false}

\pagestyle{fancy}
\lfoot{M3137y2019}
\rhead{\thepage\ из \pageref{LastPage}}

\newcommand{\R}{\mathbb{R}}
\newcommand{\Q}{\mathbb{Q}}
\newcommand{\C}{\mathbb{C}}
\newcommand{\Z}{\mathbb{Z}}
\newcommand{\B}{\mathbb{B}}
\newcommand{\N}{\mathbb{N}}

\newcommand{\const}{\text{const}}

\newcommand{\teormin}{\textcolor{red}{!}\ }

\DeclareMathOperator*{\xor}{\oplus}
\DeclareMathOperator*{\equ}{\sim}
\DeclareMathOperator{\Ln}{\text{Ln}}
\DeclareMathOperator{\sign}{\text{sign}}
\DeclareMathOperator{\Sym}{\text{Sym}}
\DeclareMathOperator{\Asym}{\text{Asym}}
% \DeclareMathOperator{\sh}{\text{sh}}
% \DeclareMathOperator{\tg}{\text{tg}}
% \DeclareMathOperator{\arctg}{\text{arctg}}
% \DeclareMathOperator{\ch}{\text{ch}}

\DeclarePairedDelimiter{\ceil}{\lceil}{\rceil}
\DeclarePairedDelimiter{\abs}{\left\lvert}{\right\rvert}

\setmainfont{Linux Libertine}

\theoremstyle{plain}
\newtheorem{axiom}{Аксиома}
\newtheorem{lemma}{Лемма}

\theoremstyle{remark}
\newtheorem*{remark}{Примечание}
\newtheorem*{exercise}{Упражнение}
\newtheorem*{consequence}{Следствие}
\newtheorem*{example}{Пример}
\newtheorem*{observation}{Наблюдение}

\theoremstyle{definition}
\newtheorem{theorem}{Теорема}
\newtheorem*{definition}{Определение}
\newtheorem*{obozn}{Обозначение}

\setlength{\parindent}{0pt}

\newcommand{\dbltilde}[1]{\accentset{\approx}{#1}}
\newcommand{\intt}{\int\!}

% magical thing that fixes paragraphs
\makeatletter
\patchcmd{\CatchFBT@Fin@l}{\endlinechar\m@ne}{}
  {}{\typeout{Unsuccessful patch!}}
\makeatother

\newcommand{\get}[2]{
    \ExecuteMetaData[#1]{#2}
}

\newcommand{\getproof}[2]{
    \iftoggle{useproofs}{\ExecuteMetaData[#1]{#2proof}}{}
}

\newcommand{\getwithproof}[2]{
    \get{#1}{#2}
    \getproof{#1}{#2}
}

\newcommand{\import}[3]{
    \subsection{#1}
    \getwithproof{#2}{#3}
}

\newcommand{\given}[1]{
    Дано выше. (\ref{#1}, стр. \pageref{#1})
}

\renewcommand{\ker}{\text{Ker }}
\newcommand{\im}{\text{Im }}
\newcommand{\grad}{\text{grad}}

\usepackage{gensymb}
\usepackage{ulem}
\usepackage{sectsty}
\usepackage{tensor}

\subsectionfont{\fontsize{14}{15}\selectfont}
\setlength{\parindent}{0pt}
\setlength\arraycolsep{2pt}

\lhead{Линейная алгерба и аналитическая геометрия}
\cfoot{}

\DeclareMathSymbol{\mlq}{\mathord}{operators}{``}
\DeclareMathSymbol{\mrq}{\mathord}{operators}{`'}
\DeclarePairedDelimiter\Bracket{\lbrack}{\rbrack}

\newcommand{\doref}[1]{Дано выше. \textit{(\ref{#1}, стр. \pageref{#1})}}
\newcommand\Warning{%
 \makebox[1.4em][c]{%
 \makebox[0pt][c]{\raisebox{.1em}{\small!}}%
 \makebox[0pt][c]{\color{red}\Large$\bigtriangleup$}}}%

\begin{document}
\section{Векторная алгебра}
\subsection{Системы координат на плоскости и в пространстве.}
\begin{definition}
    \textbf{Координатной осью} называется ориентируемая прямая, имеющая начало отсчета $0$ и снабженная масштабом $E$.
\end{definition}
\begin{definition}
    \textbf{Система координат} --- прямоугольная, если угол между осями координат прямой.
\end{definition}
\begin{definition}
    \textbf{Декартова прямоугольная система координат} --- система координат с одинаковым масштабом по всем осям.
\end{definition}
\begin{definition}
    Система координат на плоскости называется \textbf{полярной}, если положение каждой точки задаётся полярным углом $\varphi$ и полярным радиусом $\rho$.
\end{definition}
Связь прямоугольной и полярной систем:
$$x=\rho\cos\varphi,\ \ y=\rho\sin\varphi$$
$$\rho=\sqrt{x^2+y^2},\ \ \cos\varphi=\frac{x}{\rho},\ \ \sin\varphi=\frac{y}{\rho}$$
\begin{definition}
    Система координат в пространстве называется \textbf{цилиндрической}, если положение каждой точки задаётся полярным углом $\varphi$, полярным радиусом $\rho$ и высотой над плоскостью $z$.
\end{definition}
Связь прямоугольной и цилиндрической систем такая же, как прямоугольной и полярной.
\begin{definition}
    Система координат называется \textbf{сферической}, если положение каждой точки определяется радиальным расстоянием $\rho$, азимутальным $\varphi$ и зенитным $\Theta$ углами.
\end{definition}
Связь прямоугольной и сферической систем:
$$x=\rho\cos\varphi\cos\Theta, \ \ y=\rho\sin\varphi\cos\Theta, \ \ z=\rho\sin\Theta$$
\subsection{Векторы и основные действия с ними \textit{(сложение, умножение на число)}.}
\begin{definition}
    Направленные отрезки \textbf{эквивалентны}, если они:
    \begin{enumerate}
        \item Лежат на параллельных прямых
        \item Сонаправлены
        \item Имеют одинаковые длины
    \end{enumerate}
\end{definition}
\begin{definition}
    \textbf{Вектор} --- класс эквивалентности направленных отрезков
\end{definition}
\begin{definition}
    \textbf{Сумма} двух векторов $\vec a$ и $\vec b$ --- вектор $\vec c=\vec a + \vec b$, полученный по правилу треугольника или параллелограмма
\end{definition}
\begin{definition}
    \textbf{Произведением} вектора $\vec a$ на число $\lambda$ является вектор $\vec b=\lambda\vec a$, такой что:
    \begin{enumerate}
        \item $|\vec b|=|\lambda||\vec a|$
        \item $\lambda>0\Rightarrow \vec a\uparrow\uparrow\vec b$
        \item $\lambda<0\Rightarrow \vec a\uparrow\mathrel{\mspace{-1mu}}\downarrow\vec b$
        \item $\lambda=0\Rightarrow \vec b=\vec 0$
    \end{enumerate}
\end{definition}

\subsection{Векторное введение координат. Координаты вектора.}
Для любого вектора $\vec a$, заданного на оси $l$ существует единственное представление $\vec a = x_a \cdot \vec e$, где $x_a \in \R$ и $|\vec e| = 1$. Тогда $\vec e$ --- \textbf{базис} и $x_a$ --- \textbf{координата} вектора $\vec a$ в базисе $\{\vec e\}$

\subsection{Свойства основных действий над векторами.}
\begin{enumerate}
    \item $\vec a + \vec b=\vec b + \vec a$ --- коммутативность
    \item $(\vec a + \vec b) + \vec c = \vec a + (\vec b + \vec c) = \vec a + \vec b + \vec c$ --- ассоциативность
    \item $\exists \vec 0 : \vec a + \vec 0 = \vec 0 + \vec a = \vec a$ --- нейтральный элемент
    \item $\forall \vec a \ \ \exists (-\vec a) : \vec a + (- \vec a) = (- \vec a) + \vec a = \vec 0$ --- наличие обратного элемента
    \item $\alpha(\beta\vec a) = \beta(\alpha \vec a) = (\alpha\beta)\vec a$ --- ассоциативность
    \item $(\alpha+\beta)\vec a = \alpha\vec a + \beta\vec a$ --- дистрибутивность
    \item $\alpha(\vec a + \vec b)=\alpha\vec a + \alpha\vec b$ --- дистрибутивность
\end{enumerate}

\subsection{Скалярное произведение векторов и его свойства.}
\begin{definition}
    \textbf{Угол} между $\vec a$ и $\vec b$ --- угол $\leq 180\degree$, заключенный между представителями соответствующих классов эквивалентности, отложенных от одной точки.
\end{definition}
\begin{definition}
    \textbf{Скалярное произведение} --- число, равное: $(\vec a, \vec b) = |\vec a||\vec b|\cos\varphi$
\end{definition}
Свойства скалярного произведения:
\begin{enumerate}
    \item $(\vec a, \vec b)=(\vec b, \vec a)$
    \item $(\vec a, \vec b)=0 \Leftrightarrow \vec a\perp\vec b,\quad \vec a,\vec b\not=0$
    \item $(\vec a, \vec b)=|\vec a|\cdot \text{Пр}_{\vec a}^{\perp} \vec b=|\vec b|\cdot \text{Пр}_{\vec b}^{\perp} \vec a$
    \item $(\lambda\vec a + \mu\vec b, \vec c)=(\lambda\vec a, \vec c)+(\mu\vec b, \vec c)$
\end{enumerate}
\subsection{Векторное произведение и его свойства.}
\begin{definition}
    Тройка $\{\vec a, \vec b, \vec c\}$ --- \textbf{правая}, если если располагаясь по направлению вектора $\vec c$ наблюдатель видит, что кратчайший поворот от $\vec a$ к
    $\vec b$ происходит по часовой стрелке.
\end{definition}
\begin{definition}
    \textbf{Векторное произведение} $\vec a$ и $\vec b$ --- вектор $\vec c$, такой что:
    \begin{enumerate}
        \item $|\vec c|=|\vec a||\vec b|\sin\angle(\vec a, \vec b)$
        \item $\vec c\perp \vec a, \vec c\perp \vec b$
        \item $\{\vec a, \vec b, \vec c\}$ --- правая тройка
    \end{enumerate}
\end{definition}
Свойства векторного произведения:
\begin{enumerate}
    \item $\vec a\times \vec b=-\vec b\times \vec a, \quad \vec a\times \vec b=\vec 0 \Leftrightarrow \vec a\parallel\vec b$
    \item $(\alpha\vec a)\times \vec b=\alpha(\vec a\times \vec b)=\vec a\times (\alpha\vec b)$
    \item $(\vec a + \vec b)\times \vec c=\vec a\times \vec c+\vec b\times \vec c$
\end{enumerate}

\subsection{Смешанное произведение векторов и его свойства.}
\begin{definition}
    \textbf{Смешанное произведение}: $(\vec a, \vec b, \vec c)=\vec a \cdot (\vec b \times \vec c)$
\end{definition}
Векторы $\vec a, \vec b$ и $\vec c$ компланарны тогда и только тогда, когда их смешанное произведение равно нулю.
Свойства смешанного произведения:
\begin{enumerate}
    \item $\vec a\cdot (\vec b \times \vec c)=\vec b\cdot (\vec c \times \vec a)=\vec c\cdot (\vec a \times \vec b)$
    \item $\vec a\cdot (\vec c \times \vec b)=-\vec a\cdot (\vec b \times \vec c)$
    \item $\vec a\cdot (\vec b \times \vec c)=(\vec a \times \vec b)\cdot \vec c$
\end{enumerate}

\subsection{Двойное векторное произведение и его свойства.}
$\vec a\times\vec b\times\vec c$

Свойства:
\begin{enumerate}
    \item $\vec a\times\vec b\times\vec c=\vec b(\vec a\cdot\vec c) - \vec c(\vec a\cdot \vec b)$ --- можно запомнить как \textit{``бац минус цаб''}
    \item Тождество Якоби:
          $$\vec a\times\vec b\times\vec c+\vec b\times\vec c\times\vec a + \vec c\times\vec a\times\vec b=\vec 0$$
\end{enumerate}

\subsection{Замена координат при переходе к новой системе отсчета. Матрица перехода.}
Переход от одного базиса $\{\vec e_1, \vec e_2\}$ к другому базису $\{\vec f_1, \vec f_2\}$: $$\begin{cases}
        \vec f_1=t_1^1\vec e_1+t_1^2\vec e_2 \\
        \vec f_2=t_2^1\vec e_1+t_2^2\vec e_2
    \end{cases}$$

$$\begin{pmatrix}
        \vec f_1 & \vec f_2
    \end{pmatrix} = \begin{pmatrix}
        \vec e_1 & \vec e_2
    \end{pmatrix} \begin{pmatrix}
        t_1^1 & t_1^2 \\
        t_2^1 & t_2^2 \\
    \end{pmatrix}$$

\begin{definition}
    $T$ называется \textbf{матрица перехода}
\end{definition}

С учётом переноса начала координат: $$X=A+T\cdot X'$$

Парралельный перенос: $T=\begin{pmatrix}
        1 & 0 \\
        0 & 1
    \end{pmatrix}, A=\begin{pmatrix}
        \alpha^1 \\
        \alpha^2
    \end{pmatrix}$

Сжатие-растяжение: $T=\begin{pmatrix}
        \lambda_1 & 0         \\
        0         & \lambda_2
    \end{pmatrix}, A=\vec 0$

Поворот на угол $\varphi: T=\begin{pmatrix}
        \cos\varphi & -\sin\varphi \\
        \sin\varphi & \cos\varphi
    \end{pmatrix}, A=\vec 0$

\section{Аналитическая геометрия}

\subsection{Уравнения линий и поверхностей.}
\begin{definition}
    \textbf{Уравнение линии} --- такое равенство, что координаты любой точки на линии удовлетворяют этому равенству и координаты любой точки не на линии не удовлетворяют.
\end{definition}
Способы задания линий в $\R^2$:
\begin{enumerate}
    \item В явном виде: $y=f(x), x=g(y)$
    \item В неявном виде: $F(x,y)=0$
    \item Параметрически: $x=x(t), y=y(t)$
\end{enumerate}
Способы задания линий в $\R^3$:
\begin{enumerate}
    \item В неявном виде: $\begin{cases}F_1(x,y,z)=0 \\ F_2(x,y,z)=0\end{cases}$
    \item Параметрически: $x=x(t), y=y(t), z=z(t)$
\end{enumerate}
Способы задания поверхностей в $\R^3$:
\begin{enumerate}
    \item В явном виде: $x=f(y,z);\ \ y=g(x,z);\ \ z=h(x,y)$
    \item В неявном виде: $F(x,y,z)=0$
    \item Параметрически: $x=x(u,v);\ \ y=y(u,v);\ \ z=z(u,v)$
\end{enumerate}

Эти способы задают не только линии и поверхности. \textit{(Также могут быть точки, множества линий/поверхностей и т.д.)}

\begin{definition}
    \textbf{Целый алгебраический полином} --- уравнение вида $$\sum\limits_{i=1}^k\alpha_ix^{m_i}y^{n_i},\ \ m_i,n_i\in\N$$

    \textbf{Порядок} такого полинома $p=\max_{i}\{m_i+n_i\}$
\end{definition}

\subsection{Уравнения прямой на плоскости и в пространстве. Взаимное расположение прямых.}
\subsubsection{Уравнения прямой на плоскости и в пространстве}
Возьмём произвольную прямую линию $l$ в $\R^2$ или $\R^3$. Пусть $\vec s=\begin{pmatrix} m & n & p\end{pmatrix}^T$ --- направляющий вектор этой прямой. Зафиксируем точку $M_0$ на $l$ с радиус-вектором $\vec r_0=\begin{pmatrix}x_0&y_0&z_0\end{pmatrix}^T$ и выразим точку $M$ на $l$ с радиус-вектором $\vec r=\begin{pmatrix}x&y&z\end{pmatrix}^T$.

$\overrightarrow{M_0M} || \vec s$, т.к. они параллельны $l \Rightarrow \exists t\in\R:\overrightarrow{M_0M}=t\cdot\vec s$

$\overrightarrow{M_0M}=\vec r-\vec r_0$ \textit{(по определению)} $\Rightarrow \vec r-\vec r_0=t\cdot\vec s\Rightarrow \vec r=\vec r_0+t\cdot \vec s$

\begin{definition}
    Это \textbf{векторное уравнение прямой}.
\end{definition}

Домножим обе части векторно на $\vec s$.
$$\vec r\times\vec s=\vec r_0\times\vec s+t\cdot\vec s\times\vec s$$
$$\vec r\times\vec s=\vec r_0\times\vec s$$
$$\vec b:=\vec r_0\times\vec s, \quad \vec r\times\vec s=\vec b$$

Спроектируем векторное уравнение прямой на каждую из осей, получим параметрическое уравнение прямой:
$$\begin{cases}
        x=x_0+m\cdot t \\
        y=y_0+n\cdot t \\
        z=z_0+p\cdot t \\
    \end{cases}$$

Выразим из каждого уравнения $t$ и приравняем:
$$\frac{x-x_0}{m}=\frac{y-y_0}{n}=\frac{z-z_0}{p}$$

Если даны две точки на прямой $M_0$ и $M_1$ с радиус-вектором $\vec r_1=\begin{pmatrix}x_1& y_1& z_1\end{pmatrix}^T$, можем получить направляющий вектор для $l$, который равен $\vec r_1-\vec r_0$. Подставим это в векторное уравнение прямой:
$$\vec r=\vec r_0 + (\vec r_1 - \vec r_0)t$$

Сделаем переход, аналогичный предыдущему:
$$\begin{cases}
        x=x_0+(x_1-x_0)\cdot t \\
        y=y_0+(y_1-y_0)\cdot t \\
        z=z_0+(z_1-z_0)\cdot t \\
    \end{cases}$$
$$\frac{x-x_0}{x_1-x_0}=\frac{y-y_0}{y_1-y_0}=\frac{z-z_0}{z_1-z_0}$$

\subsubsection{Взаимное расположение прямых}
Угол между прямыми равен углу между их направляющими векторами, а также равен углу между нормалями к этим векторам:
$$\cos\varphi=\left|\frac{(\vec s_1, \vec s_2)}{|\vec s_1||\vec s_2|}\right|=\left|\frac{(\vec n_1, \vec n_2)}{|\vec n_1||\vec n_2|}\right|$$

Парралельность прямых:
$$L_1||L_2\Leftrightarrow \vec s_1||\vec s_2 \Leftrightarrow \vec s_2=\alpha \vec s_1\Leftrightarrow\frac{m_2}{m_1}=\frac{n_2}{n_1}=\frac{p_2}{p_1}$$

Перпендикулярность:
$$L_1\perp L_2\Leftrightarrow \vec s_1\perp\vec s_2\Leftrightarrow (\vec s_1, \vec s_2)=0\Leftrightarrow m_1m_2+n_1n_2+p_1p_2=0$$

\subsection{Частные виды уравнений прямой на плоскости. Расстояние от точки до прямой.}
\subsubsection{Уравнение прямой с угловым коэффициентом}
$$\frac{x-x_0}{m}=\frac{y-y_0}{n}$$
$$y-y_0=\frac{n}{m}(x-x_0)$$
Этот переход можно делать, только если $m\not=0$
$$k:=\frac{m}{n}=\tg \alpha \quad \alpha\text{--- угол между прямой и осью } x$$
$$b:=y_0-kx_0, \quad y=kx+b$$
Геометрический смысл $b$ --- длина отрезка между началом координат и точкой пересечения прямой с осью $y$.
\subsubsection{Уравнение прямой в отрезках на осях}
Возьмем $a$ --- длина отрезка между началом координат и точкой пересечения прямой с осью $x$. Тогда:
$$a:=\frac{-b}{k}$$
$$\frac{x}{a}+\frac{y}{b}=1$$
\subsubsection{Нормальное уравнение прямой}
Возьмём нормаль к $\vec s$ --- это будет $\vec n$, при этом его возьмём таким, что:
$$|\vec n|=1, \quad \angle(\vec n, \vec r_0)<\frac{\pi}{2}$$

Домножим векторное уравнение на $\vec n$:
$$(\vec r-\vec r_0,\vec n)=(\vec s,\vec n)$$
$$(\vec r-\vec r_0,\vec n)=0$$
$$(\vec r,\vec n) - (\vec r_0,\vec n)=0$$
$$\vec n:=(\cos \alpha, \cos \beta), \quad p:=(\vec r_0, \vec n)$$
$$x\cos\alpha+y\cos\beta-p=0$$
$p$ --- \textbf{прицельный параметр}, его геометрический смысл - расстояние
от начала отсчета до прямой.

Возьмём произвольную нормаль к $\vec s$ --- вектор $\vec N=(A,B)$
$$A(x-x_0)+B(y-y_0)=0$$
$$C:=Ax_0+By_0,\quad Ax+By+C=0$$

\subsubsection{Расстояние от точки до прямой}
Найдём расстояние между точкой $M$ и прямой $L$, заданной уравнением с прицельным параметром.
$$L:x\cos\alpha+y\cos\beta-p=0$$

Проведем $L_1||L$ через $M$:
$$L_1:x\cos\alpha+y\cos\beta-p_1=0, \quad p_1=x_1\cos\alpha+y_1\cos\beta$$
$$\rho(L,M)=\rho(L,L_1)=|p_1-p|=|x_1\cos\alpha+y_1\cos\beta-p|$$

\subsection{Уравнения плоскости в пространстве. Взаимное расположение плоскостей.}
\subsubsection{Векторное параметрическое уравнение плоскости}
Зададим плоскость двумя непараллельными векторами $\vec s_1$ и $\vec s_2$ и точкой $M_0(x_0,y_0,z_0)$. Выразим произвольную точку $M$ из плоскости с радиус-вектором $\vec r=(x,y,z)$:
$$\vec r=\vec r_0+t_1\vec s_1+t_2\vec s_2$$
\begin{definition}
    Это \textbf{векторное параметрическое уравнение плоскости}.
\end{definition}
Спроектируем на каждую ось:
$$\begin{cases}
        x=x_0+t_1m_1+t_2m_2 \\
        y=y_0+t_1n_1+t_2n_2 \\
        z=z_0+t_1p_1+t_2p_2 \\
    \end{cases}$$

\subsubsection{Общее уравнение плоскости}
Умножим векторное уравнение на $\vec n=\vec s_1\times\vec s_2$ скалярно:
$$(\vec r, \vec n)=(\vec r_0+t_1\vec s_1+t_2\vec s_2, \vec n)$$
$$(\vec r-\vec r_0, \vec n)=(t_1\vec s_1+t_2\vec s_2, \vec n)$$
$$(\vec r-\vec r_0, \vec n)=0$$

В проекции:
$$A(x-x_0)+B(y-y_0)+C(z-z_0)=0$$
$$D:=-Ax_0-By_0-Cz_0, \quad Ax+By+Cz+D=0$$

\subsubsection{Уравнение плоскости, проходящей через три точки}
По аналогии с $\R^2$:
$$(\vec r-\vec r_1,\vec r_2-\vec r_1,\vec r_3-\vec r-1)=0$$

\subsubsection{Нормальное уравнение плоскости}
По аналогии с $\R^2$:
$$p:=(\vec r_0, \vec n)=\text{Пр}^\perp_{\vec n},\quad x\cos\alpha+y\cos\beta+z\cos\gamma-p=0$$

\subsubsection{Угол между плоскостями}
Это угол между нормалями плоскостей.

\subsubsection{Парралельность плоскостей}
$$\mathcal L_1||\mathcal L_2\Leftrightarrow \vec n_1||\vec n_2\Leftrightarrow \vec n_1=\alpha\vec n_2\Leftrightarrow \frac{A_1}{A_2}=\frac{B_1}{B_2}=\frac{C_1}{C_2}=\frac{D_1}{D_2}$$

\subsubsection{Перпендикулярность плоскостей}
$$\mathcal L_1\perp \mathcal L_2\Leftrightarrow \vec n_1\perp\vec n_2\Leftrightarrow (\vec n_1,\vec n_2)=0\Leftrightarrow A_1A_2+B_1B_2+C_1C_2=0$$

\subsection{Взаимное расположение прямой и плоскости в пространстве.}
\subsubsection{Угол между прямой и плоскостью}
$$\sin\varphi=\left|\frac{(\vec s, \vec n)}{|\vec s||\vec n|}\right|$$
\subsubsection{Парралельность прямой и плоскости}
$$\mathcal L||L\Leftrightarrow \vec s\perp \vec n\Leftrightarrow (\vec s,\vec n)=0$$
\subsubsection{Перпендикулярность прямой и плоскости}
$$\mathcal L\perp L\Leftrightarrow \vec s\ ||\, \vec n\Leftrightarrow \vec n=\alpha\vec s\Leftrightarrow\frac{A}{m}=\frac{B}{n}=\frac{C}{p}$$

\subsection{Эллипс: геометрическое определение, каноническое уравнение, симметрия и форма эллипса.}
\begin{definition}
    \textbf{Эллипс} --- множество точек, сумма расстояний от которых до двух заданных точек \textit{(фокусов)} --- постоянная величина.
\end{definition}

Пусть фокусы --- точки $F_1, F_2$, $|F_1F_2|=2с$ --- фокусное расстояние, $\vec r_1, \vec r_2$ --- векторы от точки эллипса до фокусов. Тогда по определению:
$$|\vec r_1|+|\vec r_2|=2a=const$$
$\varepsilon=\frac{c}{a}$ --- \textbf{эксцентриситет} эллипса.
$$0\leq c\leq a\Rightarrow 0\leq \varepsilon\leq 1$$
При $c=0$ эллипс --- окружность, при $c=a$ --- отрезок.

Каноническое уравнение эллипса:
$$b^2:=a^2-c^2,\quad \frac{x^2}{a^2}+\frac{y^2}{b^2}=1$$
$a$ --- длина \textbf{большой полуоси} эллипса, $b$ --- длина \textbf{малой полуоси}.

Очевидны предельные значения координат эллипса:
$$|x|\leq a, \ \ |y|\leq b$$
Кроме того, эллипс симметричен относительно обеих осей и относительно начала отсчета.

\subsection{Эллипс: полярное
    уравнение, параметрические уравнения, директрисы, уравнение касательной к
    эллипсу.}

\subsubsection{\textcolor{red}{Полярное уравнение}}

\subsubsection{Параметрическое уравнение эллипса}
$$\begin{cases}
        x=a\cos t \\
        y=b\sin t
    \end{cases}$$
Это уравнение можно проверить подстановкой в каноническое.

\subsubsection{Уравнение касательной к эллипсу}
Касательная к эллипсу в точке $M_0(x_0,y_0)$ задается следующим уравнением:
$$\frac{xx_0}{a^2}+\frac{yy_0}{b^2}=1$$

\subsubsection{Директрисы}
\begin{definition}
    \textbf{Директрисами} эллипса называются прямые, параллельные малой оси эллипса и проходящие от нее на расстоянии $\frac{a}{\varepsilon}$.
\end{definition}
$$d_1:=\frac{a}{\varepsilon}+x,\quad d_2:=\frac{a}{\varepsilon}-x$$
$$\frac{r_1}{d_1}=\frac{r_2}{d_2}=\varepsilon$$

\subsection{Окружность.}
\begin{definition}
    \textbf{Окружность} --- частный случай эллипса при $c=0,\ \  r_1=r_2=r$
\end{definition}
Свойства:
\begin{enumerate}
    \item $r_1+r_2=2a=2r, \ \ a=b=r, \ \ \varepsilon=0$
    \item Каноническое уравнение:
          $$\frac{x^2}{r^2}+\frac{y^2}{b^2}=1\Rightarrow x^2+y^2=r^2$$
    \item $|x|\leq r, \ \ |y|\leq r$
    \item $\rho =r$
    \item $x=r\cos t, \ \ y=r\sin t$
    \item Касательная в точке $M_0(x_0, y_0):\ \ \frac{xx_0}{r^2}+\frac{yy_0}{r^2}=1$
\end{enumerate}

\section{Алгебраические структуры. СЛАУ}
\subsection{Алгебраические структуры: группа, кольцо, поле}
\begin{definition}
    \textbf{Полугруппа} --- множество $G$ с заданной на нём бинарной ассоциативной замкнутой операцией $\circ$, т.е. $$(g_1\circ g_2)\circ g_3 = g_1\circ (g_2 \circ g_3)$$
\end{definition}
\begin{definition}
    \textbf{Группа} --- полугруппа, где выбран нейтральный элемент и для каждого элемента есть обратный:
    \begin{enumerate}
        \item Нейтральный элемент $e: e\circ g=g\circ e=g$
        \item Обратный элемент: $\forall g\in G \ \ \exists g^{-1} \ \ g\circ g^{-1}=g^{-1}\circ g=e$
    \end{enumerate}
\end{definition}
\begin{definition}
    \textbf{Абелева группа} --- группа с коммутативной операцией, т.е. $$\forall g_1, g_2\in G \ \ g_1\circ g_2=g_2\circ g_1$$
\end{definition}
\begin{definition}
    \textbf{Кольцо} --- множество с двумя бинарными операциями $\{R, \mlq+\mrq, \mlq\cdot\mrq\}$, которое является абелевой группой относительно сложения, полугруппой относительно умножения и эти операции согласованны \textit{(дистрибутивны)}:
    $$r_1\cdot(r_2+r_3)=r_1\cdot r_2+r_1\cdot r_3;\quad (r_2+r_3)\cdot r_1=r_2\cdot r_1+r_3\cdot r_1$$
\end{definition}
\begin{definition}
    \textbf{Поле} --- множество с двумя бинарными операциями $\{R, \mlq+\mrq, \mlq\cdot\mrq\}$, где эти операции согласованны и:
    \begin{enumerate}
        \item $\{K, \mlq + \mrq\}$ --- абелева группа
        \item $\{K\setminus\{0\}, \mlq \cdot \mrq\}$ --- абелева группа
    \end{enumerate}
\end{definition}
\subsection{Алгебраические структуры: линейное пространство, алгебра}
\begin{definition}
    \textbf{Модуль над кольцом $R$} --- абелева группа $\{G, \mlq + \mrq\}$ с операцией $R\times G\to G$, записываемой как $rg$ и для которой выполняется следующее:
    \begin{enumerate}
        \item $(r_1+r_2)g=r_1g+r_2g$
        \item $r(g_1+g_2)=rg_1+rg_2$
        \item $(r_1r_2)g=r_1(r_2g)$
    \end{enumerate}
\end{definition}
\begin{definition} \label{linear_space}
    \textbf{Линейное пространство} --- модуль над кольцом, которое также является полем.
\end{definition}
\begin{definition}
    \textbf{Вектор} --- элемент линейного пространства.
\end{definition}
\begin{definition}
    \textbf{Алгебра} --- модуль над кольцом, где сам модуль также является кольцом.
\end{definition}
\subsection{Поле комплексных чисел}
$i^2:=-1$

$\mathbb{C} = \{a+bi : \forall a,b\in\R\}$

\textbf{Модуль} комплексного числа $c$: $|c|=r=\sqrt{a^2+b^2}$, если $c=a+bi$

\textbf{Аргумент} комплексного числа $c$: $\varphi=\arg(c)=\arg(a+bi)=2\arctan\left(\frac{b}{\sqrt{a^2+b^2}+a}\right)$

Тогда $c=r(\cos \varphi + i\sin \varphi)$

\textbf{Дополнение} комплексного числа $c$ записывается как $\overline c = \overline{a+bi}=a-bi$
\subsection{Линейное пространство. Примеры линейных пространств.}
\doref{linear_space}

Примеры: \begin{enumerate}
    \item $X=\{x=\begin{pmatrix}
                  \xi^1 & \ldots & \xi^n
              \end{pmatrix}^T, \xi^i\in\R\}$ \textit{(или $\mathbb{C}$)}
    \item $\mathcal P_n=\{\text{многочлены } p(t):\deg p(t)\leq n, n\in\N\}$
\end{enumerate}
\subsection{Линейная зависимость векторов. Основные леммы о линейной зависимости.}
\begin{definition}
    \textbf{Линейной комбинацией} называется следующее выражение: $$\alpha_1x_1+\ldots+\alpha_nx_n,$$ где $\{x_i\}_{i=1}^n$ --- вектора, $\{\alpha_i\}_{i=1}^n$ --- коэффициенты.
\end{definition}
\begin{definition}
    Набор векторов $\{x_i\}_{i=1}^n$ называется \textbf{линейнонезависимым}, если не существует его линейной комбинации, где не все коэффициенты равны $0$, а сама комбинация равна $0_X$:
    $$\nexists \{\alpha_i\}_{i=1}^n : \exists i : \alpha_i\not=0 \quad \alpha_1x_1+\ldots+\alpha_nx_n=0_X$$

    Иначе набор называется \textbf{линейно зависимым}
\end{definition}
\begin{lemma}
    Любой набор, содержащий нулевой вектор, является линейнозависимым.
\end{lemma}
\begin{lemma}
    Набор, содержащий линейнозависимый поднабор, является линейнозависимым.
\end{lemma}
\begin{lemma}
    Любой поднабор линейнонезависимого набора также является линенйнонезависимым.
\end{lemma}
\begin{lemma}
    Набор векторов линейнозависим тогда и только тогда, когда хотя бы один из векторов набора выражается линейной комбинацией остальных.

    $$\exists k\in\{1\ldots n\}: x_k=\sum\limits_{i=1, i\not=k}^n\alpha^ix_i \Leftrightarrow \{x_i\}_{i=1}^n \text{ --- ЛЗ}$$
\end{lemma}
\subsection{Базис и размерность линейного пространства.}
\begin{definition}
    Набор векторов называется \textbf{полным} в линейном пространстве $X$, если любой вектор этого пространства можно выразить как линейную комбинацию этого набора:
    $$\forall x\in X \quad \exists \{\alpha_i\}_{i=1}^n \quad x=\sum\limits_{i=1}^n\alpha^ix_i$$
\end{definition}
\begin{definition}
    Набор векторов называется \textbf{базисом} пространства $X$, если он является полным и ЛНЗ.
\end{definition}
\begin{definition}
    Линейное пространство называется \textbf{конечномерным}, если в нём существует конечный полный набор векторов
\end{definition}
\begin{definition}
    \textbf{Размерность пространства} $\dim X$ --- количество векторов в его базисе.
\end{definition}
\subsection{Изоморфизм линейных пространств.}
\begin{definition}
    \textbf{Изоморфизм} --- биекция, сохраняющая линейность, установленная между двумя линейными пространствами над одним и тем же полем:

    $$\begin{cases}
            x_1\leftrightarrow y_1 \\
            x_2\leftrightarrow y_2
        \end{cases} \Rightarrow \begin{cases}
            x_1+x_2\leftrightarrow y_1+y_2 \\
            \alpha x_1\leftrightarrow \alpha y_1
        \end{cases}$$
\end{definition}
\subsection{Подпространства линейного пространства: определение, примеры, линейная оболочка, линейное многообразие.}
\begin{definition}
    \textbf{Подпространство} линейного пространства $X$ --- замкнутое множество $L\subset X$
\end{definition}
\begin{example}
    \begin{enumerate}
        \item $X$ и $\{0\}$ называются тривиальными подпространствами
        \item Прямая и плоскость, содержащие начало координат --- подпространство $E_3$
        \item $\R^{m<n}$ --- подпространство $\R^n$
        \item Множество симметричных $n\times n$ матриц --- подпространство $\R^n_n$
        \item Множество полиномов с членами только чётных степеней --- подпространство $\mathcal{P}_n$
    \end{enumerate}
\end{example}
\begin{definition}
    \textbf{Линейная оболочка} набора векторов --- множество всех линейных комбинаций этих векторов: $$\mathcal{L}(x_1\ldots x_n)=\left\{\sum\limits_{i=1}^k \alpha^ix_i \ \ |\ \ \forall \alpha_1\ldots \alpha_n\right\}$$
\end{definition}
\begin{definition}
    \textbf{Линейное многообразие}, параллельное подпространству $L$ линейного пространства $X$ --- множество $M$: $$M=\{y\in X : y=x_0+x \quad \forall x\in L\},\ x_0\in X$$
\end{definition}
\subsection{Подпространства линейного пространства: сумма и пересечение подпространств, прямая сумма, дополнение.}
\begin{definition}
    \textbf{Пересечение подпространств} $L_1$ и $L_2$ --- множество $L'$, такое что: $$L'=\{x\in X : x\in L_1 \text{ и } x\in L_2\}$$
\end{definition}
\begin{definition}
    \textbf{Сумма подпространств} $L_1$ и $L_2$ --- множество $L''$, такое что: $$L'э=\{x\in X : x=x_1+x_2 \ \ \forall x_1 \in L_1, x_2 \in L_2\}$$
\end{definition}
\begin{definition}
    \textbf{Прямая сумма подпространств} $L_1$ и $L_2$ --- множество $L=L_1\dot+L_2$, такое что: $$L=\{x\in X : x!=x_1+x_2 \ \ \forall x_1 \in L_1, x_2 \in L_2\}$$
\end{definition}
\begin{definition}
    Если $X=L_1\dot+L_2$, $L_1$ --- \textbf{дополнение} $L_2$ до $X$
\end{definition}
\subsection{Линейные алгебраические системы. Геометрическое исследование систем. Теорема Крамера \textit{(геометрическая формулировка)}.}
\begin{definition}
    $$\begin{cases}
            \alpha_1^1\xi^1+\alpha_2^1\xi^2+\ldots+\alpha_n^1\xi^n=\beta^1 \\
            \alpha_1^2\xi^1+\alpha_2^2\xi^2+\ldots+\alpha_n^2\xi^n=\beta^2 \\
            \vdots                                                         \\
            \alpha_1^m\xi^1+\alpha_2^m\xi^2+\ldots+\alpha_n^m\xi^n=\beta^m
        \end{cases}$$ --- \textbf{линейная алгебраическая система}, $\alpha$ --- коэффициенты, $\beta$ --- свободные члены, $\xi$ --- неизвестные
\end{definition}
\begin{definition}
    \textbf{Решение системы} --- такой набор, при подстановке которого равенства становятся верными.
\end{definition}
\begin{definition}
    \textbf{Совместная система} --- система, у которой есть решение.
\end{definition}
\begin{definition}
    \textbf{Определенная система} --- совместная система, которая имеет единственное решение.
\end{definition}
\begin{definition}
    \textbf{Однородная система} --- система, у которой все свободные члены равны $0$.
\end{definition}
Запишем в векторной форме: $\sum\limits_{i=1}^n a_i\xi^i=b$

\begin{theorem}
    Если $m=n$ и $\{a_i\}_{i=1}^n$ --- ЛНЗ, система совместна и определена, т.е. есть единственное решение.
\end{theorem}
\subsection{Геометрическое исследование систем. Теорема Кронекера-Капелли \textit{(геометрическая формулировка)} и ее следствия.}
Рассмторим случай, когда $\dim\mathcal{L}\{a_1\ldots a_n\}=r\leq n$. Тогда можно занумеровать $a$ так, что $\{a_i\}_{i=1}^r$ --- ЛНЗ и переписать систему как: $$a_1\xi^1+\ldots+a_r\xi^r=b-a_{r+1}\xi^{r+1}-\ldots-a_n\xi^n$$
\begin{theorem}
    $b\in\mathcal L \Leftrightarrow$ система совместна \textit{(можно все $\xi$ справа занулить и представить b через левую часть)}. Если $r=m$, система определена, иначе --- нет.
\end{theorem}
\begin{corollary}
    Однородная система:
    \begin{enumerate}
        \item Всегда совместна, т.к. существует тривиальное решение
        \item Имеет нетривиальные решения тогда и только тогда, когда $r<m$
        \item Является неопределенной тогда и только тогда, когда $m<n$
    \end{enumerate}
\end{corollary}
\subsection{Альтернатива Фредгольма для линейной системы уравнений.}
\begin{theorem}
    Если $m=n$, то:
    \begin{enumerate}
        \item Или однородная система имеет только тривиальное решение, и неоднородная система совместна и определена для любого $b$
        \item Или существуют нетривиальные решения однородной системы и неоднородная система совместна не при любых $b$
    \end{enumerate}
    Пояснение: в первом случае $\{a_i\}$ ЛНЗ $\Rightarrow \forall b$ можно выразить как линейную комбинацию $\{a_i\}$ единственным образом. Во втором случае $\{a_i\}$ ЛЗ $\Rightarrow$ не любой $b$ можно выразить как линейную комбинацию $\{a_i\}$.
\end{theorem}
\subsection{Фундаментальная система решений линейной однородной системы. Общее решение однородных и неоднородных систем.}
\begin{definition}
    \textbf{Фундаментальной системой решений} линейной однородной системы уравнений называется любая система из $n − r$ линейнонезависимых решений этой системы, то есть базис пространтва решений однородной системы.
\end{definition}
Любое решение можно представить в виде общего решения: $$z=z'+\sum\limits_{i=1}^n c_ix_i,$$ где $\{x_i\}_{i=1}^n$ --- ФСР.

\section{Полилинейные формы. \sout{Определители}}

\subsection{Перестановки.}
\begin{definition}
    \textbf{Перестановкой} из $a_1, a_2, \ldots , a_n$ из первых $n$ чисел натурального ряда называется расположение их в некотором фиксированном порядке.
\end{definition}
\begin{definition}
    Перестановка $1,2,\ldots, n$ --- \textbf{базовая}.
\end{definition}
\begin{definition}
    \textbf{Транспозиция перестановки} $t^p_q$ --- обмен местами двух элементов этой перестановки.
\end{definition}
\begin{definition}
    \textbf{Беспорядок\textit{(инверсия)}} в перестановке --- когда большее число стоит перед меньшим.
\end{definition}
\begin{definition}
    Чётность числа беспорядков в перестановке $\Leftrightarrow$ чётность перестановки
\end{definition}

\subsection{Отображения. Линейные формы. Сопряженное пространство.}
\begin{definition}
    \textbf{Отображение} из $X$ в $Y$ ($f:X\to Y$) сопоставляет каждому $x\in X$ элемент $y\in Y$
\end{definition}
\begin{definition}
    \textbf{Линейная форма} --- линейное отображение из линейного пространства $X$ в линейное пространство $Y$:
    $$f(x_1+x_2)=f(x_1)+f(x_2), \quad f(\alpha x)=\alpha f(x)$$
\end{definition}
\begin{definition}
    $f=g\Leftrightarrow (f,x)=(g,x) \ \ \forall x\in X$
\end{definition}
\begin{definition}
    $\theta$ --- \textbf{нуль-форма}, если $(\theta, x)=0 \ \ \forall x\in X$
\end{definition}
\begin{definition}
    $h=f+g \Leftrightarrow (h,x)=(f,x)+(g,x) \ \ \forall x\in X$
\end{definition}
\begin{definition}
    $l=\alpha f \Leftrightarrow (l, x)=\alpha(f,x) \ \ \forall x\in X$
\end{definition}
\begin{definition}
    \textbf{Пространство, сопряженное с $X$} --- пространство линейных форм, заданных на $X$ и обозначаемое $X^*$
\end{definition}

\subsection{Полилинейные формы (ПЛФ): основные определения, тензор, эквивалентное задание ПЛФ.}
\begin{remark}
    Т.к. в этом разделе много сумм, не будем их писать: $$\sum\limits_{j=1}^n\varphi_j\xi^j\equiv \varphi_j\xi^j$$
\end{remark}
\begin{remark}
    \textit{(от автора) } Я буду втыкать \Warning везде, где есть скрытые суммы.
\end{remark}
\begin{definition}
    \textbf{Полилинейная форма} --- функция от $p$ векторов и $q$ линейных форм, принимающая значения из поля $K$: $$U: X^p\times X^{*^{q}}\to K,$$ линейная по всем аргументам: $$U(x_1\ldots \alpha x_i'+x_i''\ldots x_n,y^1\ldots y^n)=\alpha U(x_1\ldots x_i'\ldots x_n,y^1\ldots y^n)+U(x_1\ldots x_i''\ldots x_n,y^1\ldots y^n),$$ такая ПЛФ имеет \textbf{валентность} $(p,q)$
\end{definition}
\begin{definition}
    \textbf{Тензор} ПЛФ $W$ валентности $(p,q)$ --- набор из $n^{p+q}$ чисел:
    $$w^{j_1\ldots j_q}_{i_1\ldots i_q}=W(e_{i_1}\ldots e_{i_p},f^{j_1}\ldots f^{j_q})$$
    $$\forall t\in\{1\ldots p\}\ \ i_t\in\{1\ldots n\}; \quad \forall t\in\{1\ldots q\}\ \ j_t\in\{1\ldots n\},$$ \textbf{ранг} этого тензора $(p, q)$.
\end{definition}
\begin{remark}
    Для удобства можно писать так: $w^{j_1\ldots j_q}_{i_1\ldots i_q}=w^{\vec j}_{\vec i}$
\end{remark}
\begin{theorem}
    Задание ПЛФ эквивалентно заданию ее тензора в паре базисов пространств $X$ и $X^*$
\end{theorem}
\subsection{Базис линейного пространства ПЛФ валентности $(p,q)$.}
\begin{definition}
    $U+V$ и $\lambda U$ заданы так же, как для линейных форм.
\end{definition}
\begin{theorem}
    Множество всех ПЛФ валентности $(p, q)$ --- линейное пространство $\Omega^p_q$ над полем $K$.
\end{theorem}
Рассмотрим набор ПЛФ $\{{}^{s_1\ldots s_p}_{t_1\ldots t_p}W\}$, такой что:
$${}^{s_1\ldots s_p}_{t_1\ldots t_p}W(x_1\ldots x_p,y^1\ldots y^q)=\xi_1^{s_1}\xi_2^{s_2}\cdots \xi_p^{s_p}\eta^1_{t_1}\eta^2_{t_2}\cdots \eta^q_{t_q},$$
т.е. ${}^{\vec s}_{\vec t} W$ --- произведение $s_i$-ой координаты $i$-того вектора и $t_i$-ой координаты $i$-той формы. Запишем в виде тензора:
$${}^{\vec s}_{\vec t} w^{\vec j}_{\vec i}=\delta_{i_1}^{s_1}\delta_{i_2}^{s_2}\cdots \delta_{i_p}^{s_p}\delta_{t_1}^{j_1}\delta_{t_2}^{j_2}\cdots\delta_{t_q}^{j_q},$$
т.е. ${}^{\vec s}_{\vec t} w^{\vec j}_{\vec i}=1$ только если $\vec s=\vec i$ и $\vec t=\vec j$, иначе $0$.
\begin{theorem}
    $\{{}^{\vec s}_{\vec t} W\}$ --- базис $\Omega^p_q$
\end{theorem}
$\dim\Omega^p_q=n^{p+q}$
\subsection{Произведение полилинейных форм и его свойства.}
$$W=U\cdot V\Leftrightarrow W(x_1\ldots x_{p_1+p_2}, y^1\ldots y^{q_1+q_2})=U(x_1\ldots x_{p_1}, y^1\ldots y^{q_1})\cdot V(x_1\ldots x_{p_2}, y^1\ldots y^{q_2})$$
Свойства:
\begin{enumerate}
    \item $U\cdot V\not=V\cdot U$
    \item $U\cdot(V\cdot W)=(U\cdot V)\cdot W=U\cdot V\cdot W$
    \item $U\cdot \Theta_{\Omega^{p_2}_{q_2}}=\Theta_{\Omega^{p_1}_{q_1}}\cdot V=\Theta_{\Omega^{p_1+p_2}_{q_1+q_2}}$
    \item $U\cdot(V+W)=U\cdot V+U\cdot W$
    \item $(\alpha\cdot U)\cdot V=U\cdot (\alpha\cdot V)$
    \item Пусть $\{f^i\}_{i=1}^n$ --- базис $X^*$. Тогда набор
          $${}^{s_1\ldots s_p}W=f^{s_1}\cdots f^{s_p}$$
          образует базис в $\Omega^p_0$:
          $${}^{s_1\ldots s_p}W=f^{s_1}(x_1)\cdots f^{s_p}(x_p)=\xi^{s_1}_1\cdots \xi^{s_p}_p$$
    \item Более общий случай: $\{f^i\}_{i=1}^n$ --- базис $X^*$, $\{\hat x^i\}_{i=1}^n$ --- базис $X^{**}$, тогда в $\Omega^p_q$ базис:
          $${}^{s_1\ldots s_p}_{t_1\ldots t_q}W=f^{s_1}\cdots f^{s_p}\cdot \hat x_{t_1}\cdots \hat x_{t_q}$$
\end{enumerate}
\subsection{Симметричные и антисимметричные ПЛФ. Операции симметризации и антисимметризации.}

\begin{definition}
    ПЛФ $U\in\Omega^p_0$ \textbf{симметричная}, если порядок аргументов не влияет на значение $U$:
    $$\forall (j_1\ldots j_p) \text{ --- перестановки }\quad U(x_1\ldots x_p)=U(x_{j_1}\ldots x_{j_p})$$
\end{definition}
\begin{definition}
    ПЛФ $U\in\Omega^p_0$ \textbf{антисимметричная}, если любая транспозиция её аргументов меняет знак значения $U$, т.е. произвольная перестановка меняет знак столько раз, сколько в ней транспозиций:
    $$\forall (j_1\ldots j_p) \text{ --- перестановки }\quad U(x_1\ldots x_p)=(-1)^{\Bracket{j_1\ldots j_p} U(x_{j_1}\ldots x_{j_p})},$$ где $\Bracket{j_1\ldots j_p}$ --- чётность перестановки.
\end{definition}
\begin{remark}
    Пространство симметричных ПЛФ --- подпространство $\Omega^p_0$ и обозначается $\Sigma^p$
\end{remark}
\begin{remark}
    Пространство антисимметричных ПЛФ --- подпространство $\Omega^p_0$ и обозначается $\Lambda^p$
\end{remark}
\begin{definition}
    \textbf{Симметризация} --- операция получения симметричной ПЛФ из произвольной ПЛФ, называется $\Sym$ и выполняется следующим образом:
    $$\Sym W=U,\quad U(x_1\ldots x_p)=\frac{1}{p!}\sum_{(j_1\ldots j_p)} W(x_{j_1}\ldots x_{j_p})$$
\end{definition}
\begin{definition}
    \textbf{Антисимметризация} --- операция получения антисимметричной ПЛФ из произвольной ПЛФ, называется $\Asym$ и выполняется следующим образом:
    $$\Asym W=V,\quad V(x_1\ldots x_p)=\frac{1}{p!}\sum_{(j_1\ldots j_p)} (-1)^{\Bracket{j_1\ldots j_p}} W(x_{j_1}\ldots x_{j_p})$$
\end{definition}
Свойства $\Sym$ и $\Asym$:
\begin{enumerate}
    \item Линейность
    \item Дистрибутивность
    \item Композиция:
          $$\Sym\Sym=\Sym, \quad \Asym\Asym=\Asym, \quad \Sym\Asym=0, \quad \Asym\Sym=0$$
\end{enumerate}
\subsection{Базис линейного пространства антисимметричных ПЛФ валентности $(p, 0)$. Доказательство полноты.}
Базис $\Lambda^p$ --- набор ПЛФ $\{{}^{s_1\ldots s_p}F\}$, такой что:
$${}^{s_1\ldots s_p}F=p!\Asym({}^{s_1\ldots s_p}W) \text{ и } 1\leq s_1\leq s_2\leq \ldots \leq s_p\leq n,$$
где $\{{}^{s_1\ldots s_p}W\}$ --- базис $\Omega^p_0$

\begin{proof}
    Докажем полноту.

    Рассмотрим произвольную форму $U\in\Lambda^p$ и разложим её по базису $\Omega^p_0$

    $$\Warning\quad U={}^{s_1\ldots s_p}Wu_{s_1\ldots s_p}$$

    $$\Warning\quad U=\Asym U=\Asym({}^{s_1\ldots s_p}Wu_{s_1\ldots s_p})=u_{s_1\ldots s_p}\Asym{}^{s_1\ldots s_p}W=$$
    $$\Warning\quad u_{s_1\ldots s_p}\frac{1}{p!}p!\Asym{}^{s_1\ldots s_p}W=u_{s_1\ldots s_p}\frac{1}{p!}({}^{s_1\ldots s_p}F)=$$
    $$\frac{1}{p!}\sum\limits_{1\leq s_1\leq \ldots \leq s_p\leq n}\sum\limits_{(\sigma_1\ldots\sigma_p)}(-1)^{\Bracket{\sigma_{s_1}\ldots\sigma_{s_p}}}({}^{s_1\ldots s_p} F)(-1)^{\Bracket{\sigma_{s_1}\ldots\sigma_{s_p}}}u_{s_1\ldots s_p}=$$
    $$\frac{1}{p!}\sum\limits_{1\leq s_1\leq \ldots \leq s_p\leq n}p!({}^{s_1\ldots s_p} F)u_{s_1\ldots s_p}=\sum\limits_{1\leq s_1\leq \ldots \leq s_p\leq n}({}^{s_1\ldots s_p} F)u_{s_1\ldots s_p}$$
    Таким образом, мы разложили произвольную $U\in\Lambda^p$ по $\{F\}$.
\end{proof}

$\dim\Lambda^p=C_n^p$
\subsection{Базис линейного пространства антисимметричных ПЛФ валентности $(p, 0)$. Доказательство линейной независимости.}
Читайте в конспекте Александра Игоревича, я на такое не готов.

\subsection{Внешнее умножение ПЛФ и его свойства.}
\begin{definition}
    \textbf{Внешнее произведение} $U\in\Lambda^p$ на $V\in\Lambda^r$ --- ПЛФ следующего вида:
    $$U\wedge V=\frac{(p+r)!}{p!r!}\Asym(U\cdot V)$$

    $U\wedge V\in\Lambda^{p+q}$
\end{definition}
Свойства:
\begin{enumerate}
    \item $p+q>n\Rightarrow U\wedge V=\Theta$
    \item $U\wedge V=(-1)^{pr} V\wedge U$
    \item $U\wedge (V\wedge W)=(U\wedge V)\wedge W=U\wedge V\wedge W=\frac{(p+r+s)!}{p!r!s!}\Asym(U\cdot V\cdot W)$
    \item $U\wedge (V+W)=U\wedge V + U\wedge W$
    \item $(\alpha U)\wedge V=U\wedge (\alpha V)=\alpha(U\wedge V)$
    \item $U\wedge\Theta=\Theta\wedge V=\Theta$
    \item $\{f^i\}_{i=1}^n$ --- базис $X^*$. Тогда:
          $$\forall 1\leq i_1\leq \ldots \leq i_p\leq n:\quad {}^{i_1\ldots i_p}F=f^{i_1}\wedge\ldots \wedge f^{i_p}$$
\end{enumerate}
\end{document}