\documentclass[12pt, a4paper]{article}

\usepackage{lastpage}
\usepackage{mathtools}
\usepackage{xltxtra}
\usepackage{libertine}
\usepackage{amsmath}
\usepackage{amsthm}
\usepackage{amsfonts}
\usepackage{amssymb}
\usepackage{enumitem}
\usepackage{xcolor}
\usepackage[left=1.5cm, right=1.5cm, top=2cm, bottom=2cm, bindingoffset=0cm, headheight=15pt]{geometry}
\usepackage{fancyhdr}
\usepackage[russian]{babel}
% \usepackage[utf8]{inputenc}
\usepackage{catchfilebetweentags}
\usepackage{accents}
\usepackage{calc}
\usepackage{etoolbox}
\usepackage{mathrsfs}
\usepackage{wrapfig}

\providetoggle{useproofs}
\settoggle{useproofs}{false}

\pagestyle{fancy}
\lfoot{M3137y2019}
\rhead{\thepage\ из \pageref{LastPage}}

\newcommand{\R}{\mathbb{R}}
\newcommand{\Q}{\mathbb{Q}}
\newcommand{\C}{\mathbb{C}}
\newcommand{\Z}{\mathbb{Z}}
\newcommand{\B}{\mathbb{B}}
\newcommand{\N}{\mathbb{N}}

\newcommand{\const}{\text{const}}

\newcommand{\teormin}{\textcolor{red}{!}\ }

\DeclareMathOperator*{\xor}{\oplus}
\DeclareMathOperator*{\equ}{\sim}
\DeclareMathOperator{\Ln}{\text{Ln}}
\DeclareMathOperator{\sign}{\text{sign}}
\DeclareMathOperator{\Sym}{\text{Sym}}
\DeclareMathOperator{\Asym}{\text{Asym}}
% \DeclareMathOperator{\sh}{\text{sh}}
% \DeclareMathOperator{\tg}{\text{tg}}
% \DeclareMathOperator{\arctg}{\text{arctg}}
% \DeclareMathOperator{\ch}{\text{ch}}

\DeclarePairedDelimiter{\ceil}{\lceil}{\rceil}
\DeclarePairedDelimiter{\abs}{\left\lvert}{\right\rvert}

\setmainfont{Linux Libertine}

\theoremstyle{plain}
\newtheorem{axiom}{Аксиома}
\newtheorem{lemma}{Лемма}

\theoremstyle{remark}
\newtheorem*{remark}{Примечание}
\newtheorem*{exercise}{Упражнение}
\newtheorem*{consequence}{Следствие}
\newtheorem*{example}{Пример}
\newtheorem*{observation}{Наблюдение}

\theoremstyle{definition}
\newtheorem{theorem}{Теорема}
\newtheorem*{definition}{Определение}
\newtheorem*{obozn}{Обозначение}

\setlength{\parindent}{0pt}

\newcommand{\dbltilde}[1]{\accentset{\approx}{#1}}
\newcommand{\intt}{\int\!}

% magical thing that fixes paragraphs
\makeatletter
\patchcmd{\CatchFBT@Fin@l}{\endlinechar\m@ne}{}
  {}{\typeout{Unsuccessful patch!}}
\makeatother

\newcommand{\get}[2]{
    \ExecuteMetaData[#1]{#2}
}

\newcommand{\getproof}[2]{
    \iftoggle{useproofs}{\ExecuteMetaData[#1]{#2proof}}{}
}

\newcommand{\getwithproof}[2]{
    \get{#1}{#2}
    \getproof{#1}{#2}
}

\newcommand{\import}[3]{
    \subsection{#1}
    \getwithproof{#2}{#3}
}

\newcommand{\given}[1]{
    Дано выше. (\ref{#1}, стр. \pageref{#1})
}

\renewcommand{\ker}{\text{Ker }}
\newcommand{\im}{\text{Im }}
\newcommand{\grad}{\text{grad}}

\lhead{Линейная алгерба}
\cfoot{}

\DeclareMathSymbol{\mlq}{\mathord}{operators}{``}
\DeclareMathSymbol{\mrq}{\mathord}{operators}{`'}

\newcommand{\doref}[1]{Дано выше. \textit{(\ref{#1}, стр. \pageref{#1})}}
\setlength{\parindent}{0pt}

\begin{document}
\section{Векторная алгебра}
\section{Аналитическая геометрия}
\section{Алгебраические структуры. СЛАУ}
\subsection{Алгебраические структуры: группа, кольцо, поле}
\begin{definition}
    \textbf{Полугруппа} --- множество $G$ с заданной на нём бинарной ассоциативной операцией $\circ$, т.е. $$(g_1\circ g_2)\circ g_3 = g_1\circ (g_2 \circ g_3)$$
\end{definition}
\begin{definition}
    \textbf{Группа} --- полугруппа, где выбран нейтральный элемент и для каждого элемента есть обратный:
    \begin{enumerate}
        \item Нейтральный элемент $e: e\circ g=g\circ e=g$
        \item Обратный элемент: $\forall g\in G \ \ \exists g^{-1} \ \ g\circ g^{-1}=g^{-1}\circ g=e$
    \end{enumerate}
\end{definition}
\begin{definition}
    \textbf{Абелева группа} --- группа с коммутативной операцией, т.е. $$\forall g_1, g_2\in G \ \ g_1\circ g_2=g_2\circ g_1$$
\end{definition}
\begin{definition}
    \textbf{Кольцо} --- множество с двумя бинарными операциями $\{R, \mlq+\mrq, \mlq\cdot\mrq\}$, которое является абелевой группой относительно сложения, полугруппой относительно умножения и эти операции согласованны \textit{(дистрибутивны)}:
    $$r_1\cdot(r_2+r_3)=r_1\cdot r_2+r_1\cdot r_3;\quad (r_2+r_3)\cdot r_1=r_2\cdot r_1+r_3\cdot r_1$$
\end{definition}
\begin{definition}
    \textbf{Поле} --- множество с двумя бинарными операциями $\{R, \mlq+\mrq, \mlq\cdot\mrq\}$, где эти операции согласованны и:
    \begin{enumerate}
        \item $\{K, \mlq + \mrq\}$ --- абелева группа
        \item $\{K\setminus\{0\}, \mlq \cdot \mrq\}$ --- абелева группа
    \end{enumerate}
\end{definition}
\subsection{Алгебраические структуры: линейное пространство, алгебра}
\begin{definition}
    \textbf{Модуль над кольцом $R$} --- абелева группа $\{G, \mlq + \mrq\}$ с операцией $R\times G\to G$, записываемой как $rg$ и для которой выполняется следующее:
    \begin{enumerate}
        \item $(r_1+r_2)g=r_1g+r_2g$
        \item $r(g_1+g_2)=rg_1+rg_2$
        \item $(r_1r_2)g=r_1(r_2g)$
    \end{enumerate}
\end{definition}
\begin{definition} \label{linear_space}
    \textbf{Линейное пространство} --- модуль над кольцом, которое также является полем.
\end{definition}
\begin{definition}
    \textbf{Вектор} --- элемент линейного пространства.
\end{definition}
\begin{definition}
    \textbf{Алгебра} --- модуль над кольцом, где сам модуль также является кольцом.
\end{definition}
\subsection{Поле комплексных чисел}
$i^2:=1$

$\C = \{a+bi : \forall a,b\in\R\}$

\textbf{Модуль} комплексного числа $c$: $|c|=r=\sqrt{a^2+b^2}$, если $c=a+bi$

\textbf{Аргумент} комплексного числа $c$: $\varphi=\arg(c)=\arg(a+bi)=2\arctan\left(\frac{b}{\sqrt{a^2+b^2}+a}\right)$

Тогда $c=r(\cos \varphi + i\sin \varphi)$

\textbf{Дополнение} комплексного числа $c$ записывается как $\overline c = \overline{a+bi}=a-bi$
\subsection{Линейное пространство. Примеры линейных пространств.}
\doref{linear_space}

Примеры: \begin{enumerate}
    \item $X=\{x=\begin{pmatrix}
    \xi^1,\ldots \xi^n
\end{pmatrix}^T, \xi^i\in\R\}$ \textit{(или $\C$)}
    \item $\mathcal P_n=\{\text{многочлены} p(t):\deg p(t)\leq n, n\in\N\}$
\end{enumerate}
\subsection{Линейная зависимость векторов. Основные леммы о линейной зависимости.}
\begin{definition}
    \textbf{Линейной комбинацией} называется следующее выражение: $$\alpha_1x_1+\ldots+\alpha_nx_n,$$ где $\{x_i\}_{i=1}^n$ --- вектора, $\{\alpha_i\}_{i=1}^n$ --- коэффициенты.
\end{definition}
\begin{definition}
    Набор векторов $\{x_i\}_{i=1}^n$ называется \textbf{линейнонезависимым}, если не существует его линейной комбинации, где не все коэффициенты равны $0$, а сама комбинация равна $0_X$:
    $$\not\exists \{\alpha_i\}_{i=1}^n : \exists i : \alpha_i\not=0 \quad \alpha_1x_1+\ldots+\alpha_nx_n=0_X$$

    Иначе набор называется \textbf{линейно зависимым}
\end{definition}
\begin{lemma}
    Любой набор, содержащий нулевой вектор, является линейнозависимым.
\end{lemma}
\begin{lemma}
    Набор, содержащий линейнозависимый поднабор, является линейнозависимым.
\end{lemma}
\begin{lemma}
    Любой поднабор линейнонезависимого набора также является линенйнонезависимым.
\end{lemma}
\begin{lemma}
    Набор векторов линейнозависим тогда и только тогда, когда хотя бы один из векторов набора выражается линейной комбинацией остальных.

    $$\exists k\in\{1\ldots n\}: x_k=\sum\limits_{i=1, i\not=k}^n\alpha^ix_i \Leftrightarrow \{x_i\}_{i=1}^n \text{ --- ЛЗ}$$
\end{lemma}
\subsection{Базис и размерность линейного пространства.}
\begin{definition}
    Набор векторов называется \textbf{полным} в линейном пространстве $X$, если любой вектор этого пространства можно выразить как линейную комбинацию этого набора:
    $$\forall x\in X \quad \exists \{\alpha_i\}_{i=1}^n \quad x=\sum\limits_{i=1}^n\alpha^ix_i$$
\end{definition}
\begin{definition}
    Набор векторов называется \textbf{базисом} пространства $X$, если он является полным и ЛНЗ.
\end{definition}
\begin{definition}
    Линейное пространство называется \textbf{конечномерным}, если в нём существует конечный полный набор векторов
\end{definition}
\begin{definition}
    \textbf{Размерность пространства} $\dim X$ --- количество векторов в его базисе.
\end{definition}
\subsection{Изоморфизм линейных пространств.}
\begin{definition}
    \textbf{Изоморфизм} --- биекция, сохраняющая линейность, установленная между двумя линейными пространствами над одним и тем же полем:

    $$\begin{cases}
        x_1\leftrightarrow y_1 \\
        x_2\leftrightarrow y_2
    \end{cases} \Rightarrow \begin{cases}
        x_1+x_2\leftrightarrow y_1+y_2 \\
        \alpha x_1\leftrightarrow \alpha y_1
    \end{cases}$$
\end{definition}
\subsection{Подпространства линейного пространства: определение, примеры, линейная оболочка, линейное многообразие.}
\begin{definition}
    \textbf{Подпространство} линейного пространства $X$ --- замкнутое множество $L\subset X$
\end{definition}
\begin{example}
    \begin{enumerate}
        \item $X$ и $\{0\}$ называются тривиальными подпространствами
        \item Прямая и плоскость, содержащие начало координат --- подпространство $E_3$
        \item $\R^{m<n}$ --- подпространство $\R^n$
        \item Множество симметричных $n\times n$ матриц --- подпространство $\R^n_n$
        \item Множество полиномов с членами только чётных степеней --- подпространство $\mathcal{P}_n$
    \end{enumerate}
\end{example}
\begin{definition}
    \textbf{Линейная оболочка} набора векторов --- множество всех линейных комбинаций этих векторов: $$\mathcal{L}(x_1\ldots x_n)=\left\{\sum\limits_{i=1}^k \alpha^ix_i \ \ |\ \ \forall \alpha_1\ldots \alpha_n\right\}$$
\end{definition}
\begin{definition}
    \textbf{Линейное многообразие}, параллельное подпространству $L$ линейного пространства $X$ --- множество $M$: $$M=\{y\in X : y=x_0+x \quad \forall x\in L\}$$
\end{definition}
\subsection{Подпространства линейного пространства: сумма и пересечение подпространств, прямая сумма, дополнение.}
\begin{definition}
    \textbf{Пересечение подпространств} $L_1$ и $L_2$ --- множество $L'$, такое что: $$L'=\{x\in X : x\in L_1 \text{ и } x\in L_2\}$$
\end{definition}
\begin{definition}
    \textbf{Сумма подпространств} $L_1$ и $L_2$ --- множество $L''$, такое что: $$L'э=\{x\in X : x=x_1+x_2 \ \ \forall x_1 \in L_1, x_2 \in L_2\}$$
\end{definition}
\begin{definition}
    \textbf{Прямая сумма подпространств} $L_1$ и $L_2$ --- множество $L=L_1\dot+L_2$, такое что: $$L=\{x\in X : x!=x_1+x_2 \ \ \forall x_1 \in L_1, x_2 \in L_2\}$$
\end{definition}
\begin{definition}
    Если $X=L_1\dot+L_2$, $L_1$ --- \textbf{дополнение} $L_2$ до $X$
\end{definition}
\subsection{Линейные алгебраические системы. Геометрическое исследование систем. Теорема Крамера \textit{(геометрическая формулировка)}.}
\begin{definition}
    $$\begin{cases}
        \alpha_1^1\xi^1+\alpha_2^1\xi^2+\ldots+\alpha_n^1\xi^n=\beta^1 \\
        \alpha_1^2\xi^1+\alpha_2^2\xi^2+\ldots+\alpha_n^2\xi^n=\beta^2 \\
        \vdots \\
        \alpha_1^m\xi^1+\alpha_2^m\xi^2+\ldots+\alpha_n^m\xi^n=\beta^m
    \end{cases}$$ --- \textbf{линейная алгебраическая система}, $\alpha$ --- коэффициенты, $\beta$ --- свободные члены, $\xi$ --- неизвестные
\end{definition}
\begin{definition}
    \textbf{Решение системы} --- такой набор, при подстановке которого равенства становятся верными.
\end{definition}
\begin{definition}
    \textbf{Совместная система} --- система, у которой есть решение.
\end{definition}
\begin{definition}
    \textbf{Определенная система} --- совместная система, которая имеет единственное решение.
\end{definition}
\begin{definition}
    \textbf{Однородная система} --- система, у которой все свободные члены равны $0$.
\end{definition}
Запишем в векторной форме: $\sum\limits_{i=1}^n a_i\xi^i=b$

\begin{theorem}
    Если $m=n$ и $\{a_i\}_{i=1}^n$ --- ЛНЗ, система совместна и определена, т.е. есть единственное решение.
\end{theorem}
\subsection{Геометрическое исследование систем. Теорема Кронекера-Капелли \textit{(геометрическая формулировка)} и ее следствия.}
Рассмторим случай, когда $\dim\mathcal{L}\{a_1\ldots a_n\}=r\leq m$. Тогда можно переписать систему как: $$a_1\xi^1+\ldots+a_r\xi^r=b-a_{r+1}\xi^{r+1}-\ldots-a_n\xi^n$$
\begin{theorem}
    $b\in\mathcal L \Leftrightarrow$ система совместна. Если $r=n$, система определена, иначе --- нет.
\end{theorem}
\begin{consequence}
    Однородная система:
    \begin{enumerate}
        \item Всегда совместна, т.к. существует тривиальное решение
        \item Имеет нетривиальные решения тогда и только тогда, когда $r<n$
        \item Является неопределенной тогда и только тогда, когда $m<n$
    \end{enumerate}
\end{consequence}
\subsection{Альтернатива Фредгольма для линейной системы уравнений.}
\begin{theorem}
    Если $m=n$, то:
    \begin{enumerate}
        \item Или однородная система имеет только тривиальное решение, и неоднородная система совместна и определена для любого $b$
        \item Или существуют нетривиальные решения однородной системы и неоднородная система совместна не при любых $b$
    \end{enumerate}
\end{theorem}
\subsection{Фундаментальная система решений линейной однородной системы. Общее решение однородных и неоднородных систем.}
\begin{definition}
    \textbf{Фундаментальной системой решений} линейной однородной системы уравнений называется любая система из $n − r$ линейнонезависимых решений этой системы, то есть базис пространтва решений однородной системы.
\end{definition}
Любое решение можно представить в виде общего решения: $$z=z'+\sum\limits_{i=1}^n c_ix_i,$$ где $\{x_i\}_{i=1}^n$ --- ФСР.
\section{Полилинейные формы. Определители}
\end{document}