\documentclass[12pt, a4paper]{article}

\usepackage{lastpage}
\usepackage{mathtools}
\usepackage{xltxtra}
\usepackage{libertine}
\usepackage{amsmath}
\usepackage{amsthm}
\usepackage{amsfonts}
\usepackage{amssymb}
\usepackage{enumitem}
\usepackage{xcolor}
\usepackage[left=1.5cm, right=1.5cm, top=2cm, bottom=2cm, bindingoffset=0cm, headheight=15pt]{geometry}
\usepackage{fancyhdr}
\usepackage[russian]{babel}
% \usepackage[utf8]{inputenc}
\usepackage{catchfilebetweentags}
\usepackage{accents}
\usepackage{calc}
\usepackage{etoolbox}
\usepackage{mathrsfs}
\usepackage{wrapfig}

\providetoggle{useproofs}
\settoggle{useproofs}{false}

\pagestyle{fancy}
\lfoot{M3137y2019}
\rhead{\thepage\ из \pageref{LastPage}}

\newcommand{\R}{\mathbb{R}}
\newcommand{\Q}{\mathbb{Q}}
\newcommand{\C}{\mathbb{C}}
\newcommand{\Z}{\mathbb{Z}}
\newcommand{\B}{\mathbb{B}}
\newcommand{\N}{\mathbb{N}}

\newcommand{\const}{\text{const}}

\newcommand{\teormin}{\textcolor{red}{!}\ }

\DeclareMathOperator*{\xor}{\oplus}
\DeclareMathOperator*{\equ}{\sim}
\DeclareMathOperator{\Ln}{\text{Ln}}
\DeclareMathOperator{\sign}{\text{sign}}
\DeclareMathOperator{\Sym}{\text{Sym}}
\DeclareMathOperator{\Asym}{\text{Asym}}
% \DeclareMathOperator{\sh}{\text{sh}}
% \DeclareMathOperator{\tg}{\text{tg}}
% \DeclareMathOperator{\arctg}{\text{arctg}}
% \DeclareMathOperator{\ch}{\text{ch}}

\DeclarePairedDelimiter{\ceil}{\lceil}{\rceil}
\DeclarePairedDelimiter{\abs}{\left\lvert}{\right\rvert}

\setmainfont{Linux Libertine}

\theoremstyle{plain}
\newtheorem{axiom}{Аксиома}
\newtheorem{lemma}{Лемма}

\theoremstyle{remark}
\newtheorem*{remark}{Примечание}
\newtheorem*{exercise}{Упражнение}
\newtheorem*{consequence}{Следствие}
\newtheorem*{example}{Пример}
\newtheorem*{observation}{Наблюдение}

\theoremstyle{definition}
\newtheorem{theorem}{Теорема}
\newtheorem*{definition}{Определение}
\newtheorem*{obozn}{Обозначение}

\setlength{\parindent}{0pt}

\newcommand{\dbltilde}[1]{\accentset{\approx}{#1}}
\newcommand{\intt}{\int\!}

% magical thing that fixes paragraphs
\makeatletter
\patchcmd{\CatchFBT@Fin@l}{\endlinechar\m@ne}{}
  {}{\typeout{Unsuccessful patch!}}
\makeatother

\newcommand{\get}[2]{
    \ExecuteMetaData[#1]{#2}
}

\newcommand{\getproof}[2]{
    \iftoggle{useproofs}{\ExecuteMetaData[#1]{#2proof}}{}
}

\newcommand{\getwithproof}[2]{
    \get{#1}{#2}
    \getproof{#1}{#2}
}

\newcommand{\import}[3]{
    \subsection{#1}
    \getwithproof{#2}{#3}
}

\newcommand{\given}[1]{
    Дано выше. (\ref{#1}, стр. \pageref{#1})
}

\renewcommand{\ker}{\text{Ker }}
\newcommand{\im}{\text{Im }}
\newcommand{\grad}{\text{grad}}

\lhead{Линейная алгерба}
\cfoot{}
\rfoot{Лекция 2}

\newcommand\Warning{%
 \makebox[1.4em][c]{%
 \makebox[0pt][c]{\raisebox{.1em}{\small!}}%
 \makebox[0pt][c]{\color{red}\Large$\bigtriangleup$}}}%

\begin{document}

\begin{remark}
    \(\{x_i\}_{i=1}^k\) --- ЛЗ \(\Leftrightarrow \{x_i\}_{i=1}^k\) обнуляет все базисные ПЛФ из $\Lambda^K$ ($C^k_n$ штук)
\end{remark}

\(\sphericalangle C=\begin{bmatrix}
    \xi_1^1 & \xi_2^1 & \ldots & x_n^1 \\
    \xi_1^2 & \xi_2^2 & \ldots & x_n^2 \\
    \vdots & \vdots & \ddots & \vdots \\
    \xi_1^n & \xi_2^n & \ldots & x_n^n \\
\end{bmatrix}\)

$$\det C = \det \{x_1\ldots x_n\} \stackrel{\triangle}{=} {}^{1\ldots n}F(x_1\ldots x_n) = \sum\limits_{(i_1\ldots i_n)} (-1)^{[i_1\ldots i_n]} \xi_1^{i_1}\ldots \xi_n^{i_n}$$

\begin{definition}
    \textbf{Рангом} $r$ матрицы $A_{m\times n}$ называется порядок её наибольшего отличного от нуля минора.
\end{definition}

\begin{remark}
    $rank(A) \quad rg(A) \quad rang(A)$

    $\exists L_{j_1\ldots j_r}^{i_1\ldots i_r}\not=0$, но $] \exists L_{j_1\ldots j_{r+1}}^{i_1\ldots i_{r+1}} \not = 0$
\end{remark}

\(\sphericalangle C=\begin{bmatrix}
    c_{11} & c_{21} & \ldots & c_{n1} \\
    0 & c_{22} & \ldots & c_{n2} \\
    \vdots & \vdots & \ddots & \vdots \\
    0 & 0 & \ldots & c_{nn} \\
\end{bmatrix}\)


$$\begin{bmatrix}
    c_{11} & c_{21} & \ldots & c_{n1} & 0 & \ldots & 0 \\
    0 & c_{22} & \ldots & c_{n2} & 0 & \ldots & 0\\
    \vdots & \vdots & \ddots & \vdots & \vdots & \ddots & \vdots\\
    0 & 0 & \ldots & c_{nn} & 0 & \ldots & 0\\
    0 & \ldots &\ldots &\ldots &\ldots & \ldots & 0 \\
\end{bmatrix}$$

\begin{enumerate}
    \item $L_1^1=C_1 \Rightarrow rg A \geq 1$
    \item $L_1^1=C_1C_{22}\not=0 \Rightarrow rg A \geq 2$
    \item $L_1^1=C_1C_{22}C_{33}\not=0 \Rightarrow rg A \geq 3$
    
    $\vdots$
    \item $L_{1\ldots r}^{1\ldots r+1} = \prod\limits_{i=1}^r c_{ii}\not=0 \Rightarrow rg A\geq r$
    \item $L_{1\ldots r}^{1\ldots r+1} = 0 \Rightarrow rg A= r$
\end{enumerate}
$\Rightarrow rg A\leq \min(m, n)$

\begin{theorem}
    О базисном миноре

    \begin{enumerate}
        \item Число ЛНЗ строк (столбцов) матрицы $A$ равно её рангу
        \item Любая строка (столбец) матрицы $A$ может быть представлена в виде ЛК строк (столбцов), входящих в её минор наибольшего порядка, отличного от нуля (базисный минор)
    \end{enumerate}
\end{theorem}
\begin{proof}
    \begin{enumerate}
        \item Следует из критерия ЛНЗ
        \item Строки (столбцы), входящие в базисный минор, образуют максимальный ЛНЗ поднабор всех строк (столбцов) матрицы $A$.        
    \end{enumerate}
\end{proof}

\begin{theorem}
    Крамера

    $\sphericalangle$ СЛАУ: $\sum\limits_{j=1}^n a_{ij} \xi^j = b_i$, такую что $A=||a_j^i||_{i,j=1}^n \quad \det A\not=0$

    Тогда:
    \begin{enumerate}
        \item СЛАУ совместна и определена
        \item $\xi^j=\frac{\triangle_j}{\triangle},\quad \triangle_j=(a_1\ldots a_{j-1}, b, a_{j+1}\ldots a_n)$
    \end{enumerate}
\end{theorem}
\begin{proof}
    \begin{enumerate}
        \item $\det A = \det \{a_1\ldots a_n\} \not= 0 \Rightarrow \{a_j\}_{j=1}^n$ --- ЛНЗ $\Rightarrow$ базис $\R^n\ni b$
        \item $\triangle_j=\det\{a_1\ldots a_{j-1}, b, a_{j+1}\ldots a_n\}=\det\{a_1\ldots a_{j-1}, \sum\limits_{j=1}^n a_{j} \xi^j, a_{j+1}\ldots a_n\}=$
        
        $=\sum\limits_{j=1}^n \xi^j\det\{a_1\ldots a_{j-1},  a_{j} , a_{j+1}\ldots a_n\}=\xi^k\cdot\triangle$
    \end{enumerate}
\end{proof}

\begin{theorem}
    Кронекера-Капелли

    $\sphericalangle \sum\limits_{j=1}^n a_j\xi^j=b, ] A = \begin{bmatrix}
        a_1 & a_2 & \ldots & a_n
    \end{bmatrix}$

    $\tilde A = \begin{bmatrix}
        a_1 & a_2 & \ldots & a_n & | & b
    \end{bmatrix}$

    СЛАУ совместна $\Leftrightarrow rg \tilde A = rg A$
\end{theorem}
\begin{proof}
    Тривиально.
\end{proof}

\section{Тензорная алгерба}

\subsection{Преобразование координат в $X$ и $X^*$}

%<*преобразованиекоординат>
$\sphericalangle \{e_j\}$ --- базис $X$

$\sphericalangle \{\tilde e_k\}$ --- базис $X$

$\Rightarrow \forall k \ \ \tilde e_k=\sum_{j=1}^n t_k^je_j$

\begin{definition}
    Набор $T=||t_j^i||$ образует матрицу, которая называется \textbf{матрицей перехода} от базиса $\{e_j\}$ к базису $\{\tilde e_k\}$
\end{definition}

\begin{remark}
    $\sphericalangle E=\begin{bmatrix}
        e_1 & e_2 & \ldots & e_n
    \end{bmatrix}, \tilde E=\begin{bmatrix}
        \tilde e_1 & \tilde e_2 & \ldots & \tilde e_n
    \end{bmatrix} \Rightarrow \tilde E = ET$
\end{remark}

\begin{lemma}
    $] \xi$ --- координаты вектора $x$ в базисе $\{e_j\}$
    
    $] \tilde \xi$ --- координаты вектора $x$ в базисе $\{\tilde e_k\}$

    Тогда $\xi=T\tilde\xi$ или $\tilde \xi = S \xi, S=T^{-1}$
\end{lemma}
\begin{proof}
    $x=\sum\limits_{k=1}^n\tilde\xi^k \tilde e_k = \sum\limits_{k=1}^n \tilde x^k \sum\limits_{j=1}^n t_k^j e_j=\sum\limits_{j=1}^n (\sum\limits_{k=1}^n \tilde \xi^k t_k^j)e_j=\sum\limits_{j=1}^n \xi^j e_j \Rightarrow \xi=T\tilde\xi$
\end{proof}

\begin{lemma}
    $] \{f^l\}$ --- базис $X^*$, сопряженный $\{e_j\}$, т.е. $f^l(e_j)=\delta_j^l$
    
    $] \{\tilde f^m\}$ --- базис $X^*$, сопряженный $\{\tilde e_k\}$, т.е. $\tilde f^m(\tilde e_k)=\delta_m^k$

    $] F=\begin{bmatrix}
        f^1 & f^2 & \ldots & f^n
    \end{bmatrix}^T,\quad \tilde F = \begin{bmatrix}
        \tilde f^1 & \tilde f^2 & \ldots & \tilde f^n
    \end{bmatrix}^T$

    Тогда $F=T \tilde F$ или $f^l=\sum\limits_{m=1}^n t^l_m \tilde f^m$
\end{lemma}
\begin{proof}
    $\sphericalangle (\tilde f^m, \tilde e_k)=\delta^m_k=(\tilde f^m, \sum\limits_{j=1}^n t_k^j e_j)=\sum\limits_{j=1}^n t_k^j (\tilde f^m, e_j)=\sum\limits_{j=1}^n t_k^j \sum\limits_{l=1}^n a_l^m (f^l, e_j)=\sum\limits_{j=1}^n t_k^j a_j^m$
    
    $\Rightarrow \sum\limits_{j=1}^n a_j^mt_k^j=\delta^m_k$ или $AT=I$ --- единичная матрица $\Rightarrow A=T^{-1}$
\end{proof}

\begin{lemma}
    $]\varphi$ --- коэфф. ЛФ в $\{e_j\}$

    $]\tilde\varphi$ --- коэфф. ЛФ в $\{\tilde e_k\}$

    $\Rightarrow \tilde \varphi = \varphi T$
\end{lemma}
\begin{proof}
    $] g$ --- ЛФ, $\varphi_j=g(e_j) \quad \tilde \varphi_k = g(\tilde e_k)$

    $$\varphi_k=g(\tilde e_k) = g\left(\sum\limits_{j=1}^n t_k^j e_j\right)=\sum\limits_{j=1}^n t_k^j g(e_j)=\sum\limits_{j=1}^n t_k^j\varphi_j$$

    $\Rightarrow \tilde \varphi = \varphi T$
\end{proof}

Итого:
$$\tilde E = ET \quad \tilde F = T^{-1}F \quad \tilde\xi = T^{-1}\xi \quad \tilde \varphi = \varphi T$$
%</преобразованиекоординат>

\begin{definition}
    Величины, которые преобразуются при замене базиса так же, как базисные векторы, называются \textbf{ковариантными} величинами.

    Величины, которые преобразуются при замене базиса противоположным базисным векторам образом, называются \textbf{контравариантными} величинами.    
\end{definition}

\begin{remark}
    $\xi$ --- контрвариантная величина. Верхний индекс называется контравариантным, нижний --- ковариантным.
\end{remark}

$] W\in\Omega^p_q$ --- ПЛФ $(p, q)$

$] \{e_j\}_{j=1}^n$ --- базис $X$, $\{f^k\}_{k=1}^n$ --- базис $X^*$

$$\Rightarrow \omega^{j_1\ldots j_n}_{i_1\ldots i_n} \stackrel{def}{=} W(e_{i_1}\ldots e_{i_p}f^{j_1}\ldots f^{j_q})$$

$$\{e_j\}\xrightarrow{T} \{\tilde e_k\} \quad \{f^l\}\xrightarrow{T^{-1}} \{\tilde f^m\}$$

Пусть в паре базисов $\{\tilde e_k\}$ и $\{\tilde f^m\}$ ПЛФ $W$ имеет тензор $\tilde w_{s_1\ldots s_p}^{t_1\ldots t_q}=W(\tilde e_{s_1}\ldots \tilde e_{s_p}, \tilde f^{t_1}\ldots \tilde f^{t_q})=$

$$=\Warning W(t_{s_1}^{i_1}e_{i_1} \ldots t_{s_p}^{i_p}e_{i_p}, \sigma_{j_1}^{t_1}f^{j_1}\ldots \sigma_{j_q}^{t_q}f^{j_q})=$$
$$=t^{i_1}_{s_1}\ldots t^{i_p}_{s_p} \sigma^{j_1}_{t_1}\ldots \sigma^{j_q}_{t_q}W(e_{s_1}\ldots e_{s_p}, f^{t_1}\ldots f^{t_q})$$

\begin{definition}
    \begin{enumerate}
        \item \textbf{Вектором} называется величина, преобразующаяся по контравариантному закону
        \item \textbf{Линейной формой} называется величина, преобразующаяся по ковариантному закону
        \item \textbf{Тензором} типа $(p, q)$ называется величина, преобразующаяся $p$ раз по ковариантному закону и $q$ раз по контравариантному.
    \end{enumerate}
\end{definition}

\subsection{Операции с тензорами}

\begin{enumerate}
    \item $] w, v$ --- тензоры типа $(p, q)$. Тогда $w+\alpha w$ --- тензор $(p, q)$
    \begin{proof}
        Тривиально.
    \end{proof}
    \item Транспонирование
    
    $t^{(st)} : \omega^{j_1\ldots j_s \ldots j_t\ldots j_q}_{i_1\ldots i_p}\mapsto\omega^{j_1\ldots j_t \ldots j_s\ldots j_q}_{i_1\ldots i_p}$
\end{enumerate}

\begin{remark}
    Переставлять можно только индексы одного типа
\end{remark}
\begin{lemma}
    Транспонирование сохраняет тензорную природу величины.
\end{lemma}

\begin{enumerate}[resume]
    \item Свертка: $$\stackrel{k\wedge s}{\omega}^{j_1\ldots j_n}_{i_1\ldots i_n} = \sum\limits_{m=1}^n \omega^{j_1\ldots \stackrel{k}{\not m}\ldots j_q}_{i_1\ldots \stackrel{s}{\not m} \ldots i_p}$$
\end{enumerate}
\begin{remark}
    Операцию свертки можно выполнять только по индексам разных типов
\end{remark}
\begin{lemma}
    Свертка сохраняет тензорную природу
\end{lemma}
\begin{lemma}
    $$\stackrel{\stackrel{l\wedge m}{k \wedge s}}{\omega}=\stackrel{\stackrel{k \wedge s}{l\wedge m}}{\omega}$$
\end{lemma}
\begin{proof}
    От перестановки мест слагаемых конечная сумма не меняется.
\end{proof}

\begin{enumerate}[resume]
    \item Тензорное произведение
    
    $\omega (p_1, q_1); v (p_2, q_2) \quad \omega\otimes v = a$
    $$w^{j_1\ldots j_{q_1}}_{i_1\ldots i_{p_1}} \cdot v^{j_{q_1+1} \ldots j_{q_1 + q_2}}_{i_{p_1+1}\ldots i_{p_1+p_2}}=a^{j_1\ldots j_{q_1+q_2}}_{i_1\ldots i_{p_1+p_2}}$$
\end{enumerate}
\begin{lemma}
    Результат тензорного произведения является тензором типа $(p_1+p_2, q_1+q_2)$
\end{lemma}

\end{document}