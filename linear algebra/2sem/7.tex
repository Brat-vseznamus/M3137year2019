\documentclass[12pt, a4paper]{article}

\usepackage{lastpage}
\usepackage{mathtools}
\usepackage{xltxtra}
\usepackage{libertine}
\usepackage{amsmath}
\usepackage{amsthm}
\usepackage{amsfonts}
\usepackage{amssymb}
\usepackage{enumitem}
\usepackage{xcolor}
\usepackage[left=1.5cm, right=1.5cm, top=2cm, bottom=2cm, bindingoffset=0cm, headheight=15pt]{geometry}
\usepackage{fancyhdr}
\usepackage[russian]{babel}
% \usepackage[utf8]{inputenc}
\usepackage{catchfilebetweentags}
\usepackage{accents}
\usepackage{calc}
\usepackage{etoolbox}
\usepackage{mathrsfs}
\usepackage{wrapfig}

\providetoggle{useproofs}
\settoggle{useproofs}{false}

\pagestyle{fancy}
\lfoot{M3137y2019}
\rhead{\thepage\ из \pageref{LastPage}}

\newcommand{\R}{\mathbb{R}}
\newcommand{\Q}{\mathbb{Q}}
\newcommand{\C}{\mathbb{C}}
\newcommand{\Z}{\mathbb{Z}}
\newcommand{\B}{\mathbb{B}}
\newcommand{\N}{\mathbb{N}}

\newcommand{\const}{\text{const}}

\newcommand{\teormin}{\textcolor{red}{!}\ }

\DeclareMathOperator*{\xor}{\oplus}
\DeclareMathOperator*{\equ}{\sim}
\DeclareMathOperator{\Ln}{\text{Ln}}
\DeclareMathOperator{\sign}{\text{sign}}
\DeclareMathOperator{\Sym}{\text{Sym}}
\DeclareMathOperator{\Asym}{\text{Asym}}
% \DeclareMathOperator{\sh}{\text{sh}}
% \DeclareMathOperator{\tg}{\text{tg}}
% \DeclareMathOperator{\arctg}{\text{arctg}}
% \DeclareMathOperator{\ch}{\text{ch}}

\DeclarePairedDelimiter{\ceil}{\lceil}{\rceil}
\DeclarePairedDelimiter{\abs}{\left\lvert}{\right\rvert}

\setmainfont{Linux Libertine}

\theoremstyle{plain}
\newtheorem{axiom}{Аксиома}
\newtheorem{lemma}{Лемма}

\theoremstyle{remark}
\newtheorem*{remark}{Примечание}
\newtheorem*{exercise}{Упражнение}
\newtheorem*{consequence}{Следствие}
\newtheorem*{example}{Пример}
\newtheorem*{observation}{Наблюдение}

\theoremstyle{definition}
\newtheorem{theorem}{Теорема}
\newtheorem*{definition}{Определение}
\newtheorem*{obozn}{Обозначение}

\setlength{\parindent}{0pt}

\newcommand{\dbltilde}[1]{\accentset{\approx}{#1}}
\newcommand{\intt}{\int\!}

% magical thing that fixes paragraphs
\makeatletter
\patchcmd{\CatchFBT@Fin@l}{\endlinechar\m@ne}{}
  {}{\typeout{Unsuccessful patch!}}
\makeatother

\newcommand{\get}[2]{
    \ExecuteMetaData[#1]{#2}
}

\newcommand{\getproof}[2]{
    \iftoggle{useproofs}{\ExecuteMetaData[#1]{#2proof}}{}
}

\newcommand{\getwithproof}[2]{
    \get{#1}{#2}
    \getproof{#1}{#2}
}

\newcommand{\import}[3]{
    \subsection{#1}
    \getwithproof{#2}{#3}
}

\newcommand{\given}[1]{
    Дано выше. (\ref{#1}, стр. \pageref{#1})
}

\renewcommand{\ker}{\text{Ker }}
\newcommand{\im}{\text{Im }}
\newcommand{\grad}{\text{grad}}

\lhead{Линейная алгерба}
\cfoot{}
\rfoot{Лекция 7}

\begin{document}

\section*{Алгебра скалярных полиномов}

$] K$ --- поле, над которым задано множество полиномов $K_\infty[\lambda]$, также обозначается $P_\infty[K]$
$$P_\infty[K]=\{p_n(\lambda)=\sum_{i=1}^n \alpha_i \lambda^i \quad \forall n\}$$

\begin{remark}
    $P_\infty[K]$ --- линейное пространство:

    $] p, q\in P_\infty[K]; \lambda\in K \Rightarrow \begin{cases}
        (p+q)(\lambda)=p(\lambda)+q(\lambda) \\
        (\lambda p)(\lambda) = \alpha p(\lambda)
    \end{cases} \Rightarrow P_\infty[K]$ --- линейное пространство
\end{remark}

\begin{remark}
    $P_\infty[K]$ --- коммутативная алгебра

    Зададим операцию умножения в $P_\infty[K]$:
    $$\forall p, q\in P_\infty[K] \quad (p\cdot q)(\lambda)=p(\lambda)q(\lambda)$$
    $$(p\cdot q)(\lambda)=p(\lambda)q(\lambda)=q(\lambda)p(\lambda)=(qp)(\lambda) \Rightarrow \text{коммутативность}$$
    $$(p\cdot q)\cdot r = p\cdot (q\cdot r) = p\cdot q\cdot r$$
    $$(p+q)r=pr+qr$$
    $$(\lambda p)q=p(\lambda q)=\lambda(pq)$$

    Нейтральный элемент:
    \begin{itemize}
        \item по сложению: $0(\lambda)=0$
        \item по умножению: $1(\lambda)=1$
    \end{itemize}
\end{remark}

\begin{remark}
    $\{1, t, t^2 \ldots t^n \ldots \}$ --- базис $P_\infty[K] \Rightarrow \dim P_\infty[K]=\infty$
\end{remark}

\begin{definition}
    \textbf{Идеалом} $J$ алгебры $P_\infty[K]$ называется такое её подпространство, что
    $$\forall q\in J \ \ \forall p\in P_\infty[K] \quad q\cdot p\in J$$
\end{definition}
\begin{example}
    Тривиальные идеалы:
    \begin{itemize}
        \item $\{0\}$
        \item $P_\infty[K]$
    \end{itemize}
\end{example}
\begin{lemma}
    $J$ --- линейное подпространство $P_\infty[K]$
\end{lemma}
\begin{proof}
    $] q_1, q_2\in J \quad q_1+q_2\in J?$

    $q_1, q_2\in J \Rightarrow \forall p \ \ q_1p, q_2p\in J$

    $q_1=r\tilde q_1, q_2=r\tilde q_2 \quad (q_1+q_2)p=r(\tilde q_1+\tilde q_2)p$

    $(\tilde q_1+\tilde q_2)p\in P_\infty[K] \Rightarrow r(\tilde q_1+\tilde q_2)p\in J$
\end{proof}
\begin{lemma}
    $J$ --- подалгебра $P_\infty[K]$
\end{lemma}
\begin{proof}
    $$(q_1\cdot q_2)p=q_1(q_2 p)\in J$$
\end{proof}
\begin{example}
    $J_\alpha=\{p\in P_\infty[K] : p(\alpha)=0\}$ --- идеал
\end{example}
\begin{lemma}
    $] q\in P_\infty[K] \Rightarrow J_q=q\cdot P_\infty[K]$ --- идеал в $P_\infty[K]$
\end{lemma}
\begin{proof}
    $] r\in J_q \Rightarrow \exists p\in P_\infty[K] : r = q\cdot p$

    $] \tilde p\in P_\infty[K]$

    $r\tilde p = (qp)\tilde p=q(p\tilde p)$
    
    $p\tilde p \in P_\infty[K] \Rightarrow q(p\tilde p)\in q\cdot P_\infty[K]=J_q \Rightarrow J_q \text{ --- идеал}$
\end{proof}
\begin{definition}
    Полином $q : J_q = q\cdot P_\infty[K]$ называется \textbf{порождающим полиномом} идеала $J_q$
\end{definition}
\begin{remark}
    Если идеал содержит $1(\lambda)$, то данный идеал совпадает с $P_\infty[K]$:
    $$J_1=1\cdot P_\infty[K]=P_\infty[K]$$
\end{remark}
\begin{definition}
    $] J_1$ и $J_2$ --- идеалы в $P_\infty[K]$
    \begin{enumerate}
        \item Суммой $J_1 + J_2$ называется множество
        $$J_s=\{p\in P_\infty[K] : p = p_1 + p_2 \quad p_1\in J_1, p_2\in J_2\}$$
        \item Пересечением $J_1 \cap J_2$ называется множество:
        $$J_r=\{p\in P_\infty[K] : p\in J_1 \wedge p\in J_2\}$$
    \end{enumerate}
\end{definition}
\begin{lemma}
    $J_s$ и $J_r$ --- идеалы в $P_\infty[K]$
\end{lemma}
\begin{proof}
    $J_s=J_1+J_2$ --- идеал?

    $] q\in J_s \Rightarrow q=q_1+q_2 \quad q_1\in J_1, q_2\in J_2$

    $] p\in P_\infty[K] \quad qp = (q_1+q_2)p=q_1p+q_2p$

    $q_1p \in J_1, q_2p \in J_2 \Rightarrow q_1p+q_2p\in J_s$

    $J_r=J_1\cap J_2$ --- идеал?

    $] q\in J_r \Rightarrow q\in J_1; q\in J_2$

    $] p\in P_\infty[K] \quad qp \in J_1; qp\in J_2 \Rightarrow qp \in J_r$
\end{proof}

\begin{definition}
    Нетривиальный полином минимальной степени, содержащийся в идеале, называется \textbf{минимальным полиномом} идеала.
\end{definition}
\begin{lemma}
    Любой полином идеала $J$ делится на $p_J$ без остатка:
    $$] p\in J \Rightarrow p\mid p_J$$
\end{lemma}
\begin{proof}
    $] \exists p : p\nmid p_J \Rightarrow p=qp_J+r; \deg r < \deg p_J \Rightarrow r=p-qp_J$ : $\min$ полином --- противоречие.
\end{proof}
\begin{remark}
    Если $p_1$ и $p_2$ --- минимальные полиномы $J$ $\Rightarrow p_1=\alpha p_2; \alpha\in K$
\end{remark}
\begin{theorem}
    Минимальный полином идеала является его порождающим полиномом.
\end{theorem}
\begin{proof}
    $\forall p\in J \quad p\mid p_J\Rightarrow p=p_J\cdot q\in p_J\cdot P_\infty[K]$

    $\forall p\in q\cdot P_\infty[K] \Rightarrow p=qr; r\in P_\infty[K] \Rightarrow \forall p\mid q \Rightarrow q=p_J$
\end{proof}

\begin{lemma}
    Сравнение идеалов:
    $$J_1\subset J_2 \Leftrightarrow p_{J_1}\mid p_{J_2}$$
\end{lemma}
\begin{proof}
    ``$\Rightarrow$''

    $J_1\subset J_2 \Rightarrow p_{J_1}\in J_2 \Rightarrow p_{J_1}\mid p_{J_2}$

    ``$\Leftarrow$''

    $] p_{J_1}\mid p_{J_2} \Rightarrow p_{J_1} = rp_{J_2}$

    $\forall q\in J_1 \quad q=\tilde q p_{J_1}=\tilde r P_{J_2} \Rightarrow q\mid p_{J_2} \Rightarrow J_1\subset J_2$
\end{proof}
\begin{lemma}
    О минимальном полиноме пересечения

    $$J_1\leftrightarrow p_{J_1} \ \ J_2\leftrightarrow p_{J_2} \Rightarrow J_r = J_1\cap J_2 \leftrightarrow r_J=\text{НОК}(p_{J_1}, p_{J_2})$$
\end{lemma}
\begin{proof}
    $J_r=J_1\cap J_2 \Rightarrow J_r\subset J_1 \wedge J_r\subset J_2 \Rightarrow r_J \mid p_{J_1} \wedge r_J\mid p_{J_2} \Rightarrow r_J=\text{НОК}(p_{J_1}, p_{J_2})$
\end{proof}

\begin{lemma}
    О минимальном полиноме суммы

    $$J_s=J_1 + J_2 \Rightarrow S_J=\text{НОД}(p_{J_1}, p_{J_2})$$
\end{lemma}
\begin{proof}
    $J_s=J_1+J_2 \Rightarrow J_S\supset J_1 \wedge J_S\supset J_2 \Rightarrow p_{J_1} \mid S_J \wedge p_{J_2} \mid S_j \Rightarrow S_j=\text{НОД}(p_{J_1}, p_{J_2})$
\end{proof}

\begin{theorem}
    О взаимно простых полиномах

    $] p_1, p_2$ --- взаимно простые, т.е. $\text{НОД}(p_1, p_2)=1 \Rightarrow \exists q_1, q_2\in P_\infty[K] : p_1q_1 + p_2q_2=1$
\end{theorem}
\begin{proof}
    $p_1\leftrightarrow J_1=p_1 P_\infty[K]$

    $p_2\leftrightarrow J_2=p_2 P_\infty[K]$

    $\text{НОД}(p_1, p_2)=1\leftrightarrow J_1+J_2=P_\infty[K]$

    $p_1q_1 + p_2q_2=1$
\end{proof}

\begin{theorem}
    Обобщение

    $$p_1\ldots p_k \in P_\infty[K], \text{НОД}(p_1\ldots p_k)=1\Rightarrow \exists q_1\ldots q_k : \sum_{i=1}^k p_iq_i=1$$
\end{theorem}
\begin{proof}
    Аналогично.
\end{proof}

\begin{remark}
    $] p = p_1\cdot p_2\cdots p_k, \{p_i\}$ взаимно простые $\Rightarrow \exists q_1\ldots q_k : p_1'q_1+p_2'q_2+\ldots+p_k'q_k=1, p_j'=\frac{p}{p_j}$
\end{remark}

\section*{Алгебра операторных полиномов}

$] \varphi : X\to X$ --- линейный оператор

\begin{definition}
    \textbf{Операторным полиномом} $p(\varphi)$ называется полином вида:
    $$p(\varphi)=\sum_{i=1}^n \alpha_i\varphi^i, \varphi^0=I$$
\end{definition}

\begin{definition}
    $\mathcal P_\varphi=\{p_n(\varphi)\quad \forall n\}$ --- \textbf{множество операторных полиномов}
\end{definition}

\begin{lemma}
    $\mathcal P_\varphi$ --- линейное пространство
\end{lemma}
\begin{lemma}
    $\mathcal P_\varphi$ --- коммутативная алгебра
\end{lemma}
\begin{proof}
    $$\forall p(\varphi), q(\varphi) \quad p(\varphi)q(\varphi)=q(\varphi)p(\varphi) \Leftrightarrow \varphi^m\varphi^n=\varphi^n\varphi^m$$
\end{proof}

$\sphericalangle S_\varphi : P_\infty[K] \to \mathcal P_\varphi$

$$p(\lambda)=\sum_{i=1}^n \alpha_i\lambda^i \stackrel{S_\varphi}{\mapsto} p(\varphi)=\sum_{i=1}^n \alpha_i\varphi^i$$

\begin{lemma}
    $S_\varphi$ --- гомоморфизм алгебр $P_\infty[K]$ и $\mathcal P_\varphi$
\end{lemma}
\begin{proof}
    $$p(\lambda)+q(\lambda)=(p+q)(\lambda)=\sum_{i=1}^n(\lambda_i+\beta_i)\lambda^i\mapsto \sum_{i=1}^n(\lambda_i+\beta_i)\varphi^i=(p+q)(\varphi)=p(\varphi)+q(\varphi)$$
    $$\alpha p(\lambda)\mapsto \alpha p(\varphi) \text{ аналогично}$$
    $$p(\lambda)q(\lambda)=(pq)(\lambda)=\sum_{i=1}^n\sum_{j=1}^n \alpha_i\beta_j\lambda^{i+j}\mapsto \sum_{i=1}^n\sum_{j=1}^n \alpha_i\beta_j\varphi^{i+j}=(pq)(\varphi)=p(\varphi)q(\varphi)$$
    $$1\mapsto \varphi^0=I$$
\end{proof}

\begin{lemma}
    $p_1, p_2 \in P_\infty[K], \text{НОД}(p_1, p_2)=1\Rightarrow \exists q_1, q_2\in P_\infty[K] : p_1(\varphi)q_1(\varphi)+p_2(\varphi)q_2(\varphi)=I$
\end{lemma}
\begin{proof}
    Применим к обеим частям $p_1(\lambda)q_1(\lambda)+p_2(\lambda)q_2(\lambda)=1$ отображение $S_\varphi$ :
    $$p_1(\varphi)q_1(\varphi)+p_2(\varphi)q_2(\varphi)=I$$
\end{proof}

\begin{theorem}
    О сумме ядер.

    $] p=p_1\cdot p_2, \ \ p_1, p_2\in P_\infty[K]$

    $] p_1, p_2$ --- взаимно простые

    Тогда
    $$\ker p(\varphi)=\ker p_1(\varphi)\dot+\ker p_2(\varphi)$$
    Т.е., по определению $\dot+$:
    $$\forall x\in\ker p(\varphi) \ \ \exists! x_1\in\ker p_1(\varphi), x_2\in \ker p_2(\varphi) \quad x=x_1+x_2$$
\end{theorem}
\begin{proof}
    Покажем, что $\ker p_1(\varphi)+\ker p_2(\varphi)\subset \ker p(\varphi)$

    $] x_j\in \ker p_j(\varphi) \Rightarrow$
    $$p(\varphi)(x)=p(\varphi)(x_1+x_2)=p_1(\varphi)p_2(\varphi)x_1+p_1(\varphi)p_2(\varphi)x_2=0+0=0$$
    Покажем, что $\ker p(\varphi)\subset \ker p_1(\varphi)+\ker p_2(\varphi)$

    $] x\in\ker p(\varphi)$

    $$\text{НОД}(p_1, p_2)=1 \Rightarrow p_1(\varphi)q_1(\varphi)+p_2(\varphi)q_2(\varphi)=I$$
    $$x=Ix = p_1(\varphi)q_1(\varphi)x+p_2(\varphi)q_2(\varphi)x$$
    $$p_2(\varphi)q_2(\varphi)x\in\ker p_1(\varphi)\Leftarrow p_1(\varphi)p_2(\varphi)q_2(\varphi)x=p(\varphi)q_2(\varphi)x=0$$

    Покажем, что $\ker p_1(\varphi)+\ker p_2(\varphi)$ --- прямая сумма

    $\sphericalangle \ker p_1(\varphi)\cap \ker p_2(\varphi)\ni z$
    $$z=Iz=p_1(\varphi)q_1(\varphi)z+p_2(\varphi)q_2(\varphi)z=0+0=0\Rightarrow \dim \ker p_1(\varphi)\cap\ker p_2(\varphi)=0\Rightarrow$$
    $$\Rightarrow \ker p_1(\varphi)+\ker p_2(\varphi) \text{ --- прямая сумма}$$
\end{proof}

\begin{remark}
    Пусть $p=p_1\ldots p_k, \{p_k\}$ --- взаимно простые $\Rightarrow$
    $$\ker p(\varphi)=\dot+\sum_{j=1}^k \ker p_j(\varphi)$$
\end{remark}

\end{document}