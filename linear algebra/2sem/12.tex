\documentclass[12pt, a4paper]{article}

\usepackage{lastpage}
\usepackage{mathtools}
\usepackage{xltxtra}
\usepackage{libertine}
\usepackage{amsmath}
\usepackage{amsthm}
\usepackage{amsfonts}
\usepackage{amssymb}
\usepackage{enumitem}
\usepackage{xcolor}
\usepackage[left=1.5cm, right=1.5cm, top=2cm, bottom=2cm, bindingoffset=0cm, headheight=15pt]{geometry}
\usepackage{fancyhdr}
\usepackage[russian]{babel}
% \usepackage[utf8]{inputenc}
\usepackage{catchfilebetweentags}
\usepackage{accents}
\usepackage{calc}
\usepackage{etoolbox}
\usepackage{mathrsfs}
\usepackage{wrapfig}

\providetoggle{useproofs}
\settoggle{useproofs}{false}

\pagestyle{fancy}
\lfoot{M3137y2019}
\rhead{\thepage\ из \pageref{LastPage}}

\newcommand{\R}{\mathbb{R}}
\newcommand{\Q}{\mathbb{Q}}
\newcommand{\C}{\mathbb{C}}
\newcommand{\Z}{\mathbb{Z}}
\newcommand{\B}{\mathbb{B}}
\newcommand{\N}{\mathbb{N}}

\newcommand{\const}{\text{const}}

\newcommand{\teormin}{\textcolor{red}{!}\ }

\DeclareMathOperator*{\xor}{\oplus}
\DeclareMathOperator*{\equ}{\sim}
\DeclareMathOperator{\Ln}{\text{Ln}}
\DeclareMathOperator{\sign}{\text{sign}}
\DeclareMathOperator{\Sym}{\text{Sym}}
\DeclareMathOperator{\Asym}{\text{Asym}}
% \DeclareMathOperator{\sh}{\text{sh}}
% \DeclareMathOperator{\tg}{\text{tg}}
% \DeclareMathOperator{\arctg}{\text{arctg}}
% \DeclareMathOperator{\ch}{\text{ch}}

\DeclarePairedDelimiter{\ceil}{\lceil}{\rceil}
\DeclarePairedDelimiter{\abs}{\left\lvert}{\right\rvert}

\setmainfont{Linux Libertine}

\theoremstyle{plain}
\newtheorem{axiom}{Аксиома}
\newtheorem{lemma}{Лемма}

\theoremstyle{remark}
\newtheorem*{remark}{Примечание}
\newtheorem*{exercise}{Упражнение}
\newtheorem*{consequence}{Следствие}
\newtheorem*{example}{Пример}
\newtheorem*{observation}{Наблюдение}

\theoremstyle{definition}
\newtheorem{theorem}{Теорема}
\newtheorem*{definition}{Определение}
\newtheorem*{obozn}{Обозначение}

\setlength{\parindent}{0pt}

\newcommand{\dbltilde}[1]{\accentset{\approx}{#1}}
\newcommand{\intt}{\int\!}

% magical thing that fixes paragraphs
\makeatletter
\patchcmd{\CatchFBT@Fin@l}{\endlinechar\m@ne}{}
  {}{\typeout{Unsuccessful patch!}}
\makeatother

\newcommand{\get}[2]{
    \ExecuteMetaData[#1]{#2}
}

\newcommand{\getproof}[2]{
    \iftoggle{useproofs}{\ExecuteMetaData[#1]{#2proof}}{}
}

\newcommand{\getwithproof}[2]{
    \get{#1}{#2}
    \getproof{#1}{#2}
}

\newcommand{\import}[3]{
    \subsection{#1}
    \getwithproof{#2}{#3}
}

\newcommand{\given}[1]{
    Дано выше. (\ref{#1}, стр. \pageref{#1})
}

\renewcommand{\ker}{\text{Ker }}
\newcommand{\im}{\text{Im }}
\newcommand{\grad}{\text{grad}}

\lhead{Линейная алгерба}
\cfoot{}
\rfoot{Лекция 12}

\renewcommand{\thesubsection}{\arabic{subsection}.}
\makeatletter
\renewcommand*{\@seccntformat}[1]{\csname the#1\endcsname\hspace{0.1cm}}
\makeatother

\begin{document}

\section*{Тензоры в еквлидовом пространстве}

\subsection{Естественный изоморфизм $X$ и $X^*$}

$] X$ --- линейное пространство, $X^*$ сопряжено $X$

В $X$ введена еквлидова структура $\forall x,y\in X \ \ \langle x,y\rangle \in K$, удовлетворяющая свойствам \textit{(см. прошлые лекции)}

$\sphericalangle \langle x,y\rangle = \tilde x(y), \tilde x\in X^*$, т.е. для фиксированного $x$ отображение $\langle x,y\rangle$ это форма $\tilde x\in X^*$

\begin{lemma}
    Скалярное произведение устанавливает естественный изоморфизм пространств $X$ и $X^*$:
    $$x\in X \leftrightarrow \tilde x\in X^*$$
\end{lemma}
\begin{proof}
    Необходимо и достаточно показать, что это биекция, сохраняющая линейную структуру \textit{(по определению изоморфизма)}

    \begin{enumerate}
        \item Биективность: \textit{(от противного)}
        \begin{enumerate}
            \item Слева направо $$] \exists \tilde x_1, \tilde x_2 \in X^* : x\leftrightarrow \tilde x_1, x\leftrightarrow \tilde x_2$$
            $$\tilde x_1(y) - \tilde x_2(y) = \langle x, y\rangle - \langle x, y\rangle = 0 \Rightarrow \tilde x_1 - \tilde x_2 = \Theta \Rightarrow \tilde x_1 = \tilde x_2$$
        
            \item Справа налево $$] \tilde x \leftrightarrow x_1, \tilde x \leftrightarrow x_2$$
            $$0 = \tilde x(y) - \tilde x(y) = \langle x_1, y\rangle - \langle x_2, y\rangle = \langle x_1-x_2, y\rangle$$
            $$\Rightarrow x_1-x_2=0\Rightarrow x_1=x_2$$
        \end{enumerate}
        \item Линейность: \textcolor{red}{скипнуто}
    \end{enumerate}
\end{proof}

$\sphericalangle \{e_j\}_{j=1}^n$ --- базис $X$, $\{f^k\}$ --- базис $X^*$, сопряженный базису $\{e_j\}$

$f^k(e_j)=\langle e^k, e_j\rangle = \delta_j^k$

\begin{definition}
    $\{e_j\}_{j=1}^n, \{e^k\}_{k=1}^n$ --- \textbf{биортогональные базисы} $X$, если:
    $$\langle e^k, e_j\rangle = \delta_j^k$$
\end{definition}

Разложим $\{e_i\}$ по $\{e^k\}$:
$$e_i=\sum_{k=1}^n \alpha_{ik} e^k$$
Домножим скалярно на $e_j$:
$$\langle e_i, e_j\rangle = \langle \sum_{k=1}^n \alpha_{ik} e^k, e_j\rangle=\sum_{k=1}^n \alpha_{ik} \langle e^k, e_j \rangle=\sum_{k=1}^n \alpha_{ik} \delta^i_j=\alpha_{ij}$$
Таким образом, метрический тензор монжо получить из разложения $\{e_i\}$ по $\{e^k\}$

\begin{lemma}
    $$e_i = \sum_{k=1}^n g_{ik}e^k \quad e^k = \sum_{i=1}^n g^{ki} e_i$$
\end{lemma}
\begin{proof}
    выше
\end{proof}

\begin{lemma}
    О переходе в базис, биортогональный исходному.
    $$x \in X \quad x = \sum_{j=1}^n \xi^j e_j, x = \sum_{k=1}^n \xi_k e^k$$
    Тогда:
    $$\xi^j = \sum_{k=1}^n \xi_k g^{ki}, \xi_k = \sum_{j=1}^n \xi^j g_{jk}$$
\end{lemma}
\begin{proof}
    $$x = \sum_{j=1}^n \xi^j e_j = \sum_{j=1}^n \xi^j \sum_{k=1}^n g_{jk}e^k = \sum_{k=1}^n \left(\sum_{j=1}^n \xi^j g_{jk}\right)e^k$$
\end{proof}

\begin{lemma}
    $$x = \sum_{j=1}^n \xi^j e_j \Rightarrow \xi^j = \langle e^i, x\rangle$$
\end{lemma}
\begin{proof}
    $$\langle e^i, x\rangle = \langle e^i, \sum_{j=1}^n \xi^j e_j\rangle = \sum_{j=1}^n \langle e^i, e_j\rangle\xi^j =\xi^i$$
\end{proof}

\begin{remark}
    $$x = \sum_{j=1}^n \langle e^j, x\rangle e_j = \sum_{k=1}^n \langle x, e_k\rangle e^k$$
\end{remark}

\begin{lemma}
    Явный вид изоморфиза $X\simeq X^*$:
    $$x = \sum_{k=1}^n \langle x, e_k\rangle e^k \mapsto \sum_{k=1}^n \langle x, e_k\rangle f^k$$
\end{lemma}

\begin{remark}
    Биортогональный базис к нормированному базису --- он сам:

    $\{e_j\}_{j=1}^n$ --- ортономированный $\Rightarrow g_{ij}=\delta^i_j \Rightarrow g^{ij} = \delta^i_j \Rightarrow e_j=\sum_{k=1}^n g_{jk}e^k = \sum_{k=1}^n \delta_{jk} e^k = e^j$
\end{remark}

\subsection{Сопряженные и эрмитовские операторы}

\begin{definition}
    Оператор $\varphi^* : X \to X$ называется \textbf{сопряженными оператору} $\varphi : X \to X$, если: $$\langle x, \varphi y \rangle = \langle \varphi^* x , y \rangle$$
\end{definition}

\begin{theorem}
    В конечномерном евклидовом простраснстве $\forall \varphi \ \exists! \varphi^*$
\end{theorem}
\begin{proof}
    Рассматриваем $\C$, поэтому черта --- комплексное сопряжение.

    $\{e_j\}_{j=1}^n$ --- базис $X$
    $$\varphi(e_i) = \sum_{j=1}^n a_i^j e_j$$
    $$x = \sum_{j=1}^n \xi^j e_j \quad y = \sum_{k=1}^n \eta^k e_k \quad \langle x , \varphi y\rangle \stackrel{?}{=} \langle \varphi^* x, y \rangle$$
    $$\langle x, \varphi y \rangle = \langle \sum_{j=1}^n \xi^j e_j, \varphi\left(\sum_{k=1}^n \eta^k e_k\right)\rangle = \sum_{j,k=1}^n \overline \xi^j \eta^k \langle e_j, \varphi e_k\rangle = \sum_{i,j,k=1}^n \overline \xi^j \eta^k a_k^i \underbrace{\langle e_j, e_i\rangle}_{\delta_{ji}}=$$
    $$=\sum_{j,k=1}^n \overline\xi^j \eta^k a_k^j = \sum_{k=1}^n \left(\sum_{j=1}^n \overline{\overline a_k^i \xi_i }\right)\eta^k=\sum_{k=1}^n \overline \zeta_k\eta^k$$
    $$\zeta_k = \sum_{i=1}^n \overline a_k^i \xi_i = \zeta^k$$
    $$x\leftrightarrow \xi \Rightarrow \varphi^* x \leftrightarrow \zeta \quad \zeta = \overline A^T \xi$$
    $$\varphi \leftrightarrow A \Rightarrow \varphi^* = \overline A^T \stackrel{def}{=} A^+$$
\end{proof}

Свойства операции сопряжения оператора:
\begin{enumerate}
    \item $(\varphi^*)^* = \varphi$
    \item $(\varphi\pm\psi)^*=\varphi^*\pm\psi^*$
    \item $(\alpha \varphi)^* = \overline \alpha \varphi^*$
    \item $(\varphi\psi)^* = \psi^*\varphi^*$ --- порядок важен, умножение операторов не коммутативно
\end{enumerate}

\begin{definition}
    Оператор, обладающий свойством $\varphi^*=\varphi$ называется \textbf{эрмитовским} или \textbf{самосопряженным}
\end{definition}

\begin{example}
    $\varphi\leftrightarrow A = \begin{bmatrix}
        2 & 2i \\
        -2i & 5
    \end{bmatrix} \quad A^T = \begin{bmatrix}
        2 & -2i \\
        2i & 5
    \end{bmatrix} \quad A^+ = \begin{bmatrix}
        2 & 2i \\
        -2i & 5
    \end{bmatrix} = A$
\end{example}

\begin{lemma}
    $\varphi$ эрмитов $\Rightarrow \sigma_\varphi\subset\R$
\end{lemma}
\begin{proof}
    $x$ --- СВ $\Rightarrow \varphi x = \lambda x$:
    $$\langle \varphi x, x \rangle = \langle \lambda x, x \rangle = \overline \lambda \langle x, x \rangle$$
    $$\langle x, \varphi x \rangle = \langle x, \lambda x \rangle = \lambda \langle x, x \rangle$$
    $$\lambda = \overline \lambda \Rightarrow \lambda\in\R$$
\end{proof}

\begin{lemma}
    $\varphi$ эрмитов, $\lambda_1, \lambda_2\in\sigma_\varphi : \lambda_1 \not= \lambda_2$

    \textcolor{red}{Скипнуто}
\end{lemma}

\begin{definition}
    $\varphi : X \to X$, $L$ --- инвариантное подпространство $\varphi$

    $L$ называется \textbf{приводящим подпространством} $\varphi$, если $L^\perp$ --- тоже инвариантное подпространство $\varphi$
\end{definition}

\begin{lemma}
    Любое инвариантное подпространство $L$ эрмитова оператора является приводящим.
\end{lemma}
\begin{proof}
    $x\in L, y\in L^\perp$
    $$0 = \langle \varphi x, y \rangle = \langle x, \varphi y \rangle = 0 \Rightarrow L^\perp\ \ inv$$
\end{proof}

\begin{theorem}
    Эримтов оператор --- скалярного типа.
\end{theorem}
\begin{proof}
    От противного.

    $\{x_j\}_{j=1}^m$ --- набор собственных векторов эрмитова оператора $\varphi, m<\dim X$

    $L = \mathcal{L}\{x_1\ldots  x_m\}, \dim L = m$

    $L$ --- инвариантное подпространство $X \Rightarrow L$ приводящее для $X$

    $\Rightarrow L^\perp$ инвариантное подпространство $\varphi \Rightarrow \exists$ хотя бы 1 собственный вектор оператора $\varphi$, $] y$ --- СВ $\varphi$

    $y\perp L \Rightarrow \{x_1\ldots x_m y\}$ --- ЛНЗ, противоречие.
\end{proof}

\begin{consequence}
    \begin{enumerate}
        \item Эрмитов оператор имеет скалярный тпип
        \item $\forall \lambda : r_i = n_i$
        \item Из собственных векторов эрмитова оператора можно построить ортономированный базис.
    \end{enumerate}
\end{consequence}

\textcolor{red}{Скипнуто}

\begin{theorem}
    Спектральная теорема для эрмитова оператора

    $\varphi : X \to X$, $\varphi e_j = \lambda_j e_j$, $\{e_j\}$ --- ортономированный базис СВ

    $$\Rightarrow \varphi = \sum_{i=1}^n \lambda_i \mathcal{P}_{\lambda_i} = \sum_{i=1}^n \lambda_i \langle e_i, \cdot \rangle e_i$$
\end{theorem}

\end{document}