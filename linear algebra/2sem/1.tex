\documentclass[12pt, a4paper]{article}

\usepackage{lastpage}
\usepackage{mathtools}
\usepackage{xltxtra}
\usepackage{libertine}
\usepackage{amsmath}
\usepackage{amsthm}
\usepackage{amsfonts}
\usepackage{amssymb}
\usepackage{enumitem}
\usepackage{xcolor}
\usepackage[left=1.5cm, right=1.5cm, top=2cm, bottom=2cm, bindingoffset=0cm, headheight=15pt]{geometry}
\usepackage{fancyhdr}
\usepackage[russian]{babel}
% \usepackage[utf8]{inputenc}
\usepackage{catchfilebetweentags}
\usepackage{accents}
\usepackage{calc}
\usepackage{etoolbox}
\usepackage{mathrsfs}
\usepackage{wrapfig}

\providetoggle{useproofs}
\settoggle{useproofs}{false}

\pagestyle{fancy}
\lfoot{M3137y2019}
\rhead{\thepage\ из \pageref{LastPage}}

\newcommand{\R}{\mathbb{R}}
\newcommand{\Q}{\mathbb{Q}}
\newcommand{\C}{\mathbb{C}}
\newcommand{\Z}{\mathbb{Z}}
\newcommand{\B}{\mathbb{B}}
\newcommand{\N}{\mathbb{N}}

\newcommand{\const}{\text{const}}

\newcommand{\teormin}{\textcolor{red}{!}\ }

\DeclareMathOperator*{\xor}{\oplus}
\DeclareMathOperator*{\equ}{\sim}
\DeclareMathOperator{\Ln}{\text{Ln}}
\DeclareMathOperator{\sign}{\text{sign}}
\DeclareMathOperator{\Sym}{\text{Sym}}
\DeclareMathOperator{\Asym}{\text{Asym}}
% \DeclareMathOperator{\sh}{\text{sh}}
% \DeclareMathOperator{\tg}{\text{tg}}
% \DeclareMathOperator{\arctg}{\text{arctg}}
% \DeclareMathOperator{\ch}{\text{ch}}

\DeclarePairedDelimiter{\ceil}{\lceil}{\rceil}
\DeclarePairedDelimiter{\abs}{\left\lvert}{\right\rvert}

\setmainfont{Linux Libertine}

\theoremstyle{plain}
\newtheorem{axiom}{Аксиома}
\newtheorem{lemma}{Лемма}

\theoremstyle{remark}
\newtheorem*{remark}{Примечание}
\newtheorem*{exercise}{Упражнение}
\newtheorem*{consequence}{Следствие}
\newtheorem*{example}{Пример}
\newtheorem*{observation}{Наблюдение}

\theoremstyle{definition}
\newtheorem{theorem}{Теорема}
\newtheorem*{definition}{Определение}
\newtheorem*{obozn}{Обозначение}

\setlength{\parindent}{0pt}

\newcommand{\dbltilde}[1]{\accentset{\approx}{#1}}
\newcommand{\intt}{\int\!}

% magical thing that fixes paragraphs
\makeatletter
\patchcmd{\CatchFBT@Fin@l}{\endlinechar\m@ne}{}
  {}{\typeout{Unsuccessful patch!}}
\makeatother

\newcommand{\get}[2]{
    \ExecuteMetaData[#1]{#2}
}

\newcommand{\getproof}[2]{
    \iftoggle{useproofs}{\ExecuteMetaData[#1]{#2proof}}{}
}

\newcommand{\getwithproof}[2]{
    \get{#1}{#2}
    \getproof{#1}{#2}
}

\newcommand{\import}[3]{
    \subsection{#1}
    \getwithproof{#2}{#3}
}

\newcommand{\given}[1]{
    Дано выше. (\ref{#1}, стр. \pageref{#1})
}

\renewcommand{\ker}{\text{Ker }}
\newcommand{\im}{\text{Im }}
\newcommand{\grad}{\text{grad}}

\lhead{Линейная алгерба}
\cfoot{}
\rfoot{Практика 1}

\begin{document}

\section{Тензорная алгебра}

\subsection{Напоминание теории}

Напоминание: ПЛФ --- штука, которая берет набор из $p$ векторных пространств и $q$ сопряженных пространств и эта штука линейна по всем аргументам.
\[W:X\times X\times \ldots \times X\times X^*\times X^*\times\ldots\times X^*\to K\]

Полилинейность: $W(\ldots \tilde x_s + \alpha\overline x_s\ldots ) = W(\ldots \tilde x_s\ldots ) + \alpha W(\ldots \overline x_s)$

$] \{e_j\}_{j=1}^n$ --- базис \(X\), \(\{f^k\}_{k=1}^n\) --- сопр. базис \(X^*\)

\[x_i=\sum\limits_{j=1}^n \xi_i^j e_j,\ \ y^l=\sum\limits_{k=1}^n \eta ^l_k f^k\]

\[W(x_1\ldots x_p, y^1\ldots y^q)=\xi_1^{j_1}\xi_2^{j_2}\cdots \xi_p^{j_p}\eta^1_{k_1}\eta^2_{k_2}\cdots \eta^q_{k_q} W(e_{j_1} \ldots e_{j_p} f^{k_1}\ldots f^{k_q})=\]
\[=\xi_1^{j_1}\xi_2^{j_2}\cdots \xi_p^{j_p}\eta^1_{k_1}\eta^2_{k_2}\cdots \eta^q_{k_q} w_{\vec j}^{\vec k}\]

В произвольном базисе любой ПЛФ $W$ взаимооднозначно сопоставляется тензор $w$.

\(\dim X = n =\dim X^*, \quad n^{p+q}\) --- размерность пространства ПЛФ, \((p, q)\) --- валентность. $r:=p+q$

Частные случаи:
\begin{itemize}
    \item \(r=0 \quad w=\const\)
    \item \(r=1 \quad \dim=n^1=n \Rightarrow \begin{cases}
        X \Rightarrow w=\begin{bmatrix}
            w^1 \\
            \vdots \\
            w^n
        \end{bmatrix} \\
        X^*\Rightarrow w = \begin{bmatrix}
            w_1 & \ldots & w_n
        \end{bmatrix}
    \end{cases}\)
    \item \(r = 2\quad \dim=n^2\quad w\leftrightarrow \begin{bmatrix}
        w_{11} & \ldots & w_{1n} \\
        \vdots & \ddots & \vdots \\
        w_{n1} & \ldots & w_{nn} \\
    \end{bmatrix}\)
\end{itemize}

Индексы читаются слева направо и сверху вниз, сначала строка, потом столбец.

\begin{example}
    \(n=2\quad W-(0, 2)\)

    \[w=\begin{bmatrix}
        w^{11} & w^{12} \\
        w^{21} & w^{22}
    \end{bmatrix}\]
\end{example}

\begin{itemize}
    \item \(r=3\quad n^3\). Можно писать любым из следующих вариантов: $w^{ijk}, w^{ij}_k, w^{i}_{jk}, w_{ijk}$. Последний индекс называется индексом слоя.
\end{itemize}

\begin{example}
    \(n=3\)
    \[w=
    \left[
    \begin{array}{@{}ccc|ccc|ccc@{}}
        w^{1}_{11} & w^{1}_{21} & w^{1}_{31} & w^{1}_{12} & \ldots &\ldots&\ldots&\ldots&\ldots\\
        w^{2}_{11} & w^{2}_{21} & w^{2}_{31} & \ldots \\
        w^{3}_{11} & w^{3}_{21} & w^{3}_{31} & \ldots \\
    \end{array}
    \right]
    \]
\end{example}
\begin{example}
    \(\varepsilon_{ijk}=\begin{cases}
        1 & (i, j, k) \text{ --- чётню} \\
        -1 & (i, j, k) \text{ --- нечётно} \\
        0 & \text{ иначе}
    \end{cases}\)
    \[\varepsilon=
    \left[
    \begin{array}{@{}ccc|ccc|ccc@{}}
        0&0&0&0&0&-1&0&1&0\\
        0&0&1&0&0&0&-1&0&0\\
        0&-1&0&1&0&0&0&0&0\\
    \end{array}
    \right]
    \]
\end{example}

\begin{itemize}
    \item \(r=4, n=2\)
    \[w=
    \left[
    \begin{array}{@{}cc|cc@{}}
        w_{11}^{11} & w_{11}^{12} &.&. \\
        w_{11}^{21} & w_{11}^{22} &.&. \\
        \hline
        .&.&.&.\\
        .&.&.&.\\
    \end{array}
    \right]
    \]
\end{itemize}

\begin{example}
    $n=2$

    $\delta^{ijk}_l=\begin{cases}
        1 & i=j\not=k=l \\
        -1 & i=k\not=j=l \\
        0 & \text{иначе}
    \end{cases}$

    \[w=
    \left[
    \begin{array}{@{}cc|cc@{}}
        0&0&0&-1\\
        0&1&0&0\\
        \hline
        0&0&1&0\\
        -1&0&0&0\\
    \end{array}
    \right]
    \]
\end{example}

\subsection{Операции над тензорами}

\subsubsection{Линейные операции}

Эти операции аналогичны операциям на соответствующих матрицах.

\subsubsection{Тензорное произведение}

\(a^{ij}\cdot b_k=w^{ij}_k\)

\begin{example}
    $a^i_j\to A=\begin{bmatrix}
        1 & 2 \\
        3 & 4
    \end{bmatrix}$

    $b_k\to B=\begin{bmatrix}
        5 & 6
    \end{bmatrix}$

    $] c=a\otimes b \to c^i_{jk}=a^i_j\cdot b_k$

    \[c=
    \left[
    \begin{array}{@{}cc|cc@{}}
        5 & 10 & 6 & 12 \\
        15 & 20 & 18 & 24
    \end{array}
    \right]
    \]

    $] d=b\otimes a \to d=b_k\cdot a^i_j$

    \[d=
    \left[
    \begin{array}{@{}cc|cc@{}}
        5 & 6 & 10 & 12 \\
        15 & 18 & 20 & 24
    \end{array}
    \right]
    \]

    Запись сначала векторов, потом форм называется консолидация.
\end{example}

\end{document}