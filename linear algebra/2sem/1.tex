\documentclass[12pt, a4paper]{article}

\usepackage{lastpage}
\usepackage{mathtools}
\usepackage{xltxtra}
\usepackage{libertine}
\usepackage{amsmath}
\usepackage{amsthm}
\usepackage{amsfonts}
\usepackage{amssymb}
\usepackage{enumitem}
\usepackage{xcolor}
\usepackage[left=1.5cm, right=1.5cm, top=2cm, bottom=2cm, bindingoffset=0cm, headheight=15pt]{geometry}
\usepackage{fancyhdr}
\usepackage[russian]{babel}
% \usepackage[utf8]{inputenc}
\usepackage{catchfilebetweentags}
\usepackage{accents}
\usepackage{calc}
\usepackage{etoolbox}
\usepackage{mathrsfs}
\usepackage{wrapfig}

\providetoggle{useproofs}
\settoggle{useproofs}{false}

\pagestyle{fancy}
\lfoot{M3137y2019}
\rhead{\thepage\ из \pageref{LastPage}}

\newcommand{\R}{\mathbb{R}}
\newcommand{\Q}{\mathbb{Q}}
\newcommand{\C}{\mathbb{C}}
\newcommand{\Z}{\mathbb{Z}}
\newcommand{\B}{\mathbb{B}}
\newcommand{\N}{\mathbb{N}}

\newcommand{\const}{\text{const}}

\newcommand{\teormin}{\textcolor{red}{!}\ }

\DeclareMathOperator*{\xor}{\oplus}
\DeclareMathOperator*{\equ}{\sim}
\DeclareMathOperator{\Ln}{\text{Ln}}
\DeclareMathOperator{\sign}{\text{sign}}
\DeclareMathOperator{\Sym}{\text{Sym}}
\DeclareMathOperator{\Asym}{\text{Asym}}
% \DeclareMathOperator{\sh}{\text{sh}}
% \DeclareMathOperator{\tg}{\text{tg}}
% \DeclareMathOperator{\arctg}{\text{arctg}}
% \DeclareMathOperator{\ch}{\text{ch}}

\DeclarePairedDelimiter{\ceil}{\lceil}{\rceil}
\DeclarePairedDelimiter{\abs}{\left\lvert}{\right\rvert}

\setmainfont{Linux Libertine}

\theoremstyle{plain}
\newtheorem{axiom}{Аксиома}
\newtheorem{lemma}{Лемма}

\theoremstyle{remark}
\newtheorem*{remark}{Примечание}
\newtheorem*{exercise}{Упражнение}
\newtheorem*{consequence}{Следствие}
\newtheorem*{example}{Пример}
\newtheorem*{observation}{Наблюдение}

\theoremstyle{definition}
\newtheorem{theorem}{Теорема}
\newtheorem*{definition}{Определение}
\newtheorem*{obozn}{Обозначение}

\setlength{\parindent}{0pt}

\newcommand{\dbltilde}[1]{\accentset{\approx}{#1}}
\newcommand{\intt}{\int\!}

% magical thing that fixes paragraphs
\makeatletter
\patchcmd{\CatchFBT@Fin@l}{\endlinechar\m@ne}{}
  {}{\typeout{Unsuccessful patch!}}
\makeatother

\newcommand{\get}[2]{
    \ExecuteMetaData[#1]{#2}
}

\newcommand{\getproof}[2]{
    \iftoggle{useproofs}{\ExecuteMetaData[#1]{#2proof}}{}
}

\newcommand{\getwithproof}[2]{
    \get{#1}{#2}
    \getproof{#1}{#2}
}

\newcommand{\import}[3]{
    \subsection{#1}
    \getwithproof{#2}{#3}
}

\newcommand{\given}[1]{
    Дано выше. (\ref{#1}, стр. \pageref{#1})
}

\renewcommand{\ker}{\text{Ker }}
\newcommand{\im}{\text{Im }}
\newcommand{\grad}{\text{grad}}

\lhead{Линейная алгерба}
\cfoot{}
\rfoot{Лекция 1}

\begin{document}

\section{Повторение}

\subsection{Пространство ассиметричных ПЛФ}

Рассмотрим пространство полилинейных форм \(\Omega_0^p\) и базис \(\{{}^{s_1\ldots s_p}W\}\). \[{}^{s_1\ldots s_p}F:=p!\Asym \{{}^{s_1\ldots s_p}W\}\]
\[\{{}^{s_1\ldots s_p}F | 1\leq s_1<s_2<\ldots<s_p\leq n\}\text{ --- базис}\Rightarrow \dim=C_n^p\]

\begin{itemize}
    \item \(p=0 \Rightarrow \dim \Lambda^p=1\Rightarrow k\)
    \item \(p=1 \Rightarrow \dim \Lambda^p=n\Rightarrow X^*\)
    \item \(p=n \Rightarrow \dim \Lambda^p=1\Rightarrow \) псевдоскаляр

          Единственный элемент базиса: \[{}^{1\ldots n} F(x_1\ldots x_p)=p!\Asym({}^{1\ldots n}W)(x_1\ldots x_n)=p!\frac{1}{p!}\sum\limits_{(s_1\ldots s_n)} (-1)^{[s_1\ldots s_n]} {}^{s_1\ldots s_n} W(x_{s_1}\ldots x_{s_n})=\]
          \[=\sum\limits_{(s_1\ldots s_n} (-1)^{[s_1\ldots s_n]} \xi^1_{s_1}\xi^2_{s_2}\ldots \xi^n_{s_n}\stackrel{\triangle}{=}\det ||\xi_j^i||\]
\end{itemize}

\(\forall V\subset \Lambda^n\quad V=\alpha {}^{1\ldots n} F\)

\(C_n^{n-p}=C_n^p\Rightarrow \Lambda^{n-p}\simeq \Lambda^p\)

\subsection{Внешнее произведение ПЛФ}

\[U\in \Lambda^{p_1}, V\in \Lambda^{p_2} \quad U\cdot V\stackrel{?}{\in} \Lambda^p\]

\[\sphericalangle Z=U\cdot V\not\in \Lambda^{p_1+p_2}\]

\[Z(x_1\ldots x_{p_1}\ldots x_{p_1+p_2})=U(x_1\ldots x_{p_1})\cdot V(x_{p_1+1}\ldots X_{p_1+p_2})\]

Обычное произведение ПЛФ не замкнуто, но мы хотим сделать алгебру над ПЛФ \(\Rightarrow\) создадим \textbf{внешнее произведение}.

\[\in \Lambda^{p_1}, V\in \Lambda^{p_2} \quad U\wedge V := \frac{(p_1+p_2)!}{p_1!p_2!} \Asym (U\cdot V) \in \Lambda^{p_1+p_2}\]

Свойства:
\begin{enumerate}
    \item $p+q>n\Rightarrow U\wedge V=\Theta$, т.к. пространство \(\Lambda^{p_1+p_2}\) есть нуль-пространство.
    \item $U\wedge V=(-1)^{pq} V\wedge U$
          \begin{proof}
              \[U\wedge V(x_1\ldots x_{p+q})=\frac{(p+q)!}{p!q!}\frac{1}{(p+q)!}\sum\limits_{(i_1\ldots i_{p+q})} U(x_{i_1}\ldots x_{i_p}) W(x_{i_{[+1}}\ldots x_{i_{p+q}})(-1)^{[i_1\ldots i_{p+q}]}=\]\[=(-1)^{pq} V\wedge U\]
          \end{proof}
    \item $U\wedge (V\wedge W)=(U\wedge V)\wedge W=U\wedge V\wedge W=\frac{(p+q+s)!}{p!q!s!}\Asym(U\cdot V\cdot W)$
          \begin{proof}
              \[(U\wedge V)\wedge W=\left(\frac{(p+q)!}{p!q!}\Asym (U\cdot V)\right)\wedge W=\]
              \[=\frac{(p+q)!}{p!q!}\frac{(p+q+s)!}{(p+q)!s!} \Asym(\Asym(U\cdot V)\cdot W)=\frac{(p+q+s)!}{p!q!s!}\Asym(U\cdot V\cdot W)=\]
              \[=U\wedge V\wedge W\]
          \end{proof}
    \item $U\wedge (V+W)=U\wedge V + U\wedge W$
    \item $(\alpha U)\wedge V=U\wedge (\alpha V)=\alpha(U\wedge V),\quad \alpha\in K$
    \item $U\wedge\Theta=\Theta\wedge V=\Theta$
    \item $\{f^i\}_{i=1}^n$ --- базис $X^*$. Тогда \({}^{s_1\ldots s_p}F=f^{s_1}\wedge\ldots\wedge f^{s_p}\) --- базис \(\Lambda^p\)
          \begin{proof}
              \[{}^{s_1\ldots s_p} W=f^{s_1}\cdot \ldots \cdot f^{s_p}\]
              \[{}^{s_1\ldots s_p} F=p!\Asym(f^{s_1}\cdot f^{s_2} \cdot \ldots \cdot f^{s_p})=p!\frac{2!}{2!} \Asym(f^{s_1}\cdot f^{s_2} \cdot \ldots \cdot f^{s_p})=\]
              \[=\Asym\left(\frac{p!}{2!}\frac{(1+1)!}{1!1!}\Asym(f^{s_1}\cdot f^{s_2}) \cdot f^{s_3}\ldots f^{s_n}\right)=\]
              \[=\Asym\left(\frac{p!}{2!} f^{s_1}\wedge f^{s_2} \cdot f^{s_3}\ldots f^{s_n}\right)=\]
              \[=\frac{p!}{2!}\Asym\left( f^{s_1}\wedge f^{s_2} \cdot f^{s_3}\ldots f^{s_n}\right)=\]
              \[=\frac{p!}{3!}\Asym\left( f^{s_1}\wedge f^{s_2} \wedge f^{s_3}\cdot \ldots f^{s_n}\right)=\]
              \[=f^{s_1}\wedge\ldots\wedge f^{s_p}\]
          \end{proof}
\end{enumerate}

\section{Определители и их свойства}

\begin{definition}
    \(\det\{x_1\ldots x_n\}={}^{1\ldots n}F(x_1\ldots x_n)=f^1\wedge\ldots\wedge f^n(x_1\ldots x_n)\)
\end{definition}

\(x_i=\sum\limits_{j_i=1}^n\xi_i^{j_i}e_{j_i}\) --- разложение \(\Rightarrow \det\{x_1\ldots x_n\}=\sum\limits_{(j_1\ldots j_n)}(-1)^{[j_1\ldots j_n]} \xi_1^{j_1}\ldots \xi_n^{j_n}\)

\(\sphericalangle C=\begin{bmatrix}
    \xi_1^1 & \xi_2^1 & \ldots & x_n^1  \\
    \xi_1^2 & \xi_2^2 & \ldots & x_n^2  \\
    \vdots  & \vdots  & \ddots & \vdots \\
    \xi_1^n & \xi_2^n & \ldots & x_n^n  \\
\end{bmatrix}\)


Свойства:
\begin{enumerate}
    \item \(\det C=\det C^T\)
          \begin{proof}
              \(\det C = \sum\limits_{(i_1\ldots i_n)}(-1)^{[i_1\ldots i_n]}\xi_1^{i_1}\xi_2^{i_2}\ldots \xi_n^{i_n} = \sum\limits_{(j_1\ldots j_n)}(-1)^{[j_1\ldots j_n]} \xi^1_{j_1}\xi^2_{j_2}\ldots \xi^n_{j_n}=\det C^T\)
          \end{proof}
    \item \(\det \{x_1\ldots x_s\ldots x_t\ldots x_n\}=-\det \{x_1\ldots x_t\ldots x_s\ldots x_n\}\)
    \item \(\{x_i\}_{i=1}^n\) --- ЛЗ \(\Rightarrow \det \{x_1\ldots x_n\}=0\)
    \item \(\det\{x_1\ldots x_s+\sum\limits_{i=1}^n\alpha_i x_i\ldots x_n\}=\det\{x_1\ldots x_n\}\)
    \item \[\det\{x_1\ldots x_k\ldots x_n\}=f^1\wedge f^2\wedge \ldots f^n(x_1\ldots \sum\limits_{m=1}^{n}\xi_k^me_m\ldots x_n)=\]\[=\sum\limits_{m=1}^{n}\xi^m_k f^1\wedge f^2\wedge\ldots\wedge f^n(x_1\ldots e_m\ldots x_n)=\sum\limits_{m=1}^{n}\xi_k^m(-1)=\ldots\]
          \begin{remark}
              \(f^1\cdot f^2\cdot \ldots \cdot f^n(x_1\ldots x_k\ldots x_n)=f^1(x_1)f^1(x_2)\ldots f^k(x_k)\ldots f^n(x_n)\)

              \(\sphericalangle \Asym\left(f^1\cdot f^2\cdot \ldots \cdot f^n(x_1\ldots x_n) \right)=\frac{1}{n!}\sum\limits_{(s_1\ldots s_n)}(-1)^{[s_1\ldots s_n]} f^1\cdot \ldots f^n(x_{s_1}\ldots x_{x_n})=\)

              \(=\frac{1}{n!}\sum\limits_{(s_1\ldots s_n)}(-1)^{[s_1\ldots s_n]} f^{s_1}\cdot \ldots f^{s_n}(x_1\ldots x_n)\)

              Заметим, что если \(m\not=k\), то результат \(0\). Когда \(m=k\), остается \(n\) вариантов.
          \end{remark}

          \[\ldots =\sum\limits_{m=1}^n \xi_k^m (-1)^{|m-k|} f^1\wedge \ldots f^{k-1}\wedge f^{k+1}\wedge\ldots\wedge f^n(x_1\ldots x_{k-1},x_{k+1}\ldots x_n)=\]
          \[\sum\limits_{m=1}^n \xi_k^m(-1)^{|m-k|}M_k^m\]
\end{enumerate}
\begin{definition}
    \textbf{Минор} элемента \(\xi_k^m\) --- \(\det M_k^m\) матрицы, полученной из матрицы \(C\) вычеркиванием столбца номер \(K\) и строки с номером \(m\).
\end{definition}

\begin{definition}
    \textbf{Алгебраическим дополнением} элемента \(\xi_k^m\) называется число \((-1)^{m+k} M_k^m\)

    \(\det C = \sum\limits_{k=1}^n \xi_k^m A_k^m\)

    \(\det C = \sum\limits_{k=1}^n \xi_m^k A_m^k\)
\end{definition}

\begin{theorem}
    Лапласа

    \[\det C = \sum\limits_{k_1\ldots k_s} (-1)^{m_1+k_1+m_2+k_2+\ldots+m_s+k_s}=L^{m_1\ldots m_s}_{k_1\ldots k_s} M^{m_1\ldots m_s}_{k_1\ldots k_s}\]

    \(L\) --- минор порядка \(s\), \(M\) --- доп. минор порядка \(n-s\).
\end{theorem}

\begin{corollary}
    \[\begin{bmatrix}
            A_{11} & A_{12} & \ldots & A_{1n} \\
            0      & A_{22} & \ldots & A_{2n} \\
            \vdots & \vdots & \ddots & \vdots \\
            0      & 0      & \ldots & A_{nn}
        \end{bmatrix}\]
\end{corollary}

\begin{theorem}
    Критерий линейной зависимости набора векторов

    \(\{x_i\}_{i=1}^k\) --- ЛЗ \(\Leftrightarrow \forall V\in \Lambda^k \quad V(x_1\ldots x_k)=0\)
\end{theorem}
\begin{proof}
    ``\(\Leftarrow\)'' очевидно

    ``\(\Rightarrow\)'' докажем от противного.

    \(] \{x_i\}_{i=1}^k\) --- ЛНЗ : \(\forall V\in \Lambda^k \quad V(x_1\ldots x_k)=0\)

    \(\sphericalangle \{x_1\ldots x_k, y_1\ldots y_{n-k}\}\) --- базис \(X\)

    \(] \{f^1\ldots f^n\}\) --- базис \(X^*\)

    \(\sphericalangle \{f^j\}_{j=1}^k\quad f^1\wedge\ldots\wedge f^k(x_1\ldots x_k)=\tilde C\not=0\Rightarrow \) противоречие.
\end{proof}

% \begin{remark}
%     \(\{x_i\}\)
% \end{remark}

\end{document}