\documentclass[12pt, a4paper]{article}

\usepackage{lastpage}
\usepackage{mathtools}
\usepackage{xltxtra}
\usepackage{libertine}
\usepackage{amsmath}
\usepackage{amsthm}
\usepackage{amsfonts}
\usepackage{amssymb}
\usepackage{enumitem}
\usepackage{xcolor}
\usepackage[left=1.5cm, right=1.5cm, top=2cm, bottom=2cm, bindingoffset=0cm, headheight=15pt]{geometry}
\usepackage{fancyhdr}
\usepackage[russian]{babel}
% \usepackage[utf8]{inputenc}
\usepackage{catchfilebetweentags}
\usepackage{accents}
\usepackage{calc}
\usepackage{etoolbox}
\usepackage{mathrsfs}
\usepackage{wrapfig}

\providetoggle{useproofs}
\settoggle{useproofs}{false}

\pagestyle{fancy}
\lfoot{M3137y2019}
\rhead{\thepage\ из \pageref{LastPage}}

\newcommand{\R}{\mathbb{R}}
\newcommand{\Q}{\mathbb{Q}}
\newcommand{\C}{\mathbb{C}}
\newcommand{\Z}{\mathbb{Z}}
\newcommand{\B}{\mathbb{B}}
\newcommand{\N}{\mathbb{N}}

\newcommand{\const}{\text{const}}

\newcommand{\teormin}{\textcolor{red}{!}\ }

\DeclareMathOperator*{\xor}{\oplus}
\DeclareMathOperator*{\equ}{\sim}
\DeclareMathOperator{\Ln}{\text{Ln}}
\DeclareMathOperator{\sign}{\text{sign}}
\DeclareMathOperator{\Sym}{\text{Sym}}
\DeclareMathOperator{\Asym}{\text{Asym}}
% \DeclareMathOperator{\sh}{\text{sh}}
% \DeclareMathOperator{\tg}{\text{tg}}
% \DeclareMathOperator{\arctg}{\text{arctg}}
% \DeclareMathOperator{\ch}{\text{ch}}

\DeclarePairedDelimiter{\ceil}{\lceil}{\rceil}
\DeclarePairedDelimiter{\abs}{\left\lvert}{\right\rvert}

\setmainfont{Linux Libertine}

\theoremstyle{plain}
\newtheorem{axiom}{Аксиома}
\newtheorem{lemma}{Лемма}

\theoremstyle{remark}
\newtheorem*{remark}{Примечание}
\newtheorem*{exercise}{Упражнение}
\newtheorem*{consequence}{Следствие}
\newtheorem*{example}{Пример}
\newtheorem*{observation}{Наблюдение}

\theoremstyle{definition}
\newtheorem{theorem}{Теорема}
\newtheorem*{definition}{Определение}
\newtheorem*{obozn}{Обозначение}

\setlength{\parindent}{0pt}

\newcommand{\dbltilde}[1]{\accentset{\approx}{#1}}
\newcommand{\intt}{\int\!}

% magical thing that fixes paragraphs
\makeatletter
\patchcmd{\CatchFBT@Fin@l}{\endlinechar\m@ne}{}
  {}{\typeout{Unsuccessful patch!}}
\makeatother

\newcommand{\get}[2]{
    \ExecuteMetaData[#1]{#2}
}

\newcommand{\getproof}[2]{
    \iftoggle{useproofs}{\ExecuteMetaData[#1]{#2proof}}{}
}

\newcommand{\getwithproof}[2]{
    \get{#1}{#2}
    \getproof{#1}{#2}
}

\newcommand{\import}[3]{
    \subsection{#1}
    \getwithproof{#2}{#3}
}

\newcommand{\given}[1]{
    Дано выше. (\ref{#1}, стр. \pageref{#1})
}

\renewcommand{\ker}{\text{Ker }}
\newcommand{\im}{\text{Im }}
\newcommand{\grad}{\text{grad}}

\lhead{Линейная алгерба}
\cfoot{}
\rfoot{Практика 7}

\begin{document}

%<*собственныйвектор>
$\varphi : X\to X$
\begin{definition}
    $x\in X$ --- \textbf{собственный вектор} $\varphi$, если
    $$x\not=0 \quad \varphi x = \lambda x, \quad \lambda\in K$$
    $\lambda$ --- \textbf{собственное значение} $\varphi$, соответствующее $x$
\end{definition}

\begin{definition}
    \textbf{Спектр} $\sigma_\varphi=\{\lambda_1\ldots \lambda_n\}$ --- множество всех собственных значений вектора
\end{definition}
%</собственныйвектор>

\begin{example}
    Найти спектр и собственные вектора оператора $\varphi$, заданного матрицей:
    $$A=\begin{bmatrix}
        3 & -2 & 6 \\
        -2 & 6 & 3 \\
        6 & 3 & -2
    \end{bmatrix}$$

    Найдем спектр:
    $$\chi_\varphi(\lambda)=|A_\lambda E|=\begin{vmatrix}
        3 - \lambda& -2 & 6 \\
        -2 & 6 - \lambda& 3 \\
        6 & 3 & -2 - \lambda
    \end{vmatrix}=$$
    $$=(3-\lambda)((6-\lambda)(-2-\lambda)-9)+2(2(2+\lambda)-18)+6(-6-6(6-\lambda))=$$
    $$=(3-\lambda)(\lambda^2-4\lambda-21)+4(\lambda-7)-36(7-\lambda)=$$
    $$=(3-\lambda)(\lambda^2-4\lambda-21)+40(\lambda-7)=$$
    $$=(3-\lambda)(\lambda-7)(\lambda+3)+40(\lambda-7)=$$
    $$(49-\lambda^2)(\lambda-7)=(\lambda-7)^2(\lambda+7)$$

    $$\sigma_\varphi=\text{корни }\chi_\varphi=\{-7, 7^{(2)}\}$$

    Найдем собственные вектора:
    \begin{enumerate}
        \item $\lambda=-7$
        $$A\xi=\lambda\xi\Rightarrow (A-\lambda E)\xi=0 \text{ --- однор. СЛАУ}$$
        $$\begin{bmatrix}
            3 + 7 & -2 & 6 \\
            -2 & 6 + 7 & 3 \\
            6 & 3 & -2 + 7
        \end{bmatrix}=\begin{bmatrix}
            10 & -2 & 6 \\
            -2 & 13 & 3 \\
            6 & 3 & 5
        \end{bmatrix}\sim\begin{bmatrix}
            -2 & 13 & 3 \\
            0 & 42 & 14 \\
            0 & 63 & 21
        \end{bmatrix}\sim\begin{bmatrix}
            -2 & 13 & 3 \\
            0 & 3 & 1 \\
            0 & 0 & 0
        \end{bmatrix}$$

        $]\xi^3$ --- параметр $\Rightarrow \begin{cases}
            -2 \xi^1 + 13 \xi^2 = -3 \xi^3 \\
            3\xi^2 = -\xi^3
        \end{cases}$

        $] \xi^3=3 \Rightarrow \xi^2=-1, \xi^1 = -2 \Rightarrow \xi=\begin{bmatrix}
            2 \\
            -1 \\
            3
        \end{bmatrix}$

        Собственный вектор один \textit{(см. СЛАУ)}
        
        \item $\lambda=7$
        $$\begin{bmatrix}
            3 -7 & -2 & 6 \\
            -2 & 6 -7 & 3 \\
            6 & 3 & -2 -7
        \end{bmatrix}\sim\begin{bmatrix}
            -4 & -2 & 6 \\
            -2 & -1 & 3 \\
            6 & 3 & -9
        \end{bmatrix}\sim\begin{bmatrix}
            -2 & -1 & 3 \\
            0 & 0 & 0 \\
            0 & 0 & 0
        \end{bmatrix} \Rightarrow \text{два собственных вектора}$$
        $] \xi^2, \xi^3$ --- параметры

        $\xi^2=2, \xi^3=0\Rightarrow \xi^1=-1$

        $\xi^2=0, \xi^3=2\Rightarrow \xi^1=3$

        $\xi_2=\begin{bmatrix}
            -1 \\
            2 \\
            0
        \end{bmatrix} \quad \begin{bmatrix}
            3 \\
            0 \\
            2
        \end{bmatrix}$

        $\{\xi_j\}_{j=1}^3$ --- базис $X$
    \end{enumerate}

    Проверка:
    $$\begin{bmatrix}
        3 & -2 & 6 \\
        -2 & 6 & 3 \\
        6 & 3 & -2
    \end{bmatrix}\begin{bmatrix}
        2 \\
        -1 \\
        3
    \end{bmatrix}=-7\begin{bmatrix}
        2 \\
        -1 \\
        3
    \end{bmatrix}$$

    Проверим, что $A$ в базисе из собственных векторов диагональна:
    $$\tilde A = T^{-1}AT$$
    $$T^{-1}=\frac{1}{\det T}\tilde T^{T}$$
    $$\det T = \begin{vmatrix}
        2 & -1 & 3 \\
        -1 & 2 & 0 \\
        3 & 0 & 2
    \end{vmatrix}=-28$$
    $$\tilde T^T=\begin{bmatrix}
        4 & 2 & -6 \\
        2 & -5 & -3 \\
        -6 & -3 & 3
    \end{bmatrix}$$
    $$T^{-1}=\frac{-1}{28}\begin{bmatrix}
        4 & 2 & -6 \\
        2 & -5 & -3 \\
        -6 & -3 & 3
    \end{bmatrix}$$
    $$\tilde A=\frac{-1}{28}\begin{bmatrix}
        4 & 2 & -6 \\
        2 & -5 & -3 \\
        -6 & -3 & 3
    \end{bmatrix}\begin{bmatrix}
        3 & -2 & 6 \\
        -2 & 6 & 3 \\
        6 & 3 & -2
    \end{bmatrix}\begin{bmatrix}
        2 & -1 & 3 \\
        -1 & 2 & 0 \\
        3 & 0 & 2
    \end{bmatrix}=$$
    $$=\frac{-1}{28}\begin{bmatrix}
        4 & 2 & -6 \\
        2 & -5 & -3 \\
        -6 & -3 & 3
    \end{bmatrix}\begin{bmatrix}
        14 & 7 & -21 \\
        -7 & -14 & 0 \\
        21 & 0 & -14
    \end{bmatrix}=-\frac{1}{4}\begin{bmatrix}
        28 & 0 & 0 \\
        0 & -28 & 0 \\
        0 & 0 & -28
    \end{bmatrix}$$
\end{example}

\end{document}