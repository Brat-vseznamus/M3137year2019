\documentclass[12pt, a4paper]{article}

\usepackage{lastpage}
\usepackage{mathtools}
\usepackage{xltxtra}
\usepackage{libertine}
\usepackage{amsmath}
\usepackage{amsthm}
\usepackage{amsfonts}
\usepackage{amssymb}
\usepackage{enumitem}
\usepackage{xcolor}
\usepackage[left=1.5cm, right=1.5cm, top=2cm, bottom=2cm, bindingoffset=0cm, headheight=15pt]{geometry}
\usepackage{fancyhdr}
\usepackage[russian]{babel}
% \usepackage[utf8]{inputenc}
\usepackage{catchfilebetweentags}
\usepackage{accents}
\usepackage{calc}
\usepackage{etoolbox}
\usepackage{mathrsfs}
\usepackage{wrapfig}

\providetoggle{useproofs}
\settoggle{useproofs}{false}

\pagestyle{fancy}
\lfoot{M3137y2019}
\rhead{\thepage\ из \pageref{LastPage}}

\newcommand{\R}{\mathbb{R}}
\newcommand{\Q}{\mathbb{Q}}
\newcommand{\C}{\mathbb{C}}
\newcommand{\Z}{\mathbb{Z}}
\newcommand{\B}{\mathbb{B}}
\newcommand{\N}{\mathbb{N}}

\newcommand{\const}{\text{const}}

\newcommand{\teormin}{\textcolor{red}{!}\ }

\DeclareMathOperator*{\xor}{\oplus}
\DeclareMathOperator*{\equ}{\sim}
\DeclareMathOperator{\Ln}{\text{Ln}}
\DeclareMathOperator{\sign}{\text{sign}}
\DeclareMathOperator{\Sym}{\text{Sym}}
\DeclareMathOperator{\Asym}{\text{Asym}}
% \DeclareMathOperator{\sh}{\text{sh}}
% \DeclareMathOperator{\tg}{\text{tg}}
% \DeclareMathOperator{\arctg}{\text{arctg}}
% \DeclareMathOperator{\ch}{\text{ch}}

\DeclarePairedDelimiter{\ceil}{\lceil}{\rceil}
\DeclarePairedDelimiter{\abs}{\left\lvert}{\right\rvert}

\setmainfont{Linux Libertine}

\theoremstyle{plain}
\newtheorem{axiom}{Аксиома}
\newtheorem{lemma}{Лемма}

\theoremstyle{remark}
\newtheorem*{remark}{Примечание}
\newtheorem*{exercise}{Упражнение}
\newtheorem*{consequence}{Следствие}
\newtheorem*{example}{Пример}
\newtheorem*{observation}{Наблюдение}

\theoremstyle{definition}
\newtheorem{theorem}{Теорема}
\newtheorem*{definition}{Определение}
\newtheorem*{obozn}{Обозначение}

\setlength{\parindent}{0pt}

\newcommand{\dbltilde}[1]{\accentset{\approx}{#1}}
\newcommand{\intt}{\int\!}

% magical thing that fixes paragraphs
\makeatletter
\patchcmd{\CatchFBT@Fin@l}{\endlinechar\m@ne}{}
  {}{\typeout{Unsuccessful patch!}}
\makeatother

\newcommand{\get}[2]{
    \ExecuteMetaData[#1]{#2}
}

\newcommand{\getproof}[2]{
    \iftoggle{useproofs}{\ExecuteMetaData[#1]{#2proof}}{}
}

\newcommand{\getwithproof}[2]{
    \get{#1}{#2}
    \getproof{#1}{#2}
}

\newcommand{\import}[3]{
    \subsection{#1}
    \getwithproof{#2}{#3}
}

\newcommand{\given}[1]{
    Дано выше. (\ref{#1}, стр. \pageref{#1})
}

\renewcommand{\ker}{\text{Ker }}
\newcommand{\im}{\text{Im }}
\newcommand{\grad}{\text{grad}}

\lhead{Линейная алгерба}
\cfoot{}
\rfoot{Практика 5}

\usepackage{mathrsfs}

\begin{document}

\section{Линейный оператор}

\begin{example}
    $\sphericalangle \R^2_2 \quad \varphi : A\mapsto A^T$

    $A_\varphi?$ в стандартном базисе

    Стандартный матричный базис: $E_1=\begin{bmatrix}
        1 & 0 \\
        0 & 0
    \end{bmatrix} \quad E_2=\begin{bmatrix}
        0 & 1 \\
        0 & 0
    \end{bmatrix} \quad E_3 = \begin{bmatrix}
        0 & 0 \\
        1 & 0
    \end{bmatrix} \quad E_3 = \begin{bmatrix}
        0 & 0 \\
        0 & 1
    \end{bmatrix}$

    $$\varphi(E_1)=E_1 \Rightarrow A_{\varphi1}=\begin{pmatrix}
        1 & 0 & 0 & 0
    \end{pmatrix}^T$$
    $$\varphi(E_2)=E_3 \Rightarrow A_{\varphi2}=\begin{pmatrix}
        0 & 0 & 1 & 0
    \end{pmatrix}^T$$
    $$\varphi(E_3)=E_2 \Rightarrow A_{\varphi3}=\begin{pmatrix}
        0 & 1 & 0 & 0
    \end{pmatrix}^T$$
    $$\varphi(E_4)=E_4 \Rightarrow A_{\varphi4}=\begin{pmatrix}
        0 & 0 & 0 & 1
    \end{pmatrix}^T$$
    $$A_\varphi=\begin{bmatrix}
        1 & 0 & 0 & 0 \\
        0 & 0 & 1 & 0 \\
        0 & 1 & 0 & 0 \\
        0 & 0 & 0 & 1
    \end{bmatrix}$$
    $$Ker \varphi = \{0\}$$
    $$Im \varphi = \R_2^2$$
    $] M = \begin{bmatrix}
        \alpha & \beta \\
        \gamma & \delta
    \end{bmatrix} \leftrightarrow \begin{pmatrix}
        \alpha & \beta & \gamma & \delta
    \end{pmatrix}^T=m$
    
    $Ker\varphi : A_\varphi m=0$ --- это СЛАУ

    Заметим, что эта СЛАУ имеет только тривиальные решения $\Rightarrow \dim Ker \varphi =0 \Rightarrow \dim Im \varphi = \dim \R_2^2 - \dim Ker \varphi = 4 - 0 = 4$
\end{example}

\begin{example}
    $\sphericalangle \mathcal{P}^{x,y}_2$

    $$\varphi p := x\frac{\partial p}{\partial x} + y\frac{\partial p}{\partial y}$$

    $A_\varphi?$ в базисе $\{x^2 \ \ xy \ \ x^2\}$

    $Ker \varphi?$

    $$\varphi(x^2)=2x^2 \quad \varphi(xy) = 2xy \quad \varphi(y^2)=2y^2$$
    $$A_\varphi=\begin{bmatrix}
        2 & 0 & 0 \\
        0 & 2 & 0 \\
        0 & 0 & 2
    \end{bmatrix} \Rightarrow Ker \varphi = \{0\}$$
\end{example}

\begin{example}
    $$\varphi p := y\frac{\partial p}{\partial x} - x\frac{\partial p}{\partial y}$$
    $$\varphi(xy) = y^2-x^2$$
    $$\varphi(y^2) = -2xy$$
    $$A_\varphi = \begin{bmatrix}
        0 & -1 & 0 \\
        2 & 0 & -2 \\
        0 & 1 & 0
    \end{bmatrix}$$
    Найдём ядро:
    $$\begin{bmatrix}
        0 & -1 & 0 \\
        2 & 0 & -2 \\
        0 & 1 & 0
    \end{bmatrix} ~ \begin{bmatrix}
        1 & 0 & -1 \\
        0 & 1 & 0 \\
        0 & 0 & 0
    \end{bmatrix} \Rightarrow \begin{cases}
        x_1 - x_3 = 0 \\
        x_2 = 0
    \end{cases}$$
    ФСР: $\begin{bmatrix}
        1 \\ 0 \\ 1
    \end{bmatrix}$

    $$Ker \varphi = \mathscr{L}(x^2+y^2)$$
\end{example}

\begin{example}
    Найти ядро и образ оператора $\varphi$, заданного в некотором базисе матрицей:

    $\varphi : X\simeq \R^5 \to Y\simeq \R^5$

    $$A_\varphi = \begin{bmatrix}
        1 & 1 & 1 & 2 & -1 \\
        2 & 1 & -2 & -1 & 1 \\
        3 & -1 & 0 & -1 & 1 \\
        1 & -2 & 2 & 0 & 0 \\
        2 & -2 & -1 & -3 & 2
    \end{bmatrix} \sim \begin{bmatrix}
        1 & 0 & 0 & 0 & 0 \\
        2 & -1 & -4 & -5 & 3\\
        3 & -4 & -3 & -7 & 4 \\
        1 & -3 & 1 & -2 & 1 \\
        2 & -4 & -3 & -7 & 4
    \end{bmatrix} \sim \begin{bmatrix}
        1 & 0 & 0 & 0 \\
        2 & 1 & 0 & 0 \\
        3 & 4 & 13 & -8\\
        1 & 3 & 13 & -8\\
        2 & 4 & 13 & -8\\
    \end{bmatrix}\sim\begin{bmatrix}
        1 & 0 & 0 \\
        2 & 1 & 0 \\
        3 & 4 & 1 \\
        1 & 3 & 1 \\
        2 & 4 & 1 \\
    \end{bmatrix}$$

    Это базис $Im \varphi$
\end{example}

\begin{example}
    $\varphi : \R^5\to\R^3$

    $$A_\varphi = \begin{bmatrix}
        1 & 3 & 5 & 7 & 9 \\
        1 & -2 & 3 & -4 & 5 \\
        2 & 11 & 12 & 25 & 22
    \end{bmatrix} \quad a=\begin{bmatrix} 1 \\ 2 \\ 3\end{bmatrix}$$
    Найдём прообраз $a$
    $$A_\varphi x = a$$
    $$] x = \begin{pmatrix}
        \xi^1 & \xi^2 & \xi^3 & \xi^4 & \xi^5
    \end{pmatrix}^T$$

    СЛАУ:
    $$\begin{bmatrix}
        1 & 3 & 5 & 7 & 9 & 1 \\
        1 & -2 & 3 & -4 & 5 & 2 \\
        2 & 11 & 12 & 25 & 22 & 1
    \end{bmatrix}$$
\end{example}

\begin{example}
    $\varphi : \R^4 \to \R^3$
    $$A_\varphi = \begin{bmatrix}
        3 & 1 & -2 & 4 \\
        2 & -3 & 6 & -5 \\
        8 & -1 & 2 & 3
    \end{bmatrix} \quad L:\begin{cases}
        x_1 + x_2 = 0 \\
        x_1 - x_3 = 0
    \end{cases}$$
    Найдем прообраз $L$ --- найдем ФСР $L$ и полный прообраз каждого элемента ФСР.
\end{example}

\begin{example}
    $\varphi : \R^3 \to \mathbb{C}_2^2$

    $$\begin{pmatrix}
        \xi^1 & \xi^2 & \xi^3
    \end{pmatrix}^T \mapsto \begin{bmatrix}
        \xi^3 & \xi^1+i\xi^2 \\
        \xi^1 - i\xi^3 & -\xi^3
    \end{bmatrix}$$

    $Ker \varphi = \{0\}$

    $A_\varphi?$

    $$C_2^2 = \left\{
        \begin{bmatrix}
            1 & 0 \\
            0 & -1
        \end{bmatrix} \quad \begin{bmatrix}
            0 & 1 \\
            1 & 0
        \end{bmatrix} \quad \begin{bmatrix}
            0 & i \\
            -i & 0
        \end{bmatrix}    
    \right\}$$

    $$\varphi(e_1)=\begin{bmatrix}
        0 & 1 \\
        1 & 0
    \end{bmatrix} \quad \varphi(e_2) = \begin{bmatrix}
        0 & i \\
        -i & 0
    \end{bmatrix}\quad \varphi(e_3) = \begin{bmatrix}
        1 & 0 \\
        0 & -1
    \end{bmatrix}$$
\end{example}

\end{document}