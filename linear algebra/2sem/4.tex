\documentclass[12pt, a4paper]{article}

\usepackage{lastpage}
\usepackage{mathtools}
\usepackage{xltxtra}
\usepackage{libertine}
\usepackage{amsmath}
\usepackage{amsthm}
\usepackage{amsfonts}
\usepackage{amssymb}
\usepackage{enumitem}
\usepackage{xcolor}
\usepackage[left=1.5cm, right=1.5cm, top=2cm, bottom=2cm, bindingoffset=0cm, headheight=15pt]{geometry}
\usepackage{fancyhdr}
\usepackage[russian]{babel}
% \usepackage[utf8]{inputenc}
\usepackage{catchfilebetweentags}
\usepackage{accents}
\usepackage{calc}
\usepackage{etoolbox}
\usepackage{mathrsfs}
\usepackage{wrapfig}

\providetoggle{useproofs}
\settoggle{useproofs}{false}

\pagestyle{fancy}
\lfoot{M3137y2019}
\rhead{\thepage\ из \pageref{LastPage}}

\newcommand{\R}{\mathbb{R}}
\newcommand{\Q}{\mathbb{Q}}
\newcommand{\C}{\mathbb{C}}
\newcommand{\Z}{\mathbb{Z}}
\newcommand{\B}{\mathbb{B}}
\newcommand{\N}{\mathbb{N}}

\newcommand{\const}{\text{const}}

\newcommand{\teormin}{\textcolor{red}{!}\ }

\DeclareMathOperator*{\xor}{\oplus}
\DeclareMathOperator*{\equ}{\sim}
\DeclareMathOperator{\Ln}{\text{Ln}}
\DeclareMathOperator{\sign}{\text{sign}}
\DeclareMathOperator{\Sym}{\text{Sym}}
\DeclareMathOperator{\Asym}{\text{Asym}}
% \DeclareMathOperator{\sh}{\text{sh}}
% \DeclareMathOperator{\tg}{\text{tg}}
% \DeclareMathOperator{\arctg}{\text{arctg}}
% \DeclareMathOperator{\ch}{\text{ch}}

\DeclarePairedDelimiter{\ceil}{\lceil}{\rceil}
\DeclarePairedDelimiter{\abs}{\left\lvert}{\right\rvert}

\setmainfont{Linux Libertine}

\theoremstyle{plain}
\newtheorem{axiom}{Аксиома}
\newtheorem{lemma}{Лемма}

\theoremstyle{remark}
\newtheorem*{remark}{Примечание}
\newtheorem*{exercise}{Упражнение}
\newtheorem*{consequence}{Следствие}
\newtheorem*{example}{Пример}
\newtheorem*{observation}{Наблюдение}

\theoremstyle{definition}
\newtheorem{theorem}{Теорема}
\newtheorem*{definition}{Определение}
\newtheorem*{obozn}{Обозначение}

\setlength{\parindent}{0pt}

\newcommand{\dbltilde}[1]{\accentset{\approx}{#1}}
\newcommand{\intt}{\int\!}

% magical thing that fixes paragraphs
\makeatletter
\patchcmd{\CatchFBT@Fin@l}{\endlinechar\m@ne}{}
  {}{\typeout{Unsuccessful patch!}}
\makeatother

\newcommand{\get}[2]{
    \ExecuteMetaData[#1]{#2}
}

\newcommand{\getproof}[2]{
    \iftoggle{useproofs}{\ExecuteMetaData[#1]{#2proof}}{}
}

\newcommand{\getwithproof}[2]{
    \get{#1}{#2}
    \getproof{#1}{#2}
}

\newcommand{\import}[3]{
    \subsection{#1}
    \getwithproof{#2}{#3}
}

\newcommand{\given}[1]{
    Дано выше. (\ref{#1}, стр. \pageref{#1})
}

\renewcommand{\ker}{\text{Ker }}
\newcommand{\im}{\text{Im }}
\newcommand{\grad}{\text{grad}}

\lhead{Линейная алгерба}
\cfoot{}
\rfoot{Лекция 4}

\begin{document}

\section{Обратный оператор}

\subsection{Единица. Обратный элемент}

$] A(K)$ --- алгебра над полем $K$

\begin{definition}
    Единицей алгебры называется такой её элемент $e\in A$, что
    $$\forall a\in A \quad a\cdot e = e\cdot a = a$$
\end{definition}

Кроме того, существуют левая и правая единицы:
$$e_L : e_L\cdot a = a \quad e_R : a\cdot e_R = a \quad \forall a\in A$$

\begin{lemma}
    Если в $A \ \ \exists e_L$ и  $e_R \Rightarrow e_L=e_R\stackrel{\triangle}{=} e$
\end{lemma}
\begin{proof}
    $$e_L=e_Le_R=e_R$$
\end{proof}

\begin{lemma}
    $\exists!e$
\end{lemma}
\begin{proof}
    Тривиально.
\end{proof}

$\sphericalangle x,y\in A : x\cdot y = e$
\begin{definition}
    Если $x\cdot y = e$, то $x$ называется \textbf{левым обратным} к $y$, а $y$ называется правым обратным к $x$.
\end{definition}

\begin{definition}
    $] z\in A, x : xz=zx=e$, тогда $x$ --- \textbf{обратный} к $z$, обозначается $x=z^{-1}$, при этом $z$ называется \textbf{обратимым}.
\end{definition}

\begin{lemma}
    Если $y, z\in A$ $\exists x$ --- левый обратимый и $y$ --- правый обратимый. Тогда:
    \begin{enumerate}
        \item $z$ --- обратимый
        \item $z^{-1}=y\cdot x$
    \end{enumerate}
\end{lemma}
\begin{proof}
    $\sphericalangle z\cdot yx=e\cdot x=x$

    $\sphericalangle x(zy)=x \quad (xz)y=y \Rightarrow x=y \Rightarrow z$ --- обратимый.
\end{proof}

\begin{example}
    \begin{itemize}
        \item $\R \quad e = 1 \quad a^{-1} = \frac{1}{a}$
        \item $\C \quad e = 1 + 0\cdot i \quad z^{-1}=\frac{z^*}{|z|^2}$
        \item $\R^4 \quad e = 1 + 0i + 0j + 0k \quad q^{-1}=\frac{q^*}{|q|^2}$
    \end{itemize}
\end{example}

\subsection{Обратная матрица}

$\sphericalangle K_n^n$ --- алгебра матриц
\begin{definition}
    \textbf{Единичной} называется матрица $E$, такая что $\forall A\in K_n^n$:
    $$AE=EA=A$$
\end{definition}
\begin{remark}
    Единичная матрица --- диагональна:
    $$E=\begin{bmatrix}
        1 & 0 & \ldots & 0 \\
        0 & 1 & \ldots & 0 \\
        \vdots & \vdots & \ddots & \vdots \\
        0 & 0 & \ldots & 1 \\
    \end{bmatrix}$$
\end{remark}

\begin{definition}
    \textbf{Обратной} к матрице $A$ называется матрица $A^{-1}$:
    $$A^{-1}A=AA^{-1}=E$$
\end{definition}

\begin{theorem}
    $\exists A^{-1} \Leftrightarrow \det A \not=0$
\end{theorem}

Способы вычисления $A^{-1}$

\subsubsection{Метод Гаусса}

%<*методгаусса>
$$\left[\begin{array}{c|c}
    A & E
\end{array}\right] \sim \left[\begin{array}{c|c}
    E & A^{-1}
\end{array}\right]$$

\begin{proof}
    $$\left[\begin{array}{c|c}
        A & E
    \end{array}\right] = \left[\begin{array}{c|c}
        T_1A & T_1E
    \end{array}\right] = \left[\begin{array}{c|c}
        T_2T_1A & T_2T_1E
    \end{array}\right] = \ldots = \left[\begin{array}{c|c}
        T_n\ldots T_1A & T_n\ldots T_1
    \end{array}\right] = \left[\begin{array}{c|c}
        E & T_n\ldots T_1
    \end{array}\right]$$
    $$\sphericalangle T_n\ldots T_1A=E\Rightarrow A^{-1}=T_n\ldots T_1$$
\end{proof}
%</методгаусса>

%<*присоединеннаяматрица>
\begin{theorem}
    $$A^{-1}=\frac{1}{\det A}\tilde A^T$$
\end{theorem}
\begin{proof}
    $AB=E \Rightarrow B=\frac{1}{\det A}\tilde A^T$ --- надо доказать.

    $$\sum\limits_{j=1}^n \alpha_j^i\beta_k^j = \delta_k^i$$
    $$] \delta^i_{k_0}=\begin{pmatrix}
        0 & 0 & \ldots & 0 & 1_{k_0} & 0 & \ldots & 0
    \end{pmatrix}^T = b$$

    $$\beta^j_{k_0} = \xi^j \quad \alpha_j^i = a_j$$

    $$\sum\limits_{j=1}^n a_j\xi^j = b \quad \xi^j = \frac{\Delta_j}{\Delta}$$
    $$\Delta_j=\det A(a_j\to b)$$
\end{proof}
%</присоединеннаяматрица>

\subsection{Обратный оператор}

$\sphericalangle \varphi : X\to X$

%<*обратныйоператор>
\begin{definition}
    Обратным к оператору $\varphi$ называется оператор $\varphi^{-1}$:
    $$\varphi^{-1}\varphi=\varphi\varphi^{-1}=\mathcal{I}$$
\end{definition}
%</обратныйоператор>

$\mathcal{L}(X,X)\simeq K_n^n$

%<*другойкритерийобратимости>
\begin{theorem}
    Оператор $\varphi$ обратим, если $\exists$ базис, в котором его матрица невырождена
\end{theorem}
%</другойкритерийобратимости>

\begin{remark}
    $\tilde A=T^{-1}AT \quad \det\tilde A = \det(T{-1}AT)=\det T^{-1}\det A\det T$
\end{remark}

%<*ядроиобраз>
$\sphericalangle \varphi : X\to Y$

\begin{definition}
    \textbf{Ядро} $\varphi$ :
    $$\ker \varphi=\{x\in X : \varphi x=0\}$$
\end{definition}
\begin{remark}
    $\ker \varphi\subset X$
\end{remark}
\begin{lemma}
    $\ker \varphi$ --- ЛП
\end{lemma}
\begin{definition}
    \textbf{Образ} $\varphi$:
    $$\im \varphi = \{y\in Y : \exists x : \varphi(x)=y\}$$
\end{definition}
\begin{remark}
    $$\im \varphi \subset Y$$
\end{remark}
\begin{lemma}
    $\im \varphi$ --- ЛП
\end{lemma}

\begin{theorem}
    О ядре и образе

    $$] \varphi : X\to X \Rightarrow \dim \ker \varphi + \dim \im \varphi = \dim X$$
\end{theorem}
\begin{proof}
    $] \dim \ker \varphi = K$

    $] \{e_1\ldots e_k\}$ --- базис $\ker \varphi \Rightarrow \varphi(e_j)=0 \ \ \forall j=1..k$

    $\sphericalangle \{e_1\ldots e_k; e_{k+1}\ldots e_n\}$ --- базис $X$

    $\sphericalangle x=\sum\limits_{j=1}^n \xi^j e_j \quad \sphericalangle \varphi(x) = \sum\limits_{j=1}^n \xi^j\varphi(e_j)=\sum\limits_{j=k+1}^n \xi^j \varphi(e_j)$

    $\{\varphi(e_{k+1})\ldots \varphi(e_{n})\}$ --- полный для $\im$, т.к. любой $x\in \im$ можно по нему разложить. Докажем ЛНЗ от обратного:

    $] \{\varphi(e_j)\}_{j=k+1}^n$ --- ЛЗ $\Rightarrow \exists \alpha^j : \sum\limits_{j=k+1}^n \alpha^j\varphi(e_j)=0\Rightarrow\varphi\left(\sum\limits_{j=k+1}^n \alpha^je_j\right)=0\Rightarrow$
    $$\begin{cases}
        \text{или } \sum\limits_{j=k+1}^n \alpha^je_j \in \ker \varphi \Rightarrow \text{ЛК } e_{k+1}\ldots e_n \text{ разложима по } e_1\ldots e_k \text{ --- противоречие} \\
        \text{или } \sum\limits_{j=k+1}^n \alpha^je_j =0 \Rightarrow \alpha^j=0 \Rightarrow \text{ ЛНЗ}
    \end{cases}$$
    $\Rightarrow \{\varphi(e_j)\}_{j=k+1}^n$ --- базис $\im \varphi$.
\end{proof}
%</ядроиобраз>

О существовании $\varphi^{-1}$

$\sphericalangle \varphi: X\to Y$

$\varphi : \forall x \ \ \exists! y\in \im \varphi : \varphi(x)=y$

$\exists \varphi^{-1} : \forall y\in \im \varphi \ \ \exists! x\in X : \varphi^{-1}y=x$

Обратный оператор существует только к изоморфизмам.

%<*существованиеобратногоопреатора>
\begin{theorem}
    $\sphericalangle \varphi : X\to X$
    
    $\exists \varphi^{-1} \Leftrightarrow \dim \im \varphi=\dim X$ или $\dim \ker \varphi = 0$
\end{theorem}
\begin{proof}
    $\dim \im \varphi=\dim X \Leftrightarrow \im \varphi \simeq X \Rightarrow \varphi $ --- сюръекция, $\dim \ker \varphi = 0 \Rightarrow \forall y \ \ \exists x : \varphi x = y \Rightarrow \varphi$ --- инъекция
\end{proof}
%</существованиеобратногоопреатора>

\section{Внешняя степень ЛОп}

%<*определитель>
$\sphericalangle \Lambda^p \quad \{{}^{i_1\ldots i_p} F\}_{1\leq i_1<i_2<\ldots<i_p\leq n}$ --- базис $\Lambda^p$

$${}^{i_1\ldots i_p} F=f^{i_1}\wedge f^{i_2}\wedge \ldots \wedge f^{i_p} \quad \dim \Lambda^p=C^p_n$$

$] \{x_i\}_{i=1}^n$ --- набор векторов
$$\det \{x_1\ldots x_n\}:={}^{1\ldots n}F(x_1\ldots x_n)$$

$\sphericalangle \Lambda_p \quad \{{}_{i_1\ldots i_p} F\})_{1\leq i_1<i_2<\ldots<i_p\leq n}$ --- базис $\Lambda_p$

$$\dim\Lambda_p=C_n^p \quad {}_{i_1\ldots i_p} F = \hat x_{i_1}\wedge \hat x_{i_2}\wedge\ldots\wedge\hat x_{i_p}\simeq x_1\wedge x_2\wedge\ldots\wedge x_n$$

$] \{e_j\}_{j=1}^n$ --- базис $X \Rightarrow x_i=\xi^{j_i}_i e_{j_i}$

$${}_{1\ldots n} F = \xi_1^{j_1}\xi_2^{j_2}\ldots \xi_n^{j_n}(e_{j_1}\wedge e_{j_2}\wedge\ldots\wedge e_{j_n})=\sum\limits_{(j_1\ldots j_n)} (-1)^{[j_1\ldots j_n]} \xi^1_{j_1}\ldots \xi^n_{j_n} (e_{1}\wedge e_{2}\wedge\ldots\wedge e_{n})=$$
$$=\det [\xi^1_{j_1}\ldots \xi^n_{j_n}](e_{1}\wedge e_{2}\wedge\ldots\wedge e_{n})$$

\begin{definition}
    \textbf{Определителем} набора векторов $\{x_i\}_{i=1}^n$ называется число $\det [x_1\ldots x_n]$, такое, что:
    $$x_1\wedge x_2\wedge\ldots\wedge x_n = \det[x_1\ldots x_n]e_1\wedge e_2\wedge\ldots\wedge e_n$$
\end{definition}
\begin{lemma}
    $$\text{от } \Lambda^p\ \ \det \{x_1\ldots x_n\}=\det[x_1\ldots x_n] \text{ от } \Lambda_p$$
\end{lemma}
\begin{proof}
    $$\det \{x_1\ldots x_n\}={}^{1\ldots n} F(x_1\ldots x_n) = \sum\limits_{(j_1\ldots j_n)} (-1)^{[j_1\ldots j_n]} \xi_1^{j_1}\xi_2^{j_2}\ldots \xi_n^{j_n}e_1\wedge e_2\wedge\ldots\wedge e_n=$$
    $$=\det\{x_1\ldots x_n\}e_1\wedge e_2\wedge\ldots\wedge e_n$$
    $$=\det[x_1\ldots x_n]e_1\wedge e_2\wedge\ldots\wedge e_n$$
\end{proof}
%</определитель>

%<*определительвнешняястепень>
\begin{definition}
    $\sphericalangle \varphi : X \to X$

    \textbf{Внешней степенью} $\varphi^{\Lambda_p}$ оператора $\varphi$ называется отображение:
    $$\varphi^{\Lambda_p}(x_1\wedge x_2\wedge\ldots\wedge x_n) = \varphi(x_1)\wedge\ldots\wedge\varphi(x_p)$$
\end{definition}

\begin{remark}
    $$\varphi^{\Lambda_p} : \Lambda_p\to \Lambda_p$$
\end{remark}

$\sphericalangle p=n$
$$\varphi^{\Lambda_n}(e_1\wedge e_2\wedge\ldots\wedge e_n)=\varphi(e_1)\wedge\varphi(e_2)\wedge\ldots\wedge\varphi(e_n)=a^{j_1}_1e_{j_1}\wedge\ldots\wedge a^{j_n}_1e_{j_n}=$$
$$=a_1^{j_1}\ldots a_n^{j_n}(e_{j_1}\wedge\ldots\wedge e_{j_n})=\sum\limits_{(j_1\ldots j_n)}(-1)^{[j_1\ldots j_n]}a^1_{j_1}a^2_{j_2}\ldots a^n_{j_n} e_1\wedge\ldots\wedge e_n=\det A_\varphi e_1\wedge\ldots\wedge e_n$$

\begin{definition}
    \textbf{Определителем} линейного оператора $\varphi$ называется число, такое что:
    $$\det \varphi = \det [\varphi(e_1)\wedge\ldots\wedge \varphi(e_n)]=\det A_\varphi e_1\wedge\ldots\wedge e_n$$
\end{definition}

\begin{remark}
    $$\forall \omega\in\Lambda_n \quad \varphi^{\Lambda_n}\omega=\det \varphi \cdot \omega$$
    $$\omega\in\Lambda_n\Rightarrow \omega=\alpha e_1\wedge\ldots\wedge e_n$$
    $$\varphi^{\Lambda_n}\omega=\alpha\varphi^{\Lambda_n}(e_1\wedge\ldots\wedge e_n)=\alpha\det \varphi e_1\wedge\ldots\wedge e_n = \det \varphi \cdot \omega$$
\end{remark}
%</определительвнешняястепень>

\begin{definition}
    \textbf{Грассманова алгеба} --- алгебра по внешнему произведению
\end{definition}

%<*умножениеопределителей>
\begin{theorem}
    $$\det (\varphi\psi) = \det \varphi\det \psi$$
\end{theorem}
\begin{proof}
    $$\sphericalangle \det(\varphi\psi)e_1\wedge\ldots\wedge e_n = (\varphi\psi)^{\Lambda_n}e_1\wedge\ldots\wedge e_n=\varphi(\psi(e_1))\wedge\ldots\wedge \varphi(\psi(e_n))=$$
    $$=\varphi^{\Lambda_n}(\psi(e_1)\wedge\ldots\wedge \psi(e_n))=\det \varphi\det \psi e_1\wedge\ldots\wedge e_n$$
\end{proof}
%</умножениеопределителей>



\end{document}