\documentclass[12pt, a4paper]{article}

\usepackage{lastpage}
\usepackage{mathtools}
\usepackage{xltxtra}
\usepackage{libertine}
\usepackage{amsmath}
\usepackage{amsthm}
\usepackage{amsfonts}
\usepackage{amssymb}
\usepackage{enumitem}
\usepackage{xcolor}
\usepackage[left=1.5cm, right=1.5cm, top=2cm, bottom=2cm, bindingoffset=0cm, headheight=15pt]{geometry}
\usepackage{fancyhdr}
\usepackage[russian]{babel}
% \usepackage[utf8]{inputenc}
\usepackage{catchfilebetweentags}
\usepackage{accents}
\usepackage{calc}
\usepackage{etoolbox}
\usepackage{mathrsfs}
\usepackage{wrapfig}

\providetoggle{useproofs}
\settoggle{useproofs}{false}

\pagestyle{fancy}
\lfoot{M3137y2019}
\rhead{\thepage\ из \pageref{LastPage}}

\newcommand{\R}{\mathbb{R}}
\newcommand{\Q}{\mathbb{Q}}
\newcommand{\C}{\mathbb{C}}
\newcommand{\Z}{\mathbb{Z}}
\newcommand{\B}{\mathbb{B}}
\newcommand{\N}{\mathbb{N}}

\newcommand{\const}{\text{const}}

\newcommand{\teormin}{\textcolor{red}{!}\ }

\DeclareMathOperator*{\xor}{\oplus}
\DeclareMathOperator*{\equ}{\sim}
\DeclareMathOperator{\Ln}{\text{Ln}}
\DeclareMathOperator{\sign}{\text{sign}}
\DeclareMathOperator{\Sym}{\text{Sym}}
\DeclareMathOperator{\Asym}{\text{Asym}}
% \DeclareMathOperator{\sh}{\text{sh}}
% \DeclareMathOperator{\tg}{\text{tg}}
% \DeclareMathOperator{\arctg}{\text{arctg}}
% \DeclareMathOperator{\ch}{\text{ch}}

\DeclarePairedDelimiter{\ceil}{\lceil}{\rceil}
\DeclarePairedDelimiter{\abs}{\left\lvert}{\right\rvert}

\setmainfont{Linux Libertine}

\theoremstyle{plain}
\newtheorem{axiom}{Аксиома}
\newtheorem{lemma}{Лемма}

\theoremstyle{remark}
\newtheorem*{remark}{Примечание}
\newtheorem*{exercise}{Упражнение}
\newtheorem*{consequence}{Следствие}
\newtheorem*{example}{Пример}
\newtheorem*{observation}{Наблюдение}

\theoremstyle{definition}
\newtheorem{theorem}{Теорема}
\newtheorem*{definition}{Определение}
\newtheorem*{obozn}{Обозначение}

\setlength{\parindent}{0pt}

\newcommand{\dbltilde}[1]{\accentset{\approx}{#1}}
\newcommand{\intt}{\int\!}

% magical thing that fixes paragraphs
\makeatletter
\patchcmd{\CatchFBT@Fin@l}{\endlinechar\m@ne}{}
  {}{\typeout{Unsuccessful patch!}}
\makeatother

\newcommand{\get}[2]{
    \ExecuteMetaData[#1]{#2}
}

\newcommand{\getproof}[2]{
    \iftoggle{useproofs}{\ExecuteMetaData[#1]{#2proof}}{}
}

\newcommand{\getwithproof}[2]{
    \get{#1}{#2}
    \getproof{#1}{#2}
}

\newcommand{\import}[3]{
    \subsection{#1}
    \getwithproof{#2}{#3}
}

\newcommand{\given}[1]{
    Дано выше. (\ref{#1}, стр. \pageref{#1})
}

\renewcommand{\ker}{\text{Ker }}
\newcommand{\im}{\text{Im }}
\newcommand{\grad}{\text{grad}}

\usepackage{bm}
\usepackage{xcolor}
% \usepackage[raggedright]{titlesec}
\usepackage{sectsty}
\usepackage{catchfilebetweentags}

\allsectionsfont{\raggedright}
\subsectionfont{\fontsize{14}{15}\selectfont}

% magical thing that fixes paragraphs
\makeatletter
\def\CatchFBT@sanitize{%
   \@sanitize
   \@makeother\{%
   \@makeother\}%
}
\makeatother

\newcommand{\import}[3]{
    \subsection{#1}
    \ExecuteMetaData[#2]{#3}
}

\lhead{Конспект к второму опросу}
\cfoot{}
\rfoot{}

\setlength\parindent{0pt}

\begin{document}
\section{Определения и формулировки}

\import{Внутренняя точка множества, открытое множество, внутренность}{opros.tex}{внутренность}

\import{Предельная точка множества}{4.tex}{предельнаяточка}

\import{Замкнутое множество, замыкание, граница}{4.tex}{замкнутоемножество}

\ExecuteMetaData[5.tex]{замыкание}

\ExecuteMetaData[5.tex]{граница}

\import{Изолированная точка, граничная точка}{4.tex}{изолированнаяточка}

\ExecuteMetaData[5.tex]{граничнаяточка}

\import{Описание внутренности множества}{4.tex}{описаниевнутренности}

\import{Описание замыкания множества в терминах пересечений}{5.tex}{замыканиечерезпересечения}

\import{Верхняя, нижняя границы; супремум, инфимум}{5.tex}{верхняяграница}

\ExecuteMetaData[5.tex]{нижняяграница}

\ExecuteMetaData[5.tex]{супремуминфимум}

\import{Техническое описание супремума}{5.tex}{техническоеописаниесупремума}

\import{Последовательность, стремящаяся к бесконечности}{5.tex}{последовательностьстремящаясякбесконечности}

\import{Компактное множество}{6.tex}{компактноемножество}

\import{Секвенциальная компактность}{7.tex}{секвенциальнокомпактноемножество}

\import{Определения предела отображения (3 шт)}{6.tex}{пределотображения}

\import{Определения пределов в $\overline\R$}{opros.tex}{пределывrсчертой}

\import{Предел по множеству}{8.tex}{пределпомножеству}

\import{Односторонние пределы}{8.tex}{односторонниепределы}

\import{Непрерывное отображение}{8.tex}{непрерывноеотображение}

\import{Непрерывность слева}{8.tex}{непрерывностьслева}

\import{Разрыв, разрывы первого и второго рода}{8.tex}{точкаразрыва}

\ExecuteMetaData[8.tex]{родточкиразрыва}

\import{О большое, о маленькое}{10.tex}{осимволика}

\begin{remark}
    О большое и о малое --- разные вопросы в табличке.
\end{remark}

\import{Эквивалентные функции, таблица эквивалентных}{10.tex}{}

Эквивалентные функции даны выше.

\ExecuteMetaData[10.tex]{таблицаэквивалентных}

\import{Асимптотически равные \textit{(сравнимые)} функции}{10.tex}{асимптотическиравныефункции}

\import{Асимптотическое разложение}{10.tex}{асимтотическоеразложение}

\import{Наклонная асимптота графика}{10.tex}{наклоннаяасимптота}

\import{Путь в метрическом пространстве}{11.tex}{путь}

\import{Линейно связное множество}{11.tex}{линейносвязноемножество}

\import{Функция, дифференцируемая в точке и производная}{12.tex}{дифференциируемая}

\begin{remark}
    Это два разных билета.
\end{remark}

\section{Теоремы}

\import{Теорема об открытых и замкнутых множествах в пространстве и в подпространстве}{6.tex}{открытыеизамкнутыемножествавпространствеиподпространстве}

\import{Теорема о компактности в пространстве и в подпространстве}{6.tex}{окомпактностивпространствеиподпространстве}

\import{Простейшие свойства компактных множеств}{7.tex}{опростейшихсвойствахкомпактныхмножеств}

\import{Лемма о вложенных параллелепипедах}{7.tex}{овложенныхпараллелепипедах}

\import{Компактность замкнутого параллелепипеда в $\R^m$}{7.tex}{компактностьпарллелепипеда}

\import{Теорема о характеристике компактов в $\R^m$}{7.tex}{охарактеристикекомпактоввrm}

\import{Эквивалентность определений Гейне и Коши}{7.tex}{эквивалентностьопределенийгейнеикоши}

\import{Единственность предела, локальная ограниченность отображения, имеющего предел, теорема о стабилизации знака}{7.tex}{оединственностипредела}

\ExecuteMetaData[7.tex]{олокальнойограниченностиотображенияимеющегопредел}

\ExecuteMetaData[7.tex]{остабилизациизнака}

\import{Арифметические свойства пределов отображений. Формулировка для $\overline\R$}{7.tex}{обарифметическихсвойствахпредела}

\ExecuteMetaData[7.tex]{арифметическиесвойствапределадляoverliner}

\import{Принцип выбора Больцано--Вейерштрасса}{8.tex}{принципвыборабольцановейерштрасса}

\import{Сходимость в себе и ее свойства}{8.tex}{сходящаясявсебе}

\ExecuteMetaData[9.tex]{свойствасходимостивсебе}

\import{Критерий Коши для последовательностей и отображений}{8.tex}{критерийбольцанокошидляотображений}

\subsubsection{Для последовательностей}

\ExecuteMetaData[8.tex]{критерийкошидляпоследовательностей}

\import{Теорема о пределе монотонной функции}{8.tex}{определемонотоннойфункции}

\import{Свойства непрерывных отображений: арифметические, стабилизация знака, композиция}{8.tex}{арифметическиесвойстванепрерывныхотображений}

\subsubsection{Стабилизация знака}

\ExecuteMetaData[9.tex]{стабилизациязнака}

\subsubsection{Непрерывность композиции непрерывных отображений}

\ExecuteMetaData[10.tex]{онепрерывностикомпозиции}

\import{Непрерывность композиции и соответствующая теорема для пределов}{10.tex}{}

Непрервыность композиции дана выше.

\subsubsection{Теорема о пределе композиции непрерывных отображений}

\ExecuteMetaData[10.tex]{определекомпозиции}

\import{Теорема о замене на эквивалентную при вычислении пределов. Таблица эквивалентных}{10.tex}{озамененаэквивалентную}

Таблица эквивалентных дана выше. 

\import{Теорема единственности асимптотического разложения}{10.tex}{оединственностиасимтотическогоразложения}

\import{Теорема о топологическом определении непрерывности}{10.tex}{топологическоеопределениенепрерывности}

\import{Теорема Вейерштрасса о непрерывном образе компакта. Следствия}{10.tex}{вейерштрассаонепрерывномобразекомпакта}

\ExecuteMetaData[10.tex]{следствиявейерштрасса}

\import{Лемма о связности отрезка}{11.tex}{освязностиотрезка}

\import{Теорема Больцано-Коши о промежуточном значении}{11.tex}{больцанокошиопромежуточномзначении}

\import{Теорема о сохранении промежутка}{11.tex}{осохранениипромежутка}

\subsection{Теорема Больцано-Коши о сохранении линейной связности}

$X, Y$ --- метрические пространства, $f:X\to Y$ --- непрерывное и сюръекция

$X$ --- линейно связное множество. Тогда $Y$ --- линейно связное множество.

\import{Описание линейно связных множеств в $\R$}{11.tex}{линейносвзяноевr}

\import{Теорема о бутерброде}{11.tex}{теоремаобутерброде}

\subsection{Теорема о вписанном $n$-угольнике максимальной площади}

\ExecuteMetaData[11.tex]{овписанномnугольникемаксимальнойплощади}

\import{Теорема о непрерывности монотонной функции. Следствие о множестве точек разрыва}{11.tex}{онепрерывностимонотонныхфункций}

\ExecuteMetaData[12.tex]{омножестветочекразрыва}

\import{Теорема о существовании и непрерывности обратной функции}{12.tex}{осуществованииинепрерывностиобратнойфункции}

\end{document}