\documentclass[12pt, a4paper]{article}

\usepackage{lastpage}
\usepackage{mathtools}
\usepackage{xltxtra}
\usepackage{libertine}
\usepackage{amsmath}
\usepackage{amsthm}
\usepackage{amsfonts}
\usepackage{amssymb}
\usepackage{enumitem}
\usepackage{xcolor}
\usepackage[left=1.5cm, right=1.5cm, top=2cm, bottom=2cm, bindingoffset=0cm, headheight=15pt]{geometry}
\usepackage{fancyhdr}
\usepackage[russian]{babel}
% \usepackage[utf8]{inputenc}
\usepackage{catchfilebetweentags}
\usepackage{accents}
\usepackage{calc}
\usepackage{etoolbox}
\usepackage{mathrsfs}
\usepackage{wrapfig}

\providetoggle{useproofs}
\settoggle{useproofs}{false}

\pagestyle{fancy}
\lfoot{M3137y2019}
\rhead{\thepage\ из \pageref{LastPage}}

\newcommand{\R}{\mathbb{R}}
\newcommand{\Q}{\mathbb{Q}}
\newcommand{\C}{\mathbb{C}}
\newcommand{\Z}{\mathbb{Z}}
\newcommand{\B}{\mathbb{B}}
\newcommand{\N}{\mathbb{N}}

\newcommand{\const}{\text{const}}

\newcommand{\teormin}{\textcolor{red}{!}\ }

\DeclareMathOperator*{\xor}{\oplus}
\DeclareMathOperator*{\equ}{\sim}
\DeclareMathOperator{\Ln}{\text{Ln}}
\DeclareMathOperator{\sign}{\text{sign}}
\DeclareMathOperator{\Sym}{\text{Sym}}
\DeclareMathOperator{\Asym}{\text{Asym}}
% \DeclareMathOperator{\sh}{\text{sh}}
% \DeclareMathOperator{\tg}{\text{tg}}
% \DeclareMathOperator{\arctg}{\text{arctg}}
% \DeclareMathOperator{\ch}{\text{ch}}

\DeclarePairedDelimiter{\ceil}{\lceil}{\rceil}
\DeclarePairedDelimiter{\abs}{\left\lvert}{\right\rvert}

\setmainfont{Linux Libertine}

\theoremstyle{plain}
\newtheorem{axiom}{Аксиома}
\newtheorem{lemma}{Лемма}

\theoremstyle{remark}
\newtheorem*{remark}{Примечание}
\newtheorem*{exercise}{Упражнение}
\newtheorem*{consequence}{Следствие}
\newtheorem*{example}{Пример}
\newtheorem*{observation}{Наблюдение}

\theoremstyle{definition}
\newtheorem{theorem}{Теорема}
\newtheorem*{definition}{Определение}
\newtheorem*{obozn}{Обозначение}

\setlength{\parindent}{0pt}

\newcommand{\dbltilde}[1]{\accentset{\approx}{#1}}
\newcommand{\intt}{\int\!}

% magical thing that fixes paragraphs
\makeatletter
\patchcmd{\CatchFBT@Fin@l}{\endlinechar\m@ne}{}
  {}{\typeout{Unsuccessful patch!}}
\makeatother

\newcommand{\get}[2]{
    \ExecuteMetaData[#1]{#2}
}

\newcommand{\getproof}[2]{
    \iftoggle{useproofs}{\ExecuteMetaData[#1]{#2proof}}{}
}

\newcommand{\getwithproof}[2]{
    \get{#1}{#2}
    \getproof{#1}{#2}
}

\newcommand{\import}[3]{
    \subsection{#1}
    \getwithproof{#2}{#3}
}

\newcommand{\given}[1]{
    Дано выше. (\ref{#1}, стр. \pageref{#1})
}

\renewcommand{\ker}{\text{Ker }}
\newcommand{\im}{\text{Im }}
\newcommand{\grad}{\text{grad}}

\lhead{Конспект по математическому анализу}
\rfoot{October 7, 2019}

\begin{document}

\section{Скалярное произведение}

\begin{definition}
    Для $X$ --- линейного пространства \textit{(над $\mathbb{R}, \mathbb{C}$)} $\varphi:X\times X \to \mathbb{R} (\mathbb{C})$ называется \textbf{скалярным произведением}. Обозначается $\varphi(x,y)=\langle x,y\rangle$
\end{definition}

\begin{enumerate}
    \item $\langle \alpha_1 x_1+\alpha_2 x_2 \rangle = \alpha_1\langle x_1,y\rangle+\alpha_2\langle x_2,y\rangle$\
    \item $\langle y,x\rangle = \overline{\langle x,y\rangle}$
    \item $\langle x,x\rangle \geq 0 \quad \langle x,x\rangle=0 \Leftrightarrow x=0$
\end{enumerate}

\begin{definition}
    $\overline{x}$ --- \textbf{комплексное сопряжение}, для вещественных чисел $\overline{x} = x$.    
\end{definition}

\begin{enumerate}
    \item Над $\mathbb{C}$: $\langle x,\beta_1 y_1 + \beta_2 y_2 \rangle = \overline{\langle x,\beta_1 y_1 + \beta_2 y_2 \rangle} = \overline{\beta_1\langle y_1, x\rangle + \beta_2 \langle y_2,x \rangle} = \overline{\beta_1} \langle x, y_1\rangle + \overline{\beta_2} \langle x,y_2\rangle$
    \item Над $\mathbb{R}$: $\langle y,x\rangle = \langle x,y\rangle$
    \item $\langle 0,x \rangle = \langle x,0 \rangle = \langle 0\cdot a,x \rangle = 0\langle a,x \rangle=0$
\end{enumerate}

\begin{lemma}
    Неравенство КБШ \textit{(Коши-Буняковского-Шварца)}
    
    %<*неравенствокошибуняковского>
    Для $X$ --- линейного пространства \textit{(над $\mathbb{R}, \mathbb{C}$)}

    $\forall x,y\in X\quad |\langle x,y\rangle|^2\leq \langle x,x\rangle \langle y,y\rangle$
    %</неравенствокошибуняковского>
\end{lemma}

\begin{proof}
    %<*неравенствокошибуняковскогоproof>

    Возьмём $\lambda\in\mathbb{R} (\mathbb{C})$

    При $y=0$ тривиально, пусть $y\not = 0$

    $$0\leq \langle x + \lambda y,x + \lambda y\rangle = \langle x,x\rangle + \lambda\langle y,x\rangle + \overline\lambda\langle x,y\rangle+\lambda\overline\lambda\langle y,y\rangle$$

    $$\lambda := -\frac{\langle x,y\rangle}{\langle y,y\rangle}, \overline\lambda = -\frac{\langle y,x\rangle}{\langle y,y\rangle}$$

    $$0\leq\langle x,x\rangle - \frac{\langle x,y\rangle}{\langle y,y\rangle}\langle y,x\rangle - \frac{\langle y,x\rangle}{\langle y,y\rangle}\langle x,y\rangle+\frac{\langle x,y\rangle\langle y,x\rangle}{\langle y,y\rangle}$$

    $$0\leq \langle x,x\rangle - \frac{\langle x,y\rangle}{\langle y,y\rangle}\langle y,x\rangle$$

    $$\frac{\langle x,y\rangle}{\langle y,y\rangle}\langle y,x\rangle\leq \langle x,x\rangle$$

    $$\langle x,y\rangle\langle y,x\rangle \leq \langle x,x\rangle \langle y,y\rangle$$

    $$|\langle x,y\rangle|^2 \leq \langle x,x\rangle \langle y,y\rangle$$

    %</неравенствокошибуняковскогоproof>

    На википедии есть доказательство проще.
\end{proof}

Пример в $\mathbb{R}^m$: $\langle x,y\rangle = x_1y_1+x_2y_2+\ldots+x_my_m$ --- Евклидово скалярное произведение

Пример в $\mathbb{C}^m$: $\langle x,y\rangle = x_1\overline y_1+x_2\overline y_2+\ldots+x_m\overline y_m$

\begin{lemma}
    Для лин. пространства $X$, скалярного произведения $\langle \cdot ,\cdot \rangle$
    
    $\rho:X\to\mathbb{R} \quad \rho(x)=\sqrt{\langle x,x\rangle}$ --- норма
\end{lemma}

\begin{proof}
    Докажем, что $\rho$ удовлетворяет всем леммам нормы.
    \begin{enumerate}
        \item $\rho(x)\geq 0\quad \rho(x)=0\Leftrightarrow \langle x,x\rangle=0\Leftrightarrow x=0$
        \item $\rho(\alpha x) = \sqrt{\alpha\overline\alpha\langle x,x\rangle}=|\alpha|\sqrt{\langle x,x\rangle}=|\alpha|\rho(x)$
        \item $\rho(x+y)\leq \rho(x)+\rho(y)$
        $$\langle x+y,x+y\rangle \leq (\sqrt{\langle x,x\rangle} + \langle y,y\rangle)^2$$

        $$\langle x,x\rangle+\langle x,y\rangle+\langle y,x\rangle+\langle y,y\rangle\leq \langle x,x\rangle + \langle y,y\rangle + 2\sqrt{\langle x,x\rangle\langle y,y\rangle}$$

        $$2\Re\langle x,y\rangle\leq 2\sqrt{\langle x,x\rangle\langle y,y\rangle}$$

        $$\Re\langle x,y\rangle \leq |\langle x,y\rangle|\leq\sqrt{\langle x,x\rangle\langle y,y\rangle}$$
    \end{enumerate}
\end{proof}

$||x|| = \sqrt{\sum\limits_{i=1}^m x_i^2}$ - норма в $\mathbb{R}^m$

$\rho(x,y)=||x-y||$ - метрика в $\mathbb{R}^m$

Не все нормы порождены скалярным произведением, например: $||x||=\max\limits_i |x_i|$

\begin{lemma}
    О непрерывности скалярного произведения.
    %<*онепрерывностискалярногопроизведения>
    $X$ - лин. пространство со скалярным произведением, $||\cdot||$ --- норма, порожденная скалярным произведением.

    Тогда $\forall (x_n) x_n\to x, \forall (y_n) y_n\to y, \quad \langle x_n,y_m\rangle\to\langle x,y\rangle$
    %</онепрерывностискалярногопроизведения>
\end{lemma}

%<*онепрерывностискалярногопроизведенияproof>
\begin{proof}
    $$|\langle x_n,y_m\rangle - \langle x,y\rangle|=|\langle x_n,y_n\rangle - \langle x_n,y\rangle + \langle x_n,y\rangle - \langle x,y\rangle|\leq|\langle x_n,y_n\rangle-\langle x_n,y\rangle| + |\langle x_n,y\rangle-\langle x,y\rangle\leq$$
    $$\leq |\langle x_n,y_n-y\rangle| + |\langle x_n-x,y\rangle|\leq ||x_n||\cdot||y_n-y||+||x_n-x||\cdot||y|| \to 0$$

    По теореме о двух городовых чтд.
\end{proof}
%</онепрерывностискалярногопроизведенияproof>

\begin{lemma}
    %<*опокоординатнойсходимости>
    О покоординатной сходимости в $\mathbb{R}^m$

    $(x^{(n)})$ --- последовательность векторов в $\mathbb{R}^m$
    
    в $\mathbb{R}^m$ задано евклидово скалярное пространство и норма.

    Тогда $(x^{(n)})\to x \Leftrightarrow \forall i\in\{1,2,\ldots m\} \ \ x_i^{(n)}\underset{n\to+\infty}\to x_i$
    %</опокоординатнойсходимости>
\end{lemma}

\begin{remark}
    В $\mathbb{R}^{\infty}$ не выполняется
\end{remark}

%<*опокоординатнойсходимостиproof>
\begin{proof}
    Модуль координаты $\leq$ нормы всего вектора:
    $$|x_i^{(n)}-x_i|\leq ||x^{(n)}-x||\leq \sqrt{m}\max\limits_{1\leq i\leq m} |x_i^{n}-x_i|$$
    Первое неравенство доказывает $\Rightarrow$, второе неравенство доказывает $\Leftarrow$
\end{proof}
%</опокоординатнойсходимостиproof>

\begin{definition}
    \textbf{Параллелепипед} в $\mathbb{R}^m$

    $a,b\in\mathbb{R}^m \quad [a,b]=\{x \in\mathbb{R}^m : \forall i\in\{1\ldots m\} \ \ a_i\leq x_i\leq b_i \} = [a_1b_1]\times[a_2b_2]\times\ldots\times[a_mb_m]$
\end{definition}

\begin{definition}
    \textbf{Куб} в $\mathbb{R}^m$

    $[(a_1-R,a_2-R,\ldots a_m-R), (a_1+R,a_2+R,\ldots a_m+R)]$
\end{definition}

$$\overline{B(a,R)}\subset\text{Куб}(a,R)\subset \overline{B(a,\sqrt mR)}$$

\begin{proof}
    Докажем 1: $\overline{B(a,R)}\subset\text{Куб}(a,R)$

    $$x\in \overline{B(a,R)}$$

    $$\forall i \quad |x_i-a_i|\leq ||x-a||\leq R \Rightarrow x\in\text{Куб}(a,R)$$

    Докажем 2: $\text{Куб}(a,R)\subset \overline{B(a,\sqrt mR)}$

    $$x\in\text{Куб}(a,R)\quad ||x-a||\leq\sqrt m \max\limits_{1\leq i\leq m} |x_i-a_i|\leq \sqrt mR$$
\end{proof}

\section{Точки и множества в метрическом пространстве}

В этом параграфе $(X,\rho)$ - метрическое пространство, $a\in X, D\subset X$.

\begin{definition}
    $a$ --- \textbf{внутренняя точка} множества $D$, если $\exists U(a) : U(a)\subset D$

    $\exists r>0 : B(a,r)\subset D$
\end{definition}

\begin{definition}
    $D$ - \textbf{открытое множество} $\forall a\in D : a$ --- внутренняя точка $D$. 
\end{definition}

Пример:
\begin{enumerate}
    \itemsep0em
    \item $X$ - откр.
    \item \O - откр.
    \item $B(a,r)$ - откр.
\end{enumerate}

\begin{proof}
    Докажем 3.

    $x\in B(a,r)$, доказать: $x$ - внутр. точка

    Возьмём $R<r-\rho(a,x)$. Докажем, что $B(x,R)\subset B(a,r)$

    $y\in B(x,R)$. Докажем, что $y\in B(a,r)$

    $\rho(y,a)\leq\rho(y,x)+\rho(x,a)<R+\rho(x,a)<r$
\end{proof}

\begin{theorem}
    О свойствах открытых множеств.

    \begin{enumerate}
        \item $(G_\alpha)_{\alpha\in A}$ - семейство открытых множеств в $(X,\rho)$
        
        Тогда $\bigcup\limits_{\alpha\in A} G_\alpha$ - открыто в $X$.

        \item $G_1,G_2,\ldots G_n$ - открыто в $X$.
        
        Тогда $\bigcap\limits_{i=1}^{n} G_i$ - открыто в $X$.
    \end{enumerate}

\end{theorem}

\begin{proof}
    \begin{enumerate}
        \item Пусть $x\in \bigcup_{\alpha\in A} G_\alpha$
        
        Тогда $\exists \alpha_0 \quad x\in G_{\alpha_0}$ --- откр. $\exists r_0 : B(x,r_0)\subset G_{\alpha_0} \Rightarrow B(x,r_0)\subset\bigcup_{\alpha\in A} G_\alpha$

        \item $x\in \prod_{i=1}^n G_i \Rightarrow \forall i\in\{1\ldots n\} \quad x\in G_i \Rightarrow \exists r_i>0 : B(x,r_i)\subset G_i$
        
        $r:=min(r_1\ldots r_n)$

        $\forall i \quad B(x,r)\subset G_i$, т.е. $B(x,r)\subset\bigcap G_i$
    \end{enumerate}
\end{proof}

\begin{remark}
    Для $n=\infty$ не выполняется: $(-\frac{1}{n},\frac{1}{n})$ - откр. в $\mathbb{R}$

    $\bigcup_{n=1}^{+\infty}(-\frac{1}{n},\frac{1}{n})=\{0\}$ не откр. в $\mathbb{R}$
\end{remark}

\begin{definition}
    \textbf{Внутренность} $D$ $Int(D)=\{x\in D : x \text{ --- внутр. точка }D\}$
\end{definition}

\begin{remark}
    %<*описаниевнутренности>
    \begin{enumerate}
        \item $Int D$ - откр. множество
        \item $Int D = \bigcup\limits_{\substack{D\supset G \\ G\text{ --- открыт}}}$ --- максимальное открытое множество, содержащееся в $D$
        \item $D$ --- откр. в $X \Leftrightarrow D=Int D$
    \end{enumerate}
    %</описаниевнутренности>
\end{remark}

\begin{definition}
    %<*предельнаяточка>
    $a$ --- {\bf предельная точка} множества $D$, если $$\forall \dot U(a) \ \ \dot U(a)\cap D\not = \text{\O}$$
    %</предельнаяточка>
\end{definition}

Пример: $D=(0,1), X=\mathbb{R}$

\begin{tabular}{c|c}
    $a$&Пред. точка? \\
    \hline
    -1& Нет, $B(-1, \frac{1}{2})\cap D=\text{\O}$\\
    \hline
    $\frac{1}{2}$ & Да, $B(\frac{1}{2}, \frac{1}{2})\subset D$\\
    \hline
    0 & Да, $B(0,\frac{1}{2})\cap D = (0, \frac{1}{2})$
\end{tabular}

\begin{remark}
    $a$ - пред. точка $D$
    \begin{enumerate}
        \item $\forall U(a) \quad U(a)\cap D$ - бесконечное
        \item $\exists (x_n)$ --- последовательность точек $D$, $x_n\underset{n\to +\infty}\to a$
    \end{enumerate}
\end{remark}

\begin{definition}
    %<*изолированнаяточка>
    $a$ --- \textbf{изолированная точка} $D$, если $a\in D$ и $a$ --- не предельная, то есть:
    $$\exists U(a) \quad U(a)\cap D = \{a\}$$
    %</изолированнаяточка>
\end{definition}

Пример --- $\mathbb{N}$

\begin{definition}
    %<*замкнутоемножество>
    $D$ --- \textbf{замкнутое множество}, если оно содержит все свои предельные точки.
    %</замкнутоемножество>
\end{definition}

Пример: $X, \text{\O}, [0,1], \overline{B(a,R)}, \{a\}$ --- замкнутые

Пример: $(0, 1)$ --- в $\mathbb{R}$ незамкнутое

\begin{theorem}
    $D$ --- замкнуто $\Leftrightarrow D^c=X\setminus D$ \textit{(дополнение)} --- открыто.
\end{theorem}

%<*освязиоткрытыхизамкнутыхмножествproof>
\begin{proof}
    Докажем $\Rightarrow$: $D$ --- замкн. $\Rightarrow^? X\setminus D$

    $x\in X\setminus D \Rightarrow x$ --- не пред. точка $D$, т.к. $D$ содержит все свои пред. точки и $x\not\in D$ 
    
    $\Rightarrow\exists r : B(x,r)\subset X\setminus D$

    Докажем $\Leftarrow$: $X\setminus D$ --- откр., $D$ --- замкн.?, т.е. $\forall x\in\{\text{пр.точки D}\} \ \ ?x\in D$

    Если $x\in D$ --- тривиально.

    $x\not\in D\quad x\in X\setminus D$

    $\exists U(x)\subset X\setminus D \Rightarrow$ $x$ - не пред. точка
\end{proof}
%</освязиоткрытыхизамкнутыхмножествproof>

\begin{remark}
    Если $D$ --- не замкнуто, то это НЕ значит, что $D$ --- открыто, например $(0,1]$ --- не замкнуто и не открыто.
\end{remark}

\begin{theorem}
    О свойствах замкнутых множеств.

    %<*освойствахзамкнутыхмножеств>
    \begin{enumerate}
        \item $(G_\alpha)_{\alpha\in A}$ - семейство открытых множеств в $(X,\rho)$
    
        Тогда $\bigcup\limits_{\alpha\in A} G_\alpha$ - открыто в $X$.
    
        \item $G_1,G_2,\ldots G_n$ - открыто в $X$.
        
        Тогда $\bigcap\limits_{i=1}^{n} G_i$ - открыто в $X$.
    \end{enumerate}
    %</освойствахзамкнутыхмножеств>
\end{theorem}

%<*освойствахзамкнутыхмножествproof>
\begin{proof}
    \begin{enumerate}
        \item $x_0\in\bigcup\limits_{\alpha\in A} G_\alpha$
        
        $\exists \alpha_0:x_0\in G_{\alpha_0}$

        $G_{\alpha_0}$ --- открыто $\Rightarrow \exists U(x_0)\subset G_{\alpha_0} \subset \bigcup\limits_{\alpha\in A} G_\alpha \Rightarrow x_0$ --- внтуренняя точка $\bigcup\limits_{\alpha\in A} G_\alpha \Rightarrow \bigcup\limits_{\alpha\in A} G_\alpha$ --- открыто, т.к. в нём все точки внутренние.

        \item $x_0\in\bigcap\limits_{\alpha\in A} G_\alpha$
        
        $\forall \alpha\in A : x_0\in G_\alpha$

        $\forall \alpha\in A \ \ G_\alpha$ --- открыто $\Rightarrow \exists B_\alpha(x_0, r_\alpha)\subset G_\alpha$

        $\forall x_0 : \exists U(x_0)=B(x_0, \min\limits_\alpha r_\alpha)\subset \bigcap\limits_{\alpha\in A} G_\alpha \Rightarrow x_0$ --- внутренняя точка $\bigcap\limits_{\alpha\in A} G_\alpha \Rightarrow \bigcap\limits_{\alpha\in A} G_\alpha$ --- открыто, т.к. в нём все точки внутренние.
    \end{enumerate}
\end{proof}
%</освойствахзамкнутыхмножествproof>

\end{document}