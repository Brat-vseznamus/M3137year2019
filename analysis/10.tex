\documentclass[12pt, a4paper]{article}

\usepackage{lastpage}
\usepackage{mathtools}
\usepackage{xltxtra}
\usepackage{libertine}
\usepackage{amsmath}
\usepackage{amsthm}
\usepackage{amsfonts}
\usepackage{amssymb}
\usepackage{enumitem}
\usepackage{xcolor}
\usepackage[left=1.5cm, right=1.5cm, top=2cm, bottom=2cm, bindingoffset=0cm, headheight=15pt]{geometry}
\usepackage{fancyhdr}
\usepackage[russian]{babel}
% \usepackage[utf8]{inputenc}
\usepackage{catchfilebetweentags}
\usepackage{accents}
\usepackage{calc}
\usepackage{etoolbox}
\usepackage{mathrsfs}
\usepackage{wrapfig}

\providetoggle{useproofs}
\settoggle{useproofs}{false}

\pagestyle{fancy}
\lfoot{M3137y2019}
\rhead{\thepage\ из \pageref{LastPage}}

\newcommand{\R}{\mathbb{R}}
\newcommand{\Q}{\mathbb{Q}}
\newcommand{\C}{\mathbb{C}}
\newcommand{\Z}{\mathbb{Z}}
\newcommand{\B}{\mathbb{B}}
\newcommand{\N}{\mathbb{N}}

\newcommand{\const}{\text{const}}

\newcommand{\teormin}{\textcolor{red}{!}\ }

\DeclareMathOperator*{\xor}{\oplus}
\DeclareMathOperator*{\equ}{\sim}
\DeclareMathOperator{\Ln}{\text{Ln}}
\DeclareMathOperator{\sign}{\text{sign}}
\DeclareMathOperator{\Sym}{\text{Sym}}
\DeclareMathOperator{\Asym}{\text{Asym}}
% \DeclareMathOperator{\sh}{\text{sh}}
% \DeclareMathOperator{\tg}{\text{tg}}
% \DeclareMathOperator{\arctg}{\text{arctg}}
% \DeclareMathOperator{\ch}{\text{ch}}

\DeclarePairedDelimiter{\ceil}{\lceil}{\rceil}
\DeclarePairedDelimiter{\abs}{\left\lvert}{\right\rvert}

\setmainfont{Linux Libertine}

\theoremstyle{plain}
\newtheorem{axiom}{Аксиома}
\newtheorem{lemma}{Лемма}

\theoremstyle{remark}
\newtheorem*{remark}{Примечание}
\newtheorem*{exercise}{Упражнение}
\newtheorem*{consequence}{Следствие}
\newtheorem*{example}{Пример}
\newtheorem*{observation}{Наблюдение}

\theoremstyle{definition}
\newtheorem{theorem}{Теорема}
\newtheorem*{definition}{Определение}
\newtheorem*{obozn}{Обозначение}

\setlength{\parindent}{0pt}

\newcommand{\dbltilde}[1]{\accentset{\approx}{#1}}
\newcommand{\intt}{\int\!}

% magical thing that fixes paragraphs
\makeatletter
\patchcmd{\CatchFBT@Fin@l}{\endlinechar\m@ne}{}
  {}{\typeout{Unsuccessful patch!}}
\makeatother

\newcommand{\get}[2]{
    \ExecuteMetaData[#1]{#2}
}

\newcommand{\getproof}[2]{
    \iftoggle{useproofs}{\ExecuteMetaData[#1]{#2proof}}{}
}

\newcommand{\getwithproof}[2]{
    \get{#1}{#2}
    \getproof{#1}{#2}
}

\newcommand{\import}[3]{
    \subsection{#1}
    \getwithproof{#2}{#3}
}

\newcommand{\given}[1]{
    Дано выше. (\ref{#1}, стр. \pageref{#1})
}

\renewcommand{\ker}{\text{Ker }}
\newcommand{\im}{\text{Im }}
\newcommand{\grad}{\text{grad}}

\lhead{Конспект по матанализу}
\cfoot{}
\rfoot{November 25, 2019}

\begin{document}
    \begin{example}
        $\langle a,b \rangle \quad f:\langle a, b\rangle \to \R$

        $\forall n\in\N x^n$ --- непрерывно

        Любой многочлен непрерывен, выражение вида

        $$\frac{a_nx^n+a_{n-1}x^{n-1}+\ldots+a_0}{b_mx^m+b_{m-1}x^{m-1}+\ldots+b_0}$$

        тоже непрерывно на области определения.
    \end{example}
    \begin{theorem}
        О непрерывности композиции

        $f:D\subset X\to Y \quad g:E\subset Y\to Z \quad f(D)\subset E$

        $f$ --- непр. в $x_0\in D$, $g$ --- непр. в $f(x_0)$

        Тогда $g\circ f$ непр. в $x_0$
    \end{theorem}
    \begin{proof}
        По Гейне.

        Проверяем, что $\forall (x_n): x_n\in D, x_n\to x_0 \quad g(f(x_n))\xrightarrow{?} g(f(x_0))$

        $y_n:=f(x_n)\xrightarrow[n\to+\infty]{} f(x_0)$

        $y_n\in E$

        $\Rightarrow g(y_n)\to g(y_0)$
    \end{proof}
    \begin{remark}
        $f(x)=x\sin\frac{1}{x}$
        
        $g(x)=|sign(x)|$

        $x\to 0 \ \ f(x)\to 0$

        $y\to 0 \ \ g(y)\to 1$

        $x\to 0 \ \ g(f(x))\to 1?$ --- неверно

        Но: $x_n=\frac{1}{\pi n}\to 0 \ \ f(x_n)=0 \ \ g(f(x_n))\to 0$
    \end{remark}
    \begin{theorem}
        О пределе композиции

        $f:D\subset X\to Y \quad g:E\subset Y\to Z \quad f(D)\subset E$

        $a$ --- предельн. точка $D \quad f(x)\xrightarrow[x\to a]{} A$

        $A$ --- предельн. точка $E \quad g(y)\xrightarrow[y\to A]{} B$

        $\exists V(a) \quad \forall x\in \dot V(a)\cap D \quad f(x)\not=A \quad (*)$

        Тогда $g(f(x))\xrightarrow[x\to a]{} B$
    \end{theorem}
    \begin{proof}
        По Гейне.

        Проверяем, что $\forall (x_n): \substack{x_n\in D \\ x_n\to a \\ x_n\not=a} \quad g(f(x_n))\xrightarrow{?} B$

        $y_n:=f(x_n)\xrightarrow[n\to+\infty]{} A$

        $y_n\in E$

        При больших $N\quad y_n\in V(a) \Rightarrow y_n\not=A$

        $\Rightarrow g(y_n)\to B$
    \end{proof}
    \begin{remark}
        Вместо $(*)$ можно рассмотреть условие $A\in E \quad g$ --- непр. в $A$.
    \end{remark}
    \begin{theorem}
        Топологическое определение непрерывности

        $f: X\to Y$ --- непр. на $X \Leftrightarrow \forall G\subset Y$, откр. $f^{-1}(G)$ --- откр. в $X$.
    \end{theorem}
    \begin{proof}
        ``$\Rightarrow$'' $x_0\in f^{-1}(G) \quad ?\exists V(x_0)\subset f^{-1}(G)$

        $f$ --- непр. в $x_0 \quad \forall U(f(x_0)) \quad W(x_0) \quad \forall x\in W \quad f(x)\in U$

        $f(x_0)\in G$ --- откр. $\Rightarrow \exists U_1(f(x_0))\subset G$

        Для $U_1 \quad \exists W(x_0):x\in W \quad f(x)\in U_1\subset G$

        $W(x_0)\subset f^{-1}(G)$

        ``$\Leftarrow$'' $x_0\in X$ ? непр. $f$ в $x_0$

        $\forall U(f(x_0)) \quad \exists W(x_0) \quad \forall x\in W \quad \forall f(x)\in U$ --- надо проверить

        $U(f(x_0))$ --- откр. $\Rightarrow f^{-1}(U(f(x_0)))$ --- откр., а $x_0\in f^{-1}(U(f(x_0)))$, значит $\exists W(x_0) \subset f^{-1}(U(f(x_0)))$

        Для любого $x\in W(x_0)$ будет выполняться $f(x)\in U(f(x_0))$
    \end{proof}
    \begin{remark}
        $f:[0, 2]\to\R$

        $f(x)=x$

        $f^{-1}((1, +\infty))=(1, 2]$ --- открыто в $[0, 2]$
    \end{remark}
    \begin{theorem}
        Вейерштрасса о непрерывном образе компакта.

        $f:X\to Y$ --- непр. на $X$

        Если $X$ --- комп., то $f(X)$ --- комп.
    \end{theorem}
    \begin{lemma}
        $A\subset \R, A$ --- ограничено и замкнуто $\Rightarrow \sup A\in A$
    \end{lemma}
    \begin{proof}
        По техническому описанию $\sup A$ если $\sup A\not\in A \Rightarrow \sup A$ --- предельная точка $A$.

        Для $\varepsilon=\frac{1}{n} \quad \exists x_n\in A:\sup A - \frac{1}{n}<x_n\leq \sup A$, т.е. $x_n\to \sup A$
    \end{proof}
    \begin{proof}
        $?f(X)$ --- комп.

        $f(X)\subset \bigcup G_\alpha \quad G_\alpha$ --- откр. в $Y$.

        $X\subset \bigcup f^{-1}(G_\alpha)$ --- откр. т.к. $f$ --- непр. $\xRightarrow[X \text{ --- комп.}]{} \exists \alpha_1\ldots\alpha_n \quad X\subset\bigcup\limits_{i=1}^n f^{-1}(G_{\alpha_i}) \Rightarrow f(X)\subset\bigcup\limits_{i=1}^n G_{\alpha_i}$
    \end{proof}
    \begin{consequence}
        Непрерывный образ компакта замкнут и ограничен.
    \end{consequence}
    \begin{consequence}
        \textit{(1-я теорема Вейерштрасса)}
        
        $f:[a,b]\to\R$ --- непр.

        Тогда $f$ --- огр.
    \end{consequence}
    \begin{consequence}
        $f:X\to\R$

        $X$ --- комп., $f$ --- непр. на $X$

        Тогда $\exists\max\limits_{X} f, \min\limits_{X} f$

        $\exists x_0, x_1: \forall x\in X \quad f(x_0)\leq f(x)\leq f(x_1)$
    \end{consequence}
    \begin{consequence}
        $f:[a, b]\to\R$ --- непр.

        $\exists\max f, \min f$
    \end{consequence}

    \section{$O$-символика}
    \begin{definition}
        %<*осимволика>
        $f,g:D\subset X\to\R \ \ x_0$ --- пр. точка $D$

        Если $\exists V(x_0) \ \ \exists \varphi:V(x_0)\cap D\to\R \quad f(x)=g(x)\varphi(x)$ при $x\in V(x_0)\cap D$
        \begin{enumerate}
            \item $\varphi$ --- ограничена. Тогда говорят $f=O(g)$ при $x\to x_0$
            
            ``$f$ ограничена по сравнению с $g$ при $x\to x_0$''
            \item $\varphi(x)\xrightarrow[x\to x_0]{}0 \quad f$ --- беск. малая по отношению к $g$ при $x\to x_0$, $f=o(g)$
            \item $\varphi(x)\xrightarrow[x\to x_0]{}1 \quad f$ и $g$ экв. при $x\to x_0 \quad f\equ\limits_{x\to x_0} g$ 
        \end{enumerate}
        %</осимволика>

        $g, f:D\subset X\to\R$
    \end{definition}
    \begin{definition}
        $\exists c>0 \ \ \forall x\in D \ \ f=O(g) \ \ |f(x)|<c|g(x)|$ --- $f$ ограничена по сравнению с $g$ на множестве $D$.
    \end{definition}
    \begin{definition}
        %<*асимптотическиравныефункции>
        В условиях прошлых определений $f=O(g), g=O(f) \Leftrightarrow f\asymp g$ --- асимптотически сравнимы на множестве $D$, ``величины одного порядка''.
        %</асимптотическиравныефункции>
    \end{definition}
    \begin{remark}
        Первое определение $\Leftrightarrow f=O(g)$ на $V(x_0)\cap D$ в смысле второго определения $\Leftrightarrow \frac{f}{g}$ --- огр. на $V(x_0)\cap D$ \textit{(если $g\not=0$)}

        Второе определение $\xLeftrightarrow[g\not=0]{} \frac{f}{g}\to 0$

        Третье определение $\frac{f}{g}\to 1$ \textit{(если $g\not=0$)}
    \end{remark}
    \begin{consequence}
        \begin{enumerate}
            \item $f\equ g, x\to x_0 \Leftrightarrow f=g+o(g), x\to x_0 \Leftrightarrow f=g+o(f), x\to x_0$
            \begin{proof}
                $$\frac{f}{g}\to1, x\to x_0$$
                $$\frac{f(x)}{g(x)}=1+\alpha(x)$$
                $$\alpha(x)\xrightarrow[x\to0]{} 0$$
                $$f(x)=g(x)+\alpha(x)g(x)=g(x)+o(x)$$
            \end{proof}
            Аналогично для $\frac{g}{f}=1$.

            \item $f=o(g) \Rightarrow f=O(g)$
            \begin{proof}
                $f(x)=\alpha(x)g(x) \quad \alpha(x)=0\Rightarrow \alpha(x)$ --- огр.
            \end{proof}
            \item $\alpha\not=0 \quad f\equ\limits_{x\to x_0}\alpha g$. Тогда $f\asymp g, x\to x_0$
            \begin{proof}
                $$\varepsilon:=\frac{\alpha}{2} \quad \exists V(x_0) \quad \forall x\in V(x_0)\cap D \quad \frac{\alpha}{2}<\frac{f(x)}{g(x)}<\frac{3}{2}\alpha$$
            \end{proof}
        \end{enumerate}
    \end{consequence}
    \begin{example}
        \begin{enumerate}
            \item $$\frac{\sin x}{x}\xrightarrow[x\to0]{} 1 \quad \sin x = x + o(x), x\to 0$$
            \item $$\frac{1 - \cos x}{x^2}\xrightarrow[x\to0]{} \frac{1}{2} \quad \cos x = 1-\frac{x^2}{2}+o(x^2)$$
            $$\frac{1-\cos x}{x^2}=\frac{1}{2}+o(\frac{1}{2}), x\to 0$$
            $$\cos x=1-\frac{1}{2}x^2+x^2o(\frac{1}{2})$$
            \item $$\frac{e^x-1}{x}\xrightarrow[x\to0]{} 1 \quad e^x=1+x+o(x)$$
            \item $$\frac{\ln(1+x)}{x}\xrightarrow[x\to0]{} 1 \quad \ln(1+x)=x+o(x)$$
            \item $$(1+x)^\alpha=1+\alpha o(x), x\to 0$$
        \end{enumerate}
    \end{example}
    \begin{theorem}
        $f,\tilde f, g, \tilde g:D\subset X \to \R$

        $x_0$ --- предельная точка $D$

        $f\equ \tilde f, g\equ \tilde g$ при $x\to x_0$

        Тогда $$\lim\limits_{x\to x_0} f(x)g(x)=\lim\limits_{x\to x_0}\tilde f(x)\tilde g(x)$$
        , т.е. если $\exists$ один из пределов, то $\exists$ и второй и имеет место равенство $$\lim\limits_{x\to x_0} \frac{f(x)}{g(x)} = \lim\limits_{x\to x_0} \frac{\tilde f(x)}{\tilde g(x)}$$
        , если $x_0$ лежит в области определения $\frac{f}{g}$
    \end{theorem}
    \begin{proof}
        $$f(x)g(x) = \tilde f(x)\tilde g(x)\frac{f}{\tilde f}\frac{g}{\tilde g}\to \tilde f(x)\tilde g(x) \cdot 1\cdot 1$$
    \end{proof}
    \begin{remark}
        В условиях теоремы $\lim\limits_{x\to x_0} f+g\not=\lim\limits_{x\to x_0} (\tilde f + \tilde g)$
    \end{remark}
    \subsection{Асимптотическое разложение}

    \begin{definition}
        %<*асимтотическоеразложение>
        $g_n:D\subset X\to\R \quad x_0$ --- пред. точка $D$

        $\forall n \quad g_{n+1}(x)=o(g_n), x\to x_0$

        \begin{example}
            $g_n(x)=x^n, n=0,1,2\ldots \ \ x\to 0 \quad g_{n+1}=xg_n, x\to 0$
        \end{example}

        $(g_n)$ называется \textbf{шкала асимптотического разложения}.
        
        $f:D\to\R$ Если $f(x)=c_0g_0(x)+c_1g_1(x)+\ldots+c_ng_n(x)+o(g_n)$, то это асимптотическое разложение $f$ по шкале $(g_n)$
        %</асимтотическоеразложение>
    \end{definition}
    \begin{theorem}
        $f, g_n: D\subset X\to \R \quad x_0$ --- предельная точка $D$

        $\forall n \ \ g_{n+1}=o(g_n), x\to x_0$

        $\exists U(x_0) \ \ \forall n \ \ \forall x\in\dot U(x_0) \quad g(x)\not=0$

        Если $f(x)=c_0g_0(x)+\ldots+c_ng_n(x)+o(g_n(x))$

        $f(x)=d_0g_0(x)+\ldots+d_mg_m(x)+o(g_m(x))$

        $] n\leq m$

        Тогда $c_0=d_0, c_1=d_1 \ldots c_n=d_n$
    \end{theorem}
    \begin{proof}
        $k:=min\{i:c_i\not=d_i\}$

        $$f(x)=c_0g_0+\ldots+c_{k-1}g_{k-1}+c_kg_k+o(g_k)$$

        $$f(x)=c_0g_0+\ldots+c_{k-1}g_{k-1}+d_kg_k+o(g_k)$$

        $$0=(c_k-d_k)g_k+o(g_k)$$

        $$d_k-c_k=\frac{o(g_k)}{g_k(x)}\xrightarrow[x\to x_0]{}0$$
    \end{proof}
    \begin{example}
        %<*наклоннаяасимптота>
        Пусть $f(x)=Ax+B+o(1), x\to +\infty$

        Прямая $y=Ax+B$ --- наклонная асимптота к графику $f$ при $x\to +\infty$
        %</наклоннаяасимптота>
    \end{example}
\end{document}