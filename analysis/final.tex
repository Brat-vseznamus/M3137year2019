\documentclass[12pt, a4paper]{article}

\usepackage{lastpage}
\usepackage{mathtools}
\usepackage{xltxtra}
\usepackage{libertine}
\usepackage{amsmath}
\usepackage{amsthm}
\usepackage{amsfonts}
\usepackage{amssymb}
\usepackage{enumitem}
\usepackage{xcolor}
\usepackage[left=1.5cm, right=1.5cm, top=2cm, bottom=2cm, bindingoffset=0cm, headheight=15pt]{geometry}
\usepackage{fancyhdr}
\usepackage[russian]{babel}
% \usepackage[utf8]{inputenc}
\usepackage{catchfilebetweentags}
\usepackage{accents}
\usepackage{calc}
\usepackage{etoolbox}
\usepackage{mathrsfs}
\usepackage{wrapfig}

\providetoggle{useproofs}
\settoggle{useproofs}{false}

\pagestyle{fancy}
\lfoot{M3137y2019}
\rhead{\thepage\ из \pageref{LastPage}}

\newcommand{\R}{\mathbb{R}}
\newcommand{\Q}{\mathbb{Q}}
\newcommand{\C}{\mathbb{C}}
\newcommand{\Z}{\mathbb{Z}}
\newcommand{\B}{\mathbb{B}}
\newcommand{\N}{\mathbb{N}}

\newcommand{\const}{\text{const}}

\newcommand{\teormin}{\textcolor{red}{!}\ }

\DeclareMathOperator*{\xor}{\oplus}
\DeclareMathOperator*{\equ}{\sim}
\DeclareMathOperator{\Ln}{\text{Ln}}
\DeclareMathOperator{\sign}{\text{sign}}
\DeclareMathOperator{\Sym}{\text{Sym}}
\DeclareMathOperator{\Asym}{\text{Asym}}
% \DeclareMathOperator{\sh}{\text{sh}}
% \DeclareMathOperator{\tg}{\text{tg}}
% \DeclareMathOperator{\arctg}{\text{arctg}}
% \DeclareMathOperator{\ch}{\text{ch}}

\DeclarePairedDelimiter{\ceil}{\lceil}{\rceil}
\DeclarePairedDelimiter{\abs}{\left\lvert}{\right\rvert}

\setmainfont{Linux Libertine}

\theoremstyle{plain}
\newtheorem{axiom}{Аксиома}
\newtheorem{lemma}{Лемма}

\theoremstyle{remark}
\newtheorem*{remark}{Примечание}
\newtheorem*{exercise}{Упражнение}
\newtheorem*{consequence}{Следствие}
\newtheorem*{example}{Пример}
\newtheorem*{observation}{Наблюдение}

\theoremstyle{definition}
\newtheorem{theorem}{Теорема}
\newtheorem*{definition}{Определение}
\newtheorem*{obozn}{Обозначение}

\setlength{\parindent}{0pt}

\newcommand{\dbltilde}[1]{\accentset{\approx}{#1}}
\newcommand{\intt}{\int\!}

% magical thing that fixes paragraphs
\makeatletter
\patchcmd{\CatchFBT@Fin@l}{\endlinechar\m@ne}{}
  {}{\typeout{Unsuccessful patch!}}
\makeatother

\newcommand{\get}[2]{
    \ExecuteMetaData[#1]{#2}
}

\newcommand{\getproof}[2]{
    \iftoggle{useproofs}{\ExecuteMetaData[#1]{#2proof}}{}
}

\newcommand{\getwithproof}[2]{
    \get{#1}{#2}
    \getproof{#1}{#2}
}

\newcommand{\import}[3]{
    \subsection{#1}
    \getwithproof{#2}{#3}
}

\newcommand{\given}[1]{
    Дано выше. (\ref{#1}, стр. \pageref{#1})
}

\renewcommand{\ker}{\text{Ker }}
\newcommand{\im}{\text{Im }}
\newcommand{\grad}{\text{grad}}

\usepackage{bm}
\usepackage{xcolor}
\usepackage{sectsty}
\usepackage{catchfilebetweentags}

\allsectionsfont{\raggedright}
\subsectionfont{\fontsize{14}{15}\selectfont}

% magical thing that fixes paragraphs
\makeatletter
\def\CatchFBT@sanitize{%
   \@sanitize
   \@makeother\{%
   \@makeother\}%
}
\makeatother

\newcommand{\import}[3]{
    \subsection{#1}
    \ExecuteMetaData[#2]{#3}
}

\setlength\parindent{0pt}

\lhead{Конспект к экзамену}
\cfoot{}
\begin{document}
\section{Определения}
\ExecuteMetaData[opros.tex]{определения}
\ExecuteMetaData[opros2.tex]{определения}
\import{Счётное множество}{11.tex}{счётноемножество}
\import{Мощность континуума}{11.tex}{мощностьконтинуума}
\import{Фундаментальная последовательность}{8.tex}{сходящаясявсебе}
\import{Полное метрическое пространство}{8.tex}{полноепространство}
\import{Классы функций $C^n([a,b])$}{14.tex}{классыфункций}
\textcolor{red}{?}
\import{Производная $n$-го порядка}{13.tex}{производнаяnгопорядка}
\textcolor{red}{?}
\import{Многочлен Тейлора $n$-го порядка}{13.tex}{многочлентейлора}
\import{Разложения Тейлора основных элементарных функций}{14.tex}{разложениятейлора}
\section{Теоремы}
\import{Законы де Моргана}{opros.tex}{законыдеморгана}
\begin{proof}
    \ExecuteMetaData[1.md]{законыдеморганаproof}
\end{proof}
\import{Неравенство Коши-Буняковского, евклидова норма в $\R^m$}{opros.tex}{неравенствокошибуняковскогоевклдиованормавrm}
Неравенство Коши-Буняковского следует из тождества Лагранжа. Докажем его:
\begin{proof}
    \ExecuteMetaData[1.md]{тождестволагранжаproof}
    Таким образом, $$\sum\limits_{(i,k)\in A\times B}(a_ib_i)^2=\sum\limits_{(i,k)\in A\times B}a_i^2\sum\limits_{(i,k)\in A\times B}b_k^2 + \frac{1}{2}\sum\limits_{(i,k)\in A\times B}(a_ib_k-a_kb_i)^2 \geq \sum\limits_{(i,k)\in A\times B}a_i^2\sum\limits_{(i,k)\in A\times B}b_k^2$$
\end{proof}
\import{Аксиома Архимеда. Плотность множества рациональных чисел в $\R$}{opros.tex}{аксиомаархимедаплотностьqвr}
\begin{proof}
    \ExecuteMetaData[2.md]{плотностьqвrproof}
\end{proof}
\import{Неравенство Бернулли}{opros.tex}{неравенствобернулли}
\begin{proof}
    База: $n=1: \ (1+x)^1\geq 1 + x$

    Переход: Дано неравенство $(1+x)^n\geq 1+nx$, оно верно при каком-то $n$. Докажем, что $(1+x)^{n+1}\geq 1 + (n+1)x$

    $$(1+x)^{n+1} = (1+x)(1+x)^n\geq (1+x)(1+nx) = \\ 1 + (n+1)x + nx^2 \geq 1 + (n+1)x$$
\end{proof}
\import{Единственность предела и ограниченность сходящейся последовательности}{3.tex}{оединственностипредела}
\ExecuteMetaData[3.tex]{оединственностипределаproof}
\import{Теорема о предельном переходе в неравенствах для последовательностей и для функций}{opros.tex}{теоремаопредельномпереходевнеравенствахдляпоследовательностейидляфункций}
\ExecuteMetaData[3.tex]{определьномпереходевнеравенствахдляrproof}
\begin{proof}
    По Гейне.

    $\forall (x_n)\to a, x_n\in X, x_n\not=a$:

    $$f(x_n)\to A, g(x_n)\to B, \forall x \ \ f(x) \leq g(x) \Rightarrow f(x_n)\leq g(x_n) \Rightarrow A\leq B$$
\end{proof}
\import{Теорема о двух городовых}{3.tex}{одвухгородовых}
\ExecuteMetaData[3.tex]{одвухгородовыхproof}
\subsection{Бесконечно малая последовательность}
Произведение бесконечно малой последовательности на ограниченную --- бесконечная последовательность, т.е. $(x_n)$ --- беск. малая, $(y_n)$ --- ограничена $\Rightarrow x_ny_n$ --- беск. малая
\begin{proof}
    Возьмём $K$ такое, что $\forall n \ \ |y_n|\leq K$.

    $$\forall \varepsilon > 0 \ \ \exists N \ \ \forall n > N \ \ |x_n|\leq\frac{\varepsilon}{K}$$

    $$|x_ny_n|\leq\frac{\varepsilon}{K}K=\varepsilon \Rightarrow x_ny_n\to0$$
\end{proof}
\import{Теорема об арифметических свойствах предела последовательности в нормированном пространстве и в $R$}{3.tex}{арифметическиесвойствапредела}
\ExecuteMetaData[3.tex]{арифметическиесвойствапределаproof}
\ExecuteMetaData[3.tex]{арифметическиесвойствапределавr}
\import{Неравенство Коши-Буняковского в линейном пространстве, норма, порожденная скалярным произведением}{4.tex}{неравенствокошибуняковского}
\begin{proof}
\ExecuteMetaData[4.tex]{неравенствокошибуняковскогоproof}
\end{proof}
\ExecuteMetaData[3.tex]{метрикапорожденнаянормой}
\subsection{Леммы о непрерывности скалярного произведения и покоординатной сходимости в $\R^n$}
\subsubsection{О покоординатной сходимости в $\R^m$}
\ExecuteMetaData[4.tex]{опокоординатнойсходимости}
\ExecuteMetaData[4.tex]{опокоординатнойсходимостиproof}
\subsubsection{О непрерывности скалярного произведения}
\ExecuteMetaData[4.tex]{онепрерывностискалярногопроизведения}
\ExecuteMetaData[4.tex]{онепрерывностискалярногопроизведенияproof}
\subsection{Открытость открытого шара}
$B(a,r)=\{x\in X : \rho(a,x)<r\}$ --- открыт
\begin{proof}
    $x_0\in B(a, r)$

    Докажем, что $x_0$ --- внутренняя, т.е. $\exists U(x_0)\subset B(a, r)$

    $k:=r-\rho(a, x_0)$

    Докажем, что $B(x_0, k)\subset B(a, r)$

    $$\forall x\in B(x_0, k) \quad \rho(x, x_0)<k$$
    
    $$\rho(a, x_0) + \rho(x, x_0)<r$$

    $$\rho(x, a) \leq \rho(a, x_0) + \rho(x, x_0)<r$$
\end{proof}
\import{Теорема о свойствах открытых множеств}{4.tex}{освойствахзамкнутыхмножеств}
\ExecuteMetaData[4.tex]{освойствахзамкнутыхмножествproof}
\end{document}