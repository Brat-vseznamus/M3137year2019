\documentclass[12pt, a4paper]{article}

\usepackage{lastpage}
\usepackage{mathtools}
\usepackage{xltxtra}
\usepackage{libertine}
\usepackage{amsmath}
\usepackage{amsthm}
\usepackage{amsfonts}
\usepackage{amssymb}
\usepackage{enumitem}
\usepackage{xcolor}
\usepackage[left=1.5cm, right=1.5cm, top=2cm, bottom=2cm, bindingoffset=0cm, headheight=15pt]{geometry}
\usepackage{fancyhdr}
\usepackage[russian]{babel}
% \usepackage[utf8]{inputenc}
\usepackage{catchfilebetweentags}
\usepackage{accents}
\usepackage{calc}
\usepackage{etoolbox}
\usepackage{mathrsfs}
\usepackage{wrapfig}

\providetoggle{useproofs}
\settoggle{useproofs}{false}

\pagestyle{fancy}
\lfoot{M3137y2019}
\rhead{\thepage\ из \pageref{LastPage}}

\newcommand{\R}{\mathbb{R}}
\newcommand{\Q}{\mathbb{Q}}
\newcommand{\C}{\mathbb{C}}
\newcommand{\Z}{\mathbb{Z}}
\newcommand{\B}{\mathbb{B}}
\newcommand{\N}{\mathbb{N}}

\newcommand{\const}{\text{const}}

\newcommand{\teormin}{\textcolor{red}{!}\ }

\DeclareMathOperator*{\xor}{\oplus}
\DeclareMathOperator*{\equ}{\sim}
\DeclareMathOperator{\Ln}{\text{Ln}}
\DeclareMathOperator{\sign}{\text{sign}}
\DeclareMathOperator{\Sym}{\text{Sym}}
\DeclareMathOperator{\Asym}{\text{Asym}}
% \DeclareMathOperator{\sh}{\text{sh}}
% \DeclareMathOperator{\tg}{\text{tg}}
% \DeclareMathOperator{\arctg}{\text{arctg}}
% \DeclareMathOperator{\ch}{\text{ch}}

\DeclarePairedDelimiter{\ceil}{\lceil}{\rceil}
\DeclarePairedDelimiter{\abs}{\left\lvert}{\right\rvert}

\setmainfont{Linux Libertine}

\theoremstyle{plain}
\newtheorem{axiom}{Аксиома}
\newtheorem{lemma}{Лемма}

\theoremstyle{remark}
\newtheorem*{remark}{Примечание}
\newtheorem*{exercise}{Упражнение}
\newtheorem*{consequence}{Следствие}
\newtheorem*{example}{Пример}
\newtheorem*{observation}{Наблюдение}

\theoremstyle{definition}
\newtheorem{theorem}{Теорема}
\newtheorem*{definition}{Определение}
\newtheorem*{obozn}{Обозначение}

\setlength{\parindent}{0pt}

\newcommand{\dbltilde}[1]{\accentset{\approx}{#1}}
\newcommand{\intt}{\int\!}

% magical thing that fixes paragraphs
\makeatletter
\patchcmd{\CatchFBT@Fin@l}{\endlinechar\m@ne}{}
  {}{\typeout{Unsuccessful patch!}}
\makeatother

\newcommand{\get}[2]{
    \ExecuteMetaData[#1]{#2}
}

\newcommand{\getproof}[2]{
    \iftoggle{useproofs}{\ExecuteMetaData[#1]{#2proof}}{}
}

\newcommand{\getwithproof}[2]{
    \get{#1}{#2}
    \getproof{#1}{#2}
}

\newcommand{\import}[3]{
    \subsection{#1}
    \getwithproof{#2}{#3}
}

\newcommand{\given}[1]{
    Дано выше. (\ref{#1}, стр. \pageref{#1})
}

\renewcommand{\ker}{\text{Ker }}
\newcommand{\im}{\text{Im }}
\newcommand{\grad}{\text{grad}}

\lhead{Математический анализ}
\cfoot{}
\rfoot{2.11.2020}

\begin{document}

\subsection*{Потенциальные векторные поля}

\begin{definition}
    Интеграл векторного поля $V$ \textbf{не зависит от пути} в области $O$:

    $\forall A, B\in O \ \ \forall \gamma_1, \gamma_2$ --- кусочно-гладкие из $A$ в $B$:
    $$\int_{\gamma_1} \sum v_i dx_i = \int_{\gamma_2} \sum v_i dx_i$$
\end{definition}

\begin{theorem}[характеризация потенциальных векторных полей в терминах интегралов]
    %<*характеризацияпотенциальныхвекторныхполейвтерминахинтегралов>
    $V$ --- векторное поле в области $O$. Тогда эквивалентны следующие:
    \begin{enumerate}
        \item $V$ --- потенциально
        \item $\int_\gamma \sum v_i dx_i$ не зависит от пути в $O$
        \item $\forall \gamma$ --- кусочно-гладкий, замкнутый в $O$ $\int_\gamma \sum v_i dx_i = 0$
    \end{enumerate}
    %</характеризацияпотенциальныхвекторныхполейвтерминахинтегралов>
\end{theorem}
\begin{proof}\itemfix
    \begin{itemize}
        \item [1$\Rightarrow$2] Обобщенная формула Ньютона-Лейбница
        \item [2$\Rightarrow$3] $\gamma$ --- петля: $[a, b]\to O$. $\gamma(a) = \gamma(b) = A$

              Рассмотрим постоянный путь $\tilde \gamma : [a, b]\to 0, t\mapsto A$. По свойству 2: $\int_\gamma = \int_{\tilde \gamma} \langle V, \gamma' \rangle dt = 0$
        \item [3$\Rightarrow$2] $\gamma_1, \gamma_2$ --- пути с общим началом и концом. Тогда $\gamma := \gamma_2^- \gamma_1$ --- петля. $\gamma$ --- кусочно гладкий $\Rightarrow \int_\gamma = 0$
              % TODO: рисунок
        \item [2$\Rightarrow$1] Фиксируем $A\in O$.

              $\forall x\in O$ выберем кусочно-гладкий путь $\gamma_x$ из $A$ в $x$. Проверим, что $f(x) := \int_{\gamma_x} \sum v_i dx_i$ --- потенциал.

              Достаточно проверить, что $\cfrac{\partial f}{\partial x_1} = V_1$ в $O$.

              Фиксируем $x\in O$. $\gamma_0(t) = x + th e_1, t\in[0, 1]$, $\gamma_0'(t) = (h, 0 \ldots 0) = he_1$

              \begin{figure}[h]
                  \centering
                  \includesvg[scale=0.7]{images/pot_vfield_char.svg}
              \end{figure}

              \begin{align*}
                  f(x + he_1) - f(x) & = \int_{\gamma_{x+he_1}} - \int_{\gamma_x}        \\
                                     & = \int_{\gamma_0\gamma_x} - \int_{\gamma_x}       \\
                                     & = \int_{\gamma_0} = \int_0^1 V_1(\gamma_0(t)) hdt \\
                                     & = \?
              \end{align*}
    \end{itemize}
\end{proof}

\subsection*{Локально-потенциальные векторные поля}

\begin{lemma}
    $V$ --- гладкое, потенциальное в $O$

    Тогда
    \begin{equation}
        \forall x\in O \ \ \forall k,j \ \ \frac{\partial V_k}{\partial x_j}(x) = \frac{\partial V_j}{\partial x_k}(x) \label{производные поля}
    \end{equation}
\end{lemma}
\begin{proof}
    Непрерывные производные не изменяются при порядке дифференцирования:
    $$\frac{\partial V_k}{\partial x_j}(x) = \frac{\partial^2 f}{\partial x_k \partial x_j}(x) = \frac{\partial^2 f}{\partial x_j \partial x_k}(x) = \frac{\partial V_j}{\partial x_k}(x)$$
\end{proof}

\begin{exercise}
    Даны 4 векторных поля в $\R^2$: $(x, y), (x, -y), (y, x), (-y, x)$. Вычеркните лишнее.
\end{exercise}

\begin{theorem}[лемма Пуанкаре]\itemfix
    %<*леммапуанкаре>
    \begin{itemize}
        \item $O\subset \R^m$ --- выпуклая область
        \item $V : O\to\R^m$ --- векторное поле
        \item $V$ удовлетворяет \ref{производные поля}, в т.ч. $V$ --- гладкое.
    \end{itemize}
    Тогда $V$ --- потенциальное.
    %</леммапуанкаре>
\end{theorem}
%<*леммапуанкареproof>
\begin{proof}
    Фиксируем $A\in O$
    $$\forall x\in O \ \ \gamma_x(t) := A + t(x-A), t\in[0, 1]$$
    $$\gamma'_x(t) = x - A$$
    $$f(x) := \int_{\gamma_x} \sum v_i dx_i = \int_0^1 \sum_{k=1}^m V_k(A+t(x-A))(x_k - A_k)dt$$
    Проверим, что $f$ --- потенциал.
    \begin{align}
        \frac{\partial f}{\partial x_j} (x) & = \text{правило Лейбница}                                                                      \nonumber                         \\
                                            & = \int_0^1 V_j(A + t(x-A)) + \sum_{k=1}^m \frac{\partial V_k}{\partial x_j}(\ldots) t (x_k - A_k) dt \nonumber                   \\
                                            & = \int_0^1 V_j(A + t(x-A)) + \sum_{k=1}^m \frac{\partial V_j}{\partial x_k}(\ldots) t (x_k - A_k) dt \label{по производной поля} \\
                                            & = \int_0^1 (tV_j(A + t(x-A)))'_t dt \nonumber                                                                                    \\
                                            & = tV_j(A + t(x-A))\Big|_{t=0}^{t=1} \nonumber                                                                                    \\
                                            & = V_j(x) \nonumber
    \end{align}
    \ref{по производной поля}: по \ref{производные поля}.
\end{proof}
%</леммапуанкареproof>

\begin{remark}
    Это же доказательство проходит для ``звёздных'' областей.
\end{remark}

\begin{definition}
    $V$ --- \textbf{локально потенциальное} векторное поле в $O$, если $\forall x\in O \ \ \exists U(x) : V$ --- потенциально в $U(x)$
\end{definition}

\begin{corollary}[лемма Пуанкаре]\itemfix
    \begin{itemize}
        \item $O\subset \R^m$ --- любая область
        \item $V\in C^1(O)$
        \item $V$ удовлетворяет \ref{производные поля}.
    \end{itemize}
    Тогда $V$ --- локально потенциально.
\end{corollary}

\subsection*{Равномерная сходимость функциональных рядов \textit{(продолжение)}}

\begin{manualtheorem}{3'}[о дифференцировании ряда по параметру]\itemfix
    \begin{itemize}
        \item $u_n\in C^1\langle a, b\rangle$
        \item $\sum u_n(x) = S(x)$ (поточечная сходимость)
        \item $\sum u_n'(x) = \varphi(x)$ (равномерная сходимость)
    \end{itemize}
    Тогда:
    \begin{enumerate}
        \item $S(x) \in C^1\langle a, b\rangle$
        \item $S' = \varphi$ на $\langle a,b \rangle$
    \end{enumerate}
    То есть $\left(\sum u_n(x)\right)' = \sum u_n'(x)$
\end{manualtheorem}
\begin{proof}
    Следует из теоремы 3.

    $S_n \to S$ поточечно, $S'_n \rightrightarrows \varphi$
\end{proof}

\begin{example}
    Формула Вейерштрасса:
    $$\frac{1}{\Gamma(x)} = xe^{\gamma x} \prod_{k=1}^\infty \left(1+\frac{x}{k}\right)e^{-\frac{x}{k}}, x > 0$$
    $\gamma$ --- постоянная Эйлера.
    $$-\ln \Gamma(x) = \ln x + \gamma x + \sum_{k=1}^\infty\underbrace{\left(\ln\left(1+\frac{x}{k}\right) - \frac{x}{k}\right)}_{u_k(x)}$$
    Зафиксируем $x_0$.
    $$u'_k(x) = \frac{1}{1+\frac{x}{k}} \frac{1}{k} - \frac{1}{k} = \frac{1}{x+k} - \frac{1}{k} = \frac{-x}{(x+k)k}$$
    Пусть $M>x_0$. Тогда
    $$\left|\frac{-x}{(x+k)k}\right| \le \frac{M}{k^2}$$
    при $x\in(0, M)$.

    Тогда $\sum\cfrac{-x}{(x+k)k}$ равномерно сходится на $(0, M)$, значит $\ln \Gamma(x) \in C^1(0, M)$, $\cfrac{-x}{(x+k)k}$ --- непр. $\Rightarrow \sum\cfrac{-x}{(x+k)k}$ --- непр. $\Rightarrow \ln \Gamma(x)\in C^1(0, +\infty) \Rightarrow \Gamma(x) \in C^1(0, +\infty)$
\end{example}

\begin{remark}
    Фактически, теорема 3' устанавливает, что $\sum u'_n(x)$ --- непр.
\end{remark}

\begin{remark}
    \begin{align*}
        -\frac{\Gamma'(x)}{\Gamma(x)} & = \frac{1}{x} + \gamma - \sum_{k=1}^{+\infty} \frac{x}{(x+k)k}                        \\
        \Gamma'(x)                    & = -\Gamma(x)\left(\frac{1}{x} + \gamma - \sum_{k=1}^{+\infty} \frac{x}{(x+k)k}\right)
    \end{align*}
    Изучив равномерную сходимость $\left(\cfrac{x}{(x+k)k}\right)'$, получаем, что $\Gamma\in C^2(0, +\infty)$ и т.д. $\Rightarrow \Gamma\in C^\infty(0, +\infty)$
\end{remark}

\begin{manualtheorem}{4'}[о почленном предельном переходе в суммах]\itemfix
    \begin{itemize}
        \item $u_n : E\subset X\to\R$
        \item $X$ --- метрическое пространство
        \item $x_0$ --- предельная точка $E$
        \item $\forall n \exists \text{конечный} \lim\limits_{x\to x_0} u_n(x) = a_n$
        \item $\sum u_n(x)$ равномерно сходится на $E$.
    \end{itemize}
    Тогда:
    \begin{enumerate}
        \item $\sum a_n$ --- сходится
        \item $\sum a_n = \lim\limits_{x\to x_0} \sum\limits_{n=1}^{+\infty} u_n(x)$
    \end{enumerate}
    $$\lim_{x\to x_0} \sum_{n=0}^{+\infty} u_n(x) = \sum_{n=0}^{+\infty} \lim_{x\to x_0} u_n(x)$$
\end{manualtheorem}
\begin{proof}\itemfix
    \begin{enumerate}
        \item $? \sum a_n$ --- сходится

              $S_n(x) = \sum_{k=0}^n u_k(x), S_n^a = \sum_{k=1}^n a_k$

              Проверим, что $S_n^a$ --- фундаментальная:
              \begin{equation}
                  |S^a_{n+p} - S^a_n| \le |S^a_{n+p} - S_{n+p}(x)| + |S_{n+p}(x) - S_n(x)| + |S_n(x) - S_n^a| \label{условие фундаментальности}
              \end{equation}
              Из равномерной сходимости $\sum u_n(x)$
              $$\forall \varepsilon > 0 \ \ \exists N \ \ \forall n > N \ \ \forall p\in\N \ \ \forall x\in E \ \ |S_{n+p} - S_n(x)| < \frac{\varepsilon}{3}$$
              \textit{(Это критерий Больцано-Коши для равномерной сходимости)}

              Зададим $\varepsilon$ по $N$, выберем $n, n+p$ и возьмём $x$ близко к $x_0$ : $|S_{n+p}^a - S_{n+p}(x)| < \cfrac{\varepsilon}{3}$ $|S_{n}^a - S_{n}(x)| < \cfrac{\varepsilon}{3}$

              Тогда выполнено \ref{условие фундаментальности}, т.е. $|S_{n+p} - S_n^a| < \cfrac{\varepsilon}{3} + \cfrac{\varepsilon}{3} + \cfrac{\varepsilon}{3} < \varepsilon$

        \item $\sum a_n \stackrel{?}{=} \lim\limits_{x\to x_0} \sum u_n(x)$

              Сведём к теореме Стокса-Зайдля.

              $\tilde u_n(x) = \begin{cases}
                      u_n(x), & x\in E\setminus \{x_0\} \\
                      a_n,    & x=x_0
                  \end{cases}$ --- задано на $U\cup \{x_0\}$, непрерывно в $x_0$.

              $\sum \tilde u_n(x)$ --- равномерно сходится на $E\cup \{x_0\} \Rightarrow$ сумма ряда непрерывна в $x_0$.

              $$\lim_{x\to x_0} \sum u_n(x) = \lim_{x\to x_0} \sum \tilde u_n(x) = \sum \tilde u_n(x_0) = \sum a_n$$
              $$\sup_x \left|\sum_{k=n}^{+\infty} \tilde u_k(x)\right| \le \underbrace{\sup_{x\in E\setminus\{x_0\}} \left|\sum_{k=n}^{+\infty} u_k(x)\right|}_{\to 0} + \underbrace{\left|\sum_{k=n}^{+\infty} a_k\right|}_{\to0}$$
    \end{enumerate}
\end{proof}

\begin{remark}
    Теорема 4' верна для случая $u_n : E\subset X \to Y$, где $Y$ --- полное нормированное пространство.
\end{remark}

\begin{manualtheorem}{4}[о перестановке двух предельных переходов]\itemfix
    \begin{itemize}
        \item $f_n : E\subset X\to\R$
        \item $x_0$ --- предельная точка $E$
        \item $f_n \xrightrightarrows[n\to+\infty]{E} S(x)$
        \item $f_n(x) \xrightarrow[x\to x_0]{} A_n$
    \end{itemize}
    Тогда:
    \begin{enumerate}
        \item $\exists \lim\limits_{n\to+\infty} A_n = A\in \R$
        \item $S(x) \xrightarrow[x\to x_0]{} A$
    \end{enumerate}
    То есть пунктирное преобразование верно:
    $$\begin{tikzcd}[ampersand replacement=\&]
            f_n \arrow[r, yshift=0.7ex]\arrow[r, yshift=-0.7ex] \arrow[swap]{d}{x\to x_0} \& S(x) \arrow[dashed]{d}{x\to x_0} \\
            f_n' \arrow[dashed, swap]{r}{n\to+\infty} \& \varphi
        \end{tikzcd}$$
\end{manualtheorem}
\begin{proof}
    $u_1 = f_1, \ldots u_k = f_k - f_{k-1}\ldots$

    $a_1 = A_1, \ldots a_k = A_k - A_{k-1}$

    Тогда $f_n = u_1 + u_2 + \ldots + u_n$, $A_n = a_1 + a_2 + \ldots + a_n$

    В эти обозначениях $\sum u_k(x)$ равномерно сходится к сумме $S(x)$.

    $u_k(x)\xrightarrow[x\to x_0]{} a_k$

    Тогда по т. 4' $\sum_{k=1}^n a_k = A_n$ имеет конечный предел при $n\to+\infty$.

    $$\lim_{x\to x_0} \sum u_k(x) = \lim_{x\to x_0} S(x) = \sum a_k = A$$
\end{proof}

\begin{remark}
    Здесь можно было бы вместо $n$ рассматривать ``непрерывный параметр'' $t$.

    $f_n(x) \leftrightarrow \?$
\end{remark}

\end{document}