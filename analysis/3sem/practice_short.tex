\documentclass[12pt, a4paper]{article}

\usepackage{lastpage}
\usepackage{mathtools}
\usepackage{xltxtra}
\usepackage{libertine}
\usepackage{amsmath}
\usepackage{amsthm}
\usepackage{amsfonts}
\usepackage{amssymb}
\usepackage{enumitem}
\usepackage{xcolor}
\usepackage[left=1.5cm, right=1.5cm, top=2cm, bottom=2cm, bindingoffset=0cm, headheight=15pt]{geometry}
\usepackage{fancyhdr}
\usepackage[russian]{babel}
% \usepackage[utf8]{inputenc}
\usepackage{catchfilebetweentags}
\usepackage{accents}
\usepackage{calc}
\usepackage{etoolbox}
\usepackage{mathrsfs}
\usepackage{wrapfig}

\providetoggle{useproofs}
\settoggle{useproofs}{false}

\pagestyle{fancy}
\lfoot{M3137y2019}
\rhead{\thepage\ из \pageref{LastPage}}

\newcommand{\R}{\mathbb{R}}
\newcommand{\Q}{\mathbb{Q}}
\newcommand{\C}{\mathbb{C}}
\newcommand{\Z}{\mathbb{Z}}
\newcommand{\B}{\mathbb{B}}
\newcommand{\N}{\mathbb{N}}

\newcommand{\const}{\text{const}}

\newcommand{\teormin}{\textcolor{red}{!}\ }

\DeclareMathOperator*{\xor}{\oplus}
\DeclareMathOperator*{\equ}{\sim}
\DeclareMathOperator{\Ln}{\text{Ln}}
\DeclareMathOperator{\sign}{\text{sign}}
\DeclareMathOperator{\Sym}{\text{Sym}}
\DeclareMathOperator{\Asym}{\text{Asym}}
% \DeclareMathOperator{\sh}{\text{sh}}
% \DeclareMathOperator{\tg}{\text{tg}}
% \DeclareMathOperator{\arctg}{\text{arctg}}
% \DeclareMathOperator{\ch}{\text{ch}}

\DeclarePairedDelimiter{\ceil}{\lceil}{\rceil}
\DeclarePairedDelimiter{\abs}{\left\lvert}{\right\rvert}

\setmainfont{Linux Libertine}

\theoremstyle{plain}
\newtheorem{axiom}{Аксиома}
\newtheorem{lemma}{Лемма}

\theoremstyle{remark}
\newtheorem*{remark}{Примечание}
\newtheorem*{exercise}{Упражнение}
\newtheorem*{consequence}{Следствие}
\newtheorem*{example}{Пример}
\newtheorem*{observation}{Наблюдение}

\theoremstyle{definition}
\newtheorem{theorem}{Теорема}
\newtheorem*{definition}{Определение}
\newtheorem*{obozn}{Обозначение}

\setlength{\parindent}{0pt}

\newcommand{\dbltilde}[1]{\accentset{\approx}{#1}}
\newcommand{\intt}{\int\!}

% magical thing that fixes paragraphs
\makeatletter
\patchcmd{\CatchFBT@Fin@l}{\endlinechar\m@ne}{}
  {}{\typeout{Unsuccessful patch!}}
\makeatother

\newcommand{\get}[2]{
    \ExecuteMetaData[#1]{#2}
}

\newcommand{\getproof}[2]{
    \iftoggle{useproofs}{\ExecuteMetaData[#1]{#2proof}}{}
}

\newcommand{\getwithproof}[2]{
    \get{#1}{#2}
    \getproof{#1}{#2}
}

\newcommand{\import}[3]{
    \subsection{#1}
    \getwithproof{#2}{#3}
}

\newcommand{\given}[1]{
    Дано выше. (\ref{#1}, стр. \pageref{#1})
}

\renewcommand{\ker}{\text{Ker }}
\newcommand{\im}{\text{Im }}
\newcommand{\grad}{\text{grad}}

\lhead{Краткий конспект практик}
\cfoot{}
\rfoot{}

\begin{document}

\section{Равномерная сходимость последовательности}

Практически все задачи решаются следующим образом:
\begin{enumerate}
    \item Находим кандидата на роль \(f\) по формуле \(f = \lim\limits_{n\to +\infty} f_n(x)\). Предел берется при фиксированном \(x\). \(f\) может зависеть от \(x\) и может быть разрывным \textit{(например, 2751.б)}
    \item Проверяем, что \(\rho(f, f_n) \xrightarrow{n\to +\infty} 0\), где \(\rho(f, f_n) = \sup\limits_{x\in E} |f(x) - f_n(x)|\)

          Методы нахождения супремума:
          \begin{enumerate}
              \item \textbf{Прямой}: \(\sup\limits_{x\in(0, +\infty)} \cfrac{1}{x + n} = \cfrac{1}{n} \to 0\)
              \item \textbf{Оценка сверху} \textit{(доказывает равн. сходимость)}: \( \sup\limits_{x\in[0, 1]} \left|\cfrac{x + x^2}{1 + n + x}\right| \leq \cfrac{2}{1 + n}\to 0\)
              \item \textbf{Оценка снизу} \textit{(доказывает отсутствие равн. сходимости)} --- обычно подстановка конкретного \(x\) \textit{(он может зависеть от \(n\))}:
                    \[\sup_{x} \left|\sin\left( \frac{x}{n} \right)\right| \stackrel{x: = n}{ \ge } \sin(1) \not\to 0\]
              \item \textbf{Оценка снизу пределом}: \(\sup\limits_{g\in E} g(x) \geq \lim\limits_{x\to A} g(x)\), где \(A\) --- предельная точка \(E\).
          \end{enumerate}
\end{enumerate}

Есть более простой признак \textbf{отсутствия} равн. сходимости:
\[f_n(x) \rightrightarrows f \implies \forall x\in E \ \ f_n(x) \to f(x)\]

\section{Равномерная сходимость рядов}

\[\sum_{n = 1}^{ +\infty} u_n(x) = S(x) \quad S_N \rightrightarrows S \text{ на } E\]

Методы доказательства:
\begin{enumerate}
    \item \textbf{По определению} \textit{(см. равн. сходимость последовательностей)} --- самый простой вариант, для него нужен способ посчитать частную сумму. Это либо телескоп, либо прогрессия. Иногда из дроби можно получить телескоп разложением на простые дроби.
    \item \textbf{По абсурдности} если \(u_n(x)\not\rightrightarrows 0\), то сумма не сходится.
    \item \textbf{Признак Вейерштрасса}

          \(\sum u_n(x), x\in E\):
          \begin{enumerate}
              \item \(\forall x\in E : |u_n(x)| \leq C_n\)
              \item \(\sum C_n\) --- сходится
          \end{enumerate}
          Тогда ряд равномерно сходится.

          Обычно берут \(C_n = \frac{1}{n^\alpha}, \alpha > 1\), но иногда нужно думать про сходимость ряда \(C_n\), т.к. она не очевидна. Вольфрам в помощь.

          При придумывании \(C_n\) можно найти точку экстремума максимума \(|u_n(x)|\) (\(x = \dots\)) через \(u_n(x)'_x\) и подставить такой \(x\).
    \item \textbf{Критерий Больцано-Коши} обычно доказывает \textbf{отсутствие} равномерной сходимости, хотя его можно использовать и для обратного.
          \[\exists \varepsilon > 0 \ \ \forall N \ \ \exists n > N, \exists m\in \N, \exists x \quad  |u_{n + 1}(x) + \dots + u_{n + m}(x)| > \varepsilon\]
          Тогда равномерной сходимости нет.

          Идея в том, чтобы иметь большие ( \( \geq \frac{C}{n}\) ) слагаемые, для этого надо придумать соответствующий \(x\). Часто используется оценка суммы \(|\sum_{i = n + 1}^{n + m} u_i(x)| \geq \min u_i(x) \cdot m\).

          Для обратного нужно построить отрицание критерия.
    \item \textbf{Признак Дирихле} для \(\sum a_n(x)b_n(x)\):
          \begin{enumerate}
              \item Частичные суммы \(\sum a_n\) равномерно ограничены:
                    \[\exists C_a \ \ \forall N \ \ \forall x\in E \ \ \left|\sum_{k = 1}^N a_k(x)\right| \leq C_a\]
              \item \begin{enumerate}
                        \item При фиксированном \(x\) функция \(b_n(x)\) монотонна по \(n\)
                        \item \(b_n(x) \rightrightarrows 0\) на \(E\) при \(n\to +\infty\)
                    \end{enumerate}
          \end{enumerate}
    \item \textbf{Признак Абеля} для \(\sum a_n(x)b_n(x)\):
          \begin{enumerate}
              \item \(\sum a_n(x)\) равномерно сходится на \(E\)
              \item \begin{enumerate}
                        \item \(b_n(x)\) монотонно по \(n\)
                        \item \(b_n(x)\) равномерно ограничено:
                              \[\exists C_b \ \ \forall N \ \ \forall x\in E \ \ |b_n(x)| \leq C_b\]
                    \end{enumerate}
          \end{enumerate}
\end{enumerate}

\section{Свойства через ряды}

\begin{enumerate}
    \item \begin{itemize}
              \item \(\sum u_n(x) = f(x)\)
              \item \(u_n(x)\) непр. в \(x_0\)
              \item Ряд равномерно сходится в \(U(x_0)\)
          \end{itemize}
          Тогда \(f\) \textbf{непр}. в \(x_0\)
    \item \begin{itemize}
              \item \(\sum u_n'(x) = \varphi(x)\)
              \item \(\sum u_n'(x)\) равномерно сходится в \(U(x_0)\)
          \end{itemize}
          Тогда \(f\) --- \textbf{дифф}. в \(x_0, f'(x) = \varphi(x)\)
    \item \begin{itemize}
              \item \(\sum u_n(x) \) равномерно сходится на \([a, b]\)
              \item \(u_n\) непр. на \([a, b]\)
          \end{itemize}
          Тогда \(\int_a^b f(x)dx = \sum \int_a^b u_n(x) dx\)
\end{enumerate}

Когда требуют равномерную сходимость в \(U(x_0) \ \ \forall x_0\in E\), можно пытаться доказать равномерную сходимость в \(E\). Это проще сделать, но не всегда возможно.

\section{Степенные ряды}

Степенной ряд --- ряд вида \(\sum a_n(x - x_0)^n\). Он сходится при \(\left|x - x_0\right|< R, R = \frac{1}{\overline \lim\sqrt[n]{|a_n|}}\)

Верхний предел --- наибольший предел из пределов всех подпоследовательностей.

Иногда ответ выдает \(R = \lim \left|\frac{a_n}{a_{n + 1}}\right|\), но не всегда.

И ещё возможно сходится при \(x = x_0 \pm R\). Сходимость при таком \(x\) находится путём подстановки соответствующего \(x\) в ряд. Но этот ряд не простой, в нем не будет работать признак Даламбера и Коши.

Можно решать заменой на эквивалентное (возможно по модулю), если это не помогает, то применяется Лейбниц или Дирихле.

\section{Разложение фукнций}

Мы знаем, что если \(f(x) = \sum a_n(x - x_0)^n\), то это ряд Тейлора, т.е. \(a_n = \frac{f^{(n)} (x_0)}{n!} \).

У нас есть пять основных разложений:
\begin{align*}
    e^z        & = 1 + z + \frac{z^2}{2} + \dots + \frac{x^n}{n!} + \dots                                                    \\
    \sin x     & = x - \frac{x^3}{3!} + \dots + ( - 1)^{n - 1} \frac{x^{2n - 1}}{(2n - 1)!} + \dots                          \\
    \cos x     & = 1 - \frac{x^2}{2!} + \dots + ( - 1)^n \frac{x^{2n}}{(2n)!} + \dots                                        \\
    (1 + x)^m  & = 1 + mx + \frac{m(m - 1)}{2!}x^2 + \dots + \frac{m(m - 1)\dots(m - n + 1)}{n!} + \dots \quad x\in( - 1, 1) \\
    \ln(1 + x) & = x - \frac{x^2}{2} + \frac{x^3}{3} - \dots + ( - 1)^{n - 1}\frac{x^n}{n} + \dots \quad x\in( - 1, 1]
\end{align*}

Функции вида \(f(x) = (x - x_0)^\alpha \cdot g(x)\), где \(g = \sum a_n(x - x_0)^n\), раскладываются по формуле \(f = \sum a_n(x - x_0)^{n + \alpha}\), то есть разложение можно домножить на \((x - x_0)^\alpha\).

Композиция сохраняется разложением.

Можно найти разложение производной (\(f'(x)\)), потом проинтегрировать и найти искомое. Константа находится подстановкой \(x = x_0\), при ней ряд после интеграции \( = 0\), \(f(x_0)\) может не быть \(0\).

\section*{Трюки}

\begin{enumerate}
    \item \[\frac{t}{1 + t^2} \leq \frac{1}{2}\]
    \item \[\left|xe^{ - x^2n}\right| \leq \frac{C}{\sqrt{n}}\]
    \item \[a + b \geq 2\sqrt{ab}\]
    \item \(\sum \frac{( - 1)^n}{n^p} \) сходится при \(p > 0\) и расходится при \(p \leq 0\)
    \item \[\sum \frac{( - 1)^n}{n} = -\ln 2\]
    \item \[\sum_{n = 1}^N \sin (nx) = \sin \frac{nx}{2} \sin \frac{(n + 1)x}{2} \frac{1}{\sin \frac{x}{2}}\]
    \item \[\frac{\sin x}{\sin \frac{x}{2}} = 2\cos \frac{x}{2}\]
    \item \[\left|\sum_{n = 1}^N \sin (nx)\right| \leq \frac{1}{|e^{ix} - 1|}\]
    \item Если \(u_n(x)\) монотонно по \(n\), то:
          \[\left|\sum_{n \geq N} ( - 1)^n u_n(x) \right| \leq u_N(x)\]
    \item Если есть равномерная сходимость ряда в \(U(0)\), то \(\sum u_n(x) \xrightarrow{x\to 0} \sum u_n(0)\).
\end{enumerate}

\end{document}