\documentclass[12pt, a4paper]{article}

\usepackage{lastpage}
\usepackage{mathtools}
\usepackage{xltxtra}
\usepackage{libertine}
\usepackage{amsmath}
\usepackage{amsthm}
\usepackage{amsfonts}
\usepackage{amssymb}
\usepackage{enumitem}
\usepackage{xcolor}
\usepackage[left=1.5cm, right=1.5cm, top=2cm, bottom=2cm, bindingoffset=0cm, headheight=15pt]{geometry}
\usepackage{fancyhdr}
\usepackage[russian]{babel}
% \usepackage[utf8]{inputenc}
\usepackage{catchfilebetweentags}
\usepackage{accents}
\usepackage{calc}
\usepackage{etoolbox}
\usepackage{mathrsfs}
\usepackage{wrapfig}

\providetoggle{useproofs}
\settoggle{useproofs}{false}

\pagestyle{fancy}
\lfoot{M3137y2019}
\rhead{\thepage\ из \pageref{LastPage}}

\newcommand{\R}{\mathbb{R}}
\newcommand{\Q}{\mathbb{Q}}
\newcommand{\C}{\mathbb{C}}
\newcommand{\Z}{\mathbb{Z}}
\newcommand{\B}{\mathbb{B}}
\newcommand{\N}{\mathbb{N}}

\newcommand{\const}{\text{const}}

\newcommand{\teormin}{\textcolor{red}{!}\ }

\DeclareMathOperator*{\xor}{\oplus}
\DeclareMathOperator*{\equ}{\sim}
\DeclareMathOperator{\Ln}{\text{Ln}}
\DeclareMathOperator{\sign}{\text{sign}}
\DeclareMathOperator{\Sym}{\text{Sym}}
\DeclareMathOperator{\Asym}{\text{Asym}}
% \DeclareMathOperator{\sh}{\text{sh}}
% \DeclareMathOperator{\tg}{\text{tg}}
% \DeclareMathOperator{\arctg}{\text{arctg}}
% \DeclareMathOperator{\ch}{\text{ch}}

\DeclarePairedDelimiter{\ceil}{\lceil}{\rceil}
\DeclarePairedDelimiter{\abs}{\left\lvert}{\right\rvert}

\setmainfont{Linux Libertine}

\theoremstyle{plain}
\newtheorem{axiom}{Аксиома}
\newtheorem{lemma}{Лемма}

\theoremstyle{remark}
\newtheorem*{remark}{Примечание}
\newtheorem*{exercise}{Упражнение}
\newtheorem*{consequence}{Следствие}
\newtheorem*{example}{Пример}
\newtheorem*{observation}{Наблюдение}

\theoremstyle{definition}
\newtheorem{theorem}{Теорема}
\newtheorem*{definition}{Определение}
\newtheorem*{obozn}{Обозначение}

\setlength{\parindent}{0pt}

\newcommand{\dbltilde}[1]{\accentset{\approx}{#1}}
\newcommand{\intt}{\int\!}

% magical thing that fixes paragraphs
\makeatletter
\patchcmd{\CatchFBT@Fin@l}{\endlinechar\m@ne}{}
  {}{\typeout{Unsuccessful patch!}}
\makeatother

\newcommand{\get}[2]{
    \ExecuteMetaData[#1]{#2}
}

\newcommand{\getproof}[2]{
    \iftoggle{useproofs}{\ExecuteMetaData[#1]{#2proof}}{}
}

\newcommand{\getwithproof}[2]{
    \get{#1}{#2}
    \getproof{#1}{#2}
}

\newcommand{\import}[3]{
    \subsection{#1}
    \getwithproof{#2}{#3}
}

\newcommand{\given}[1]{
    Дано выше. (\ref{#1}, стр. \pageref{#1})
}

\renewcommand{\ker}{\text{Ker }}
\newcommand{\im}{\text{Im }}
\newcommand{\grad}{\text{grad}}

\lhead{Математический анализ}
\cfoot{}
\rfoot{16.11.2020}

\begin{document}

\subsection*{Гомотопия}

Неформально гомотопия --- непрерывная деформация объектов. У нас рассматриваемые объекты --- пути.

\begin{definition}
    %<*гомотопия>
    \textbf{Гомотопия} двух (\textit{непрерывных}) путей \(\gamma_0, \gamma_1 : [a, b] \to O\subset \R^m\) это непрерывное отображение \(\Gamma : \underbrace{[a, b]}_{t} \times \underbrace{[0, 1]}_{u} \to O\), такое что:
    \begin{itemize}
        \item \(\Gamma(\circ , 0) = \gamma_0\)
        \item \(\Gamma(\circ , 1) = \gamma_1\)
    \end{itemize}

    Гомотопия \textbf{связанная} (\textit{не связная}), если:
    \begin{itemize}
        \item \(\gamma_0(a) = \gamma_1(a)\)
        \item \(\gamma_0(b) = \gamma_1(b)\)
        \item \(\forall u\in[0, 1]\ \Gamma(a, u) = \gamma_0(a), \Gamma(b, u) = \gamma_1(b)\)
    \end{itemize}

    \begin{figure}[h]
        \centering
        \includesvg[scale=1]{images/гомотопия_связанная.svg}
        \caption{Связанная гомотопия.\\ Пунктиром --- \(\Gamma(\circ, u)\) для различных \(u\)}
    \end{figure}

    Гомотопия \textbf{петельная}, если:
    \begin{itemize}
        \item \(\gamma_0(a) = \gamma_0(b)\)
        \item \(\gamma_1(a) = \gamma_1(b)\)
        \item \(\forall u\in[0, 1]\ \Gamma(a, u) = \Gamma(b, u)\)
    \end{itemize}

    \begin{figure}[h]
        \centering
        \includesvg[scale=1]{images/гомотопия_петельная.svg}
        \caption{Петельная гомотопия.\\ Пунктиром --- \(\Gamma(\circ, u)\) для различных \(u\)}
    \end{figure}

    \pagebreak
    %</гомотопия>
\end{definition}

\begin{theorem}\itemfix
    %<*интегралпосвязанногомотопнымпутям>
    \begin{itemize}
        \item \(V\) --- локально потенциальное векторное поле в \(O\subset\R^m\)
        \item \(\gamma_0, \gamma_1\) --- связанно гомотопные пути
    \end{itemize}
    Тогда \(\int_{\gamma_0} \sum V_i dx_i = \int_{\gamma_1} \sum V_idx_i\)
    %</интегралпосвязанногомотопнымпутям>
\end{theorem}

\begin{remark}
    То же самое верно для петельных гомотопий.
\end{remark}

%<*интегралпосвязанногомотопнымпутямproof>
\begin{proof}
    Пусть \(\Gamma\) --- гомотопия \(\gamma_0\) и \(\gamma_1\).

    \(\gamma_u(t) : = \Gamma(t, u), t\in[a, b], u\in[0, 1]\)

    \[\Phi(u) = \int_{\gamma_u} \sum V_i dx_i\]

    Мы хотим доказать, что \(\Phi(u) = \const\). Докажем более простой факт, что \(\Phi\) --- локально постоянна, тогда в силу компактности и связности отрезка \(\Phi\) будет постоянна.

    Определение локально постоянной функции:
    \[\forall u_0\in[0, 1]\ \ \exists W(u_0) : \forall u\in W(u_0)\cap [0, 1]\quad \Phi(u) = \Phi(u_0)\]

    \(\Gamma\) --- непр. на \([a, b] \times [0, 1]\) --- комп. \( \Rightarrow \Gamma\) равномерно непрерывна:
    \[\forall \delta > 0\ \exists \sigma > 0\ \forall t, t' : |t - t'| < \sigma\ \forall u,u': |u - u'|< \sigma\quad |\Gamma(t, u) - \Gamma(t', u')| < \frac{\delta}{2} \]

    Возьмём \(\delta\) из леммы о похожести близких путей (\ref{лемма 3, лекция 9}) для пути \(\gamma_{u_0}\).

    Если \(|u - u_0| < \sigma \ \ |\Gamma(t, u) - \Gamma(t, u_0)|< \frac{\delta}{2}\) при \(t\in[a, b]\), т.е. \(\gamma_u\) и \(\gamma_{u_0}\) похожи по лемме о похожести близких путей. Хочется сказать, что интегралы по \(\gamma_u\) и \(\gamma_{u_0}\) таким образом равны, однако это не обосновано, для этого необходимо, чтобы пути были кусочно-гладкими.

    Построим кусочно-гладкий путь \(\tilde \gamma_{u_0}\), \(\frac{\delta}{4}\)-близкий к \(\gamma_{u_0}\), т.е.
    \[\forall t\in[a, b] \ \ |\gamma_{u_0}(t) - \tilde \gamma_{u_0}(t)| < \frac{\delta}{4}\]
    и кусочно-гладкий путь \(\tilde \gamma_u\), \(\frac{\delta}{4}\)-близкий к \(\gamma_{u}\). Тогда \(\tilde \gamma_{u_0}\) и \(\tilde \gamma_u\) - \(\delta\)-близкие к \(\gamma_{u_0}\) \( \Rightarrow \) они \(V\)-похожи \( \Rightarrow \) \[\int_{\gamma_u} \sum V_i dx_i\defeq \int_{\tilde \gamma_u} \dots = \int_{\tilde \gamma_{u_0}} \dots \defeq \int_{\gamma_{u_0}} \dots \]
    Таким образом, \(\Phi(u) = \Phi(u_0)\) при \(|u - u_0|< \delta\), т.е. \(\Phi\) --- локально постоянна.
\end{proof}
%</интегралпосвязанногомотопнымпутямproof>

\begin{definition}
    %<*односвязнаяобласть>
    Область \(O\subset \R^m\) --- \textbf{односвязная}, если любой замкнутый путь в ней гомотопен постоянному пути.

    Простыми словами --- в \(O\) нет дырок, иначе путь вокруг дырки нельзя было бы стянуть.

    \begin{figure}[h]
        \centering
        \includesvg[scale=1]{images/односвязно.svg}
        \caption{Стягивание замкнутого пути (сплошной линией) к постоянному пути (точке)}
    \end{figure}
    %</односвязнаяобласть>
\end{definition}

\begin{remark}\itemfix
    \begin{enumerate}
        \item Выпуклая область --- односвязная.

              \begin{figure}[h]
                  \centering
                  \includesvg[scale=0.9]{images/гомотетия.svg}
                  \caption{Применение гомотетии с центром \(A\)}
              \end{figure}

              Это доказывается тем, что для любого пути можно применить гомотетию в качестве гомотопии: \(\Gamma(t, u) = F_{1 - u}(\gamma(t))\), где \(F_{\alpha}\) --- гомотетия с центром в произвольной точке \(A\), лежащей внутри области, ограниченной путём \(\gamma\), и коэффициентом \(\alpha\)

              \begin{remark}
                  Гомотетия --- равномерное стягивание всех точек к одной.
              \end{remark}
        \item Гомеоморфный образ односвязного множества --- односвязен.

              \(\Phi : O \to O'\) --- гомеоморфизм, \(\gamma\) --- петля в \(O'\), \(\Phi^{ - 1}(\gamma)\) --- петля в \(O\).

              \(\Gamma : [a, b] \times [0, 1] \to O\) --- гомотопия \(\Phi^{ - 1}(\gamma)\) и постоянного пути \(\tilde \gamma\equiv A\)

              \(\Phi \circ \Gamma\) --- гомотопия \(\gamma\) с постоянным путём \(\dbltilde \gamma \equiv \Phi(A)\)
    \end{enumerate}
\end{remark}

\begin{theorem}\itemfix
    %<*леммапуанкареводносвязнойобласти>
    \begin{itemize}
        \item \(O\subset \R^m\) --- односвязная область
        \item \(V\) --- локально потенциальное векторное поле в \(O\)
    \end{itemize}
    Тогда \(V\) --- потенциальное в \(O\)
    %</леммапуанкареводносвязнойобласти>
\end{theorem}

%<*леммапуанкареводносвязнойобластиproof>
\begin{proof}
    \(V\) --- локально потенциально, \(\sphericalangle \gamma_0\) --- кусочно-гладкая петля, тогда \(\gamma_0\) гомотопна постоянному пути \(\gamma_1\) \( \Rightarrow \)
    \[\int_{\gamma_0} = \int_{\gamma_1} = \int_a^b \langle V(\gamma_1(t)), \underbrace{\gamma'_1(t)}_{\equiv0} \rangle dt = 0\]
    Тогда по теореме о характеризации потенциальных векторных полей в терминах интегралов \(V\) потенциально.
\end{proof}
%</леммапуанкареводносвязнойобластиproof>

\begin{corollary}
    Теорема Пуанкаре верна в односвязной области.
\end{corollary}

Пусть даны две плоскости, соединенные гвоздём, между плоскостями есть зазор. На гвоздь надета веревочка в виде петли. Можно ли снять веревочку с гвоздя?

\begin{theorem}[о веревочке]\itemfix
    %<*оверевочке>
    \begin{itemize}
        \item \(O = \R^2\setminus \{(0, 0)\}\)
        \item \(\gamma : [0, 2\pi] \to O, t\mapsto (\cos t, \sin t)\)
    \end{itemize}
    Тогда эта петля нестягиваема.

    Неформальная формулировка: пусть даны две плоскости, соединенные гвоздём, между плоскостями есть зазор. На гвоздь надета веревочка в виде петли. Можно ли снять веревочку с гвоздя?

    \begin{figure}[h]
        \centering
        \includesvg[scale=1.3]{images/веревочка.svg}
        \caption{Веревочка (жирным), надетая на ``гвоздь'' (цилиндр)}
    \end{figure}
    %</оверевочке>
\end{theorem}

%<*оверевочкеproof>
\begin{proof}
    \(V(x, y) = \left( \cfrac{ -y}{x^2 + y^2}, \cfrac{x}{x^2 + y^2}\right)\) --- векторное поле в \(\R^2\)
    \[\frac{\partial V_1}{\partial y} = \frac{ -(x^2 + y^2) + 2 y^2}{(x^2 + y^2)^2} = \frac{y^2 - x^2}{(x^2 + y^2)^2} \]
    \[\frac{\partial V_2}{\partial x} = \frac{(x^2 + y^2) - 2x^2}{(x^2 + y^2)^2} = \frac{y^2 - x^2}{(x^2 + y^2)^2}\]
    Таким образом, \(\cfrac{\partial V_1}{\partial y} = \cfrac{\partial V_2}{\partial x}\) в области \(O\). Тогда по лемме Пуанкаре \(V\) --- локально потенциально.

    При этом
    \begin{align*}
        \int_{\gamma}\sum V_idx_i & = \int_0^{2\pi} \frac{ -\sin t}{\cos^2 t + \sin^2 t} ( - \sin t)dt + \frac{\cos t}{\cos^2 t + \sin^2 t} \cos t dt \\
                                  & = \int_0^{2\pi} 1dt = 2\pi
    \end{align*}

    Таким образом, если бы существовал постоянный путь \(\tilde \gamma\), которому \(\gamma\) гомотопен, то \(\int_\gamma = \int_{\tilde \gamma} = 0\), но это не так.
\end{proof}
%</оверевочкеproof>

\subsection*{Степенные ряды}

\begin{example}
    \begin{enumerate}
        \item \(\sum_{n = 0}^{+\infty} z^n, R = \cfrac{1}{\overline \lim \sqrt[n]{1}} = 1 , |z| < 1\) --- сходится, \(|z|> 1\) --- расходится, \(|z|= 1\) --- расходится, т.к. слагаемые \(\not\to 0\)
        \item \(\sum \cfrac{z^n}{n}, R = \cfrac{1}{\overline \lim \sqrt[n]{\frac{1}{n}}} = 1\)
              \begin{enumerate}
                  \item \(z = 1, \sum \frac{1}{n}\) --- расходится
                  \item \(z = - 1, \sum \frac{( - 1)^n}{n} \) --- сходится
                  \item \(z = e^{i\varphi}, \varphi \neq 0, 2\pi \ \ \sum \cfrac{e^{in\varphi}}{n} = \sum \cfrac{\cos n\varphi + i \sin n\varphi}{n} \) --- сходится по признаку Дирихле.
              \end{enumerate}
        \item \(\sum \frac{z^n}{n^2}, R = 1, |z|= 1 \Rightarrow \left|\frac{z^n}{n^2}\right| \leq \frac{1}{n^2}\) сходится.
        \item \(\sum n!z^n, R = \cfrac{1}{\overline \lim \sqrt[n]{n!}} \approx \cfrac{1}{\overline \lim \sqrt[n]{n^ne^{ - n}\sqrt{2\pi n}}} = \cfrac{1}{\overline \lim \frac{n}{e}} = 0\), в \(0\) сходится, в остальных точках расходится.
        \item \(\sum \cfrac{z^n}{n!}, R = +\infty\) --- везде сходится.
    \end{enumerate}
\end{example}

\begin{theorem}[о равномерной сходимости и непрерывности степенного ряда]\itemfix
    %<*оравномернойсходимостиинепрерывностистепенногоряда>
    \begin{itemize}
        \item \(\sum a_n(z - z_0)^n\)
        \item \(0 < R \leq +\infty\)
    \end{itemize}
    Тогда:
    \begin{enumerate}
        \item \(\forall r : 0 < r < R\) ряд сходится равномерно на \(\overline{B(z_0, r)}\)
        \item \(f(z) = \sum a_n(z - z_0)^n\) --- непрерывна в \(B(z_0, r)\)
    \end{enumerate}
    %</оравномернойсходимостиинепрерывностистепенногоряда>
\end{theorem}

%<*оравномернойсходимостиинепрерывностистепенногорядаproof>
\begin{proof}\itemfix
    \begin{enumerate}
        \item Если \(0 < r < R\), то при \(z - z_0 = r\) ряд абсолютно сходится, т.е. \(\sum |a_n| r^n < +\infty\)

              Признак Вейерштрасса:
              \begin{enumerate}
                  \item При \(|z - z_0|\leq r\) \(|a_n(z - z_0)^n| \leq |a_n|r^n\)
                  \item \(\sum |a_n|r^n < +\infty\)
              \end{enumerate}
              \( \Rightarrow \) есть сходимость на \(\overline{B(z_0, r)}\)
        \item Следствие из пункта 1 и теоремы Стокса-Зайдля.

              Если \(z\) удовлетворяет \(|z - z_0| < R\), то \(\exists r_0 < R : z\in B(z_0, r_0)\)

              На \(B(z_0, r_0)\) есть равномерная сходимость \( \Rightarrow \) \(f\) непр. в точке \(z\).
    \end{enumerate}
\end{proof}
%</оравномернойсходимостиинепрерывностистепенногорядаproof>

\begin{definition}
    \(f : \mathbb{C} \to \mathbb{C}\). Тогда производная \(f\) это:
    \[f'(z_0) = \lim_{z\to z_0} \frac{f(z) - f(z_0)}{z - z_0}\]
\end{definition}

\begin{remark}
    \(f(z_0 + h) = f(z_0) + f'(z_0)h + o(|h|), h\in\mathbb{C}\)
\end{remark}

\begin{lemma}\itemfix
    \begin{itemize}
        \item \(w, w_0\in\mathbb{C}\)
        \item \(|w|< r\)
        \item \(|w_0|< r\)
    \end{itemize}
    Тогда \(|w^n - w^n_0|\le n r^n |w - w_0|, n\in\N\).
\end{lemma}
\begin{proof}
    \[w^n - w^n_0 = (w - w_0)(w^{n - 1} + \underbrace{w^{n - 2}w_0}_{\text{по модулю} \le r^{n-1}} + \dots + w^{n - 1}_0)\]
\end{proof}

\begin{lemma}[о дифференцируемости степенного ряда]\itemfix
    %<*одифференцируемостистепенногоряда>
    \begin{itemize}
        \item [(A)] \(\sum_{n = 0}^{\infty} a_n(z - z_0)^n, 0 < R < +\infty\)
        \item [(A')] \(\sum_{n = 1}^{\infty} na_n (z - z_0)^{n - 1}\)
    \end{itemize}
    Тогда:
    \begin{enumerate}
        \item Радиус сходимости (A') равен \(R\)
        \item \(\forall z\in B(z_0,R) \ \ \exists f'(z)\) и \(f'(z) = \sum n a_n(z - z_0)^n\)
    \end{enumerate}
    %</одифференцируемостистепенногоряда>
\end{lemma}

%<*одифференцируемостистепенногорядаproof>
\begin{proof}\itemfix
    \begin{enumerate}
        \item По формуле Адамара.

              Ряд (A') сходится при каком-то \(z\) \( \Leftrightarrow \sum n a_n(z - z_0)^n\) --- сходится.

              \[\frac{1}{\overline \lim \sqrt[n]{n|a_n|}} = \frac{1}{1\cdot \overline \lim \sqrt[n]{|a_n|}} = R\]
        \item \(\sphericalangle a\in B(z_0, R) , \exists r < R : a\in B(z_0, r)\)

              \(a = z_0 + w_0, |w_0|< r\)

              \(z = z_0 + w\)

              \begin{figure}[h]
                  \centering
                  \includesvg[scale=1]{images/штука.svg}
              \end{figure}

              \begin{equation}
                  \frac{f(z) - f(a)}{z - a} = \sum_{n = 0}^{ +\infty} a_n\frac{(z - z_0)^n - (a - z_0)^n}{z - a} = \sum_{n = 1}^{ +\infty} \underbrace{a_n \frac{w^n - w^n_0}{w - w_0}}_{\substack{\text{модуль по лемме} \\ n r^{n-1}|a_n|}} \label{комплексная производная для a'}
              \end{equation}
              \(\sum nr^{n - 1}|a_n|\) сходится по пункту 1.

              То есть ряд \eqref{комплексная производная для a'} в круге \(z\in B(z_0, r)\)

              \[\lim \frac{f(z) - f(a)}{z - a} = \sum_{n = 1}^{ +\infty} a_n\lim \frac{(z - z_0)^n - (a - z_0)^n}{z - a} = \sum n a_n(a - z_0)^{n - 1}\]
    \end{enumerate}
\end{proof}
%</одифференцируемостистепенногорядаproof>

\end{document}