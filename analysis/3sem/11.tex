\documentclass[12pt, a4paper]{article}

\usepackage{lastpage}
\usepackage{mathtools}
\usepackage{xltxtra}
\usepackage{libertine}
\usepackage{amsmath}
\usepackage{amsthm}
\usepackage{amsfonts}
\usepackage{amssymb}
\usepackage{enumitem}
\usepackage{xcolor}
\usepackage[left=1.5cm, right=1.5cm, top=2cm, bottom=2cm, bindingoffset=0cm, headheight=15pt]{geometry}
\usepackage{fancyhdr}
\usepackage[russian]{babel}
% \usepackage[utf8]{inputenc}
\usepackage{catchfilebetweentags}
\usepackage{accents}
\usepackage{calc}
\usepackage{etoolbox}
\usepackage{mathrsfs}
\usepackage{wrapfig}

\providetoggle{useproofs}
\settoggle{useproofs}{false}

\pagestyle{fancy}
\lfoot{M3137y2019}
\rhead{\thepage\ из \pageref{LastPage}}

\newcommand{\R}{\mathbb{R}}
\newcommand{\Q}{\mathbb{Q}}
\newcommand{\C}{\mathbb{C}}
\newcommand{\Z}{\mathbb{Z}}
\newcommand{\B}{\mathbb{B}}
\newcommand{\N}{\mathbb{N}}

\newcommand{\const}{\text{const}}

\newcommand{\teormin}{\textcolor{red}{!}\ }

\DeclareMathOperator*{\xor}{\oplus}
\DeclareMathOperator*{\equ}{\sim}
\DeclareMathOperator{\Ln}{\text{Ln}}
\DeclareMathOperator{\sign}{\text{sign}}
\DeclareMathOperator{\Sym}{\text{Sym}}
\DeclareMathOperator{\Asym}{\text{Asym}}
% \DeclareMathOperator{\sh}{\text{sh}}
% \DeclareMathOperator{\tg}{\text{tg}}
% \DeclareMathOperator{\arctg}{\text{arctg}}
% \DeclareMathOperator{\ch}{\text{ch}}

\DeclarePairedDelimiter{\ceil}{\lceil}{\rceil}
\DeclarePairedDelimiter{\abs}{\left\lvert}{\right\rvert}

\setmainfont{Linux Libertine}

\theoremstyle{plain}
\newtheorem{axiom}{Аксиома}
\newtheorem{lemma}{Лемма}

\theoremstyle{remark}
\newtheorem*{remark}{Примечание}
\newtheorem*{exercise}{Упражнение}
\newtheorem*{consequence}{Следствие}
\newtheorem*{example}{Пример}
\newtheorem*{observation}{Наблюдение}

\theoremstyle{definition}
\newtheorem{theorem}{Теорема}
\newtheorem*{definition}{Определение}
\newtheorem*{obozn}{Обозначение}

\setlength{\parindent}{0pt}

\newcommand{\dbltilde}[1]{\accentset{\approx}{#1}}
\newcommand{\intt}{\int\!}

% magical thing that fixes paragraphs
\makeatletter
\patchcmd{\CatchFBT@Fin@l}{\endlinechar\m@ne}{}
  {}{\typeout{Unsuccessful patch!}}
\makeatother

\newcommand{\get}[2]{
    \ExecuteMetaData[#1]{#2}
}

\newcommand{\getproof}[2]{
    \iftoggle{useproofs}{\ExecuteMetaData[#1]{#2proof}}{}
}

\newcommand{\getwithproof}[2]{
    \get{#1}{#2}
    \getproof{#1}{#2}
}

\newcommand{\import}[3]{
    \subsection{#1}
    \getwithproof{#2}{#3}
}

\newcommand{\given}[1]{
    Дано выше. (\ref{#1}, стр. \pageref{#1})
}

\renewcommand{\ker}{\text{Ker }}
\newcommand{\im}{\text{Im }}
\newcommand{\grad}{\text{grad}}

\lhead{Математический анализ}
\cfoot{}
\rfoot{23.11.2020}

\begin{document}

\subsection*{Степенные ряды}

\begin{corollary}\itemfix
    \begin{itemize}
        \item \(f(z) = \sum_n a_n(z - z_0)^n\)
        \item \(|z - z_0| < R\)
        \item \(0 < R\le +\infty\)
    \end{itemize}
    Тогда: \(f\in C^{\infty}(B(z_0, R))\) и все производные можно найти почленным дифференцированием.
\end{corollary}
\begin{proof}
    Это очевидно из леммы о дифференцируемости степенного ряда. Если в некоторой точке \(a\) нет гладкости, то она не лежит в \(B(z_0, R)\)
\end{proof}

\begin{theorem}[из ТФКП]\itemfix
    \begin{itemize}
        \item \(f\) комплексно дифференцируема в \(z_0\) \textit{(на самом деле, в некоторой области, но нас не волнует формальность в этой теореме)}.
    \end{itemize}
    Тогда \(f =\sum a_n(z - z_0)^n\) и \(R =\) расстояние от \(z_0\) до ближайшей особой точки.
\end{theorem}

\begin{example}
    \(f = \cfrac{1}{1 + x^2}, z_0 = 0\)

    \begin{figure}[h]
        \centering
        \includesvg[scale=1]{images/примерразложениефункции.svg}
        \caption{Пунктиром --- круг сходимости степенного ряда}
    \end{figure}
\end{example}

%<*следствиеобинтегрировании>
\begin{corollary}\itemfix
    \begin{itemize}
        \item \(f(x) = \sum a_n(x - x_0)^n\)
        \item \(a_n, x_0, x\in\R\)
    \end{itemize}
    Тогда:
    \begin{enumerate}
        \item \(\sum \frac{a_n}{n + 1}(x - x_0)^{n + 1}\) имеет тот же радиус сходимости, что и \(f\)
        \item \(\int_{x_0}^x f(t)dt = \sum \frac{a_n}{n + 1}(x - x_0)^{n + 1}\)
    \end{enumerate}
\end{corollary}
%</следствиеобинтегрировании>

%<*следствиеобинтегрированииexample>
\begin{example}
    \[\int f(x) dx = \sum \frac{a_n}{n + 1}(x - x_0)^{n + 1} + C\]
    \[f(x) : = \arcctg x\]
    \[f' = \frac{ -1}{1 + x^2} = - 1 + x^2 - x^4 + \dots \]
    Проинтегрируем:
    \begin{align*}
        \arcctg x & = C - x + \frac{x^3}{3} - \frac{x^5}{5} + \dots             \\
        \arcctg x & = \arcctg 0 - x + \frac{x^3}{3} - \frac{x^5}{5} + \dots     \\
        \arcctg x & = \frac{\pi}{2} - x + \frac{x^3}{3} - \frac{x^5}{5} + \dots
    \end{align*}
    То же самое можно получить, взяв определенный интеграл.
\end{example}
%</следствиеобинтегрированииexample>

\subsection*{Метод Абеля суммирования числовых рядов}

\begin{theorem}\itemfix
    %<*методабеля>
    \begin{itemize}
        \item \(\sum\limits_{n = 0}^{ +\infty} c_n\) --- сходится
        \item \(c_n\in \C\)
        \item \(f(x) =\sum c_nx^n\)
        \item \(R\ge 1\)
        \item \( - 1 < x < 1\)
    \end{itemize}
    Тогда \(\lim\limits_{x\to 1-0} f(x) = \sum c_n\)
    %</методабеля>
\end{theorem}
%<*методабеляproof>
\begin{proof}
    Ряд \(\sum c_nx^n\) равномерно сходится на \([0, 1]\) по признаку Абеля при \(a_n(x) = c_n, b_n(x) = x^n\).

    Функции \(c_nx^n\) непрерывны на \([0, 1] \xRightarrow{\text{Стокса-Зайдля}} \sum c_n x^n\) --- непр. на \([0, 1]\)
\end{proof}
%</методабеляproof>

\begin{example}
    \[\sum_{n = 1}^{ +\infty} \frac{1}{n(n + 1)} = \sum \frac{1}{n} - \frac{1}{n + 1} = 1 - \lim_{N\to +\infty} \frac{1}{N + 1} = 1\]
    \[f(x) = \sum_{n = 1}^{ +\infty} \frac{x^{n + 1}}{n(n + 1)}\]
    \[f'(x) = \sum_{n = 1}^{ +\infty} \frac{x^{n}}{n}\]
    \[f''(x) = \sum_{n = 1}^{ +\infty} x^{n - 1} = 1 + x + x^2 + \dots = \frac{1}{1 - x}\]
    Интегрируем:
    \[f'(x) = \sum \frac{x^n}{n} = -\ln(1 - x) + C\]
    При \(x = 0\) \(C = 0\)

    Ещё раз интегрируем:
    \[\sum \frac{x^{n + 1}}{n(n + 1)} = -\int \ln(1 - x) dx = (1 - x)\ln{(1 - x)} + x + C\]
    При \(x = 0\) \(C = 0\)
    \[\sum \frac{1}{n(n + 1)} = \lim_{x\to 1 - 0} (1 - x)\ln(1 - x) + x = 1\]
\end{example}

%<*теоремаабеля>
\begin{corollary}[т. Абеля]\itemfix
    \begin{itemize}
        \item \(\sum a_n = A\)
        \item \(\sum b_n = B\)
        \item \(c_n = a_0b_n + a_1 b_{n - 1} + \dots + a_n b_0\)
        \item \(\sum c_n = C\)
    \end{itemize}
    Тогда \(C = AB\)
\end{corollary}
%</теоремаабеля>
%<*теоремаабеляproof>
\begin{proof}
    \(f(x) = \sum a_n x^n, g(x) = \sum b_n x^n, h(x) = \sum c_nx^n, x\in[0, 1]\)

    При \(x < 1\) есть абсолютная сходимость \(f(x)\) и \(g(x)\). Можно перемножать: \(f(x)g(x) = h(x)\), при \(x\to 1 - 0\ \ \) \(A\cdot B = C\)
\end{proof}
%</теоремаабеляproof>

\subsection*{Экспонента (комплексной переменной)}

\begin{definition}
    \[\exp(z) : = \sum_{n = 0}^{ +\infty} \frac{z^n}{n!} \]
\end{definition}

Свойства:
%<*свойстваэкспоненты>
\begin{enumerate}
    \item \(\exp(0) = 1\)
    \item \(\exp(z)' = \sum\limits_{n = 1}^{ +\infty} \frac{z^{n - 1}}{(n - 1)!} = \sum\limits_{k = 0}^{ +\infty} \frac{z^{k}}{k!} = \exp z\)
    \item \(\forall z, w\in \C \ \ \exp(z + w) = \exp z \cdot \exp w\)
\end{enumerate}
%</свойстваэкспоненты>
%<*свойстваэкспонентыproof>
\begin{proof}\itemfix
    \begin{enumerate}
        \item \[\exp(0) = \sum_{n = 0}^{ +\infty} \frac{0^n}{n!} = 1\]
        \item \[\exp(z)' = \sum_{n = 1}^{+\infty} \frac{z^{n - 1}}{(n - 1)!} = \sum_{n = 0}^{ +\infty} \frac{z^n}{n!} = \exp(z)\]
        \item
              \begin{align*}
                  \exp(z + w) & = \sum_{n = 0}^{+\infty} \frac{(z + w)^n}{n!}                                                               \\
                              & = \sum_{n = 0}^{ +\infty} \sum_{k = 0}^{n} \binom{n}{k}\frac{z^k w^{n - k}}{n!}                             \\
                              & = \sum_{n = 0}^{ +\infty} \sum_{k = 0}^{n} \frac{z^k w^{n - k}}{k!(n - k)!}                                 \\
                              & = \left( \sum_{n = 0}^{+\infty} \frac{z^n}{n!} \right) \left( \sum_{n = 0}^{+\infty} \frac{w^n}{n!} \right) \\
                              & = \exp(z)\exp(w)
              \end{align*}
    \end{enumerate}
\end{proof}
%</свойстваэкспонентыproof>

Возвращаем кредит: в первом семестре говорилось, что \(\exists f_0\) --- показательная функция, такая что \(f(x + y) = f(x)\cdot f(y)\) и \(\lim\limits_{x\to 0} \frac{f_0(x) - 1}{x} = 1\)

\(f_0(x) : = \exp(x)\)
\[\lim_{x\to 0} \frac{\exp(x) - 1}{x} = \exp'(0) = 1\]

% \begin{theorem}
%     \[\forall z, w\in \C \ \ \exp(z + w) = \exp z \cdot \exp w\]
% \end{theorem}
% \begin{exercise}
%     Доказать к следующей лекции
% \end{exercise}

\section*{Теория меры}

\subsection*{Системы множеств}

Здесь и далее система \(\iff \) множество, так говорится, чтобы избежать ``множество множеств''

\begin{obozn}
    \(A_i\) --- множества. Тогда \(\bigsqcup\limits_i A_i\) --- дизъюнктное объединение.

    \(A_i\) --- попарно не пересекаются \( \iff \) ``\(A_i\) --- дизъюнктно''
\end{obozn}

\begin{definition}
    \(X\) --- множество, тогда \(2^X\) --- система всех подмножеств \(X\).

    %<*полукольцо>
    \(\mathcal P \subset 2^X\) --- \textbf{полукольцо}, если:
    \begin{itemize}
        \item \(\text{\O}\subset \mathcal P\)
        \item \(\forall A,B \in \mathcal P \ \ A\cap B\in \mathcal{P}\)
        \item \(\forall A, A' \in \mathcal{P} \ \ \exists \text{ кон. и дизъюнктные } B_1\dots B_n\in \mathcal{P} : A\setminus A' = \bigsqcup\limits_i B_i\)
    \end{itemize}
    %</полукольцо>
\end{definition}

\begin{example}
    \begin{enumerate}
        \item \(2^X\) --- полукольцо
        \item \(X = \R^2, \mathcal{P} =\) ограниченные подмножества, в том числе \O
        \item \begin{definition}
                  %<*ячейка>
                  Ячейка в \(\R^m\) это \([a, b) = \{x\in\R^m : \forall i \ \ x_i\in[a_i, b_i)\} \)

                  \begin{figure}[h]
                      \centering
                      \includesvg[scale=1.3]{images/ячейка.svg}
                      \caption{\([a, b)\) --- ячейка в \(\R^2\)}
                  \end{figure}
                  %</ячейка>
              \end{definition}

              \(\mathcal{P}\) --- множество ячеек в \(\R^m\) --- полукольцо.

              \begin{proof}
                  \(\sphericalangle m = 2\)
                  \begin{enumerate}
                      \item Очевидно
                      \item \(A\cap B = [a, a') \cap [b, b') = \{(x_1, x_2) \in\R^2 : \forall i = 1, 2 \ \ \max(a_i, b_i) \leq x_i < \min(a_i', b_i')\} \)
                      \item \(A\setminus A' = \bigsqcup\limits_{i = 1}^8 B_i \)
                  \end{enumerate}
              \end{proof}

              \begin{figure}[h]
                  \centering
                  \includesvg[scale=1]{images/вырезаниеячейки.svg}
              \end{figure}
        \item \(A = \{1, 2, 3,4,5,6\} \)

              \[\forall i \ \ A_i : = A\]

              \[X = \bigotimes\limits_{i = 1}^{ +\infty} A_i = \{(a_1, a_2, a_3, \dots) , \forall i\ \ a_i\in A_i\} \]

              \[\sigma = \begin{pmatrix} i_1 & i_2 & \dots & i_k \\ \alpha_1 & \alpha_2 & \dots & \alpha_k \end{pmatrix}, k\in\N\cup \{0\} , \forall l \ \ \alpha_l\in A_{i_l} \]

              \[\mathcal{P} = \{X_\sigma\}_\sigma \]

              \[X_\sigma = \{a\in X : a_{i_1} = \alpha_1, \dots, a_{i_k} = \alpha_k\} \]

              \(\mathcal{P}\) --- полукольцо

              \begin{enumerate}
                  \item \(\text{\O} = X_\sigma, \sigma = \begin{pmatrix} 1 & 1 \\ 1 & 2 \end{pmatrix} \)
                  \item \(X_\sigma\cap X_{\sigma'} = X_{\sigma\cup \sigma'}\)
                  \item \(X_\sigma \setminus X_{\sigma'}\)

                        \(\sigma = \begin{pmatrix} 1 \\ 6 \end{pmatrix} , \sigma' = \begin{pmatrix} 2 \end{pmatrix} 1\)

                        \(X_\sigma \setminus X_{\sigma'}\) = на первой координате 6, на второй --- не 1 = \(X_{\sigma_2} \cap X_{\sigma_3} \cap \dots \cap X_{\sigma_6}, \sigma_k = \begin{pmatrix} 1 & 2 \\ 6 & k \end{pmatrix} \).
              \end{enumerate}
        \item Ячейки с рациональными координатами.
    \end{enumerate}
\end{example}

Свойства:
\begin{enumerate}
    \item Как показывают примеры:
          \begin{enumerate}
              \item \(A\in \mathcal{P} \not \Rightarrow A^c = X\setminus A\in \mathcal{P}\)
              \item \(A, B\in \mathcal{P}\), нельзя утверждать, что:
                    \begin{itemize}
                        \item \(A\cap B\not\in \mathcal{P}\)
                        \item \(A\setminus  B\not\in \mathcal{P}\)
                        \item \(A\triangle  B = (A\setminus B)\cup (B\setminus A)\not\in \mathcal{P}\)
                    \end{itemize}
          \end{enumerate}
    \item Модернизируем утверждение 3:

          \(A, A_1 \dots A_n \in \mathcal{P}\). Тогда \(A\setminus (A_1\cup A_2 \cup \dots \cup A_n)\) --- представимо в виде дизъюнктного объединения элементом \(\mathcal{P}\)

          \begin{proof}
              Докажем по индукции по \(n\).

              База ( \(n = 1\)) --- аксиома 3.

              Переход:

              \begin{align*}
                  A\setminus (A_1\cup \dots \cup A_n) & = (A\setminus (A_1\cup \dots \cup A_{n - 1}))\setminus A_n \\
                                                      & = \left( \bigsqcup_{i = 1}^k B_i \right)\setminus A_n      \\
                                                      & = \bigsqcup_{i = 1}^k (B_i \setminus A_n)                  \\
                                                      & = \bigsqcup_{i = 1}^k \bigsqcup_{j = 1}^{l_i} D_{ij}       \\
              \end{align*}
          \end{proof}
\end{enumerate}

\begin{definition}
    %<*алгебраподмножеств>
    \(\mathfrak A \subset 2^X\) --- \textbf{алгебра подмножеств} в \(X\), если:
    \begin{enumerate}
        \item \(\forall A,B\in \mathfrak A \ \ A\setminus B\in\mathfrak A\)
        \item \(X\in\mathfrak A\)
    \end{enumerate}
    %</алгебраподмножеств>
\end{definition}

Свойства:
\begin{enumerate}
    \item \(\text{\O} = X\setminus X\in \mathfrak A\)
    \item \(A\cap B = A\setminus (A\setminus B)\in \mathfrak A\)
    \item \(A^c = X\setminus A\in \mathfrak A\)
    \item \(A\cup B\in\mathfrak A\), потому что \((A\cup B)^c = A^c \cap B^c\)
    \item \(A_1\dots A_n \in \mathfrak A \Rightarrow \bigcup\limits_{i = 1}^n A_i, \bigcap\limits_{i = 1}^n A_i\in \mathfrak A\) --- по индукции
    \item Всякая алгебра есть полукольцо

          Обратное неверно, см. пример 2.
\end{enumerate}

\begin{example}
    \begin{enumerate}
        \item \(2^X\)
        \item \(X = \R^2, \mathfrak A\) --- ограниченные подмножества или их дополнения.
    \end{enumerate}
\end{example}

\end{document}