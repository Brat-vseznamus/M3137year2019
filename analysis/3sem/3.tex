\documentclass[12pt, a4paper]{article}

\usepackage{lastpage}
\usepackage{mathtools}
\usepackage{xltxtra}
\usepackage{libertine}
\usepackage{amsmath}
\usepackage{amsthm}
\usepackage{amsfonts}
\usepackage{amssymb}
\usepackage{enumitem}
\usepackage{xcolor}
\usepackage[left=1.5cm, right=1.5cm, top=2cm, bottom=2cm, bindingoffset=0cm, headheight=15pt]{geometry}
\usepackage{fancyhdr}
\usepackage[russian]{babel}
% \usepackage[utf8]{inputenc}
\usepackage{catchfilebetweentags}
\usepackage{accents}
\usepackage{calc}
\usepackage{etoolbox}
\usepackage{mathrsfs}
\usepackage{wrapfig}

\providetoggle{useproofs}
\settoggle{useproofs}{false}

\pagestyle{fancy}
\lfoot{M3137y2019}
\rhead{\thepage\ из \pageref{LastPage}}

\newcommand{\R}{\mathbb{R}}
\newcommand{\Q}{\mathbb{Q}}
\newcommand{\C}{\mathbb{C}}
\newcommand{\Z}{\mathbb{Z}}
\newcommand{\B}{\mathbb{B}}
\newcommand{\N}{\mathbb{N}}

\newcommand{\const}{\text{const}}

\newcommand{\teormin}{\textcolor{red}{!}\ }

\DeclareMathOperator*{\xor}{\oplus}
\DeclareMathOperator*{\equ}{\sim}
\DeclareMathOperator{\Ln}{\text{Ln}}
\DeclareMathOperator{\sign}{\text{sign}}
\DeclareMathOperator{\Sym}{\text{Sym}}
\DeclareMathOperator{\Asym}{\text{Asym}}
% \DeclareMathOperator{\sh}{\text{sh}}
% \DeclareMathOperator{\tg}{\text{tg}}
% \DeclareMathOperator{\arctg}{\text{arctg}}
% \DeclareMathOperator{\ch}{\text{ch}}

\DeclarePairedDelimiter{\ceil}{\lceil}{\rceil}
\DeclarePairedDelimiter{\abs}{\left\lvert}{\right\rvert}

\setmainfont{Linux Libertine}

\theoremstyle{plain}
\newtheorem{axiom}{Аксиома}
\newtheorem{lemma}{Лемма}

\theoremstyle{remark}
\newtheorem*{remark}{Примечание}
\newtheorem*{exercise}{Упражнение}
\newtheorem*{consequence}{Следствие}
\newtheorem*{example}{Пример}
\newtheorem*{observation}{Наблюдение}

\theoremstyle{definition}
\newtheorem{theorem}{Теорема}
\newtheorem*{definition}{Определение}
\newtheorem*{obozn}{Обозначение}

\setlength{\parindent}{0pt}

\newcommand{\dbltilde}[1]{\accentset{\approx}{#1}}
\newcommand{\intt}{\int\!}

% magical thing that fixes paragraphs
\makeatletter
\patchcmd{\CatchFBT@Fin@l}{\endlinechar\m@ne}{}
  {}{\typeout{Unsuccessful patch!}}
\makeatother

\newcommand{\get}[2]{
    \ExecuteMetaData[#1]{#2}
}

\newcommand{\getproof}[2]{
    \iftoggle{useproofs}{\ExecuteMetaData[#1]{#2proof}}{}
}

\newcommand{\getwithproof}[2]{
    \get{#1}{#2}
    \getproof{#1}{#2}
}

\newcommand{\import}[3]{
    \subsection{#1}
    \getwithproof{#2}{#3}
}

\newcommand{\given}[1]{
    Дано выше. (\ref{#1}, стр. \pageref{#1})
}

\renewcommand{\ker}{\text{Ker }}
\newcommand{\im}{\text{Im }}
\newcommand{\grad}{\text{grad}}

\lhead{Математический анализ}
\cfoot{}
\rfoot{21.9.2020}

\begin{document}

% \begin{example}
%     Загадка. $f(x, y) = x^2 + Axy^2 + y^4$.

%     $(a, 0)$ --- подозрительная точка, $d^2f = 2dx^2 \ge 0$

%     $f(x_0 + h) \stackrel{\text{Тейлор}}{=} f(x_0) + \underbrace{df(x_0, h)}_{0} + \frac{1}{2}d^2 f(x_0, h) + \frac{1}{3!} d^3f(x_0, h) + \frac{1}{4!}d^4f(x_0, h)$
% \end{example}

В первом семестре была задача нахождения максимального по площади вписанного $n$-угольника.
Мы выяснили, что если максимум существует, то он достигается правильным $n$-угольником (\textit{суждение про сдвиг точки}). Кроме того, мы доказали, что максимум существует, сделаем это снова, но другим методом.

\begin{proof}
    Пусть внутренние углы многоугольника $\varphi_1\ldots \varphi_n$, тогда
    \begin{align*}
        S & = \frac{1}{2} r^2(\sin \varphi_1 + \ldots + \sin \varphi_n)                                                 \\
          & = \frac{1}{2} r^2(\sin \varphi_1 + \ldots + \sin \varphi_{n-1} - \sin (\varphi_1 - \ldots - \varphi_{n-1}))
    \end{align*}

    Очевидно $\forall i \ \ 0<\varphi_1<\pi \quad \pi < \varphi_1+\ldots+\varphi_{n-1} < 2\pi$.

    Найдём максимум путём дифференцирования. Это требует существования максимума внутри области определения. Если все неравенства сделать нестрогими, то область определения становится замкнутой, очевидно ограниченной $\Rightarrow \exists \max$ по теореме Вейерштрасса. Кроме того, максимум не лежит на границе области определения из очевидных геометрических соображений.

    $$\frac{\partial S}{\partial \varphi_i} = 0 \Leftrightarrow \cos \varphi_i = \cos(\varphi_1+\ldots+\varphi_{n-1})$$
    \begin{align*}
        \cos \varphi_i - \cos(\varphi_1+\ldots+\varphi_{n-1})   & = 0              \\
        2\sin \frac{n\varphi}{2} \sin \frac{\pi - 2n\varphi}{2} & = 0              \\
        \varphi                                                 & = \frac{2\pi}{n} \\
    \end{align*}
\end{proof}

\section*{Диффеоморфизмы}

\begin{definition}
    \textbf{Область} --- открытое связное множество.
\end{definition}

\begin{definition}
    $F : \underbrace{O}_{\text{область}}\subset \R^m\to \R^m$ --- \textbf{диффеоморфизм}, если:
    \begin{itemize}
        \item $F$ обратимо
        \item $F$ дифференцируемо
        \item $F^{-1}$ дифференцируемо
    \end{itemize}
\end{definition}
\begin{remark}
    $Id = F\circ F^{-1} = F^{-1}\circ F$

    $E = F' (F^{-1})' \Rightarrow \forall x \ \ \det F'(x) \not=0$
\end{remark}

\begin{lemma}[о почти локальной иньективности]\itemfix
    \begin{itemize}
        \item $F : O\subset\R^m\to\R^m$
        \item $F$ дифф. в $x_0\in O$
        \item $\det F'(x_0)\not=0$
    \end{itemize}
    Тогда $\exists c>0, \delta>0 \ \ \forall h < \delta \ \ |F(x_0+h)-F(x_0)| > C|h|$
\end{lemma}
\begin{proof}
    Если $F$ --- линейное отображение:
    $$|F(x_0+h) - F(x_0)| = |F(h)| = |F'(x_0)h| \ge ||F'(x_0)||\cdot |h| \ge \frac{1}{||(F'(x_0))^{-1}||}|h|$$
    $$|F(x_0+h) - F(x_0)| = |F'(x_0)h + \alpha(h)|h|| \ge c|h| - \frac{c}{2}|h| \ge \frac{c}{2}|h|$$
\end{proof}

\begin{theorem}[о сохранении области]\itemfix
    \begin{itemize}
        \item $F : O\subset\R^m \to\R^m$
        \item $F$ дифф.
        \item $\forall x\in O \ \ \det F'(x)\not=0$
    \end{itemize}
    Тогда $F(O)$ --- открыто.
\end{theorem}

\begin{remark}
    $O$ --- связно, $F$ --- непр. $\Rightarrow F(O)$ связно.
\end{remark}

\begin{proof}
    $x_0 \in O \Rightarrow F(x_0)\in F(O)$ --- внутренняя? в $F(O)$

    По лемме $\exists c, \delta : \forall h \in \overline{B(0, \delta)} \ \ |F(x_0+h) - F(x_0)| > C|h|$

    В частности $F(x_0 + h)\not=F(x_0)$ при $|h| = \delta$

    $r := \frac{1}{2}\rho(y_0, F(S(x_0, \delta))$

    \textcolor{red}{Не дописано}

    % Если $y\in B(y_0\ldots)$, то $\rho(y_0\ldots)$
\end{proof}

\end{document}