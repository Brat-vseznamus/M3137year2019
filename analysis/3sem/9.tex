\documentclass[12pt, a4paper]{article}

\usepackage{lastpage}
\usepackage{mathtools}
\usepackage{xltxtra}
\usepackage{libertine}
\usepackage{amsmath}
\usepackage{amsthm}
\usepackage{amsfonts}
\usepackage{amssymb}
\usepackage{enumitem}
\usepackage{xcolor}
\usepackage[left=1.5cm, right=1.5cm, top=2cm, bottom=2cm, bindingoffset=0cm, headheight=15pt]{geometry}
\usepackage{fancyhdr}
\usepackage[russian]{babel}
% \usepackage[utf8]{inputenc}
\usepackage{catchfilebetweentags}
\usepackage{accents}
\usepackage{calc}
\usepackage{etoolbox}
\usepackage{mathrsfs}
\usepackage{wrapfig}

\providetoggle{useproofs}
\settoggle{useproofs}{false}

\pagestyle{fancy}
\lfoot{M3137y2019}
\rhead{\thepage\ из \pageref{LastPage}}

\newcommand{\R}{\mathbb{R}}
\newcommand{\Q}{\mathbb{Q}}
\newcommand{\C}{\mathbb{C}}
\newcommand{\Z}{\mathbb{Z}}
\newcommand{\B}{\mathbb{B}}
\newcommand{\N}{\mathbb{N}}

\newcommand{\const}{\text{const}}

\newcommand{\teormin}{\textcolor{red}{!}\ }

\DeclareMathOperator*{\xor}{\oplus}
\DeclareMathOperator*{\equ}{\sim}
\DeclareMathOperator{\Ln}{\text{Ln}}
\DeclareMathOperator{\sign}{\text{sign}}
\DeclareMathOperator{\Sym}{\text{Sym}}
\DeclareMathOperator{\Asym}{\text{Asym}}
% \DeclareMathOperator{\sh}{\text{sh}}
% \DeclareMathOperator{\tg}{\text{tg}}
% \DeclareMathOperator{\arctg}{\text{arctg}}
% \DeclareMathOperator{\ch}{\text{ch}}

\DeclarePairedDelimiter{\ceil}{\lceil}{\rceil}
\DeclarePairedDelimiter{\abs}{\left\lvert}{\right\rvert}

\setmainfont{Linux Libertine}

\theoremstyle{plain}
\newtheorem{axiom}{Аксиома}
\newtheorem{lemma}{Лемма}

\theoremstyle{remark}
\newtheorem*{remark}{Примечание}
\newtheorem*{exercise}{Упражнение}
\newtheorem*{consequence}{Следствие}
\newtheorem*{example}{Пример}
\newtheorem*{observation}{Наблюдение}

\theoremstyle{definition}
\newtheorem{theorem}{Теорема}
\newtheorem*{definition}{Определение}
\newtheorem*{obozn}{Обозначение}

\setlength{\parindent}{0pt}

\newcommand{\dbltilde}[1]{\accentset{\approx}{#1}}
\newcommand{\intt}{\int\!}

% magical thing that fixes paragraphs
\makeatletter
\patchcmd{\CatchFBT@Fin@l}{\endlinechar\m@ne}{}
  {}{\typeout{Unsuccessful patch!}}
\makeatother

\newcommand{\get}[2]{
    \ExecuteMetaData[#1]{#2}
}

\newcommand{\getproof}[2]{
    \iftoggle{useproofs}{\ExecuteMetaData[#1]{#2proof}}{}
}

\newcommand{\getwithproof}[2]{
    \get{#1}{#2}
    \getproof{#1}{#2}
}

\newcommand{\import}[3]{
    \subsection{#1}
    \getwithproof{#2}{#3}
}

\newcommand{\given}[1]{
    Дано выше. (\ref{#1}, стр. \pageref{#1})
}

\renewcommand{\ker}{\text{Ker }}
\newcommand{\im}{\text{Im }}
\newcommand{\grad}{\text{grad}}

\lhead{Математический анализ}
\cfoot{}
\rfoot{9.11.2020}

\begin{document}

\subsection*{Интеграл локального потенциального векторного поля по непрерывному пути}

\begin{lemma}[о гусенице]\itemfix
    %<*огусенице>
    \begin{itemize}
        \item \(\gamma : [a, b] \to O \subset \R^m\) --- непр.
    \end{itemize}
    Тогда \(\exists \) дробление \(a = t_0 < t_1 < \dots < t_n = b\) и \(\exists \) шары \(B_1 \dots B_n \subset O : \gamma[t_{k - 1}, t_k] \subset B_k\).

    \begin{figure}[h]
        \centering
        \includesvg[scale=0.7]{images/огусенице.svg}
        \caption{``Гусеница'' --- покрытие пути шарами}
    \end{figure}
    %</огусенице>
\end{lemma}
%<*огусеницеproof>
\begin{proof}
    \(\forall c\in[a, b]\) возьмём \(B_c := B(\gamma(c), \underbrace{r_c}_{\text{произвольн.}}) \subset O\).

    \(\overline{\alpha_c} : = \inf\{ \alpha \in [a, b] : \gamma[\alpha, c] \subset B_c\}\)

    \(\overline{\beta_c} : = \inf\{ \alpha \in [a, b] : \gamma[c, \beta] \subset B_c\}\) --- момент первого выхода после посещения точки $\gamma(c)$

    Возьмём \((\alpha_c, \beta_c) : \overline \alpha_c < \alpha_c < c < \beta_c < \overline \beta_c\)

    Таким образом \(c \mapsto (\alpha_c, \beta_c)\) --- открытое покрытие $[a, b]$, если для \(c = a\) или \(c = b\) вместо \(\alpha_c, \beta_c\) брать \([a, \beta_a), (\alpha_b, b]\)

    \([a, b]\) --- компактно \( \implies [a, b] \subset \bigcup_{\text{кон.}} (\alpha_c, \beta_c)\)

    \? ни один интервал не накрывается целиком остальными \( \Leftrightarrow \forall (\alpha_c, \beta_c)\: \exists d_c\), принадлежащая ``только этому'' интервалу.

    \begin{figure}[h]
        \centering
        \includesvg[scale=0.7]{images/t_choice.svg}
        \caption{Выбор точек $t_k$}
    \end{figure}

    Точка \(t_k\) выбирается на \(d_k, d_{k + 1}\) и \(t_k \in (\alpha_k, \beta_k) \cap (\alpha_{k + 1}, \alpha_{k + 1})\).

    \(\gamma([t_{k - 1}, t_k]) = \gamma(\alpha_k, \beta_k) \subset B_k\)
\end{proof}
%</огусеницеproof>

\begin{remark}
    \(\forall \delta > 0 \) мы можем требовать, чтобы все \(r_k < \delta\)
\end{remark}
\begin{remark}
    В силу произвольности \(r_c\) можно требовать, чтобы шары \(B_c\) удовлетворяют некоторому локальному условию.

    Например пусть \(V\) --- локально потенциальное поле в \(O\). Мы можем требовать, чтобы во всех шарах существовал потенциал \(V\). Тогда будем называть \(\{B_k\}\) \(V\)-гусеницей.
\end{remark}

\begin{definition}\itemfix
    \begin{itemize}
        \item \(V\) --- локально потенциальное поле в \(O\subset\R^m\)
    \end{itemize}
    \(\gamma, \tilde \gamma : [a, b] \to O\) называются похожими \textit{(\(V\)-похожими)}, если у них есть общая \(V\)-гусеница:

    \(\exists t_0 = a < t_1 < t_2 < \dots < t_n = b\ \ \exists \) шары \( B_k \subset O :\)
    \[
        \gamma[t_{k - 1}, t_k] \subset B_k, \tilde \gamma [t_{k - 1}, t_k] \subset B_k
    \]
\end{definition}
\begin{corollary}\itemfix
    \begin{itemize}
        \item \(V\) --- локально потенциальное поле в \(O\subset\R^m\)
    \end{itemize}

    Тогда любой путь \(V\)-похож на ломаную:

    \begin{figure}[h]
        \centering
        \includesvg[scale=0.7]{images/ломаная.svg}
        \caption{Построение ломаной \textit{(розовая)} по пути \textit{(чёрный)} с помощью \(V\)-гусеницы \textit{(круги)}}.
    \end{figure}
\end{corollary}

\begin{lemma}[о равенстве интегралов локально-потенциальных векторных путей по похожим путям]\itemfix
    %<*оравенствеинтегралов>
    \label{о равенстве интегралов}
    \begin{itemize}
        \item \(V\) --- локально-потенциальное векторное поле в \(O\subset \R^m\)
        \item \(\gamma, \tilde \gamma : [a, b] \to O\) --- \(V\)-похожие, кусочно гладкие
        \item \(\gamma(a) = \tilde \gamma(a), \gamma(b) = \tilde \gamma(b)\)
    \end{itemize}
    Тогда \(\int_\gamma \sum V_i dx_i = \int_{\tilde \gamma} \sum V_i dx_i\)
    %</оравенствеинтегралов>
\end{lemma}
%<*оравенствеинтеграловproof>
\begin{proof}
    Рассмотрим общую \(V\)-гусеницу. Пусть \(f_k\) --- потенциал \(V\) в шаре \(B_k\), \(a = t_0 < t_1 < \dots < t_n = b\)

    Сдвинем потенциалы прибавлением константы, так что \(f_k(\gamma(t_k)) = f_{k + 1}(\gamma(t_k))\) при \(k = 1 \dots n\)

    Тогда
    \begin{align}
        \int_\gamma \sum_i V_i dx_i & = \sum \int_{[t_{k - 1}, t_k]} \dots   \nonumber                   \\
                                    & = \sum f_k(\gamma(t_k)) - f_k(\gamma(t_{k - 1})) \label{обобщ. НЛ} \\
                                    & = f_n(\gamma(b)) - f_1(\gamma(a))\nonumber
    \end{align}

    \ref{обобщ. НЛ}: По обобщенной формуле Ньютона-Лейбница.

    Для \(\tilde \gamma\) воспользуемся свойством: \(f_k\Big|_{B_k \cap B_{k + 1}} = f_{k + 1}\Big|_{B_k \cap B_{k + 1}}\) и тогда аналогично
    \[\int_{\tilde \gamma} \sum v_i dx_i = f_n(\tilde \gamma(b)) - f_1(\tilde \gamma(a))\]
\end{proof}
%</оравенствеинтеграловproof>

\begin{remark}
    Вместо условия ``\(\gamma(a) = \tilde \gamma(a), \gamma(b) = \tilde \gamma(b)\)'' можно взять условие: \(\gamma, \tilde \gamma\) --- петли. Тогда утверждение леммы тоже верно.
\end{remark}

\begin{lemma}\itemfix
    %<*опохожестиблизкихпутей>
    \label{лемма 3, лекция 9}
    \begin{itemize}
        \item \(\gamma : [a, b] \to O\) --- непр.
        \item \(V\) --- локально-потенциальное векторное поле в \(O\subset \R^m\)
    \end{itemize}
    Тогда \(\exists \delta > 0 :\) если \(\tilde \gamma, \dbltilde \gamma : [a, b] \to O\) таковы, что:
    \[
        \forall t \in [a, b] \ \ |\gamma(t) - \tilde\gamma(t)| < \delta, |\gamma(t) - \dbltilde\gamma(t)| < \delta
    \]
    Тогда \(\gamma, \tilde\gamma, \dbltilde\gamma\) \(V\)-похожи.
    %</опохожестиблизкихпутей>
\end{lemma}
%<*опохожестиблизкихпутейproof>
\begin{proof}
    Берём \(V\)-гусеницу для \(\gamma\).

    \(\delta_k\)-окрестность множества \(A\) \( : = \{x : \exists a\in A \ \ \rho(a, x) < \delta\} = \bigcap\limits_{a\in A} B(a, \delta)\)

    \begin{figure}[h]
        \centering
        \includesvg[scale=0.7]{images/deltakокрестность.svg}
        \caption{\(\delta_k\)-окрестность множества \(\gamma[t_{k - 1}, t_k]\)}
    \end{figure}

    \[
        \forall k \ \ \exists \delta_k > 0 : \left( \delta_k\text{-окрестность } \gamma[t_{k_1}, t_k] \right) \subset B_k
    \]

    Это следует из компактности:

    Пусть \(B_k = B(w, r)\), функция \(t \in [\gamma_{k - 1}m\, \gamma_k] \mapsto \rho(\gamma(t), w)\) непрерывна \( \Rightarrow \) достигается \(\max \), \(\rho(\gamma(t), w) \le r_0 < r\)

    \(\delta_k : = \frac{r - r_0}{2}, \delta : = \min(\delta_1 \dots \delta_k) \)
\end{proof}
%</опохожестиблизкихпутейproof>

\begin{definition}[Интеграл локального потенциального векторного поля \(V\) по непрерывному пути \(\gamma\)]
    %<*интеграллокальнопотенциальноговекторногопляпонепрерывномупути>
    Возьмём \(\delta > 0\) из леммы \ref{лемма 3, лекция 9}.

    Пусть \(\tilde \gamma\) --- \(\delta\)-близкий кусочно-гладкий путь, т.е. \(\forall t \ \ | \gamma(t) - \tilde \gamma(t) | < \delta \).

    Полагаем \(I(V, \gamma) : = I(V, \tilde \gamma)\).

    Корректность \textit{(нет произвольности)} следует из лемм \ref{лемма 3, лекция 9} и \ref{о равенстве интегралов}
    %</интеграллокальнопотенциальноговекторногопляпонепрерывномупути>
\end{definition}

\subsection*{Равномерная сходимость функциональных рядов \textit{(продолжение)}}

\begin{theorem}[признак Дирихле]\itemfix
    %<*признакдирихле>
    \begin{itemize}
        \item \(\sum a_n(x) b_n(x)\) --- вещественный ряд.
        \item \(x\in X\)
        \item Частичные суммы ряда \(\sum a_n\) равномерно ограничены :
              \[
                  \exists C_a \ \ \forall N \ \ \forall x\in X \ \ \left|\sum_{k = 1}^n a_k(x)\right| \le C_a
              \]
        \item \(\forall x\) последовательность \(b_n(x)\) --- монотонна по \(n\) и \(b_n(x)\xrightrightarrows[n\to +\infty]{X} 0\)
    \end{itemize}
    Тогда ряд \(\sum a_n(x)b_n(x)\) равномерно сходится на \(X\)
    %</признакдирихле>
\end{theorem}
%<*признакдирихлеproof>
\begin{proof}
    Преобразование Абеля \textit{(суммирование по частям)}
    \[\sum_{M \le k \le N} a_kb_k = A_N b_N - A_{M - 1} b_{M - 1} + \sum_{M \le k \le N - 1} A_k(b_k - b_{k+1})\]
    \begin{align}
        \left|\sum_{k = m}^N a_k(x)b_k(x)\right| & \le C_A |b_N| + C_A |b_{M - 1}| + \sum_{M \le k \le N - 1} C_A|b_k - b_{k + 1}|  \nonumber                       \\
                                                 & \le C_A\left( |b_N(x)| + |b_{M - 1}(x)| + \sum_{k = M}^{N - 1} |b_k - b_{k+1}| \right) \label{телескоп по знаку} \\
                                                 & \le C_A\left( 2|b_N(x)| + |b_{M - 1}(x)| + |b_M(x)| \right) \nonumber
    \end{align}

    \ref{телескоп по знаку} : Все разности одного знака \(\Rightarrow\) телескопически \( = \pm (b_M - b_N)\)

    \[
        \forall \varepsilon > 0 \ \ \exists K : \forall l > K \ \ \forall x\in X \ \ |b_l(x)| < \frac{\varepsilon}{4C_A}
    \]

    Значит, при \(M, N > K \ \ \forall x\in X\):
    \[\left|\sum_{k = m}^N a_k(x)b_k(x)\right| < \varepsilon\]

    Это критерий Больцано-Коши равномерной сходимости ряда.
\end{proof}
%</признакдирихлеproof>

\begin{example}
    \(f(x) = \sum\limits_{n = 1}^\infty \cfrac{\sin nx}{n^2} , x\in \R\)

    \begin{enumerate}
        \item \(f(x)\) --- непр. на \(\R\)

              \(\left|\cfrac{\sin nx}{n^2}\right| \le \cfrac{1}{n^2}, \sum \cfrac{1}{n^2}\) сходится. По признаку Вейерштрасса ряд равномерно сходится на \(\R\) \(\Rightarrow\) ряд \(f\) --- непр. на \(\R\)
        \item \(f\) --- дифф.?

              По теореме 3' \(\sum f'_n(x)\) --- ? равномерно сходится в окрестности \(x_0\). Если да, то $f\in C^1(V(x_0))$.

              $f' = \sum \cfrac{\cos nx}{n}$, но при $x=2\pi k$ $\sum$ расходится.

              Применим признак Дирихле для \(a_n = \cos nx, b_n = \frac{1}{n}, x\in[\varepsilon, 2\pi - \varepsilon]\)

              \begin{align*}
                  |\cos x + \cos 2n + \dots + \cos nx| & = |\Re(e^{ix} + e^{2ix} + .. + e^{nix})                \\
                                                       & \le \left|e^{ix} \frac{e^{nix} - 1}{e^{ix} - 1}\right| \\
                                                       & = |e^{ix}| \frac{|e^{nix} - 1|}{|e^{ix} - 1|}          \\
                                                       & \le \frac{2}{|e^{ix} - 1|}                             \\
                                                       & \le \frac{2}{e^{i\varepsilon} - 1}                     \\
                                                       & = : C_A
              \end{align*}

              \(b_n\) --- монотонно, \(\to 0\), не зависит от \(x\), поэтому \(\rightrightarrows 0\)

              Таким образом, призак Дирихле сработал и \(f'(x) = \sum \frac{\cos nx}{n} \) при \(x \in (0, 2\pi)\)
    \end{enumerate}
\end{example}

\begin{exercise}
    \begin{enumerate}
        \item При \(x = 2\pi k\) \(f(x)\) не дифференцируемая.
        \item Существует ли \(f''(x)\) при \(x\in(0, 2\pi)\)?
        \item Если $q(x) = \sum\frac{\sin nx}{\sqrt n}, x\in(0, 2\pi)$: \begin{enumerate}
                  \item Непрерывна?
                  \item Дифференцируема?
              \end{enumerate}
    \end{enumerate}
\end{exercise}

\section*{Степенные ряды}

\(B(r_0, r)\subset \C\) --- открытый круг

\begin{definition}
    %<*степеннойряд>
    Степенной ряд: \(\sum\limits_{n = 0}^{ +\infty } a_n(z - z_0)^n\), где \(z_0\in \C, a_n\in C, z\) --- переменная \(\in C\)
    %</степеннойряд>
\end{definition}

\begin{theorem}[о круге сходимости степенных рядов]\itemfix
    %<*кругстепенногоряда>
    \begin{itemize}
        \item \(\sum a_n(z - z_0)^n\) --- степенной ряд
    \end{itemize}
    Тогда выполняется ровно один из трех случаев:
    \begin{enumerate}
        \item Ряд сходится при всех \(z\in C\)
        \item Ряд сходится только при \(z = z_0\)
        \item \(\exists R\in(0, +\infty)\) :
              \begin{enumerate}
                  \item при \(|z - z_0|< R\) ряд абсолютно сходится
                  \item при \(|z - z_0|> R\) ряд расходится
              \end{enumerate}
    \end{enumerate}
    %</кругстепенногоряда>
\end{theorem}

\begin{remark}
    Ряд не может никогда не сходиться, т.к. при \(z = z_0\) ряд\( = a_0 + 0 + 0 + \dots = a_0\).
\end{remark}

%<*кругстепенногорядаproof>
\begin{proof}
    Применим признак Коши: \(\overline{\lim}\sqrt[n]{|a_n|} = r\), если \(r < 1\), ряд сходится, если \(r > 1\), ряд расходится.

    \[
        \overline{\lim} \sqrt[n]{ |a_n| \cdot |z - z_0|^n } = \overline{\lim} \sqrt[n]{|a_n|}|z - z_0| = |z - z_0|\overline{\lim} \sqrt[n]{|a_n|}
    \]
    \begin{enumerate}
        \item \(\overline{\lim} = 0\). Тогда \(r = 0\), есть абсолютная сходимость при всех \(z\).
        \item \(\overline{\lim} = +\infty\). Тогда \(r = +\infty\) при \(z\neq z_0\). При \(z = z_0\) сходимость очевидна.
        \item \(\overline{\lim} \neq 0, +\infty\). Тогда \(|z - z_0|\overline{\lim} \sqrt[n]{|a_n|} < 1 \Leftrightarrow |z - z_0| < \frac{1}{\overline \lim \sqrt[n]{|a_n|}} \defeq R\)
    \end{enumerate}
\end{proof}
%</кругстепенногорядаproof>

\begin{definition}
    %<*формулаадамара>
    \(\sum a_n(z - z_0)^n\), тогда число \(R = \frac{1}{\overline \lim_n \sqrt[n]{|a_n|}}\). Это \textbf{формула Адамара}.
    %</формулаадамара>
\end{definition}

\end{document}
