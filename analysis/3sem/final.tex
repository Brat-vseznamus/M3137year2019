\documentclass[12pt, a4paper]{article}

\usepackage{lastpage}
\usepackage{mathtools}
\usepackage{xltxtra}
\usepackage{libertine}
\usepackage{amsmath}
\usepackage{amsthm}
\usepackage{amsfonts}
\usepackage{amssymb}
\usepackage{enumitem}
\usepackage{xcolor}
\usepackage[left=1.5cm, right=1.5cm, top=2cm, bottom=2cm, bindingoffset=0cm, headheight=15pt]{geometry}
\usepackage{fancyhdr}
\usepackage[russian]{babel}
% \usepackage[utf8]{inputenc}
\usepackage{catchfilebetweentags}
\usepackage{accents}
\usepackage{calc}
\usepackage{etoolbox}
\usepackage{mathrsfs}
\usepackage{wrapfig}

\providetoggle{useproofs}
\settoggle{useproofs}{false}

\pagestyle{fancy}
\lfoot{M3137y2019}
\rhead{\thepage\ из \pageref{LastPage}}

\newcommand{\R}{\mathbb{R}}
\newcommand{\Q}{\mathbb{Q}}
\newcommand{\C}{\mathbb{C}}
\newcommand{\Z}{\mathbb{Z}}
\newcommand{\B}{\mathbb{B}}
\newcommand{\N}{\mathbb{N}}

\newcommand{\const}{\text{const}}

\newcommand{\teormin}{\textcolor{red}{!}\ }

\DeclareMathOperator*{\xor}{\oplus}
\DeclareMathOperator*{\equ}{\sim}
\DeclareMathOperator{\Ln}{\text{Ln}}
\DeclareMathOperator{\sign}{\text{sign}}
\DeclareMathOperator{\Sym}{\text{Sym}}
\DeclareMathOperator{\Asym}{\text{Asym}}
% \DeclareMathOperator{\sh}{\text{sh}}
% \DeclareMathOperator{\tg}{\text{tg}}
% \DeclareMathOperator{\arctg}{\text{arctg}}
% \DeclareMathOperator{\ch}{\text{ch}}

\DeclarePairedDelimiter{\ceil}{\lceil}{\rceil}
\DeclarePairedDelimiter{\abs}{\left\lvert}{\right\rvert}

\setmainfont{Linux Libertine}

\theoremstyle{plain}
\newtheorem{axiom}{Аксиома}
\newtheorem{lemma}{Лемма}

\theoremstyle{remark}
\newtheorem*{remark}{Примечание}
\newtheorem*{exercise}{Упражнение}
\newtheorem*{consequence}{Следствие}
\newtheorem*{example}{Пример}
\newtheorem*{observation}{Наблюдение}

\theoremstyle{definition}
\newtheorem{theorem}{Теорема}
\newtheorem*{definition}{Определение}
\newtheorem*{obozn}{Обозначение}

\setlength{\parindent}{0pt}

\newcommand{\dbltilde}[1]{\accentset{\approx}{#1}}
\newcommand{\intt}{\int\!}

% magical thing that fixes paragraphs
\makeatletter
\patchcmd{\CatchFBT@Fin@l}{\endlinechar\m@ne}{}
  {}{\typeout{Unsuccessful patch!}}
\makeatother

\newcommand{\get}[2]{
    \ExecuteMetaData[#1]{#2}
}

\newcommand{\getproof}[2]{
    \iftoggle{useproofs}{\ExecuteMetaData[#1]{#2proof}}{}
}

\newcommand{\getwithproof}[2]{
    \get{#1}{#2}
    \getproof{#1}{#2}
}

\newcommand{\import}[3]{
    \subsection{#1}
    \getwithproof{#2}{#3}
}

\newcommand{\given}[1]{
    Дано выше. (\ref{#1}, стр. \pageref{#1})
}

\renewcommand{\ker}{\text{Ker }}
\newcommand{\im}{\text{Im }}
\newcommand{\grad}{\text{grad}}

\usepackage{sectsty}

\allsectionsfont{\raggedright}
\subsectionfont{\fontsize{14}{15}\selectfont}

\lhead{Итоговый конспект}
\cfoot{}
\rfoot{}

\settoggle{useproofs}{true}

\begin{document}

\section{Определения}

\import{Мультииндекс и обозначения с ним}{1.tex}{мультииндекс}

\import{\teormin Формула Тейлора (различные виды записи)}{1.tex}{формулатейлорадифференциал}
\get{1.tex}{формулатейлорамультииндекс}

\import{$n$-й дифференциал}{1.tex}{дифференциал}

\import{\teormin Норма линейного оператора}{1.tex}{нормалоп}

\import{Положительно-, отрицательно-, незнако- определенная квадратичная форма}{2.tex}{формы}

\import{Локальный максимум, минимум, экстремум}{2.tex}{экстремум}

\import{Диффеоморфизм}{3.tex}{диффеоморфизм}

\subsection{Формулировка теоремы о локальной обратимости}
\get{3.tex}{олокальнойобратимости}

\subsection{Формулировка теоремы о локальной обратимости в терминах систем уравнений}
\get{3.tex}{олокальнойобратимостисистема}

\import{Формулировка теоремы о неявном отображении в терминах систем уравнений}{4.tex}{онеявномотображениивтерминахсистемы}

\import{\teormin Простое $k$-мерное гладкое многообразие в $\R^m$}{4.tex}{простоеkмерноегладкоемногообразие}

\import{Касательное пространство к $k$-мерному многообразию в $\R^m$}{5.tex}{касательноепространство}

\import{Относительный локальный максимум, минимум, экстремум}{5.tex}{локальныйминимум}

\import{\teormin Формулировка достаточного условия относительного экстремума}{6.tex}{достаточноеусловиеэкстремума}

\import{Поточечная сходимость последовательности функций на множестве }{5.tex}{поточеченаясходимость}

\import{\teormin Равномерная сходимость последовательности функций на множестве }{5.tex}{равномернаясходимость}

\import{Равномерная сходимость функционального ряда}{6.tex}{равномернаясходимостьряда}

\import{Формулировка критерия Больцано-Коши для равномерной сходимости}{6.tex}{критерийбольцанокоши}

\import{\teormin Степенной ряд, радиус сходимости степенного ряда, формула Адамара}{9.tex}{}
\get{9.tex}{степеннойряд}
\get{9.tex}{формулаадамара}


\import{Признак Абеля равномерной сходимости функционального ряда}{9.tex}{}
\?

\import{}{9.tex}{}\?
\import{}{9.tex}{}\?

\import{Кусочно-гладкий путь}{7.tex}{путь}
\get{7.tex}{кусочногладкоеотображение}

\import{Векторное поле}{7.tex}{векторноеполе}

\import{Интеграл векторного поля по кусочно-гладкому пути}{7.tex}{интегралвекторногополяпокусочногладкомупути}

\import{\teormin Потенциал, потенциальное векторное поле}{7.tex}{потенциальноевекторноеполе}

\import{Локально потенциальное векторное поле}{8.tex}{локальнопотенциальноеполе}

\import{\teormin Интеграл локально-потенциального векторного поля по произвольному пути}{9.tex}{интеграллокальнопотенциальноговекторногопляпонепрерывномупути}

\import{Гомотопия путей связанная и петельная}{10.tex}{гомотопия}

\import{Односвязная область}{10.tex}{односвязнаяобласть}

\section{Теоремы}

\import{Лемма о дифференцировании ``сдвига''}{1.tex}{леммаодиффсдвига}

\import{\teormin Многомерная формула Тейлора (с остатком в форме Лагранжа и Пеано)}{1.tex}{}
\subsubsection{В форме Лагранжа}
\getwithproof{1.tex}{тейлорлагранж}
\subsubsection{В форме Пеано}
\getwithproof{1.tex}{тейлорпеано}

\import{Теорема о пространстве линейных отображений}{1.tex}{опространствелинейныхотображений}

\import{Лемма об условиях, эквивалентных непрерывности линейного оператора}{1.tex}{эквивалентностьнепрерывности}

\import{Теорема Лагранжа для отображений}{2.tex}{лагранжадляотображений}

\import{Теорема об обратимости линейного отображения, близкого к обратимому}{2.tex}{обобратимости}

\import{Теорема о непрерывно дифференцируемых отображениях}{2.tex}{онепрдифф}

\import{Теорема Ферма. Необходимое условие экстремума. Теорема Ролля}{2.tex}{ферма}
\get{2.tex}{необходимоеусловиеэкстремума}
\getwithproof{2.tex}{ролля}

\import{Лемма об оценке квадратичной формы и об эквивалентных нормах}{2.tex}{леммаобоценкенормы}

\import{\teormin Достаточное условие экстремума}{2.tex}{достаточноеусловиеэкстремума}

\import{Лемма о ``почти локальной инъективности''}{3.tex}{опочтилокальнойиньективности}

\import{Теорема о сохранении области}{3.tex}{осохраненииобласти}

\import{Следствие о сохранении области для отображений в пространство меньшей размерности}{3.tex}{следствиеосохраненииобласти}

\import{Теорема о гладкости обратного отображения}{3.tex}{огладкостиобратногоотображения}

\import{\teormin Теорема о неявном отображении}{4.tex}{онеявномотображении}

\import{Теорема о задании гладкого многообразия системой уравнений}{4.tex}{озаданиигладкогомногообразиясистемойуравнений}

\import{Следствие о двух параметризациях}{4.tex}{одвухпараметризациях}

\import{Лемма о корректности определения касательного пространства}{5.tex}{касательноепространство}
\get{5.tex}{касательноепространство-proof}

\import{Касательное пространство в терминах векторов скорости гладких путей}{5.tex}{касательноепространствовтерминахвекторовскоростигладкихпутей}

\import{Касательное пространство к графику функции и к поверхности уровня}{5.tex}{касательноепространствокграфику}
\getwithproof{5.tex}{линииуровня}

\import{\teormin Необходимое условие относительного локального экстремума}{5.tex}{необходимоеусловиеотносительногоэкстремума}

\import{Вычисление нормы линейного оператора с помощью собственных чисел}{5.tex}{нормалоп}

\import{\teormin Теорема Стокса--Зайдля о непрерывности предельной функций. Следствие для рядов}{6.tex}{стоксазайдля}

\import{Метрика в пространстве непрерывных функций на компакте, его полнота}{6.tex}{метрикавcx}

\import{Теорема о предельном переходе под знаком интеграла. Следствие для рядов}{6.tex}{предельныйпереходподинтегралом}

\import{Правило Лейбница дифференцирования интеграла по параметру}{6.tex}{правилолейбница}

\import{Теорема о предельном переходе под знаком производной. Дифференцирование функционального ряда}{6.tex}{предельныйпереходподпроизводной}

\textcolor{red}{Дифференцирование функционального ряда?}

\import{\teormin Признак Вейерштрасса равномерной сходимости функционального ряда}{7.tex}{признаквейерштрасса}

\import{Дифференцируемость $\Gamma$ функции}{8.tex}{дифференцируемостьгаммафункции}

\import{Теорема о предельном переходе в суммах}{8.tex}{определьномпереходевсуммах}

\import{Теорема о перестановке двух предельных переходов}{8.tex}{оперестановкедвухпредельныхпереходов}

\import{Признак Дирихле равномерной сходимости функционального ряда}{9.tex}{признакдирихле}

\import{Теорема о круге сходимости степенного ряда}{9.tex}{кругстепенногоряда}

\import{Теорема о непрерывности степенного ряда}{10.tex}{оравномернойсходимостиинепрерывностистепенногоряда}

\import{Теорема о дифференцировании степенного ряда. Следствие об интегрировании. Пример.}{10.tex}{одифференцируемостистепенногоряда}

\textcolor{red}{Следствие об интегрировании?}

\textcolor{red}{Пример?}

\import{}{9.tex}{}\?
\import{}{9.tex}{}\?
\import{}{9.tex}{}\?

\import{Простейшие свойства интеграла векторного поля по кусочно-гладкому пути}{7.tex}{свойства}

\import{\teormin Обобщенная формула Ньютона--Лейбница}{7.tex}{обобщеннаяформуланьютоналейбница}

\import{Характеризация потенциальных векторных полей в терминах интегралов}{8.tex}{характеризацияпотенциальныхвекторныхполейвтерминахинтегралов}

\import{\teormin Необходимое условие потенциальности гладкого поля. Лемма Пуанкаре}{8.tex}{необходимоеусловиепотенциальности}

Лемма Пуанкаре:
\getwithproof{8.tex}{леммапуанкаре}
Лемма Пуанкаре о локальной потенциальности:
\get{8.tex}{леммапуанкаре2}

\import{Лемма о гусенице}{9.tex}{огусенице}

\import{Лемма о равенстве интегралов по похожим путям}{9.tex}{оравенствеинтегралов}

\import{Лемма о похожести путей, близких к данному}{9.tex}{опохожестиблизкихпутей}

\import{Равенство интегралов по гомотопным путям}{10.tex}{интегралпосвязанногомотопнымпутям}

\import{Теорема Пуанкаре для односвязной области}{10.tex}{леммапуанкареводносвязнойобласти}

\import{Теорема о веревочке}{10.tex}{оверевочке}

\end{document}