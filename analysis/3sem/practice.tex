\documentclass[12pt, a4paper]{article}

\usepackage{lastpage}
\usepackage{mathtools}
\usepackage{xltxtra}
\usepackage{libertine}
\usepackage{amsmath}
\usepackage{amsthm}
\usepackage{amsfonts}
\usepackage{amssymb}
\usepackage{enumitem}
\usepackage{xcolor}
\usepackage[left=1.5cm, right=1.5cm, top=2cm, bottom=2cm, bindingoffset=0cm, headheight=15pt]{geometry}
\usepackage{fancyhdr}
\usepackage[russian]{babel}
% \usepackage[utf8]{inputenc}
\usepackage{catchfilebetweentags}
\usepackage{accents}
\usepackage{calc}
\usepackage{etoolbox}
\usepackage{mathrsfs}
\usepackage{wrapfig}

\providetoggle{useproofs}
\settoggle{useproofs}{false}

\pagestyle{fancy}
\lfoot{M3137y2019}
\rhead{\thepage\ из \pageref{LastPage}}

\newcommand{\R}{\mathbb{R}}
\newcommand{\Q}{\mathbb{Q}}
\newcommand{\C}{\mathbb{C}}
\newcommand{\Z}{\mathbb{Z}}
\newcommand{\B}{\mathbb{B}}
\newcommand{\N}{\mathbb{N}}

\newcommand{\const}{\text{const}}

\newcommand{\teormin}{\textcolor{red}{!}\ }

\DeclareMathOperator*{\xor}{\oplus}
\DeclareMathOperator*{\equ}{\sim}
\DeclareMathOperator{\Ln}{\text{Ln}}
\DeclareMathOperator{\sign}{\text{sign}}
\DeclareMathOperator{\Sym}{\text{Sym}}
\DeclareMathOperator{\Asym}{\text{Asym}}
% \DeclareMathOperator{\sh}{\text{sh}}
% \DeclareMathOperator{\tg}{\text{tg}}
% \DeclareMathOperator{\arctg}{\text{arctg}}
% \DeclareMathOperator{\ch}{\text{ch}}

\DeclarePairedDelimiter{\ceil}{\lceil}{\rceil}
\DeclarePairedDelimiter{\abs}{\left\lvert}{\right\rvert}

\setmainfont{Linux Libertine}

\theoremstyle{plain}
\newtheorem{axiom}{Аксиома}
\newtheorem{lemma}{Лемма}

\theoremstyle{remark}
\newtheorem*{remark}{Примечание}
\newtheorem*{exercise}{Упражнение}
\newtheorem*{consequence}{Следствие}
\newtheorem*{example}{Пример}
\newtheorem*{observation}{Наблюдение}

\theoremstyle{definition}
\newtheorem{theorem}{Теорема}
\newtheorem*{definition}{Определение}
\newtheorem*{obozn}{Обозначение}

\setlength{\parindent}{0pt}

\newcommand{\dbltilde}[1]{\accentset{\approx}{#1}}
\newcommand{\intt}{\int\!}

% magical thing that fixes paragraphs
\makeatletter
\patchcmd{\CatchFBT@Fin@l}{\endlinechar\m@ne}{}
  {}{\typeout{Unsuccessful patch!}}
\makeatother

\newcommand{\get}[2]{
    \ExecuteMetaData[#1]{#2}
}

\newcommand{\getproof}[2]{
    \iftoggle{useproofs}{\ExecuteMetaData[#1]{#2proof}}{}
}

\newcommand{\getwithproof}[2]{
    \get{#1}{#2}
    \getproof{#1}{#2}
}

\newcommand{\import}[3]{
    \subsection{#1}
    \getwithproof{#2}{#3}
}

\newcommand{\given}[1]{
    Дано выше. (\ref{#1}, стр. \pageref{#1})
}

\renewcommand{\ker}{\text{Ker }}
\newcommand{\im}{\text{Im }}
\newcommand{\grad}{\text{grad}}

\lhead{Практика}
\cfoot{}
\rfoot{}

\begin{document}

\section*{Практика 8}

\subsection*{Равномерная сходимость последовательности функций}

\begin{definition}
    \(f_n \rightrightarrows f\) на \(E\), если \(\rho(f_n, f) \to 0\), где \(\rho(f_n, f) := \sup\limits_{x\in E} |f_n(x) - f(x)|\)
\end{definition}

\begin{remark}
    Есть более простой признак: \(f_n(x) \rightrightarrows f(x) \implies \forall x\in E \ \ f_n(x) \to f(x)\)
\end{remark}

\begin{exercise}[Демидович, 2749]
    \(f_n(x) = \frac{1}{x + n}, E = (0, +\infty)\). Есть ли равномерная сходимость?

    \begin{enumerate}
        \item Ищем кандидата на роль \(f\).

              При фиксированном \(x\) посчитаем \(\lim\limits_{n\to + \infty} f_n(x)\):
              \[\lim\limits_{n\to + \infty} f_n(x) = \lim\limits_{n\to +\infty} \frac{1}{x + n} = 0\]

              Таким образом, \(f(x)\equiv 0\)
        \item Проверяем равномерную сходимость.
              \[\rho(f_n, f) = \sup_{x\in(0, +\infty)} \left|\frac{1}{x + n} - 0\right| = \sup_{x\in(0, +\infty)} \frac{1}{x + n} = \frac{1}{n} \to 0\]
    \end{enumerate}

    \textbf{Ответ}: равномерная сходимость есть.
\end{exercise}

\begin{exercise}
    \[f_n(x) = \frac{n^2x + x^2n + 20}{n + nx + n^2x^2 + 1} \quad x\in(0, +\infty) \]
    \begin{enumerate}
        \item Ищем \(f\).
              \[f(x) = \lim_{n\to +\infty} f_n(x) = \frac{1}{x}\]
              Этот предел нашелся заменой на эквивалентную в числителе и знаменателе
        \item \(\rho(f_n, f)\)
              \[f_n - f = \frac{n^2x^2 + x^3n + 20x - n - nx - n^2x^2 - 1}{x(n + nx + n^2x^2 + 1)}\]
              \[\sup_{x\in(0, +\infty)} \frac{|x^3n + 20x - n - nx - 1|}{x(n + nx + n^2x^2 + 1)} = ?\]
              Такой супремум сложно берется, если вообще берется. Но можно попробовать его оценить сверху. Заметим, что у нас \(n\) фиксировано, а \(x\) мы двигаем. Если двигать \(x\to 0\), то числитель ненулевая константа, а знаменатель произвольно мал. Таким образом, \(\sup = +\infty\) и равномерной сходимости нет.

              Что произойдет, если в условии \(x\in[1, +\infty)\)? Подставим \(x = 1\), тогда получается \(\cfrac{1}{n^2}\). Если \(x = n\), то дробь \(\cfrac{\approx n^4}{n\cdot n^4} \approx \cfrac{1}{n}\). Пока что никакие \(x\) не дают большие значения. Если взять \(x = n^2\), получается \(\frac{n^7}{n^8}\).

              Пусть \(n > 20\). Тогда
              \[\sup \dots \le \frac{5x^3 n}{x^3 n^2} \le \frac{5}{n} \to 0\]
    \end{enumerate}
\end{exercise}

\section*{Практика 9}

Когда оцениваем \(\sup\), можно говорить, что \(\sup \geq \lim_{x\to A}\), где \(A\) --- произвольная константа \textit{(к которой можно стремиться в \(E\) (предельная точка))}.

\[\ctg x \stackrel{x\to 0}\approx \frac{1}{x}\]
\[\arctg x \stackrel{x\to +\infty}{\approx} \frac{\pi}{2} - \frac{1}{x}\]
Таким образом, \(\arctg x \equ_{x\to +\infty} \frac{\pi}{2} - x\), это проверяется вычислением предела \(\lim\limits_{x\to +\infty} \cfrac{\frac{\pi}{2} -\arctg x}{\frac{1}{x}}\) по правилу Лопиталя.

\subsection*{Равномерная сходимость рядов}

\[\sum_{n = 1}^{ +\infty} u_n(x) = S(x) \quad S_N \rightrightarrows S \text{ на } E\]

Рассмотрим простой случай, где сходимость доказывается по определению.

\begin{exercise}[2769]
    \[\sum_{n = 0}^{ +\infty} x^n(1 - x) \quad x\in[0, 1]\]
    \[S_N = \sum_{n = 0}^N x^n - x^{n + 1} \stackrel{\text{телескоп}}{=} 1-x^{N + 1} \xrightarrow{N\to +\infty} \begin{cases}
            1, & x < 1 \\
            0, & x = 1
        \end{cases}\]
    \[\sup_{x\in[0, 1]} |S_N - S| = \sup_{x\in[0, 1)} x^{N + 1} = 1 \not\to 0 \]
    Мы игнорируем \(x = 1\), т.к. там \(\sup = 0\)
\end{exercise}

\begin{exercise}[2771]
    \[\sum_{n = 1}^{\infty} \frac{x}{((n - 1)x + 1)(nx + 1)} \quad x\in (0, +\infty)\]
    \[S_N = \sum_{n = 1}^{N} \frac{x}{((n - 1)x + 1)(nx + 1)} = \sum_{n = 1}^N - \frac{1}{nx + 1} + \frac{1}{(n - 1)x + 1} \stackrel{\text{телескоп}}{ =} 1 - \frac{1}{Nx + 1} \]
    \[\lim S_N = 1\]
    \[\sup_x|S_N - S| = \sup_x \frac{1}{Nx + 1} \geq 1 \not\to 0\]
\end{exercise}

\textbf{Критерий Больцано-Коши} обычно доказывает отсутствие равномерной сходимости, хотя его можно использовать и для обратного.
\[\exists \varepsilon > 0 \ \ \forall N \ \ \exists n > N, \exists m\in \N, \exists x \quad  |u_{n + 1}(x) + \dots + u_{n + m}(x)| > \varepsilon\]
Тогда равномерной сходимости нет.

Докажем предыдущий номер по критерию Больцано-Коши.

\[\exists \varepsilon = \frac{1}{100}  \ \ \forall N \ \ \exists n > N, \exists m = n, \exists x = \frac{1}{n} \quad \sum_{k = n + 1}^{2n} \frac{1}{n} \frac{1}{(\frac{k}{n} + 1)(\frac{k + 1}{n} + 1)} \geq \frac{n}{n} \frac{1}{3\cdot 4} > \frac{1}{100}\]

\textbf{Признак Вейерштрасса}

\(\sum u_n(x), x\in E\):
\begin{enumerate}
    \item \(\forall x\in E : |u_n(x)| \leq C_n\)
    \item \(\sum C_n\) --- сходится
\end{enumerate}
Тогда ряд равномерно сходится.

\begin{example}
    \(x\in [0, \frac{1}{2}], \sum x^n(1 - x)\)
    \[|x^n(1 - x)| \leq \frac{1}{2^n}, \sum \frac{1}{2^n} \text{ сходится}\]
\end{example}

\begin{exercise}[2774]\itemfix
    \begin{itemize}
        \item [(a)] \[f_n(x) = \frac{1}{x^2 + n^2} \quad E = \R\]
              \[\left|\frac{1}{x^2 + n^2}\right| \leq \frac{1}{n^2}, \sum \frac{1}{n^2} \text{ сходится}\]
              \textbf{Ответ}: сходится равномерно
        \item [(a')] \[f_n(x) = \frac{x}{x^2 + n^2} \quad E = \R\]
              \[\left(\frac{x}{x^2 + n^2}\right)' = \frac{x^2 + n^2 - 2x^2}{(x^2 + n^2)^2} \]
              Таким образом, максимум достигается при \(x = n\).
              \[C_n: = \max f_n = \frac{n}{n^2 + n^2} = \frac{1}{2n} \quad \sum \frac{1}{n} \text{ расходится}\]
              Применим критерий Коши.
              \[\exists \varepsilon = \frac{1}{100}  \ \ \forall N \ \ \exists n > N, \exists m = n, \exists x = n\]
              \[\frac{n}{n^2 + (n + 1)^2} + \frac{n}{n^2 + (n + 2)^2} + \dots + \frac{n}{n^2 + (2n)^2} \ge \frac{n}{5n} = \frac{1}{5} > \varepsilon\]


        \item [(в)] \[\sum \frac{x}{1 + n^4x^2} \quad x\in (0, +\infty)\]
              \[C_n: = \sup_{x\in E} \frac{x}{1 + n^4x^2} \stackrel{\frac{1}{n^2}}= \frac{\frac{1}{n}}{1 + n^4 \frac{1}{n^4}} = \frac{1}{2n^2}\]
    \end{itemize}
\end{exercise}

\section*{Практика 10}

\textbf{Признак Дирихле} для \(\sum a_n(x)b_n(x)\):
\begin{enumerate}
    \item Частичные суммы \(\sum a_n\) равномерно ограничены:
          \[\exists C_a \ \ \forall N \ \ \forall x\in E \ \ \left|\sum_{k = 1}^N a_k(x)\right| \leq C_a\]
    \item \begin{enumerate}
              \item При фиксированном \(x\) функция \(b_n(x)\) монотонна по \(n\)
              \item \(b_n(x) \rightrightarrows 0\) на \(E\) при \(n\to +\infty\)
          \end{enumerate}
\end{enumerate}

\textbf{Признак Абеля} для \(\sum a_n(x)b_n(x)\):
\begin{enumerate}
    \item \(\sum a_n(x)\) равномерно сходится на \(E\)
    \item \begin{enumerate}
              \item \(b_n(x)\) монотонно по \(n\)
              \item \(b_n(x)\) равномерно ограничено:
                    \[\exists C_b \ \ \forall N \ \ \forall x\in E \ \ |b_n(x)| \leq C_b\]
          \end{enumerate}
\end{enumerate}

\begin{exercise}
    \[\sum \frac{\sin \frac{\pi n}{6} }{n + \sin x} \quad x\in \R \]
    \[a_n(x) : = \sin \frac{\pi n}{6} \quad b_n(x) : = \frac{1}{n + \sin x}\]
    \(a_n(x)\) ограничена, т.к. за 12 шагов мы обходим всю окружность и приходим назад в \(0\). Таким образом, \(\left|\sum_{n = 1}^N a_n\right| \leq 11\)

    Монотонность \(b_n\) тривиальна, равномерная сходимость к нулю тоже.
\end{exercise}

\begin{exercise}
    \[\sum \frac{( - 1)^{\frac{n(n - 1)}{2} }}{\sqrt[3]{n^2 + e^x}} \quad |x| \leq 10 \]
    \[a_n = ( - 1)^{\frac{n(n - 1)}{2} } \quad b_n = \frac{1}{\sqrt[3]{n^2 + e^x}}\]
    \(a_n\) имеет вид \(1, - 1, - 1, 1, 1, - 1, - 1, 1 \dots \). Тогда \(\left|\sum a_n\right| \leq 4\)

    \(b_n\) очевидно.
\end{exercise}

\begin{exercise}
    \[\sum \frac{\sin nx}{\sqrt{n} + \cos nx} \quad x\in\R\]
    \begin{enumerate}
        \item \(x\in[1, 2]\)

              Заметим, что \(\sum \frac{\sin nx}{\sqrt{n}} \) --- равномерно сходится по Дирихле, при \(a_n = \sin nx\). Рассмотрим разность исходного ряда и этого ряда:

              \[\frac{1}{2}\sum \frac{\sin 2nx}{\sqrt{n}(\sqrt{n} + \cos nx)} \]
              Аналогично рассмотрим \(\sum \frac{\sin 2nx}{n} \). Так можно продолжить несколько итераций и мы получим сходимость или расходимость.
        \item \(x\in\R\)

              Равномерной сходимости нет по критерию Коши:
              \[\exists \varepsilon > 0 \ \ \forall N \ \ \exists n > N, \exists m = n, \exists x = \frac{1}{n} \quad  \left|\frac{\sin \frac{n + 1}{n} }{\sqrt{n + 1} + \cos \frac{n + 1}{n}} + \dots + \frac{\sin 2}{\sqrt{2n} + \cos 2}  \right| > \frac{C}{\sqrt{n}} \cdot n > \varepsilon\]
    \end{enumerate}
\end{exercise}

\begin{exercise}[Кудрявцев, том 2, параграф 18, задача 13]\itemfix
    \begin{enumerate}
        \item \[\sum \frac{x^2 \sin(n \sqrt{x})}{1 + n^3x^4} \quad E = [0, +\infty)\]
              Найдём экстремум члена суммы:
              \[\left(\frac{x^2 \sin(n \sqrt{x})}{1 + n^3x^4}\right)'_x = \frac{(1 + n^3x^4)(2x \sin(n \sqrt{x}) + x^2 \cos (n \sqrt{x}) \frac{1}{2 \sqrt{x}}) - x^2 \sin(n \sqrt{x})\cdot 4n^3x^3}{(1 + n^3x^4)^2} \]
              Не находится.

              Найдём экстремум следующего:
              \[\left|\frac{x^2 \sin(n \sqrt{x})}{1 + n^3x^4}\right| \leq \frac{x^2}{1 + n^3x^4}\]
              \[\left( \frac{x^2}{1 + n^3x^4} \right)' = \frac{2x (1 + n^3x^4) - x^2n^3\cdot 4x}{\dots }\]
              \[2x (1 + n^3x^4) - x^2n^3\cdot 4x = 0\]
              \[x = \frac{1}{n^{\frac{3}{4}}}\]
              Таким образом:
              \[\left|\frac{x^2 \sin(n \sqrt{x})}{1 + n^3x^4}\right| \leq \frac{1}{2n^{\frac{3}{2}}} \text{ сходится}\]
              \textbf{Ответ}: равномерно сходится по Вейерштрассу
        \item \[\sum \frac{n^2x}{(n^2 + 1)(1 + n^4x^2)\arctg(1 + x)} \quad E = (0, +\infty)\]
              Вспомним волшебное школьное неравенство:
              \[\frac{t}{1 + t^2} \leq \frac{1}{2}\]
              Пусть \(t = n^2x\)
              \[\left|\frac{n^2x}{(n^2 + 1)(1 + n^4x^2)\arctg(1 + x)}\right| \leq \frac{1}{2(n^2 + 1)\arctg 1} \sim \frac{C}{n^2}\]
              \textbf{Ответ}: равномерно сходится по Вейерштрассу
        \item \[\sum \frac{x\sin(x + n)}{n^2x^2 + n + 1} \quad E = [0, +\infty)\]
              \[\left|\frac{x\sin(x + n)}{n^2x^2 + n + 1}\right| \leq \frac{x}{n^2x^2 + n + 1}\]
              Экстремум \(x =\sqrt{\frac{n + 1}{2n^2}} \)
              \[\frac{x}{n^2x^2 + n + 1} \leq \frac{\sqrt{\frac{n + 1}{2n^2}}}{1.5(n + 1)} \leq \frac{\frac{10}{\sqrt{n}}}{n + 1} \text{ сходится}  \]
              \textbf{Ответ}: равномерно сходится по Вейерштрассу
        \item \[\sum \frac{xe^{ - x^2n}}{\sqrt{n\ln^3(n + 1)}} \quad E = \R\]
              \[\left|xe^{ - x^2n}\right| \leq \frac{C}{\sqrt{n}}\]
              \[\left|\frac{xe^{ - x^2n}}{\sqrt{n\ln^3(n + 1)}}\right| \leq \frac{C}{n\ln^{\frac{3}{2}}(n + 1)} \text{ сходится}\]
              \textbf{Ответ}: равномерно сходится по Вейерштрассу
        \item \[\sum \left( \frac{x^2}{1 + nx^3} \right)^3 \quad E = [0, +\infty)\]
              Экстремум: \(x = \frac{2}{n^{\frac{1}{3}}}\)

              Аналогично решается.
        \item \[\sum \frac{\sin \frac{n}{x}\sin \frac{x}{n}}{1 + nx^2} \quad E = [0, +\infty)\]
              \[\left|\frac{\sin \frac{n}{x}\sin \frac{x}{n}}{1 + nx^2}\right| \leq \frac{1 \frac{x}{n}}{1 + nx^2} = \frac{x}{n + n^2x^2} \stackrel{a + b \geq 2\sqrt{ab}}{\leq} \frac{C}{n^{\frac{3}{2}}}\]
        \item \[\sum e^{ - nx} \quad E = (0, +\infty)\]
              По признаку Коши при \(x = \frac{1}{n}\) члены суммы \(\approx \frac{1}{e} \Rightarrow \) расходится.

              Также можно было сказать, что \(\sup e^{ - nx} = 1 \Rightarrow\) члены суммы не \(\rightrightarrows 0 \Rightarrow\) расходится.
        \item \[\sum \frac{e^{ - \frac{x}{n}} \cos nx}{x^2 + n^2x} \]
              \begin{enumerate}
                  \item \(E = [\frac{1}{10}, +\infty)\) --- очевиден, т.к. \(|u_n(x)| \leq \frac{1\cdot 1}{\frac{1}{10}n^2} = \frac{C}{n^2}\) сходится.
                  \item \(E = (0, +\infty)\) --- не сходится, т.к. \(\sup \left|u_n(x)\right| = +\infty,  u_n(x) \not\rightrightarrows 0\)
              \end{enumerate}
        \item \[\sum \frac{\sin(nx)}{(1 + nx)\sqrt{nx}} \]
              \begin{enumerate}
                  \item \((2, 3)\)
                  \item \((0, \pi)\)
              \end{enumerate}

              Аналогично.
        \item \[\sum \frac{( - 1)^n}{\sqrt{n}} \arctg x^n \quad E = [1, +\infty)\]
              \(a_n = \frac{( - 1)^n}{\sqrt{n}}\) --- сходится и называется ряд Лейбница
              \[b_n = \arctg x^n \quad \left|b_n(x)\right| \leq \frac{\pi}{2} \text{ и монотонно}\]

        \item \[\sum \frac{( - 1)^n}{n} \frac{x^n}{x^n + 1} \quad E = [1, +\infty)\]
              \[a_n = \frac{( - 1)^n}{n} \]
              \[b_n = \frac{x^n}{x^n + 1} < 1 \text{ и монотонно}\]
    \end{enumerate}
\end{exercise}

\section*{Практика 11}

\textbf{Важные правила}
\begin{enumerate}
    \item \begin{itemize}
              \item \(\sum u_n(x) = f(x)\)
              \item \(u_n(x)\) непр. в \(x_0\)
              \item Ряд равномерно сходится в \(U(x_0)\)
          \end{itemize}
          Тогда \(f\) \textbf{непр}. в \(x_0\)
    \item \begin{itemize}
              \item \(\sum u_n'(x) = \varphi(x)\)
              \item \(\sum u_n'(x)\) равномерно сходится в \(U(x_0)\)
          \end{itemize}
          Тогда \(f\) --- \textbf{дифф}. в \(x_0, f'(x) = \varphi(x)\)
    \item \begin{itemize}
              \item \(\sum u_n(x) \) равномерно сходится на \([a, b]\)
              \item \(u_n\) непр. на \([a, b]\)
          \end{itemize}
          Тогда \(\int_a^b f(x)dx = \sum \int_a^b u_n(x) dx\)
\end{enumerate}

\begin{exercise}\itemfix
    \begin{enumerate}
        \item \[\sum \frac{\arctg nx}{\sqrt[3]{n^4 + x^2}} \text{ непр. при } x\in\R\]
              Непрерывность слагаемых \(\forall x_0\) очевидна.

              \[\left|\frac{\arctg nx}{\sqrt[3]{n^4 + x^2}}\right| \leq \frac{\frac{\pi}{2}}{n^{\frac{4}{3}}} \text{ равномерно сходится по Вейерштрассу } \Rightarrow \text{ ряд сходится равномерно на }\R \quad \square\]
        \item \[\sum \frac{( - 1)^n}{x^2 +\sqrt{n }} \text{ непр. при } x\in[2, 5] \]
              \[\left|\sum_{n \geq N} \frac{( - 1)^n}{x^2 +\sqrt{n}} \right| \leq \frac{1}{x^2 + \sqrt{N}} \leq \frac{1}{\sqrt{n}}\]
              Тогда ряд равномерно сходится по определению (остатки ряда \(\rightrightarrows S = 0\)).
        \item \[\int_{\ln 2}^{\ln 5} \left( \sum ne^{ - nx} \right)dx = ?\]
              Докажем равномерную сходимость \(\sum ne^{ - nx}, x\in[\ln 2, \ln 5]\)
              \[|ne^{ - nx}| \leq ne^{ - n\ln 3} \leq \frac{n}{2^n} \text{ сходится}\]
              Таким образом ряд равномерно сходится, поэтому:
              \[\int_{\ln 2}^{\ln 5} \left( \sum ne^{ - nx} \right)dx = \sum \int_{\ln 2}^{\ln 5} ne^{ - nx} dx = \sum - e^{ -nx} \Big|_{\ln 2}^{\ln 5}\]
        \item \[\sum \frac{\sin nx}{n^2 \ln^2(n + 1)} \text{ дифф. при } x\in\R\]
              \[\sum \left(\frac{\sin nx}{n^2 \ln^2(n + 1)}\right)' = \sum \frac{\cos nx}{n \ln^2(n + 1)} \]
              Надо доказать равномерную сходимость этого ряда:
              \begin{enumerate}
                  \item Либо на \(\R\)
                  \item Либо \(\forall x_0\) в \(U(x_0)\)
              \end{enumerate}
              Ряд очевидно равн. сходится на \(\R\) по признаку Вейерштрасса.
        \item \(\sum \frac{\cos nx}{n^2}\) имеет непрерывную производную на \((0, 2\pi)\)
              Докажем равномерную сходимость \( -\sum \frac{\sin nx}{n}\) в \(\R\) или в любой окрестности.

              По критерию Коши в \(\R\) её нет ( \(x = \frac{1}{n}, m = n\)), поэтому докажем в любой окрестности. Пусть эта окрестность \((\alpha, \beta)\):
              \[a_n: = \sin nx\]
              Частичные суммы \(a_n\) равномерно ограничены, это записано в трюках.

              \[b_n = \frac{1}{n} \rightrightarrows 0\]
        \item \(\zeta(x) = \sum \frac{1}{n^x}, x\in(1, +\infty)\). Доказать: \(\zeta \in C^{+\infty} (1, +\infty)\)
              Равномерной сходимости на \((1, +\infty)\) нет, потому что по критерию Коши при \(x = 1 + \frac{1}{n}\). Докажем для окрестности \((\alpha, \beta)\).

              По Вейерштрассу сходится:
              \[\left|\frac{1}{n^x}\right| \leq \frac{1}{n^\alpha} \text{ сходится}\]
        \item Где \(f(x) = \sum e^{ - n^2x^2} \cos nx\) непрерывна?
              При \(x \neq 0\) равномерная сходимость доказывается аналогично предыдущему пункту. При \(x = 0\) ряд расходится, т.к. это ряд \(\sum 1\).
        \item \[\lim_{x \to 0} \sum \frac{x^2}{1 + n^2x^2} = ?\]
              Если есть равномерная сходимость ряда в \(U(0)\), то \(\sum u_n(x) \xrightarrow{x\to 0} \sum u_n(0)\). Докажем равн. сходимость по Вейерштрассу:
              \[\left|\frac{x^2}{1 + n^2x^2}\right| \leq \left|\frac{1}{\frac{1}{x^2} + n^2}\right| \leq \frac{1}{n^2}\]
    \end{enumerate}
\end{exercise}

\section*{Практика 12}

\subsection*{Степенные ряды}

Степенной ряд --- ряд вида \(\sum a_n(x - x_0)^n\). Он сходится при \(\left|x - x_0\right|< R, R = \frac{1}{\overline \lim\sqrt[n]{|a_n|}}\)

Иногда ответ выдает \(R = \lim \left|\frac{a_n}{a_{n + 1}}\right|\), но не всегда.

И ещё возможно сходится при \(x = x_0 \pm R\). Сходимость при таком \(x\) находится путём подстановки соответствующего \(x\) в ряд. Но этот ряд не простой, в нем не будет работать признак Даламбера и Коши.

Можно решать заменой на эквивалентное (возможно по модулю), если это не помогает, то применяется Лейбниц или Дирихле.

\begin{exercise}[2812-\dots]\itemfix
    \begin{enumerate}
        \item \[\sum \frac{x^n}{n^e}\]
              \[R = \frac{1}{\overline \lim\sqrt[n]{\frac{1}{n^p}}} = \overline \lim\sqrt[n]{n^p} = \lim e^{\frac{1}{n} p\ln n} = 1\]
              Таким образом, ряд сходится при \(x \in ( - 1, 1)\). Проверим \(x = \pm 1\)
              При \(x = 1\) \(\sum \frac{1}{n^p}\) сходится при \(p > 1\) и расходится при \(p \leq 1\).

              При \(x = - 1\) \(\sum \frac{( - 1)^n}{n^p} \) сходится при \(p > 0\) и расходится при \(p \leq 0\).
        \item \[\sum \frac{3^n + ( - 2)^n}{n}(x + 1)^n\]
              Несложно угадать, что \(R = \frac{1}{3}\). Проверим это вычислением:
              \[\frac{1}{\lim\sqrt[n]{\frac{3^n +( - 2)^n}{n} }} = \frac{1}{3}\]
              Таким образом, ряд сходится при \(x \in ( - \frac{1}{3}, \frac{1}{3})\). Проверим \(x = \pm \frac{1}{3}\)
              \[\sphericalangle x = \frac{1}{3} \quad \frac{3^{n} + ( - 2)^n}{3^n \cdot n} \sim \frac{1}{n} \text{ расходится} \]
              \[\sphericalangle x = - \frac{1}{3} \quad ( - 1)^n\frac{3^{n} + ( - 2)^n}{3^n \cdot n} = \sum \frac{( - 3)^n}{3^n \cdot n} + \frac{2^n}{3^n \cdot} \text{ сходится}\]
        \item \[\sum \frac{(n!)^2}{(2n)!} x^n\]
              \[\lim \left|\frac{a_n}{a_{n + 1}} \right| = \lim \frac{n!^2 (2n + 2)!}{(n + 1)!^2 (2n)!} = \lim \frac{(2n + 2)(2n + 1)}{(n + 1)^2} = 4\]
              Таким образом, ряд сходится при \(x \in ( - 4, 4)\). Проверим \(x = \pm 4\)
              \[\sphericalangle x = 4 \quad \frac{(n!)^2}{(2n)!} 4^n \sim \frac{n^{2n} e^{ - 2n} (2\pi n)}{(2n)^{2n} e^{ - 2n}\sqrt{4\pi n}} 4^n = \sqrt{\pi n} \text{ расходится}\]
              \[\sphericalangle x = - 4 \quad \sum \frac{(n!)^2}{(2n)!} 4^n ( - 1)^n\]
              \[\frac{a_{n+1}}{a_n} = \frac{4n^2 + 8n + 4}{4n^2 + 6n + 2} > 1\]
              Итого при \(x = \pm 4\) расходится, при \(x\in ( - 4, 4)\) сходится.
        \item \[\sum \alpha^{n^2} x^n \quad \alpha\in(0,1)\]
              \[R = \frac{1}{\lim\sqrt[n]{\alpha^{n^2}}} = \frac{1}{\lim \alpha^{n}} = +\infty\]
    \end{enumerate}
\end{exercise}

\subsection*{Разложение функции в ряд \textit{(Тейлора)}}

Мы знаем, что если \(f(x) = \sum a_n(x - x_0)^n\), то это ряд Тейлора, т.е. \(a_n = \frac{f^{(n)} (x_0)}{n!} \).

У нас есть пять основных разложений:
\begin{align*}
    e^z        & = 1 + z + \frac{z^2}{2} + \dots + \frac{x^n}{n!} + \dots                                                    \\
    \sin x     & = x - \frac{x^3}{3!} + \dots + ( - 1)^{n - 1} \frac{x^{2n - 1}}{(2n - 1)!} + \dots                          \\
    \cos x     & = 1 - \frac{x^2}{2!} + \dots + ( - 1)^n \frac{x^{2n}}{(2n)!} + \dots                                        \\
    (1 + x)^m  & = 1 + mx + \frac{m(m - 1)}{2!}x^2 + \dots + \frac{m(m - 1)\dots(m - n + 1)}{n!} + \dots \quad x\in( - 1, 1) \\
    \ln(1 + x) & = x - \frac{x^2}{2} + \frac{x^3}{3} - \dots + ( - 1)^{n - 1}\frac{x^n}{n} + \dots \quad x\in( - 1, 1]
\end{align*}

\begin{exercise}[2851-\dots]
    Разложить в степенной ряд функцию:
    \begin{itemize}
        \item \(e^{ - x^2}\)
              \[e^{ - x^2} = 1 - x^2 + \frac{x^4}{2!} - \frac{x^6}{3!} + \dots \quad x\in\R\]
        \item \(\cos^2 x\)
              \[\cos^2x = \frac{1}{2} + \frac{1}{2}\cos 2x = \frac{1}{2} + \frac{1}{2}\left( 1 - \frac{4x^2}{2!} + \dots + ( - 1)^n \frac{(2x)^{2n}}{(2n)!} + \dots \right) \quad x\in\R\]
        \item \(\cfrac{x^{10}}{1 - x}\)
              \[\frac{x^{10}}{1 - x} = x^{10} \frac{1}{1 - x} = x^{10} + x^{11} + \dots \quad x\in ( - 1, 1)\]
        \item \(\cfrac{x}{\sqrt{1 - 2x}}\)
              \[\frac{x}{\sqrt{1 - 2x}} = x(1 - 2x)^{ - \frac{1}{2}} = x\left( 1 + \left( - \frac{1}{2} \right)( - 2x) + \frac{\frac{ - 1}{2} \frac{ - 3}{2} }{2} ( - 2x)^2 + \dots + \underbrace{\frac{\frac{ -1}{2} \frac{ - 3}{2} \frac{ - 2n - 1}{2}}{n!}( - 2)^n}_{\frac{(2n - 1)!!}{n!!}x^n } \right)\]
        \item \(\cfrac{x}{1 + x - 2x^2}\)
              \[\frac{x}{1 + x - 2x^2} = x(1 - (x - 2x)^2 + (x - 2x^2)^2 + \dots) \quad x - 2x^2\in( - 1, 1)\]
              Из такого вида неудобно получать коэффициенты при \(x^n\).
              \[\frac{x}{(1 - x)(1 + 2x)} = - \frac{\frac{1}{3}}{1 + 2x} + \frac{\frac{1}{3}}{1 - x} = - \frac{1}{3}\left( 1 - 2x + (2x)^2 - (2x)^3  + \dots - (1 + x + x^2 \dots )\right)\]
              \(|2x|< 1\)
        \item \(f(x) = \arcsin x\)
              \[f'(x) = \frac{1}{\sqrt{1 - x^2}} = (1 - x^2)^{ -\frac{1}{2}} = 1 + x^2 - \frac{3}{8} x^4 + \dots \]
              Дальше интегрируем и получаем ответ.
    \end{itemize}
\end{exercise}

\begin{exercise}
    Проверить, что \(\cos^2 x + \sin^2 x = 1\) через ряды, ``в лоб''
\end{exercise}

\begin{exercise}[2873]
    \(f(x) = (1 + x)\ln(1 + x)\)
    Можно не думать и действовать так:
    \[f' = \ln(1 + x) + 1 = 1 + x - \frac{x^2}{2} + \frac{x^3}{3} - \dots \]
    \[f = x + \frac{x^2}{2} - \frac{x^3}{6} + \frac{x^4}{12} - \dots + x^n \frac{( - 1)^n}{(n - 1)n} + \dots \]
    Но у нас ещё есть несовпадение на константу. Это очевидно проверяется подстановкой \(x = 0 : f(0) = 0\), ряд тоже \(0\), поэтому этой константы нет.

    Можно подумать и сказать следующее:
    \[f = x - \frac{x^2}{2} + \frac{x^3}{3} - \dots + x^2 - x^3 + \frac{x^4}{3} + \dots \]
\end{exercise}

\begin{exercise}
    \[f(x) = \frac{1}{4}\ln \frac{1 + x}{1 - x} + \frac{1}{2}\arctg x = \frac{1}{4}\ln(1 + x) - \frac{1}{4}\ln(1 - x) + \frac{1}{2}\arctg x\]
    Разложение \(\arctg x\) получается дифференцированием и потом интегрированием.
    \[\arctg x = x - \frac{x^3}{3} + \frac{x^5}{5} - \frac{x^7}{7} + \dots , \const = 0\]
\end{exercise}

\begin{exercise}
    \[f(x) = \arctg \frac{2 - 2x}{1 + 4x} \]
    \[f'(x) = \frac{1}{1 + \left( \frac{2 - 2x}{1 + 4x}  \right)^2} \frac{ - 2(1 + 4x) - 4(2 - 2x)}{(1 + 4x)^2} = \frac{1}{(1 + 4x)^2 + (2 - 2x)^2} \cdot ( - 10)\]
    \[ = \frac{ -10}{5 + 20x} =- \frac{2}{1 + 4x^2} = - 2(1 - 4x^2 + 16x^4 - \dots + ( - 1)^n (4x^2)^n + \dots )\]
    \[f(x) = \arctg 2 + \sum_{n = 0}^{+\infty} \frac{( - 1)^{n - 1} 2^{2n + 1} x^{2n + 1}}{2n + 1} \]
    Таким образом:
    \[\arctg \frac{2 - 2x}{1 + 4x} = \arctg 2 - \arctg 2x\]
\end{exercise}

На контрольной работе будут вопросы, похожие на:
\begin{itemize}
    \item Сходится ли равномерно последовательность функций?
    \item Сходится ли равномерно функциональный ряд?
    \item Задает ли ряд непрерывную функцию на множестве?
    \item Задает ли ряд дифференцируемую функцию на множестве?
    \item Разложить функцию в ряд
    \item Найти множество сходимости ряда
    \item Найти сумму числового/степенного ряда
\end{itemize}

\section*{Практика 13}

% \begin{exercise}[2906]
%     \[x - \frac{x^3}{3} + \frac{x^5}{5} + \dots = ?\]
% \end{exercise}

% \subsection*{Вычисление сумм рядов}

% Нам уже известны:
% \begin{enumerate}
%     \item \(\sum_{n = 0}^{ +\infty} q^n = \frac{1}{1 - q}\)
%     \item \(e = 1 + \frac{1}{1!} + \frac{1}{2!} + \frac{1}{3!} + \dots \)
%     \item \(\sum \frac{1}{n^2} = \frac{\pi^2}{6}\), \(\sum \frac{( - 1)^{n + 1}}{n} = \ln 2\), \(1 - \frac{1}{3} + \frac{1}{5} - \frac{1}{7} + \dots = \frac{\pi}{4}\)
%     \item Телескопические \(\sum\limits_{n = 1}^{ +\infty} (a_k - a_{k + 1}) = a_1 - \lim\limits_{n\to +\infty} a_n\)
% \end{enumerate}

% \begin{exercise}[2968]
%     \[\frac{1}{1\cdot 3} + \frac{1}{3\cdot 5} + \frac{1}{5\cdot 7} + \dots = \frac{1}{2}\left( 1 - \frac{1}{3} + \frac{1}{3} - \frac{1}{5} + \frac{1}{5} - \frac{1}{7} + \dots \right) = \frac{1}{2}\left(1 - \lim \frac{1}{2n + 1}\right) = \frac{1}{2}\]
%     Решим по-другому:
%     \[f(x) : = \sum \frac{1}{(2n - 1)(2n + 1)}x^{2n + 1}\]
%     \[\left(\frac{f'}{x}\right)' =\sum_{n = 1}^{ +\infty} x^{2n - 2} = \frac{1}{1 - x^2} = \frac{1}{2}\left( \frac{1}{1 - x} + \frac{1}{1 + x} \right)\]
%     \[f' = \frac{x}{2}\left( \ln(1 + x) - \ln(1 - x) \right)\]
%     \[f =\int \frac{x}{2}(\ln(1 + x) -\ln(1 - x))dx = \frac{x^2}{4}\ln \frac{1 + x}{1 - x} -\int \frac{x^2}{} \textcolor{red}{\dots}\]
% \end{exercise}

% \begin{exercise}[2993]
%     \[\sum \frac{2n - 1}{n^2(n + 1)^2} \]
% \end{exercise}


\end{document}