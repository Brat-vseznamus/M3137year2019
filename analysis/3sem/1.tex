\documentclass[12pt, a4paper]{article}

\usepackage{lastpage}
\usepackage{mathtools}
\usepackage{xltxtra}
\usepackage{libertine}
\usepackage{amsmath}
\usepackage{amsthm}
\usepackage{amsfonts}
\usepackage{amssymb}
\usepackage{enumitem}
\usepackage{xcolor}
\usepackage[left=1.5cm, right=1.5cm, top=2cm, bottom=2cm, bindingoffset=0cm, headheight=15pt]{geometry}
\usepackage{fancyhdr}
\usepackage[russian]{babel}
% \usepackage[utf8]{inputenc}
\usepackage{catchfilebetweentags}
\usepackage{accents}
\usepackage{calc}
\usepackage{etoolbox}
\usepackage{mathrsfs}
\usepackage{wrapfig}

\providetoggle{useproofs}
\settoggle{useproofs}{false}

\pagestyle{fancy}
\lfoot{M3137y2019}
\rhead{\thepage\ из \pageref{LastPage}}

\newcommand{\R}{\mathbb{R}}
\newcommand{\Q}{\mathbb{Q}}
\newcommand{\C}{\mathbb{C}}
\newcommand{\Z}{\mathbb{Z}}
\newcommand{\B}{\mathbb{B}}
\newcommand{\N}{\mathbb{N}}

\newcommand{\const}{\text{const}}

\newcommand{\teormin}{\textcolor{red}{!}\ }

\DeclareMathOperator*{\xor}{\oplus}
\DeclareMathOperator*{\equ}{\sim}
\DeclareMathOperator{\Ln}{\text{Ln}}
\DeclareMathOperator{\sign}{\text{sign}}
\DeclareMathOperator{\Sym}{\text{Sym}}
\DeclareMathOperator{\Asym}{\text{Asym}}
% \DeclareMathOperator{\sh}{\text{sh}}
% \DeclareMathOperator{\tg}{\text{tg}}
% \DeclareMathOperator{\arctg}{\text{arctg}}
% \DeclareMathOperator{\ch}{\text{ch}}

\DeclarePairedDelimiter{\ceil}{\lceil}{\rceil}
\DeclarePairedDelimiter{\abs}{\left\lvert}{\right\rvert}

\setmainfont{Linux Libertine}

\theoremstyle{plain}
\newtheorem{axiom}{Аксиома}
\newtheorem{lemma}{Лемма}

\theoremstyle{remark}
\newtheorem*{remark}{Примечание}
\newtheorem*{exercise}{Упражнение}
\newtheorem*{consequence}{Следствие}
\newtheorem*{example}{Пример}
\newtheorem*{observation}{Наблюдение}

\theoremstyle{definition}
\newtheorem{theorem}{Теорема}
\newtheorem*{definition}{Определение}
\newtheorem*{obozn}{Обозначение}

\setlength{\parindent}{0pt}

\newcommand{\dbltilde}[1]{\accentset{\approx}{#1}}
\newcommand{\intt}{\int\!}

% magical thing that fixes paragraphs
\makeatletter
\patchcmd{\CatchFBT@Fin@l}{\endlinechar\m@ne}{}
  {}{\typeout{Unsuccessful patch!}}
\makeatother

\newcommand{\get}[2]{
    \ExecuteMetaData[#1]{#2}
}

\newcommand{\getproof}[2]{
    \iftoggle{useproofs}{\ExecuteMetaData[#1]{#2proof}}{}
}

\newcommand{\getwithproof}[2]{
    \get{#1}{#2}
    \getproof{#1}{#2}
}

\newcommand{\import}[3]{
    \subsection{#1}
    \getwithproof{#2}{#3}
}

\newcommand{\given}[1]{
    Дано выше. (\ref{#1}, стр. \pageref{#1})
}

\renewcommand{\ker}{\text{Ker }}
\newcommand{\im}{\text{Im }}
\newcommand{\grad}{\text{grad}}

\lhead{Математический анализ}
\cfoot{}
\rfoot{7.9.2020}

\begin{document}

\subsection*{Полиномиальная формула}

\begin{definition}
    %<*мультииндекс>
    \textbf{Мультииндекс} --- вектор $\alpha = (\alpha_1, \alpha_2\ldots \alpha_n), \alpha_i\in\Z_+$

    \begin{enumerate}
        \item $|\alpha| \defeq \alpha_1+\alpha_2+\ldots+\alpha_n$
        \item $x^\alpha \defeq x_1^{\alpha_1}x_2^{\alpha_2}\ldots x_n^{\alpha_n} \quad (x\in\R^n)$
        \item $\alpha! \defeq \alpha_1!\alpha_2!\ldots\alpha_n!$
        \item $f^{(\alpha)}_{(x)} \defeq \frac{\partial^{|\alpha|}}{\partial x^\alpha} f \defeq \frac{\partial^{|\alpha|}f}{\partial x_1^{\alpha_1}\partial x_2^{\alpha_2}\ldots \partial x_m^{\alpha_m}}$
    \end{enumerate}
    %</мультииндекс>
\end{definition}

$$(a_1+a_2+\ldots+a_n)^r=\sum_{i_1=1}^n\sum_{i_2=1}^n\ldots \sum_{i_r=1}^n a_{i_1}a_{i_2}\ldots a_{i_r}=\sum_{\alpha : |\alpha| = r} \frac{r!}{\alpha!} a^\alpha$$

Это обобщение следующих формул:
\begin{enumerate}
    \item $(a+b)^2 = a^2 + 2ab + b^2$
    \item $(a_1+a_2)^n = \sum\limits_{\alpha_1+\alpha_2=n}\frac{n!}{\alpha_1!\alpha_2!} a_1^{\alpha_1}a_2^{\alpha_2}$ \textit{(биномиальная формула)}
\end{enumerate}

\begin{lemma}
    \itemfix
    %<*леммаодиффсдвига>
    \begin{itemize}
        \item $f: E\subset \R^m \to\R$
        \item $f\in C^r(E)$ --- это подразумевает, что $E$ открыто
        \item $a\in E$
        \item $h\in \R^m : \forall t\in[-1, 1] \quad a+th\in E$
        \item $\varphi(t) = f(a+th)$
    \end{itemize}
    Тогда при $1\le k\le r$:
    $$\varphi^{(k)}(0)=\sum_{j : |j|=k} \frac{k!}{j!} h^j \frac{\partial^k f}{\partial x^j} (a)$$
    %</леммаодиффсдвига>
\end{lemma}

%<*леммаодиффсдвигаproof>
\begin{proof}
    Как на лекции:
    $$\varphi'(t)=\sum_{i=1}^m \frac{\partial f}{\partial x_i}(a+th)h_i$$
    $$\varphi''(t)=\sum_{i=1}^m \left(\frac{\partial f}{\partial x_i}(a+th)\right)'h_i=\sum_{i=1}^m \sum_{i_2=1}^m \frac{\partial^2 f}{\partial x_i\partial x_{i_2}}(a+th)h_ih_{i_2}$$
    $$\varphi''(0)=\frac{\partial^2 f}{\partial x_1^2}h_1^2 + \frac{\partial^2 f}{\partial x_2^2}h_2^2 + \ldots + \frac{\partial^2 f}{\partial x_m^2}h_m^2 + 2\left( \frac{\partial^2 f}{\partial x_1\partial x_2}h_1h_2 + \frac{\partial^2 f}{\partial x_1\partial x_3}h_1h_3 + \ldots \right)$$
    $$\varphi^{(k)}(0)=\sum_{i_1=1}^m\sum_{i_2=1}^m\ldots \sum_{i_k=1}^m \frac{\partial^k f(a)}{\partial x_{i_1}\partial x_{i_2}\ldots \partial x_{i_k}} h_{i_1}h_{i_2}\ldots h_{i_k}$$

    Формальное доказательство по индукции:

    Индукционное предположение:
    \[\varphi^{k} (t) = \sum_{i_1=1}^m\sum_{i_2=1}^m\ldots \sum_{i_k=1}^m \frac{\partial^k f(a + th)}{\partial x_{i_1}\partial x_{i_2}\ldots \partial x_{i_k}} h_{i_1}h_{i_2}\ldots h_{i_k}\]

    \textbf{База}:
    \[\varphi'(t)=\sum_{i_1=1}^m \frac{\partial f}{\partial x_{i_1}}(a+th)h_{i_1}\]

    \textbf{Переход}:
    \begin{align*}
        \varphi^{(k + 1)}(t)
         & = \left(\varphi^k (t)\right)'                                                                                                                                                                                                    \\
         & = \left(\sum_{i_1=1}^m\sum_{i_2=1}^m\ldots \sum_{i_k=1}^m \frac{\partial^k f(a + th)}{\partial x_{i_1}\partial x_{i_2}\ldots \partial x_{i_k}} h_{i_1}h_{i_2}\ldots h_{i_k}\right)'\,                                            \\
         & = \sum_{i_1=1}^m\sum_{i_2=1}^m\ldots \sum_{i_k=1}^m \left(\frac{\partial^k f(a + th)}{\partial x_{i_1}\partial x_{i_2}\ldots \partial x_{i_k}}\right)' h_{i_1}h_{i_2}\ldots h_{i_k}                                              \\
         & = \sum_{i_1=1}^m\sum_{i_2=1}^m\ldots \sum_{i_k=1}^m \sum_{i_{k+1} = 1}^m\frac{\partial^{k + 1} f(a + th)}{\partial x_{i_1}\partial x_{i_2}\ldots \partial x_{i_k} \partial x_{i_{k+1}}} h_{i_{k+1}} h_{i_1}h_{i_2}\ldots h_{i_k} \\
    \end{align*}
\end{proof}
%</леммаодиффсдвигаproof>

\begin{theorem}[Формула Тейлора в терминах мультииндекса]
    \itemfix
    %<*тейлорлагранж>
    \begin{itemize}
        \item $f\in C^{r+1}(E)$ --- это подразумевает $E\subset\R^m, f:E\to\R$
        \item $x\in B(a, R)\subset E$
    \end{itemize}
    Тогда $\exists t\in(0, 1)$:
    %<*формулатейлорамультииндекс>
    $$f(x) = \sum_{\alpha : 0\le |\alpha|\le r} \frac{f^{(\alpha)}(a)}{\alpha!}(x-a)^\alpha + \underbrace{\sum_{\alpha : |\alpha|=r+1} \frac{f^{(\alpha)}(a+t(x-a))}{\alpha!}(x-a)^\alpha}_{\text{Остаток в форме Лагранжа}}$$
    %</формулатейлорамультииндекс>
    %</тейлорлагранж>
\end{theorem}

Раскроем мультииндексы:

$$f(x)=\sum_{k=0}^r\left( \sum_{\substack{(\alpha_1\ldots \alpha_m): \\ \alpha_i\ge 0 \\ \sum \alpha_i = k}} \frac{1}{\alpha_1!\alpha_2!\ldots\alpha_m!} \frac{\partial^k f(a)}{(\partial x_1)^{\alpha_1}\ldots (\partial x_m)^{\alpha_m}} (x_1-a_1)^{\alpha_1}(x_2-a_2)^{\alpha_2}\ldots (x_m-a_m)^{\alpha_m} \right)$$
Ещё + аналогичный остаток.

$$f(a+h)=\sum_{k=0}^r\sum\ldots \frac{1}{\alpha_1!\alpha_2!\ldots\alpha_m!} \frac{\partial^k f(a)}{(\partial x_1)^{\alpha_1}\ldots (\partial x_m)^{\alpha_m}} h_1^{\alpha_1}h_2^{\alpha_2}\ldots h_m^{\alpha_m}$$
Тут тоже + аналогичный остаток.

%<*тейлорлагранжproof>
\begin{proof}
    Кажется, это теперь почти очевидно.

    $\varphi(t) = а(a+th)$, где $h=x-a$. Тогда $\varphi(0)=f(a)$

    $$\varphi(t) = \varphi(0) + \frac{\varphi'(0)}{1!}+\ldots+\frac{\varphi^{(r)}(0)}{r!}t^r+\frac{\varphi^{(r+1)}(\overline t)}{(r+1)!}t^{r+1}$$
    $$f(x) = \underbrace{\sum_{\alpha : 0\le |\alpha|\le r} \frac{f^{(\alpha)}(a)}{\alpha!}(x-a)^\alpha}_{\text{Многочлен Тейлора}} + \underbrace{\sum_{\alpha : |\alpha| = r+1} \frac{f^{(\alpha)}(a+\Theta(x-a))}{\alpha!}(x-a)^\alpha}_{\mathcal O (|x-1|^r)}$$
    По лемме:
    $$f(x)=f(a) + \sum_{k=1}^r\sum_{\alpha : |\alpha|=k}\frac{f^{(\alpha)}}{\alpha!}h^\alpha + \sum_{\alpha : |\alpha| = r+1} \frac{f^{(\alpha)}(a+\Theta(x-a))}{\alpha!}h^\alpha$$
\end{proof}
%</тейлорлагранжproof>

\begin{definition}
    %<*дифференциал>
    $$\sum_{\alpha : |\alpha|=k} k!\frac{f^{(\alpha)}(a)}{\alpha!}h^\alpha \defeq k\text{-й \textbf{дифференциал} функции $f$ в точке $a$ относительно $h$} \defeq d^k f(a, h)$$
    %</дифференциал>
\end{definition}

Перепишем $f(x)$ через дифференциал:

%<*формулатейлорадифференциал>
$$f(a+h)=\sum_{k=0}^r\frac{d^k f(a, h)}{k!} + \frac{1}{(r+1)!} d^{r+1} f(a+\Theta h, h)$$
\[f(a+h)=\sum_{k=0}^r\frac{d^k f(a, h)}{k!} + o(|h|^r)\]
%</формулатейлорадифференциал>

$f(a)$ это $d^k f(a, h)$ при $k=0$, поэтому он зашел под сумму.

\begin{example}
    $\sphericalangle k=2$

    $$d^2 f = f''_{x_1, x_1}(a) h_1^2 + f''_{x_2, x_2}(a) h_2^2 + \ldots + f''_{x_m, x_m}(a) h_m^2 + 2\sum_{i<j} f''_{x_i, x_j}h_ih_j$$
\end{example}

Заметим, что $k!/\alpha!$ - число способов реализовать дифференцирование, т.е. в каком порядке брать частные производные.

В дифференциалах работает композиция: $d^{k+1}f = d(d^k f)$

Покажем это на примере:
$$df = f'_{x_1}h_1 + f'_{x_2}h_2 + \ldots + f'_{x_m}h_m$$
$$d^2 f = (f''_{x_1x_1}h_1+f''_{x_1x_2}h_2+\ldots + f''_{x_1x_m}h_m)h_1 + (f''_{x_2x_1}h_1+f''_{x_2x_2}h_2+\ldots + f''_{x_2x_m}h_m)h_2 + \ldots = $$
$$= (f'_{x_1})'h_1 + (f'_{x_2})'h_2 + \ldots = d(df)$$

\begin{theorem}[Формула Тейлора с остатком в форме Пеано]
    %<*тейлорпеано>
    $$f(a+h) = \sum_{\alpha : 0 \le |\alpha| \le r} \frac{f^{(\alpha)}(a)}{\alpha!}h^\alpha + o(|h|^r)$$
    %</тейлорпеано>
\end{theorem}
%<*тейлорпеаноproof>
\begin{proof}
    \textcolor{red}{Отсутствует}
\end{proof}
%</тейлорпеаноproof>

\begin{exercise}
    Если $f(a+h) = \underbrace{T_r(h)}_{\text{Многочлен степени $\le r$}} + o(|h|^r)$, то $T_r(h)$ --- многочлен Тейлора
\end{exercise}

\begin{example}
    $(0.97)^{1.02} = ?$

    $f(x, y) \defeq x^y$

    $f(0.97, 1.02)=?$

    Здесь все производные в $(1, 1)$, это не указывается ради краткости.
    $$f(x, y) = f(1, 1) + f'_x (x-1) + f'_y(y-1) + \frac{f''_{xx}}{2!}(x-1)^2 + \frac{f''_{yy}}{2!}(y-1)^2 + f''_{xy}(x-1)(y-1) + o(\ldots)$$

    \begin{itemize}
        \item $f'_x = yx^{y-1} \to 1$
        \item $f'_y = x^y \ln x \to 0$
        \item $f''_{xx} = y(y-1)x^{y-2} \to 0$
        \item $f''_{yy} = x^y \ln^2 x \to 0$
        \item $f''_{xy} = x^{y-1} + yx^{y-1} \ln x \to 1$
    \end{itemize}

    $$f(0.97, 1.02) \approx 1 + 1 (-0.03) + 1 (-0.03)(0.02)$$
\end{example}

\begin{definition}
    $\mathcal L(\R^m, \R^n)$ --- \textbf{множество линейных отображений} $\R^m \to \R^n$, также обозначается $\text{Lin}(\R^m, \R^n), L_{m, n}$

    Это линейное пространство:
    \begin{enumerate}
        \item $(F+G)(x)=F(x)+G(x)$
        \item $(\alpha F)x = \alpha (Fx)$
    \end{enumerate}
\end{definition}

\begin{obozn}[Норма линейного оператора]
    %<*нормалоп>
    $$A\in \mathcal L(\R^m, \R^n) \quad ||A|| \defeq \sup\limits_{\substack{x\in\R^m :\\ |x|=1}} |Ax|$$
    %</нормалоп>
\end{obozn}

\begin{remark}
    \itemfix
    \begin{enumerate}
        \item $\sup \Leftrightarrow \max$ в силу компактности сферы
        \item $A = (a_{ij})$ \textit{(оператор задается матрицей)}, тогда $||A||\le \sqrt{\sum\limits_{i, j} a_{ij}^2}$ по лемме об оценке нормы линейного отображения ( \(||Ax|| \le C||x||\))
        \item $\forall x \in \R^m \quad |Ax| \le ||A||\cdot |x|$
              \begin{proof}\itemfix
                  \begin{enumerate}
                      \item $x=0$ --- тривиально, $0\le 0$
                      \item $x\not=0\ $ \(\ \tilde x = \frac{x}{|x|} \Rightarrow |\tilde{x}| = 1 \Rightarrow |Ax|\le ||A||\)
                            \[|Ax| = |A(|x|\tilde x)| = ||x| A\tilde x| = |x|\cdot |A\tilde x| \le ||A||\cdot |x|\]
                  \end{enumerate}
              \end{proof}
        \item Если $\exists c > 0 : \forall x\in\R^m \ \ |Ax|\le C|x|$, то $||A||\le C$
    \end{enumerate}
\end{remark}

\begin{example}
    \begin{enumerate}
        \item $m=l=1 \quad A \leftrightarrow a_{11} \quad x \mapsto a_{11}x \quad ||A||=a$
        \item $m=1, l$ --- любое. $A : \R\to\R^l \quad A\leftrightarrow \begin{pmatrix}
                      a_1    \\
                      \vdots \\
                      a_l
                  \end{pmatrix} = \vec a \quad t\mapsto t\vec a \quad ||A||=|a|$
        \item $m$ --- любое, $l=1 \quad A : \R^m\to\R \quad A\leftrightarrow \vec a \quad x\mapsto (\vec a, x) \quad ||A||=\sup\limits_{\substack{x\in\R^m :\\ |x|=1}} |(\vec a, x)| = |\vec a|$
        \item $m$ --- любое, $l$ --- любое. $||A||=\sup|Ax|$ --- мы не знаем, как такое считать.
    \end{enumerate}
\end{example}

\begin{lemma}
    \itemfix
    %<*эквивалентностьнепрерывности>
    \begin{itemize}
        \item $X, Y$ --- линейные нормированные пространства
        \item $A\in \mathcal L(X, Y)$
    \end{itemize}
    Тогда эквивалентны следующие утверждения:
    \begin{enumerate}
        \item $A$ --- ограниченный оператор, т.е. $||A||$ --- конечно
        \item $A$ --- непрерывно в нуле
        \item $A$ --- непрерывно всюду в $X$
        \item $A$ --- равномерно непрерывно
    \end{enumerate}
    %</эквивалентностьнепрерывности>
\end{lemma}

%<*эквивалентностьнепрерывностиproof>
\begin{proof}
    \itemfix
    \begin{enumerate}
        \item 4 $\Rightarrow$ 3 $\Rightarrow$ 2 --- очевидно.
        \item 2 $\Rightarrow$ 1

              Непрерывность в $0$: $\forall \varepsilon \ \ \exists \delta : \forall x : |x| \le \delta \quad |Ax|<\varepsilon$

              $\sphericalangle \varepsilon = 1, |x|=1 : |Ax| = \left|A\frac{1}{\delta}\delta x\right| = \frac{1}{\delta} |A \delta x| \le \frac{1}{\delta}$
        \item 1 $\Rightarrow$ 4

              $$\forall \varepsilon > 0 \ \ \exists \delta := \frac{\varepsilon}{||A||} \ \ \forall x_1, x_0 : |x_1-x_0| < \delta$$
              \[|Ax_1 - Ax_0| = |A(x_1-x_0)| \le ||A|| \cdot |x_1 - x_0| < \varepsilon\]
    \end{enumerate}
\end{proof}
%</эквивалентностьнепрерывностиproof>

\begin{theorem}[о пространстве линейных отображений]
    \itemfix
    %<*опространствелинейныхотображений>
    \begin{enumerate}
        \item Отображение $A \to ||A||$ в $\mathcal L(\R^m, \R^n)$ --- норма, т.е.:
              \begin{enumerate}
                  \item $||A|| \ge 0$
                  \item $||A|| = 0 \Rightarrow A = 0_{n \times m}$
                  \item $\forall \lambda \in \R \ \ ||\lambda A|| = |\lambda| ||A||$
                  \item $||A+B|| \le ||A|| + ||B||$
              \end{enumerate}
        \item $A\in \mathcal L(\R^m, \R^n), B\in \mathcal L(\R^n, \R^k) \Rightarrow ||BA|| \le ||B||\cdot ||A||$
    \end{enumerate}
    %</опространствелинейныхотображений>
\end{theorem}
%<*опространствелинейныхотображенийproof>
\begin{proof}
    \itemfix
    \begin{enumerate}
        \item $||A|| = \sup\limits_{|x| = 1} |Ax|$

              a, b, c --- очевидно.

              d: $|(A+B)x| = |Ax + Bx| \le |Ax| + |Bx| \le (||A|| + ||B||) |x|$

              По замечанию 3 $||A+B|| \le ||A|| + ||B||$
        \item $|BAx| = |B (Ax)| \le ||B||\cdot|Ax|\le ||B||\cdot||A||$
    \end{enumerate}
\end{proof}
%</опространствелинейныхотображенийproof>

\begin{remark}
    В $\mathcal L(X, Y)$:

    $$||A|| = \sup_{|x|=1} |Ax| = \sup_{|x| \le 1} |Ax| = \sup_{|x| < 1} |Ax| = \sup_{|x| \not= 0} \frac{|Ax|}{|x|} = \inf \{ C \in \R : \forall x \ \ |Ax| \le C|x|\}$$
\end{remark}

\end{document}