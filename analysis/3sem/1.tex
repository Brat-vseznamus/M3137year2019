\documentclass[12pt, a4paper]{article}

\usepackage{lastpage}
\usepackage{mathtools}
\usepackage{xltxtra}
\usepackage{libertine}
\usepackage{amsmath}
\usepackage{amsthm}
\usepackage{amsfonts}
\usepackage{amssymb}
\usepackage{enumitem}
\usepackage{xcolor}
\usepackage[left=1.5cm, right=1.5cm, top=2cm, bottom=2cm, bindingoffset=0cm, headheight=15pt]{geometry}
\usepackage{fancyhdr}
\usepackage[russian]{babel}
% \usepackage[utf8]{inputenc}
\usepackage{catchfilebetweentags}
\usepackage{accents}
\usepackage{calc}
\usepackage{etoolbox}
\usepackage{mathrsfs}
\usepackage{wrapfig}

\providetoggle{useproofs}
\settoggle{useproofs}{false}

\pagestyle{fancy}
\lfoot{M3137y2019}
\rhead{\thepage\ из \pageref{LastPage}}

\newcommand{\R}{\mathbb{R}}
\newcommand{\Q}{\mathbb{Q}}
\newcommand{\C}{\mathbb{C}}
\newcommand{\Z}{\mathbb{Z}}
\newcommand{\B}{\mathbb{B}}
\newcommand{\N}{\mathbb{N}}

\newcommand{\const}{\text{const}}

\newcommand{\teormin}{\textcolor{red}{!}\ }

\DeclareMathOperator*{\xor}{\oplus}
\DeclareMathOperator*{\equ}{\sim}
\DeclareMathOperator{\Ln}{\text{Ln}}
\DeclareMathOperator{\sign}{\text{sign}}
\DeclareMathOperator{\Sym}{\text{Sym}}
\DeclareMathOperator{\Asym}{\text{Asym}}
% \DeclareMathOperator{\sh}{\text{sh}}
% \DeclareMathOperator{\tg}{\text{tg}}
% \DeclareMathOperator{\arctg}{\text{arctg}}
% \DeclareMathOperator{\ch}{\text{ch}}

\DeclarePairedDelimiter{\ceil}{\lceil}{\rceil}
\DeclarePairedDelimiter{\abs}{\left\lvert}{\right\rvert}

\setmainfont{Linux Libertine}

\theoremstyle{plain}
\newtheorem{axiom}{Аксиома}
\newtheorem{lemma}{Лемма}

\theoremstyle{remark}
\newtheorem*{remark}{Примечание}
\newtheorem*{exercise}{Упражнение}
\newtheorem*{consequence}{Следствие}
\newtheorem*{example}{Пример}
\newtheorem*{observation}{Наблюдение}

\theoremstyle{definition}
\newtheorem{theorem}{Теорема}
\newtheorem*{definition}{Определение}
\newtheorem*{obozn}{Обозначение}

\setlength{\parindent}{0pt}

\newcommand{\dbltilde}[1]{\accentset{\approx}{#1}}
\newcommand{\intt}{\int\!}

% magical thing that fixes paragraphs
\makeatletter
\patchcmd{\CatchFBT@Fin@l}{\endlinechar\m@ne}{}
  {}{\typeout{Unsuccessful patch!}}
\makeatother

\newcommand{\get}[2]{
    \ExecuteMetaData[#1]{#2}
}

\newcommand{\getproof}[2]{
    \iftoggle{useproofs}{\ExecuteMetaData[#1]{#2proof}}{}
}

\newcommand{\getwithproof}[2]{
    \get{#1}{#2}
    \getproof{#1}{#2}
}

\newcommand{\import}[3]{
    \subsection{#1}
    \getwithproof{#2}{#3}
}

\newcommand{\given}[1]{
    Дано выше. (\ref{#1}, стр. \pageref{#1})
}

\renewcommand{\ker}{\text{Ker }}
\newcommand{\im}{\text{Im }}
\newcommand{\grad}{\text{grad}}

\lhead{Математический анализ}
\cfoot{}
\rfoot{8.9.2020}

\begin{document}

\subsection*{Полиномиальная формула}

\begin{definition}
    \textbf{Мультииндекс} --- вектор $\alpha = (\alpha_1, \alpha_2\ldots \alpha_n), \alpha_i\in\Z_+$

    $|\alpha| \defeq \alpha_1+\alpha_2+\ldots+\alpha_n$

    $x^\alpha \defeq x_1^{\alpha_1}x_2^{\alpha_2}\ldots x_n^{\alpha_n} \quad (x\in\R^n)$

    $\alpha! \defeq \alpha_1!\alpha_2!\ldots\alpha_n!$

    $f^{(\alpha)}_{(x)}=\frac{\partial^{|\alpha|}}{\partial x^\alpha} f = \frac{\partial^{|\alpha|}f}{\partial x_1^{\alpha_1}\partial x_2^{\alpha_2}\ldots \partial x_m^{\alpha_m}}$
\end{definition}

$$(a_1+a_2+\ldots+a_n)^r=\sum_{i_1=1}^n\sum_{i_2=1}^n\ldots \sum_{i_n=1}^n a_{i_1}a_{i_2}\ldots a_{i_n}=\sum_{\alpha : |\alpha| = r} \frac{r!}{\alpha!} a^\alpha$$

Это обобщение следующих формул:
\begin{enumerate}
    \item $(a+b)^2 = a^2 + 2ab + b^2$
    \item $(a_1+a_2)^n = \sum\limits_{\alpha_1+\alpha_2=n}\frac{n!}{\alpha_1!\alpha_2!} a_1^{\alpha_1}a_2^{\alpha_2}$ \textit{(биномиальная формула)}
\end{enumerate}

\begin{lemma}
    \itemfix
    \begin{itemize}
        \item $f: E\subset \R^m \to\R$
        \item $f\in C^r(E)$ --- это подразумевает, что $E$ открыто
        \item $a\in E$
        \item $h\in \R^m : \forall t\in[-1, 1] \quad a+th\in E$
        \item $\varphi(t) = f(a+th)$
    \end{itemize}
    Тогда при $1\le k\le r$:
    $$\varphi^{(k)}(0)=\sum_{j : |j|=k} \frac{k!}{j!} h^j \frac{\partial^k f}{\partial x^j} (a)$$
\end{lemma}

\begin{proof}
    $$\varphi'(t)=\sum_{i=1}^m \frac{\partial f}{\partial x_i}(a+th)h_i$$
    $$\varphi''(t)=\sum_{i=1}^m \left(\frac{\partial f}{\partial x_i}(a+th)\right)'h_i=\sum_{i=1}^m \sum_{i_2=1}^m \frac{\partial^2 f}{\partial x_i\partial x_{i_2}}(a+th)h_ih_{i_2}$$
    $$\varphi''(0)=\frac{\partial^2 f}{\partial x_1^2}h_1^2 + \frac{\partial^2 f}{\partial x_2^2}h_2^2 + \ldots + \frac{\partial^2 f}{\partial x_m^2}h_m^2 + 2\left( \frac{\partial^2 f}{\partial x_1\partial x_2}h_1h_2 + \frac{\partial^2 f}{\partial x_1\partial x_3}h_1h_3 + \ldots \right)$$
    $$\varphi^{(r)}(t)=\sum_{i_1=1}^m\sum_{i_2=1}^m\ldots \sum_{i_r=1}^m \frac{\partial^r f}{\partial x_{i_1}\partial x_{i_2}\ldots \partial x_{i_r}} h_{i_1}h_{i_2}\ldots h_{i_r}$$
\end{proof}

\begin{theorem}[Формула Тейлора в терминах мультииндекса]
    \itemfix
    \begin{itemize}
        \item $f\in C^{r+1}(E)$ --- это подразумевает $E\subset\R^m, f:E\to\R$
        \item $x\in B(a, R)\subset E$
    \end{itemize}
    Тогда $\exists t\in(0, 1)$:
    $$f(x) = \sum_{\alpha : 0\le |\alpha|\le r} \frac{f^{(\alpha)}(a)}{\alpha!}(x-a)^\alpha + \underbrace{\sum_{\alpha : |\alpha|=r+1} \frac{f^{(\alpha)}(a+t(x-a))}{(\alpha+1)!}(x-a)^\alpha}_{\text{Остаток в форме Лагранжа}}$$
\end{theorem}

Раскроем мультииндексы:

$$f(x)=\sum_{k=0}^r\left( \sum_{\substack{(\alpha_1\ldots \alpha_m): \\ \alpha_i\ge 0 \\ \sum \alpha_i = k}} \frac{1}{\alpha_1!\alpha_2!\ldots\alpha_m!} \frac{\partial^k f(a)}{(\partial x_1)^{\alpha_1}\ldots (\partial x_m)^{\alpha_m}} (x_1-a_1)^{\alpha_1}(x_2-a_2)^{\alpha_2}\ldots (x_m-a_m)^{\alpha_m} \right)$$
Ещё + аналогичный остаток.

$$f(a+h)=\sum_{k=0}^r\sum\ldots \frac{1}{\alpha_1!\alpha_2!\ldots\alpha_m!} \frac{\partial^k f(a)}{(\partial x_1)^{\alpha_1}\ldots (\partial x_m)^{\alpha_m}} h_1^{\alpha_1}h_2^{\alpha_2}\ldots h_m^{\alpha_m}$$
Тут тоже + аналогичный остаток.

\begin{proof}
    Кажется, это теперь почти очевидно.

    $\varphi(t) = а(a+th)$, где $h=x-a$. Тогда $\varphi(0)=f(a)$

    $$\varphi(t) = \varphi(0) + \frac{\varphi'(0)}{1!}+\ldots+\frac{\varphi^{(r)}(0)}{r!}t^r+\frac{\varphi^{(r+1)}(\overline t)}{(r+1)!}t^{r+1}$$
    $$f(x) = \sum_{\alpha : 0\le |\alpha|\le r} \frac{f^{(\alpha)}(a)}{\alpha!}(x-a)^\alpha + \sum_{\alpha : |\alpha| = r+1} \frac{f^{(\alpha)}(a+\Theta(x-a))}{(\alpha+1)!}(x-a)^\alpha$$
    По лемме:
    $$f(x)=f(a) + \sum_{k=1}^r\sum_{\alpha : |\alpha|=k}\frac{f^{(\alpha)}}{\alpha!}h^\alpha + \sum_{\alpha : |\alpha| = r+1} \frac{f^{(\alpha)}(a+\Theta(x-a))}{\alpha!}h^\alpha$$
\end{proof}

\begin{remark}
    $$\sum_{\alpha : |\alpha|=k} k!\frac{f^{(\alpha)}}{\alpha!}h^\alpha \defeq k\text{-й дифференциал функции $f$ в точке $k$} \defeq d^k f(a, h)$$
\end{remark}

Перепишем $f(x)$ через дифференциал:

$$f(x)=\sum_{k=0}^r\frac{d^k f(a, h)}{k!} + \frac{1}{(k+1)!} d^{k+1} f(a+\Theta h, h)$$

$f(a)$ это $d^k f(a, h)$ при $k=0$, поэтому он зашел под сумму.

\begin{example}
    $\sphericalangle k=2$

    $$d^2 f = f''_{x_1, x_1}(a) h_1^2 + f''_{x_2, x_2}(a) h_2^2 + \ldots + f''_{x_m, x_m}(a) h_m^2 + 2\sum_{i<j} f''_{x_i, x_j}h_ih_j$$
\end{example}

Заметим, что $k!/\alpha!$ - число способов реализовать дифференцирование, т.е. в каком порядке брать частные производные.

В дифференциалах работает композиция: $d^{k+1}f = d(d^k f)$

Покажем это на примере:
$$df = f'_{x_1}h_1 + f'_{x_2}h_2 + \ldots + f'_{x_m}h_m$$
$$d^2 f = (f''_{x_1x_1}h_1+f''_{x_1x_2}h_2+\ldots + f''_{x_1x_m}h_m)h_1 + (f''_{x_2x_1}h_1+f''_{x_2x_2}h_2+\ldots + f''_{x_2x_m}h_m)h_2 + \ldots = $$
$$= (f'_{x_1})'h_1 + (f'_{x_2})'h_2 + \ldots = d(df)$$

\end{document}