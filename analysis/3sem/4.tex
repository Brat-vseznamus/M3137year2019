\documentclass[12pt, a4paper]{article}

\usepackage{lastpage}
\usepackage{mathtools}
\usepackage{xltxtra}
\usepackage{libertine}
\usepackage{amsmath}
\usepackage{amsthm}
\usepackage{amsfonts}
\usepackage{amssymb}
\usepackage{enumitem}
\usepackage{xcolor}
\usepackage[left=1.5cm, right=1.5cm, top=2cm, bottom=2cm, bindingoffset=0cm, headheight=15pt]{geometry}
\usepackage{fancyhdr}
\usepackage[russian]{babel}
% \usepackage[utf8]{inputenc}
\usepackage{catchfilebetweentags}
\usepackage{accents}
\usepackage{calc}
\usepackage{etoolbox}
\usepackage{mathrsfs}
\usepackage{wrapfig}

\providetoggle{useproofs}
\settoggle{useproofs}{false}

\pagestyle{fancy}
\lfoot{M3137y2019}
\rhead{\thepage\ из \pageref{LastPage}}

\newcommand{\R}{\mathbb{R}}
\newcommand{\Q}{\mathbb{Q}}
\newcommand{\C}{\mathbb{C}}
\newcommand{\Z}{\mathbb{Z}}
\newcommand{\B}{\mathbb{B}}
\newcommand{\N}{\mathbb{N}}

\newcommand{\const}{\text{const}}

\newcommand{\teormin}{\textcolor{red}{!}\ }

\DeclareMathOperator*{\xor}{\oplus}
\DeclareMathOperator*{\equ}{\sim}
\DeclareMathOperator{\Ln}{\text{Ln}}
\DeclareMathOperator{\sign}{\text{sign}}
\DeclareMathOperator{\Sym}{\text{Sym}}
\DeclareMathOperator{\Asym}{\text{Asym}}
% \DeclareMathOperator{\sh}{\text{sh}}
% \DeclareMathOperator{\tg}{\text{tg}}
% \DeclareMathOperator{\arctg}{\text{arctg}}
% \DeclareMathOperator{\ch}{\text{ch}}

\DeclarePairedDelimiter{\ceil}{\lceil}{\rceil}
\DeclarePairedDelimiter{\abs}{\left\lvert}{\right\rvert}

\setmainfont{Linux Libertine}

\theoremstyle{plain}
\newtheorem{axiom}{Аксиома}
\newtheorem{lemma}{Лемма}

\theoremstyle{remark}
\newtheorem*{remark}{Примечание}
\newtheorem*{exercise}{Упражнение}
\newtheorem*{consequence}{Следствие}
\newtheorem*{example}{Пример}
\newtheorem*{observation}{Наблюдение}

\theoremstyle{definition}
\newtheorem{theorem}{Теорема}
\newtheorem*{definition}{Определение}
\newtheorem*{obozn}{Обозначение}

\setlength{\parindent}{0pt}

\newcommand{\dbltilde}[1]{\accentset{\approx}{#1}}
\newcommand{\intt}{\int\!}

% magical thing that fixes paragraphs
\makeatletter
\patchcmd{\CatchFBT@Fin@l}{\endlinechar\m@ne}{}
  {}{\typeout{Unsuccessful patch!}}
\makeatother

\newcommand{\get}[2]{
    \ExecuteMetaData[#1]{#2}
}

\newcommand{\getproof}[2]{
    \iftoggle{useproofs}{\ExecuteMetaData[#1]{#2proof}}{}
}

\newcommand{\getwithproof}[2]{
    \get{#1}{#2}
    \getproof{#1}{#2}
}

\newcommand{\import}[3]{
    \subsection{#1}
    \getwithproof{#2}{#3}
}

\newcommand{\given}[1]{
    Дано выше. (\ref{#1}, стр. \pageref{#1})
}

\renewcommand{\ker}{\text{Ker }}
\newcommand{\im}{\text{Im }}
\newcommand{\grad}{\text{grad}}

\lhead{Математический анализ}
\cfoot{}
\rfoot{28.9.2020}

\begin{document}

$\sphericalangle \R^{m+n}, (x, y) \in\R^{m+n}, x=\begin{pmatrix}
        x_1 & \cdots & x_m
    \end{pmatrix}, y = \begin{pmatrix}
        y_1 & \cdots & y_n
    \end{pmatrix}$

$F : O\subset\R^{m+n} \to\R^m$

$$F' = \begin{pmatrix}
        \frac{\partial F_1}{\partial x_1} & \cdots & \frac{\partial F_1}{\partial x_m} & \frac{\partial F_1}{\partial y_1} & \cdots & \frac{\partial F_1}{\partial y_n} \\
        \vdots                            & \ddots & \vdots                            & \vdots                            & \ddots & \vdots                            \\
        \frac{\partial F_n}{\partial x_1} & \cdots & \frac{\partial F_n}{\partial x_m} & \frac{\partial F_n}{\partial y_1} & \cdots & \frac{\partial F_n}{\partial y_n} \\
    \end{pmatrix}$$

\begin{theorem}[о неявном отображении]\itemfix
    %<*онеявномотображении>
    \begin{itemize}
        \item $F: O\subset\R^{m+n}\to\R^n$
        \item $O$ откр.
        \item $F\in C^r (O, \R^n)$
        \item $(a, b)\in O$
        \item $F(a, b) = 0$
        \item $\det F'_y(a, b)\not=0$
    \end{itemize}
    Тогда:
    \begin{enumerate}
        \item $\exists$ откр. $P\subset\R^{m}, a\in P$

              $\exists$ откр. $Q\subset\R^{n}, b\in Q$

              $\exists!\ \Phi : P\to Q\in C^r : \forall x \in P \ \ F(x, \Phi(x)) = 0$
        \item $\Phi'(x) = -\left(F'_y(x, \Phi(x))\right)^{-1}\cdot F'_x(x, \Phi(x))$
    \end{enumerate}
    %</онеявномотображении>

    В терминах системы уравнений:
    % украдено у Yulya3102
    %<*онеявномотображениивтерминахсистемы>
    Дана система из $n$ функций, $f_i\in C^r$.
    \[\begin{cases}
            f_1(x_1 \dots x_m, y_1 \dots y_n) = 0 \\
            \vdots                                \\
            f_n(x_1 \dots x_m, y_1 \dots y_n) = 0
        \end{cases}\]
    \[\frac{\partial F}{\partial y} = \begin{pmatrix}
            \frac{\partial F_1}{\partial y_1} & \cdots & \frac{\partial F_1}{\partial y_n} \\
            \vdots                            & \ddots & \vdots                            \\
            \frac{\partial F_n}{\partial y_1} & \cdots & \frac{\partial F_n}{\partial y_n} \\
        \end{pmatrix}\]
    Пусть \((a, b) = (a_1 \dots a_m, b_1 \dots b_n)\) --- решение системы и \(\det \left(\frac{\partial F}{\partial y} (a, b)\right) \neq 0\). Тогда \(\exists U(a) \subset \R^m\) и \(\exists ! \ \Phi : P\to Q\in C^r : \forall x \in P \ \ F(x, \Phi(x)) = 0\) такие, что \(\forall x\in U(a)\) \(x, \Phi(x)\) --- тоже решение системы.
    %</онеявномотображениивтерминахсистемы>
\end{theorem}
%<*онеявномотображенииproof>
\begin{proof}\itemfix
    \begin{itemize}
        \item [$1 \Rightarrow 2$:] $F(x, \Phi(x)) = 0 \Rightarrow F'_x(x, \Phi(x)) + F'_y(x, \Phi(x))\Phi'(x) = 0$
        \item [$1$:] $\tilde F: O\to\R^{m+n} : (x, y)\mapsto(x, F(x, y)), \tilde F(a, b) = (a, 0)$
              $$F' = \left(\begin{array}{c|c}
                          E_m  & 0    \\
                          \hline
                          F'_x & F'_y
                      \end{array}\right)$$

              Очевидно $\det \tilde F'\not=0$ в $(a, b)$, значит $\exists U(a, b) : \tilde F\Big|_{U}$ --- диффеоморфизм

              \begin{tikzpicture}
                  \draw[-{>[length=2mm,width=3mm]}] (-1, 0) -- (5, 0) node[anchor=north west] {$x$};
                  \draw[-{>[length=2mm,width=3mm]}] (0, -1) -- (0, 5) node[anchor=south east] {$y$};
                  \draw[green, thick] (0, 3) -- (0, 2) node[anchor=south east] {$P_2$};
                  \draw[green, thick] (3, 0) -- (2, 0) node[anchor=north west] {$P_1$};
                  \filldraw[black] (2.5, 0) circle (1pt) node[anchor=south] {$a$};
                  \filldraw[black] (0, 2.5) circle (1pt) node[anchor=west] {$b$};
                  \draw[dashed] (2, 3) rectangle (3, 2) node[anchor=north west] {$U$};

                  \draw[-{>[length=2mm,width=3mm]}] (7, 0) -- (13, 0) node[anchor=north west] {$\R^m$};
                  \draw[-{>[length=2mm,width=3mm]}] (8, -1) -- (8, 5) node[anchor=south east] {$\R^n$};
                  \draw[green, thick] (11, 0) -- (10, 0) node[anchor=north east] {$P_1$};
                  \filldraw[black] (10.5, 0) circle (1pt) node[anchor=south] {$a$};
                  \draw[] (10.5, 0) \irregularcircle{0.5cm}{1mm} node[anchor=south west] {$P$};

                  \draw[->] (4.5, 4) -- (6.5, 4) node[midway, above] {$\tilde F$};
                  \draw[->] (10, 0.5) -- (3.5, 2.5) node[midway, above] {$\psi$};
              \end{tikzpicture}

              \begin{enumerate}
                  \item $U = P_1 \times Q$ --- можно так считать
                  \item $V = \tilde F(U)$
                  \item $\tilde F$ --- диффеоморфизм на $U \Rightarrow \exists \Psi = \tilde F^{-1} : V\to U$
                  \item $\tilde F$ не меняет первые $m$ координат $\Rightarrow \Psi(u, v) = (u, H(u, v)),  H : V\to\R^n$.
                  \item ``Ось $x$'' $\Leftrightarrow$ ``ось $y$'', $P:=``\text{ось } u'' = \R^m\times{a}\cap V$, $P$ --- откр. в $\R^m$, $P=P_1$
                  \item $\Phi(x) := H(x, 0)$

                        $F\in C^r \Rightarrow \tilde F\in C^r \Rightarrow \Psi\in C^r \Rightarrow H\in C^r \Rightarrow \Phi\in C^r$

                        Единственность: $(x, y) = \Psi(\tilde F(x, y)) = \Psi(x, 0) = (x, H(x, 0)) = (x, \Phi(x))$
              \end{enumerate}
    \end{itemize}
\end{proof}
%</онеявномотображенииproof>

\begin{definition}\itemfix
    \begin{itemize}
        \item $M\subset\R^m$
        \item $k\in\{1 \dots m\}$
    \end{itemize}
    $M$ --- \textbf{простое $k$-мерное \textit{(непрерывное)} многообразие} в $\R^m$, если оно гомеоморфно некоторому открытому множеству $O\subset\R^m$

    Т.е. $\exists \underbrace{\Phi}_{\text{параметризация}} : \underbrace{O}_{\text{откр.}} \subset\R^k \xrightarrow{\text{сюрьекция}} M$ --- непр., обратимо и $\Phi^{-1}$ непрерывно.
\end{definition}

\begin{definition}
    %<*простоеkмерноегладкоемногообразие>
    $M\subset\R^m$ --- \textbf{простое $k$-мерное $C^r$-гладкое многообразие} в $\R^m$, если:
    \begin{itemize}
        \item $\exists \Phi : O\subset\R^k \to\R^m$
        \item $\Phi(O) = M$
        \item $\Phi\in C^r$
        \item $\forall x\in O \ \ \rg \Phi'(x)=k$
    \end{itemize}
    %</простоеkмерноегладкоемногообразие>
\end{definition}

\begin{example}\itemfix
    \begin{enumerate}
        \item Полусфера в $\R^3 = \{(x, y, z)\in\R^3 : r = 0, x^2 + y^2 + z^2 = r^2\}$

              $\Phi : (x, y)\mapsto (x, y, \sqrt{r^2 - x^2 - y^2})$

              $\Phi : B(0, r)\subset\R^2 \to\R^3$

              $\Phi\in C^{\infty}$

              $\Phi' = \begin{pmatrix}
                      1                             & 0                             \\
                      0                             & 1                             \\
                      \frac{-x}{\sqrt{r^2-x^2-y^2}} & \frac{-y}{\sqrt{r^2-x^2-y^2}}
                  \end{pmatrix}, \rg \Phi' = 2$
        \item Цилиндр $= \{(x, y, z)\in\R^3 : x^2 + y^2 = r^2, z\in(a, b)\}$

              $\Phi : [0, 2\pi]\times (0, h) \to\R^3$


              $(\varphi, z)\mapsto (r\cos \varphi, r\sin \varphi, z)$ --- не иньективно.

              Не существует $\Phi : \underbrace{O}_{\text{односвязн.}}\subset\R^2 \to$ цилиндр $\subset\R^3$, потому что топология: в цилиндре есть дырка, в $O$ --- нет.

              Если мы допускаем дырку в $O$, то $(x, y)\mapsto \left(\frac{rx}{\sqrt{x^2+y^2}}, \frac{ry}{\sqrt{x^2+y^2}}, \sqrt{x^2+y^2} - 1\right)$ --- параметризация.
        \item Сфера в $\R^3$ без точки

              $\Phi : (0, 2\pi)\times\left(-\frac{\pi}{2}, \frac{\pi}{2}\right) \to \R^{3}$

              $(\varphi, \psi) \mapsto \begin{pmatrix}
                      R\cos \varphi \cos \psi \\
                      R\sin \varphi \cos \psi \\
                      R\sin \psi
                  \end{pmatrix}$
    \end{enumerate}
\end{example}

\begin{theorem}\itemfix
    %<*озаданиигладкогомногообразиясистемойуравнений>
    \begin{itemize}
        \item $M\subset \R^m$
        \item $1\le k\le m$ (случай $k=m$ тривиален)
        \item $1\le r\le\infty$
        \item $p\in M$
    \end{itemize}

    Тогда эквивалентны следующие утверждения:
    \begin{enumerate}
        \item $\exists U(p) \subset\R^m$ --- окрестность $p$ в $\R^m : M\cap U$ --- $k$-мерное $C^r$-гладкое многообразие.
        \item $\exists \tilde U(p)\subset \R^m$ и функции $f_1, f_2\ldots f_{m-k} : \tilde U\to\R$, все $f_i\in C^r$

              $x\in M \cap \tilde U \Leftrightarrow f_1(x)=f_2(x)=\ldots=0$, при этом $\grad f_1(p)\ldots \grad f_{m-k}(p)$ --- ЛНЗ.
    \end{enumerate}
    %</озаданиигладкогомногообразиясистемойуравнений>
\end{theorem}
%<*озаданиигладкогомногообразиясистемойуравненийproof>
\begin{proof}\itemfix
    \begin{itemize}
        \item [$1 \Rightarrow 2$:] $\Phi$ --- параметризация $O\subset\R^k\to\R^m, \Phi\in C^r, p=\Phi(t^0)$

              $\rg \Phi'(t^0) = k$

              Пусть $\det \left(\cfrac{\partial \Phi_i}{\partial t_j}(t^0)\right)_{i,j = 1\ldots k} \not=0$

              Пусть $L : \R^m\to\R^k$ --- проекция на первые $k$ координат: $(x_1\ldots x_m)\mapsto (x_1\ldots x_k)$

              Тогда $(L \circ \Phi)'$ --- невырожденный оператор $\Rightarrow$ локальный диффеоморфизм. Тогда если $W(t^0)$ --- окрестность точки $t^0$, то $L\circ \Phi : W \to V\subset\R^k$ --- диффеоморфизм.

              \begin{tikzpicture}
                  \draw[-{>[length=2mm,width=3mm]}] (-1, 0) -- (5, 0) node[anchor=north west] {$x$};
                  \draw[-{>[length=2mm,width=3mm]}] (0, -1) -- (0, 5) node[anchor=south east] {$y$};

                  \node at (-0.5, -0.5) {$\R^2$};
                  %   \draw[green, thick] (0, 3) -- (0, 2) node[anchor=south east] {$P_2$};
                  %   \draw[green, thick] (3, 0) -- (2, 0) node[anchor=north west] {$P_1$};
                  \node (t) [draw, circle, minimum size = 20mm, dashed, label=above:$W(t^0)$] at (2.5, 2.5) {$t^0$};
                  % \filldraw[black] (2.5, 2.5) circle (1pt) node[anchor=south] {$t^0$};
                  % \circle{2} 
                  %   \filldraw[black] (0, 2.5) circle (1pt) node[anchor=west] {$b$};
                  %   \draw[dashed] (2, 3) rectangle (3, 2) node[anchor=north west] {$U$};

                  \draw[-{>[length=2mm,width=3mm]}] (7, 0) -- (13, 0) node[anchor=north west] {$\R^k$};
                  \draw[-{>[length=2mm,width=3mm]}] (8, -1) -- (8, 5) node[anchor=south east] {$\R^{m-k}$};
                  %   \draw[green, thick] (11, 0) -- (10, 0) node[anchor=north east] {$P_1$};
                  %   \filldraw[black] (10.5, 0) circle (1pt) node[anchor=south] {$a$};
                  %   \draw[] (10.5, 0) \irregularcircle{0.5cm}{1mm} node[anchor=south west] {$P$};
                  \draw (10, 4) .. controls (12, 3.5) .. (13, 2);
                  \draw (9.5, 3) .. controls (11.5, 2.5) .. (12.5, 1);
                  \draw (10, 4) -- (9.5, 3);
                  \draw (13, 2) -- (12.5, 1);

                  \node (circ) at (10.5, 3.3) {};
                  \draw[] (circ) \irregularcircle{0.4cm}{0.5mm};
                  \draw[] ($ (circ) - (0, 5) $) \irregularcircle{0.4cm}{0.5mm};
                  \draw[->] (circ) -- ($ (circ) - (0, 5) $) node[midway, left] {$L$};

                  \draw[->] ($ (circ) - (0, 5) + (-0.3, 0.3) $) .. controls (7, 1) .. (t) node[midway, above] {$\Psi$};
                  \draw[->] (t) .. controls (6, 0) .. ($ (circ) - (0, 5) + (-0.5, 0) $) node[midway, below] {$L \circ \Phi$};

                  \draw[->] (4.5, 4) -- (6.5, 4) node[midway, above] {$\Phi$};
                  %   \draw[->] (10, 0.5) -- (3.5, 2.5) node[midway, above] {$\psi$};
              \end{tikzpicture}

              Множество $\Phi(W)$ --- график некоторого отображения $H : V\to\R^{m-k}$

              Пусть $\Psi = (L \circ \Phi)^{-1}$

              Берем $x' \in V$, тогда $(x', U(x')) = \Phi(\Psi(x'))$, т.е. $H\in C^r$

              Множество $\Phi(W)$ открыто в $M \Rightarrow \Phi(W) = M \cap \tilde U$, где $\tilde U$ открыто в $\R^m$

              $\tilde U\subset V\times \R^{m-k}$

              Пусть $f_j : \tilde U \to\R, x\mapsto H_j(L(x)) - x_{k+j}$. Тогда $x\in M\cap \tilde U (=\Phi(W)) \Leftrightarrow f_j(x)=0$

              $\begin{pmatrix}
                      \grad f_1(p) \\
                      \vdots       \\
                      \grad f_{m-k}(p)
                  \end{pmatrix} = \begin{pmatrix}
                      \frac{\partial H_1}{\partial x_1}     & \ldots & \frac{\partial H_1}{\partial x_k}     & -1 & 0  & \ldots & 0      \\
                      \vdots                                & \ddots & \vdots                                & 0  & -1 & \ldots & \vdots \\
                      \frac{\partial H_{m-k}}{\partial x_1} & \ldots & \frac{\partial H_{m-k}}{\partial x_k} & 0  & 0  & \ldots & -1     \\
                  \end{pmatrix}$

              $\rg = k \Rightarrow$ ЛНЗ

        \item [$2 \Rightarrow 1$:] $F := (f_1\ldots f_{m-k})$

              $I := \begin{pmatrix}
                      \frac{\partial f_1}{\partial x_1}(p)     & \ldots & \frac{\partial f_1}{\partial x_k}(p)     \\
                      \vdots                                   & \ddots & \vdots                                   \\
                      \frac{\partial f_{m-k}}{\partial x_1}(p) & \ldots & \frac{\partial f_{m-k}}{\partial x_k}(p) \\
                  \end{pmatrix}$

              Градиенты ЛНЗ $\Rightarrow \rg I = m-k$.

              Пусть ранг реализуется на последних $m-k$ столбцах, т.е. $$\det\left(\frac{\partial f_i}{\partial x_{k+j}} (p)\right)_{i, j = 1\ldots m-k} \not = 0$$

              $F(x_1\ldots x_k, x_{k+1}\ldots x_m) = 0$ при $x\in U$

              По т. о неявном отображении:

              $\exists P$ --- окр. $(x_1\ldots x_k)$ в $\R^m$

              $\exists Q$ --- окр. $(x_{k+1}\ldots x_m)$ в $\R^{m-k}$

              $\exists H\in C^r : P\to Q : F(x', H(x')) = 0$ для $x'\in P$

              Тогда $\Phi : P\to\R^m : (x_1\ldots x_k)\mapsto(x_1\ldots x_k, H_1(x_1\ldots x_k), H_2(x_1\ldots x_k) \ldots H_{m-k}(x_1\ldots x_k)$

              $\Phi$ --- гомеоморфизм $P$ и $M \cap \tilde U, \Phi$ --- фактически проекция.

    \end{itemize}
\end{proof}
%</озаданиигладкогомногообразиясистемойуравненийproof>

\begin{corollary}[о двух параметризациях]\itemfix
    %<*одвухпараметризациях>
    \begin{itemize}
        \item $M\subset\R^m$ --- $k$-мерное $C^r$-гладкое многообразие
        \item $p\in M$
        \item $\exists$ две параметризации:

              $\Phi_1 : O_1\subset\R^k \to U(p)\cap M\subset\R^m, \Phi_1(t^0) = p$

              $\Phi_2 : O_2\subset\R^k \to U(p)\cap M\subset\R^m, \Phi_2(s^0) = p$
    \end{itemize}

    Тогда $\exists$ диффеоморфизм $\Psi : O_1\to O_2$, такой что $\Phi_1 = \Phi_2 \circ \Psi$
    %</одвухпараметризациях>
\end{corollary}
%<*одвухпараметризацияхproof>
\begin{proof}\itemfix
    \begin{itemize}[align=left]
        \item [Частный случай:] Пусть $\rg \Phi'_1(t^0), \rg \Phi'_2(s^0)$ достигается на первых $k$ столбцах.

              Тогда $\Phi_1 = \Phi_2 \circ \underbrace{(L \circ \Phi_2)^{-1} \circ (L \circ \Phi_1)}_{\Theta \text{ --- искомый диффеоморфизм}}$
        \item [Общий случай:] $\Phi_1 = \Phi_2 \circ (\Phi_2 \circ L_2)^{-1} \circ (L_2 \circ L_1^{-1}) \circ (L_1\circ \Phi_1)$

              $$L_2 \circ L_1^{-1} = L_2 \circ \Phi_1 \circ (L \circ \Phi_1)^{-1} \in C^r$$

              Гладкость очевидна в силу гладкости всех элементов.

              Невырожденность мы не доказали, поэтому то, что это диффеоморфизм --- ещё не доказано. Возможно, это будет на следующей лекции.
    \end{itemize}
\end{proof}
%</одвухпараметризацияхproof>

\end{document}