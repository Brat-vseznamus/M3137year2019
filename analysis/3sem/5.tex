\documentclass[12pt, a4paper]{article}

\usepackage{lastpage}
\usepackage{mathtools}
\usepackage{xltxtra}
\usepackage{libertine}
\usepackage{amsmath}
\usepackage{amsthm}
\usepackage{amsfonts}
\usepackage{amssymb}
\usepackage{enumitem}
\usepackage{xcolor}
\usepackage[left=1.5cm, right=1.5cm, top=2cm, bottom=2cm, bindingoffset=0cm, headheight=15pt]{geometry}
\usepackage{fancyhdr}
\usepackage[russian]{babel}
% \usepackage[utf8]{inputenc}
\usepackage{catchfilebetweentags}
\usepackage{accents}
\usepackage{calc}
\usepackage{etoolbox}
\usepackage{mathrsfs}
\usepackage{wrapfig}

\providetoggle{useproofs}
\settoggle{useproofs}{false}

\pagestyle{fancy}
\lfoot{M3137y2019}
\rhead{\thepage\ из \pageref{LastPage}}

\newcommand{\R}{\mathbb{R}}
\newcommand{\Q}{\mathbb{Q}}
\newcommand{\C}{\mathbb{C}}
\newcommand{\Z}{\mathbb{Z}}
\newcommand{\B}{\mathbb{B}}
\newcommand{\N}{\mathbb{N}}

\newcommand{\const}{\text{const}}

\newcommand{\teormin}{\textcolor{red}{!}\ }

\DeclareMathOperator*{\xor}{\oplus}
\DeclareMathOperator*{\equ}{\sim}
\DeclareMathOperator{\Ln}{\text{Ln}}
\DeclareMathOperator{\sign}{\text{sign}}
\DeclareMathOperator{\Sym}{\text{Sym}}
\DeclareMathOperator{\Asym}{\text{Asym}}
% \DeclareMathOperator{\sh}{\text{sh}}
% \DeclareMathOperator{\tg}{\text{tg}}
% \DeclareMathOperator{\arctg}{\text{arctg}}
% \DeclareMathOperator{\ch}{\text{ch}}

\DeclarePairedDelimiter{\ceil}{\lceil}{\rceil}
\DeclarePairedDelimiter{\abs}{\left\lvert}{\right\rvert}

\setmainfont{Linux Libertine}

\theoremstyle{plain}
\newtheorem{axiom}{Аксиома}
\newtheorem{lemma}{Лемма}

\theoremstyle{remark}
\newtheorem*{remark}{Примечание}
\newtheorem*{exercise}{Упражнение}
\newtheorem*{consequence}{Следствие}
\newtheorem*{example}{Пример}
\newtheorem*{observation}{Наблюдение}

\theoremstyle{definition}
\newtheorem{theorem}{Теорема}
\newtheorem*{definition}{Определение}
\newtheorem*{obozn}{Обозначение}

\setlength{\parindent}{0pt}

\newcommand{\dbltilde}[1]{\accentset{\approx}{#1}}
\newcommand{\intt}{\int\!}

% magical thing that fixes paragraphs
\makeatletter
\patchcmd{\CatchFBT@Fin@l}{\endlinechar\m@ne}{}
  {}{\typeout{Unsuccessful patch!}}
\makeatother

\newcommand{\get}[2]{
    \ExecuteMetaData[#1]{#2}
}

\newcommand{\getproof}[2]{
    \iftoggle{useproofs}{\ExecuteMetaData[#1]{#2proof}}{}
}

\newcommand{\getwithproof}[2]{
    \get{#1}{#2}
    \getproof{#1}{#2}
}

\newcommand{\import}[3]{
    \subsection{#1}
    \getwithproof{#2}{#3}
}

\newcommand{\given}[1]{
    Дано выше. (\ref{#1}, стр. \pageref{#1})
}

\renewcommand{\ker}{\text{Ker }}
\newcommand{\im}{\text{Im }}
\newcommand{\grad}{\text{grad}}

\lhead{Математический анализ}
\cfoot{}
\rfoot{12.10.2020}

\begin{document}

\begin{lemma}\itemfix
    \begin{itemize}
        \item $\Phi : O\subset\R^k \to\R^m$
        \item $\Phi\in C^r$
        \item $\Phi$ --- параметризация многообразия $U(p)\cap M$, где $p\in M$, $M$ --- гладкое $k$-мерное многообразие $\Rightarrow U(p)\cap M$ --- простое многообразие
    \end{itemize}
    Тогда образ $\Phi'(t^0) : \R^k \to \R^m$ есть $k$-мерное линейное подпространство в $\R^m$. Оно не зависит от $\Phi$.
\end{lemma}
\begin{proof}
    $\rg \Phi'(t^0) = k$ по определению параметризации $\Rightarrow$ искомое очевидно. Если взять другую параметризацию $\Phi_1$, то $$\Phi = \Phi_1 \circ \Psi$$
    $$\Phi' = \Phi'_1 \Psi'$$
    $\Psi'(t^0)$ --- невырожденный оператор $\Rightarrow$ образ $\Phi'$ = образ $\Phi_1'$
\end{proof}

\begin{definition}
    $k$-мерное пространство из леммы --- касательное пространство к $M$ в точке $p$, обозначается $T_pM$.
\end{definition}

\begin{example}
    $M$ --- окружность в $\R^2$, задается параметризацией $\Phi : t\mapsto \begin{pmatrix}
            \cos t \\
            \sin t
        \end{pmatrix}, t^0 := \frac{\pi}{4}$
    $$\Phi'(t^0) = \begin{pmatrix}
            -\frac{\sqrt{2}}{2} \\
            \frac{\sqrt{2}}{2}
        \end{pmatrix}$$
    Тогда $h \mapsto \begin{pmatrix}
            -\frac{\sqrt{2}}{2} \\
            \frac{\sqrt{2}}{2}
        \end{pmatrix} h$
    % \begin{tikzpicture}
    %     \draw[-{>[length=2mm,width=3mm]}] (-3, 0) -- (3, 0) node[anchor=north west] {$x$};
    %     \draw[-{>[length=2mm,width=3mm]}] (0, -3) -- (0, 3) node[anchor=north west] {$y$};
    % \end{tikzpicture}
\end{example}

\begin{example}
    Афинное подпространство $\{p + v, v\in T_pM\}$ --- называется афинным касательным подпространством.
\end{example}

\begin{remark}\itemfix
    \begin{enumerate}
        \item $v\in T_pM$. Тогда $\exists$ путь $\gamma_V : [-\varepsilon, \varepsilon] \to M$, такой что $\gamma(0)=p, \gamma'(0)=v$
        \item Пусть $\gamma : [-\varepsilon, \varepsilon]\to M, \gamma(0) = p$ --- гладкий путь. Тогда $\gamma'(0) \in T_pM$
        \item $f : O\subset\R^m \to\R, y=f(x)$ --- поверхность в $\R^{m+1}$, задается точками $(x, y)$.

              Тогда (аффинная) касательная плоскость в точке $(a, b)$ задается уравнением
              $$y-b = f'_{x_1}(a)(x_1-a_1) + f'_{x_2}(a)(x_2-a_2) + \ldots + f'_{x_m}(a)(x_m-a_m)$$
        \item $\Phi(x_1\ldots x_m) = 0$

              $\Phi : O\subset\R^m \to\R$

              $\Phi(a) = 0$

              Уравнение касательной к плоскости $\Phi'_{x_1}(a)(x_1-a_1) + \ldots + \Phi'_{x_m}(a)(x_m-a_m) = 0$

              %   $\gamma$ --- путь в $M$ : $\Phi(\gamma(s)) = 0, \Phi'(\gamma(s))\gamma'(s) = 0, \?\gamma'(s)$

              По определению дифференцируемости $\Phi$ в точке $a$:
              $$\Phi(x) = \Phi(a) + \Phi'_{x_1}(x_1-a_1) + \ldots + \Phi'_{x_m}(x_m-a_m) + \xcancel{O()}$$

              \?
    \end{enumerate}
\end{remark}
\begin{proof}
    % \begin{tikzpicture}
    %     \draw[-{>[length=2mm,width=3mm]}] (0, 0) -- (5, 0) node[anchor=north west] {$x$};
    %     \draw[-{>[length=2mm,width=3mm]}] (0, 0) -- (0, 5) node[anchor=north west] {$y$};
    %     \draw[-{>[length=2mm,width=3mm]}] (0, 0) -- (-2, -2) node[anchor=north west] {$z$};
    %     \draw (0.5, 3) .. controls (2, 2.5) .. (3, 1);
    %     \draw (0.5, 3) .. controls (2.5, 3.5) .. (4, 3);
    %     \draw (3, 1) -- (4, 3);
    %     \node at (2, -2) {$\R^m$};
    % \end{tikzpicture}
    \begin{enumerate}
        \item $h:=\left(\Phi'(t_0)\right)^{-1}(v)$

              $$\tilde \gamma_v(s) := t_0 + su, s\in[-\varepsilon, \varepsilon]$$
              $$\gamma_v(s) := \Phi(\tilde \gamma_v(s))$$
              $$\gamma_v'(0) = \Phi'(\tilde \gamma_v(0))\tilde \gamma_v(0)' = \Phi'(\tilde \gamma_v(0))u = v$$
        \item $$\gamma(s) = \Phi \Psi L \gamma(\delta)$$
              $$\gamma' = \Phi' \Psi' L' \gamma'(\delta) \in T_{\?} M \text{ при } s=0$$
        \item $\Phi : O\subset\R^m \to\R^{m+1}$

              $\Phi(x) = (x, f(x))$

              $$\Phi' = \begin{bmatrix}
                      1        & 0        & \ldots & 0        \\
                      0        & 1        & \ldots & 0        \\
                      \vdots   & \vdots   & \ddots & \vdots   \\
                      0        & 0        & \ldots & 1        \\
                      f'_{x_1} & f'_{x_2} & \ldots & f'_{x_m}
                  \end{bmatrix}$$
              Рассмотрим произвольный вектор $\begin{pmatrix}
                      \alpha_1 \\
                      \vdots   \\
                      \alpha_m \\
                      \beta
                  \end{pmatrix}$. В каких случаях он принадлежит образу $\Phi'$?

              $$\Phi' \vec x = \begin{bmatrix}
                      1        & 0        & \ldots & 0        \\
                      0        & 1        & \ldots & 0        \\
                      \vdots   & \vdots   & \ddots & \vdots   \\
                      0        & 0        & \ldots & 1        \\
                      f'_{x_1} & f'_{x_2} & \ldots & f'_{x_m}
                  \end{bmatrix} \begin{pmatrix}
                      \alpha_1 \\
                      \vdots   \\
                      \alpha_m \\
                      \beta
                  \end{pmatrix} = \?$$
    \end{enumerate}
\end{proof}

\begin{remark}
    $y(x) = f(a) + f'_{x_1}(a)(x_1-a_1)+\ldots + f'_{x_m}(a)(x_m-a_m)$

    % TODO
    % \begin{tikzpicture}
    %     \draw[-{>[length=2mm,width=3mm]}] (-3, 0) -- (3, 0) node[anchor=north west] {$x$};
    %     \draw[-{>[length=2mm,width=3mm]}] (0, -4) -- (0, 2) node[anchor=north west] {$y$};
    %     \node at (2, 0.3) {$y(x)$};
    % \end{tikzpicture}
\end{remark}

\section{Относительный экстремум}

\begin{example}
    Найти наибольшее/наименьшее значение выражения $f(x, y) = x + y$ при условии $x^2+y^2=1$.

    Рассмотрим линии уровня, т.е. $f(x, y) = C$:

    % \begin{tikzpicture}
    %     \draw[-{>[length=2mm,width=3mm]}] (-3, 0) -- (3, 0) node[anchor=north west] {$x$};
    %     \draw[-{>[length=2mm,width=3mm]}] (0, -3) -- (0, 3) node[anchor=north west] {$y$};
    %     \draw (0, 0) circle (30pt);
    %     \node at (1.2, 0.2) {$1$};
    %     \draw[pattern=north east lines] (-3, -3) rectangle (3, 3);
    % \end{tikzpicture}
\end{example}

\textcolor{red}{Не дописано}

% \begin{definition}
%     $f: O\subset\R^{m+n} \to\R$
% \end{definition}

\end{document}