\chapter{12 апреля}

В третьем семестре у нас был криволинейный интеграл функции и векторного поля вдоль кривой \(\gamma : [a, b] \to \R^m\):
\[\int_\gamma f \underbrace{dS}_{|\gamma'| dt} \quad \int \ev{F, \gamma'} dt\]
Мера на кривой, т.е. гладком одномерном многообразии, с параметризацией \(\gamma\), является мерой Лебега в \(\R^1\) с весом \(|\gamma'|\). Такой интеграл --- первого рода.

В общем случае интеграл II рода по \(m - 1\)-мерной поверхности в \(\R^m\) от векторного поля \(F\) есть: \(\int \ev{F, n_0} dS_{m - 1}\)

\begin{definition}[мера Лебега на \(k\)-мерном многообразии в \(\R^m\)]\itemfix
    \begin{itemize}
        \item \(\Phi : O \subset \R^k \to \R^m\)
        \item \(\exists \Phi'_1 \dots \Phi'_k\)
    \end{itemize}

    \(\lambda_k(\text{ПРЛП}(\Phi_1' \dots \Phi_k'))\)\footnote{Объём параллелепипеда на этих векторах} --- это и будет плотность меры.
\end{definition}

\section{Формула Грина}

\begin{theorem}[формула грина]\itemfix
    %<*формулагрина>
    \label{формула грина}
    \begin{itemize}
        \item \(D\subset \R^2\) --- компактное, связное, односвязное\footnote{Любая петля стягиваема}, ограниченное множество.
        \item \(D\) ограничено кусочно-гладкой кривой \(\partial D\)
        \item \((P, Q)\) --- гладкое векторное поле в окрестности \(D\)
    \end{itemize}

    Пусть \(\partial D\) ориентирована согласованно с ориентацией \(D\) \textit{(против часовой стрелки)} --- обозначим \(\partial D^{ +}\). Тогда:
    \[\iint_D \frac{\partial Q}{\partial x} - \frac{\partial P}{\partial y} dx dy = \int_{\partial D^{ +}} Pdx + Qdy\]
    %</формулагрина>
\end{theorem}
%<*формулагринаproof>
\begin{proof}
    Ограничимся случаем \(D\) --- ``криволинейный четырёхугольник''.

    \begin{figure}[h]
        \includesvg{images/криволинейный_четырехугольник.svg}
        \centering
        \caption{Криволинейный четырёхугольник с \(\partial D\)}
    \end{figure}

    \(\partial D\) состоит из путей \(\gamma_1 \dots \gamma_4\), где \(\gamma_2\) и \(\gamma_4\) --- вертикальные отрезки\footnote{Возможно, вырожденные}, \(\gamma_1\) и \(\gamma_3\) --- гладкие кривые --- можно считать, что это графики функций \(\varphi_1(x), \varphi_3(x)\).

    Аналогично можно описать \(\partial D\) по отрезкам, параллельным оси \(OY\). % TODO: расшифровка сокращений угадана

    Проверим, что \( - \iint_D \frac{\partial P}{\partial y} dx dy = \int_{\partial D^{ +}} Pdx + 0 dy\)

    \begin{align*}
        - \iint_D \frac{\partial P}{\partial y} dx dy & = -\int_a^b dx \int_{\varphi_1(x)}^{\varphi_3(x)} \frac{\partial P}{\partial y} dy \\
                                                      & = -\int_a^b P(x, \varphi_3(x)) - P(x, \varphi_1(x)) dx
    \end{align*}
    \begin{align*}
        \int_{\partial D^{ +}} Pdx + 0 dy & = \int_{\gamma_1} + \underbrace{\int_{\gamma_2}}_0 + \int_{\gamma_3} + \underbrace{\int_{\gamma_4}}_0 \\
                                          & = \int_a^b P(x, \gamma_1(x)) dx - \int_a^b P(x, \gamma_3(x)) dx
    \end{align*}

    Таким образом, искомое доказано.
\end{proof}
%</формулагринаproof>

\begin{remark}
    Теорема верна для любой области \(D\) с кусочно-гладкой границей, которую можно разрезать на криволинейные четырёхугольники.

    Такой разрез --- безобидное действие, которое можно выполнить вертикальным разрезанием по середине:
    \begin{figure}[h]
        \includesvg[scale=0.7]{images/разрез.svg}
        \centering
    \end{figure}

    Кажется, такими разрезами можно достичь искомого в любой области, но мы не будем это утверждать.
\end{remark}

\begin{theorem}[формула Стокса]\itemfix
    %<*формуластокса>
    \begin{itemize}
        \item \(\Omega\) --- простое гладкое двумерное многообразие в \(\R^3\) \textit{(двустороннее)}
        \item \(\Phi : G \subset \R^2 \to \R^3\) --- параметризация \(\Omega\)
        \item \(L^{ +}\) -- граница \(G\)
        \item \(n_0\) --- сторона \(\Omega\)
        \item \(\partial \Omega\) --- кусочно-гладкая кривая
        \item \(\partial \Omega^{ +}\) --- кривая с согласованной ориентацией
        \item \((P, Q, R)\) --- гладкое векторное поле в окрестности \(\Omega\)
    \end{itemize}

    Тогда:
    \[\int_{\partial \Omega^{ +}} P dx + Q dy + R dz = \iint_\Omega \left( \frac{\partial R}{\partial y} - \frac{\partial Q}{\partial z} \right) dy dz + \left( \frac{\partial P}{\partial z} - \frac{\partial R}{\partial x} \right) dz dx + \left( \frac{\partial Q}{\partial x} - \frac{\partial P}{\partial y} \right) dx dy\]
    %</формуластокса>
\end{theorem}

\begin{remark}
    \(dx dy = - dy dx, dx dx = 0\)
    \[d P dx + dQ dy + dR dz = (P'_x dx + P'_y dy + P'_z dz) dx + \dots \]
\end{remark}

%<*формуластоксаproof>
\begin{proof}
    Ограничимся случаем \(\Omega \in C^2\), т.е. параметризация \(\Omega\) дважды гладко дифференцируема.

    Достаточно показать, что:
    \[\int_{\partial \Omega^{ +}} P dx = \iint_\Omega \frac{\partial P}{\partial z} dz dx - \frac{\partial P}{\partial y} dx dy\]

    Пусть \(\Phi = (x(u, v), y(u, v), z(u, v))\).

    Запараметризуем \(L^+\) как \(\gamma : [a, b] \to \R^2, t \mapsto (u(t), v(t))\). Тогда \(\Phi \circ \gamma\) --- параметризация \(\partial\Omega^+\). Тогда \((\Phi \circ \gamma)' = \Phi' \cdot \gamma'\)

    \begin{align*}
        \int_{\partial \Omega^{ +}} P dx & = \int_{L^+} P \left( \frac{\partial x}{\partial u} u' + \frac{\partial x}{\partial v} v' \right) dt                                                                                                           \\
                                         & = \int_{L^{ +}} P \left( \frac{\partial x}{\partial u} du + \frac{\partial x}{\partial v} dv \right)                                                                                                           \\
                                         & \symrefeq{по грину} \iint_G \frac{\partial}{\partial u} \left( P \frac{\partial x}{\partial v} \right) - \frac{\partial}{\partial v} \left( P \frac{\partial x}{\partial u} \right) du dv                      \\
                                         & \symrefeq{дифференцирование произведения} \iint_G (\cancel{P'_x x'_u} + P'_y y'_u + P'_z z'_u)x'_v + \cancel{P \cdot x''_{uv}} - (\cancel{P'_x x'_v} + P'_y y'_v + P'_z z'_v) x'_u - \cancel{P x''_{uv}} du dv \\
                                         & = \iint_G \frac{\partial P}{\partial z} (z'_u x'_v - z'_v x'_u) - \frac{\partial P}{\partial y} (x'_u y'_v - x'_v y'_u) du dv                                                                                  \\
                                         & = \iint_G \frac{\partial P}{\partial z} dz dx - \frac{\partial P}{\partial x} dx dy
    \end{align*}
    \blfootnote{\eqref{по грину}: по \nameref{формула грина}}
    \blfootnote{\eqref{дифференцирование произведения}: это дифференцирование произведения}
\end{proof}
%</формуластоксаproof>

\section{Ряды Фурье \textit{(возвращение)}}

\begin{theorem}\itemfix
    %<*овложении>
    \begin{itemize}
        \item \(\mu E < +\infty\)
        \item \(1 \leq s < r \leq +\infty\)
    \end{itemize}

    Тогда:
    \begin{enumerate}
        \item \(L^r(E, \mu) \subset L^s(E, \mu)\)
        \item \(||f||_s \leq \mu E^{\frac{1}{s} - \frac{1}{r}} \cdot ||f||_r\)
    \end{enumerate}
    %</овложении>
\end{theorem}
%<*овложенииproof>
\begin{proof}
    1 следует из 2, т.к. если \(f \in L^r(E, \mu)\), то \(||f||_s\) конечно. Докажем 2.

    При \(r = \infty\) очевидно:
    \[\left(\int_E |f|^S d\mu \right)^{\frac{1}{s}} \leq \esup |f| \cdot \mu E^{\frac{1}{s}}\]

    При \(r < +\infty\) \(p : = \frac{r}{s}, q : = \frac{r}{r - s}\)

    \begin{align*}
        ||f||_s^s & = \int_E |f|^s d\mu                                                                                                                             \\
                  & = \int_E |f|^s \cdot 1 d\mu                                                                                                                     \\
                  & \leq \left( \int_E |f|^{s \cdot \frac{r}{s}} d\mu \right)^{\frac{s}{r}} \cdot \left( \int_E 1^{\frac{r}{r - s}} d\mu \right)^{\frac{r - s}{r} } \\
                  & \leq ||f||_r^s \mu E^{1 - \frac{s}{r}}
    \end{align*}
\end{proof}
%</овложенииproof>
\begin{corollary}
    \(\mu E < +\infty, 1 \leq s, r \leq +\infty, f_n \xrightarrow{L^r} f\). Тогда \(f_n \xrightarrow{L^s} f\)
\end{corollary}
\begin{proof}
    \(||f_n - f||_s \leq \mu E^{\frac{1}{s} - \frac{1}{r}} \cdot ||f_n - f||_r \to 0\)
\end{proof}

\begin{theorem}[о сходиомсти в \(L^p\) и по мере]\itemfix
    %<*осходимостивlpипомере>
    \begin{itemize}
        \item \(1 \leq p < +\infty\)
        \item \(f_n \in L^p(X, \mu)\)
    \end{itemize}
    Тогда
    \begin{enumerate}
        \item \(f \in L^p, f_n \xrightarrow{L^p} f \Rightarrow f_n \xRightarrow{\mu} f\)
        \item \begin{itemize}
                  \item \(f_n \xRightarrow{\mu} f\) (либо \(f_n \to f\) п.в.)
                  \item \(|f_n| \leq g\)
                  \item \(g \in L^p\)
              \end{itemize}
              Тогда \(f \in L^p\) и \(f_n \to f\) в \(L^p\)
    \end{enumerate}
    %</осходимостивlpипомере>
\end{theorem}
%<*осходимостивlpипомереproof>
\begin{proof}\itemfix
    \begin{enumerate}
        \item Пусть \(X_n(\varepsilon) = X(|f_n - f| \geq \varepsilon)\)
              \[\mu X_n(\varepsilon) = \int_{X_n(\varepsilon)} 1 d\mu \leq \frac{1}{\varepsilon^p} \int_{X_n(\varepsilon)} |f_n - f|^p d\mu \leq \frac{1}{\varepsilon^p} ||f_n - f||_p^p \to 0\]
        \item Пусть \(f_n \Rightarrow f\). Тогда по теореме Рисса \(\exists n_k : f_{n_k} \to f\) почти везде. \(|f| \leq g\) почти везде. \(|f_n - f|^p \leq (2g)^p\) --- суммируема, т.к. \(g \in L^p\).
              \[||f_n - f||_p^p = \int_X |f_n - f|^p \xrightarrow{\text{т. Лебега}} 0\]
    \end{enumerate}
\end{proof}
%</осходимостивlpипомереproof>

\subsection{Напоминание}

\begin{itemize}
    \item Фундаментальная последовательность : \(\forall \varepsilon > 0 \ \ \exists N \ \ \forall k, n > N \ \ ||f_n - f_k|| < \varepsilon\), т.е. \(||f_n - f_k|| \xrightarrow{n,k \to +\infty} 0\)
    \item \(f_n \to f \Rightarrow  (f_n)\) --- фундаментальная, т.к. \(||f_n - f_k|| \leq \underbrace{||f_n - f||}_{ \to 0} + \underbrace{||f - f_k||}_{ \to 0}\)
    \item \(C(K)\) --- пространство непрерывных функций на компакте \(K\).

          \(||f|| = \max_K |f|\). Утверждение: \(C(K)\) --- полное, т.е. любая фундаментальная последовательность сходится.
\end{itemize}

\begin{exercise}
    \(L^{\infty}(X, \mu)\) --- полное % TODO
\end{exercise}

\begin{theorem}
    %<*полнотаlp>
    \(L^p(X, \mu), 1 \leq p < +\infty\) --- полное.
    %</полнотаlp>
\end{theorem}
%<*полнотаlpproof>
\begin{proof}
    Рассмотрим \(f_n\) --- фундаментальную. Куда бы она могла сходиться?

    Пусть \(\varepsilon = \frac{1}{2}\). Тогда \(\exists N_1 \ \ \forall n_1, k > N_1 \ \ ||f_{n_1} - f_k||_p < \frac{1}{2}\). Зафиксируем какой-либо \(n_1\).

    Аналогично для \(\varepsilon = \frac{1}{4}\).

    В общем случае \(\sum_k ||f_{n_{k+1}} - f_{n_k}||_p \leq \sum_k \frac{1}{2^k} = 1\). Рассмотрим ряд \(S(x) = \sum |f_{n_{k+1}}(x) - f_{n_k}(x)|, S(x) \in [0, +\infty]\) и его частичные суммы \(S_N\).
    \[||S_N||_p \leq \sum_{k = 1}^N ||f_{n_{k+1}} - f_{n_k}||_p < 1\]
    Таким образом, \(\int_X S_N^p < 1\). По теореме Фату \(\int_X S^p d\mu < 1\), т.е. \(S^p\) --- суммируемо \( \Rightarrow S\) почти везде конечно.

    \(f(x) = f_{n_1}(x) + \sum_{k = 1}^{+\infty} (f_{n_{k+1}}(x) - f_{n_k}(x))\) --- его частичные суммы это \(f_{n_{N+1}}(x)\), т.е. сходимость этого ряда почти везде означает, что \(f_{n_k} \to f\) почти везде. Таким образом, кандидат --- \(f\). Проверим, что \(||f_n - f||_p \to 0\):
    \[\forall \varepsilon > 0 \ \ \exists N \ \ \forall m,n > N \ \ ||f_n - f_m||_p < \varepsilon\]
    Берём \(m = n_k > N\).
    \[||f_n - f_{n_k}||_p^p = \int_X |f_n - f_{n_k}|^p d\mu < \varepsilon^p\]

    Это выполнено при всех достаточно больших \(k\). Тогда по теореме Фату \(\int_X |f_n - f|^p d\mu < \varepsilon^p\), т.е. \(||f_n - f||_p < \varepsilon\).
\end{proof}
%</полнотаlpproof>

\begin{definition}
    \(Y\) --- метрическое пространство, \(A \subset Y\), \(A\) --- \textbf{\textit{(всюду)} плотно} в \(Y\), если:
    \[\forall y\in Y \ \ \forall U(y) \ \ \exists a \in A : a\in U(y)\]
\end{definition}
\begin{example}
    \(\Q\) плотно в \(\R\).
\end{example}

\begin{lemma}\itemfix
    %<*плотностьступенчатых>
    \begin{itemize}
        \item \((X, \mathfrak{A}, \mu)\)
        \item \(1 \leq p \leq +\infty\)
    \end{itemize}

    Множество ступенчатых функций \textit{(из \(L^p\))} плотно в \(L^p\).
    %</плотностьступенчатых>
\end{lemma}
%<*плотностьступенчатыхproof>
\begin{proof}\itemfix
    \begin{enumerate}
        \item \(p = \infty\)

              \(\sphericalangle f\in L^{\infty}\). Изменив \(f\) на множестве меры \(0\), считаем, что \(|f| \leq ||f||_\infty\), т.к. \(f > A\) на множестве меры \(0\).

              Тогда из доказательство теоремы о характеризации неотрицательных функций с помощью ступенчатых \(\exists\) ступенчатые функции \(\varphi_n\), такие что \(0 \leq \varphi_n \rightrightarrows f^{ +}\) и \(\psi_n\), такие что \(0 \leq \psi_n \rightrightarrows f^{ -}\)

              Тогда сколь угодно близко к \(f\) можно найти ступенчатую функцию вида \(\varphi_n + \psi_n\), т.е. \(|f - \varphi_n - \psi_n| \leq \frac{1}{n}\), что и требовалось показать.

        \item \(p < +\infty\). Пусть \(f \geq 0\).

              \(\exists \varphi_{n} \geq 0\) ступенчатые : \(\varphi_n \uparrow f\)
              \[||\varphi_n - f||_p^p = \int_X \underbrace{|\varphi_n - f|^p}_{ \leq |f|^p \text{ --- мажоранта}} \xrightarrow{\text{т. Лебега}} 0\]

              Если \(f\) любого знака, то при рассмотрении срезок искомое очевидно.
    \end{enumerate}
\end{proof}
%</плотностьступенчатыхproof>

\begin{remark}
    \(\varphi \in L^p\) --- ступенчатая \(\Rightarrow \mu X (\varphi \neq 0) < +\infty\)
\end{remark}

\begin{definition}
    \(f : \R^m \to \R\) --- \textbf{финитная}, если \(\exists B(0, r) : f\equiv 0\) вне \(B(0, r)\).
\end{definition}

\begin{obozn}
    \(C_0(\R^m)\) --- непрерывные финитные функции
\end{obozn}

Очевидно, что \(\forall p \geq 1 \ \ C_0(\R^m) \subset L^p(\R^m, \lambda_m)\)

\begin{definition}
    Топологическое пространство \(X\) \textbf{нормальное}, если:
    \begin{enumerate}
        \item Точки \(X\) суть замкнутые множества
        \item \(\forall F_1, F_2 \subset X\) --- замкнутых\footnote{И непересекающихся, но это не было сказано на лекции.}, тогда \(\exists U(F_1), U(F_2)\) --- открыты, \(U(F_1) \cap U(F_2) = \emptyset\)
    \end{enumerate}
\end{definition}

Загадка: \(\R^m\) --- нормальные.\footnote{Если очень хочется, то \href{https://math.stackexchange.com/questions/2872410/proof-that-every-metric-space-is-normal}{здесь} можно почитать доказательство того, что все метрические пространства \textit{(коим \(\R^m\) является)} нормальные.}
