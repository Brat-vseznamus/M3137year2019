\documentclass[12pt, a4paper]{article}

\usepackage{lastpage}
\usepackage{mathtools}
\usepackage{xltxtra}
\usepackage{libertine}
\usepackage{amsmath}
\usepackage{amsthm}
\usepackage{amsfonts}
\usepackage{amssymb}
\usepackage{enumitem}
\usepackage{xcolor}
\usepackage[left=1.5cm, right=1.5cm, top=2cm, bottom=2cm, bindingoffset=0cm, headheight=15pt]{geometry}
\usepackage{fancyhdr}
\usepackage[russian]{babel}
% \usepackage[utf8]{inputenc}
\usepackage{catchfilebetweentags}
\usepackage{accents}
\usepackage{calc}
\usepackage{etoolbox}
\usepackage{mathrsfs}
\usepackage{wrapfig}

\providetoggle{useproofs}
\settoggle{useproofs}{false}

\pagestyle{fancy}
\lfoot{M3137y2019}
\rhead{\thepage\ из \pageref{LastPage}}

\newcommand{\R}{\mathbb{R}}
\newcommand{\Q}{\mathbb{Q}}
\newcommand{\C}{\mathbb{C}}
\newcommand{\Z}{\mathbb{Z}}
\newcommand{\B}{\mathbb{B}}
\newcommand{\N}{\mathbb{N}}

\newcommand{\const}{\text{const}}

\newcommand{\teormin}{\textcolor{red}{!}\ }

\DeclareMathOperator*{\xor}{\oplus}
\DeclareMathOperator*{\equ}{\sim}
\DeclareMathOperator{\Ln}{\text{Ln}}
\DeclareMathOperator{\sign}{\text{sign}}
\DeclareMathOperator{\Sym}{\text{Sym}}
\DeclareMathOperator{\Asym}{\text{Asym}}
% \DeclareMathOperator{\sh}{\text{sh}}
% \DeclareMathOperator{\tg}{\text{tg}}
% \DeclareMathOperator{\arctg}{\text{arctg}}
% \DeclareMathOperator{\ch}{\text{ch}}

\DeclarePairedDelimiter{\ceil}{\lceil}{\rceil}
\DeclarePairedDelimiter{\abs}{\left\lvert}{\right\rvert}

\setmainfont{Linux Libertine}

\theoremstyle{plain}
\newtheorem{axiom}{Аксиома}
\newtheorem{lemma}{Лемма}

\theoremstyle{remark}
\newtheorem*{remark}{Примечание}
\newtheorem*{exercise}{Упражнение}
\newtheorem*{consequence}{Следствие}
\newtheorem*{example}{Пример}
\newtheorem*{observation}{Наблюдение}

\theoremstyle{definition}
\newtheorem{theorem}{Теорема}
\newtheorem*{definition}{Определение}
\newtheorem*{obozn}{Обозначение}

\setlength{\parindent}{0pt}

\newcommand{\dbltilde}[1]{\accentset{\approx}{#1}}
\newcommand{\intt}{\int\!}

% magical thing that fixes paragraphs
\makeatletter
\patchcmd{\CatchFBT@Fin@l}{\endlinechar\m@ne}{}
  {}{\typeout{Unsuccessful patch!}}
\makeatother

\newcommand{\get}[2]{
    \ExecuteMetaData[#1]{#2}
}

\newcommand{\getproof}[2]{
    \iftoggle{useproofs}{\ExecuteMetaData[#1]{#2proof}}{}
}

\newcommand{\getwithproof}[2]{
    \get{#1}{#2}
    \getproof{#1}{#2}
}

\newcommand{\import}[3]{
    \subsection{#1}
    \getwithproof{#2}{#3}
}

\newcommand{\given}[1]{
    Дано выше. (\ref{#1}, стр. \pageref{#1})
}

\renewcommand{\ker}{\text{Ker }}
\newcommand{\im}{\text{Im }}
\newcommand{\grad}{\text{grad}}

\lhead{Математический анализ}
\cfoot{}
\rfoot{1.3.2021}

\begin{document}

\begin{theorem}[об абсолютной непрерывности интеграла]\itemfix
    \begin{itemize}
        \item \((X, \mathfrak{A}, \mu)\) --- пространство с мерой
        \item \(f : X \to \overline \R\)
        \item \(f\) суммируемо
    \end{itemize}

    Тогда \(\forall \varepsilon > 0 \ \ \exists \delta > 0 \ \ \forall E \text{ --- изм. } \mu E < \delta : \left|\int_E f\right|< \varepsilon\)
\end{theorem}
\begin{corollary}
    \?
\end{corollary}
\begin{proof}
    \[X_n : = X(|f| \geq n)\]
    \[X_n \subset X_{n+1} \subset \dots \quad \mu\left( \bigcap X_n \right) \symrefeq{почти везде конечна} 0\]
    \begin{equation}
        \forall \varepsilon > 0 \ \ \exists n_\varepsilon \ \ \int_{X_{n_\varepsilon}} |f| < \frac{\varepsilon}{2} \label{непрерывность сверху меры}
    \end{equation}

    Пусть \(\delta : = \frac{\varepsilon}{2n_\varepsilon} \). Тогда при \(\mu E < \delta\):
    \[|\int_E f| \leq \int_E |f| = \int_{E\cap X_{n_\varepsilon}} |f| + \int_{E\cap X_{n_\varepsilon}^c} \leq \int_{X_{n_\varepsilon}} |f| + \int_{E\cap X_{n_\varepsilon}^c} n_\varepsilon < \frac{\varepsilon}{2} + \underbrace{\mu E}_{\delta} \cdot n_\varepsilon \leq \varepsilon\]

    \begin{itemize}
        \item \eqref{непрерывность сверху меры}: По непрерывности сверху меры \(A \mapsto \int_A |f| d\mu\)
        \item \eqref{почти везде конечна}: Т.к. \(f\) почти везде конечна.
    \end{itemize}
\end{proof}

\begin{remark}
    Следующие два свойства не эквивалентны:
    \begin{enumerate}
        \item \(f_n \xRightarrow{\mu} f \ \ \forall \varepsilon > 0 \ \ \mu X(|f_n - f| > \varepsilon) \to 0\)
        \item \(\int_X |f_n - f| d\mu \to 0\)
    \end{enumerate}

    Из 1 не следует 2: пусть \((X, \mathfrak{A}, \mu) = (\R, \mathfrak{M}, \lambda), f_n = \frac{1}{nx}\). Тогда \(f_n \xRightarrow{\lambda} 0\), но \(\int |f_n - f|\) \?
\end{remark}

\begin{theorem}[Лебега]\itemfix
    \begin{itemize}
        \item \((X, \mathfrak{A}, \mu)\) --- пространство с мерой
        \item \(f_n, f\) --- измеримо и почти везде конечно
        \item \(f_n \xRightarrow{\mu} f\)
        \item \(\exists g\) : \begin{enumerate}
                  \item \(\forall n \ \ |f_n| \symref{``1''}{\leq} g\) почти везде
                  \item  \(g\) --- суммируемо на \(X\)
              \end{enumerate}
    \end{itemize}

    Тогда: \(f_n, f\) --- суммируемы и \(\int_X |f_n - f| d\mu \xrightarrow{n \to +\infty} 0\), и тем более \(\int_X f_n d\mu \to \int_X f d\mu\)
\end{theorem}
\begin{proof}
    \(f_n\) --- суммируемы в силу неравенства \eqref{``1''}, \(f\) суммируемо в силу следствия теоремы Рисса, тем более \(|\int_X f_n - \int_X f| \leq  \int_X |f_n - f| \to 0\)

    \begin{enumerate}
        \item \(\mu X < +\infty\)

              Зафиксируем \(\varepsilon\). \(X_n : = X(|f_n - f| > \varepsilon)\)

              \(f_n \Rightarrow f\), т.е. \(\mu X_n \to 0\)

              \[\int_X |f_n - f| = \int_{X_n} + \int_{X_n^c} = \underbrace{\int_{X_n} 2g}_{\xrightarrow[\text{сл. т. об абс. непр.}]{n \to +\infty} 0} + \int_{X_n^c} \varepsilon d\mu < \varepsilon + \varepsilon \mu X\]

        \item \(\mu X = +\infty\)
    \end{enumerate}
\end{proof}

\textcolor{red}{Не дописано}

\end{document}