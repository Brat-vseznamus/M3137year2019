\chapter{19 апреля}

\begin{theorem}[Формула Остроградского для \dots]\itemfix
    \begin{itemize}
        \item \(V = \{(x, y, z) : (x, y)\in G \subset \R^2 \ \ f(x, y) \leq z \leq F(x, y) \} \)
        \item \(G\) --- компакт
        \item \(\partial G\) --- кусочно-гладкое
        \item \(f, F\in C^1\)
    \end{itemize}

    Фиксируем внешнюю сторону поверхности, \(\mathbb R :\) окрестность \(V \to \R, \mathbb R \in C^1\)

    Тогда
    \[\iiint_V \frac{\partial R}{\partial z} dxdydz = \iint_{\partial V_\text{внешн.}} R dx dy = \iint_{\partial V} 0 dy dz + 0 dz dx + R dx dy\]
\end{theorem}
\begin{proof}
    % TODO: иллюстрация

    \begin{align*}
        \iiint_V \frac{\partial R}{\partial z} & = \iint_G dx dy \int_{f(x, y)}^{F(x, y)} \frac{\partial R}{\partial z} dz                              \\
                                               & = \iint_G R(x, y, F(x, y)) dx dy - \iint_G R(x, y, f(x, y)) dx dy                                      \\
                                               & = \iint_{\Omega_F} R(x, y, z) dx dy + \iint_{\Omega_f} R dx dy + \underbrace{\iint_{\Omega} R dx dy}_0
    \end{align*}
\end{proof}

\begin{corollary}[обощенная формула Остроградского]
    \[\iiint_V \frac{\partial P}{\partial x} + \frac{\partial Q}{\partial y} + \frac{\partial R}{\partial z} dx dy dz = \iint_{\partial V_{\text{внеш.}}} P dy dz + Q dz dx + R dx dy\]
\end{corollary}

\begin{definition}
    \(V\) --- гладкое векторное поле. Тогда \textbf{дивергенция} \(\div V = \frac{\partial P}{\partial x} + \frac{\partial Q}{\partial y} + \frac{\partial R}{\partial z}\)
\end{definition}

Наблюдение:
\[\div V(a) = \lim_{\varepsilon \to 0} \frac{1}{\frac{4}{3}\pi \varepsilon^3} \iiint_{B(a, \varepsilon)} \div V dx dy dz = \lim_{\varepsilon \to 0} \frac{1}{\frac{4}{3}\pi \varepsilon^3} \iint_{S(a, \varepsilon)} \ev{V, \pi_0} dS\]
, не зависит от координат.

Физический смысл --- мы измеряем поток воды и обнаруживаем, что поток по замкнутой поверхности пропадает или же появляется. Тогда \(\div V\) --- мера\footnote{Не та, что в теории меры; ``измерение''} интенсивности стока/истока.

\begin{corollary}
    \(l \in \R^3, f\in C^1(\text{окр.}(V))\)
    \[\iiint_V \frac{\partial f}{\partial l} dx dy dz = \iint_{\partial V} f \cdot \ev{l, n_0} dS\]
\end{corollary}
\begin{proof}
    Загадка.
\end{proof}

\begin{definition}
    \textbf{Ротор \textit{(вихрь)}}
    \[\rot V = \begin{pmatrix} \frac{\partial R}{\partial y} - \frac{\partial Q}{\partial z} & \frac{\partial P}{\partial z} - \frac{\partial R}{\partial x} & \frac{\partial Q}{\partial x} - \frac{\partial P}{\partial y} \end{pmatrix}\]
\end{definition}

\begin{remark}
    Поле \(V = (P, Q, R)\) --- потенциально \(\Leftrightarrow \exists f : V = \nabla f\). По теореме Пуанкаре при \(\Omega\) --- односвязной: \(V\) --- потенциально \(\Leftrightarrow \rot V = 0\), т.к. \(\rot(\nabla f)\equiv 0\)
\end{remark}

\begin{definition}
    Векторное поле \(A = (A_1,A_2,A_3)\) \textbf{соленоидально} в области \(\Omega \subset \R^3\), если \(\exists \) гладкое векторное поле \(B\) в \(\Omega\), такое что \(A = \rot B\).
\end{definition}

\begin{theorem}[Пуанкаре\({}'\)]\itemfix
    \begin{itemize}
        \item \(\Omega\) --- открытое \?
        \item \(A\) --- векторное поле в \(\Omega\)
        \item \(A \in C^1\)
    \end{itemize}

    Тогда \(A\) --- соленоидально \(\Leftrightarrow \div A = 0\)
\end{theorem}
\begin{proof}
    \begin{itemize}
        \item [\( \Rightarrow \)] \(\div \rot B \equiv 0\)
        \item [\( \Leftarrow \)] Дано: \(A_{1}'{}_x + A_2'{}_y + A_3'{}_z = 0\). Найдём векторный потенциал \(B = (B_1, B_2, B_3)\), где \(A = \rot B\).

              Пусть \(B_3 \equiv 0\).
              \[\begin{cases} B_3'{}_y - B_2'{}_z = A_1 \\ B_1'{}_z - B_3'{}_x = A_2 \\ B_2'{}_x - B_1'{}_y = A_3 \end{cases}\]
              \begin{align}
                  - B_2'{}_z          & = A_1 \label{штука1} \\
                  - B_1'{}_z          & = A_2 \label{штука2} \\
                  B_2'{}_x - B_2'{}_y & = A_3 \label{штука3} \\
              \end{align}

              \begin{align*}
                  \eqref{штука1} \quad B_2 & = - \int_{z_0}^z A_1 dz + \varphi(x, y)                           \\
                  \eqref{штука2} \quad B_1 & = \int_{z_0}^z A_2 dz                                             \\
                  \eqref{штука3} \quad A_3 & =  -\int_{z_0}^z A_1'x dz + \varphi'_x - \int_{z_0}^z A_2'{}_y dz \\
              \end{align*}

              \[\int_{z_0}^z A_3'{}_z + \varphi'_x = A_3 \Leftrightarrow A_3(x, y, z) - A_3(x, y, z_0) + \varphi'_x = A_3(x, y, z) \Leftrightarrow \varphi'_x = A_3(x, y, z_0)\]

              Отсюда найдём \(\varphi = \int_{x_0}^x A_3(x, y, z_0)dx\)
    \end{itemize}
\end{proof}

\begin{lemma}[Урысона]\itemfix
    \begin{itemize}
        \item \(X\) нормальное
        \item \(F_0, F_1 \subset X\) --- замкнутые
        \item \(F_0 \cap F_1 = \emptyset\)
    \end{itemize}

    Тогда \(\exists f : X \to \R\) непрерывное, \(0 \leq f \leq 1\), \(f\Big|_{F_0} \equiv 0, f\Big|_{F_1}\equiv 1\)
\end{lemma}
\begin{proof}
    Переформулируем нормальность: если \(F\subset G\), \(F\) замкнутое, \(G\) открытое, то \(\exists U(F)\) --- открытое, такое что \(F \subset U(F) \subset \overline{U(F)} \subset G\)

    \[F \leftrightarrow F_0 \quad G \leftrightarrow (F_1)^c \quad F_0 \subset \underbrace{U(F_0)}_{G_0} \subset \underbrace{\overline{U(F_0)}}_{\overline{G_0}} \subset \underbrace{F_1^c}_{G_1}\]

    Строим \(G_{\frac{1}{2}}\):
    \[G_0 \subset \overline{G_0} \subset \underbrace{U(\overline{G_0})}_{G_\frac{1}{2}} \subset \underbrace{\overline{U(\overline{G_0})}}_{\overline{G_{\frac{1}{2}}}}\]

    Строим \(G_{\frac{1}{4}}\):
    \[\overline{G_{\frac{1}{2}}} \subset \underbrace{U(\overline{G_{\frac{1}{2}}})}_{G_\frac{1}{4}} \subset \overline{U(\overline{G_{\frac{1}{2}}})}\]

    Таким образом, \(\forall \) двоично рациональной \(\alpha \in [0, 1]\) задаётся открытое множество \(G_\alpha\).

    \[f(x) : = \inf \{\alpha \text{ --- двоично рациональная } : x\in G_\alpha\} \]

    \(f\) --- непрерывно \(\xLeftrightarrow{?} f^{ - 1}(a, b)\) --- всегда открыто.

    Достаточно проверить:
    \begin{enumerate}
        \item \(\forall b \ \ f^{ - 1}( -\infty, b)\) --- открыто
        \item \(\forall a \ \ f^{ - 1}( -\infty, a]\) --- замкнуто
    \end{enumerate}

    \begin{enumerate}
        \item \(f^{ - 1}( - \infty, b) = \bigcup\limits_{\substack{q < b \\ q \text{ дв. рац.}}} G_q\) --- открыто. Почему это так?

              \(f^{ - 1}( - \infty, b) \subset \bigcup\), т.к. \(f(x) = b_0 < b\). Возьмём \(q : b_0 < q < b\). Тогда \(x\in G_q\)

              \(f^{ - 1}( - \infty, b) \supset \bigcup\) очевидно, т.к. при \(x\in G_q \ \ f(x \leq q < b)\).

        \item \unfinished
    \end{enumerate}
\end{proof}

