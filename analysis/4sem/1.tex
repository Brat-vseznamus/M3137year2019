\documentclass[12pt, a4paper]{article}

\usepackage{lastpage}
\usepackage{mathtools}
\usepackage{xltxtra}
\usepackage{libertine}
\usepackage{amsmath}
\usepackage{amsthm}
\usepackage{amsfonts}
\usepackage{amssymb}
\usepackage{enumitem}
\usepackage{xcolor}
\usepackage[left=1.5cm, right=1.5cm, top=2cm, bottom=2cm, bindingoffset=0cm, headheight=15pt]{geometry}
\usepackage{fancyhdr}
\usepackage[russian]{babel}
% \usepackage[utf8]{inputenc}
\usepackage{catchfilebetweentags}
\usepackage{accents}
\usepackage{calc}
\usepackage{etoolbox}
\usepackage{mathrsfs}
\usepackage{wrapfig}

\providetoggle{useproofs}
\settoggle{useproofs}{false}

\pagestyle{fancy}
\lfoot{M3137y2019}
\rhead{\thepage\ из \pageref{LastPage}}

\newcommand{\R}{\mathbb{R}}
\newcommand{\Q}{\mathbb{Q}}
\newcommand{\C}{\mathbb{C}}
\newcommand{\Z}{\mathbb{Z}}
\newcommand{\B}{\mathbb{B}}
\newcommand{\N}{\mathbb{N}}

\newcommand{\const}{\text{const}}

\newcommand{\teormin}{\textcolor{red}{!}\ }

\DeclareMathOperator*{\xor}{\oplus}
\DeclareMathOperator*{\equ}{\sim}
\DeclareMathOperator{\Ln}{\text{Ln}}
\DeclareMathOperator{\sign}{\text{sign}}
\DeclareMathOperator{\Sym}{\text{Sym}}
\DeclareMathOperator{\Asym}{\text{Asym}}
% \DeclareMathOperator{\sh}{\text{sh}}
% \DeclareMathOperator{\tg}{\text{tg}}
% \DeclareMathOperator{\arctg}{\text{arctg}}
% \DeclareMathOperator{\ch}{\text{ch}}

\DeclarePairedDelimiter{\ceil}{\lceil}{\rceil}
\DeclarePairedDelimiter{\abs}{\left\lvert}{\right\rvert}

\setmainfont{Linux Libertine}

\theoremstyle{plain}
\newtheorem{axiom}{Аксиома}
\newtheorem{lemma}{Лемма}

\theoremstyle{remark}
\newtheorem*{remark}{Примечание}
\newtheorem*{exercise}{Упражнение}
\newtheorem*{consequence}{Следствие}
\newtheorem*{example}{Пример}
\newtheorem*{observation}{Наблюдение}

\theoremstyle{definition}
\newtheorem{theorem}{Теорема}
\newtheorem*{definition}{Определение}
\newtheorem*{obozn}{Обозначение}

\setlength{\parindent}{0pt}

\newcommand{\dbltilde}[1]{\accentset{\approx}{#1}}
\newcommand{\intt}{\int\!}

% magical thing that fixes paragraphs
\makeatletter
\patchcmd{\CatchFBT@Fin@l}{\endlinechar\m@ne}{}
  {}{\typeout{Unsuccessful patch!}}
\makeatother

\newcommand{\get}[2]{
    \ExecuteMetaData[#1]{#2}
}

\newcommand{\getproof}[2]{
    \iftoggle{useproofs}{\ExecuteMetaData[#1]{#2proof}}{}
}

\newcommand{\getwithproof}[2]{
    \get{#1}{#2}
    \getproof{#1}{#2}
}

\newcommand{\import}[3]{
    \subsection{#1}
    \getwithproof{#2}{#3}
}

\newcommand{\given}[1]{
    Дано выше. (\ref{#1}, стр. \pageref{#1})
}

\renewcommand{\ker}{\text{Ker }}
\newcommand{\im}{\text{Im }}
\newcommand{\grad}{\text{grad}}

\lhead{Математический анализ}
\cfoot{}
\rfoot{8.2.2021}

\begin{document}

\begin{lemma}[о структуре компактного оператора]\itemfix
    %<*оструктурекомпактногооператора>
    \begin{itemize}
        \item \(V : \R^m \to \R^m\) --- линейный оператор
        \item \(\det V \neq 0\)
    \end{itemize}
    Тогда \(\exists \) ортонормированные базисы \(g_1 \dots g_m\) и \(h_1 \dots h_m\), а также \(\exists s_1 \dots s_m > 0\), такие что:
    \[\forall x\in\R^m \quad V(x) = \sum_{i = 1}^m s_i\langle x, g_i\rangle h_i\]

    И \(|\det V| = s_1s_2 \dots s_m\).
    %</оструктурекомпактногооператора>
\end{lemma}

\begin{remark}
    Эта лемма из функционального анализа, что такое компактный оператор --- мы не знаем.
\end{remark}

%<*оструктурекомпактногооператораproof>
\begin{proof}
    \(W : = V^* V\) --- самосопряженный оператор \textit{(матрица симметрична относительно диагонали)}.

    Из линейной алгебры мы знаем, что такой оператор имеет:
    \begin{itemize}
        \item Собственные числа: \(c_1 \dots c_m\) --- вещественные \textit{(возможно с повторениями)}
        \item Собственные векторы: \(g_1 \dots g_m\) --- ортонормированные
    \end{itemize}

    \begin{remark}
        Пока мы в \(\R^m\) \textit{(а не в \(\mathbb{C}^m\))}, \({}^*\) есть транспонирование. В комплексном случае ещё берется сопряжение.
    \end{remark}

    \[c_i \ev{g_i, g_i} \symrefeq{собственный вектор} \ev{W g_i, g_i} \symrefeq{из линала} \ev{V g_i, V g_i} > 0\]

    \begin{itemize}
        \item \ref{собственный вектор}: т.к. \(g_i\) --- собственный вектор для \(W\) с собственным значением \(c_i\).
        \item \ref{из линала}: из линейной алгебры:
              \[W_{kl} = \sum_{i = 1}^m V_{ik}V_{il}\]
              \[\ev{W g_i, g_i} = \sum_{k, l, j} V_{jk} V_{jl} g_k^{(i)} g_l^{(i)} = \ev{V g_i, V g_i}\]
    \end{itemize}

    Таким образом, \(c_i > 0\).

    \[s_i : = \sqrt{c_i}\]
    \[h_i : = \frac{1}{s_i} V g_i\]

    \[\ev{h_i, h_j} \defeqfor{h_i} \frac{1}{s_i s_j}\ev{V g_i, V g_j} \symrefeq{из линала 2} \frac{1}{s_i s_j}\ev{W g_i, g_j} \symrefeq{собственный вектор 2} \frac{c_i}{s_i s_j}\ev{g_i, g_j} \symrefeq{кронекер} \delta_{ij}\]

    \begin{itemize}
        \item \ref{из линала 2}: из линейной алгебры, аналогично предыдущему.
        \item \ref{собственный вектор 2}: т.к. \(g_i\) --- собственный вектор для \(W\) с собственным значением \(c_i\).
        \item \ref{кронекер}: при \(i \neq j\) \(\ev{g_i, g_j} = 0\) в силу ортогональности, а при \(i = j\) \(\ev{g_i, g_j} = 1\) в силу ортонормированности и \(\frac{c_i}{s_i s_j} = \frac{c_i}{\sqrt{c_i}\sqrt{c_i}} = 1\)
    \end{itemize}

    \begin{remark}
        \(\delta_{ij} = \begin{cases}
            1, & i = j    \\
            0, & i \neq j
        \end{cases}\) --- символ Кронекера.
    \end{remark}

    Таким образом, \(\{h_i\}\) ортонормирован.

    \[V(x) \defeqfor{x} V\left( \sum_{i = 1}^m \ev{x, g_i} g_i \right) \symrefeq{линейность V} \sum_{i = 1}^m \ev{x, g_i} V(g_i) \defeqfor{h_i} \sum s_i \ev{x, g_i} h_i\]

    \begin{itemize}
        \item \ref{линейность V}: в силу линейности \(V\)
    \end{itemize}

    \[(\det V)^2 \symrefeq{мультипликативность det} \det(V^* V) \defeqfor{W} \det W \symrefeq{определитель базис} c_1 \dots c_m\]

    \begin{itemize}
        \item \ref{мультипликативность det}: в силу мультипликативности \(\det\) и инвариантности относительно транспонирования.
        \item \ref{определитель базис}: т.к. \(\det\) инвариантен по базису и в базисе собственных векторов \(\det W = c_1 \dots c_m\).
    \end{itemize}

    \[|\det V| = \sqrt{c_1} \dots \sqrt{c_m} = s_1 \dots s_m\]
\end{proof}
%</оструктурекомпактногооператораproof>

\begin{theorem}[о преобразовании меры Лебега под действием линейного отображения]\itemfix
    %<*опреобразованиимерылебегаподдействиемлинейногоотображения>
    \begin{itemize}
        \item \(V : \R^m \to \R^m\) --- линейное отображение
    \end{itemize}
    Тогда \(\forall E \in \mathfrak{M}^m \ \ V(E) \in \mathfrak{M}^m\) и \(\lambda(V(E)) = |\det V| \cdot \lambda E\)
    %</опреобразованиимерылебегаподдействиемлинейногоотображения>
\end{theorem}

%<*опреобразованиимерылебегаподдействиемлинейногоотображенияproof>
\begin{proof}\itemfix
    \begin{enumerate}
        \item Если \(\det V = 0 \quad \text{Im}(V)\) --- подпространство в \(\R^m\) \( \Rightarrow \lambda(\text{Im}(V)) = 0\) по следствию 6 лекции 15 третьего семестра. Тогда \(\forall E \ \ V(E)\subset \text{Im}(V) \Rightarrow \lambda(V(E)) = 0\)
        \item Если \(\det V \neq 0 \quad \mu E : = \lambda(V(E))\) --- мера, инвариантная относительно сдвигов. Это было доказано в конце прошлого семестра:
              \[\mu(E + a) = \lambda(V(E + a)) = \lambda(V(E) + V(a)) = \lambda(V(E)) = \mu E\]

              \( \Rightarrow \exists k : \mu = k \lambda\) по недоказанной теореме из прошлого семестра.

              Мы хотим найти \(k\), для этого нужно что-нибудь померять. Померяем что-то очень простое, например \(Q = \{\sum \alpha_i g_i\ |\ \alpha_i \in [0, 1]\} \) --- единичный куб на векторах \(g_i\).

              \(V(g_i) = s_i h_i\). Таким образом, \(V(Q) = \{\sum \alpha_i s_i h_i \ |\ \alpha_i \in [0, 1]\} \).

              \[\mu Q = \lambda(V(Q)) = s_1 \dots s_m = |\det V| = |\det V| \underbrace{\lambda Q}_{= 1}\]

              Таким образом, \(k = |\det V|\)
    \end{enumerate}
\end{proof}
%</опреобразованиимерылебегаподдействиемлинейногоотображенияproof>

\section*{Интеграл}

\subsection*{Измеримые функции}

\begin{definition}\itemfix
    \begin{enumerate}
        \item \(E\) --- множество, \(E = \bigsqcup\limits_{\text{кон.}} e_i\) --- разбиение множества.
        \item %<*ступенчатаяфункция>
              \(f : X \to \R\) --- \textbf{ступенчатая}, если:
              \[\exists\ \text{разбиение} \ X = \bigsqcup\limits_{\text{кон.}} e_i : \forall i \ \ f\Big|_{e_i} = \const_i = c_i\]

              При этом разбиение называется \textbf{допустимым} для этой функции.
              %</ступенчатаяфункция>
    \end{enumerate}
\end{definition}

\begin{example}\itemfix
    \begin{enumerate}
        \item Характеристическая функция множества \(E\subset X\) : \(\chi_E(x) = \begin{cases}
                  1, & x\in E            \\
                  0, & x\in X\setminus E
              \end{cases}\)
        \item \(f = \sum\limits_{\text{кон.}} c_i \chi_{e_i}\), где \(X = \bigsqcup e_i\)
    \end{enumerate}
\end{example}

\begin{figure}[h]
    \centering
    \includesvg{images/ступенчатая_функция.svg}
    \caption{Ступенчатая функция}
\end{figure}

\begin{prop}\itemfix
    \begin{enumerate}
        \item \(\forall f, g\) --- ступенчатые:

              \(\exists \) разбиение \(X\), допустимое и для \(f\), и для \(g\):

              \[f = \sum\limits_{\text{кон.}} c_i \chi_{e_i} \quad g = \sum\limits_{\text{кон.}} b_k \chi_{a_k}\]
              \[f = \sum\limits_{i, k} c_i \chi_{e_i \cap a_k} \quad g = \sum\limits_{i, k} b_k \chi_{e_i \cap a_k}\]
        \item \(f, g\) --- ступенчатые, \(\alpha\in\R\)

              Тогда \(f + g, \alpha f, fg, \max(f, g), \min(f, g), |f|\) --- ступенчатые.
    \end{enumerate}
\end{prop}

\begin{definition}
    \(f : E\subset X \to \overline \R, a\in\R\)

    \(E(f < a) = \{x\in E : f(x) < a\} \) --- лебегово множество функции \(f\)

    Аналогично можно использовать \(E(f \leq a), E(f > a), E(f \geq a)\)
\end{definition}

\begin{remark}
    \[E(f \geq a) = E(f < a)^c \quad E(f < a) = E(f \geq a)^c\]
    \[E(f \leq a) = \bigcap_{b > a} E(f < b) = \bigcap_{n\in\N} E\left(f < a + \frac{1}{n}\right)\]
\end{remark}

\begin{definition}\itemfix
    %<*измеримаяфункция>
    \begin{itemize}
        \item \((X, \mathfrak{A}, \mu)\) --- пространство с мерой
        \item \(f : X \to \overline \R\)
        \item \(E\in \mathfrak{A}\)
    \end{itemize}

    \(f\) \textbf{измерима} на множестве \(E\), если \(\forall a\in\R \ \ E(f < a)\) измеримо, т.е. \(\in \mathfrak{A}\)
    %</измеримаяфункция>
\end{definition}

Вместо ``\(f\) измерима на \(X\)'' говорят просто ``измерима''.

Если \(X = \R^m\), мера --- мера Лебега, тогда \(f\) --- измеримо по Лебегу.

\begin{remark}
    Эквивалентны:
    \begin{enumerate}
        \item \(\forall a \ \ E(f < a)\) --- измеримо
        \item \(\forall a \ \ E(f \leq a)\) --- измеримо
        \item \(\forall a \ \ E(f > a)\) --- измеримо
        \item \(\forall a \ \ E(f \geq a)\) --- измеримо
    \end{enumerate}
\end{remark}

\begin{proof}
    Тривиально по соображениям выше.
\end{proof}

\begin{example}\itemfix
    \begin{enumerate}
        \item \(E\subset X, E\) --- измеримо \( \Rightarrow \chi_E\) --- измеримо.
              \[E(\chi_E < a) = \begin{cases}
                      \emptyset,     & a < 0           \\
                      X \setminus E, & 0 \leq a \leq 1 \\
                      X ,            & a > 1
                  \end{cases}\]
        \item \(f:\R^m \to \R\) --- непрерывно. Тогда \(f\) --- измеримо по Лебегу.
              \begin{proof}
                  \(f^{ - 1}(( -\infty, a))\) открыто по топологическому определению открытости, а любое открытое множество измеримо по Лебегу.
              \end{proof}
    \end{enumerate}
\end{example}

\begin{prop}\itemfix
    \begin{enumerate}
        \item \(f\) измеримо на \(E\) \( \Rightarrow \forall a\in\R \ \ E(f = a)\) измеримо.

              В обратную сторону неверно, пример --- \(f(x) = x + \chi_\text{неизм.}\)

        \item \(f\) --- измеримо \( \Rightarrow \forall \alpha\in\R \ \ \alpha f\) --- измеримо.

              \begin{proof}
                  \(E(\alpha f < a) = \begin{cases}
                      E(f < \frac{a}{\alpha}), & \alpha > 0           \\
                      E(f > \frac{a}{\alpha}), & \alpha < 0           \\
                      E,                       & \alpha = 0, a \geq 0 \\
                      \emptyset,               & \alpha = 0, a < 0    \\
                  \end{cases}\)
              \end{proof}
        \item \(f\) --- измеримо на \(E_1, E_2, \dots \Rightarrow f\) измеримо на \(E = \bigcup E_k\)
        \item \(f\) --- измеримо на \(E, E_{\text{изм.}}'\subset E \Rightarrow f\) измеримо на \(E'\)
              \begin{proof}
                  \(E'(f < a) = E(f < a)\cap E'\)
              \end{proof}
        \item \(f \neq 0\), измеримо на \(E \Rightarrow \frac{1}{f}\) измеримо на \(E\).
        \item \(f \geq 0\), измеримо на \(E, \alpha\in\R \Rightarrow f^\alpha\) измеримо на \(E\).

              \textcolor{red}{Это неверно}, т.к. при \(f \equiv 0, \alpha = - 1\) \(\ \not\exists f^\alpha\)
    \end{enumerate}
\end{prop}

\begin{theorem}
    %<*обизмеримостипределовисупрмемумов>
    \(f_n\) --- измеримо на \(X\). Тогда:
    \begin{enumerate}
        \item \(\sup f_n, \inf f_n\) измеримо.
        \item \(\overline \lim f_n, \underline \lim f_n\) измеримо.
        \item Если \(\forall x \ \ \exists \lim\limits_{n \to +\infty} f_n(x) = h(x)\), то \(h(x)\) измеримо.
    \end{enumerate}
    %</обизмеримостипределовисупрмемумов>
\end{theorem}
%<*обизмеримостипределовисупрмемумовproof>
\begin{proof}\itemfix
    \begin{enumerate}
        \item \(g = \sup f_n \quad X(g > a) \symrefeq{два шага} \bigcup_n X(f_n > a)\) и счётное объединение измеримых множеств измеримо.

              \ref{два шага}: \begin{itemize}
                  \item \(X(g > a) \subset \bigcup_n X(f_n > a)\), т.к. если \(x\in X(g > a)\), то \(g(x) > a\).
                        \[\sup_n f_n(x) = g(x) \neq a \Rightarrow \exists n : f_n(x) > a\]
                  \item \(X(g > a) \supset \bigcup_n X(f_n > a)\), т.к. если \(x\in X(f_n > a)\), то \(f_n(x) > a\), следовательно \(g(x) > a\).
              \end{itemize}

        \item \((\overline \lim f_n)(x) = \inf_n (s_n = \sup (f_n(x), f_{n+1}(x), \dots )) \). Т.к. \(\sup\) и \(\inf\) измерим, \(\overline \lim f_n\) тоже измерим.
        \item Очевидно, т.к. если \(\exists \lim\), то \(\lim = \overline \lim = \underline \lim\)
    \end{enumerate}
\end{proof}
%</обизмеримостипределовисупрмемумовproof>

\subsection*{Меры Лебега-Стилтьеса}

\(\R, \mathcal{P}^1, g : \R \to \R\) возрастает, непрерывно.

\(\mu[a, b) : = g(b) - g(a)\) --- \(\sigma\)-конечный объем \textit{(и даже \(\sigma\)-конечная мера на \(\mathcal{P}^1\))}

Также можно определить для монотонной, но непрерывной \(g\). Тогда в точках разрыва \(\exists g(a + 0), g(a - 0)\). Пусть \(\mu[a, b) = g(b - 0) - g(a - 0)\). Такое изменение нужно, потому что исходное \(\mu\) не является объемом для разрывных функций.

Применим теорему о лебеговском продолжении меры. Получим меру \(\mu_g\) на некоторой \(\sigma\)---алгебре. Это мера Лебега-Стилтьеса.

\begin{example}
    \(g(x) = [x]\), тогда мера ячейки --- количество целых точек в этой ячейке.
\end{example}

Если \(\mu_g\) определена на Борелевской \(\sigma\)-алгебре, то она называется мерой Бореля-Стилтьеса.

\end{document}