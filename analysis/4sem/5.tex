\chapter{15 марта}

\begin{definition}\itemfix
    \begin{itemize}
        \item \(X, \mathfrak{A}, \mu\) --- пространство с мерой
        \item \(\nu : \mathfrak{A} \to \overline \R\) --- мера
    \end{itemize}

    \textbf{Плотность меры} \(\nu\) относительно \(\mu\) есть положительная измеримая функция \(\omega : X \to \overline \R\), такая что:
    \[\forall B\in \mathfrak{A} \ \ \nu B = \int_B \omega d\mu\]
\end{definition}

\begin{theorem}[критерий плотности]\itemfix
    \label{критерий плотности}
    \begin{itemize}
        \item \(X, \mathfrak{A}, \mu\) --- пространство с мерой
        \item \(\nu\) --- мера
        \item \(\omega : X \to \overline\R\)
        \item \(\omega \geq 0\)
        \item \(\omega\) измеримо
    \end{itemize}

    Тогда \(\omega\) --- плотность \(\nu\) относительно \(\mu \Leftrightarrow\):
    \[\forall A\in \mathfrak{A} \ \ \mu A \cdot \inf_A \omega \leq \nu(A) \leq \mu A \sup_A \omega\]
    При этом \(0\cdot \infty\) считается \( = 0\).
\end{theorem}

\begin{example}[отсутствие плотности]
    \(X = \R, \mathfrak{A} = \mathfrak{M}^1, \mu = \lambda_1\)

    \(\nu\) --- одноточечная мера: \(\nu(A) =\begin{cases}
        1, & 0\in A     \\
        0, & 0\notin A
    \end{cases}\)

    Необходимое условие существования плотности --- \(\mu A = 0 \Rightarrow \nu A = 0\)

    Это и достаточное условие по теореме Радона-Никодима\footnote{Возможно, мы разберём её в конце семестра.}.
\end{example}

\begin{proof}[Доказательство теоремы~\nameref{критерий плотности}]\itemfix
    \begin{itemize}
        \item [``\( \Rightarrow \)''] Очевидно.
        \item [``\( \Leftarrow \)''] Рассмотрим \(\omega > 0\). Общность не умаляется, т.к. пусть \(e = X(\omega = 0)\), тогда \(\nu(e) \defeq \int_e \omega d\mu = 0\), поэтому в случае \(A\cap e \neq \emptyset\) всё ещё только лучше.

              Фиксируем число \(q\in(0, 1)\).
              \[A_j : = A(q^j \leq \omega < q^{j - 1}), j\in\Z\]
              \[A = \bigsqcup_{j\in \Z} A_j\]
              \[\mu A_j \cdot q^j \symref{неравенство 11}{\leq} \nu A_j \symref{неравенство 12}{\leq} \mu A_j \sup_{A_j} q^{j - 1}\]
              \[\mu A_j \cdot q^j  \symref{неравенство 21}{\leq} \int_{A_j} \omega d\mu \symref{неравенство 22}{\leq} \mu A_j q^{j - 1}\]
              Тогда:
              \begin{align*}
                  q\cdot \int_A \omega d\mu & \leq q\cdot \sum \int_{A_j} \omega d\mu                            \\
                                            & \symref{по неравенство 22}{\leq} \sum q^j \mu A_j                  \\
                                            & \symref{по неравенство 11}{\leq} \underbrace{\sum \nu A_j}_{\nu A} \\
                                            & \symref{по неравенство 12}{\leq} \frac{1}{q} \sum q^j \mu A_j      \\
                                            & \symref{по неравенство 21}{\leq} \frac{1}{q}\sum \int_{A_j}        \\
                                            & = \frac{1}{q}\int_A \omega d\mu
              \end{align*}
              \blfootnote{\eqref{по неравенство 11}: по~\eqref{неравенство 11}}
              \blfootnote{\eqref{по неравенство 12}: по~\eqref{неравенство 12}}
              \blfootnote{\eqref{по неравенство 21}: по~\eqref{неравенство 21}}
              \blfootnote{\eqref{по неравенство 22}: по~\eqref{неравенство 22}}

              То есть:
              \[q \int_A \omega d\mu \leq \nu A \leq \frac{1}{q}\int_A \omega d\mu\]
              Тогда предельный переход при \(q \to 1 - 0\) дает искомое.
    \end{itemize}
\end{proof}

\begin{lemma}\itemfix
    \begin{itemize}
        \item \(f, g\) суммируемы
        \item \((X, \mathfrak{A}, \mu)\) --- пространство с мерой
        \item \(\forall A\in \mathfrak{A} \ \ \int_A f = \int_A g\)
    \end{itemize}

    Тогда \(f = g\) почти везде.
\end{lemma}

\begin{proof}
    \(h: = f - g\). Дано: \(\forall A \ \ \int_A h = 0\); доказать --- \(h = 0\) почти везде.

    \[A_{ +} : = X(h \geq 0) \quad A_{ - } : = X(h < 0) \quad X = A_{ +} \sqcup A_{ -}\]
    \[\int_{A_{ +}} |h| = \int_{A_{ +}} h = 0 \quad \int_{A_{ -}} |h| = -\int_{A_{ -}} h = 0 \implies \int_X |h| = 0 \implies h = 0 \text{ п.в.}\]
\end{proof}

\begin{remark}
    Если \(\mathcal{L}(X)\) --- линейное пространство, отображение \(l_A : f \mapsto \int_A f\) есть линейный функционал. Таким образом, множество функционалов \(\{l_A, A \in \mathfrak{A}\}\) разделяет точки, т.е. \(\forall f \neq g \in \mathcal{L}(X) \ \ \exists A : l_A(f) \neq l_A(g)\)
\end{remark}

\begin{remark}
    В \(\R^m\) \(a = (a_1 \dots a_m)\), \(l_a : x \mapsto a_1x_1 + \dots + a_n x_n\). Тогда \(\forall x, y\in\R^m \ \ \exists a : l_a(x) =\footnote{Кажется, здесь должно быть ``\(\neq\)''}\ l_a(y)\).
\end{remark}

\section{Возвращаемся в \(\R^m\)}

\begin{lemma}[о мере образа малых кубических ячеек]\itemfix
    \label{о мере образа малых кубических ячеек}
    \begin{itemize}
        \item \(\Phi : O \subset \R^m \to \R^m\)
        \item \(O\) открыто
        \item \(a\in O\)
        \item \(\Phi\in C^1\)
        \item \(c > |\det \Phi'(a)| \neq 0\)
    \end{itemize}

    Тогда \(\exists \delta > 0 \ \ \forall \text{ куба } Q\subset B(a, \delta), a\in Q\) выполняется неравенство \(\lambda \Phi(Q) < c \lambda Q\)

    \begin{remark}
        Здесь можно считать, что \(Q\) --- замкнутые кубы.
    \end{remark}
\end{lemma}
\begin{proof}
    \(L: = \Phi'(a)\) --- обратимо\footnote{Т.к. \(\det \Phi'(a) \neq 0\).}
    \[\Phi(x) = \Phi(a) + L (x - a) + o(x - a)\]
    \[\underbrace{a + L^{ - 1}(\Phi(x) - \Phi(a))}_{\Psi(x)} = x + o\footnote{Это не то же самое \(o\), что строчкой выше.}(x - a)\]
    \[\forall \varepsilon > 0 \ \ \exists \text{ шар } B_{\varepsilon\footnotemark}(a) \ \ \forall x \in B_{\varepsilon}(a) \ \ |\Psi(x) - x| < \frac{\varepsilon}{\sqrt{m}}|x - a|\]
    \footnotetext{Это не радиус шара, а параметр.}

    Пусть \(Q \subset B_\varepsilon(a), a \in Q, Q\) --- куб со стороной \(h\).

    При \(x\in Q\):

    \[|x - a| \leq \sqrt{m}h\]
    \[|\Psi(x) - x| \symref{по определению шара}{<} \frac{\varepsilon}{\sqrt{m}} |x - a| \leq \varepsilon h\]
    \blfootnote{\eqref{по определению шара}: т.к. \(x\in B_\varepsilon(a)\)}

    Тогда \(\Psi(Q) \subset \) куб со стороной \((1 + 2\varepsilon)h\), т.к. при \(x,y\in Q\)
    \begin{align*}
        |\Psi(x)_i - \Psi(y)_i| & \leq |\Psi(x)_i - x_i| + |x_i - y_i| + |\Psi(y)_i - y_i| \\
                                & \leq |\Psi(x) - x| + h + |\Psi(y) - y|                   \\
                                & \leq (1 + 2\varepsilon)h
    \end{align*}

    \[\lambda(\Psi(Q)) \leq (1 + 2\varepsilon)^m \cdot \lambda Q\]

    \(\Psi\) и \(\Phi\) отличаются только сдвигом и линейным отображением.
    \[\lambda \Phi(Q) = |\det L| \cdot \lambda \Psi(Q) \leq |\det L| (1 + 2\varepsilon)^m \cdot \lambda Q\]

    Выбираем \(\varepsilon\) такое, чтобы \(|\det L| (1 + 2\varepsilon)^m < c\), потом берём \(\delta =\) радиус \(B_\varepsilon(a)\)
\end{proof}

\begin{lemma}\itemfix
    \label{лемма без имени}
    \begin{itemize}
        \item \(f : O \subset \R^m \to \R\)
        \item \(O\) открыто
        \item \(f\) непрерывна
        \item \(A\) измеримо
        \item \(A \subset Q \subset \overline Q \subset O\)
        \item \(Q\) --- кубическая ячейка
    \end{itemize}

    Тогда:
    \[\inf_{\substack{G : A\subset G \\ G \text{ откр. } \subset O}} \lambda(G) \cdot \sup_G f = \lambda A \cdot \sup_A f\]
\end{lemma}
\begin{proof}
    Упражнение. % TODO: доказать
\end{proof}

\begin{theorem}\itemfix
    \label{мера лебега при диффеоморфизме}
    \begin{itemize}
        \item \(\Phi : O\subset \R^m \to \R^m\)
        \item \(\Phi\) диффеоморфизм
    \end{itemize}

    Тогда
    \[\forall A\in \mathfrak{M}^m, A \subset O \ \ \lambda \Phi(A) = \int_A |\det \Phi'(x)| d\lambda(x)\]
\end{theorem}

\begin{proof}
    \begin{obozn}\itemfix
        \begin{itemize}
            \item \(J_\Phi(x) = |\det \Phi'(x)|\)
            \item \(\nu A : = \lambda \Phi(A)\) --- мера
        \end{itemize}
    \end{obozn}

    Надо доказать, что \(J_\Phi\) --- плотность \(\nu\) относительно \(\lambda\).

    Достаточно проверить условие теоремы~\nameref{критерий плотности}, что \(\forall \) измеримого \(A\):
    \[\inf_A J_\Phi \cdot \lambda A \leq \nu(A) \symref{искомое неравенство}{\leq} \sup_A J_\Phi \cdot \lambda A\]

    Достаточно проверить только правое неравенство, т.к. левое неравенство --- правое неравенство для \(\Phi(A)\) и о��ображения \(\Phi^{-1}\)
    \[\inf \frac{1}{|\det (\Phi')|} \cdot \lambda \Phi(A) \leq \lambda A \]

    \begin{enumerate}
        \item Проверяем~\eqref{искомое неравенство} для случая \(A\) --- кубическая ячейка, \(A\subset\overline A \subset O\)

              От противного: \(\lambda Q \cdot \sup_Q J_\Phi < \nu (Q)\)

              Возьмём \(C > \sup_Q J_\Phi : C \cdot \lambda Q < \nu (Q)\).

              Запускаем половинное деление: режем \(Q\) на \(2^m\) более мелких кубических ячеек. Выберем ``мелкую'' ячейку \(Q_1 \subset Q : C \cdot \lambda Q_1 < \nu Q_1\). Опять делим на \(2^m\) частей, берём \(Q_2 Ж С\cdot \lambda Q_2 < \nu Q_2\) и т.д.

              \(a \in \bigcap \overline Q_i\)

              \begin{equation}
                  Q_1 \supset Q_2 \supset \dots \quad \quad \forall n \ \ C \cdot \lambda Q_n < \nu Q_n \label{свойство кубов}
              \end{equation}
              \(C > \sup_Q J_\Phi = \sup_{\overline Q} J_\Phi\), в частности \(c > |\det \Phi'(a)|\). Мы получили противоречие с леммой~\nameref{о мере образа малых кубических ячеек}: в сколько угодно малой окрестности \(a\) имеются кубы \(\overline Q_n\), где выполнено~\eqref{свойство кубов}

        \item Проверяем~\eqref{искомое неравенство} для случая \(A\) открыто.

              Это очевидно, т.к. \(A = \bigsqcup Q_j, Q_j\) --- кубическая ячейка, \(Q_j \subset \overline Q_j \subset A\)
              \begin{equation}
                  \nu A = \sum \lambda Q_j \leq \sum \mu Q_j \sup_{Q_j} J_\Phi \leq \sup_A J_\Phi \cdot \sum \mu Q_j = \sup_A J_\Phi \cdot \lambda A \label{доказательство для открытых}
              \end{equation}

        \item По лемме~\ref{лемма без имени} неравенство~\eqref{искомое неравенство} выполнено для всех измеримых \(A\):

              \(O = \bigsqcup Q_j\) --- кубы \(Q_j \subset \overline Q_j \subset O\), \(A = \bigsqcup \underbrace{A \cap Q_j}_{A_j}\)
              \[\nu A_j \leq \nu G \leq \sup_G J_\Phi \cdot \lambda G \Rightarrow \nu A_j \leq \inf_G (\sup J_\Phi \cdot \lambda G) = \sup_{A_j} f \cdot \lambda A_j\]

              Аналогично формуле~\eqref{доказательство для открытых} получаем \(\nu A \leq \sup_A f \cdot \lambda A\)
    \end{enumerate}
\end{proof}

\begin{theorem}\itemfix
    \begin{itemize}
        \item \(\Phi : O \subset \R^m \to \R^m\)
        \item \(\Phi\) дифференцируемо
    \end{itemize}

    Тогда \(\forall\) измеримой \(f \geq 0\), заданной на \(O' = \Phi(O)\):
    \[\int_{O'} f(y)d\lambda = \int_O f(\Phi(x)) \cdot J_\Phi \cdot d \lambda, J_\Phi(x) = |\det \Phi'(x)|\]

    То же самое верно для суммируемой \(f\).
\end{theorem}
\begin{proof}
    Применяем теорему~\nameref{о вычислении интеграла по взвешенному образу меры} при \(X = Y = \R^m, \mathfrak{A} = \mathfrak{B} = \mathfrak{M}^m, \mu = \lambda, \nu(A) = \lambda(\Phi(A))\):
    \[\int_B f d\nu = \int_{\Phi^{-1}B} f(\Phi(x)) \omega(x) d\mu\]

    По теореме~\ref{мера лебега при диффеоморфизме} \(\lambda(B) = \int_{\Phi^{-1}(O)} J_\Phi d \lambda\), т.е. \(\lambda\) --- взвешенный образ исходной меры по отношению к \(\Phi\).
\end{proof}

\begin{example}\itemfix
    \begin{enumerate}
        \item Полярные координаты в \(\R^2\):
              \[\Phi = \begin{cases} x = r\cos \varphi \\ y = r\sin \varphi \end{cases} \quad \Phi : \{(r, \varphi), r > 0, \varphi\in(0, 2\pi)\} \to \R^2\]
              \[\Phi' = \begin{pmatrix} \cos \varphi & - r \sin \varphi \\ \sin \varphi & r \cos \varphi \end{pmatrix} \quad \det \Phi' = r \quad J_\Phi = r \]
              \[\iint_\Omega f(x, y)d\lambda_r = \iint_{\Phi^{ - 1}(\Omega)} f(r \cos \varphi, r \sin \varphi) \cdot r d \lambda_2\]
        \item Сферические координаты в \(\R^3\):

              \[\begin{cases}
                      x = r\cos \varphi\cos \psi \\
                      y = r\sin \varphi\cos \psi \\
                      z = r\sin \psi
                  \end{cases}\]
              \[\begin{cases}
                      r > 0                 \\
                      \varphi \in (0, 2\pi) \\
                      \psi\in \left( -\frac{\pi}{2}, \frac{\pi}{2} \right)
                  \end{cases}\]

              \[\Phi' = \begin{pmatrix}
                      \cos \varphi\cos \psi  & - r\sin \varphi \cos \psi & - r\cos \varphi \sin \psi \\
                      \sin \varphi \cos \psi & r\cos \varphi\cos \psi    & - r\sin \varphi\sin \psi  \\
                      \sin \psi              & 0                         & r\cos \psi
                  \end{pmatrix} \quad J_\Phi = r^2\cos \psi\]
              \[\det \Phi' = r^2(\sin^2\psi\cos \psi + \cos^3\psi) = r^2\cos \psi\]
    \end{enumerate}
\end{example}

