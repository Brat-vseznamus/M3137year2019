\documentclass[12pt, a4paper]{article}

\usepackage{lastpage}
\usepackage{mathtools}
\usepackage{xltxtra}
\usepackage{libertine}
\usepackage{amsmath}
\usepackage{amsthm}
\usepackage{amsfonts}
\usepackage{amssymb}
\usepackage{enumitem}
\usepackage{xcolor}
\usepackage[left=1.5cm, right=1.5cm, top=2cm, bottom=2cm, bindingoffset=0cm, headheight=15pt]{geometry}
\usepackage{fancyhdr}
\usepackage[russian]{babel}
% \usepackage[utf8]{inputenc}
\usepackage{catchfilebetweentags}
\usepackage{accents}
\usepackage{calc}
\usepackage{etoolbox}
\usepackage{mathrsfs}
\usepackage{wrapfig}

\providetoggle{useproofs}
\settoggle{useproofs}{false}

\pagestyle{fancy}
\lfoot{M3137y2019}
\rhead{\thepage\ из \pageref{LastPage}}

\newcommand{\R}{\mathbb{R}}
\newcommand{\Q}{\mathbb{Q}}
\newcommand{\C}{\mathbb{C}}
\newcommand{\Z}{\mathbb{Z}}
\newcommand{\B}{\mathbb{B}}
\newcommand{\N}{\mathbb{N}}

\newcommand{\const}{\text{const}}

\newcommand{\teormin}{\textcolor{red}{!}\ }

\DeclareMathOperator*{\xor}{\oplus}
\DeclareMathOperator*{\equ}{\sim}
\DeclareMathOperator{\Ln}{\text{Ln}}
\DeclareMathOperator{\sign}{\text{sign}}
\DeclareMathOperator{\Sym}{\text{Sym}}
\DeclareMathOperator{\Asym}{\text{Asym}}
% \DeclareMathOperator{\sh}{\text{sh}}
% \DeclareMathOperator{\tg}{\text{tg}}
% \DeclareMathOperator{\arctg}{\text{arctg}}
% \DeclareMathOperator{\ch}{\text{ch}}

\DeclarePairedDelimiter{\ceil}{\lceil}{\rceil}
\DeclarePairedDelimiter{\abs}{\left\lvert}{\right\rvert}

\setmainfont{Linux Libertine}

\theoremstyle{plain}
\newtheorem{axiom}{Аксиома}
\newtheorem{lemma}{Лемма}

\theoremstyle{remark}
\newtheorem*{remark}{Примечание}
\newtheorem*{exercise}{Упражнение}
\newtheorem*{consequence}{Следствие}
\newtheorem*{example}{Пример}
\newtheorem*{observation}{Наблюдение}

\theoremstyle{definition}
\newtheorem{theorem}{Теорема}
\newtheorem*{definition}{Определение}
\newtheorem*{obozn}{Обозначение}

\setlength{\parindent}{0pt}

\newcommand{\dbltilde}[1]{\accentset{\approx}{#1}}
\newcommand{\intt}{\int\!}

% magical thing that fixes paragraphs
\makeatletter
\patchcmd{\CatchFBT@Fin@l}{\endlinechar\m@ne}{}
  {}{\typeout{Unsuccessful patch!}}
\makeatother

\newcommand{\get}[2]{
    \ExecuteMetaData[#1]{#2}
}

\newcommand{\getproof}[2]{
    \iftoggle{useproofs}{\ExecuteMetaData[#1]{#2proof}}{}
}

\newcommand{\getwithproof}[2]{
    \get{#1}{#2}
    \getproof{#1}{#2}
}

\newcommand{\import}[3]{
    \subsection{#1}
    \getwithproof{#2}{#3}
}

\newcommand{\given}[1]{
    Дано выше. (\ref{#1}, стр. \pageref{#1})
}

\renewcommand{\ker}{\text{Ker }}
\newcommand{\im}{\text{Im }}
\newcommand{\grad}{\text{grad}}

\lhead{Математический анализ}
\cfoot{}
\rfoot{22.2.2021}

\begin{document}

\section*{Интеграл \textit{(продолжение)}}

\begin{definition}
    Если оказалось, что \(\int_X f^{ +}, \int_X f^{ -}\) оба конечны, то \(f\) называется \textbf{суммируемой}.
\end{definition}

\begin{remark}\itemfix
    \begin{enumerate}
        \item Если \(f\) измеримо и \( \geq \), то интеграл определения 3 = интегралу определения 2.
    \end{enumerate}
\end{remark}

\begin{definition}[4]\itemfix
    \begin{itemize}
        \item \(E\subset X\) --- измеримо
        \item \(f\) измеримо на \(X\)
    \end{itemize}
    \[\int_E f d\mu : = \int_X f \cdot \chi_E\]
\end{definition}

\begin{remark}\itemfix
    \begin{itemize}
        \item \(f = \sum \alpha_k \chi_{E_k} \Rightarrow \int_E f = \sum \alpha_k \mu (E_k\cap E)\)
        \item \(\int_E f d\mu = \sup \{\int_E g : 0 \leq g \leq f \text{ на } E, g \text{ --- ступ.}\} \) и мы считаем, что \(g \equiv 0\) вне \(E\).
        \item \(\int_E f\) не зависит от значений \(f\) вне множества \(E\).
    \end{itemize}
\end{remark}

\begin{prop}\itemfix
    \((X, \mathfrak{A}, \mu)\) --- пространство с мерой, \(E\subset X\) --- измеримо, \(g, f\) --- измеримо.
    \begin{enumerate}
        \item Монотонность \(f \leq g : \int_E f \leq \int_E g\)

              \begin{proof}\itemfix
                  \begin{enumerate}
                      \item При \(f, g \geq 0\) --- очевидно из определения.
                      \item При произвольных \(f, g\ \) \(f^{ +} \leq g^{ +}\) и \(f^{ -} \geq g^{ -}\) \textit{(очевидно из определения)}. Из предыдущего случая \(\int_E f^{ +} \leq \int_E g^{ +}, \int_E f^{ -} \geq \int_E g^{ -}\).
                  \end{enumerate}
              \end{proof}

        \item \(\int_E 1 d\mu = \mu E, \int_E 0 d\mu = 0\)
        \item \(\mu E = 0 \Rightarrow \int_E f = 0\)

              \begin{proof}\itemfix
                  \begin{enumerate}
                      \item \(f\) --- ступ. Тривиально.
                      \item \(f\) --- измеримо, \(f \geq 0\). \(\sup 0 = 0\), поэтому искомое выполнено.
                      \item \(\int f^{ +}, \int f^{ -} = 0 \Rightarrow \int f = 0\)
                  \end{enumerate}
              \end{proof}

              \begin{remark}
                  \(f\) --- измерима. Тогда \(f\) суммируема \(\Leftrightarrow \int |f| < +\infty\)
              \end{remark}
              \begin{proof}\itemfix
                  \begin{itemize}
                      \item [ \( \Leftarrow \)] следует из \(f^{ +}, f^{ -} \leq |f|\)
                      \item [ \( \Rightarrow \)] будет доказано позже на этой лекции. \label{доказано позже}
                  \end{itemize}
              \end{proof}

        \item \(\int_E ( - f) = - \int_E f, \forall c\in\R \ \ \int_E cf = c\int_E f\)

              \begin{proof}\itemfix
                  \begin{enumerate}
                      \item \(( - f)^{ +} = f^{ -}, ( - f)^{ -} = f^{ +}\), тогда искомое очевидно.
                      \item Можно считать \(c > 0\) без потери общности, тогда для \(f \geq 0\) тривиально.
                  \end{enumerate}
              \end{proof}

        \item \(\exists \int_E f d\mu\). Тогда \(|\int_E f d\mu| \leq \int_E |f|d\mu\)

              \begin{proof}
                  \[ -|f| \leq f \leq |f|\]
                  \[ -\int |f| \leq \int f \leq \int |f|\]
                  \[ \left|\int f\right| \leq \int |f| \]
              \end{proof}

        \item \(\mu E < +\infty, a \leq f \leq b\). Тогда
              \[a\mu E \leq \int_E f \leq b\mu E\]

              \begin{corollary}
                  \(f\) --- измеримо на \(E\), \(f\) --- ограничено на \(E\), \(\mu E < +\infty\). Тогда \(f\) суммируемо на \(E\)
              \end{corollary}

        \item \(f\) суммируема на \(E\). Тогда \(f\) почти везде конечна.
              \begin{proof}\itemfix
                  \begin{enumerate}
                      \item \(f \geq 0\) и \(f = +\infty\) на \(A\subset E\). Тогда \(\int_E f \geq n \mu A \ \ \forall n\in\N\) \( \Rightarrow \mu A = 0\)
                      \item В произвольном случае аналогично со срезками.
                  \end{enumerate}
              \end{proof}
    \end{enumerate}
\end{prop}

\begin{lemma}\itemfix
    \begin{itemize}
        \item \(A = \bigsqcup\limits_{i = 1}^{+\infty} A_i\) --- измеримо
        \item \(g\) --- ступенчато
        \item \(g \geq 0\)
    \end{itemize}
    Тогда \[\int_A g d\mu = \sum_{i = 1}^{+\infty} \int_{A_i} g d\mu\]
\end{lemma}
\begin{proof}
    \begin{align*}
        \int_A g d\mu & = \sum_{\text{кон.}} \alpha_k \mu(E_k\cap A)                     \\
                      & = \sum_k \sum_i \underbrace{\alpha_k \mu(E_k\cap A_i)}_{ \geq 0} \\
                      & \symrefeq{переставлять можно} \sum_i \sum_k \dots                \\
                      & = \sum_i \int_{A_i} g d\mu
    \end{align*}

    \ref{переставлять можно}: переставлять можно, т.к. члены суммы \( \geq 0\).
\end{proof}

\begin{theorem}\itemfix
    \begin{itemize}
        \item \(A = \bigsqcup A_i\) --- измеримо
        \item \(f : X \to \overline \R\) --- измеримо на \(A\)
        \item \(f \geq 0\)
    \end{itemize}

    Тогда \[\int_A f d\mu = \sum_{i = 1}^{+\infty} \int_{A_i} f d\mu\]
\end{theorem}
\begin{proof}\itemfix
    Докажем, что части равенства \( \leq \) и \( \geq \), тогда равенство выполнено.

    \begin{itemize}
        \item [ \( \leq \)] \(\sphericalangle g : 0 \leq g \leq f\)
              \[\int_A g \symrefeq{по лемме об интеграле} \sum\int_{A_i} g \leq \sum \int_{A_i} f\]

        \item [ \( \geq \)]
              \begin{enumerate}
                  \item \(A = A_1 \sqcup A_2\)

                        \(\sphericalangle 0 \leq g_1 \leq f \chi_{A_1}, 0 \leq g_2 \leq f \chi_{A_2}\). Пусть \(E_k\) --- совместное разбиение, у \(g_1\) коэффициенты \(\alpha_k\), у \(g_2\) : \(\beta_k\).

                        \begin{align*}
                            0 \leq g_1 + g_2                                     & \leq f \chi_A \\
                            \int_{A_1} g_1 + \int_{A_2} g_2 = \int_A (g_1 + g_2) & \leq \int_A f \\
                            \int_{A_1} f + \int_{A_2} g_2                        & \leq \int_A f \\
                            \int_{A_1} f + \int_{A_2} f                          & \leq \int_A f \\
                        \end{align*}

                  \item \(A = \bigsqcup A_i\) тривиально по индукции.
                  \item \(A = \bigsqcup_{i = 1}^n A_i \cup B_n\), где \(B_n = \bigsqcup_{i > n} A_i\)
                        \[\int_A f = \sum_{i = 1}^n \int_{A_i} f + \int_{B_n} f \geq \sum_{i = 1}^n \int A_i f\]
              \end{enumerate}
    \end{itemize}

    \ref{по лемме об интеграле}: по лемме об интеграле.
\end{proof}

\begin{corollary}
    \(f \geq 0\) --- измеримо. Пусть \(\nu : \mathfrak{A} \to \overline \R_{ +}\) и \(\nu E : = \int_E f d\mu\). Тогда \(\nu\) --- мера.
\end{corollary}

\begin{corollary}[Счётная аддитивность интеграла]
    \(f\) суммируема на \(A = \bigsqcup A_i\) --- измеримо. Тогда
    \[\int_A f = \sum \int_{A_i} f\]
\end{corollary}
\begin{proof}
    Очевидно, если рассмотреть срезки.
\end{proof}
\begin{corollary}
    \(A\subset B, f \geq 0 \Rightarrow \int_A f \leq \int_B f\)
\end{corollary}

\subsection*{Предельный переход под знаком интеграла}

Пусть \(f_n \to f\). Можно ли утверждать, что \(\int_E f_n \to \int_E f\)?

\begin{example}[контр]
    \[f_n : = \frac{1}{n} \chi_{[0, n]} \quad f \equiv 0 \quad f_n \to f \quad (\text{даже } f_n \rightrightarrows f)\]
    \[\int_\R f_n = \frac{1}{n}\lambda[0, n] = 1 \not\to 0 = \int_\R f\]
\end{example}

\begin{theorem}[Леви]\itemfix
    \begin{itemize}
        \item \((X, \mathfrak{A}, \mu)\) --- пространство с мерой
        \item \(f_n\) измеримо
        \item \(\forall n \ \ 0 \leq f_n \leq f_{n+1}\) почти везде.
        \item \(f(x) : = \lim\limits_{n \to +\infty} f_n(x)\) --- эта функция определена почти везде.
    \end{itemize}

    Тогда \[\lim_{n \to +\infty} \int_X f_n d\mu = \int_X f d\mu\]
    \begin{remark}
        \(f\) задано везде, кроме множества \(e\) меры \(0\). Считаем, что \(f = 0\) на \(e\). Тогда \(f\) измеримо на \(X\).
    \end{remark}
\end{theorem}
\begin{proof}\itemfix
    \begin{itemize}
        \item [ \( \leq \)] очевидно, т.к. \(\int f_n \leq f\) почти везде, таким образом:
              \[\int_X f_n = \int_{X\setminus e} f_n + \underbrace{\int_e f_n}_0 = \int_{X\setminus e} f_n \leq \int_{X\setminus e} f \leq \int_{X} f\]
        \item [ \( \geq \)] достаточно проверить, что \(\forall \) ступенчатой \(g : 0 \leq g < f\) выполняется следующее \(\lim \int_X f_n \geq \int_X g\)

              Сильный трюк: то достаточно проверить, что \(\forall c\in(0, 1) \ \ \lim \int_X f_n \geq c \int_X g\)

              \[E_n : = X(f_n \geq cg) \quad E_1 \subset E_2 \subset \dots\]
              \(\bigcup E_n = X\), т.к. \(c < 1\)
              \[\int_X f_n \geq \int_{E_n} f_n \geq c \int_{E_n} g\]
              Тогда \(\lim \int_X f_n \geq c\cdot \lim \int_{E_n}g \symrefeq{непрерывность снизу} c\int_X g\)
    \end{itemize}

    \ref{непрерывность снизу}: по непрерывности снизу меры \(\nu : E \mapsto \int_E g\)
\end{proof}

\begin{theorem}\itemfix
    \begin{itemize}
        \item \(f, g \geq 0\)
        \item \(f, g\) измеримо на \(E\)
    \end{itemize}
    Тогда \(\int_E f + g = \int_E f + \int_E g\)
\end{theorem}
\begin{proof}\itemfix
    \begin{enumerate}
        \item \(f, g\) --- ступенчатые, т.е. \(f = \sum \alpha_k \chi_{E_k}, g = \sum \beta_k \chi_{E_k}\)
              \[\int_E f + g = \sum (\alpha_k + \beta_k)\mu(E_k\cap E) = \sum \alpha_k \mu (E_k \cap E) + \sum \beta_k \mu (E_k \cap E) = \int_E f + \int_E g\]

        \item \(f \geq 0\), измеримо. \(\exists \text{ступ. } f_n : 0 \leq f_n \leq f_{n+1} \leq \dots \ \ \lim f_n = f\)

              \(g \geq 0\), измеримо. \(\exists \text{ступ. } g_n : 0 \leq g_n \leq g_{n+1} \leq \dots \ \ \lim g_n = g\)

              \begin{align*}
                  f_n + g_n               & \to f + g                                 \\
                  \int_E f_n + g_n        & \xrightarrow{\text{т. Леви}} \int_E f + g \\
                  \int_E f_n + \int_E g_n & \to \int_E f + \int_e g                   \\
              \end{align*}
    \end{enumerate}
\end{proof}

\begin{corollary}
    \(f, g\) суммируемы на \(E\). Тогда \(f + g\) суммируемо и \(\int_{E} f + g = \int_E f + \int_E g\). Таким образом, доказано \ref{доказано позже}.
\end{corollary}

\begin{proof}[суммируемости]
    \(|f + g| \leq |f| + |g|\). Пусть \(h = f + g\). Тогда
    \begin{align*}
        h^{ +} - h^{ -}                                & = f^{ +} - f^{ -} + g^{ +} - g^{ -}                               \\
        h^{ +} + f^{ -} + g^{ - }                      & = f^{ + } + g^{ +} + h^{ -}                                       \\
        \int_E h^{ +} + \int_E f^{ -} + \int_E g^{ - } & = \int_E f^{ +} + \int_E g^{ +} + \int_E h^{ - }                  \\
        \int_E h^{ +} - \int_E f^{ - }                 & = \int_E f^{ +} + \int_E g^{ +} - \int_E f^{ - } - \int_E g^{ - } \\
    \end{align*}
\end{proof}
\begin{definition}
    \(\mathcal{L}(X)\) --- множество суммируемых функций на \(X\)
\end{definition}
\begin{corollary}[следствия]
    \(\mathcal{L}(X)\) --- линейное пространство, а отображение \(f \mapsto \int_X f\) это линейный функционал на \(\mathcal{L}(X)\) % TODO: определить функционал
    , т.е. \(\forall f_1 \dots f_n \in \mathcal{L}(X) \ \ \forall \alpha_1 \dots \alpha_n \in\R\)

    \? % TODO: дописать
\end{corollary}

\begin{theorem}[об интегрировании положительных рядов]\itemfix
    \begin{itemize}
        \item \((X, \mathfrak{A}, \mu)\) --- пространство с мерой
        \item \(E\in \mathfrak{A}\)
        \item \(u_n : X \to \overline\R\)
        \item \(u_n \geq 0\) почти везде
        \item \(u_n\) измеримо
    \end{itemize}

    Тогда
    \[\int_E \left( \sum_{n = 1}^{+\infty} u_n(x) \right) d\mu = \sum_{n = 1}^{+\infty} \int_E u_n d\mu\]
\end{theorem}
\begin{proof}
    По теореме Леви:

    \[S_n : = \sum_{k = 1}^n u_k \quad 0 \leq S_n \leq S_{n+1} \leq \dots \]
    Пусть \(S_n \to S\). Тогда \(\int_E S_n \to \int_E S\)

    % TODO: обосновать
\end{proof}

\begin{corollary}
    \(u_n\) измеримо и
    \(\sum\limits_{n = 1}^{+\infty} \int_E |u_n| < +\infty\). Тогда ряд \(\sum u_n(x)\) абсолютно сходится при почти всех \(x\).
\end{corollary}
\begin{proof}
    \[S(x) : = \sum |u_n(x)|\]
    \[\int_E S(X) = \sum \int_E |u_n|< +\infty \Rightarrow S \text{ суммируемо} \Rightarrow S \text{ почти везде конечно}\]
\end{proof}

\begin{example}
    \(x_n\in\R\) --- произвольная последовательность, \(\sum a_n\) абсолютно сходится.

    Тогда \(\sum \cfrac{a_n}{\sqrt{|x - x_n|}}\) абсолютно сходится при почти всех \(x\).
\end{example}
\begin{proof}
    Достаточно проверить абсолютную сходимость на \([ - N, N]\) почти везде.

    \begin{align*}
        \int_{[ - N, N]} \frac{|a_n| d\lambda}{\sqrt{|x - x_n|}}
         & = \int_{ - N}^N \frac{|a_n|}{\sqrt{|x - x_n|}} dx        \\
         & = |a_n|\int_{ - N - x_n}^{N - x_n} \frac{dx}{\sqrt{|x|}} \\
         & \leq |a_n|\int_{ - N}^{N} \frac{dx}{\sqrt{|x|}}          \\
         & 4 \sqrt{N} |a_n|
    \end{align*}
\end{proof}

\end{document}