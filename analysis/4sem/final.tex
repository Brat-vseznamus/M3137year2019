\documentclass[12pt, a4paper]{article}

\usepackage{lastpage}
\usepackage{mathtools}
\usepackage{xltxtra}
\usepackage{libertine}
\usepackage{amsmath}
\usepackage{amsthm}
\usepackage{amsfonts}
\usepackage{amssymb}
\usepackage{enumitem}
\usepackage{xcolor}
\usepackage[left=1.5cm, right=1.5cm, top=2cm, bottom=2cm, bindingoffset=0cm, headheight=15pt]{geometry}
\usepackage{fancyhdr}
\usepackage[russian]{babel}
% \usepackage[utf8]{inputenc}
\usepackage{catchfilebetweentags}
\usepackage{accents}
\usepackage{calc}
\usepackage{etoolbox}
\usepackage{mathrsfs}
\usepackage{wrapfig}

\providetoggle{useproofs}
\settoggle{useproofs}{false}

\pagestyle{fancy}
\lfoot{M3137y2019}
\rhead{\thepage\ из \pageref{LastPage}}

\newcommand{\R}{\mathbb{R}}
\newcommand{\Q}{\mathbb{Q}}
\newcommand{\C}{\mathbb{C}}
\newcommand{\Z}{\mathbb{Z}}
\newcommand{\B}{\mathbb{B}}
\newcommand{\N}{\mathbb{N}}

\newcommand{\const}{\text{const}}

\newcommand{\teormin}{\textcolor{red}{!}\ }

\DeclareMathOperator*{\xor}{\oplus}
\DeclareMathOperator*{\equ}{\sim}
\DeclareMathOperator{\Ln}{\text{Ln}}
\DeclareMathOperator{\sign}{\text{sign}}
\DeclareMathOperator{\Sym}{\text{Sym}}
\DeclareMathOperator{\Asym}{\text{Asym}}
% \DeclareMathOperator{\sh}{\text{sh}}
% \DeclareMathOperator{\tg}{\text{tg}}
% \DeclareMathOperator{\arctg}{\text{arctg}}
% \DeclareMathOperator{\ch}{\text{ch}}

\DeclarePairedDelimiter{\ceil}{\lceil}{\rceil}
\DeclarePairedDelimiter{\abs}{\left\lvert}{\right\rvert}

\setmainfont{Linux Libertine}

\theoremstyle{plain}
\newtheorem{axiom}{Аксиома}
\newtheorem{lemma}{Лемма}

\theoremstyle{remark}
\newtheorem*{remark}{Примечание}
\newtheorem*{exercise}{Упражнение}
\newtheorem*{consequence}{Следствие}
\newtheorem*{example}{Пример}
\newtheorem*{observation}{Наблюдение}

\theoremstyle{definition}
\newtheorem{theorem}{Теорема}
\newtheorem*{definition}{Определение}
\newtheorem*{obozn}{Обозначение}

\setlength{\parindent}{0pt}

\newcommand{\dbltilde}[1]{\accentset{\approx}{#1}}
\newcommand{\intt}{\int\!}

% magical thing that fixes paragraphs
\makeatletter
\patchcmd{\CatchFBT@Fin@l}{\endlinechar\m@ne}{}
  {}{\typeout{Unsuccessful patch!}}
\makeatother

\newcommand{\get}[2]{
    \ExecuteMetaData[#1]{#2}
}

\newcommand{\getproof}[2]{
    \iftoggle{useproofs}{\ExecuteMetaData[#1]{#2proof}}{}
}

\newcommand{\getwithproof}[2]{
    \get{#1}{#2}
    \getproof{#1}{#2}
}

\newcommand{\import}[3]{
    \subsection{#1}
    \getwithproof{#2}{#3}
}

\newcommand{\given}[1]{
    Дано выше. (\ref{#1}, стр. \pageref{#1})
}

\renewcommand{\ker}{\text{Ker }}
\newcommand{\im}{\text{Im }}
\newcommand{\grad}{\text{grad}}

\usepackage{sectsty}

\allsectionsfont{\raggedright}
\subsectionfont{\fontsize{14}{15}\selectfont}

\lhead{Итоговый конспект}
\rfoot{}

\settoggle{useproofs}{true}

\renewcommand{\import}[3]{
    \subsection{#1}
    \getwithproof{#2}{#3}
}

\begin{document}

\section{Определения}

\import{Ступенчатая функция}{1.tex}{ступенчатаяфункция}
\label{ступенчатая функция}

\import{Разбиение, допустимое для ступенчатой функции}{}{}
\given{ступенчатая функция}

\import{\teormin Измеримая функция}{1.tex}{измеримаяфункция}

\import{Свойство, выполняющееся почти везде}{2.tex}{свойствопочтивезде}

\import{Сходимость почти везде}{2.tex}{сходимостьпочтивезде}

\import{Сходимость по мере}{2.tex}{сходимостьпомере}

\import{Теорема Егорова о сходиомсти почти везде и почти равномерной сходиомсти}{2.tex}{теоремаегорова}

\import{Интеграл ступенчатой функции}{2.tex}{интеграл1}

\import{Интеграл неотрицательной измеримой функции}{2.tex}{интеграл2}

\import{Суммируемая функция}{3.tex}{суммируемаяфункция}

\import{Интеграл суммируемой функции}{2.tex}{интеграл3}

\import{Образ меры при отображении}{4.tex}{образприотображении}

\import{Взвешенный образ меры}{4.tex}{взвешенныйобразмеры}

\import{Плотность одной меры по отношению к другой}{4.tex}{плотностьмеры}

% \import{Измеримое множество на простой двумерной поверхности в \(\R^3\)}{2.tex}{}

% \import{Мера Лебега на простой двумерной поверхности в \(\R^3\)}{2.tex}{}

% \import{Поверхностный интеграл первого рода}{2.tex}{}

% \import{Произведение мер}{2.tex}{}

% \import{Теорема Фубини}{2.tex}{}

% \import{Сторона поверхности}{2.tex}{}

% \import{Задание стороны поверхности с помощью касательных реперов}{2.tex}{}

% \import{Интеграл II рода}{2.tex}{}

% \import{Ориентация контура, согласованная со стороной поверхности}{2.tex}{}

% \import{Интегральные неравенства Гельдера и Минковского}{2.tex}{}

% \import{Интеграл комплекснозначной функции}{2.tex}{}

% \import{Пространство $L^p(E,\mu)$}{2.tex}{}

% \import{Пространство $L^\infty(E,\mu)$}{2.tex}{}

% \import{Существенный супремум}{2.tex}{}

% \import{Ротор, дивергенция векторного поля}{2.tex}{}

% \import{Соленоидальное векторное поле}{2.tex}{}

% \import{Бескоординатное определение ротора и дивергенции}{2.tex}{}

% \import{Гильбертово пространство}{2.tex}{}

% \import{Ортогональный ряд}{2.tex}{}

% \import{Сходящийся ряд в гильбертовом пространстве}{2.tex}{}

% \import{Ортогональная система (семейство) векторов}{2.tex}{}

% \import{Ортонормированная система }{2.tex}{}

% \import{Коэффициенты Фурье}{2.tex}{}

% \import{Ряд Фурье в Гильбертовом пространстве}{2.tex}{}

\section{Теоремы}

\import{Лемма ``о структуре компактного оператора''}{1.tex}{оструктурекомпактногооператора}

\import{\teormin Теорема о преобразовании меры Лебега при линейном отображении}{1.tex}{опреобразованиимерылебегаподдействиемлинейногоотображения}

\import{Теорема об измеримости пределов и супремумов}{1.tex}{обизмеримостипределовисупрмемумов}

\import{Характеризация измеримых функций с помощью ступенчатых. Следствия}{2.tex}{характеризацияизмеримыхфункцийспомощьюступенчатых}
\get{2.tex}{характеризацияизмеримыхфункцийспомощьюступенчатыхcorollary}

\import{Измеримость функции, непрерывной на множестве полной меры}{2.tex}{обизмеримостифункцийнепрерывныхнамножествеполноймеры}

\import{Теорема Лебега о сходимости почти везде и сходимости по мере}{2.tex}{лебега}

\import{Теорема Рисса о сходимости по мере и сходимости почти везде}{2.tex}{рисса}

\import{Простейшие свойства интеграла Лебега}{3.tex}{свойстваинтегралалебега}

\import{Счетная аддитивность интеграла (по множеству)}{3.tex}{счётнаяадиитивностьинтеграла}

\import{Теорема Леви}{3.tex}{леви}

% \import{Линейность интеграла Лебега}{2.tex}{}

% \import{Теорема об интегрировании положительных рядов. Следствие о рядах, сходящихся почти везде}{2.tex}{}

% \import{Абсолютная непрерывность интеграла}{2.tex}{}

% \import{Теорема Лебега о мажорированной сходимости для случая сходимости по мере}{2.tex}{}

% \import{Теорема Лебега о мажорированной сходимости для случая сходимости почти везде}{2.tex}{}

% \import{Теорема Фату. Следствия}{2.tex}{}

% \import{Теорема о вычислении интеграла по взвешенному образу меры}{2.tex}{}

% \import{Критерий плотности}{2.tex}{}

% \import{Лемма о единственности плотности}{2.tex}{}

% \import{Лемма об оценке мер образов малых кубов}{2.tex}{}

% \import{Теорема о преобразовании меры при диффеоморфизме}{2.tex}{}

% \import{Теорема о гладкой замене переменной в интеграле Лебега}{2.tex}{}

% \import{Теорема о произведении мер}{2.tex}{}

% \import{Принцип Кавальери}{2.tex}{}

% \import{Теорема Тонелли}{2.tex}{}

% \import{Формула для Бета-функции}{2.tex}{}

% \import{Объем шара в $\R^m$}{2.tex}{}

% \import{Формула Грина}{2.tex}{}

% \import{Формула Стокса}{2.tex}{}

% \import{Формула Гаусса--Остроградского}{2.tex}{}

% \import{Соленоидальность бездивергентного векторного поля}{2.tex}{}

% \import{Теорема о вложении пространств $L^p$}{2.tex}{}

% \import{Теорема о сходимости в $L_p$ и по мере}{2.tex}{}

% \import{Полнота $L^p$}{2.tex}{}

% \import{Плотность в $L^p$ множества ступенчатых функций}{2.tex}{}

% \import{Лемма Урысона}{2.tex}{}

% \import{Плотность в $L^p$ непрерывных финитных функций}{2.tex}{}

% \import{Теорема о непрерывности сдвига}{2.tex}{}

% \import{Теорема о свойствах сходимости в гильбертовом пространстве}{2.tex}{}

% \import{Теорема о коэффициентах разложения по ортогональной системе}{2.tex}{}

% \import{Теорема о свойствах частичных сумм ряда Фурье. Неравенство Бесселя}{2.tex}{}

\end{document}
