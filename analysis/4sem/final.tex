\documentclass[12pt, a4paper]{article}

\usepackage{lastpage}
\usepackage{mathtools}
\usepackage{xltxtra}
\usepackage{libertine}
\usepackage{amsmath}
\usepackage{amsthm}
\usepackage{amsfonts}
\usepackage{amssymb}
\usepackage{enumitem}
\usepackage{xcolor}
\usepackage[left=1.5cm, right=1.5cm, top=2cm, bottom=2cm, bindingoffset=0cm, headheight=15pt]{geometry}
\usepackage{fancyhdr}
\usepackage[russian]{babel}
% \usepackage[utf8]{inputenc}
\usepackage{catchfilebetweentags}
\usepackage{accents}
\usepackage{calc}
\usepackage{etoolbox}
\usepackage{mathrsfs}
\usepackage{wrapfig}

\providetoggle{useproofs}
\settoggle{useproofs}{false}

\pagestyle{fancy}
\lfoot{M3137y2019}
\rhead{\thepage\ из \pageref{LastPage}}

\newcommand{\R}{\mathbb{R}}
\newcommand{\Q}{\mathbb{Q}}
\newcommand{\C}{\mathbb{C}}
\newcommand{\Z}{\mathbb{Z}}
\newcommand{\B}{\mathbb{B}}
\newcommand{\N}{\mathbb{N}}

\newcommand{\const}{\text{const}}

\newcommand{\teormin}{\textcolor{red}{!}\ }

\DeclareMathOperator*{\xor}{\oplus}
\DeclareMathOperator*{\equ}{\sim}
\DeclareMathOperator{\Ln}{\text{Ln}}
\DeclareMathOperator{\sign}{\text{sign}}
\DeclareMathOperator{\Sym}{\text{Sym}}
\DeclareMathOperator{\Asym}{\text{Asym}}
% \DeclareMathOperator{\sh}{\text{sh}}
% \DeclareMathOperator{\tg}{\text{tg}}
% \DeclareMathOperator{\arctg}{\text{arctg}}
% \DeclareMathOperator{\ch}{\text{ch}}

\DeclarePairedDelimiter{\ceil}{\lceil}{\rceil}
\DeclarePairedDelimiter{\abs}{\left\lvert}{\right\rvert}

\setmainfont{Linux Libertine}

\theoremstyle{plain}
\newtheorem{axiom}{Аксиома}
\newtheorem{lemma}{Лемма}

\theoremstyle{remark}
\newtheorem*{remark}{Примечание}
\newtheorem*{exercise}{Упражнение}
\newtheorem*{consequence}{Следствие}
\newtheorem*{example}{Пример}
\newtheorem*{observation}{Наблюдение}

\theoremstyle{definition}
\newtheorem{theorem}{Теорема}
\newtheorem*{definition}{Определение}
\newtheorem*{obozn}{Обозначение}

\setlength{\parindent}{0pt}

\newcommand{\dbltilde}[1]{\accentset{\approx}{#1}}
\newcommand{\intt}{\int\!}

% magical thing that fixes paragraphs
\makeatletter
\patchcmd{\CatchFBT@Fin@l}{\endlinechar\m@ne}{}
  {}{\typeout{Unsuccessful patch!}}
\makeatother

\newcommand{\get}[2]{
    \ExecuteMetaData[#1]{#2}
}

\newcommand{\getproof}[2]{
    \iftoggle{useproofs}{\ExecuteMetaData[#1]{#2proof}}{}
}

\newcommand{\getwithproof}[2]{
    \get{#1}{#2}
    \getproof{#1}{#2}
}

\newcommand{\import}[3]{
    \subsection{#1}
    \getwithproof{#2}{#3}
}

\newcommand{\given}[1]{
    Дано выше. (\ref{#1}, стр. \pageref{#1})
}

\renewcommand{\ker}{\text{Ker }}
\newcommand{\im}{\text{Im }}
\newcommand{\grad}{\text{grad}}

\usepackage{sectsty}

\allsectionsfont{\raggedright}
\subsectionfont{\fontsize{14}{15}\selectfont}

\lhead{Итоговый конспект}
\rfoot{}

\settoggle{useproofs}{true}

\renewcommand{\import}[3]{
    \subsection{#1}
    \getwithproof{#2}{#3}
}

\begin{document}

\section{Определения}

\import{Ступенчатая функция}{1}{ступенчатаяфункция}
\label{ступенчатая функция}

\import{Разбиение, допустимое для ступенчатой функции}{}{}
\given{ступенчатая функция}

\import{\teormin Измеримая функция}{1}{измеримаяфункция}

\import{Свойство, выполняющееся почти везде}{2}{свойствопочтивезде}

\import{\teormin Сходимость почти везде}{2}{сходимостьпочтивезде}

\import{Сходимость по мере}{2}{сходимостьпомере}

\import{Теорема Егорова о сходиомсти почти везде и почти равномерной сходиомсти}{2}{теоремаегорова}

\import{Интеграл ступенчатой функции}{2}{интеграл1}

\import{\teormin Интеграл неотрицательной измеримой функции}{2}{интеграл2}

\import{\teormin Суммируемая функция}{3}{суммируемаяфункция}

\import{Интеграл суммируемой функции}{2}{интеграл3}

\import{Образ меры при отображении}{4}{образприотображении}

\import{Взвешенный образ меры}{4}{взвешенныйобразмеры}

\import{Плотность одной меры по отношению к другой}{4}{плотностьмеры}

\import{Измеримое множество на простой двумерной поверхности в \(\R^3\)}{7}{измеримоемножествонадвумернойповерхности}
\get{7}{алгебраповерхностей}

\import{Мера Лебега на простой двумерной поверхности в \(\R^3\)}{7}{мераповерхности}

\import{\teormin Поверхностный интеграл первого рода}{7}{поверхностныйинтегралпервогорода}

\import{Произведение мер}{6}{произведениемер}

\import{\teormin Теорема Фубини}{7}{фубини}

\import{Сторона поверхности}{8}{сторонаповерхости}

\import{Задание стороны поверхности с помощью касательных реперов}{8}{реперы}

\import{\teormin Интеграл II рода}{8}{интегралвторогорода}

\import{Ориентация контура, согласованная со стороной поверхности}{8}{согласованнаяориентация}

\import{Интегральные неравенства Гельдера и Минковского}{8}{неравенствогёльдера}
\get{8}{неравенствоминеовского}

\import{Интеграл комплекснозначной функции}{8}{интегралкомплекснозначнойфункции}

\import{\teormin Пространство $L^p(E,\mu)$}{8}{пространствоlp}

\import{\teormin Пространство $L^\infty(E,\mu)$}{8}{пространствоlinfty}

\import{\teormin Существенный супремум}{8}{существенныйсупремум}

\import{\teormin Ротор, дивергенция векторного поля}{10}{ротор}
\get{10}{дивергенция}

\import{Соленоидальное векторное поле}{10}{соленоидальноеполе}

\import{Бескоординатное определение ротора и дивергенции}{10}{бескоординатнаядивергенция}
\get{10}{бескоординатныйротор}

\import{\teormin Гильбертово пространство}{11}{гильбертовопространство}

\import{Ортогональный ряд}{11}{ортогональныйряд}

\import{Сходящийся ряд в гильбертовом пространстве}{11}{сходящийсяряд}

\import{Ортогональная система (семейство) векторов}{11}{ортогональноесемейство}
\label{Ортонормированная система}

\import{\teormin Ортонормированная система}{11}{}
\given{Ортонормированная система}

\import{Коэффициенты Фурье}{11}{коэффициентфурье}
\label{ряд фурье}

\import{Ряд Фурье в Гильбертовом пространстве}{11}{}
\given{ряд фурье}

\import{Базис, полная, замкнутая ОС}{12}{базис}

\import{Тригонометрический ряд}{12}{тригонометрическийряд}

\import{Коэффициенты Фурье функции}{12}{коэффициентыфурье}

\import{Класс Липшица с константой M и показателем альфа}{12}{класслипшица}

\import{Ядро Дирихле, ядро Фейера}{13}{ядра}

\import{\teormin Свертка}{13}{свертка}

\import{\teormin Аппроксимативная единица}{14}{аппроксимативнаяединица}

\import{Усиленная аппроксимативная единица}{14}{усиленнаяаппроксимативнаяединица}

\import{Метод суммирования средними арифметическими}{14}{среднихарифметических}

\import{Суммы Фейера}{14}{суммафейера}


\section{Теоремы}

\import{Лемма ``о структуре компактного оператора''}{1}{оструктурекомпактногооператора}

\import{\teormin Теорема о преобразовании меры Лебега при линейном отображении}{1}{опреобразованиимерылебегаподдействиемлинейногоотображения}

\import{Теорема об измеримости пределов и супремумов}{1}{обизмеримостипределовисупрмемумов}

\import{\teormin Характеризация измеримых функций с помощью ступенчатых. Следствия}{2}{характеризацияизмеримыхфункцийспомощьюступенчатых}
\get{2}{характеризацияизмеримыхфункцийспомощьюступенчатыхcorollary}

\import{Измеримость функции, непрерывной на множестве полной меры}{2}{обизмеримостифункцийнепрерывныхнамножествеполноймеры}

\import{Теорема Лебега о сходимости почти везде и сходимости по мере}{2}{лебега}

\import{Теорема Рисса о сходимости по мере и сходимости почти везде}{2}{рисса}

\import{Простейшие свойства интеграла Лебега}{3}{свойстваинтегралалебега}

\import{Счетная аддитивность интеграла (по множеству)}{3}{счётнаяадиитивностьинтегралалемма}
\getwithproof{3}{счётнаяадиитивностьинтеграла}
\get{3}{счётнаяадиитивностьинтегралаcorollary}

\import{\teormin Теорема Леви}{3}{леви}

\import{Линейность интеграла Лебега}{3}{линейностьинтеграла}
\getwithproof{3}{линейностьинтегралаcorollary}

\import{Теорема об интегрировании положительных рядов. Следствие о рядах, сходящихся почти везде}{3}{обинтегрированииположительныхрядов}
\getwithproof{3}{обинтегрированииположительныхрядовcorollary}

\import{Абсолютная непрерывность интеграла}{4}{обабсолютнойнепрерывностиинтеграла}
\get{4}{обабсолютнойнепрерывностиинтегралаcorollary}

\import{\teormin Теорема Лебега о мажорированной сходимости для случая сходимости по мере}{4}{лебега1}

\import{\teormin Теорема Лебега о мажорированной сходимости для случая сходимости почти везде}{4}{лебега2}

\import{Теорема Фату. Следствия}{4}{фату}
\get{4}{фатуcorollary}

\import{Теорема о вычислении интеграла по взвешенному образу меры}{4}{овычисленииинтегралаповзвешенномуобразумерынаблюдение}
\getwithproof{4}{овычисленииинтегралаповзвешенномуобразумеры}
\getwithproof{4}{овычисленииинтегралаповзвешенномуобразумерыcorollary}

\import{Критерий плотности}{5}{критерийплотности}

\import{Лемма о единственности плотности}{5}{единственностьплотности}

\import{Лемма об оценке мер образов малых кубов}{5}{обоценкемеробразовмалыхкубов}

\import{Теорема о преобразовании меры при диффеоморфизме}{5}{леммабезимени}
\getwithproof{5}{мералебегапридиффеоморфизме}

\import{Теорема о гладкой замене переменной в интеграле Лебега}{5}{огладкойзамене}

\import{Теорема о произведении мер}{6}{опроизведениимер}

\import{Принцип Кавальери}{6}{кавальери}
\getwithproof{6}{кавальериcorollary}

\import{Теорема Тонелли}{7}{тонелли}

\import{Формула для Бета-функции}{7}{формулабетафункции}

\import{Объем шара в $\R^m$}{7}{объемшара}

\import{Формула Грина}{9}{формулагрина}

\import{\teormin Формула Стокса}{9}{формуластокса}

\import{\teormin Формула Гаусса--Остроградского}{10}{формулаостроградского}
\get{10}{формулаостроградскогоcorollary}

\import{Соленоидальность бездивергентного векторного поля}{10}{пуанкаре}

\import{Теорема о вложении пространств $L^p$}{9}{овложении}

\import{Теорема о сходимости в $L_p$ и по мере}{9}{осходимостивlpипомере}

\import{Полнота $L^p$}{9}{полнотаlp}

\import{Плотность в $L^p$ множества ступенчатых функций}{9}{плотностьступенчатых}

\import{Лемма Урысона}{10}{урысона}

\import{Плотность в $L^p$ непрерывных финитных функций}{10}{плотностьнепрерывныхфинитных}

\import{\teormin Теорема о непрерывности сдвига}{11}{непрерывностьсдвига}

\import{Теорема о свойствах сходимости в гильбертовом пространстве}{11}{свойствасходимости}

\import{Теорема о коэффициентах разложения по ортогональной системе}{11}{окоэффициентах}

\import{Теорема о свойствах частичных сумм ряда Фурье. Неравенство Бесселя}{11}{частсумм}
\get{11}{неравенствобесселя}

\import{Теорема Рисса -- Фишера о сумме ряда Фурье. Равенство Парсеваля}{12}{риссфишер}
\get{12}{равенствоперсиваля}

\import{Теорема о характеристике базиса}{12}{охарактеристикебазиса}

\import{Лемма о вычислении коэффициентов тригонометрического ряда}{12}{овычислениикоэффициентов}

\import{Теорема Римана--Лебега}{12}{риманалебега}

\import{Три следствия об оценке коэффициентов Фурье}{12}{corollaries}

\import{Принцип локализации Римана}{13}{локализацияримана}

\import{\teormin Признак Дини. Следствия}{13}{признакдини}
\get{13}{признакдиниcorollary}

\import{Корректность определения свертки}{13}{корректностьсвертки}

\import{Свойства свертки}{13}{свойствасвертки}

\import{Теорема о свойствах аппроксимативной единицы}{14}{свойстваединицы}

\import{Теорема о перманентности метода средних арифметических}{14}{оперманентности}

\import{Теорема Фейера}{14}{}

\import{Следствия из теоремы Фейера}{2}{}

\import{Теорема об интегрировании ряда Фурье}{2}{}

\import{Лемма о слабой сходимости сумм Фурье}{2}{}

\end{document}
