\documentclass[12pt, a4paper]{article}

\usepackage{lastpage}
\usepackage{mathtools}
\usepackage{xltxtra}
\usepackage{libertine}
\usepackage{amsmath}
\usepackage{amsthm}
\usepackage{amsfonts}
\usepackage{amssymb}
\usepackage{enumitem}
\usepackage{xcolor}
\usepackage[left=1.5cm, right=1.5cm, top=2cm, bottom=2cm, bindingoffset=0cm, headheight=15pt]{geometry}
\usepackage{fancyhdr}
\usepackage[russian]{babel}
% \usepackage[utf8]{inputenc}
\usepackage{catchfilebetweentags}
\usepackage{accents}
\usepackage{calc}
\usepackage{etoolbox}
\usepackage{mathrsfs}
\usepackage{wrapfig}

\providetoggle{useproofs}
\settoggle{useproofs}{false}

\pagestyle{fancy}
\lfoot{M3137y2019}
\rhead{\thepage\ из \pageref{LastPage}}

\newcommand{\R}{\mathbb{R}}
\newcommand{\Q}{\mathbb{Q}}
\newcommand{\C}{\mathbb{C}}
\newcommand{\Z}{\mathbb{Z}}
\newcommand{\B}{\mathbb{B}}
\newcommand{\N}{\mathbb{N}}

\newcommand{\const}{\text{const}}

\newcommand{\teormin}{\textcolor{red}{!}\ }

\DeclareMathOperator*{\xor}{\oplus}
\DeclareMathOperator*{\equ}{\sim}
\DeclareMathOperator{\Ln}{\text{Ln}}
\DeclareMathOperator{\sign}{\text{sign}}
\DeclareMathOperator{\Sym}{\text{Sym}}
\DeclareMathOperator{\Asym}{\text{Asym}}
% \DeclareMathOperator{\sh}{\text{sh}}
% \DeclareMathOperator{\tg}{\text{tg}}
% \DeclareMathOperator{\arctg}{\text{arctg}}
% \DeclareMathOperator{\ch}{\text{ch}}

\DeclarePairedDelimiter{\ceil}{\lceil}{\rceil}
\DeclarePairedDelimiter{\abs}{\left\lvert}{\right\rvert}

\setmainfont{Linux Libertine}

\theoremstyle{plain}
\newtheorem{axiom}{Аксиома}
\newtheorem{lemma}{Лемма}

\theoremstyle{remark}
\newtheorem*{remark}{Примечание}
\newtheorem*{exercise}{Упражнение}
\newtheorem*{consequence}{Следствие}
\newtheorem*{example}{Пример}
\newtheorem*{observation}{Наблюдение}

\theoremstyle{definition}
\newtheorem{theorem}{Теорема}
\newtheorem*{definition}{Определение}
\newtheorem*{obozn}{Обозначение}

\setlength{\parindent}{0pt}

\newcommand{\dbltilde}[1]{\accentset{\approx}{#1}}
\newcommand{\intt}{\int\!}

% magical thing that fixes paragraphs
\makeatletter
\patchcmd{\CatchFBT@Fin@l}{\endlinechar\m@ne}{}
  {}{\typeout{Unsuccessful patch!}}
\makeatother

\newcommand{\get}[2]{
    \ExecuteMetaData[#1]{#2}
}

\newcommand{\getproof}[2]{
    \iftoggle{useproofs}{\ExecuteMetaData[#1]{#2proof}}{}
}

\newcommand{\getwithproof}[2]{
    \get{#1}{#2}
    \getproof{#1}{#2}
}

\newcommand{\import}[3]{
    \subsection{#1}
    \getwithproof{#2}{#3}
}

\newcommand{\given}[1]{
    Дано выше. (\ref{#1}, стр. \pageref{#1})
}

\renewcommand{\ker}{\text{Ker }}
\newcommand{\im}{\text{Im }}
\newcommand{\grad}{\text{grad}}

\usepackage{sectsty}

\allsectionsfont{\raggedright}
\subsectionfont{\fontsize{14}{15}\selectfont}

\lhead{Итоговый конспект}
\rfoot{}

\settoggle{useproofs}{true}

\renewcommand{\import}[3]{
    \subsection{#1}
    \getwithproof{#2}{#3}
}

\begin{document}

\section{Определения}

\import{Ступенчатая функция}{1}{ступенчатаяфункция}
\label{ступенчатая функция}

\import{Разбиение, допустимое для ступенчатой функции}{}{}
\given{ступенчатая функция}

\import{\teormin Измеримая функция}{1}{измеримаяфункция}

\import{Свойство, выполняющееся почти везде}{2}{свойствопочтивезде}

\import{\teormin Сходимость почти везде}{2}{сходимостьпочтивезде}

\import{Сходимость по мере}{2}{сходимостьпомере}

\import{Теорема Егорова о сходиомсти почти везде и почти равномерной сходиомсти}{2}{теоремаегорова}

\import{Интеграл ступенчатой функции}{2}{интеграл1}

\import{\teormin Интеграл неотрицательной измеримой функции}{2}{интеграл2}

\import{\teormin Суммируемая функция}{3}{суммируемаяфункция}

\import{Интеграл суммируемой функции}{2}{интеграл3}

\import{Образ меры при отображении}{4}{образприотображении}

\import{Взвешенный образ меры}{4}{взвешенныйобразмеры}

\import{Плотность одной меры по отношению к другой}{4}{плотностьмеры}

\import{Измеримое множество на простой двумерной поверхности в \(\R^3\)}{7}{измеримоемножествонадвумернойповерхности}
\get{7}{алгебраповерхностей}

\import{Мера Лебега на простой двумерной поверхности в \(\R^3\)}{7}{мераповерхности}

% \import{\teormin Поверхностный интеграл первого рода}{2}{}

% \import{Произведение мер}{2}{}

% \import{\teormin Теорема Фубини}{2}{}

% \import{Сторона поверхности}{2}{}

% \import{Задание стороны поверхности с помощью касательных реперов}{2}{}

% \import{\teormin Интеграл II рода}{2}{}

% \import{Ориентация контура, согласованная со стороной поверхности}{2}{}

% \import{Интегральные неравенства Гельдера и Минковского}{2}{}

% \import{Интеграл комплекснозначной функции}{2}{}

% \import{\teormin Пространство $L^p(E,\mu)$}{2}{}

% \import{\teormin Пространство $L^\infty(E,\mu)$}{2}{}

% \import{\teormin Существенный супремум}{2}{}

% \import{\teormin Ротор, дивергенция векторного поля}{2}{}

% \import{Соленоидальное векторное поле}{2}{}

% \import{Бескоординатное определение ротора и дивергенции}{2}{}

% \import{\teormin Гильбертово пространство}{2}{}

% \import{Ортогональный ряд}{2}{}

% \import{Сходящийся ряд в гильбертовом пространстве}{2}{}

% \import{Ортогональная система (семейство) векторов}{2}{}

% \import{\teormin Ортонормированная система }{2}{}

% \import{Коэффициенты Фурье}{2}{}

% \import{Ряд Фурье в Гильбертовом пространстве}{2}{}

% \import{Базис, полная, замкнутая ОС}{2}{}

% \import{Тригонометрический ряд}{2}{}

% \import{Коэффициенты Фурье функции}{2}{}

% \import{Класс Липшица с константой M и показателем альфа}{2}{}

% \import{Ядро Дирихле, ядро Фейера}{2}{}

% \import{\teormin Свертка}{2}{}

% \import{\teormin Аппроксимативная единица}{2}{}

% \import{Усиленная аппроксимативная единица}{2}{}

% \import{Метод суммирования средними арифметическими}{2}{}

% \import{Суммы Фейера}{2}{}


\section{Теоремы}

\import{Лемма ``о структуре компактного оператора''}{1}{оструктурекомпактногооператора}

\import{\teormin Теорема о преобразовании меры Лебега при линейном отображении}{1}{опреобразованиимерылебегаподдействиемлинейногоотображения}

\import{Теорема об измеримости пределов и супремумов}{1}{обизмеримостипределовисупрмемумов}

\import{\teormin Характеризация измеримых функций с помощью ступенчатых. Следствия}{2}{характеризацияизмеримыхфункцийспомощьюступенчатых}
\get{2}{характеризацияизмеримыхфункцийспомощьюступенчатыхcorollary}

\import{Измеримость функции, непрерывной на множестве полной меры}{2}{обизмеримостифункцийнепрерывныхнамножествеполноймеры}

\import{Теорема Лебега о сходимости почти везде и сходимости по мере}{2}{лебега}

\import{Теорема Рисса о сходимости по мере и сходимости почти везде}{2}{рисса}

\import{Простейшие свойства интеграла Лебега}{3}{свойстваинтегралалебега}

\import{Счетная аддитивность интеграла (по множеству)}{3}{счётнаяадиитивностьинтегралалемма}
\getwithproof{3}{счётнаяадиитивностьинтеграла}
\get{3}{счётнаяадиитивностьинтегралаcorollary}

\import{\teormin Теорема Леви}{3}{леви}

\import{Линейность интеграла Лебега}{3}{линейностьинтеграла}
\getwithproof{3}{линейностьинтегралаcorollary}

\import{Теорема об интегрировании положительных рядов. Следствие о рядах, сходящихся почти везде}{3}{обинтегрированииположительныхрядов}
\getwithproof{3}{обинтегрированииположительныхрядовcorollary}

\import{Абсолютная непрерывность интеграла}{4}{обабсолютнойнепрерывностиинтеграла}
\get{4}{обабсолютнойнепрерывностиинтегралаcorollary}

\import{\teormin Теорема Лебега о мажорированной сходимости для случая сходимости по мере}{4}{лебега1}

\import{\teormin Теорема Лебега о мажорированной сходимости для случая сходимости почти везде}{4}{лебега2}

\import{Теорема Фату. Следствия}{4}{фату}
\get{4}{фатуcorollary}

\import{Теорема о вычислении интеграла по взвешенному образу меры}{4}{овычисленииинтегралаповзвешенномуобразумерынаблюдение}
\getwithproof{4}{овычисленииинтегралаповзвешенномуобразумеры}
\getwithproof{4}{овычисленииинтегралаповзвешенномуобразумерыcorollary}

\import{Критерий плотности}{5}{критерийплотности}

\import{Лемма о единственности плотности}{5}{единственностьплотности}

\import{Лемма об оценке мер образов малых кубов}{5}{обоценкемеробразовмалыхкубов}

\import{Теорема о преобразовании меры при диффеоморфизме}{5}{леммабезимени}
\getwithproof{5}{мералебегапридиффеоморфизме}

\import{Теорема о гладкой замене переменной в интеграле Лебега}{5}{огладкойзамене}

% \import{Теорема о произведении мер}{2}{}

% \import{Принцип Кавальери}{2}{}

% \import{Теорема Тонелли}{2}{}

% \import{Формула для Бета-функции}{2}{}

% \import{Объем шара в $\R^m$}{2}{}

% \import{Формула Грина}{2}{}

% \import{\teormin Формула Стокса}{2}{}

% \import{\teormin Формула Гаусса--Остроградского}{2}{}

% \import{Соленоидальность бездивергентного векторного поля}{2}{}

% \import{Теорема о вложении пространств $L^p$}{2}{}

% \import{Теорема о сходимости в $L_p$ и по мере}{2}{}

% \import{Полнота $L^p$}{2}{}

% \import{Плотность в $L^p$ множества ступенчатых функций}{2}{}

% \import{Лемма Урысона}{2}{}

% \import{Плотность в $L^p$ непрерывных финитных функций}{2}{}

% \import{\teormin Теорема о непрерывности сдвига}{2}{}

% \import{Теорема о свойствах сходимости в гильбертовом пространстве}{2}{}

% \import{Теорема о коэффициентах разложения по ортогональной системе}{2}{}

% \import{Теорема о свойствах частичных сумм ряда Фурье. Неравенство Бесселя}{2}{}

% \import{Теорема Рисса -- Фишера о сумме ряда Фурье. Равенство Парсеваля}{2}{}

% \import{Теорема о характеристике базиса}{2}{}

% \import{Лемма о вычислении коэффициентов тригонометрического ряда}{2}{}

% \import{Теорема Римана--Лебега}{2}{}

% \import{Три следствия об оценке коэффициентов Фурье}{2}{}

% \import{Принцип локализации Римана}{2}{}

% \import{\teormin Признак Дини. Следствия}{2}{}

% \import{Корректность определения свертки}{2}{}

% \import{Свойства свертки }{2}{}

% \import{Теорема о свойствах аппроксимативной единицы}{2}{}

% \import{Теорема о перманентности метода средних арифметических}{2}{}

% \import{Теорема Фейера }{2}{}

% \import{Следствия из теоремы Фейера}{2}{}

% \import{Теорема об интегрировании ряда Фурье}{2}{}

% \import{Лемма о слабой сходимости сумм Фурье}{2}{}

\end{document}
