\documentclass[12pt, a4paper]{article}

\usepackage{lastpage}
\usepackage{mathtools}
\usepackage{xltxtra}
\usepackage{libertine}
\usepackage{amsmath}
\usepackage{amsthm}
\usepackage{amsfonts}
\usepackage{amssymb}
\usepackage{enumitem}
\usepackage{xcolor}
\usepackage[left=1.5cm, right=1.5cm, top=2cm, bottom=2cm, bindingoffset=0cm, headheight=15pt]{geometry}
\usepackage{fancyhdr}
\usepackage[russian]{babel}
% \usepackage[utf8]{inputenc}
\usepackage{catchfilebetweentags}
\usepackage{accents}
\usepackage{calc}
\usepackage{etoolbox}
\usepackage{mathrsfs}
\usepackage{wrapfig}

\providetoggle{useproofs}
\settoggle{useproofs}{false}

\pagestyle{fancy}
\lfoot{M3137y2019}
\rhead{\thepage\ из \pageref{LastPage}}

\newcommand{\R}{\mathbb{R}}
\newcommand{\Q}{\mathbb{Q}}
\newcommand{\C}{\mathbb{C}}
\newcommand{\Z}{\mathbb{Z}}
\newcommand{\B}{\mathbb{B}}
\newcommand{\N}{\mathbb{N}}

\newcommand{\const}{\text{const}}

\newcommand{\teormin}{\textcolor{red}{!}\ }

\DeclareMathOperator*{\xor}{\oplus}
\DeclareMathOperator*{\equ}{\sim}
\DeclareMathOperator{\Ln}{\text{Ln}}
\DeclareMathOperator{\sign}{\text{sign}}
\DeclareMathOperator{\Sym}{\text{Sym}}
\DeclareMathOperator{\Asym}{\text{Asym}}
% \DeclareMathOperator{\sh}{\text{sh}}
% \DeclareMathOperator{\tg}{\text{tg}}
% \DeclareMathOperator{\arctg}{\text{arctg}}
% \DeclareMathOperator{\ch}{\text{ch}}

\DeclarePairedDelimiter{\ceil}{\lceil}{\rceil}
\DeclarePairedDelimiter{\abs}{\left\lvert}{\right\rvert}

\setmainfont{Linux Libertine}

\theoremstyle{plain}
\newtheorem{axiom}{Аксиома}
\newtheorem{lemma}{Лемма}

\theoremstyle{remark}
\newtheorem*{remark}{Примечание}
\newtheorem*{exercise}{Упражнение}
\newtheorem*{consequence}{Следствие}
\newtheorem*{example}{Пример}
\newtheorem*{observation}{Наблюдение}

\theoremstyle{definition}
\newtheorem{theorem}{Теорема}
\newtheorem*{definition}{Определение}
\newtheorem*{obozn}{Обозначение}

\setlength{\parindent}{0pt}

\newcommand{\dbltilde}[1]{\accentset{\approx}{#1}}
\newcommand{\intt}{\int\!}

% magical thing that fixes paragraphs
\makeatletter
\patchcmd{\CatchFBT@Fin@l}{\endlinechar\m@ne}{}
  {}{\typeout{Unsuccessful patch!}}
\makeatother

\newcommand{\get}[2]{
    \ExecuteMetaData[#1]{#2}
}

\newcommand{\getproof}[2]{
    \iftoggle{useproofs}{\ExecuteMetaData[#1]{#2proof}}{}
}

\newcommand{\getwithproof}[2]{
    \get{#1}{#2}
    \getproof{#1}{#2}
}

\newcommand{\import}[3]{
    \subsection{#1}
    \getwithproof{#2}{#3}
}

\newcommand{\given}[1]{
    Дано выше. (\ref{#1}, стр. \pageref{#1})
}

\renewcommand{\ker}{\text{Ker }}
\newcommand{\im}{\text{Im }}
\newcommand{\grad}{\text{grad}}

\lhead{Конспект по матанализу}
\cfoot{}
\rfoot{December 2, 2019}

\usepackage{xcolor}

\begin{document}
    \begin{lemma}
        О связности отрезка

        %<*освязностиотрезка>
        Промежуток $\langle a, b\rangle$ \textit{(границы могут входить, могут не входить)} --- не представим в виде объединения двух непересекающихся непустых открытых множеств

        Т.е. $\not\exists G_1, G_2 \subset \R$ --- откр.:
        \begin{itemize}
            \item $G_1\cap G_2=\text{\O}$
            \item $\langle a, b\rangle \cap G_1 \not=\text{\O} \quad \langle a, b\rangle \cap G_2 \not=\text{\O}$
            \item $\langle a, b\rangle \subset G_1\cup G_2$
        \end{itemize}
        %</освязностиотрезка>
    \end{lemma}
    %<*освязностиотрезкаproof>
    \begin{proof}
        От противного: $\alpha\in \langle a, b\rangle \cap G_1 \quad \beta\in \langle a, b\rangle \cap G_2$, пусть $\alpha<\beta$

        $$t:=\sup\{x:[\alpha, x]\subset G_1\} \quad \alpha\leq t\leq\beta$$

        $t\in G_1? \text{ нет, т.к. если да, то } t\not=\beta \text{ и } \exists U(t)=(t-\varepsilon, t+\varepsilon)\subset G_1\cap[\alpha, \beta]$, это противоречит определению $t$:

        $[\alpha, t-\frac{\varepsilon}{2}]\subset G_1$

        $(t-\varepsilon, t+\varepsilon)\subset G_1$

        $[\alpha, t+\frac{\varepsilon}{2}]\subset G_1$

        $t\in G_2?$ нет, т.к. если лежит, то $t\not=\alpha \quad \exists(t-\varepsilon, t+\varepsilon)\subset G_2 \cap [\alpha, \beta)$

        $\sup\{x:[\alpha, x]\subset G_1\}\leq t-\varepsilon$
    \end{proof}
    %</освязностиотрезкаproof>
    \begin{theorem}
        Больцано-Коши о промежуточном значении

        %<*больцанокошиопромежуточномзначении>
        $f:[a,b]\to\R$, непр. на $[a,b]$. Тогда
        $$\forall t\text{ между } f(a) \text{ и } f(b) \ \ \exists x\in[a,b] : f(x)=t$$
        %</больцанокошиопромежуточномзначении>
    \end{theorem}
    %<*больцанокошиопромежуточномзначенииproof>
    Традиционное доказательство --- бинпоиск.
    \begin{proof}
        Сразу следует из леммы о связности отрезка и топологического определения\\непрерывности.

        Если нашлось $t$, для которого доказуемое утверждение неверно, то $$[a,b]=f^{-1}(-\infty, t)\cup f^{-1}(t, +\infty)$$

        Оба множества открыты, т.к. они --- прообразы открытых множеств. Кроме того, они непусты, т.к. одно из них содержит $a$, другое содержит $b$. Итого, мы представили отрезок $[a,b]$ в виде двух непересекающихся непустых открытых множеств, противоречие по предыдущей лемме.
    \end{proof}
    %</больцанокошиопромежуточномзначенииproof>
    \begin{theorem}
        О вписанном $n$-угольнике максимальной площади
        
        %<*овписанномnугольникемаксимальнойплощади>
        Вписанный $n$-угольник максимальной площади --- правильный.
        %</овписанномnугольникемаксимальнойплощади>
    \end{theorem}
    %<*овписанномnугольникемаксимальнойплощадиproof>
    \begin{proof}
        \textcolor{red}{Чего-то геометрическое}
    \end{proof}
    %</овписанномnугольникемаксимальнойплощадиproof>
    \begin{theorem}
        Теорема о разделении колбасы.

        Кусок колбасы произвольной формы можно разделить прямой, параллельной данной, на две части одинаковой площади. \textit{(Кусок колбасы вводится, чтобы не возникало вопросов о площади.)}
    \end{theorem}
    \begin{proof}
        Пусть колбаса лежит на столе высотой $H_0$.

        $S(x)\stackrel{def}{=}S_{\text{л}}$ --- непр.

        $$|S(x+h)-S(x)|\leq hH_0$$

        $$\begin{cases}
            S(x_0)=0 \\
            S(x_1)=S
        \end{cases} \Rightarrow \exists \overline x \ \ S(\overline x)=\frac{S}{2}$$
    \end{proof}
    \begin{theorem}
        Теорема о бутерброде

        %<*теоремаобутерброде>
        Кусок хлеба и кусок колбасы, лежащие на столе, можно разрезать прямой на две равные по площади части каждый.
        %</теоремаобутерброде>
    \end{theorem}
    \begin{remark}
        Т.к. хлеб и колбаса лежат на столе, они ограничены.
    \end{remark}
    %<*теоремаобутербродеproof>
    \begin{proof}
        Рассмотрим угол $\varphi$ и разделим прямой под углом $\varphi$ колбасу на две равные по площади части.

        $S(\varphi) = S_{\text{л}}-S_{\text{п}}$ \textit{(для хлеба)}

        $S$ --- непр.

        $$|S(\varphi+h)-S(\varphi)|\leq 2ab\sin h\leq 2d^2\sin h$$

        Берём произвольный угол $\varphi_0; \varphi_0+\pi$

        $\varphi_0: S_{\text{л}} - S_{\text{п}}$

        $\varphi_0+\pi: S_{\text{п}} - S_{\text{л}}$

        $\exists \varphi \ \ S(\varphi)=0$
    \end{proof}
    %</теоремаобутербродеproof>
    \begin{remark}
        Это верно для адекватной колбасы, например она не должна состоять из двух кусков. Кроме того, она не должна состоять из двух кусков, соединённых ниткой.

        Теорема верна для выпуклых фигур, но возможно для других --- тоже.
    \end{remark}
    \begin{theorem}
        О сохранении промежутка
        
        %<*осохранениипромежутка>
        $f:\langle a,b \rangle \to\R$, непр.

        Тогда $f(\langle a,b \rangle)$ --- промежуток.
        %</осохранениипромежутка>
    \end{theorem}
    %<*осохранениипромежуткаproof>
    \begin{proof}
        \textcolor{red}{Не по Кохасю.}

        $m:=\inf f, M:=\sup f$. Докажем, что $f(\langle a,b \rangle)=\langle m,M \rangle$ путем доказательства, что $f(\langle a,b \rangle)\subset\langle m,M \rangle$ и $f(\langle a,b \rangle)\supset\langle m,M \rangle$.
        \begin{enumerate}
            \item $f(\langle a,b \rangle)\subset\langle m,M \rangle$
            $$\forall x \ \ m\leq f(x)\leq M \Rightarrow f(\langle a,b \rangle)\subset\langle m,M \rangle$$

            \item $f(\langle a,b \rangle)\supset\langle m,M \rangle$
            
            $\sphericalangle k\in\langle m,M \rangle$. По теореме Больцано-Коши о промежуточном значении $\exists c\in\langle a,b \rangle: f(c)=k\Rightarrow k\in f(\langle a,b\rangle)\Rightarrow \langle m,M \rangle\subset f(\langle a,b \rangle)$
        \end{enumerate}
    \end{proof}
    %</осохранениипромежуткаproof>
    \begin{remark}
        Это промежуток $\langle \inf f, \sup f \rangle$
    \end{remark}
    \begin{remark}
        $f=\sin \quad (0; 2\pi)\to [-1, 1]; (0; \pi)\to (0, 1]$

        Но! т. Вейерштрасса $f([a,b])$ --- замкн. промежуток
    \end{remark}
    \begin{definition}
        %<*путь>
        $Y$ --- метр. пр-во
        
        $\gamma: [a,b]\to Y$ --- непр. на $[a,b]$

        = \textbf{путь} в пространстве $Y$
        %</путь>

        $\gamma[a,b] = $ носитель пути, ``кривая''
    \end{definition}
    \begin{example}
        $\gamma : [0, 2\pi]\to\R^2$

        $t\mapsto \begin{pmatrix}
            \cos t \\
            \sin t
        \end{pmatrix}$

        $\gamma$ --- окружность.
    \end{example}
    \begin{definition}
        %<*линейносвязноемножество>
        $E\subset Y$

        $E$ --- \textbf{линейно связное}, если $\forall A,B\in E$
        $\exists$ путь $\gamma:[a,b]\to E$ такой, что:
        \begin{itemize}
            \item $\gamma(a)=A$
            \item $\gamma(b)=B$
        \end{itemize}
        %</линейносвязноемножество>
    \end{definition}
    \begin{lemma}
        %<*линейносвзяноевr>
        В $\R$ линейно связанными множествами являются только промежутки.
        %</линейносвзяноевr>
    \end{lemma}
    %<*линейносвзяноевrproof>
    \begin{proof}
        \begin{enumerate}
            \item Промежуток линейно связен.
            $$\forall A, B\in \langle a,b \rangle \quad \exists \text{ путь: } \gamma:[A,B]\Rightarrow \langle a,b \rangle; t\mapsto t$$
            \item $E\subset\R$ --- линейно связное $\xRightarrow{?} E$ --- промежуток
            
            Пусть $E$ --- не промежуток

            $\exists a,b,t : a,b\in E; a<b \quad a<t<b; t\not\in E$

            Линейная связность: $\gamma:[\alpha, \beta]\to E$

            $\gamma(\alpha)=a \quad \gamma(\beta)=b \quad \gamma$ --- непр.
        \end{enumerate}
    \end{proof}
    %</линейносвзяноевrproof>
    \begin{example}
        \textit{(ужасный)}

        $\gamma: [0,1]\to [0,1]\times[0,1]\subset\R$

        $[0, \frac{1}{4}]$ --- в третий квадрант

        $[\frac{1}{4}, \frac{2}{4}]$ --- во второй квадрант

        $[\frac{2}{4}, \frac{3}{4}]$ --- в первый

        $[\frac{3}{4}, 1]$ --- в четвертый
        
        Каждую часть разобьем таким же образом, только так, чтобы из последнего квадранта можно было перейти в дальнейший.

        Разобьем так бесконечное число раз.

        $$x\in [0,1] \quad [a_1, b_1]\supset[a_2, b_2]\supset\ldots$$

        $$K_1\supset K_2\supset K_3\supset\ldots \text{ --- квадраты}$$

        $$\bigcap K_i=\{X\}$$

        $$x\mapsto X$$

        Это кривая Пеано
    \end{example}
    \begin{definition}
        $A$ и $B$ \textbf{равномощны}, если $\exists \varphi: A\to B$ --- биекция
    \end{definition}
    \begin{definition}
        $A$ \textbf{``меньше либо равно''} $B$, обозначается $A\preccurlyeq B$

        $\exists \varphi : A\to B$ --- иньекция
    \end{definition}
    \begin{theorem}
        Кантора --- Бернштейна

        $$A\preccurlyeq B, B\preccurlyeq A \Rightarrow A \text{ и } B \text{ равномощны}$$
    \end{theorem}

    $\N$ равномощно $\Z$:

    $$\begin{cases}
        2n\mapsto n\\
    2n-1\mapsto 1-n\\
    \end{cases}$$

    \begin{definition}
        %<*счётноемножество>
        $A$ --- \textbf{счётное множество} $\Leftrightarrow$ равномощно $\N$
        %</счётноемножество>
    \end{definition}

    \setcounter{theorem}{-1}

    \begin{theorem}
        $A$ --- бесконечное $\quad \exists$ счётное $B\subset A$
    \end{theorem}

    \begin{theorem}
            $A$ --- счётное $\Rightarrow A\cap\{x\}$ --- счётное

            $A, B$ --- счётное $\Rightarrow A\cap B$ --- счётное
    \end{theorem}
    \begin{theorem}
        $\N\times\N$ --- счётное
    \end{theorem}
    \begin{theorem}
        $A\subset B, B$ --- счётное; $A$ --- бесконечное $\Rightarrow A$ --- счётное
    \end{theorem}
    \begin{consequence}
        %<*счётностьq>
        $\Q$ --- счётное
        %</счётностьq>
    \end{consequence}
    %<*счётностьqproof>
    \begin{proof}
        $$\Q_+:=\{x\in\Q:x>0\},\ \ \Q_-:=\{x\in\Q:x<0\}$$
        $$\forall q\in\N \ \ Q_p=\left\{\frac{1}{q}, \frac{2}{q}\ldots\right\} \text{ --- счётно}$$
        $$\Q_+=\bigcup\limits_{q=1}^\infty Q_p \text{ --- счётно}$$
        $$\Q_-\sim\Q_+\Rightarrow \Q_- \text{ --- счётно}$$
        $$\Q=\Q_+\cup\Q_-\cup\{0\} \text{ --- счётно}$$
    \end{proof}
    %</счётностьqproof>
    \begin{theorem}
        %<*несчетностьотрезка>
        $[0,1]$ --- несчётно
        %</несчетностьотрезка>
    \end{theorem}
    %<*несчетностьотрезкаproof>
    \begin{proof}
        Пусть $\exists \varphi: \N\to[0,1]$ --- биекция

        $[a_1, b_1]$ --- любая из частей, где нет $\varphi(1)$

        $[a_2, b_2]$ --- любая из частей, где нет $\varphi(2)$ и $[a_2, b_2]\subset [a_1, b_1]$

        $\bigcap[a_k, b_k]\supset\{x\} \quad x$ --- не имеет номера
        
        $\forall k \ \ x\in[a_k,b_k]\Rightarrow x\not=\varphi(k)$
    \end{proof}
    %</несчетностьотрезкаproof>
    \begin{definition}
        %<*мощностьконтинуума>
        $A$ равномощно $[0,1] \Rightarrow A$ имеет мощность \textbf{континуума}.
        %</мощностьконтинуума>
    \end{definition}
    \begin{theorem}
        %<*мощностьбинарныхпоследовательностей>
        $Bin=$ множество бинарных последовательностей

        $Bin$ имеет мощность континуума
        %</мощностьбинарныхпоследовательностей>
    \end{theorem}
    \begin{lemma}
        $A$ --- беск., $B$ --- счётное $\Rightarrow A\cup B$ равномощно $A$
    \end{lemma}
    \begin{proof}
        Теорема 0 + Теорема 1 \textit{(второй пункт)} + Теорема 3
    \end{proof}
    %<*мощностьбинарныхпоследовательностейproof>
    \begin{proof}
        $\varphi : Bin\to [0,1]\cup Bin_{\text{кон.}}$

        $0101\ldots\mapsto0.0101\ldots$ --- это отображение не иньективно ($0100\ldots\mapsto0.01$; $0011\ldots\mapsto0.01$)

        Иньекция достигается тем, что конечные дроби идут в $Bin_{\text{кон.}}$, а бесконечные в $[0,1]$
    \end{proof}
    %</мощностьбинарныхпоследовательностейproof>
    \begin{consequence}
        $[0,1]\times[0,1]\subset\R^2$ --- тоже континуум
    \end{consequence}
    \begin{proof}
        Докажем, что $Bin\times Bin$ равномощно $Bin$

        $(x_1x_2x_3\ldots), (y_1y_2\ldots)\mapsto x_1y_1x_2y_2x_3y_3\ldots$ --- биекция

        $$[0,1]\xrightarrow{\varphi} Bin \quad [0,1]\xrightarrow \psi Bin$$

        $$(x,y)\mapsto (\varphi(x), \psi(y))$$
    \end{proof}
    \begin{remark}
        $(0,1)$ равномощно $\R$, например через $ctg(\pi x)$
    \end{remark}

    \begin{exercise}
        Доказать, что $[0,1]$ и $(0,1)$ равномощны.
    \end{exercise}

    \begin{theorem}
        О непрерывности монотонных функций. (Важно знать формулировку)

        %<*онепрерывностимонотонныхфункций>
        $f:\langle a,b\rangle \to\R$, монотонна. Тогда
        \begin{enumerate}
            \item Точки разрыва $f$ (если есть) --- I рода
            \item $f$ --- непр. на $\langle a,b\rangle \Leftrightarrow f(\langle a,b\rangle)$ --- промежуток
        \end{enumerate}
        %</онепрерывностимонотонныхфункций>
    \end{theorem}
    %<*онепрерывностимонотонныхфункцийproof>
    \begin{proof}
        Рассмотрим $f\uparrow$
        \begin{enumerate}
            \item $\sphericalangle x_1\in\langle a,b\rangle$
            
            $$\lim\limits_{x\to x_1-0} f(x)\stackrel{def}{=}\lim\limits_{x\to x_1} f(x)|_{\langle a,x_1)}$$

            $f(x)|_{\langle a,x_1)}\uparrow$ и ограничена значением $f(x)$ $\Rightarrow$ по теореме о пределе монотонной функции $\exists\lim\limits_{x\to x_1} f(x)|_{\langle a,x_1)}=\sup f(x)|_{\langle a,x_1)}$. Аналогично $\exists\lim\limits_{x\to x_1+0}$. Таким образом, для каждой точки существует предел слева и справа.
            
            \item ``$\Rightarrow$'' следует из теоремы о сохранении промежутка.
            
            ``$\Leftarrow$''
            
            Пусть $f$ имеет разрыв в $x_0$. Тогда либо $f(x_0-0)<f(x_0)$, либо $f(x_0)<f(x_0+0)$. Рассмотрим $f(x_0)<f(x_0+0)$. В силу монотонности $f$ не принимает значений между $f(x_0)$ и $f(x_0+0) \Rightarrow $ множество значений --- не промежуток. Противоречие.
        \end{enumerate}
    \end{proof}
    %</онепрерывностимонотонныхфункцийproof>
\end{document}