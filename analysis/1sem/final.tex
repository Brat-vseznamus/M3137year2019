\documentclass[12pt, a4paper]{article}

\usepackage{lastpage}
\usepackage{mathtools}
\usepackage{xltxtra}
\usepackage{libertine}
\usepackage{amsmath}
\usepackage{amsthm}
\usepackage{amsfonts}
\usepackage{amssymb}
\usepackage{enumitem}
\usepackage{xcolor}
\usepackage[left=1.5cm, right=1.5cm, top=2cm, bottom=2cm, bindingoffset=0cm, headheight=15pt]{geometry}
\usepackage{fancyhdr}
\usepackage[russian]{babel}
% \usepackage[utf8]{inputenc}
\usepackage{catchfilebetweentags}
\usepackage{accents}
\usepackage{calc}
\usepackage{etoolbox}
\usepackage{mathrsfs}
\usepackage{wrapfig}

\providetoggle{useproofs}
\settoggle{useproofs}{false}

\pagestyle{fancy}
\lfoot{M3137y2019}
\rhead{\thepage\ из \pageref{LastPage}}

\newcommand{\R}{\mathbb{R}}
\newcommand{\Q}{\mathbb{Q}}
\newcommand{\C}{\mathbb{C}}
\newcommand{\Z}{\mathbb{Z}}
\newcommand{\B}{\mathbb{B}}
\newcommand{\N}{\mathbb{N}}

\newcommand{\const}{\text{const}}

\newcommand{\teormin}{\textcolor{red}{!}\ }

\DeclareMathOperator*{\xor}{\oplus}
\DeclareMathOperator*{\equ}{\sim}
\DeclareMathOperator{\Ln}{\text{Ln}}
\DeclareMathOperator{\sign}{\text{sign}}
\DeclareMathOperator{\Sym}{\text{Sym}}
\DeclareMathOperator{\Asym}{\text{Asym}}
% \DeclareMathOperator{\sh}{\text{sh}}
% \DeclareMathOperator{\tg}{\text{tg}}
% \DeclareMathOperator{\arctg}{\text{arctg}}
% \DeclareMathOperator{\ch}{\text{ch}}

\DeclarePairedDelimiter{\ceil}{\lceil}{\rceil}
\DeclarePairedDelimiter{\abs}{\left\lvert}{\right\rvert}

\setmainfont{Linux Libertine}

\theoremstyle{plain}
\newtheorem{axiom}{Аксиома}
\newtheorem{lemma}{Лемма}

\theoremstyle{remark}
\newtheorem*{remark}{Примечание}
\newtheorem*{exercise}{Упражнение}
\newtheorem*{consequence}{Следствие}
\newtheorem*{example}{Пример}
\newtheorem*{observation}{Наблюдение}

\theoremstyle{definition}
\newtheorem{theorem}{Теорема}
\newtheorem*{definition}{Определение}
\newtheorem*{obozn}{Обозначение}

\setlength{\parindent}{0pt}

\newcommand{\dbltilde}[1]{\accentset{\approx}{#1}}
\newcommand{\intt}{\int\!}

% magical thing that fixes paragraphs
\makeatletter
\patchcmd{\CatchFBT@Fin@l}{\endlinechar\m@ne}{}
  {}{\typeout{Unsuccessful patch!}}
\makeatother

\newcommand{\get}[2]{
    \ExecuteMetaData[#1]{#2}
}

\newcommand{\getproof}[2]{
    \iftoggle{useproofs}{\ExecuteMetaData[#1]{#2proof}}{}
}

\newcommand{\getwithproof}[2]{
    \get{#1}{#2}
    \getproof{#1}{#2}
}

\newcommand{\import}[3]{
    \subsection{#1}
    \getwithproof{#2}{#3}
}

\newcommand{\given}[1]{
    Дано выше. (\ref{#1}, стр. \pageref{#1})
}

\renewcommand{\ker}{\text{Ker }}
\newcommand{\im}{\text{Im }}
\newcommand{\grad}{\text{grad}}

\usepackage{bm}
\usepackage{xcolor}
\usepackage{sectsty}
\usepackage{catchfilebetweentags}
\usepackage{titlesec}

\setcounter{secnumdepth}{4}
\titleformat{\paragraph}
{\normalfont\normalsize\bfseries}{\theparagraph}{1em}{}
\titlespacing*{\paragraph}
{0pt}{3.25ex plus 1ex minus .2ex}{1.5ex plus .2ex}

% \allsectionsfont{\raggedright}
% \subsectionfont{\fontsize{14}{15}\selectfont}

% magical thing that fixes paragraphs
\makeatletter
\def\CatchFBT@sanitize{%
   \@sanitize
   \@makeother\{%
   \@makeother\}%
}
\makeatother

\renewcommand{\import}[3]{
    \subsection{#1}
    \ExecuteMetaData[#2]{#3}
}

\setlength\parindent{0pt}

\lhead{Конспект к экзамену}
\cfoot{}

\begin{document}

\section{Определения}

\ExecuteMetaData[opros.tex]{определения}

\ExecuteMetaData[opros2.tex]{определения}

\import{Счётное множество}{11.tex}{счётноемножество}

\import{\teormin Мощность континуума}{11.tex}{мощностьконтинуума}

\import{Фундаментальная последовательность}{8.tex}{сходящаясявсебе}

\import{Полное метрическое пространство}{8.tex}{полноепространство}

\import{Классы функций $C^n([a,b])$}{14.tex}{классыфункций}

\import{Производная $n$-го порядка}{13.tex}{производнаяnгопорядка}

\import{Многочлен Тейлора $n$-го порядка}{13.tex}{многочлентейлора}

\import{\teormin Разложения Тейлора основных элементарных функций}{14.tex}{разложениятейлора}

\section{Теоремы}

\import{Законы де Моргана}{opros.tex}{законыдеморгана}
\begin{proof}
    \ExecuteMetaData[1.md]{законыдеморганаproof}
\end{proof}

\import{Неравенство Коши-Буняковского, евклидова норма в $\R^m$}{opros.tex}{неравенствокошибуняковскогоевклдиованормавrm}
Неравенство Коши-Буняковского следует из тождества Лагранжа. Докажем его:
\begin{proof}
    \ExecuteMetaData[1.md]{тождестволагранжаproof}
    Таким образом, $$\left(\sum\limits_{(i,k)\in A\times B}a_ib_i\right)^2=\sum\limits_{(i,k)\in A\times B}a_i^2\sum\limits_{(i,k)\in A\times B}b_k^2 - \frac{1}{2}\sum\limits_{(i,k)\in A\times B}(a_ib_k-a_kb_i)^2 \leq \sum\limits_{(i,k)\in A\times B}a_i^2\sum\limits_{(i,k)\in A\times B}b_k^2$$
\end{proof}
\begin{proof}
    Альтернативное:

    \begin{align*}
        &\left(\sum_{i=1}^{n}a_i^2\right)\left(\sum_{i=1}^{n}b_i^2\right)-\left(\sum_{i=1}^{n}a_ib_i\right)^2=\\
        &\sum_{i=1}^{n}\sum_{j=1}^{n}a_i^2b_j^2-\sum_{i=1}^{n}\sum_{j=1}^{n}a_ib_ia_jb_j=\\
        &\frac12\sum_{i=1}^{n}\sum_{j=1}^{n}\left(a_i^2b_j^2+a_j^2b_i^2-2a_ib_ia_jb_j\right)=\\
        &\frac12\sum_{i=1}^{n}\sum_{j=1}^{n}(a_ib_j-a_jb_i)^2\geq0
    \end{align*}
\end{proof}

\import{Аксиома Архимеда. Плотность множества рациональных чисел в $\R$}{opros.tex}{аксиомаархимедаплотностьqвr}
\begin{proof}
    \ExecuteMetaData[2.md]{плотностьqвrproof}
\end{proof}

\import{Неравенство Бернулли}{opros.tex}{неравенствобернулли}
\begin{proof}
    База: $n=1: \ (1+x)^1\geq 1 + x$

    Переход: Дано неравенство $(1+x)^n\geq 1+nx$, оно верно при каком-то $n$. Докажем, что $(1+x)^{n+1}\geq 1 + (n+1)x$

    $$(1+x)^{n+1} = (1+x)(1+x)^n\geq (1+x)(1+nx) = \\ 1 + (n+1)x + nx^2 \geq 1 + (n+1)x$$
\end{proof}

\subsection{Единственность предела и ограниченность сходящейся последовательности}
\subsubsection{Единственность предела}
\ExecuteMetaData[3.tex]{оединственностипредела}
\ExecuteMetaData[3.tex]{оединственностипределаproof}
\subsubsection{Ограниченность сходящейся последовательности}
\ExecuteMetaData[3.tex]{ограниченностьсходящейсяпоследовательности}
\ExecuteMetaData[3.tex]{ограниченностьсходящейсяпоследовательностиproof}

\import{Теорема о предельном переходе в неравенствах для последовательностей и для функций}{opros.tex}{теоремаопредельномпереходевнеравенствахдляпоследовательностейидляфункций}
\ExecuteMetaData[3.tex]{определьномпереходевнеравенствахдляrproof}
\begin{proof}
    По Гейне.

    $\forall (x_n)\to a, x_n\in X, x_n\not=a$:

    $$f(x_n)\to A, g(x_n)\to B, \forall x \ \ f(x) \leq g(x) \Rightarrow f(x_n)\leq g(x_n) \Rightarrow A\leq B$$
\end{proof}

\import{\teormin Теорема о двух городовых}{3.tex}{одвухгородовых}
\ExecuteMetaData[3.tex]{одвухгородовыхproof}

\subsection{Бесконечно малая последовательность}
Произведение бесконечно малой последовательности на ограниченную --- бесконечно малая последовательность, т.е. $(x_n)$ --- беск. малая, $(y_n)$ --- ограничена $\Rightarrow x_ny_n$ --- беск. малая
\begin{proof}
    Возьмём $K$ такое, что $\forall n \ \ |y_n|\leq K$.

    $$\forall \varepsilon > 0 \ \ \exists N \ \ \forall n > N \ \ |x_n|\leq\frac{\varepsilon}{K}$$

    $$|x_ny_n|\leq\frac{\varepsilon}{K}K=\varepsilon \Rightarrow x_ny_n\to0$$
\end{proof}

\import{\teormin Теорема об арифметических свойствах предела последовательности в нормированном пространстве и в $\R$}{3.tex}{арифметическиесвойствапредела}
\ExecuteMetaData[3.tex]{арифметическиесвойствапределаproof}
\ExecuteMetaData[3.tex]{арифметическиесвойствапределавr}
\ExecuteMetaData[3.tex]{арифметическиесвойствапределавrproof}

\subsection{Неравенство Коши-Буняковского в линейном пространстве, норма, порожденная скалярным произведением}
\subsubsection{Неравенство Коши-Буняковского в линейном пространстве}
\ExecuteMetaData[4.tex]{неравенствокошибуняковского}
\begin{proof}
\ExecuteMetaData[4.tex]{неравенствокошибуняковскогоproof}
\end{proof}
\subsubsection{Норма, порожденная скалярным произведением}
\ExecuteMetaData[4.tex]{нормапорожденнаяскалярнымпроизведением}
\ExecuteMetaData[4.tex]{нормапорожденнаяскалярнымпроизведениемproof}
% \ExecuteMetaData[3.tex]{метрикапорожденнаянормой}

\subsection{Леммы о непрерывности скалярного произведения и покоординатной сходимости в $\R^n$}
\subsubsection{О покоординатной сходимости в $\R^m$}
\ExecuteMetaData[4.tex]{опокоординатнойсходимости}
\ExecuteMetaData[4.tex]{опокоординатнойсходимостиproof}

\subsubsection{О непрерывности скалярного произведения}
\ExecuteMetaData[4.tex]{онепрерывностискалярногопроизведения}
\ExecuteMetaData[4.tex]{онепрерывностискалярногопроизведенияproof}

\subsection{Открытость открытого шара}
$B(a,r)=\{x\in X : \rho(a,x)<r\}$ --- открыт
\begin{proof}
    $x_0\in B(a, r)$

    Докажем, что $x_0$ --- внутренняя, т.е. $\exists U(x_0)\subset B(a, r)$

    $k:=r-\rho(a, x_0)$

    Докажем, что $B(x_0, k)\subset B(a, r)$

    $$\forall x\in B(x_0, k) \quad \rho(x, x_0)<k$$
    
    $$\rho(a, x_0) + \rho(x, x_0)<r$$

    $$\rho(x, a) \leq \rho(a, x_0) + \rho(x, x_0)<r$$
\end{proof}

\import{Теорема о свойствах открытых множеств}{4.tex}{освойствахзамкнутыхмножеств}
\ExecuteMetaData[4.tex]{освойствахзамкнутыхмножествproof}

\import{Теорема о связи открытых и замкнутых множеств, свойства замкнутых множеств}{opros.tex}{освязиоткрытыхизамкнутыхмножеств}
\ExecuteMetaData[4.tex]{освязиоткрытыхизамкнутыхмножествproof}
\begin{proof}
    \begin{enumerate}
        \item $(\bigcap F_\alpha)^c=X\setminus(\bigcap F_\alpha)=\bigcup (X\setminus F_\alpha)$
        
        $F_\alpha$ --- закрыто $\Rightarrow X\setminus F_\alpha$ --- открыто $\Rightarrow \bigcup (X\setminus F_\alpha)$ --- открыто

        $(\bigcap F_\alpha)^c$ --- открыто $\Rightarrow \bigcap F_\alpha$ --- замкнуто

        \item $(\bigcup F_i)^c=\bigcap(F_i)^c$
        
        $\bigcap(F_i)^c$ --- открыто, т.к. $F_i^c$ --- открыто $\Rightarrow (\bigcup F_i)^c$ --- открыто $\Rightarrow \bigcup F_i$ --- замкнуто
    \end{enumerate}
\end{proof}

\subsection{\teormin Теорема об арифметических свойствах предела последовательности (в $\overline\R$). Неопределенности}
\subsubsection{Теорема об арифметических свойствах предела последовательности (в $\overline\R$)}
\paragraph{По Кохасю}
\ExecuteMetaData[5.tex]{арифметическиесвойствапределоввrсчертой}
\paragraph{По Виноградову}
\begin{enumerate}
    \item $x_n\to+\infty, \{y_n\}$ --- огр. снизу $\Rightarrow x_n+y_n\to+\infty$
    \item $x_n\to-\infty, \{y_n\}$ --- огр. сверху $\Rightarrow x_n+y_n\to-\infty$
    \item $x_n\to\infty, \{y_n\}$ --- огр. $\Rightarrow x_n+y_n\to\infty$
    \item $x_n\to\pm\infty, \forall n \ \ y_n>0 $ или $y_n\to b>0$ $\Rightarrow x_ny_n\to\pm\infty$
    \item $x_n\to\pm\infty, \forall n \ \ y_n<0 $ или $y_n\to b<0$ $\Rightarrow x_ny_n\to\mp\infty$
    \item $x_n\to\infty, \forall n \ \ |y_n|>0 $ или $y_n\to b\not=0$ $\Rightarrow x_ny_n\to\infty$
    \item $x_n\to a\not=0, y_n\to0, \forall n \ \ y_n\not=0 \Rightarrow \frac{x_n}{y_n}\to\infty$
    \item $x_n\to a\in\R, y_n\to+\infty \Rightarrow \frac{x_n}{y_n}\to0$
    \item $x_n\to\infty, y_n\to a\in\R \Rightarrow \frac{x_n}{y_n}\to\infty$
\end{enumerate}
% \begin{proof}
%     Докажем 4 для $x_n\to\infty$

%     $$x_n\to \infty\Leftrightarrow \forall E'>0 \ \ \exists N_1 \ \ \forall n>N_1 \ \ x_n>E'$$
%     $$y_n\to a \Leftrightarrow \forall \varepsilon>0 \ \ \exists N_2 \ \ \forall n>N_2 \ \ |y_n-a|<\varepsilon \Rightarrow y_n>a-\varepsilon$$
%     $$E':=\frac{E}{a-\varepsilon} \quad \forall n>\max(N_1, N_2) \ \ x_ny_n>E'(a-\varepsilon)=\frac{E}{a-\varepsilon}(a-\varepsilon)=E$$
% \end{proof}
\begin{proof}
    Тривиально.
\end{proof}
\subsubsection{Неопределенности}
\ExecuteMetaData[5.tex]{неопределенности}

\import{\teormin Теорема Кантора о стягивающихся отрезках}{5.tex}{теоремакантора}
\ExecuteMetaData[5.tex]{теоремакантораproof}

\import{Теорема о существовании супремума}{5.tex}{осуществованиисупремума}
\ExecuteMetaData[5.tex]{осуществованиисупремумаproof}

\import{Лемма о свойствах супремума}{6.tex}{освойствахсупремума}
\ExecuteMetaData[6.tex]{освойствахсупремумаproof}

\import{Теорема о пределе монотонной последовательности}{6.tex}{определемонотоннойпоследовательности}
\ExecuteMetaData[6.tex]{определемонотоннойпоследовательностиproof}

\import{Определение числа $e$, соответствующий замечательный предел}{opros.tex}{определениеe}

\import{Теорема об открытых и замкнутых множествах в пространстве и в подпространстве}{6.tex}{открытыеизамкнутыемножествавпространствеиподпространстве}
\ExecuteMetaData[6.tex]{открытыеизамкнутыемножествавпространствеиподпространствеproof}

\import{Теорема о компактности в пространстве и в подпространстве}{6.tex}{окомпактностивпространствеиподпространстве}
\ExecuteMetaData[6.tex]{окомпактностивпространствеиподпространствеproof}

\import{Простейшие свойства компактных множеств}{7.tex}{опростейшихсвойствахкомпактныхмножеств}
\ExecuteMetaData[7.tex]{опростейшихсвойствахкомпактныхмножествproof}

\import{Лемма о вложенных параллелепипедах}{7.tex}{овложенныхпараллелепипедах}
\ExecuteMetaData[7.tex]{овложенныхпараллелепипедахproof}

\import{Компактность замкнутого параллелепипеда в $\R^m$}{7.tex}{компактностьпарллелепипеда}
\ExecuteMetaData[7.tex]{компактностьпарллелепипедаproof}

\import{\teormin Теорема о характеристике компактов в $\R^m$}{7.tex}{охарактеристикекомпактоввrm}
\ExecuteMetaData[7.tex]{охарактеристикекомпактоввrmproof}
\ExecuteMetaData[8.tex]{охарактеристикекомпактоввrmproof}

\import{Эквивалентность определений Гейне и Коши}{7.tex}{эквивалентностьопределенийгейнеикоши}
\ExecuteMetaData[7.tex]{эквивалентностьопределенийгейнеикошиproof}

\subsection{Единственность предела, локальная ограниченность отображения, имеющего предел, теорема о стабилизации знака}
\subsubsection{Единственность предела}
\ExecuteMetaData{7.tex}{оединственностипредела}
\ExecuteMetaData[7.tex]{оединственностипределаproof}
\subsubsection{Локальная ограниченность отображения, имеющего предел}
\ExecuteMetaData[7.tex]{олокальнойограниченностиотображенияимеющегопредел}
\ExecuteMetaData[7.tex]{олокальнойограниченностиотображенияимеющегопределproof}
\subsubsection{Теорема о стабилизации знака}
\ExecuteMetaData[7.tex]{остабилизациизнака}
\ExecuteMetaData[7.tex]{остабилизациизнакаproof}

\import{Арифметические свойства пределов отображений. Формулировка для $\overline\R$}{7.tex}{обарифметическихсвойствахпредела}
\ExecuteMetaData[7.tex]{обарифметическихсвойствахпределаproof}
\ExecuteMetaData[7.tex]{арифметическиесвойствапределадляoverliner}
\ExecuteMetaData[7.tex]{арифметическиесвойствапределадляoverlinerproof}

\import{Принцип выбора Больцано--Вейерштрасса}{8.tex}{принципвыборабольцановейерштрасса}
\ExecuteMetaData[8.tex]{принципвыборабольцановейерштрассаproof}

\import{Сходимость в себе и ее свойства}{8.tex}{сходящаясявсебе}
\ExecuteMetaData[8.tex]{сходящаясявсебесвойства}
\ExecuteMetaData[8.tex]{сходящаясявсебесвойстваproof}

\subsection{Критерий Коши для последовательностей и отображений}
\subsubsection{Для последовательностей}
\ExecuteMetaData[8.tex]{критерийкошидляпоследовательностей}
\ExecuteMetaData[8.tex]{критерийкошидляпоследовательностейproof}
\subsubsection{Для отображений}
\ExecuteMetaData[8.tex]{критерийбольцанокошидляотображений}
\ExecuteMetaData[8.tex]{критерийбольцанокошидляотображенийproof}

\import{\teormin Теорема о пределе монотонной функции}{8.tex}{определемонотоннойфункции}
\ExecuteMetaData[8.tex]{определемонотоннойфункцииproof}

\subsection{Свойства непрерывных отображений: арифметические, стабилизация знака, композиция}
\subsubsection{Арифметические}
\ExecuteMetaData[8.tex]{арифметическиесвойстванепрерывныхотображений}
\ExecuteMetaData[8.tex]{арифметическиесвойстванепрерывныхотображенийproof}
\subsubsection{Стабилизация знака}
\ExecuteMetaData[9.tex]{стабилизациязнака}
\ExecuteMetaData[9.tex]{стабилизациязнакаproof}
\subsubsection{Непрерывность композиции непрерывных отображений}
\ExecuteMetaData[10.tex]{онепрерывностикомпозиции}
\ExecuteMetaData[10.tex]{онепрерывностикомпозицииproof}

\subsection{Непрерывность композиции и соответствующая теорема для пределов}
\subsubsection{Непрерывность композиции}
Дана выше.
\subsubsection{Cоответствующая теорема для пределов}
\ExecuteMetaData[10.tex]{определекомпозиции}
\ExecuteMetaData[10.tex]{определекомпозицииproof}

\subsection{\teormin Теорема о замене на эквивалентную при вычислении пределов. Таблица эквивалентных}
\subsubsection{Теорема о замене на эквивалентную при вычислении пределов}
\ExecuteMetaData[10.tex]{озамененаэквивалентную}
\ExecuteMetaData[10.tex]{озамененаэквивалентнуюproof}
\subsubsection{Таблица эквивалентных}
Дана выше. (\ref{equivtable}, стр. \pageref{equivtable})

\import{Теорема единственности асимптотического разложения}{10.tex}{оединственностиасимтотическогоразложения}
\ExecuteMetaData[10.tex]{оединственностиасимтотическогоразложенияproof}

\import{\teormin Теорема о топологическом определении непрерывности}{10.tex}{топологическоеопределениенепрерывности}
\ExecuteMetaData[10.tex]{топологическоеопределениенепрерывностиproof}

\import{\teormin Теорема Вейерштрасса о непрерывном образе компакта. Следствия}{10.tex}{вейерштрассаонепрерывномобразекомпакта}
\ExecuteMetaData[10.tex]{вейерштрассаонепрерывномобразекомпактаproof}
\ExecuteMetaData[10.tex]{следствиявейерштрасса}

\import{Лемма о связности отрезка}{11.tex}{освязностиотрезка}
\ExecuteMetaData[11.tex]{освязностиотрезкаproof}

\import{\teormin Теорема Больцано-Коши о промежуточном значении}{11.tex}{больцанокошиопромежуточномзначении}
\ExecuteMetaData[11.tex]{больцанокошиопромежуточномзначенииproof}

\import{Теорема о сохранении промежутка}{11.tex}{осохранениипромежутка}
\ExecuteMetaData[11.tex]{осохранениипромежуткаproof}

\ExecuteMetaData[opros2.tex]{больцанокошиосохранениилинейнойсвязности}
\ExecuteMetaData[opros2.tex]{больцанокошиосохранениилинейнойсвязностиproof}

\import{Описание линейно связных множеств в $\R$}{11.tex}{линейносвзяноевr}
\ExecuteMetaData[11.tex]{линейносвзяноевrproof}

\import{Теорема о бутерброде}{11.tex}{теоремаобутерброде}
\ExecuteMetaData[11.tex]{теоремаобутербродеproof}

\import{Теорема о вписанном $n$-угольнике максимальной площади}{11.tex}{овписанномnугольникемаксимальнойплощади}
\ExecuteMetaData[11.tex]{овписанномnугольникемаксимальнойплощадиproof}

\subsection{\teormin Теорема о непрерывности монотонной функции. Следствие о множестве точек разрыва}
\subsubsection{Теорема о непрерывности монотонной функции}
\ExecuteMetaData[11.tex]{онепрерывностимонотонныхфункций}
\ExecuteMetaData[11.tex]{онепрерывностимонотонныхфункцийproof}
\subsubsection{Следствие о множестве точек разрыва}
\ExecuteMetaData[12.tex]{омножестветочекразрыва}
\ExecuteMetaData[12.tex]{омножестветочекразрываproof}

\import{Теорема о существовании и непрерывности обратной функции}{12.tex}{осуществованииинепрерывностиобратнойфункции}
\ExecuteMetaData[12.tex]{осуществованииинепрерывностиобратнойфункцииproof}

\import{Счетность множества рациональных чисел}{11.tex}{счётностьq}
\ExecuteMetaData[11.tex]{счётностьqproof}

\import{Несчетность отрезка}{11.tex}{несчетностьотрезка}
\ExecuteMetaData[11.tex]{несчетностьотрезкаproof}

\import{Континуальность множества бинарных последовательностей}{11.tex}{мощностьбинарныхпоследовательностей}
\ExecuteMetaData[11.tex]{мощностьбинарныхпоследовательностейproof}

\subsection{Равносильность двух определений производной. Правила дифференцирования.}
\subsubsection{Равносильность двух определений производной}
\ExecuteMetaData[12.tex]{равносильностьопределенийпроизводной}
\ExecuteMetaData[12.tex]{равносильностьопределенийпроизводнойproof}
\subsubsection{Правила дифференцирования}
\ExecuteMetaData[12.tex]{правиладифференциирования}
\ExecuteMetaData[12.tex]{правиладифференциированияproof}

\subsection{Дифференцирование композиции и обратной функции}
\subsubsection{Дифференцирование композиции}
\ExecuteMetaData[12.tex]{дифференциированиекомпозиции}
\ExecuteMetaData[12.tex]{дифференциированиекомпозицииproof}
\subsubsection{Дифференцирование обратной функции}
\ExecuteMetaData[12.tex]{дифференциированиеобратной}
\ExecuteMetaData[12.tex]{дифференциированиеобратнойproof}

\subsection{Теорема Ферма \textit{(с леммой)}}
\subsubsection{Лемма}
\ExecuteMetaData[12.tex]{леммадлятеоремыферма}
\ExecuteMetaData[12.tex]{леммадлятеоремыферма2}
\ExecuteMetaData[12.tex]{леммадлятеоремыфермаproof}
\subsubsection{Теорема Ферма}
\ExecuteMetaData[12.tex]{теоремаферма}
\ExecuteMetaData[12.tex]{теоремафермаproof}

\subsection{Теорема Ролля. Вещественность корней многочлена Лежандра}
\subsubsection{Теорема Ролля}
\ExecuteMetaData[12.tex]{теоремаролля}
\ExecuteMetaData[12.tex]{теоремаролляproof}
\subsubsection{Вещественность корней многочлена Лежандра}
\ExecuteMetaData[12.tex]{полиномылежандра}
\ExecuteMetaData[12.tex]{полиномылежандраproof}

\subsection{\teormin Теоремы Лагранжа и Коши. Следствия об оценке приращения и о пределе производной}
\subsubsection{Теорема Лагранжа}
\ExecuteMetaData[13.tex]{теоремалагранжа}
\ExecuteMetaData[13.tex]{теоремалагранжаremark}
\ExecuteMetaData[13.tex]{теоремалагранжаproof}
\subsubsection{Теорема Коши}
\ExecuteMetaData[13.tex]{теоремакоши}
\ExecuteMetaData[13.tex]{теоремакошиproof}
\subsubsection{Следствия об оценке приращения и о пределе производной}
\ExecuteMetaData[13.tex]{следствия}
\ExecuteMetaData[13.tex]{следствияproof}

\subsection{Теорема Дарбу. Следствия}
\subsubsection{Теорема Дарбу}
\ExecuteMetaData[13.tex]{теоремадарбу}
\ExecuteMetaData[13.tex]{теоремадарбуproof}
\subsubsection{Следствия}
\ExecuteMetaData[13.tex]{теоремадарбуследствия}

\import{Теорема о свойствах показательной функции}{13.tex}{свойствапоказательнойфункции}
\ExecuteMetaData[13.tex]{свойствапоказательнойфункцииproof}

\subsection{Выражение произвольной показательной функции через экспоненту. Два следствия}
$\exists$ показательная функция $f_0$, удовлетворяющая
$$\frac{f_0(x)-1}{x}\xrightarrow[x\to0]{}1, f(x)=\sum\limits_{n=0}^{+\infty}\frac{x^n}{n!}$$
$f$ --- произвольная показательная функция. Тогда $\exists \alpha\in\R : \forall x : f(x)=f_0(\alpha x)$
\begin{proof}
    $a:=f(1), \exists \alpha : f_0(\alpha)=a$

    $f_0(\alpha x)$ и есть $f(x)$, т.к. у них совпадает значение в 0.
    
    Докажем, что $f_0(\alpha x)$ --- показ. функция.
    $$f_0(\alpha(x+y))=f_0(\alpha x + \alpha y)=f_0(\alpha x)f_0(\alpha y)$$
\end{proof}
\begin{consequence}
    $f_0$ --- единственна.
\end{consequence}
\begin{proof}
    Пусть $h(x)$ --- ещё одна такая функция $\Rightarrow \exists \alpha : h(x)=f_0(\alpha x)$
    $$1\xleftarrow[x\to0]{} \frac{h(x)-1}{x}=\frac{f_0(\alpha x)-1}{x}=\frac{f_0(\alpha x)-1}{\alpha x}\cdot \alpha\xrightarrow[x\to0]{} \alpha \Rightarrow \alpha=1$$
\end{proof}
\begin{consequence}
    $\forall a>0, a\not=1 \quad \exists! f : f(1)=a$
\end{consequence}
\begin{proof}
    Для этого $a \ \ \exists! \alpha : f_0(\alpha)=a$

    $f(x):=f_0(\alpha x)$

    $f(1)=a=f(\alpha)$
\end{proof}

\subsection{Показательная функция от произведения}
$f$ --- показ. ф-ция. $\forall r\in\R \quad f(rx)=(f(x))^r$

\begin{proof}
    Для $r\in\Q$ доказано выше. (\ref{exponentofmultiplication}, стр. \pageref{exponentofmultiplication})

    $\sphericalangle r\in\R$. $\exists a_{n}\in\Q \to r$
    
    $f(a_nx)=f(x)^{a_n}\to f(x)^r$
\end{proof}

\subsection{Формула Тейлора с остатком в форме Пеано}
Остаток: $T_n:=o((x-x_0)^{n+1}), x\to x_0$.
$$f(x)=f(x_0)+\ldots+\frac{f^{(n)}(x_0)}{n!}(x-x_0)^n+o((x-x_0)^{n+1})$$
\begin{proof}
    \begin{lemma}
        $\varphi:\langle a,b \rangle \to \R, x_0 \in \langle a,b \rangle, \varphi$ $n$ раз дифференциируема в $x_0$ и $\varphi(x_0)=\varphi'(x_0)=\ldots=\varphi^{(n)}(x_0)=0$.
        
        Тогда $\varphi(x)=o((x-x_0)^n), x\to x_0$
        \begin{proof}
            База: $n=1$.

            $$\varphi(x)=\varphi(x_0)+\varphi'(x_0)(x-x_0)+o(x-x_0)=o(x-x_0)$$

            Переход:

            $\varphi'(x)=o((x-x_0)^n)$ по индукционному переходу.

            $$r(x)=r(x_0)+r'(x_0)(x-x_0)+o(x-x_0)=0+o((x-x_0)^n)(x-x_0)+o(x-x_0)=$$
            $$=o((x-x_0)^{n+1})+o(x-x_0)=o((x-x_0)^{n+1})$$
        \end{proof}
    \end{lemma}

    $T_n:=f(x)-f(x_0)-\ldots-\frac{f^{(n)}(x_0)}{n!}(x-x_0)^n$ --- подходит в лемму $\Rightarrow T_n=o((x-x_0)^{n+1})$
\end{proof}

\subsection{\teormin Формула Тейлора с остатком в форме Лагранжа}
Остаток: $R_n:=\frac{f^{(n+1)}(c)}{(n+1)!}(x-x_0)^{n+1}$, где $c\in(x_0, x)$ \textit{(или наоборот, если $x<x_0$)}.
$$f(x)=f(x_0)+\ldots+\frac{f^{(n)}(x_0)}{n!}(x-x_0)^n+\frac{f^{(n+1)}(c)}{(n+1)!}(x-x_0)^{n+1}$$
\begin{proof}
    $$g(t):=f(x)-f(t)-f'(t)(x-t)-\ldots-\frac{f^{(n)}(t)}{n!}(x-t)^n$$
    $g(x)=0, g(x_0)\stackrel{def}{=}R_n$
    $$g'(t)=\left(f(x)-f(t)-f'(t)(x-t)-\ldots-\frac{f^{(n)}(t)}{n!}(x-t)^n\right)'$$
    $$g'(t)=0-f'(t)-(f''(t)(x-t)-f'(t))-\ldots-\left(\frac{f^{(n)}(t)}{n!}(x-t)^n\right)'$$
    $$g'(t)=-\frac{f^{(n+1)}(t)}{n!}(x-t)^n$$
    $h(x):=(x_0-x)^{n+1}, n+1>0$

    По т. Коши: \textit{(можно применить, т.к. $h'\not=0$, $g,h$ --- дифф. на $(x,x_0)$)}
    $$\frac{g(x)-g(x_0)}{h(x)-h(x_0)}=\frac{g'(c)}{h'(c)}$$
    и при этом $c\in(x,x_0)$.
    $$\frac{0-R_n}{0-(x-x_0)^{n+1}}=\frac{-\cfrac{f^{(n+1)}(t)}{n!}(x-t)^n}{-(n+1)(x-c)^n}$$
    $$R_n=\frac{f^{(n+1)}(c)}{(n+1)!}(x-x_0)^{n+1}$$
\end{proof}

\import{Метод Ньютона}{14.tex}{методньютона}

\import{Иррациональность числа $e^2$}{14.tex}{иррациональностьe2}
\ExecuteMetaData[14.tex]{иррациональностьe2proof}

\subsection{Следствие об оценке сходимости многочленов Тейлора к функции. Примеры}
$|f^{(n+1)}|\leq M$
$$|R_n(x_0)f(x)|\leq\frac{M|x-x_0|^{n+1}}{(n+1)!}$$
Пусть $f\in C^{\infty}\langle a,b \rangle, \exists M>0 : \forall n\in N, t\in\langle a,b \rangle \quad |f^{(n)}(t)|\leq M$. Тогда $$T_n(x_0)f(x)\xrightarrow[n\to\infty]{} f(x)$$

\import{Теорема о разложении рациональной функции на простейшие дроби}{14.tex}{оразложениирациональнойдроби}
\ExecuteMetaData[14.tex]{оразложениирациональнойдробиproof}

\end{document}