\documentclass[12pt, a4paper]{article}

\usepackage{lastpage}
\usepackage{mathtools}
\usepackage{xltxtra}
\usepackage{libertine}
\usepackage{amsmath}
\usepackage{amsthm}
\usepackage{amsfonts}
\usepackage{amssymb}
\usepackage{enumitem}
\usepackage{xcolor}
\usepackage[left=1.5cm, right=1.5cm, top=2cm, bottom=2cm, bindingoffset=0cm, headheight=15pt]{geometry}
\usepackage{fancyhdr}
\usepackage[russian]{babel}
% \usepackage[utf8]{inputenc}
\usepackage{catchfilebetweentags}
\usepackage{accents}
\usepackage{calc}
\usepackage{etoolbox}
\usepackage{mathrsfs}
\usepackage{wrapfig}

\providetoggle{useproofs}
\settoggle{useproofs}{false}

\pagestyle{fancy}
\lfoot{M3137y2019}
\rhead{\thepage\ из \pageref{LastPage}}

\newcommand{\R}{\mathbb{R}}
\newcommand{\Q}{\mathbb{Q}}
\newcommand{\C}{\mathbb{C}}
\newcommand{\Z}{\mathbb{Z}}
\newcommand{\B}{\mathbb{B}}
\newcommand{\N}{\mathbb{N}}

\newcommand{\const}{\text{const}}

\newcommand{\teormin}{\textcolor{red}{!}\ }

\DeclareMathOperator*{\xor}{\oplus}
\DeclareMathOperator*{\equ}{\sim}
\DeclareMathOperator{\Ln}{\text{Ln}}
\DeclareMathOperator{\sign}{\text{sign}}
\DeclareMathOperator{\Sym}{\text{Sym}}
\DeclareMathOperator{\Asym}{\text{Asym}}
% \DeclareMathOperator{\sh}{\text{sh}}
% \DeclareMathOperator{\tg}{\text{tg}}
% \DeclareMathOperator{\arctg}{\text{arctg}}
% \DeclareMathOperator{\ch}{\text{ch}}

\DeclarePairedDelimiter{\ceil}{\lceil}{\rceil}
\DeclarePairedDelimiter{\abs}{\left\lvert}{\right\rvert}

\setmainfont{Linux Libertine}

\theoremstyle{plain}
\newtheorem{axiom}{Аксиома}
\newtheorem{lemma}{Лемма}

\theoremstyle{remark}
\newtheorem*{remark}{Примечание}
\newtheorem*{exercise}{Упражнение}
\newtheorem*{consequence}{Следствие}
\newtheorem*{example}{Пример}
\newtheorem*{observation}{Наблюдение}

\theoremstyle{definition}
\newtheorem{theorem}{Теорема}
\newtheorem*{definition}{Определение}
\newtheorem*{obozn}{Обозначение}

\setlength{\parindent}{0pt}

\newcommand{\dbltilde}[1]{\accentset{\approx}{#1}}
\newcommand{\intt}{\int\!}

% magical thing that fixes paragraphs
\makeatletter
\patchcmd{\CatchFBT@Fin@l}{\endlinechar\m@ne}{}
  {}{\typeout{Unsuccessful patch!}}
\makeatother

\newcommand{\get}[2]{
    \ExecuteMetaData[#1]{#2}
}

\newcommand{\getproof}[2]{
    \iftoggle{useproofs}{\ExecuteMetaData[#1]{#2proof}}{}
}

\newcommand{\getwithproof}[2]{
    \get{#1}{#2}
    \getproof{#1}{#2}
}

\newcommand{\import}[3]{
    \subsection{#1}
    \getwithproof{#2}{#3}
}

\newcommand{\given}[1]{
    Дано выше. (\ref{#1}, стр. \pageref{#1})
}

\renewcommand{\ker}{\text{Ker }}
\newcommand{\im}{\text{Im }}
\newcommand{\grad}{\text{grad}}

\lhead{Конспект по матанализу}
\cfoot{}
\rfoot{December 16, 2019}

\begin{document}
\begin{theorem}
    Лагранжа.

    %<*теоремалагранжа>
    $f:[a,b]\to\R$ --- непр., дифф. в $(a, b)$. Тогда $\exists c\in(a,b)$, такое что:
    $$f(b)-f(a)=f'(c)(b-a)$$
    %</теоремалагранжа>
\end{theorem}
\begin{theorem}
    Коши.
    %<*теоремакоши>
    $f,g : [a,b]\to\R$

    $f,g$ --- дифф. в $(a, b); g'\not=0$ на $(a,b)$. Тогда

    $\exists c\in(a,b)$, такое что:
    $$\frac{f(b)-f(a)}{g(b)-g(a)}=\frac{f'(c)}{g'(c)}$$
    %</теоремакоши>
\end{theorem}
%<*теоремалагранжаproof>
\begin{proof}
    Следует из теоремы Коши при $g(x)=x$
\end{proof}
%</теоремалагранжаproof>
\begin{remark}
    Если $g(b)=g(a)$, то по т. Ролля $\ldots$
\end{remark}
\begin{remark}
    От автора конспекта: Кохась действительно не дописал замечание.
\end{remark}
%<*теоремалагранжаremark>
\begin{remark}
    Теорему Лагранжа можно интерпретировать как следующее: $\frac{f(b)-f(a)}{b-a}$ --- тангенс угла между хордой графика и горизонталью, а $f'(c)$ --- касательная. Таким образом, если провести хорду графика, то можно найти точку между точками пересечения графика и хорды такую, что касательная к графику будет параллельна этой хорде.
\end{remark}
%</теоремалагранжаremark>
%<*теоремакошиproof>
\begin{proof}
    Теоремы Коши.

    $F(x):=f(x)-kg(x)$

    Подберем $k$ такое, что $F(b)=F(a)$
    $$f(b)-kg(b)=f(a)-kg(a)$$
    $$k=\frac{f(b)-f(a)}{g(b)-g(a)}$$

    По т. Ролля $\exists c:F'(c)=0$
    $$f'(c)-kg'(c)=0$$
    $$k=\frac{f'(c)}{g'(c)}$$
\end{proof}
%</теоремакошиproof>
\begin{consequence}
    %<*следствия>
    \begin{enumerate}
        \item $f$ непр. на $[a,b]$, дифф. в $(a,b)$

        $\exists M : \forall x \ \ |f'(x)|\leq M$
    
        Тогда $\forall x, x+h\in[a,b]$
    
        $$|f(x+h)-f(x)|\leq M|h|$$

        \item $f$ --- непр. на $[a,b\rangle$, дифф. на $(a,b\rangle$
        
        $\exists \lim\limits_{x\to a+0}f'(x)=k\in\overline\R$

        Тогда $f'_+(a)=k$
    \end{enumerate}
    %</следствия>
\end{consequence}
%<*следствияproof>
\begin{proof}
    Следствия 2.

    $\exists a<c<a+h$, такой что:

    $$\frac{f(a+h)-f(a)}{h}=f'(c)\xrightarrow[h\to0]{} k$$
\end{proof}
%</следствияproof>

$\sphericalangle f(x)=\begin{cases}
    e^{-\frac{1}{x^2}} & ,x\not=0 \\
    0 & ,x=0
\end{cases}$

$f$ --- дифф. при $x>0$

$f'=e^{-\frac{1}{x^2}}\cdot \frac{2}{x^3}$

$\lim\limits_{x\to0} \frac{2e^{-\frac{1}{x^2}}}{x^3}=0$ (позже)

Это был контрпример --- функция, которая везде дифф., но $\not\exists\lim$

\begin{theorem}
    Дарбу.

    %<*теоремадарбу>
    $f:[a,b]\to\R$ --- дифф. на $[a,b]$

    Тогда $\forall C$ лежащего между $f'(a), f'(b)$

    $$\exists c\in(a,b) : f'(c) = C$$
    %</теоремадарбу>
\end{theorem}
%<*теоремадарбуproof>
\begin{proof}
    $F(x):=f(x)-C\cdot x$ --- у неё $\exists\max\limits_{[a,b]}$ \textit{(в силу непрерывности)}

    $F'(x)=f'(x)-C \quad F'(a)$ и $F'(b)$ разных знаков.

    \begin{enumerate}
        \item $F'(a)>0 \ \ F'(b) < 0$
        
        По лемме при $x>a$, близких к $a \ \ f(x) > f(a) \Rightarrow \max f$ достигается в $c\in(a,b)$
    \end{enumerate}
\end{proof}
%</теоремадарбуproof>
%<*теоремадарбуследствия>
\begin{consequence}
    \begin{enumerate}
        \item Функция $f'$ обладает свойством ``сохранять промежуток''
        \item $f'$ не может иметь разрывов вида ``скачок''
    \end{enumerate}
\end{consequence}
%</теоремадарбуследствия>
\section{Показательная функция}
$\forall x, y \quad f(x+y)=f(x)\cdot f(y) \quad (*)$

$f:\R\to\R$, непр.

\begin{definition}
    \textbf{Показательная функция} $f:\R\to\R$, непр.

    $\not\equiv 0, \not\equiv 1$ и удовл. $(*)$
\end{definition}
\begin{theorem}
    %<*свойствапоказательнойфункции>
    $f$ --- показ. ф-ция

    Тогда:
    \begin{enumerate}\label{exponentofmultiplication}
        \item $\forall x \ \ f(x)>0; f(0)=1$
        \item $\forall r \in\Q \quad f(rx)=(f(x))^r$ 
        \item $f$ --- строго монот.: $a:=f(1)$
        
        Тогда $a\not=1$, если $a>1$ --- возр., если $a<1$ --- убыв.

        \item Множество значений $f \quad (0, +\infty)$
        \item $\tilde f(1)=f(1)$, тогда $f=\tilde f$
    \end{enumerate}
    %</свойствапоказательнойфункции>
\end{theorem}
%<*свойствапоказательнойфункцииproof>
\begin{proof}
    \begin{enumerate}
        \item $f\not\equiv 0 \ \ \exists f(x_0)\not=0$

        $x=x_0, y=0 \quad f(x_0+0)=f(x_0)\cdot f(0)\Rightarrow f(0)=1$

        Если $f(x_1)=0$, тогда $$\forall x \quad f(x)=f(x-x_1)\cdot f(x_1) = 0$$
        $$f(x)=f\left(\frac{x}{2}\right)\cdot f\left(\frac{x}{2}\right)>0$$

        \item Как в опр. ст. с рациональным показателем
        \begin{enumerate}
            \item $r=1$
            \item $r\in\N$
            $$f(2x)=f(x+x)=f(x)\cdot f(x) = f(x^2)$$
            $$f((n+1)x)=f(nx+x)=f(nx)\cdot f(x)=(f(x))^nf(x)=(f(x))^{n+1}$$
            \item $r\in$``$-\N$''
            $$1=f(0)=f(nx+(-n)x)=f(nx)\cdot f(-nx)=(f(x))^nf(-nx)$$
            \item $r=0$
            $$f(rx)=f(0)=1=(f(x))^0$$
            \item $r=\frac{1}{n}$
            $$f(x)=f(n\cdot\frac{x}{n})=(f(\frac{x}{n}))^n$$
            $$f(\frac{1}{n}x)=(f(x))^{\frac{1}{n}}$$
            \item $r=\frac{m}{n} \quad m\in\Z, n\in\N$
            $$f(\frac{m}{n}x)=f(m\cdot(\frac{1}{n}x))=(f(\frac{1}{n}x))^m=(f(x)^{\frac{1}{n}})^m$$
        \end{enumerate}
        \item $a=1 \quad f(1)=1 \quad \forall r\in\Q \ \ f(r)=1^r=1$
        
        $f$ --- непр. и $f(x)=1$ при $x\in\Q\Rightarrow f\equiv1$

        $a>1$. Тогда $\forall x>0 \quad f(x)>1$
        $$r\in\Q, r>0 \quad f(r)=r(r\cdot 1)=(f(1))^r=a^r>1$$

        Значит $\forall x\in\R, x>0$ берем $r_k\to x (r_k\in\Q)$

        $f(r_k)\to f(x)$, значит $f(x)\geq 1$

        $$f(x)=f((x-r)+r)=f(x-r)\cdot f(r)>1$$

        $\exists r\in\Q : 0<r<x$

        возр. $x\in\R, h>0$

        $f(x+h)=f(x)\cdot f(h)$

        $f(h)>1 \Rightarrow f(x+h)>f(x)$

        $a<1$ --- аналогично.

        \item $f(\R)=(\inf f, \sup f)$
        
        $\inf f = 0 \quad \sup f = +\infty$

        $f(1)=a>1$
        
        $a^n, n\in\Z$

        \item $\tilde f(1)=f(1) \Rightarrow \forall r \quad \tilde f(r)=f(r)$
        
        $\forall x \quad r_k\to x$

        $\tilde f(r_k) =f(r_k)$

        $\tilde f(r_k)\to \tilde f(x); f(r_k)\to f(x) \Rightarrow f(x) = \tilde f(x)$
    \end{enumerate}
\end{proof}
%</свойствапоказательнойфункцииproof>
Обозначение. $f$ --- показ ф-ция, $f(1)=a$

Это значит $\forall r\in\Q \quad f(r)=a^r$

Обозначим: $f(x)=a^x$

\begin{theorem}
    $\exists$ показ. ф-ция $f_0$, удовл.:

    $$\frac{f_0 (x)-1}{x}\xrightarrow[x\to0]{} 1$$
\end{theorem}
Доказательство будет позже.
\begin{theorem}
    $f$ --- показ. ф-ция.
    
    Тогда $\exists \alpha\in\R \ \ \forall x \ \ f(x)=f_0(\alpha x)$
\end{theorem}
\begin{proof}
    $f(1)=a$

    Множество значений $f_0$ это $(0, +\infty)$

    $\exists \alpha : f_0(\alpha)=a$

    $f_0(\alpha x)$ и есть $f(x)$. Покажем это:

    $g(x):=f_0(\alpha x)$

    $g(x)$ --- показ. ф., т.к. она не тривиальна и удовлетворяет $(*)$, покажем это:

    $$g(x+y)=f_0(\alpha(x+y))=f_0(\alpha x + \alpha y)=f_0(\alpha x) \cdot f_0(\alpha y) = g(x)g(y)$$

    $$g(1)=f_0(\alpha)=a=f(1)$$
\end{proof}
\begin{consequence}
    Функция $f_0$, удовл. теореме 5, --- единственная.
\end{consequence}
\begin{proof}
    $h(x)$ --- ещё одна такая функция $\Rightarrow h(x)=f_0(\alpha x)$

    $$1\xleftarrow[xx\to0]{} \frac{h(x)-1}{x}=\frac{f_0(\alpha x)-1}{\alpha x}\cdot \alpha\xrightarrow[x\to0]{}\alpha$$
    , т.е. $\alpha=1$
\end{proof}
\begin{definition}
    $f_0$ называется \textbf{экспонента}, если:

    $f_0(x)=e^x\quad f_0(1)=e$

    $$\frac{e^x-1}{x}\xrightarrow[x\to0]{} 1$$
    , т.е. $e^x>1$ при $x>0$
\end{definition}
\begin{consequence}
    $\forall a>0, a\not=1$

    $\exists! f:f(1)=a$
\end{consequence}
\begin{proof}
    Для этого $a \ \ \exists!\alpha \ \ f_0(\alpha)=a$

    $f(x)=f_0(\alpha x)$

    $f(1)=f_0(\alpha)=a$
\end{proof}
\begin{consequence}
    $\forall x, y\in\R \quad \forall a>0, a\not=1$

    $$a^{xy}=(a^x)^y=(a^y)^x$$
\end{consequence}
\begin{proof}
    $x=0$ --- тривиально

    $x\not=0 \quad a^x=b\not=1$

    $y\in\Q \quad a^{xy}=(a^x)^y=b^y$

    $y\in\R \quad r_k\to y \quad a^{xy}\leftarrow a^{xr_k}=b^{r_k} \rightarrow b^y\Rightarrow a^{xy}=b^y$
\end{proof}
\section{Производные высших порядков}
\begin{definition}
    $f:\langle a,b \rangle\to\R$ --- дифф.

    $x\in\langle a,b \rangle$

    Если $f'$ --- дифф. в $x_0$, то $(f')'(x_0)$ --- называется \textbf{вторая производная} функции $f$.

    %<*производнаяnгопорядка>
    Пусть $n-1\in\N$ --- множество $D_{n-1}$ и $f^{(n-1)}:D_{n-1}\to\R$ определены. Пусть $D_n$ --- множество точек $x_0\in D_{n-1}$, для которых существует $\delta>0$, такое что:
    $$(x_0-\delta,x_0+\delta)\cap D_{n-1}=(x_0-\delta,x_0+\delta)\cap D$$
    и $f^{(n-1)}$ дифференцируема в точке $x_0$. Если $x_0\in D_n$, то $f$ --- дифференцируема $n$ раз в точке $x_0$. Функция
    $$f^{(n)}=(f^{(n-1)})'_{D_n}:D_n\to\R$$
    называется производной порядка $n$.
    %</производнаяnгопорядка>

    Если $\forall x\in\langle a,b\rangle\ \ \exists f^{(n)}(x)$, изучим дифференцируемость $f^{(n)}$ в точке $x_0\in\langle a,b\rangle$

    $$f^{(n+1)}(x_0)=(f^{(n)})'(x_0)$$
    $$f^{(n)}_+(x_0) = (f|_{\langle a,b\rangle\cap [x_0, +\infty)})^{(n)}(x_0)$$
\end{definition}
\begin{obozn}
    $E$ --- \textbf{пр-к} в $\R, n\in\N$

    $C^n(E)=\{f:E\to\R : f^{(n)} \text{ непр. на } E\}$

    $C(E) =$ функции, непр. на $E$

    $C^\infty(E)$:

    $C(E)\underset{\not=}{\supset}C^1(E)\underset{\not=}{\supset}C^2(E)\underset{\not=}{\supset}\ldots$
\end{obozn}
\begin{observation}
    $P(x)$ --- многочлен степени $n$

    Пусть $\begin{smallmatrix}
        P(a)=C_0 \\
        P'(a)=C_1 \\
        \vdots \\
        P^{(n)}(a)=C_n
    \end{smallmatrix}$

    $$P(x)=\alpha_0+\alpha_1(x-a)+\alpha_2(x-a)^2+\ldots+\alpha_n(x-a)^N$$
    $P(a)=\alpha_0=C_0$
    $$P'(x)=\alpha_1+2\alpha_2(x-a)+\ldots+n\alpha_n(x-a)^{n-1}$$
    $P'(a)=\alpha_1=C_1$
    $$P''(x)=2\alpha_2+6\alpha_3(x-a)+\ldots+n(n-1)\alpha_n(x-a)^{n-2}$$
    $P''(a)=2\alpha_2$
    
    $\vdots$
    $$P^{(n)}(a)=n(n-1)(n-2)\ldots 2\cdot 1\cdot \alpha_n$$

    $$P(x)=C_0+\frac{C_1}{1!}(x-a)+\frac{C_2}{2!}(x-a)^2+\ldots+\frac{C_n}{n!}(x-a)^n$$
    $$P(x)=P(a)+\frac{P'(a)}{1!}(x-a)+\ldots+\frac{P^{(n)}(a)}{n!}(x-a)^n$$
\end{observation}
\begin{definition}
    %<*многочлентейлора>
    \textbf{Многочленом Тейлора} $n$-той степени \textit{(порядка)} функции $f$ в точке $a$ называется:

    $$T_n(f, a)(x)=f(a)+\frac{f'(a)}{1!}(x-a)+\frac{f''(a)}{2!}(x-a)^2+\ldots+\frac{f^{(n)}(a)}{n!}(x-a)^n$$
    %</многочлентейлора>
\end{definition}
\end{document}