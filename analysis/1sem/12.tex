\documentclass[12pt, a4paper]{article}

\usepackage{lastpage}
\usepackage{mathtools}
\usepackage{xltxtra}
\usepackage{libertine}
\usepackage{amsmath}
\usepackage{amsthm}
\usepackage{amsfonts}
\usepackage{amssymb}
\usepackage{enumitem}
\usepackage{xcolor}
\usepackage[left=1.5cm, right=1.5cm, top=2cm, bottom=2cm, bindingoffset=0cm, headheight=15pt]{geometry}
\usepackage{fancyhdr}
\usepackage[russian]{babel}
% \usepackage[utf8]{inputenc}
\usepackage{catchfilebetweentags}
\usepackage{accents}
\usepackage{calc}
\usepackage{etoolbox}
\usepackage{mathrsfs}
\usepackage{wrapfig}

\providetoggle{useproofs}
\settoggle{useproofs}{false}

\pagestyle{fancy}
\lfoot{M3137y2019}
\rhead{\thepage\ из \pageref{LastPage}}

\newcommand{\R}{\mathbb{R}}
\newcommand{\Q}{\mathbb{Q}}
\newcommand{\C}{\mathbb{C}}
\newcommand{\Z}{\mathbb{Z}}
\newcommand{\B}{\mathbb{B}}
\newcommand{\N}{\mathbb{N}}

\newcommand{\const}{\text{const}}

\newcommand{\teormin}{\textcolor{red}{!}\ }

\DeclareMathOperator*{\xor}{\oplus}
\DeclareMathOperator*{\equ}{\sim}
\DeclareMathOperator{\Ln}{\text{Ln}}
\DeclareMathOperator{\sign}{\text{sign}}
\DeclareMathOperator{\Sym}{\text{Sym}}
\DeclareMathOperator{\Asym}{\text{Asym}}
% \DeclareMathOperator{\sh}{\text{sh}}
% \DeclareMathOperator{\tg}{\text{tg}}
% \DeclareMathOperator{\arctg}{\text{arctg}}
% \DeclareMathOperator{\ch}{\text{ch}}

\DeclarePairedDelimiter{\ceil}{\lceil}{\rceil}
\DeclarePairedDelimiter{\abs}{\left\lvert}{\right\rvert}

\setmainfont{Linux Libertine}

\theoremstyle{plain}
\newtheorem{axiom}{Аксиома}
\newtheorem{lemma}{Лемма}

\theoremstyle{remark}
\newtheorem*{remark}{Примечание}
\newtheorem*{exercise}{Упражнение}
\newtheorem*{consequence}{Следствие}
\newtheorem*{example}{Пример}
\newtheorem*{observation}{Наблюдение}

\theoremstyle{definition}
\newtheorem{theorem}{Теорема}
\newtheorem*{definition}{Определение}
\newtheorem*{obozn}{Обозначение}

\setlength{\parindent}{0pt}

\newcommand{\dbltilde}[1]{\accentset{\approx}{#1}}
\newcommand{\intt}{\int\!}

% magical thing that fixes paragraphs
\makeatletter
\patchcmd{\CatchFBT@Fin@l}{\endlinechar\m@ne}{}
  {}{\typeout{Unsuccessful patch!}}
\makeatother

\newcommand{\get}[2]{
    \ExecuteMetaData[#1]{#2}
}

\newcommand{\getproof}[2]{
    \iftoggle{useproofs}{\ExecuteMetaData[#1]{#2proof}}{}
}

\newcommand{\getwithproof}[2]{
    \get{#1}{#2}
    \getproof{#1}{#2}
}

\newcommand{\import}[3]{
    \subsection{#1}
    \getwithproof{#2}{#3}
}

\newcommand{\given}[1]{
    Дано выше. (\ref{#1}, стр. \pageref{#1})
}

\renewcommand{\ker}{\text{Ker }}
\newcommand{\im}{\text{Im }}
\newcommand{\grad}{\text{grad}}

\lhead{Конспект по матанализу}
\cfoot{}
\rfoot{December 9, 2019}

\begin{document}
\begin{proof}
    Доказательство несчётности отрезка с помощью компактности.

    Рассмотрим произвольную точку отрезка $x_k$ и её окрестность размером $\frac{1}{10^k}$. Все такие окрестности образуют открытое покрытие отрезка, но их суммарная длина $\leq\frac{1}{9}$, что меньше длины произвольного отрезка. Почему-то это даёт противоречие.
\end{proof}

\begin{consequence}
    \textit{(из теоремы о непрерывности монотонной функции)}
    %<*омножестветочекразрыва>
    У монотонной функции, заданной на промежутке, имеется не более чем счётное \textit{(НБЧС)} множество точек разрыва.
    %</омножестветочекразрыва>
\end{consequence}
\begin{example}
    $\Theta(x) = sign(x) \frac{1}{2} + \frac{1}{2}$

    $] x_k$ это $k$-тое рациональное число
    $$f(x)=\sum\limits_{k=1}^{+\infty} \frac{1}{2^k}\Theta(x-x_k)$$ 
    $f(x)$ имеет скачок в каждом рациональном значении аргумента.
\end{example}
%<*омножестветочекразрываproof>
\begin{proof}
    $f(x-0)<f(x+0)$

    Создадим инъекцию $(f(x-0), f(x+0))\leadsto q_x\in\Q$. Возьмём $q_x\in (f(x-0), f(x+0))$. Такая точка будет в силу плотности $\Q$ в $\R$. Теперь докажем, что взятие $q_x$ --- инъекция. Рассмотрим другую точку разрыва --- $y$.

    $$\exists t\in (f(x), f(y))$$

    $$f(x)\leq t\leq f(y)$$

    $$f(x)\leq f(x+0)\leq t\leq f(y-0)\leq f(y)$$

    Таким образом, $(f(x-0), f(x+0))$ и $(f(y-0), f(y+0))$ не имеют общих точек, тогда $q_x$ все разные $\Rightarrow$ взятие $q_x$ --- инъекция. Доказать инъективность достаточно, т.к. нам не нужна равномощность.
\end{proof}
%</омножестветочекразрываproof>
\begin{exercise}
    \begin{enumerate}
        \item Существует ли на плоскости более, чем счётное множество непересекающихся окружностей?
        \item Существует ли на плоскости более, чем счётное множество восьмёрок?
        \item Можно ли в счётном множестве задать такое континуальное семейство $(A_\alpha)$, что:
        $\forall \alpha\in\R \ \ A_\alpha\subset A$

        $\alpha_1\not=\alpha_2 \Rightarrow A_{\alpha_1}\not=A_{\alpha_2}$

        $\forall \alpha,\beta: \alpha<\beta \ \ A_\alpha\subset A_\beta$
    \end{enumerate}
\end{exercise}
\begin{theorem}
    О существовании и непрерывности обратной функции.

    %<*осуществованииинепрерывностиобратнойфункции>
    $f:\langle a, b\rangle\to\R$ --- непр., строго монот.

    $m:=\inf\limits_{\langle a, b\rangle} f(x), M:=\sup\limits_{\langle a, b\rangle} f(x)$. Тогда:

    \begin{enumerate}
        \item $f$ --- обратимая и $f^{-1}:\langle m, M\rangle\to\langle a,b\rangle$
        \item $f^{-1}$ строго монотонна и того же типа \textit{(возрастает или убывает)}
        \item $f^{-1}$ непрерывна
    \end{enumerate}
    %</осуществованииинепрерывностиобратнойфункции>
\end{theorem}
\begin{remark}
    Тип промежутка в $f$ и $f^{-1}$ совпадают.
\end{remark}
%<*осуществованииинепрерывностиобратнойфункцииproof>
\begin{proof}
    Пусть $f\uparrow$

    $f(\langle a,b\rangle)$ --- промежуток $\langle m, M\rangle$ \textit{(типы скобок совпадают)}

    $f$ --- строго монот. $\Rightarrow f$ --- инъекция. Тогда $f:\langle a,b\rangle\to\langle m,M\rangle$ --- биекция

    $\forall x_1<x_2 \ \ f(x_1)<f(x_2)$

    $\forall y_1<y_2 \ \ f^{-1}(y_1)<f^{-1}(y_2)$
\end{proof}
%</осуществованииинепрерывностиобратнойфункцииproof>

\section{Элементарные функции}

\begin{definition}
    Всё, для чего есть кнопочки на калькуляторе --- \textbf{элементарные функции}:

    $const, x^a, a^x, \log_a x, \sin x, \cos x, \arcsin, \arctan x$

    + конечное число арифметических действий и композиций
\end{definition}

\subsection{$x^a$}
Свойства:
\begin{enumerate}
    \item $x^{r+s}=x^rx^s$
    \item $(x^r)^s=x^{rs}$
    \item $(xy)^s=x^sy^s$
\end{enumerate}

$$f_a(x) = x^a, a\in\Q$$

Докажем непрерывность:
\begin{enumerate}
    \item $a=1 \quad f_1(x)=x$ --- непр.
    \item $a\in\N \quad f_a(x)=f_1(x)\cdot f_1(x)\ldots f_1(x)$ --- непр.
    \item $a\in$``$-\N$'' $\quad f_{-a}(x)\stackrel{def}{=}\frac{1}{f_a(x)}$
    \item $a=0 \quad f_0(x)\equiv 1$ (при $x\not=0$, доопределим $f_0(0)=1$) --- непр. в $\R$
    \item $a=\frac{1}{n}, n\in\N, n$ --- нечётно
    
    $f_n\uparrow$ строго $\quad \inf\limits_{x\in\R} f_n(x)=-\infty \quad \sup f_n=+\infty, f_n$ --- непр. $\Rightarrow$ по теореме о непрерывности монотонной функции $f^{-1}$ --- непр.

    $\exists f^{-1}_n:(-\infty, +\infty)\to\R$

    $$f_{\frac{1}{n}}(x)=f_n^{-1}(x)$$

    \item $a=\frac{1}{n}, n\in\N, n$ --- чётн.
    
    $f_n:[0, +\infty)\to[0,+\infty)$ --- строго монот., непр.

    $f(0)=0 \quad \sup f_n=+\infty$

    $\exists f^{-1} : [0, +\infty)\to[0, +\infty)$

    $$f_{\frac{1}{n}}(x):=f_n^{-1}(x)$$

    \item $a=\frac{p}{q}$ \textit{(несокр.)}, $p\in\Z, q\in\N$
    
    $$f_a:=f_\frac{1}{q}\circ f_p$$
\end{enumerate}

\section{Производная}

\begin{definition}
    %<*дифференциируемая>
    $f:\langle a,b\rangle\to\R \quad x_0\in\langle a,b\rangle$

    $f$ --- \textbf{дифференцируема}. в точке $x_0$, если $\exists A\in\R$

    $$f(x)=f(x_0)+A\cdot(x-x_0)+o(x-x_0), x\to x_0$$

    При этом $A$ называется \textbf{производной} $f$ в точке $x_0$
    %</дифференциируемая>
\end{definition}
\begin{definition}
    $f$ --- \textbf{дифференцируема} в точке $x$, если $$\exists \lim\limits_{x\to x_0}\frac{f(x)-f(x_0)}{x-x_0}=A\in\R$$
    $A$ --- производная $f$ в точке $x_0$
\end{definition}
\begin{remark}
    Второе определение не обобщимо на пространство произвольной размерности, в отличие от первого.
\end{remark}
\begin{theorem}
    %<*равносильностьопределенийпроизводной>
    Определение 1 $\Leftrightarrow$ определению 2, т.е.
    $$f(x)=f(x_0)+A\cdot(x-x_0)+o(x-x_0), x\to x_0$$
    $$\exists \lim\limits_{x\to x_0}\frac{f(x)-f(x_0)}{x-x_0}=B\in\R$$
    $$A=B$$
    %</равносильностьопределенийпроизводной>
\end{theorem}
%<*равносильностьопределенийпроизводнойproof>
\begin{proof}
    Докажем ``$\Leftarrow$''.
    
    $$f(x)=f(x_0)+A\cdot(x-x_0)+o(x-x_0), x\to x_0$$

    $$A=\frac{f(x)-f(x_0)}{x-x_0} - \frac{o(x-x_0)}{x-x_0}$$

    Докажем ``$\Rightarrow$''.

    $$\frac{f(x)-f(x_0)}{x-x_0}=A+\alpha(x) \quad \alpha(x)\xrightarrow[x\to x_0]{} 0$$
\end{proof}
%</равносильностьопределенийпроизводнойproof>
\begin{remark}
    \begin{enumerate}
        \item $f$ --- дифф. в $x_0 \Rightarrow f$ --- непр. в $x_0$
        \item $f'(x_0)$ --- обозн. для производной
        
        Если $x_0\in(a, b)$ в опр. 1, 2 $\quad x\to x_0+0$

        $f$ --- дифф. справа $\Rightarrow A$ --- правостор. производная $f'_+(x)$

        $x\to x_0-0$ слева $f_-'(x_0)$

        $$\exists f_+'(x_0)=f_-'(x_0)=f \text{ --- дифф. в} x_0$$
        
        \item $A=\pm\infty: f'(x_0)=\pm\infty$, но $f$ не дифф.
    \end{enumerate}
\end{remark}

\begin{example}
    $$f(x)=\begin{cases}
        0 \quad\quad\ \ \ ,\ x=0 \\
        x\sin\frac{1}{x} \ \ ,\ x\not=0
    \end{cases}$$

    $$\not\exists\lim\limits_{x\to0}\frac{f(x)-f(0)}{x-0}$$
\end{example}

$$f(x)=f(x_0)+f'(x_0)(x-x_0)+o(x-x_0)$$

\begin{definition}
    $$y=f(x_0)+f'(x_0)(x-x_0)$$ --- называется \textbf{касательной} к графику $y=f(x)$ в точке $x_0$
\end{definition}
\begin{theorem}
    %<*правиладифференциирования>
    $f, g:\langle a,b\rangle\to\R$, дифф. в $x_0$

    Тогда указанные ниже в левых частях дифференцируемы в $x_0$ и их производные равны.
    \begin{enumerate}
        \item $(f+g)'(x_0)=f'(x_0)+g'(x_0)$
        \item $\forall \alpha\in\R \quad (\alpha f)'(x_0)=\alpha f'(x_0)$
        \item $(fg)'(x_0)=f'(x_0)g(x_0)+f(x_0)g'(x_0)$
        \item Если $g(x_0)\not=0$: $$\left(\frac{f}{g}\right)'(x_0)=\frac{f'(x_0)g(x_0)-f(x_0)g'(x_0)}{g^2(x_0)}$$
    \end{enumerate}
    %</правиладифференциирования>
\end{theorem}
%<*правиладифференциированияproof>
\begin{proof}
    Докажем 4 по определению.

    $$\frac{\cfrac{f}{g}(x_0)+h-\cfrac{f}{g}(x_0)}{h}=\frac{\cfrac{f(x_0+h)-f(x_0)}{h}g(x_0)-f(x_0)\cfrac{g(x_0+h)-g(x_0)}{h}}{g(x_0+h)g(x_0)}\xrightarrow[h\to0]{} \text{ОК}$$
\end{proof}
%</правиладифференциированияproof>
\begin{theorem}
    О производной композиции.

    %<*дифференциированиекомпозиции>
    $f:\langle a,b\rangle\to\langle c,d\rangle \quad x\in\langle a,b\rangle \quad f$ --- дифф. в $x$

    $g:\langle c,d\rangle\to\R \quad g$ --- дифф. $y=f(x)$

    Тогда $g\circ f$ --- дифф. в $x$; $(g(f(x)))'=g'(f(x))\cdot f'(x)$
    %</дифференциированиекомпозиции>
\end{theorem}
%<*дифференциированиекомпозицииproof>
\begin{proof}
    $$f(x+h)=f(x)+f'(x) h + \alpha(h)h, \alpha(h)\xrightarrow[h\to 0]{} 0$$
    $$g(y+k)=g(y)+g'(y) k + \beta(k)k$$
    $$]f'(x)h + \alpha(h)h = k; \quad k\xrightarrow[h\to0]{}0$$
    $$g(f(x+h))=g(f(x) + f'(x)h + \alpha(h)h)=$$
    $$=g(f(x))+g'(f(x))(f'(x)h+\alpha(h)h)+\beta(k)(f'(x)h+\alpha(h)h)=$$
    $$=g(f(x))+g'(f(x))f'(x)h+g'(f(x))\alpha(h)h+\beta(k)f'(x)h+\beta(k)\alpha(h)h$$
    $$]g'(f(x))\alpha(h)h+\beta(k)f'(x)h+\beta(k)\alpha(h)h=\gamma(h)\cdot h; \quad \gamma(h)\xrightarrow[h\to0]{}0$$
\end{proof}
%</дифференциированиекомпозицииproof>
\begin{theorem}
    О производной обратной функции.

    %<*дифференциированиеобратной>
    $f:\langle a,b\rangle\to\R$ --- непр., строго монот. $x\in\langle a,b\rangle\ \ f$ --- дифф. в $x$; $f'(x)\not=0$

    По определению $f\ \ $ $\exists f^{-1}$

    Тогда $f^{-1}$ --- дифф. в $y=f(x)$ и $$(f^{-1})'(y)=\frac{1}{f'(x)}$$
    %</дифференциированиеобратной>
\end{theorem}
\begin{remark}
    $f^{-1}$ --- дифф. $\Rightarrow$ ф-ция очев.: $(f^{-1})'(f(x))\cdot f'(x)=(f^{-1}(f(x)))'=(x)'=1$
\end{remark}
%<*дифференциированиеобратнойproof>
\begin{proof}
    $\forall k \ \ \exists h: f(x+h)=y+k$

    $$h=(x+h)-x=f^{-1}(y+k)-f^{-1}(y)=\tau(k)$$

    $$\frac{f^{-1}(y+k)-f^{-1}(y)}{k}=\frac{\tau(k)}{f(x+\tau(k))-f(x)}=\cfrac{1}{\cfrac{f(x+\tau(k))-f(x)}{(x+\tau(k)) - x}}\xrightarrow[\stackrel{k\to0}{\tau(k)\to0}]{\text{по т.о непр. обр. ф}}\frac{1}{f'(x)}$$
\end{proof}
%</дифференциированиеобратнойproof>
\begin{example}
$$y=\sin x$$
$$\arcsin'y=\frac{1}{\cos x}=\frac{1}{\sqrt{1-\sin^2x}}=\frac{1}{\sqrt{1-y^2}}$$
\end{example}
\begin{exercise}
    $\arctan'y=0$
\end{exercise}

\section{Теоремы о среднем}

\begin{lemma}
    %<*леммадлятеоремыферма>
    $f:\langle a,b\rangle\to\R$ --- дифф. в $x_0\in(a,b)$; $f'(x_0)>0$

    Тогда $\exists \varepsilon > 0 \ \ \forall x : x\in(x_0, x_0+\varepsilon) \ \ f(x_0) < f(x)$

    и $\forall x : x\in(x_0 - \varepsilon, x_0) \ \ f(x_0) > f(x)$
    %</леммадлятеоремыферма>
\end{lemma}
%<*леммадлятеоремыферма2>
\begin{remark}
    Это не монотонность.
\end{remark}
%</леммадлятеоремыферма2>
%<*леммадлятеоремыфермаproof>
\begin{proof}
    $$\frac{f(x)-f(x_0)}{x-x_0}\xrightarrow[x\to x_0]{} f'(x_0)>0$$

    $$x\to x_0+0 \quad x-x_0>0 \Rightarrow f(x)-f(x_0)>0 \text{ вблизи } x_0 \textit{ (по теор. о стабилизации знака)}$$

    $$x\to x_0-0 \quad x-x_0<0 \Rightarrow f(x)-f(x_0)<0 \text{ вблизи } x_0$$
\end{proof}
%</леммадлятеоремыфермаproof>
\begin{theorem}
    Ферма.

    %<*теоремаферма>
    $f:\langle a,b\rangle \to\R$

    $x_0\in(a,b)$ --- точка максимума

    $f$ --- дифференцируема в $x_0$

    Тогда $f'(x_0)=0$
    %</теоремаферма>
\end{theorem}
%<*теоремафермаproof>
\begin{proof}
    Из леммы.

    Если $f'(x_0)>0$, то справа от $x_0$ есть $x: f(x)>f(x_0)$

    Если $f'(x_0)<0$, то слева от $x_0$ есть $x: f(x)>f(x_0)$
\end{proof}
%</теоремафермаproof>
\begin{theorem}
    Ролля.

    %<*теоремаролля>
    $f:[a,b]\to\R$ --- непр. на $[a,b]$, дифф. на $(a,b)$

    $f(a)=f(b)$. Тогда $\exists c\in(a,b) : f'(c)=0$
    %</теоремаролля>
\end{theorem}
%<*теоремаролляproof>
\begin{proof}
    По теореме Вейерштрасса.

    $x_0=\max f(x); x_1=\min f(x)$

    $\{x_0, x_1\}=\{a, b\} \Rightarrow f=const; f'\equiv0$

    Иначе: пусть $x_0\in (a,b) \xRightarrow[\text{т. Ферма}]{} f'(x_0)=0$
\end{proof}
%</теоремаролляproof>
\begin{remark}
    $f(x)=(x-a)^kg(x)$, где $g(a)\not=0$
    $$f'(x)=k(x-a)^{k-1}g(x)+(x-a)^kg'(x)=(x-a)^{k-1}(k\cdot g(x)+(x-a)\cdot g'(x))$$
\end{remark}
\begin{example}
    %<*полиномылежандра>
    $n\in\N$

    $\Ln(x)=((x^2-1)^n)^{(n)}$ --- \textbf{полиномы Лежандра} \textit{(с точностью до умножения на константу)}

    $\deg \Ln=n$

    Утверждение: $\Ln$ имеет $n$ различных вещественных корней.
    %</полиномылежандра>
\end{example}
%<*полиномылежандраproof>
\begin{proof}
    Рассмотрим $(x^2-1)^n$. У этого многочлена 2 корня $\{-1, 1\}$, каждый кратности $n$.

    Возьмём производную. По т. Ролля у этого многочлена есть корень $\in(-1, 1)$. Кроме того, $\{-1, 1\}$ все ещё корни, у них кратность $n-1$. Т.к. $\deg=2n-1$, кратность нового корня 1. На $n$-ном шаге получается $n$ корней, каждый кратности 1.
\end{proof}
%</полиномылежандраproof>
\end{document}