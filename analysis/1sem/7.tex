\documentclass[12pt, a4paper]{article}

\usepackage{lastpage}
\usepackage{mathtools}
\usepackage{xltxtra}
\usepackage{libertine}
\usepackage{amsmath}
\usepackage{amsthm}
\usepackage{amsfonts}
\usepackage{amssymb}
\usepackage{enumitem}
\usepackage{xcolor}
\usepackage[left=1.5cm, right=1.5cm, top=2cm, bottom=2cm, bindingoffset=0cm, headheight=15pt]{geometry}
\usepackage{fancyhdr}
\usepackage[russian]{babel}
% \usepackage[utf8]{inputenc}
\usepackage{catchfilebetweentags}
\usepackage{accents}
\usepackage{calc}
\usepackage{etoolbox}
\usepackage{mathrsfs}
\usepackage{wrapfig}

\providetoggle{useproofs}
\settoggle{useproofs}{false}

\pagestyle{fancy}
\lfoot{M3137y2019}
\rhead{\thepage\ из \pageref{LastPage}}

\newcommand{\R}{\mathbb{R}}
\newcommand{\Q}{\mathbb{Q}}
\newcommand{\C}{\mathbb{C}}
\newcommand{\Z}{\mathbb{Z}}
\newcommand{\B}{\mathbb{B}}
\newcommand{\N}{\mathbb{N}}

\newcommand{\const}{\text{const}}

\newcommand{\teormin}{\textcolor{red}{!}\ }

\DeclareMathOperator*{\xor}{\oplus}
\DeclareMathOperator*{\equ}{\sim}
\DeclareMathOperator{\Ln}{\text{Ln}}
\DeclareMathOperator{\sign}{\text{sign}}
\DeclareMathOperator{\Sym}{\text{Sym}}
\DeclareMathOperator{\Asym}{\text{Asym}}
% \DeclareMathOperator{\sh}{\text{sh}}
% \DeclareMathOperator{\tg}{\text{tg}}
% \DeclareMathOperator{\arctg}{\text{arctg}}
% \DeclareMathOperator{\ch}{\text{ch}}

\DeclarePairedDelimiter{\ceil}{\lceil}{\rceil}
\DeclarePairedDelimiter{\abs}{\left\lvert}{\right\rvert}

\setmainfont{Linux Libertine}

\theoremstyle{plain}
\newtheorem{axiom}{Аксиома}
\newtheorem{lemma}{Лемма}

\theoremstyle{remark}
\newtheorem*{remark}{Примечание}
\newtheorem*{exercise}{Упражнение}
\newtheorem*{consequence}{Следствие}
\newtheorem*{example}{Пример}
\newtheorem*{observation}{Наблюдение}

\theoremstyle{definition}
\newtheorem{theorem}{Теорема}
\newtheorem*{definition}{Определение}
\newtheorem*{obozn}{Обозначение}

\setlength{\parindent}{0pt}

\newcommand{\dbltilde}[1]{\accentset{\approx}{#1}}
\newcommand{\intt}{\int\!}

% magical thing that fixes paragraphs
\makeatletter
\patchcmd{\CatchFBT@Fin@l}{\endlinechar\m@ne}{}
  {}{\typeout{Unsuccessful patch!}}
\makeatother

\newcommand{\get}[2]{
    \ExecuteMetaData[#1]{#2}
}

\newcommand{\getproof}[2]{
    \iftoggle{useproofs}{\ExecuteMetaData[#1]{#2proof}}{}
}

\newcommand{\getwithproof}[2]{
    \get{#1}{#2}
    \getproof{#1}{#2}
}

\newcommand{\import}[3]{
    \subsection{#1}
    \getwithproof{#2}{#3}
}

\newcommand{\given}[1]{
    Дано выше. (\ref{#1}, стр. \pageref{#1})
}

\renewcommand{\ker}{\text{Ker }}
\newcommand{\im}{\text{Im }}
\newcommand{\grad}{\text{grad}}

% \fancyhf{}
\lhead{Конспект по матанализу}
\cfoot{}
\rfoot{November 4, 2019}

\begin{document}

\section{Пределы}

\begin{theorem}
    %<*эквивалентностьопределенийгейнеикоши>
    Определение Коши $\Leftrightarrow$ определение Гейне.
    %</эквивалентностьопределенийгейнеикоши>
\end{theorem}

%<*эквивалентностьопределенийгейнеикошиproof>
\begin{proof}
    Докажем ``$\Rightarrow$''.

    Если дана $(x_n)$, удовл. определению Коши, доказать $$\forall \varepsilon>0 \ \ \exists N \ \ \forall n > N \ \ 0<\rho(f(x_n), A)<\varepsilon$$

    $$\forall \varepsilon>0 \ \ \exists \delta>0 \ \ \forall x \in D \ \ 0<\rho(x, a)<\delta \ \ \rho(f(x), A)<\varepsilon$$

    $$\text{Для этого } \delta \ \ \exists N \ \ \forall n>N \rho(x_n,a)<\delta$$, где $x_n\in D, x_n\not = a$

    $\Rightarrow \rho(f(x_n), A)<\varepsilon$
\end{proof}
\begin{proof}
    Докажем ``$\Leftarrow$''

    Пусть определение Коши не выполняется.

    $$\exists \varepsilon>0 \ \ \forall \delta>0 \ \ \exists x\in D \ \ 0<\rho(x,a)<\delta \ \ \rho(f(x), A)\geq \varepsilon$$

    $$\delta:=\frac{1}{n} \ \ \exists x_n\in D \ \ 0<\rho(x_n,a)<\frac{1}{n} \ \ \rho(f(x_n), A)\geq\varepsilon$$

    Построена последовательность $(x_n): x_n\in D \ \ x_n\not = a \ \ \rho(x_n, a)<\frac{1}{n} \Rightarrow \rho(x_n, a)\to 0 \Rightarrow x_n\to a$. Кроме того, $\rho(f(x_n), A)\geq \varepsilon$ --- противоречит утверждению Гейне, что $f(x_n)\to A$.
\end{proof}
%</эквивалентностьопределенийгейнеикошиproof>

\begin{theorem}
    О единственности предела.
    %<*оединственностипредела>
    $f: D\subset X\to Y$, $a$ --- пред. точка $D$

    $\lim\limits_{x\to a} f(x)=A; \lim\limits_{x\to a} f(x)=B$

    Тогда $A=B$
    %</оединственностипредела>
\end{theorem}

%<*оединственностипределаproof>
\begin{proof}
    По Гейне.
    $\forall (x_n):$
    \begin{itemize}
        \itemsep0em
        \item $x_n\to a$
        \item $x_n\in D$
        \item $x_n\not=a$
    \end{itemize}

    $f(x_n)\to A, f(x_n)\to B \xRightarrow[\text{теор. о ед. предела посл.}]{} A=B$
\end{proof}
%</оединственностипределаproof>

\begin{theorem}
    %<*олокальнойограниченностиотображенияимеющегопредел>
    О локальной ограниченности отображения, имеющего предел.

    $f: D\subset X\to Y$, $a$ --- пред. точка $D$, $\exists\lim\limits_{x\to a} f(x)=A$

    Тогда $\exists V(a) : f$ --- огр. на $V(a)\cap D$, т.е. $f(V(a)\cap D)$ содержится в некотором шаре.
    %</олокальнойограниченностиотображенияимеющегопредел>
\end{theorem}
%<*олокальнойограниченностиотображенияимеющегопределproof>
\begin{proof}
    Для $\varepsilon = 1 \ \ \exists V(a) \ \ \forall x\in \dot V(a)\cap D \ \ f(x)\in U_\varepsilon(A)$

    Если $\not\exists f(a)$, ограниченность доказана. Иначе:

    $\forall x\in V(a)\cap D \ \ f(x)\in U_{\tilde{\varepsilon}} (A)$, где $\tilde \varepsilon = \max(\varepsilon, \rho(A, f(a)) + 1)$
\end{proof}
%</олокальнойограниченностиотображенияимеющегопределproof>

\begin{theorem}
    %<*остабилизациизнака>
    О стабилизации знака.

    $f: D\subset X\to Y$, $a$ --- пред. точка $D$, $\exists\lim\limits_{x\to a} f(x)=A$

    Пусть $B\in Y, B\not=A$

    Тогда $\exists V(a) \ \ \forall x \in \dot V(a)\cap D \ \ f(x)\not = B$
    %</остабилизациизнака>
\end{theorem}
%<*остабилизациизнакаproof>
\begin{proof}
    Для $$0 < \varepsilon < \rho(A, B) \ \ \exists V(a) \ \ \forall x\in \dot V(a)\cap D \ \ f(x)\in U_\varepsilon(A)$$ $U_\varepsilon(A)$ не содержит $B$.
\end{proof}
%</остабилизациизнакаproof>
\begin{corollary}
    $f:D\subset X \to \mathbb{R}$, $a$ -- пред. точка, $\lim\limits_{x\to a} f(x)=A>0 \ \ B=0$

    $\exists \dot V(a)\cap D : f(x)\not=0$

    В доказательстве $0<\varepsilon<A \ \ f(x)\in U_\varepsilon (A)=(A-\varepsilon, A+\varepsilon)$
\end{corollary}

\begin{theorem}
    Об арифметических свойствах предела

    %<*обарифметическихсвойствахпредела>
    $f,g: D\subset X\to Y$, $X$ --- метрич. пространство, $Y$ --- норм. пространство над $\mathbb{R}$, $a$ --- пред. точка $D$

    $\lim\limits_{x\to a}f(x)=A, \lim\limits_{x\to a}g(x)=B$

    $\lambda: D\to \mathbb{R}, \lim\limits_{x\to a}\lambda (x) = \lambda_0$

    Тогда:
    \begin{enumerate}
        \item $\exists\lim\limits_{x\to a} f(x)\pm g(x)$ и $\lim\limits_{x\to a} f(x)\pm g(x)=A\pm B$
        \item $\lim\limits_{x\to a} \lambda(x) f(x) = \lambda_0 A$
        \item $\lim\limits_{x\to a} ||f(x)||=||A||$
        \item Для случая $Y=\R$ и для $B\not=0$ $$\lim\limits_{x\to a}\frac{f(x)}{g(x)}=\frac{A}{B}$$
              $\frac{f}{g}$ задано на множестве $D'=D\setminus \{x:g(x)=0\}$

              $a$ --- пр. точка $D'$ по теореме о стабилизации знака $\exists V(a) \ \ \forall x\in V(a)\cap D \ \ g(x)$ --- того же знака, что и $B$, т.е. $g(x)\not = 0$

              $\dot V(a)\cap D'=\dot V(a)\cap D \Rightarrow a$ --- пред. точка для $D'$
    \end{enumerate}
    %</обарифметическихсвойствахпредела>
\end{theorem}
%<*обарифметическихсвойствахпределаproof>
\begin{proof}
    По Гейне.
    $\forall (x_n):$
    \begin{itemize}
        \itemsep0em
        \item $x_n\to a$
        \item $x_n\in D$
        \item $x_n\not=a$
    \end{itemize}

    $f(x_n) + g(x_n)\to^? A+B$ верно по теореме последовательности.

    Аналогично прочие пункты, кроме 4.

    $f(x_n)\to A$

    $g(x_n)\to B\not=0$
    $\Rightarrow \exists n_0 \ \ \forall n>n_0 \ \ g(x_n)\not=0$

    $\frac{f(x_n)}{g(x_n)}$ корректно задано при $n>n_0$.
\end{proof}
%</обарифметическихсвойствахпределаproof>
\begin{remark}
    Для $\overline \R$

    %<*арифметическиесвойствапределадляoverliner>
    Если $Y=\overline\R$, можно ``разрешить'' случай $A, B=\pm \infty$
    %</арифметическиесвойствапределадляoverliner>

    %<*арифметическиесвойствапределадляoverlinerproof>
    Тогда 3. тривиально, 1., 2. и 4. верно, если выражения $A\pm B$, $\lambda_0 A$, $\frac{A}{B}$ корректны.

    Докажем 1. как в теореме об арифметических свойствах последовательности.

    $\lim\limits_{x\to a} f(x) = +\infty; \lim\limits_{x\to a} g(x) = +\infty \Leftrightarrow \forall E_1 \ \ \exists \delta_1>0 \ \ \forall x\in D\cap V_{\delta_1}(a) \ \ f(x) > E_1 \ \ \forall E_2 \ \ \exists \delta_2>0 \ \ \forall x\in D\cap V_{\delta_2}(a) \ \ g(x) > E_2$
    %</арифметическиесвойствапределадляoverlinerproof>
\end{remark}
Это доказательство не будет спрашиваться.

\section{Компактные множества}

\begin{theorem}
    О простейших свойствах компактных множеств.
    %<*опростейшихсвойствахкомпактныхмножеств>
    $(X, \rho)$ --- метрическое пространство, $K\subset X$

    \begin{enumerate}
        \item $K$ --- комп. $\Rightarrow K$ --- замкн., $K$ --- огр.
        \item $X$ --- комп, $K$ --- замкн. $\Rightarrow K$ --- комп.
    \end{enumerate}
    %</опростейшихсвойствахкомпактныхмножеств>
\end{theorem}
%<*опростейшихсвойствахкомпактныхмножествproof>
\begin{proof}
    \begin{enumerate}
        \item $?K$ --- замкн. $?K^c$ --- откр.

              $a\not\in K$, проверим, что $\exists U(a)\subset K^c$

              $K\subset \bigcup\limits_{x\in K} B(x, \frac{1}{2}\rho(x,a))$ --- откр. покрытие

              $K$ --- комп. $\Rightarrow \exists x_1\ldots x_n \quad K\subset\bigcup\limits_{i=1}^n B(x_i, \frac{1}{2}\rho(x_i, a))$ --- открытое покрытие

              $r:=\min(\frac{1}{2}\rho(x_1, a))\ldots\frac{1}{2}\rho(x_n, a)))$

              $B(a, r)$ не пересекается ни с одним $B(x_i, \frac{1}{2}\rho(x_i, a)) \Rightarrow B(a,r)\subset K^c$

              $?K$ --- огр.

              $b\in X$

              $K\subset \bigcup\limits_{n=1}^{+\infty} B(b, n) = X$

              $K$ --- комп. $\Rightarrow K\subset \bigcup\limits_{n=1}^{m} \Rightarrow K\subset B(b, \max(n_1\ldots n_m))$

        \item $?K$ --- комп.

              $
                  \begin{cases}
                      K\subset \bigcup\limits_{\alpha\in A} G_\alpha \text{, $G_\alpha$ --- откр.} \\
                      K \text{ --- замкн., } K^c \text{ --- откр.}
                  \end{cases} \Rightarrow K^c\cup\bigcup\limits_{\alpha\in A} G_\alpha \text{ --- откр. покрытие } X \Rightarrow \\ \Rightarrow X\subset (\text{может быть } K^c)\cup \bigcup\limits_{i=1}^{n} G_{\alpha_i}$
    \end{enumerate}
\end{proof}
%</опростейшихсвойствахкомпактныхмножествproof>

\begin{lemma}
    О вложенных параллелепипедах.
    %<*овложенныхпараллелепипедах>
    $[a, b] = \{x\in\R^m: \forall i=1\ldots m \ \ a_i\leq x_i\leq b_i\}$ --- параллелепипед.

    $[a^1, b^1]\supset[a^2, b^2]\supset\ldots$ --- бесконечная последовательность параллелепипедов.

    Тогда $\bigcap\limits_{i=1}^{+\infty}[a^i, b^i]\not=$\O

    Если $diam[a^n, b^n]=||b^n-a^n||\to 0$, тогда $\exists! c\in\bigcap\limits_{i=1}^\infty[a^i, b^i]$
    %</овложенныхпараллелепипедах>
\end{lemma}
%<*овложенныхпараллелепипедахproof>
\begin{proof}
    $\forall i=1\ldots m \quad [a^1_i, b^1_i]\supset[a^2_i, b^2_i]\supset\ldots \quad \exists c_i\in \bigcap\limits_{n=1}^{+\infty}[a_i^n, b_i^n]$. $c=(c_1\ldots c_m)$ --- общая точка всех параллелепипедов.

    $|a_i^n-b_i^n|\leq ||a^n-b^n||\to 0 \Rightarrow_{\text{т. Кантора}}\exists! c_i\in\bigcap\limits_{n=1}^{+\infty}[a_i^n, b_i^n] \Rightarrow \exists! c=(c_1\ldots c_m)$
\end{proof}
%</овложенныхпараллелепипедахproof>
\begin{lemma}
    %<*компактностьпарллелепипеда>
    $[a,b]$ --- компактное множество в $\R^m$
    %</компактностьпарллелепипеда>
    $[a, b]\subset\bigcup\limits_{\alpha\in A} G_\alpha$ --- откр. в $\R^m$
\end{lemma}

%<*компактностьпарллелепипедаproof>
\begin{proof}
    Докажем, что $\exists$ кон. $\alpha=(\alpha_1\ldots\alpha_n): [a,b]\subset \bigcup\limits_{i=1}^n G_{\alpha_i}$

    Допустим, что не $\exists$

    $[a^1, b^1]:=[a,b] \Rightarrow [a^1, b^1]$ нельзя покрыть кон. набором

    $[a^2, b^2]:=$ делим $[a^1, b^1]$ на $2^m$ частей, берем любую ``часть'', которую нельзя покрыть конечным набором $G_\alpha$

    $\vdots$

    $$diam=[a^n, b^n]=\frac{1}{2}diam[a^{n-1}, b^{n-1}]=\frac{1}{2^{n-1}}diam[a^1,b^1]$$

    $$\exists c\in \bigcap\limits_{n=1}^{+\infty}[a^n, b^n]$$

    $$c\in [a, b] \subset \bigcup\limits_{\alpha\in A} G_\alpha$$

    $$\exists \alpha_0 \quad c\in G_{\alpha_0} \text{ --- откр.}$$

    $$\exists U_\varepsilon(c)\subset G_{\alpha_0}$$

    $$\exists n \quad diam[a^n, b^n] \ll \varepsilon$$

    $$\text{и тогда } [a^n, b^n]\subset U_\varepsilon(c)\subset G_{\alpha_0}$$
\end{proof}
%</компактностьпарллелепипедаproof>

\begin{remark}
    $x_n\to a$

    $\forall $ подпосл. $n_k\quad x_{n_k}\to a$
\end{remark}
\begin{remark}
    $\{n_k\}\cap \{m_k\}=\N$

    $\begin{cases}
            x_{n_k}\to a \\
            x_{m_k}\to a
        \end{cases} \Rightarrow x_n\to a$
\end{remark}

\begin{definition}
    %<*секвенциальнокомпактноемножество>
    \textbf{Секвенциально компактным} называется множество $A\subset X :
        \forall \text{ посл. } (x_n) \text{ точек } A \\
        \exists \text{ подпосл. } x_{n_k} \text{, которая сходится к точке из } A
    $
    %</секвенциальнокомпактноемножество>
\end{definition}
\begin{theorem}
    О характеристике компактов в $\R^m$.
    %<*охарактеристикекомпактоввrm>
    $K\subset \R^m$. Эквивалентны следующие утверждения:\begin{enumerate}
        \itemsep-0.5em
        \item $K$ --- замкнуто и ограничено
        \item $K$ --- компактно
        \item $K$ --- секвенциально компактно
    \end{enumerate}
    %</охарактеристикекомпактоввrm>
\end{theorem}

%<*охарактеристикекомпактоввrmproof>
\begin{proof}
    Докажем $1\Rightarrow 2$

    $K$ --- огр. $\Rightarrow K$ содержится в $[a,b]$

    $K$ --- замкн. в $\R^m \Rightarrow K$ --- замкн. в $[a,b]$

    Т.к. $[a,b]$ --- комп., по простейшему свойству компактов $K$ --- комп.
\end{proof}
\begin{proof}
    Докажем $2\Rightarrow 3$

    $\forall (x_n)$ --- точки из $K$.

    ?сходящаяся последовательность

    Если множество значений $D=\{x_n, n\in\N\}$ --- конечно, то $\exists$ сход. подпосл. очевидно.

    Пусть $D$ --- бесконечно

    Если $D$ имеет предельную точку, то $x_{m_k}\to a$

    Если $D$ --- бесконечно и не имеет предельных точек, $K\subset \bigcup\limits_{x\in K}B(x, \varepsilon_x)$, радиус такой, что в этом шаре нет точек $D$, кроме $x$ \textit{(его может тоже не быть)}. Тогда $\bigcup\limits_{x\in K}B(x, \varepsilon_x)$ --- открытое покрытие $K$. Так как каждый шар содержит 0 или 1 точку, конечное число шаров не может покрыть $K$, т.к. в $K$ бесконечное число точек \textit{(т.к. бесконечное число различных значений $D$)}. Таким образом, мы нашли открытое покрытие $K$, у которого нет конечного подпокрытия --- противоречие.

    Пусть $a\in K$ --- предельная точка. Возьмём из $B(a, r_1)$ точку $x_{n_1}$. Возьмём $r_2<r_1$ и из соответствующего шара возьмём $x_{n_2}$. При $r_n\to0$ $x_{n_k}\to a$.

    \textcolor{red}{Почему вблизи $a$ будет точка из произвольной последовательности?}
\end{proof}
%</охарактеристикекомпактоввrmproof>
\end{document}