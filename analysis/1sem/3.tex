\documentclass[12pt, a4paper]{article}

\usepackage{lastpage}
\usepackage{mathtools}
\usepackage{xltxtra}
\usepackage{libertine}
\usepackage{amsmath}
\usepackage{amsthm}
\usepackage{amsfonts}
\usepackage{amssymb}
\usepackage{enumitem}
\usepackage{xcolor}
\usepackage[left=1.5cm, right=1.5cm, top=2cm, bottom=2cm, bindingoffset=0cm, headheight=15pt]{geometry}
\usepackage{fancyhdr}
\usepackage[russian]{babel}
% \usepackage[utf8]{inputenc}
\usepackage{catchfilebetweentags}
\usepackage{accents}
\usepackage{calc}
\usepackage{etoolbox}
\usepackage{mathrsfs}
\usepackage{wrapfig}

\providetoggle{useproofs}
\settoggle{useproofs}{false}

\pagestyle{fancy}
\lfoot{M3137y2019}
\rhead{\thepage\ из \pageref{LastPage}}

\newcommand{\R}{\mathbb{R}}
\newcommand{\Q}{\mathbb{Q}}
\newcommand{\C}{\mathbb{C}}
\newcommand{\Z}{\mathbb{Z}}
\newcommand{\B}{\mathbb{B}}
\newcommand{\N}{\mathbb{N}}

\newcommand{\const}{\text{const}}

\newcommand{\teormin}{\textcolor{red}{!}\ }

\DeclareMathOperator*{\xor}{\oplus}
\DeclareMathOperator*{\equ}{\sim}
\DeclareMathOperator{\Ln}{\text{Ln}}
\DeclareMathOperator{\sign}{\text{sign}}
\DeclareMathOperator{\Sym}{\text{Sym}}
\DeclareMathOperator{\Asym}{\text{Asym}}
% \DeclareMathOperator{\sh}{\text{sh}}
% \DeclareMathOperator{\tg}{\text{tg}}
% \DeclareMathOperator{\arctg}{\text{arctg}}
% \DeclareMathOperator{\ch}{\text{ch}}

\DeclarePairedDelimiter{\ceil}{\lceil}{\rceil}
\DeclarePairedDelimiter{\abs}{\left\lvert}{\right\rvert}

\setmainfont{Linux Libertine}

\theoremstyle{plain}
\newtheorem{axiom}{Аксиома}
\newtheorem{lemma}{Лемма}

\theoremstyle{remark}
\newtheorem*{remark}{Примечание}
\newtheorem*{exercise}{Упражнение}
\newtheorem*{consequence}{Следствие}
\newtheorem*{example}{Пример}
\newtheorem*{observation}{Наблюдение}

\theoremstyle{definition}
\newtheorem{theorem}{Теорема}
\newtheorem*{definition}{Определение}
\newtheorem*{obozn}{Обозначение}

\setlength{\parindent}{0pt}

\newcommand{\dbltilde}[1]{\accentset{\approx}{#1}}
\newcommand{\intt}{\int\!}

% magical thing that fixes paragraphs
\makeatletter
\patchcmd{\CatchFBT@Fin@l}{\endlinechar\m@ne}{}
  {}{\typeout{Unsuccessful patch!}}
\makeatother

\newcommand{\get}[2]{
    \ExecuteMetaData[#1]{#2}
}

\newcommand{\getproof}[2]{
    \iftoggle{useproofs}{\ExecuteMetaData[#1]{#2proof}}{}
}

\newcommand{\getwithproof}[2]{
    \get{#1}{#2}
    \getproof{#1}{#2}
}

\newcommand{\import}[3]{
    \subsection{#1}
    \getwithproof{#2}{#3}
}

\newcommand{\given}[1]{
    Дано выше. (\ref{#1}, стр. \pageref{#1})
}

\renewcommand{\ker}{\text{Ker }}
\newcommand{\im}{\text{Im }}
\newcommand{\grad}{\text{grad}}

\usepackage{bm}

\lhead{Конспект по математическому анализу}
\rfoot{September 23, 2019}

\begin{document}

Определение предела дает функцию $N(\varepsilon)$, хорошо приспособленную
для изучения неравенства $\rho(x_n,a)<\varepsilon$ для $n\in(N;+\infty)$.
Кроме того, для последовательности $r_n=\rho(x_n, a) \quad |r_n|<\varepsilon$.

\begin{theorem}
    \textbf{О единственности предела.}
    %<*оединственностипредела>
    $(X,\rho)$ --- метрическое пр-во, $a,b\in X$, $(x_n)$ --- послед. в $X$,
    $x_n\xrightarrow[n\to +\infty]{} a$, $x_n\xrightarrow[n\to +\infty]{} b$,
    тогда $a=b$
    %</оединственностипредела>
\end{theorem}

%<*оединственностипределаproof>
\begin{proof}
    $$\text{Докажем от противного --- пусть }a\not =b. \text{ Возьмем } 0<\varepsilon<\frac{1}{2}\rho(a,b)$$
    $$\exists N(\varepsilon) \ \ \forall n>N(\varepsilon) \ \ \rho(x_n,a)<\varepsilon$$
    $$\exists K(\varepsilon) \ \ \forall n>K(\varepsilon) \ \ \rho(x_n,b)<\varepsilon$$
    $$\text{При } n>\max(N(\varepsilon),K(\varepsilon)) \quad \rho(a,b)\leq
        \rho(a,x_n)+\rho(b,x_n)<2\varepsilon<\rho(a,b) \text{ --- противоречие}$$
\end{proof}
%</оединственностипределаproof>

\begin{definition}
    $A\subset X$ --- {\bf ограничено}, если $\exists x_0\in X \ \ \exists R>0 \ \ A\subset B(x_0, R)$
\end{definition}

Пусть $b\in X$. $A$ --- огр. $\Leftrightarrow \exists r>0 \ \ A\subset B(b,r)$

$A\subset B(x_0, R)\Rightarrow A\subset B(b,\rho(x_0,b)\pm R)$

\begin{theorem}
    %<*ограниченностьсходящейсяпоследовательности>
    Если $(x,\rho)$ --- метрическое пр-во, $(x_n)$ --- послед. в $X$, $x_n$ сходится, \textbf{тогда} $x_n$ - ограничен.
    %</ограниченностьсходящейсяпоследовательности>
\end{theorem}

%<*ограниченностьсходящейсяпоследовательностиproof>
\begin{proof}
    $$\text{Пусть } a=\lim\limits_{n\to +\infty} x_n$$
    $$\forall U(a) \ \ \exists N \ \ \forall n>N \ \ x_n\in U(a)$$
    $$U(a)=B(a,\varepsilon)$$
    $$r:=max(\varepsilon, \rho(x_1, a), \rho(x_2, a) \ldots \rho(x_N, a))+1 $$
    $$\text{тогда } \forall n\in\mathbb{N} \ \ x_n\in B(a,r)$$
\end{proof}
%</ограниченностьсходящейсяпоследовательностиproof>

\section*{Порядковые свойства предела}

\begin{theorem}
    \textbf{О предельном переходе в неравенствах для $\mathbb{R}$.}
    %<*определьномпереходевнеравенствахдляr>
    Если $(x_n),(y_n)$ ---
    вещественные последовательности $x_n\to a, y_n\to b$, $\forall n \ x_n\leq y_n$, \textbf{тогда} $a\leq b$.
    %</определьномпереходевнеравенствахдляr>
\end{theorem}

%<*определьномпереходевнеравенствахдляrproof>
\begin{proof}
    $$\text{Докажем от противного. Пусть } a>b, 0<\varepsilon <\frac{a-b}{2}.$$
    $$\exists N(\varepsilon) \ \ \forall n>N \ \ a-\varepsilon<x_n<a+\varepsilon$$
    $$\exists K(\varepsilon) \ \ \forall n>K \ \ b-\varepsilon<y_n<b+\varepsilon $$
    $$\text{При } n>\max(N,K) \ \ y_n < b + \varepsilon < a-\varepsilon <x_n \text{ --- противоречие}$$
\end{proof}
%</определьномпереходевнеравенствахдляrproof>

\begin{remark}
    Если вместо "$\forall n \ \ x_n\leq y_n$" потребовать: "$\exists M \ \ \forall n>M \ \ x_n\leq y_n$", то утв. по-прежнему верно
\end{remark}

\begin{remark}
    $x_n=-\frac{1}{n} \ \ y_n=\frac{1}{n}$. тогда $x_n\to 0, y_n\to 0$. $x_n<y_n$, но пределы совпадают. То есть даже если $x_n<y_n$ строго, $a\leq b$ --- нестрого.
\end{remark}

\begin{corollary}
    $(x_n)$ --- вещественная последовательность, $a,b\in\mathbb{R}$
    \begin{enumerate}
        \item $\forall n \ \ x_n\leq a \Rightarrow \lim x_n\leq a$
        \item $\forall n \ \ x_n\geq b \Rightarrow \lim x_n\geq b$
        \item $\forall n \ \ x_n\in [a,b] \Rightarrow \lim x_n\in [a,b]$
    \end{enumerate}
\end{corollary}

\begin{theorem}
    \textbf{О двух городовых} (о сжатой последовательности).
    %<*одвухгородовых>
    Если $(x_n),(y_n),(z_n)$ - вещ. посл., $\forall n \ \ x_n\leq y_n\leq z_n, \lim x_n=\lim z_n =a$, \textbf{тогда} $\exists \lim y_n=a$
    %</одвухгородовых>
\end{theorem}

%<*одвухгородовыхproof>
\begin{proof}
    $$\forall \varepsilon>0 \ \ \exists N \ \ \forall n>N \ \ a-\varepsilon<x_n<a+\varepsilon$$
    $$\forall \varepsilon>0 \ \ \exists K \ \ \forall n>K \ \ a-\varepsilon<z_n<a+\varepsilon$$
    $$\forall \varepsilon > 0 \ \ \exists N_0=max(N,K) \ \ \forall n>N_0 \ \ a-\varepsilon<x_n\leq y_n\leq z_n<a+\varepsilon$$
    $$\text{По определению } \lim y_n=a$$
\end{proof}
%</одвухгородовыхproof>

\begin{corollary}
    $(y_n), (z_n) \ \ \forall n \ \ |y_n|\leq z_n, \ \ \exists \lim z_n=0, \text{ тогда } y_n\to 0$. Доказательство тривиально, т.к. $y_n$ ограничено $z_n$ и $-z_n$.
\end{corollary}

\begin{definition}
    %<*бесконечномалаяпоследовательность>
    $(x_n)$ --- вещ. посл. называется \textbf{бесконечно малой}, если $x_n\to 0$
    %</бесконечномалаяпоследовательность>
\end{definition}

\begin{theorem}
    Если $(x_n),(y_n)$ --- вещ. посл., $x_n$ --- беск.мал., $y_n$ --- огр., \textbf{тогда} $x_ny_n$ --- беск.мал.
\end{theorem}

\begin{proof}
    $$\exists R \ \ \forall n \ \ |y_n|<R, \text{т.к.} y_n \text{--- огр.} $$
    $$|x_ny_n|\leq R|x_n|, R|x_n|\to 0 \Rightarrow y_n\to0$$
\end{proof}

\section*{Нормированные пространства}

\begin{definition}
    Если $K$ --- поле ($K=\mathbb{R}$ или $\mathbb{C}$), $X$ --- множество, то $X$ называется линейным пространством над полем $K$ (и тогда $K$ называется полем скаляр), если определены следующие две операции:
    \begin{enumerate}
        \item $+:X\times X \to X$ --- сложение векторов
        \item $\cdot:K\times X\to X$ --- умножение векторов на скаляры
    \end{enumerate}
\end{definition}

Для этих операций выполняются соответствующие аксиомы \textit{(здесь $A,B,C\in X; a,b\in \R$ или $\C$)}:

Аксиомы сложения векторов
\begin{enumerate}\itemsep0em
    \item $A+B=B+A$
    \item $A+(B+C)=(A+B)+C$
    \item $\exists \bm 0 \in X : A+\bm 0 = a$
\end{enumerate}

Аксиомы умножения векторов на скаляры
\begin{enumerate}\itemsep0em
    \item $(A+B)\cdot a = A\cdot a + B\cdot a$
    \item $A\cdot(a+b) = A\cdot a + A\cdot b$
    \item $(ab)\cdot A = a(b\cdot A)$
    \item $\exists \bm 1 \in X : \bm 1 \cdot a = a$
\end{enumerate}

Ещё есть аксиома $\exists -A\in X : A+(-A)=0$, но у нас её не было.

\begin{definition}
    Норма - отображение $X\to\mathbb{R}, x\mapsto ||x||$, если $X$ - линейное пространство (над $\mathbb{R}$ или $\mathbb{C}$) и выполняется следующее:
    \begin{enumerate}
        \item $\forall x \ \ ||x||\geq 0, ||x||=0\Leftrightarrow x=0$
        \item $\forall x\in X \ \ \forall \lambda\in\mathbb R(\mathbb{C}) \ \ ||\lambda x||=|\lambda|\cdot||x||$
        \item Неравенство треугольника: $\forall x,y\in X \ \ ||x+y||\leq||x||+||y||$
    \end{enumerate}
\end{definition}

\begin{definition}
    Полунорма - норма без свойства $||x||=0\Leftrightarrow x=0$
\end{definition}

\begin{definition}
    Нормированное пространство --- $(X, ||\cdot||)$, где $||||$ - норма
\end{definition}

\begin{lemma}
    О свойстве полунормы.
    \begin{enumerate}
        \item $p(\sum\limits_{finite}\lambda_kx_k)\leq\sum\lambda_kp(x_k)$
        \item $p(0)=0$ - тут $0\in X$
        \item $p(-x)=p(x)$
        \item $|p(x)-p(y)|\leq p(x-y)$
    \end{enumerate}
\end{lemma}

\begin{proof}
    \begin{enumerate}
        \item $p(\lambda_1x_1+\lambda_2x_2+...)\leq p(\lambda_1x_1)+p(\lambda_2x_2+...)$
        \item тривиально
        \item тривиально
        \item $-p(x-y)\leq p(x)-p(y)\leq p(x-y)\\$
              $p(x)=p(y+(x-y))\leq p(y)+p(x-y)$
    \end{enumerate}
\end{proof}

Примеры норм:
\begin{enumerate}
    \item $X=\mathbb R^m \ \ ||x||=\sqrt{\sum\limits_{i}^m x_i^2} \\$
          $X=\mathbb{C}^m \ \ ||x||=\sqrt{\sum\limits_{i}^m |x_i|^2}$
    \item $(\mathbb R^m, ||\cdot||_\infty) \ \ ||x||_\infty=max(|x_1|,|x_2|,...,|x_m|)$
    \item $(\mathbb R^m, ||\cdot||_1) \ \ ||x||_1=\sum\limits_i^m |x_i|$
          \begin{enumerate}
              \item $p(x)=|x_1|$ --- полунорма, но не норма
          \end{enumerate}
\end{enumerate}

\begin{remark}
    %<*метрикапорожденнаянормой>
    Если $(X, ||\cdot||)$ --- норм. пр-во, тогда $\rho(x,y):=||x-y||$ --- метрика,
    порожденная нормой. Не все метрики порождены нормами, например $\rho=\frac{|x-y|}{1+|x-y|}$.
    %</метрикапорожденнаянормой>
\end{remark}

\section*{Арифметические свойства предела}

\begin{theorem}
    %<*арифметическиесвойствапредела>
    \textbf{Об арифметических свойствах предела в нормированном пространстве.}

    Если $(X, ||\cdot||)$ --- норм. пр-во, $(x_n),(y_n)$ --- посл. в $X$, $\lambda_n$ ---
    посл. скаляров, и $x_n\to x_0, y_n\to y_0, \lambda_n\to \lambda_0$, \textbf{тогда}:
    \begin{enumerate}
        \item $x_n\pm y_n\to x_0\pm y_0$
        \item $\lambda_nx_n\to \lambda_0x_0$
        \item $||x_n||\to ||x_0||$
    \end{enumerate}
    %</арифметическиесвойствапредела>
\end{theorem}

%<*арифметическиесвойствапределаproof>
\begin{proof}
    \begin{enumerate}
        \item $\forall \varepsilon \ \ \exists N_1 \ \ \forall n>N_1 \ \ ||x_n - x_0||<\varepsilon$

              $\forall \varepsilon \ \ \exists N_2 \ \ \forall n>N_2 \ \ ||y_n - y_0||<\varepsilon$

              $N:=\max(N_1, N_2)$

              $\forall \varepsilon \ \ \forall n>N \ \ ||(x_n+y_n) - (x_0+y_0)||\leq ||x_n-x_0|| + ||y_n-y_0||\leq 2\varepsilon$

        \item $||\lambda_nx_n-\lambda_0x_0||=||\lambda_nx_n-\lambda_0x_0+\lambda_0x_n-\lambda_0x_n||=||(\lambda_n-\lambda_0)x_n+(x_n-x_0)\lambda_0||\leq\\ ||(\lambda_n-\lambda_0)x_n||+||(x_n-x_0)\lambda_0||=||x_n|||\lambda_n-\lambda_0|+||x_n-x_0|||\lambda_0|$

              $|\lambda_n-\lambda_0|$ и $||x_n-x_0||$ --- бесконечно малые, $||x_n||$ и $|\lambda_n|$ --- ограниченные $\Rightarrow ||x_n|||\lambda_n-\lambda_0|+||x_n-x_0|||\lambda_0|$ --- бесконечно малая

        \item Докажем, что $|||x_n||-||x_0|||\leq ||x_n-x_0||$.

              $$||x_n||=||x_0+(x_n-x_0)||\leq ||x_0||+||x_n-x_0||\Rightarrow ||x_n||-||x_0||\leq ||x_n-x_0||$$
              Аналогично $||x_0||-||x_n||\leq ||x_n-x_0||$.

              Тогда $|||x_n||-||x_0|||\leq ||x_n-x_0||$
    \end{enumerate}
\end{proof}
%</арифметическиесвойствапределаproof>

\begin{theorem}
    %<*арифметическиесвойствапределавr>
    \textbf{Об арифметических свойствах пределов в $\mathbb{R}$}.

    Для $(x_n),(y_n)$
    --- вещ.посл., $\forall n \ \ y_n\not =0, y_0\not = 0$:
    \begin{enumerate}[resume]
        \item $\frac{x_n}{y_n}\to\frac{x_0}{y_0}$
    \end{enumerate}
    %</арифметическиесвойствапределавr>
\end{theorem}
%<*арифметическиесвойствапределавrproof>
\textcolor{red}{Доказательство взято из воздуха.}
\begin{proof}
    Докажем, что $\frac{1}{y_n}\to\frac{1}{y_0}$, если $\forall n \ \ y_n\not =0, y_0\not = 0$.

    $$\left|\frac{1}{y_n}-\frac{1}{y_0}\right|=\left|\frac{y_0-y_n}{y_ny_0} \right|$$

    В числителе бесконечно малая последовательность, в знаменателе ограниченная $\Rightarrow$ дробь --- бесконечно малая последовательность.
\end{proof}
%</арифметическиесвойствапределавrproof>

% Доказательство есть, вы там держитесь.

\end{document}