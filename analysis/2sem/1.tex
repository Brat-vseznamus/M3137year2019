\documentclass[12pt, a4paper]{article}

\usepackage{lastpage}
\usepackage{mathtools}
\usepackage{xltxtra}
\usepackage{libertine}
\usepackage{amsmath}
\usepackage{amsthm}
\usepackage{amsfonts}
\usepackage{amssymb}
\usepackage{enumitem}
\usepackage{xcolor}
\usepackage[left=1.5cm, right=1.5cm, top=2cm, bottom=2cm, bindingoffset=0cm, headheight=15pt]{geometry}
\usepackage{fancyhdr}
\usepackage[russian]{babel}
% \usepackage[utf8]{inputenc}
\usepackage{catchfilebetweentags}
\usepackage{accents}
\usepackage{calc}
\usepackage{etoolbox}
\usepackage{mathrsfs}
\usepackage{wrapfig}

\providetoggle{useproofs}
\settoggle{useproofs}{false}

\pagestyle{fancy}
\lfoot{M3137y2019}
\rhead{\thepage\ из \pageref{LastPage}}

\newcommand{\R}{\mathbb{R}}
\newcommand{\Q}{\mathbb{Q}}
\newcommand{\C}{\mathbb{C}}
\newcommand{\Z}{\mathbb{Z}}
\newcommand{\B}{\mathbb{B}}
\newcommand{\N}{\mathbb{N}}

\newcommand{\const}{\text{const}}

\newcommand{\teormin}{\textcolor{red}{!}\ }

\DeclareMathOperator*{\xor}{\oplus}
\DeclareMathOperator*{\equ}{\sim}
\DeclareMathOperator{\Ln}{\text{Ln}}
\DeclareMathOperator{\sign}{\text{sign}}
\DeclareMathOperator{\Sym}{\text{Sym}}
\DeclareMathOperator{\Asym}{\text{Asym}}
% \DeclareMathOperator{\sh}{\text{sh}}
% \DeclareMathOperator{\tg}{\text{tg}}
% \DeclareMathOperator{\arctg}{\text{arctg}}
% \DeclareMathOperator{\ch}{\text{ch}}

\DeclarePairedDelimiter{\ceil}{\lceil}{\rceil}
\DeclarePairedDelimiter{\abs}{\left\lvert}{\right\rvert}

\setmainfont{Linux Libertine}

\theoremstyle{plain}
\newtheorem{axiom}{Аксиома}
\newtheorem{lemma}{Лемма}

\theoremstyle{remark}
\newtheorem*{remark}{Примечание}
\newtheorem*{exercise}{Упражнение}
\newtheorem*{consequence}{Следствие}
\newtheorem*{example}{Пример}
\newtheorem*{observation}{Наблюдение}

\theoremstyle{definition}
\newtheorem{theorem}{Теорема}
\newtheorem*{definition}{Определение}
\newtheorem*{obozn}{Обозначение}

\setlength{\parindent}{0pt}

\newcommand{\dbltilde}[1]{\accentset{\approx}{#1}}
\newcommand{\intt}{\int\!}

% magical thing that fixes paragraphs
\makeatletter
\patchcmd{\CatchFBT@Fin@l}{\endlinechar\m@ne}{}
  {}{\typeout{Unsuccessful patch!}}
\makeatother

\newcommand{\get}[2]{
    \ExecuteMetaData[#1]{#2}
}

\newcommand{\getproof}[2]{
    \iftoggle{useproofs}{\ExecuteMetaData[#1]{#2proof}}{}
}

\newcommand{\getwithproof}[2]{
    \get{#1}{#2}
    \getproof{#1}{#2}
}

\newcommand{\import}[3]{
    \subsection{#1}
    \getwithproof{#2}{#3}
}

\newcommand{\given}[1]{
    Дано выше. (\ref{#1}, стр. \pageref{#1})
}

\renewcommand{\ker}{\text{Ker }}
\newcommand{\im}{\text{Im }}
\newcommand{\grad}{\text{grad}}

\lhead{Конспект по матанализу}
\cfoot{}
\rfoot{February 10, 2019}

\begin{document}

\section{Монотонные экстремумы}

\begin{theorem}
    \textbf{Критерий монотонности}
%<*критериймонотонности>

    $f\in C(\langle a,b\rangle)$, дифф. в $(a, b)$

    Тогда $f$ --- возрастает $\Leftrightarrow \forall x\in(a, b)\ \ f'(x) \geq 0$ 
%</критериймонотонности>
\end{theorem}
%<*критериймонотонностиproof>
\begin{proof}
    ``$\Rightarrow$'' По определению $f' \quad \frac{f(x+h)-f(x)}{h}\geq 0$

    ``$\Leftarrow$'' $x_1 > x_2$, по т. Лагранжа: $\exists c : f(x_1)-f(x_2)=f'(c)(x_1-x_2)\geq 0$
\end{proof}
%</критериймонотонностиproof>
%<*критериймонотонностиследствия1>
\begin{consequence}
    $f:\langle a,b\rangle\to\R$, тогда:

    $f=\const\Leftrightarrow (f\in C(\langle a,b\rangle)$ --- дифф. на $(a,b), f'\equiv0)$
\end{consequence}
\begin{consequence}
    $f\in C\langle a,b\rangle$, дифф. на $(a,b)$. Тогда:

    $f$ строго возрастает $\Leftrightarrow$ \textcircled{1} и \textcircled{2}

    \textcircled{1} $f'\geq 0$ на $(a,b)$

    \textcircled{2} $f'\not\equiv0$ ни на каком промежутке
\end{consequence}
%</критериймонотонностиследствия1>
%<*критериймонотонностиследствия1proof>
\begin{proof}
    ``$\Rightarrow$'' очевидно

    ``$\Leftarrow$'' По лемме о возрастании в отрезке
\end{proof}
%</критериймонотонностиследствия1proof>
%<*критериймонотонностиследствия2>
\begin{consequence}
    О доказательстве неравенств

    $g, f\in C([a,b\rangle)$, дифф. в $(a,b)$

    $f(a)\leq g(a); \forall x\in(a,b) \ \ f'(x)\leq g'(x)$

    Тогда $\forall x\in[a,b\rangle \ \ f(x)\leq g(x)$
\end{consequence}
%</критериймонотонностиследствия2>
%<*критериймонотонностиследствия2proof>
\begin{proof}
    $g-f$ --- возр., $g(a)-f(a)\geq 0$
\end{proof}
%</критериймонотонностиследствия2proof>

%<*локальныймаксимум>
\begin{definition}
    $f:E\subset\R\to\R, x_0\in E$ --- \textbf{локальный максимум функции}, если
    $$\exists U(x_0) \ \ \forall x\in U(x_0)\cap E\ \ f(x)\leq f(x_0)$$
\end{definition}

Аналогично определяется минимум.

\begin{definition}
    \textbf{Экстремум} --- точка минимума либо максимума.
\end{definition}
%</локальныймаксимум>

\begin{theorem}
%<*необходимостьидостаточностьусловиялокальногоэкстремума>
    $f:\langle a,b\rangle\to\R\quad x_0\in(a,b)\quad f$ --- дифф. на $(a,b)$

    Тогда:\begin{enumerate}
        \item $x_0$ --- лок. экстремум $\Rightarrow f'(x_0)=0$
        \item $f$ --- $n$ раз дифф. в $x_0$
        
        $f'(x_0)=f''(x_0)=\ldots=f^{(n-1)}(x_0)=0$

        Если $f^{(n)}(x_0)<0$, то $\begin{cases}
            n \text{ --- чет.}: & x_0 \text{ --- локальный максимум} \\
            n \text{ --- нечет.}: & x_0 \text{ --- не экстремум}
        \end{cases}$

        Если $f^{(n)}(x_0)>0$, то $\begin{cases}
            n \text{ --- чет.}: & x_0 \text{ --- локальный минимум} \\
            n \text{ --- нечет.}: & x_0 \text{ --- не экстремум}
        \end{cases}$
    \end{enumerate}
%</необходимостьидостаточностьусловиялокальногоэкстремума>
\end{theorem}
%<*необходимостьидостаточностьусловиялокальногоэкстремумаproof>
\begin{proof}
    \begin{enumerate}
        \item т. Ферма
        \item ф. Тейлора
        
        $$f(x)=\text{T}_n(f, x_0)(x)+o((x-x_0)^n)$$
        $$f(x)=f(x_0)+\frac{f^{(n)}(x_0)}{n!}(x-x_0)^n+o((x-x_0)^n)$$

        при $x$, близких к $x_0$:
        $$\sign(f(x)-f(x_0)) = \sign\left(\frac{f^{(n)}(x_0)}{n!}(x-x_0)^n\right)$$
        Тогда при чётном $n$ $$\sign(f(x)-f(x_0))=\sign f^{(n)}(x_0) \Rightarrow x_0 \text{ --- экстр.}$$
        При нечётном $n$ $$\sign(f(x)-f(x_0))=\begin{cases}
            f^{(n)}(x_0), & x>x_0 \\
            -f^{(n)}(x_0), & x<x_0 \\
        \end{cases} \Rightarrow x_0 \text{ --- не экстр.}$$
    \end{enumerate}
\end{proof}
%</необходимостьидостаточностьусловиялокальногоэкстремумаproof>

\section{Интеграл}

\subsection{Неопределенный интеграл}

\begin{definition}
%<*первообразная>
    $F, f:\langle a,b\rangle\to\R$

    $F$ --- \textbf{первообразная} $f$ на $\langle a,b\rangle$
    $$\forall x\in\langle a,b\rangle\quad F'(x)=f(x)$$
%</первообразная>
\end{definition}

\begin{theorem}
    \textbf{О существовании первообразной}
%<*осуществованиипервообразной>
    
    $f\in C^0(\langle a,b\rangle)$ тогда у $f$ существует первообразная.
%</осуществованиипервообразной>
\end{theorem}
\begin{proof}
    Чуть позже.
\end{proof}

\begin{theorem}
    $F$ --- первообразная $f$ на $\langle a,b\rangle$. Тогда:
    \begin{enumerate}
        \item $\forall c\in\R\quad F+c$ --- тоже первообразная
        \item Никаких других первообразных нет, т.е. если $G$ --- перв. $f$, то $\exists c\in\R : G=F+c$
    \end{enumerate}
\end{theorem}
\begin{proof}
    \begin{enumerate}
        \item очевидно
        \item $F'=f, G'=f \quad (G-F)'\equiv0\Rightarrow G-F=\const$
    \end{enumerate}
\end{proof}

\begin{definition}
%<*неопределенныйинтеграл>

    \textbf{Неопределенный интеграл} $f$ на $\langle a,b\rangle$ --- множество всех первообразных $f$:
    $$\{F+c, c\in\R\} \text{, где $F$ --- первообразная}$$
    Обозначается $\intt f=F+c$ или $\intt f(x)dx$
%</неопределенныйинтеграл>
\end{definition}

%<*таблицапервообразных>
$$\intt x^ndx=\frac{x^{(n+1)}}{n+1} + C, n\not=-1$$
$$\intt \frac{1}{x}dx=\ln x + C$$
$$\intt \sin x dx = -\cos x + C$$
$$\intt \cos x dx = \sin x + C$$
$$\intt e^xdx = e^x + C$$
$$\intt \frac{1}{\sqrt{1-x^2}}dx=\arcsin x + C$$
$$\intt \frac{1}{\sqrt{1+x^2}} dx=\ln(x+\sqrt{1+x^2}) + C \text{ --- длинный логарифм}$$
$$\intt \frac{1}{\cos^2 x}dx=\tg x + C$$
$$\intt \frac{1}{\sin^2 x}dx=-\ctg x + C$$
%</таблицапервообразных>

\begin{theorem}
    %<*свойстванеопределенногоинтеграла>
    $f, g$ имеют первообразную на $\langle a,b \rangle$. Тогда

    \begin{enumerate}
        \item Линейность: $$\intt (f + g) = \intt f + \intt g$$
        $$\forall \alpha\in\R\ \ \intt \alpha f = \alpha\intt f$$
        \item $\varphi \langle c,d\rangle\to \langle a,b\rangle$
        $$\intt f(\varphi(t))\cdot \varphi'(t)dt=\left(\intt f(x)dx \right)|_{x=\varphi(t)}=F(\varphi(t))$$
        Частный случай: $\alpha, \beta\in\R:$
        $$\intt f(\alpha t + \beta)dt=\frac{1}{\alpha}F(\alpha t+\beta)$$
        \item $f,g$ --- дифф. на $\langle a,b\rangle$; $f'g$ --- имеет первообр.
        
        Тогда $fg'$ имеет первообразную и $$\intt fg'=fg-\intt f'g$$
    \end{enumerate}
    %</свойстванеопределенногоинтеграла>
\end{theorem}
%<*свойстванеопределенногоинтегралаproof>
\begin{proof}
    \begin{enumerate}
        \item $(F + G)' = F' + G' \quad (\alpha F)' = \alpha F'$
        \item $(F(\varphi(t)))'=f(\varphi(t))\cdot \varphi'(t)$
        \item $(fg-\intt f'g)'=f'g+fg'-f'g=fg'$
    \end{enumerate}
\end{proof}
%</свойстванеопределенногоинтегралаproof>
\begin{remark}
    Если $\varphi$ обратима, то:
    $$\intt f(x)dx=\left(\intt f(\varphi(t))\varphi'(t)dt \right)|_{t:=\varphi^{-1}(x)}$$
\end{remark}
$df:=f'(x)dx$

$$\intt \frac{1}{\sqrt{1+x^2}}dx = \left[x:=\tg t\right] = \intt \frac{1}{\sqrt{1 + \tg^2 t}}\cdot \frac{1}{\cos^2 t}dt=\intt \frac{1}{\sqrt{\frac{\cos^2 t + \sin^2 t}{\cos^2 t}}}\cdot \frac{1}{\cos^2 t}dt=\intt \frac{1}{\frac{1}{\cos t}}\cdot \frac{1}{\cos^2 t}dt=$$
$$=\intt \frac{\cos t dt}{\cos^2 t}=\intt \frac{\cos t dt}{1-\sin^2 t}=[y:=\sin t]=\intt \frac{dy}{1-y^2}=\intt \frac{1}{1-y}\cdot\frac{1}{1+y}=\intt \frac{1}{2}\left(\frac{1}{1-y}+\frac{1}{1+y}\right)dy=$$
$$=\frac{1}{2}\left(-\ln(1-y)+\ln(1+y)\right)=\frac{1}{2}\ln\frac{1+y}{1-y}=\frac{1}{2}\ln\frac{1+\sin t}{1-\sin t}=\frac{1}{2}\ln\frac{1+\sin \arctg x}{1-\sin\arctg x}$$

\subsection{Гиперболические тригонометрические функции}

$$\sh t = \frac{e^t-e^{-t}}{2}\quad \ch t = \frac{e^t+e^{-t}}{2}$$
Они полезны тем, что по ним висит нить, закрепленная в двух точках.

$$\sh 2t = 2\sh t \ch t$$
$$(\ch t)^2+\left(\frac{\sh t}{i}\right)^2=1$$

$$\intt \frac{1}{\sqrt{1+x^2}}dx=[x=\sh t]=\intt \frac{1}{\sqrt{ch^2 t}}ch t dt=\intt 1dt=t$$

\subsection{Равномерно непрерывные функции}

\begin{definition}
%<*равномернаянепрерывность>
    $f:\langle a,b\rangle\subset\R\to\R$ \textbf{равномерно непрерывна} на $\langle a,b\rangle$:
    $$\forall \varepsilon>0 \ \ \exists \delta>0 \ \ \forall x_1, x_2\in\langle a,b\rangle : |x_1-x_2|<\delta \quad |f(x_1) - f(x_2)|<\varepsilon$$

    Или для метрического пространства:
    $$\forall \varepsilon>0 \ \ \exists \delta>0 \ \ \forall x_1, x_2 \ \ \rho(x_1, x_2)<\delta \quad \rho(f(x_1), f(x_2))<\varepsilon$$
    Отличие от непрерывности на отрезке в том, что $\delta$ зависит только от $\varepsilon$ и подходит для всех $x_1, x_2$.
%</равномернаянепрерывность>
\end{definition}
\begin{example}
    \begin{enumerate}
        \item $f(x)=x$ равномерно непрерывна.
        \item $f(x)=x^2$ $\langle a,b\rangle=\R\quad \varepsilon:=1\ \ \exists? \delta$
        
        $x_1:=\frac{1}{\delta} + \frac{\delta}{2}, x_2:=\frac{1}{\delta}$

        $x_1^2-x_2^2=1+\frac{\delta^2}{4}>1 \Rightarrow f$ --- не равномерно непрерывна.
    \end{enumerate}
\end{example}

\begin{theorem}
%<*теоремакантора>
    $f:X\to Y, X$ --- комп., $f$ --- непр. на $X$

    Тогда $f$ --- равномерно непр.
%</теоремакантора>
\end{theorem}
%<*теоремакантораproof>
\begin{proof}
    От противного.

    $$\exists \varepsilon>0 \ \ \forall \delta > 0 \ \ \exists x_n, \overline x_n : \rho(x_n, \overline x_n) < \delta \quad \rho(f(x_n), f(\overline x_n))\geq\varepsilon$$
    $$\delta:=\frac{1}{n} \ \ \exists x_n, \overline x_n : \rho(x_n, \overline x_n) < \delta \quad \rho(f(x_n), f(\overline x_n))\geq\varepsilon$$

    Выберем $x_{n_k}\to\tilde x, \overline x_{n_k}\to\dbltilde{x}$

    $$\rho(\tilde x, \dbltilde x)\leq \lim_{n\to\infty}\delta=0 \Rightarrow \tilde x = \dbltilde x$$

    Тогда $f(x_{n_k})\to f(\tilde x), f(\overline x_{n_k})\to f(\tilde x)$, противоречие с $\rho(f(x_n), f(\overline x_n))\geq\varepsilon$
\end{proof}
%</теоремакантораproof>

\begin{example}
    $f(x)=\sqrt x\quad X=[0, +\infty)$

    По т. Кантора: $f$ равномерно непрерывна на $[0,1]$

    При $x\geq\frac{1}{2} \quad |\sqrt{x_1} - \sqrt{x_2}|=\frac{1}{2\sqrt c}|x_1-x_2|<|x_1-x_2|$, т.е. тоже равномерно непрерывна.
\end{example}

\subsection{Конфетка: т. Брауэра о неподвижной точке}

Статья от Matousek, Zigler, Bjorner (arxiv: 1409.7890v1)

Игра Hex: два игрока --- чёрный и белый, на своем ходе красят один шестиугольник в свой цвет. Условие выигрыша --- путь искомого цвета с одной стороны в сторону нужного цвета --- две противоположные стороны имеют черный цвет, две другие --- белый.

\begin{theorem}
    Дана доска для Hex --- параллелограм $k\times l$, покрашенная в 2 цвета.

    Это выигрышная доска для одного из игроков.
\end{theorem}
\begin{proof}
    Рассмотрим первый ряд \textit{(прилегающий к чёрной стороне)}. Если в нём нет черных клеток, белый выиграл. Пойдём по границе черных и белых клеток так, что справа всегда черная клетка, слева белая. В этом пути нет самопересечений, т.к. в точке самопересечеения с обеих сторон черные клетки, мы так не идём.

    Представим доску в виде прямоугольной сетки, где вершины соединены, если из соответствующего шестиугольника можно прийти в другой соответствующий шестиугольник.
\end{proof}

\begin{theorem}
    %<*теоремаонеподвижнойточке>
    $f:[0,1]\times[0,1]\to[0,1]\times[0,1]$, непр.

    Тогда $\exists x\in[0,1]^2 : f(x)=x$, т.е. есть неподвижная точка.

    Обобщенный вариант:
    \begin{enumerate}
        \item $f:[0,1]^m\to[0,1]^m$ --- непр.
        \item $f:B(0,1)\subset\R^m\to B(0,1)$ --- непр.
        \item $f:S(0,1)\subset\R^m\to S(0,1)$ --- непр.
    \end{enumerate}
    %</теоремаонеподвижнойточке>
\end{theorem}
%<*теоремаонеподвижнойточкеproof>
\begin{proof}
    $\rho : [0, 1]^2\to\R$

    $\rho(x, y)=\max(|x_1-y_1|, |x_2-y_2|)$ --- непр. в $[0,1]^2$

    От противного --- пусть $\forall x\in[0,1]^2 \quad f(x)\not=x$

    Тогда $\forall x \quad \rho(f(x), x) > 0 \quad x\mapsto \rho(f(x), x)$ --- непр., $>0$

    По т. Вейерштрасса $\exists \varepsilon>0 \ \ \forall x\in[0,1] \ \ \rho(f(x), x))\geq \varepsilon$

    По т. Кантора для $f$: для этого $\varepsilon \ \ \exists \delta<\varepsilon:$
    $$\forall x, \overline x : ||x-\overline x||<\delta \quad ||f(x)-f(\overline x)||<\varepsilon$$
    Можно писать не $||\cdot||$, а $\rho$.

    Возьмём $n: \frac{\sqrt 2}{n}<\delta$

    Построим доску $Hex(n+1, n+1)$, где $n+1$ --- число узлов.

    Логические координаты узла $(v_1, v_2)\quad v_1,v_2\in\{0\ldots n\}$ имеют физические координаты, то есть узлу сопоставляется точка на квадрате с координатами $\left(\frac{v_1}{n}, \frac{v_2}{n}\right)$

    $K(V):=\min\{i\in\{1,2\} : |f(\frac{v}{n}) - \frac{v_i}{n}|\geq \varepsilon\}$
    %</теоремаонеподвижнойточкеproof>

    Продолжение на следующей лекции.
\end{proof}

\end{document}