\documentclass[12pt, a4paper]{article}

\usepackage{lastpage}
\usepackage{mathtools}
\usepackage{xltxtra}
\usepackage{libertine}
\usepackage{amsmath}
\usepackage{amsthm}
\usepackage{amsfonts}
\usepackage{amssymb}
\usepackage{enumitem}
\usepackage{xcolor}
\usepackage[left=1.5cm, right=1.5cm, top=2cm, bottom=2cm, bindingoffset=0cm, headheight=15pt]{geometry}
\usepackage{fancyhdr}
\usepackage[russian]{babel}
% \usepackage[utf8]{inputenc}
\usepackage{catchfilebetweentags}
\usepackage{accents}
\usepackage{calc}
\usepackage{etoolbox}
\usepackage{mathrsfs}
\usepackage{wrapfig}

\providetoggle{useproofs}
\settoggle{useproofs}{false}

\pagestyle{fancy}
\lfoot{M3137y2019}
\rhead{\thepage\ из \pageref{LastPage}}

\newcommand{\R}{\mathbb{R}}
\newcommand{\Q}{\mathbb{Q}}
\newcommand{\C}{\mathbb{C}}
\newcommand{\Z}{\mathbb{Z}}
\newcommand{\B}{\mathbb{B}}
\newcommand{\N}{\mathbb{N}}

\newcommand{\const}{\text{const}}

\newcommand{\teormin}{\textcolor{red}{!}\ }

\DeclareMathOperator*{\xor}{\oplus}
\DeclareMathOperator*{\equ}{\sim}
\DeclareMathOperator{\Ln}{\text{Ln}}
\DeclareMathOperator{\sign}{\text{sign}}
\DeclareMathOperator{\Sym}{\text{Sym}}
\DeclareMathOperator{\Asym}{\text{Asym}}
% \DeclareMathOperator{\sh}{\text{sh}}
% \DeclareMathOperator{\tg}{\text{tg}}
% \DeclareMathOperator{\arctg}{\text{arctg}}
% \DeclareMathOperator{\ch}{\text{ch}}

\DeclarePairedDelimiter{\ceil}{\lceil}{\rceil}
\DeclarePairedDelimiter{\abs}{\left\lvert}{\right\rvert}

\setmainfont{Linux Libertine}

\theoremstyle{plain}
\newtheorem{axiom}{Аксиома}
\newtheorem{lemma}{Лемма}

\theoremstyle{remark}
\newtheorem*{remark}{Примечание}
\newtheorem*{exercise}{Упражнение}
\newtheorem*{consequence}{Следствие}
\newtheorem*{example}{Пример}
\newtheorem*{observation}{Наблюдение}

\theoremstyle{definition}
\newtheorem{theorem}{Теорема}
\newtheorem*{definition}{Определение}
\newtheorem*{obozn}{Обозначение}

\setlength{\parindent}{0pt}

\newcommand{\dbltilde}[1]{\accentset{\approx}{#1}}
\newcommand{\intt}{\int\!}

% magical thing that fixes paragraphs
\makeatletter
\patchcmd{\CatchFBT@Fin@l}{\endlinechar\m@ne}{}
  {}{\typeout{Unsuccessful patch!}}
\makeatother

\newcommand{\get}[2]{
    \ExecuteMetaData[#1]{#2}
}

\newcommand{\getproof}[2]{
    \iftoggle{useproofs}{\ExecuteMetaData[#1]{#2proof}}{}
}

\newcommand{\getwithproof}[2]{
    \get{#1}{#2}
    \getproof{#1}{#2}
}

\newcommand{\import}[3]{
    \subsection{#1}
    \getwithproof{#2}{#3}
}

\newcommand{\given}[1]{
    Дано выше. (\ref{#1}, стр. \pageref{#1})
}

\renewcommand{\ker}{\text{Ker }}
\newcommand{\im}{\text{Im }}
\newcommand{\grad}{\text{grad}}

\lhead{Конспект по матанализу}
\cfoot{}
\rfoot{Лекция 14}

\renewcommand{\thesubsection}{\arabic{subsection}.}
\begin{document}

\section*{Бесконечные произведения}

\begin{definition}
    %<*бесконечноепроизведение>
    $\prod\limits_{i=1}^{+\infty} p_n$ : $\prod_N := \prod\limits_{n=1}^N p_n \ \ \lim\limits_{n\to+\infty}\prod_N=P$

    \begin{itemize}
        \item $P\in (0, +\infty) \Rightarrow \prod\limits_{i=1}^{+\infty} p_n$ сходится к $P$
        \item $P=+\infty \Rightarrow \prod\limits_{i=1}^{+\infty} p_n$ расходится к $+\infty$
        \item $P=0 \Rightarrow \prod\limits_{i=1}^{+\infty} p_n$ расходится к $0$
        \item $\not\exists\lim \prod_n :$ расходится
    \end{itemize}
    %</бесконечноепроизведение>
\end{definition}

\begin{example}
    $$\prod_{n=2}^{+\infty}\left(1-\frac{1}{n^2}\right)=\frac{3}{4}\frac{8}{9}\frac{15}{16}\ldots$$
    $$1-\frac{1}{n^2}=\frac{(n-1)(n+1)}{n^2} \quad \sideset{}{_N}\prod = \frac{1\cdot 3}{2^2}\frac{2\cdot4}{3^2}\frac{3\cdot5}{4^2}\cdots\frac{(N-1)(N+1)}{N^2} = \frac{N+1}{2N}$$
\end{example}

\begin{example}
    Формула Валлиса:
    $$\frac{2}{1}\frac{2}{3}\frac{4}{3}\frac{4}{5}\frac{6}{5}\ldots=\frac{\pi}{2}$$
    $$\left(\frac{(2n)!!}{(2n-1)!!}\right)\frac{1}{2n}$$
\end{example}

\begin{example}
    $$\cos\frac{\varphi}{2}\cdot\cos\frac{\varphi}{4}\cdots\cos\frac{\varphi}{2^n}=\frac{\sin\varphi}{2^n\sin\frac{\varphi}{2^n}}\xrightarrow[n\to+\infty]{}\frac{\sin\varphi}{\varphi}$$
    $$\prod_{n=1}^{+\infty}\cos\frac{\varphi}{2^n}=\frac{\sin\varphi}{\varphi}$$
\end{example}

\begin{definition}
    $\pi_m:=\prod\limits_{n=m+1}^{+\infty} p_n$
\end{definition}

Свойства:
\begin{enumerate}
    \item $\prod\limits_{n=1}^{+\infty} p_n$ сходится $\Leftrightarrow \forall m \ \ \pi_m$ сходится
    \item $\prod\limits_{n=1}^{+\infty} p_n$ сходится. Тогда $\pi_m\xrightarrow[m\to+\infty]{}1$
    \item $\prod\limits_{n=1}^{+\infty} p_n$ сходится $\Rightarrow p_n\to1$
    \item $\prod\limits_{n=1}^{+\infty} p_n$ сходится $\Leftrightarrow \sum\limits_{n=1}^{+\infty} \ln p_n$ сходится
\end{enumerate}

$$\pi_m = \lim_{N\to+\infty}\frac{\prod\limits_{n=1}^N p_n}{\prod\limits_{k=1}^m p_n}=\frac{P}{\prod_m}$$
$$\ln\left(\prod\limits_{n=1}^{+\infty} p_n\right)=\sum_{n=1}^N \ln p_n$$

\begin{theorem}
    %<*условиясходимостипроизведения>
    \begin{enumerate}
        \item $a_n>0$ НСНМ. Тогда $\prod$ сходится $\Leftrightarrow \sum a_n$ сходится.
        \item $\sum a_n$ сходится, $\sum a_n^2$ сходится $\Rightarrow \prod(1+a_n)$ сходится.
    \end{enumerate}
    %</условиясходимостипроизведения>
\end{theorem}

%<*условиясходимостипроизведенияproof>
\begin{proof}
    \begin{enumerate}
        \item \textcolor{red}{???}
        \item $\ln(1+a_n)=a_n-\frac{a_n^2}{2}+o(a_n^2)$
        $$\sum_{n=1}^N \ln(1+a_n)=\sum_{n=1}^N a_n - \frac{1}{2}\sum_{n=1}^N a_n^2 + \underbrace{\sum_{n=1}^N o(a_n^2)}_{\text{абс.сх}}$$
    \end{enumerate}
\end{proof}
%</условиясходимостипроизведенияproof>

\begin{lemma}
    $\prod(n, x) := \int\limits_0^n \left(1-\frac{t}{n}\right)^n t^{x-1}dt, x>0$

    Тогда $\prod(n, x) = \frac{1\cdot2\cdots n}{x\cdot(x+1)\cdots(x+n)}n^x$
\end{lemma}
\begin{proof}
    $$\prod(n, x)\stackrel{t=ny}{=}n^x\int\limits_0^1 (1-y)^ny^{x-1}dy=$$
    $$=n^x\left((1-y)^n\frac{1}{x}y^x\Bigg|_{y=0}^{y=1} + \frac{n}{x} \int\limits_0^1 (1-y)^{n-1} y^x dy \right)=$$
    $$=n^x\frac{n}{x}\int\limits_0^1 (1-y)^{n-1} y^x dy=$$
    $$=n^x\frac{n}{x}\frac{n-1}{x+1}\int\limits_0^1 (1-y)^{n-2} y^{x+1} dy=$$
    $$=\ldots=n^x\frac{n}{x}\frac{n-1}{x+1}\cdots\frac{1}{x+n-1}\int\limits_0^1 y^{x+n-1} dy$$
\end{proof}

\begin{lemma}
    %<*приближениеепределом>
    $0\le t\le n$. Тогда
    $$0\le e^{-t} - \left(1-\frac{t}{n}\right)^n \le \frac{t^2e^{-t}}{n}$$
    \textcolor{red}{Надо проверить формулировку}
    %</приближениеепределом>
\end{lemma}
%<*приближениеепределомproof>
\begin{proof}
    $$1+y\le e^y \le (1-y)^{-1}\ ,\ y\in[0,1]$$
    Это неравенство следует из выпуклости экспоненты.
    $$y:=\frac{t}{n}$$
    $$1+\frac{t}{n}\le e^{\frac{t}{n}} \le \left(1-\frac{t}{n}\right)^{-1}$$
    $$\left(1+\frac{t}{n}\right)^{-n} \stackrel{*}{\ge} e^{-t} \ge \left(1-\frac{t}{n}\right)^{n}$$
    $$e^{-t}\left(1 - \left(1+\frac{t}{n}\right)^{-n}\left(1-\frac{t}{n}\right)^n\right) \stackrel{*}{\ge} e^{-t}\left(1 - e^t\left(1-\frac{t}{n}\right)^n\right) = e^{-t} - \left(1-\frac{t}{n}\right)^{n} \ge 0$$
    $$e^{-t}\left(1 -\left(1-\frac{t^2}{n^2}\right)^n\right) \stackrel{\text{неравенство Бернулли}}{\le} \frac{t^2}{n}e^{-t}$$
\end{proof}
%</приближениеепределомproof>

\begin{theorem}
    Формула Эйлера.

    %<*формулаэйлера>
    При $x>0$ $$\lim_{n\to+\infty} \frac{1\cdot2\cdots n}{x(x+1)\cdots(x+n)}n^x=\Gamma(x)$$
    %</формулаэйлера>
\end{theorem}
\begin{remark}
    $$\Gamma(x)=\int\limits_0^{+\infty}t^{x-1}e^{-t}dt$$
    $$\Gamma(n+1)=n!$$
    $$\Gamma(x+1)=x\Gamma(x)$$
\end{remark}
%<*формулаэйлераproof>
\begin{proof}
    $$\Gamma(x) - \lim\Pi(n, x) = \int\limits_0^{+\infty} t^{x-1}e^{-t}dt - \lim\int\limits_0^n \left(1-\frac{t}{n}\right)^n t^{x-1}dt$$
    $$\lim_{n\to+\infty}\left(\underbrace{\int\limits_0^n\left(e^{-t}-\left(1-\frac{t}{n}\right)^n\right) t^{x-1} dt}_{\text{I}} + \underbrace{\int\limits_{n}^{+\infty} e^{-t}t^{x-1}dt}_{\text{II}}\right)\stackrel{?}{=}0$$
    \textcolor{red}{Доказательство скипнуто}
\end{proof}
%</формулаэйлераproof>

\begin{theorem}
    Формула Вейерштрасса.

    %<*формулавейерштрасса>
    При $x>0$:
    $$\frac{1}{\Gamma(x)} = x e^{\gamma x} \prod_{k=1}^{+\infty}\left(1+\frac{x}{k}\right)e^{-\frac{x}{k}}$$
    где $\gamma = \lim(1+\ldots+\frac{1}{n}-\ln n)$ --- постоянная Эйлера. 
    %</формулавейерштрасса>
\end{theorem}
%<*формулавейерштрассаproof>
\begin{proof}
    $$``1+a_k''=\left(1+\frac{x}{k}\right)e^{-\frac{x}{k}} = \left(1+\frac{x}{k}\right)\left(1-\frac{x}{k}+\frac{x^2}{2k^2}+\ldots\right)=$$
    $$1-\frac{x^2}{2k^2}+o(\frac{1}{k^2})$$
    \textcolor{red}{Доказательство скипнуто}
\end{proof}
%</формулавейерштрассаproof>

\begin{lemma}
    Об оценке нормы линейного отображения.

    %<*оценканормыотображения>
    $A:\R^m \to \R^l \ \ A=(a_{ij})$. Тогда $\forall x\in\R^m$: $|Ax|\le C_A|x|$, где $C_a=\left(\sum\limits_{i, j} a_{ij}^2\right)^{\frac{1}{2}}$
    %</оценканормыотображения>
\end{lemma}
%<*оценканормыотображенияproof>
\begin{proof}
    $$|Ax|^2 = \sum_{j}\left(\sum_j a_{ij}x_j\right)^2 \stackrel{\text{КБШ}}{\le} \sum_i\left(\left(\sum_j a_{ij}^2\right)\left(\sum_j x_{j}^2\right)\right)$$
\end{proof}
%</оценканормыотображенияproof>

\begin{consequence}
    Линейное отображение непрерывно всюду:
    $$|Ax-Ax_0|=|A(x-x_0)|\le C_a|x-x_0|$$
\end{consequence}

\begin{theorem}
    %<*дифференциированиекомпозиции>
    \begin{itemize}[itemsep=1mm, after=\vspace{5mm}]
        \item $F:E\subset\R^m\to\R^l$
        \item $G:I\subset\R^l\to\R^n$
        \item $F(E)\subset I$
        \item $a\in Int E$
        \item $F$ дифф. в $a$
        \item $F(a)\in Int I$
        \item $G$ дифф. в $F(a)$
    \end{itemize}

    Тогда $G\circ F$ дифф. в $a$, $(G\circ F)'(a)=G'(F(a))F'(a)$
    %</дифференциированиекомпозиции>
\end{theorem}
%<*дифференциированиекомпозицииproof>
\begin{proof}
    $b:=F(a)$. По определению:
    $$F(a+h)=F(a) + F'(a)h + \alpha(h)|h|$$
    $$G(b+k)=G(b) + G'(b)k + \beta(k)|k|$$
    $$G(F(a+h))=G(F(a))+G'(F(a))(F'(a)h+\alpha(h)|h|)+\beta(k)|k|=$$
    $$=G(F(a))+G'(F(a))F'(a)h+G'(b)\alpha(h)|h|+\beta(k)|F'(a)h+\alpha(h)|h||$$
    Надо доказать, что $\underbrace{G'(b)\alpha(h)|h|}_{\text{I}}+\underbrace{\beta(k)|F'(a)h+\alpha(h)|h||}_{\text{II}}=\gamma(h)|h|$.
    $$\text{I}=|G'(b)\alpha(h)|h||\le C_{G'(b)} |\alpha(h)||h|$$
    $$|F'(a)h+\alpha(h)|h||\le |F'(a)h|+|\alpha(h)||h|\le (C_{F'(a)}+\alpha(h))|h|$$
    \textcolor{red}{Доказательство скипнуто}
\end{proof}
%</дифференциированиекомпозицииproof>

\begin{remark}
    $(F\circ G\circ H)'(x) = F'(G(H(x)))\cdot G'(H(x)) \cdot H'(x)$
\end{remark}

% \begin{example}
%     $G:\R^m\to\R^l \quad x,v\in\R^m$

%     $$H(t):=G(x+tv), t\in \R$$
%     $H'(t_0)=? \quad \sphericalangle F:\R\to\R^m \ \ F(t) = x+tv \quad F'=\begin{pmatrix}
%         v_1 \\
%         \vdots \\
%         v_m
%     \end{pmatrix}=v$
% \end{example}

$F=(f_1, f_2\ldots f_l) \quad G=(g_1, g_2\ldots g_n) \quad H=G\circ F = (h_1\ldots h_n)$

$h_j(x_1\ldots x_m)=g_j(f_1(x_1\ldots x_m), f_2(x_1\ldots x_m)\ldots f_l(x_1\ldots x_m))$

$$\frac{\partial h_j}{\partial x_i} = \sum_k \frac{\partial g_j}{\partial y_k} \frac{\partial f_k}{\partial x_i} $$
$$h'(x)=\frac{\partial g}{\partial y_1} \frac{df_1}{d x} + \frac{\partial g}{\partial y_2} \frac{df_2}{d x}+\ldots\frac{\partial g}{\partial y_m} \frac{df_m}{d x}$$

\textcolor{red}{Скипнуто}

\begin{lemma}
    %<*дифференциированиепроизведений>
    \begin{itemize}[itemsep=1mm, after=\vspace{3mm}]
        \item $F, G: E\subset \R^m \to\R^l$
        \item $a\in IntE$
        \item $\lambda:E\to\R$
        \item $F,G,\lambda$ дифф. в $a$
    \end{itemize}

    Тогда $\lambda F, \langle F, G\rangle$ --- дифф. в $a$:
    \begin{enumerate}
        \item $(\lambda F)'(a) (h) = (\lambda'(a)h)F(A) + \lambda(a) F'(a)h$
        \item $\langle F, G\rangle'(a) (h) = \langle F'(a)h, G(a) \rangle + \langle F(a), G'(a)h\rangle$
    \end{enumerate}
    
    Здесь $h$ нигде не умножается, на него действуют операторы дифференциирования.
    %</дифференциированиепроизведений>
\end{lemma}
%<*дифференциированиепроизведенийproof>
\begin{proof}
    \begin{enumerate}
        \item Для координатной функции $l=1$:
        $$\lambda f(a+h)-\lambda f(a) = (\lambda(a)+\lambda'(a)h+o(h))(f(a)+f'(a)h+o(h))-\lambda(a)f(a)=$$
        $$=(\lambda'(a)h)f(a)+\lambda(a)f'(a)h+o(h)$$
        $$|(\lambda'(a)h)(f'(a)h)|\le C_{\lambda'(a)}|h|C_{f'(a)}|h|$$

        \item $$\langle F,G\rangle=\sum_{i=1}^l f_ig_i$$
        По линейности всего и пункту 1:
        $$\langle F,G\rangle'(a)h=\sum_i (f_ig_i)'(a)h\stackrel{1.}{=}\sum_i f_i'(a)h g_i(a)+f(a)g_i'(a)(h)=\langle F'(a)h, G(a) \rangle + \langle F(a), G'(a)h\rangle$$
    \end{enumerate}
\end{proof}
%</дифференциированиепроизведенийproof>

\begin{remark}
    $m=1$:
    $$\langle F,G\rangle'(a) = \langle F'(a), G(a) \rangle + \langle F(a), G'(a) \rangle$$
\end{remark}

\begin{theorem}
    Лагранжа для векторнозначных функций.

    %<*теоремалагранжадлявекторнозначных>
    $F:[a,b]\to\R^m$ --- непр. на $[a,b]$, дифф. на $(a, b)$

    Тогда $\exists c\in(a,b) : |F(b) - F(a)| \le |F'(c)|(b-a)$
    %</теоремалагранжадлявекторнозначных>
\end{theorem}
%<*теоремалагранжадлявекторнозначныхproof>
\begin{proof}
    $$\varphi(t):=\langle F(b) - F(a), F(t) - F(a)\rangle, t\in[a,b]$$
    $$\varphi(a) = 0 \quad \varphi(b) = |F(b) - F(a)|^2$$
    $$\varphi'(t) = \langle F(b) - F(a), F'(t)\rangle$$
    Теорема Лагранжа \textit{(для обычных функций)}:
    $$\varphi(b)-\varphi(a)=\varphi'(c)(b-a)$$
    $$|F(b)-F(a)|^2=(b-a)\langle F(b) - F(a), F'(c)\rangle \stackrel{\text{КБШ}}{\le} (b-a)|F(b)-F(a)||F'(c)|$$
\end{proof}
%</теоремалагранжадлявекторнозначныхproof>

\begin{remark}
    ``$=$'' не достигается, если ехать быстро и криво. % каво?
\end{remark}
\begin{example}
    $F:[0, 2\pi]\to\R^2 \quad t\mapsto (\cos t, \sin t) \quad F'(t)=(-\sin t, \cos t)$

    $a:=0, b:=2\pi$
    $$F(b)-F(a)=0\stackrel{?}{=}|F'(c)|(b-a)$$
    Нет.
\end{example}

\end{document}