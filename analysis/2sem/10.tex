\documentclass[12pt, a4paper]{article}

\usepackage{lastpage}
\usepackage{mathtools}
\usepackage{xltxtra}
\usepackage{libertine}
\usepackage{amsmath}
\usepackage{amsthm}
\usepackage{amsfonts}
\usepackage{amssymb}
\usepackage{enumitem}
\usepackage{xcolor}
\usepackage[left=1.5cm, right=1.5cm, top=2cm, bottom=2cm, bindingoffset=0cm, headheight=15pt]{geometry}
\usepackage{fancyhdr}
\usepackage[russian]{babel}
% \usepackage[utf8]{inputenc}
\usepackage{catchfilebetweentags}
\usepackage{accents}
\usepackage{calc}
\usepackage{etoolbox}
\usepackage{mathrsfs}
\usepackage{wrapfig}

\providetoggle{useproofs}
\settoggle{useproofs}{false}

\pagestyle{fancy}
\lfoot{M3137y2019}
\rhead{\thepage\ из \pageref{LastPage}}

\newcommand{\R}{\mathbb{R}}
\newcommand{\Q}{\mathbb{Q}}
\newcommand{\C}{\mathbb{C}}
\newcommand{\Z}{\mathbb{Z}}
\newcommand{\B}{\mathbb{B}}
\newcommand{\N}{\mathbb{N}}

\newcommand{\const}{\text{const}}

\newcommand{\teormin}{\textcolor{red}{!}\ }

\DeclareMathOperator*{\xor}{\oplus}
\DeclareMathOperator*{\equ}{\sim}
\DeclareMathOperator{\Ln}{\text{Ln}}
\DeclareMathOperator{\sign}{\text{sign}}
\DeclareMathOperator{\Sym}{\text{Sym}}
\DeclareMathOperator{\Asym}{\text{Asym}}
% \DeclareMathOperator{\sh}{\text{sh}}
% \DeclareMathOperator{\tg}{\text{tg}}
% \DeclareMathOperator{\arctg}{\text{arctg}}
% \DeclareMathOperator{\ch}{\text{ch}}

\DeclarePairedDelimiter{\ceil}{\lceil}{\rceil}
\DeclarePairedDelimiter{\abs}{\left\lvert}{\right\rvert}

\setmainfont{Linux Libertine}

\theoremstyle{plain}
\newtheorem{axiom}{Аксиома}
\newtheorem{lemma}{Лемма}

\theoremstyle{remark}
\newtheorem*{remark}{Примечание}
\newtheorem*{exercise}{Упражнение}
\newtheorem*{consequence}{Следствие}
\newtheorem*{example}{Пример}
\newtheorem*{observation}{Наблюдение}

\theoremstyle{definition}
\newtheorem{theorem}{Теорема}
\newtheorem*{definition}{Определение}
\newtheorem*{obozn}{Обозначение}

\setlength{\parindent}{0pt}

\newcommand{\dbltilde}[1]{\accentset{\approx}{#1}}
\newcommand{\intt}{\int\!}

% magical thing that fixes paragraphs
\makeatletter
\patchcmd{\CatchFBT@Fin@l}{\endlinechar\m@ne}{}
  {}{\typeout{Unsuccessful patch!}}
\makeatother

\newcommand{\get}[2]{
    \ExecuteMetaData[#1]{#2}
}

\newcommand{\getproof}[2]{
    \iftoggle{useproofs}{\ExecuteMetaData[#1]{#2proof}}{}
}

\newcommand{\getwithproof}[2]{
    \get{#1}{#2}
    \getproof{#1}{#2}
}

\newcommand{\import}[3]{
    \subsection{#1}
    \getwithproof{#2}{#3}
}

\newcommand{\given}[1]{
    Дано выше. (\ref{#1}, стр. \pageref{#1})
}

\renewcommand{\ker}{\text{Ker }}
\newcommand{\im}{\text{Im }}
\newcommand{\grad}{\text{grad}}

\lhead{Конспект по матанализу}
\cfoot{}
\rfoot{Лекция 10}

\renewcommand{\thesubsection}{\arabic{subsection}.}

\begin{document}

\section*{Несколько классических неравенств}

\subsection{Неравенство Йенсена}

%<*неравенствойенсена>
$f$ --- выпуклая на $\langle a,b \rangle$. Тогда
$$\forall x_1\ldots x_n\in\langle a,b \rangle \ \ \forall \alpha_1\ldots \alpha_n : \alpha_i\geq 0 \ \ \alpha_1+\ldots+\alpha_n=1 \quad f(\alpha_1x_1+\ldots+\alpha_nx_n)\leq \alpha_1f(x_1)+\ldots+\alpha_nf(x_n)$$
%</неравенствойенсена>

%<*неравенствойенсенаproof>
\begin{proof}
    Для $x_1=x_2=\ldots=x_n$ тривиально.
    
    $$\min x_i\leq x^*:=\alpha_1x_1+\ldots+\alpha_nx_n\leq(\alpha_1+\ldots+\alpha_n)\max(x_i)=\max(x_i)$$
    $$\Rightarrow x^*\in \langle a,b \rangle$$
    В $x^*$ можно провести опорную прямую $y=kx+b$
    $$f(x^*)=kx^*+b=\sum_{i=1}^n (\alpha_ikx_i) + b = \sum_{i=1}^n \alpha_i(kx_i + b)\leq\sum_{i=1}^n \alpha_i f(x_i)$$
\end{proof}
%</неравенствойенсенаproof>

\begin{consequence}
    Неравенство Коши.

%<*неравенствокоши>
    $$a_i>0 \quad \frac{1}{n} \sum a_i \geq \sqrt[n]{a_1\cdots a_n}$$
%</неравенствокоши>
\end{consequence}
%<*неравенствокошиproof>
\begin{proof}
    $f(x)=\ln x$ --- вогн., $\alpha_i=\frac{1}{n}$, по неравенству Йенсена:
    $$\ln\left(\frac{1}{n}a_1+\ldots+\frac{1}{n}a_n\right)\ge \frac{1}{n}\ln a_1+\ldots+\frac{1}{n}\ln a_n$$
    $$\ln\left(\frac{a_1+\ldots+a_n}{n}\right)\ge \frac{1}{n}\ln(a_1\cdots a_n)$$
    $$\ln\left(\frac{a_1+\ldots+a_n}{n}\right)\ge \ln(a_1\cdots a_n)^{\frac{1}{n}}$$
    $$\frac{a_1+\ldots+a_n}{n}\ge (a_1\cdots a_n)^{\frac{1}{n}}$$
\end{proof}
%</неравенствокошиproof>

\subsection{Интегральное неравенство Йенсена}
%<*интегральноенеравенствойенсена>
\begin{itemize}
    \item $f$ --- выпуклая на $\langle A,B \rangle$
    \item $\varphi : [a,b]\to\langle A,B \rangle$ --- непрерывная
    \item $\lambda:[a,b]\to[0,+\infty)$ --- непрерывная \textit{(для кусочно-непрерывной тоже верно)}
    \item $\int_a^b \lambda(t)dt=1$
\end{itemize}
Тогда
$$f\left(\int_a^b \lambda(t)\varphi(t)dt \right) \leq \int_a^b \lambda(t)f(\varphi(t))dt$$
%</интегральноенеравенствойенсена>

%<*интегральноенеравенствойенсенаproof>
\begin{proof}
    $m:=\inf \varphi, M:=\sup \varphi$
    $$m\leq m\int_a^b \lambda(t) \leq \int_a^b \lambda(t)\varphi(t)\leq M\int_a^b \lambda(t) = M$$
    $$x^*:=\int_a^b \lambda(t)\varphi(t)dt \Rightarrow x^*\in\langle A,B\rangle$$
    Для $m=M$ тривиально.

    $y=kx+b$ --- опорная прямая в точке $x^*$ графика $f$.
    $$f(x^*)=kx^*+b=k\int_a^b \lambda\varphi + b\int_a^b\lambda=\int_a^b \lambda(t)(k\varphi(t)+b)dt\leq$$
    $$\leq \int_a^b \lambda(t)f(\varphi(t))dt$$
\end{proof}
%</интегральноенеравенствойенсенаproof>

\begin{example}
    $$f>0, f\in C[a,b] \quad \exp\left(\frac{1}{b-a} \int_a^b \ln f(x)dx \right) \leq \frac{1}{b-a}\int_a^b f(x)dx$$
    % Правая часть --- среднее арифметическое $f$ на $[a,b]$ по интегральным суммам.

    % Левая часть:
    % $$\exp\left(\frac{1}{b-a} \int_a^b \ln f(x)dx \right) = \exp\left(\sum \frac{1}{n} \ln f(x_i) \right)=\prod_{i=1}^n \exp\left(\frac{\ln(f(x_i))}{n}\right)$$

    Возьмём логарифм от искомого неравенства:
    $$\frac{1}{b-a}\int_a^b \ln f(x)dx \leq \ln\left(\frac{1}{b-a}\int_a^b f\right)$$

    Подставим в интегральное неравенство Йенсена:
    \begin{itemize}
        \item $f\leftrightarrow \ln$
        \item $\lambda(t)\leftrightarrow\frac{1}{b-a}$
        \item $\varphi\leftrightarrow \textcolor{red}{???}$
    \end{itemize}
\end{example}

\subsection{Неравенство Гёльдера}

%<*неравенствогельдера>
$a_1\ldots a_n, b_1\ldots b_n > 0, p>1, q>1 \ \ \frac{1}{p}+\frac{1}{q}=1$. Тогда
$$\sum_{i=1}^n a_ib_i \leq \left(\sum a_i^p\right)^{\frac{1}{p}}\left(\sum b_i^q\right)\frac{1}{q}$$

Частный случай при $p=q=2$ --- неравенство Коши-Буняковского.
%</неравенствогельдера>

%<*неравенствогельдераproof>
\begin{proof}
    $f(x)=x^p, (p>1)$ --- строго выпуклая $\Rightarrow f''=p(p-1)x^{p-2}>0$
    
    По Йенсену $\left(\sum \alpha_i x_i\right)^p\leq \sum \alpha_i x_i^p$

    $$\text{Левая часть}^{\frac{1}{p}}=\sum\frac{b_i^q}{\sum b_i^q} a_ib_i^{\frac{-1}{p-1}} \sum b^q=\sum a_ib_i$$
    $$\text{Правая часть}=\sum\frac{b_i^q}{\sum b_i^q} a_i^pb_i^{-q}\left(\sum b_j^q\right)^p=\left(\sum a_i^p\right)\left(\sum b_j^q\right)^\frac{p}{q}$$
    $$\text{Правая часть}^{\frac{1}{p}}=\left(\sum a_i^p\right)^{\frac{1}{p}}\left(\sum b_j^q\right)^\frac{1}{q}$$
\end{proof}
%</неравенствогельдераproof>

Общий вид: $a_i, b_i\in\R$
$$\left|\sum a_ib_i\right|\leq \left(\sum |a_i|^p\right)^{\frac{1}{p}}\left(\sum |b_i|^q\right)^{\frac{1}{q}}$$

\subsection{Интегральное неравенство Гёльдера}

%<*интегральноенеравенствогельдера>
$p>1, q>1, \frac{1}{p}+\frac{1}{q}=1 \quad f, g\in C[a,b]$. Тогда
$$\left|\int_a^b f(x)g(x)dx\right|\leq\left(\int_a^b |f|^p\right)^{\frac{1}{p}}\left(\int_a^b |g|^q\right)^{\frac{1}{q}}$$

Неравенство КБШ в пространстве функций --- частный случай этого неравенства.
%</интегральноенеравенствогельдера>
%<*интегральноенеравенствогельдераproof>
\begin{proof}
    По интегральным суммам:
    $$x_i:=a+i\frac{b-a}{n} \quad \Delta x_i = x_i-x_{i-1} \quad a_i:=f(x_i)(\Delta x_i)^{\frac{1}{p}} \quad b_i=g(x_i)(\Delta x_i)^{\frac{1}{q}}$$
    $$a_ib_i=f(x_i)g(x_i)(\Delta x_i)$$
    $$\left|\sum_{i=1}^n f(x_i)g(x_i)\Delta x_i\right| \leq \left(\sum |f(x_i)|^p \Delta x_i\right)^{\frac{1}{p}}\left(\sum |g(x_i)|^q \Delta x_i\right)^{\frac{1}{q}}$$
    Предельный переход доказывает искомое.
\end{proof}
%</интегральноенеравенствогельдераproof>

\subsection{Неравенство Минковского}

%<*неравенствоминковского>
$p\geq 1,\ \ a_i,b_i\in\R$

$$\left(\sum_{i=1}^n |a_i+b_i|^p\right)^{\frac{1}{p}}\leq \left(\sum |a_i|^p\right)^{\frac{1}{p}}+\left(\sum |b_i|^p\right)^{\frac{1}{p}}$$

Это неравенство треугольника для нормы $||a||_p=\left(\sum |a_i|^p\right)^{\frac{1}{p}}$
%</неравенствоминковского>

%<*неравенствоминковскогоproof>
\begin{proof}
    $p=1$ тривиально, $|a_i+b_i|\le |a_i|+|b_i|$

    Докажем для положительных $a_i, b_i$, другие случаи сводятся к этому.

    По неравенству Гёльдера для $q=p/(p-1)$:
    $$\sum a_i(a_i+b_i)^{p-1} \leq \left(\sum a_i^p\right)^\frac{1}{p}\left(\sum (a_i+b_i)^{q(p-1)}\right)^{\frac{1}{q}}=\left(\sum a_i^p\right)^\frac{1}{p}\left(\sum (a_i+b_i)^p\right)^{\frac{1}{q}}$$
    $$\sum b_i(a_i+b_i)^{p-1} \leq \left(\sum b_i^p\right)^\frac{1}{p}\left(\sum (a_i+b_i)^{q(p-1)}\right)^{\frac{1}{q}}=\left(\sum b_i^p\right)^\frac{1}{p}\left(\sum (a_i+b_i)^{p}\right)^{\frac{1}{q}}$$
    Сложим эти два неравенства:
    $$\sum (a_i+b_i)^p\leq \left(\left(\sum a_i^p\right)^{\frac{1}{p}} + \left(\sum b_i^q\right)^{\frac{1}{q}}\right)\left(\sum (a_i+b_i)^p\right)^{\frac{1}{q}}$$
    $$\left(\sum (a_i+b_i)^p\right)^{1-\frac{1}{q}}\leq \left(\sum a_i^p\right)^{\frac{1}{p}} + \left(\sum b_i^q\right)^{\frac{1}{q}}$$
    $$\left(\sum (a_i+b_i)^p\right)^{\frac{1}{p}}\leq \left(\sum a_i^p\right)^{\frac{1}{p}} + \left(\sum b_i^q\right)^{\frac{1}{q}}$$
\end{proof}
%</неравенствоминковскогоproof>

\subsection{Интегральное неравенство Минковского}

$f,g\in C[a,b], p\geq 1$

$$\left(\int_a^b |f+g|^p\right)^{\frac{1}{p}} \leq \left(\int_a^b |f|^p\right)^{\frac{1}{p}} + \left(\int_a^b |g|^p\right)^{\frac{1}{p}}$$

\begin{proof}
    Оставлено как упражнение читателю.
\end{proof}

\begin{theorem}
    Признак Коши

    %<*признаккошиlitepro>
    $a_n\geq 0, K_n:=\sqrt[n]{a_n}$. Тогда:

    Lite:
    \begin{enumerate}
        \item Если $\exists q<1 : K_n\leq q$, начиная с некоторого места \textit{(НСНМ)} ($\exists N : \forall n>N$) $\Rightarrow \sum a_n$ сходится.
        \item $K_n\geq 1$ для бесконечного множества $\Rightarrow$ $\sum a_n$ расходится.
    \end{enumerate}

    Pro: $K:=\overline \lim K_n$
    \begin{enumerate}
        \item $K<1 \Rightarrow \sum a_n$ сходится
        \item $K>1 \Rightarrow \sum a_n$ расходится
    \end{enumerate}
    %</признаккошиlitepro>
\end{theorem}

%<*признаккошиliteproproof>
\begin{proof}
    Lite:
    \begin{enumerate}
        \item НСНМ $\sqrt[n]{a_n} \leq q \Leftrightarrow a_n\leq q^n$, $q_n$ сх. $\Rightarrow \sum a_n$ сх.
        \item $\sqrt[n]{a_n} \geq 1 \Leftrightarrow a_n\geq 1 \Rightarrow a_n\not\to0 \Rightarrow \sum a_n$ расх.
    \end{enumerate}
    Pro:
    \begin{enumerate}
        \item По техническому описанию $\overline \lim$ $\exists N \ \ \forall n > N \ \ K_n<q \Rightarrow$ по Lite.1 сходится.
        \item $l = \overline \lim K_n > 1, 1=l-\varepsilon$. Тогда $K_n\ge 1$ для бесконечного множества $n \Rightarrow$ по Lite.2 расходится.
    \end{enumerate}
\end{proof}
%</признаккошиliteproproof>

\begin{theorem}
    Признак Даламбера.

    %<*признакдаламбера>
    $a_n>0, D_n:=\frac{a_{n+1}}{a_n}$

    Lite:
    \begin{enumerate}
        \item $\exists q < 1$ : $D_n < q$ НСНМ $\Rightarrow \sum a_n$ сх.
        \item $D_n\geq 1$ НСНМ $\Rightarrow \sum a_n$ расх.
    \end{enumerate}
    Pro: $D:=\lim D_n$
    \begin{enumerate}
        \item $D < 1 \Rightarrow \sum a_n$ сх.
        \item $D > 1 \Rightarrow \sum a_n$ расх.
    \end{enumerate}
    %</признакдаламбера>
\end{theorem}

%<*признакдаламбераproof>
\begin{proof}
    Lite:
    \begin{enumerate}
        \item $\exists N : \frac{a_{N+1}}{a_N} < q, \frac{a_{N+2}}{a_{N+1}} < q, \ldots$
        $$\frac{a_n}{a_N} < q^{n-N}$$
        $$a_n < q^n\left(\frac{a_N}{q^N}\right)$$
        $\sum q^n$ сх. $\Rightarrow \sum a_n$ сх.
        \item $D_n \ge 1 \Leftrightarrow a_{n+1}\ge a_n$, при $n>N$ $a_n\ge a_N \Rightarrow a_n\ge A_N \Rightarrow a_n\not\to0$. Также можно аналогично пункту 1.
    \end{enumerate}
    Pro:
    \begin{enumerate}
        \item $q:=\frac{1+D}{2}$. По определению предела $\varepsilon:=q-D \ \ \exists N \ \ \forall n>N \ \ D_n<q \xRightarrow{Lite1} \sum a_n$ сх.
        \item $\varepsilon:=D-1 \ \ \exists N \ \ \forall n > N \ \ D_n>1\xRightarrow{Lite2} \sum a_n$ расх.
    \end{enumerate}
\end{proof}
%</признакдаламбераproof>

\begin{lemma}
    $a_n, b_n > 0$ $\frac{a_{n+1}}{a_n} \le \frac{b_{n+1}}{b_n}$ НСНМ.

    Тогда:
    \begin{enumerate}
        \item $\sum b_n$ сх. $\Rightarrow \sum a_n$ сх.
        \item $\sum a_n$ расх. $\Rightarrow \sum b_n$ расх.
    \end{enumerate}
\end{lemma}
\begin{proof}
    Будем игнорировать ``НСНМ''

    $$\frac{a_2}{a_1}\leq\frac{b_2}{b_1} \quad \frac{a_3}{a_2} \leq \frac{b_3}{b_2} \quad \ldots$$
    $$a_n\leq b_n \frac{a_1}{b_1}$$
    По признаку сравнения все работает.
\end{proof}

\begin{theorem}
    Признак Раабе

    %<*признакраабе>
    $a_n > 0, R_n:=n\left( \frac{a_n}{a_{n+1}} - 1 \right)$. Тогда:
    \begin{enumerate}
        \item $\exists r > 1 \ \ R_n\geq r$ НСНМ $\Rightarrow \sum a_n$ сх.
        \item $R_n\leq 1$ НСНМ $\Rightarrow \sum a_n$ сх.
    \end{enumerate}
    %</признакраабе>
\end{theorem}

\textcolor{red}{Еще следствие}

%<*признакраабеproof>
\begin{proof}
    \begin{enumerate}
        \item $R_n\geq r\Leftrightarrow \frac{a_n}{a_{n+1}}\ge 1 + \frac{r}{n}$
        $$\frac{\left(1+\frac{1}{n}\right)^s - 1}{\frac{1}{n}}<r \Leftrightarrow \left(1+\frac{1}{n}\right)^s < 1 + \frac{r}{n}$$
        $$b_n:=\frac{1}{n^s} \quad \frac{a_{n+1}}{a_n}=\frac{1}{\frac{a_n}{a_{n+1}}}\leq \frac{1}{1+\frac{r}{n}} < \frac{1}{\left(1+\frac{1}{n}\right)^s}=\frac{b_{n+1}}{b_n}$$
        $\sum b_n$ сх. $\Rightarrow \sum a_n$ сх. по лемме 1.
        \item $R_n \le 1 \Leftrightarrow \frac{a_n}{a_{n+1}} \le 1 + \frac{1}{n}$
        $$\frac{a_{n+1}}{a_n}\geq \frac{n}{n+1}=\frac{\frac{1}{n+1}}{\frac{1}{n}}$$
        $b_n=\frac{1}{n}$ расх. $\Rightarrow \sum a_n$ расх.
    \end{enumerate}
\end{proof}
%</признакраабеproof>

\end{document}