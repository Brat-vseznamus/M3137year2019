\documentclass[12pt, a4paper]{article}

\usepackage{lastpage}
\usepackage{mathtools}
\usepackage{xltxtra}
\usepackage{libertine}
\usepackage{amsmath}
\usepackage{amsthm}
\usepackage{amsfonts}
\usepackage{amssymb}
\usepackage{enumitem}
\usepackage{xcolor}
\usepackage[left=1.5cm, right=1.5cm, top=2cm, bottom=2cm, bindingoffset=0cm, headheight=15pt]{geometry}
\usepackage{fancyhdr}
\usepackage[russian]{babel}
% \usepackage[utf8]{inputenc}
\usepackage{catchfilebetweentags}
\usepackage{accents}
\usepackage{calc}
\usepackage{etoolbox}
\usepackage{mathrsfs}
\usepackage{wrapfig}

\providetoggle{useproofs}
\settoggle{useproofs}{false}

\pagestyle{fancy}
\lfoot{M3137y2019}
\rhead{\thepage\ из \pageref{LastPage}}

\newcommand{\R}{\mathbb{R}}
\newcommand{\Q}{\mathbb{Q}}
\newcommand{\C}{\mathbb{C}}
\newcommand{\Z}{\mathbb{Z}}
\newcommand{\B}{\mathbb{B}}
\newcommand{\N}{\mathbb{N}}

\newcommand{\const}{\text{const}}

\newcommand{\teormin}{\textcolor{red}{!}\ }

\DeclareMathOperator*{\xor}{\oplus}
\DeclareMathOperator*{\equ}{\sim}
\DeclareMathOperator{\Ln}{\text{Ln}}
\DeclareMathOperator{\sign}{\text{sign}}
\DeclareMathOperator{\Sym}{\text{Sym}}
\DeclareMathOperator{\Asym}{\text{Asym}}
% \DeclareMathOperator{\sh}{\text{sh}}
% \DeclareMathOperator{\tg}{\text{tg}}
% \DeclareMathOperator{\arctg}{\text{arctg}}
% \DeclareMathOperator{\ch}{\text{ch}}

\DeclarePairedDelimiter{\ceil}{\lceil}{\rceil}
\DeclarePairedDelimiter{\abs}{\left\lvert}{\right\rvert}

\setmainfont{Linux Libertine}

\theoremstyle{plain}
\newtheorem{axiom}{Аксиома}
\newtheorem{lemma}{Лемма}

\theoremstyle{remark}
\newtheorem*{remark}{Примечание}
\newtheorem*{exercise}{Упражнение}
\newtheorem*{consequence}{Следствие}
\newtheorem*{example}{Пример}
\newtheorem*{observation}{Наблюдение}

\theoremstyle{definition}
\newtheorem{theorem}{Теорема}
\newtheorem*{definition}{Определение}
\newtheorem*{obozn}{Обозначение}

\setlength{\parindent}{0pt}

\newcommand{\dbltilde}[1]{\accentset{\approx}{#1}}
\newcommand{\intt}{\int\!}

% magical thing that fixes paragraphs
\makeatletter
\patchcmd{\CatchFBT@Fin@l}{\endlinechar\m@ne}{}
  {}{\typeout{Unsuccessful patch!}}
\makeatother

\newcommand{\get}[2]{
    \ExecuteMetaData[#1]{#2}
}

\newcommand{\getproof}[2]{
    \iftoggle{useproofs}{\ExecuteMetaData[#1]{#2proof}}{}
}

\newcommand{\getwithproof}[2]{
    \get{#1}{#2}
    \getproof{#1}{#2}
}

\newcommand{\import}[3]{
    \subsection{#1}
    \getwithproof{#2}{#3}
}

\newcommand{\given}[1]{
    Дано выше. (\ref{#1}, стр. \pageref{#1})
}

\renewcommand{\ker}{\text{Ker }}
\newcommand{\im}{\text{Im }}
\newcommand{\grad}{\text{grad}}

\lhead{Конспект по матанализу}
\cfoot{}
\rfoot{Лекция 8}

\begin{document}

\begin{definition}
    %<*абсолютносходящийсяинтеграл>
    $f$ --- допустимая функция на $[a, b)$

    $\int_a^b f$ --- \textbf{абсолютно сходится}, если:
    \begin{enumerate}
        \item $\int_a^b f$ сходится
        \item $\int_a^b |f|$ --- сходится
    \end{enumerate}
    %</абсолютносходящийсяинтеграл>
\end{definition}

\begin{theorem}
    %<*абсолютносходящийсяинтегралтеорема>
    $f$ --- доп. на $[a, b)$. Тогда эквивалентны следующие утверждения:
    \begin{enumerate}
        \item $\int_a^b f$ абсолютно сходится
        \item $\int_a^b |f|$ сходится
        \item $\int_a^b f^+, \int_a^b f^-$ оба сходятся
    \end{enumerate}
    \begin{remark}
        $f^+ = \max(f, 0), f^- = \max(-f, 0)$
    \end{remark}
    %</абсолютносходящийсяинтегралтеорема>
\end{theorem}
%<*абсолютносходящийсяинтегралтеоремаproof>
\begin{proof}
    1$\Rightarrow$2 --- тривиально

    2$\Rightarrow$3 : $0\leq f^{\pm}\leq |f|$

    3$\Rightarrow$1 : $f=f^+ - f^- \Rightarrow \int_a^b f = \int_a^b f^+ - \int_a^b f^-$
\end{proof}
%</абсолютносходящийсяинтегралтеоремаproof>

\begin{example}
    $$\int_{10}^{+\infty} \frac{\sin x}{x}dx \stackrel{\text{по частям}}{=} \left[\begin{array}{lr}
        u = \frac{1}{x} & du = -\frac{1}{x^2} dx \\
        dv = \sin x dx & v = -\cos x
    \end{array}\right]=-\cos \frac{1}{x} \bigg|_{10}^{+\infty} - \int_{10}^{+\infty} \frac{\cos x}{x^2} dx$$
    Также можно было оставить нижнюю границу $0$, но использовать $v=1-\cos x$

    Первое слагаемое очевидно конечно, а второе конечно по абсолютной сходимости: $\left|\frac{\cos x}{x^2}\right|\leq \frac{1}{x}$. Тогда искомый интеграл сходится.
\end{example}

\begin{example}
    %<*интегралsinxxp>
    $$\int_1^{+\infty} \frac{\sin x}{x^p} dx$$
    \begin{itemize}
        \item При каких $p$ сходится?
        \item При каких $p$ абсолютно сходится?
    \end{itemize}
    \begin{enumerate}
        \item $p > 1 \Rightarrow$ абсолютно сходится, т.к. $\left|\frac{\sin x}{x^p}\right|<\frac{1}{x^{p-1}}$
        \item $p > 0 \Rightarrow$ сходится, т.к. (по частям):
        $$\int_1^{+\infty} \frac{\sin x}{x^p} = -\frac{\cos x}{x^p}\bigg|_1^{+\infty} - p \int_1^{+\infty} \frac{\cos x}{x^{p+1}}$$
        Первое конечно, второе абсолютно сходится.
        \item $p \leq 0$, по критерию Коши:
        $$\exists A_n, B_n \to b \quad \int_{A_n}^{B_n} f\not\to0 \Rightarrow \int_a^b f \text{ расходится}$$
        $$A_n:=2\pi n, B_n:=2\pi n+\pi \quad \int_{A_n}^{B_n} \frac{\sin x}{x^p} dx \geq (2\pi n)^{-p}\int_{A_n}^{B_n} \sin x \text{ расходится}$$
        Итого для $p\leq 0$ расходится.
        \item $0<p\leq 1$, абсолютная сходимость?
        $$\int_1^{+\infty} \frac{|\sin x|}{x^p}$$
        \begin{enumerate}
            \item Первый способ. $A_n:=\pi n, B_n:=2\pi n$
            $$\int_{A_n}^{B_n}\frac{|\sin x|}{x^p} \ge \frac{1}{(2\pi n)^p}\underbrace{\int_{A_n}^{B_n}|\sin x|}_{\text{площадь } n \text{ арок синуса}}=\frac{2n}{(2\pi n)^p}=Cn^{1-p}\not\to0$$
            \item Второй способ.
            $$\int_1^{+\infty} \frac{|\sin x|}{x^p} \ge \int_1^{+\infty} \frac{\sin^2 x}{x^p} = \int_1^{+\infty} \frac{1-\cos(2x)}{2x^p} = \underbrace{\int_1^{+\infty} \frac{1}{2x^p}}_{+\infty} - \underbrace{\int_1^{+\infty} \frac{\cos 2x}{2x^p}}_{\substack{\text{При } p>0 \text{ сходится} \\ \text{ как в пункте 2}}}$$
        \end{enumerate}
        Итого абсолютной сходимости нет.
    \end{enumerate}
    %</интегралsinxxp>
\end{example}

\begin{remark}
    \begin{enumerate}
        \item $\int_a^b f$ --- сходится $\not\Rightarrow f(x)\xrightarrow{x\to b-0} 0$
        $$\int_1^{+\infty} x\sin x^3 dx = \left[\begin{array}{rl}
            t = x^3 &\\
            x = \sqrt[3]{t} & dx = \frac{1}{3}t^{-2/3}dt
        \end{array}\right] = \frac{1}{3} \int_1^{+\infty} t^{-1/3} \sin t dt = \frac{1}{3} \int_1^{+\infty} \frac{\sin t}{t^{1/3}} dt$$
        Этот интеграл сходится, но $f(x)\not\to0$
        \item $\int_a^b f$ --- абсолютно сходится $\not\Rightarrow f(x)\xrightarrow{x\to b-0} 0$
    \end{enumerate}
\end{remark}
\begin{exercise}
    $\int_1^{+\infty} x\sin x^3 dx$ не сходится абсолютно
\end{exercise}

\begin{theorem}
    Признак Абеля-Дирихле.

    %<*признакабелядирихле>
    $f$ --- допустима на $[a, b)$, $g\in C^1 [a, b)$

    Если выполняется 1 или 2, то $\int_a^b fg$ --- сходится
    \begin{enumerate}
        \item \begin{enumerate}
            \item $F(A) := \int_a^A f(x)dx , A\in [a, b), F$ ограничена, т.е.:
            $$\exists K : \forall A\in [a, b) \quad \left|\int_a^A f\right|\le K$$
            \item $g(x)$ монотонна, $g(x)\xrightarrow{x\to b-0} 0$
        \end{enumerate}
        \item \begin{enumerate}
            \item $\int_a^b f(x)dx$ сходится, необязательно абсолютно
            \item $g(x)$ монотонна, $g(x)$ ограничена, т.е.: $\exists L \ \ \forall x \in [a, b) \quad |g(x)| \le L$
        \end{enumerate}
    \end{enumerate}
    1 часть --- Дирихле, 2 --- Абель.
    %</признакабелядирихле>
\end{theorem}
%<*признакабелядирихлеproof>
\begin{proof}
    \begin{enumerate}
        \item $$\int_a^b fg = F(x)g(x)\Big|_a^b - \int_a^b F(x)g'(x)dx$$
        $$\lim_{x\to b - 0} \underbrace{F(x)}_{\text{огр.}}\underbrace{g(x)}_{\text{б.м.}}=0 \Rightarrow F(x)g(x)\Big|_a^b \text{ --- конечн.}$$
        Покажем абсолютную сходимость, из нее следует обычная сходимость:
        $$\int_a^b |F(x)g'(x)|dx \le \int_a^b K\int_a^b |g'|=$$
        Можно снять модуль, т.к. $g$ монотонна $\Rightarrow$ $\sign(g')=\const$
        $$=\pm K \int_a^b g'=\pm K g(x)\Big|_a^b=\pm K(\underbrace{\lim_{x\to b - 0}g(x)}_{0} - \underbrace{g(a)}_{\text{кон.}})$$
        \item $\alpha:=\lim\limits_{x\to b-0} g(x)$ --- кон.
        $$\int_a^b fg = \underbrace{\int_a^b f\alpha}_{\text{кон. по а}} + \underbrace{\int_a^b f(g-\alpha)}_{\text{сс ходится по 1}}$$
        Пояснение насчет сходимости $\int_a^b f(g-\alpha)$:
        \begin{enumerate}
            \item $F : A \mapsto \int_a^A f$ --- ограничена, т.к. $\int_a^b f$ сходится
            \item $g\to\alpha \Rightarrow (g-\alpha)\to0$ 
        \end{enumerate}
    \end{enumerate}
\end{proof}
%</признакабелядирихлеproof>

\begin{exercise}
    $$\int_{10}^{+\infty} \sin(x^3-x) dx = \int_{10}^{+\infty} \underbrace{(3x^2-1)\sin(x^3-x)}_{f}\underbrace{\frac{1}{3x^2-1}}_{g}dx$$
    Сходится по признаку Дирихле.
    
    \textcolor{red}{Дальше в лекции была проверка на абсолютную сходимость.}
\end{exercise}

\begin{example}
    Интеграл Дирихле.

    %<*интегралдирихле>
    $$\int_0^{+\infty} \frac{\sin x}{x}dx = \frac{\pi}{2}$$
    %</интегралдирихле>
\end{example}
%<*интегралдирихлеproof>
\begin{proof}
    $$\cos x + \cos 2x + \ldots + \cos nx = \frac{\sin\left(n+\frac{1}{2}\right)x}{2\sin \frac{x}{2}}-\frac{1}{2}$$
    $$2\sin \frac{x}{2}\cos x + 2\sin \frac{x}{2}\cos 2x + \ldots + 2\sin \frac{x}{2}\cos nx = \sin\left(n+\frac{1}{2}\right)x -\frac{1}{2}$$
    $$\sin\frac{3}{2}x-\sin\frac{1}{2}x+\ldots+\sin\left(n+\frac{1}{2}\right)x - \sin\left(n-\frac{1}{2}\right)x = \sin\left(n+\frac{1}{2}\right)x -\frac{1}{2}$$
    $$\int_0^\pi \cos kx = \frac{1}{k}\sin kx\Big|_0^\pi=0$$
    Проинтегрируем исходное выражение по $[0, \pi]$:
    $$0=\int_0^\pi\ldots=\int_0^\pi \frac{\sin\left(n+\frac{1}{2}\right)x}{2\sin\frac{x}{2}}dx-\frac{\pi}{2}$$
    $$\int_0^\pi \frac{\sin\left(n+\frac{1}{2}\right)x}{2\sin\frac{x}{2}}dx=\frac{\pi}{2}$$
    $$\int_0^\pi \frac{\sin\left(n+\frac{1}{2}\right)x}{\sin x}dx = \left[\begin{array}{ll}
        y = \left(n+\frac{1}{2}\right)x &\\
        x = \frac{1}{n+\frac{1}{2}} & dx = \frac{1}{n+\frac{1}{2}}dy
    \end{array}\right] = $$
    $$ = \int_0^{\left(n+\frac{1}{2}\right)\pi} \frac{\sin y}{\frac{1}{n+\frac{1}{2}}y}\frac{1}{n+\frac{1}{2}}dy = \int_0^{\left(n+\frac{1}{2}\right)\pi} \frac{\sin y}{y}dy$$
    Итого:
    $$\int_0^\pi \frac{\sin\left(n+\frac{1}{2}\right)x}{x}dx = \int_0^{\left(n+\frac{1}{2}\right)\pi} \frac{\sin y}{y}dy$$
    Проверим:
    $$\int_0^\pi \frac{\sin\left(n+\frac{1}{2}\right)x}{2\sin\frac{x}{2}}dx -  \int_0^\pi \frac{\sin\left(n+\frac{1}{2}\right)x}{x}dx \xrightarrow[n\to+\infty]{} 0$$

    $$\int_0^\pi \sin\left(n+\frac{1}{2}\right)x h(x)dx$$
    $$h(x)=\frac{1}{2\sin\frac{x}{2}}-\frac{1}{x}=\frac{x-2\sin\frac{x}{2}}{2x\sin\frac{x}{2}}=\frac{\mathcal O(x^3)}{x^2}\xrightarrow[x\to0]{}0$$
    $$\int_0^\pi \sin\left(n+\frac{1}{2}\right)x h(x)dx = \left[\begin{array}{ll}
        f = h(x) \\
        g' = \sin\left(n+\frac{1}{2}\right)x
    \end{array}\right]=$$
    $$=\frac{-1}{n+\frac{1}{2}}\cos\left(n+\frac{1}{2}\right)x h(x)\Big|_0^\pi + \frac{1}{n+\frac{1}{2}}\int_0^\pi \cos\left(n+\frac{1}{2}\right) h'(x)dx$$
    $$h'(x) = -\frac{\cos\frac{x}{2}}{4\sin^2\frac{x}{2}}+\frac{1}{x^2}=\frac{x^2\cos\frac{x}{2} - 4\sin^2 \frac{x}{2}}{4x^2\sin^2\frac{x}{2}} = \frac{x^2\left(1-\frac{x^2}{4}+o(x^3)\right) - 4\left(\frac{x}{2}-\frac{x^3}{48}+o(x^3)\right)}{4x^2\sin^2\frac{x}{2}}\xrightarrow[x\to0]{} \const$$
    $\Rightarrow h'(0)=\const$ (той, которая $\lim$) и $h\in C^1[0,\pi]$

    $$\int_0^\pi \sin\left(n+\frac{1}{2}\right)x h(x)dx=\underbrace{\frac{-1}{n+\frac{1}{2}}\underbrace{\cos\left(n+\frac{1}{2}\right)x}_{\text{огр.}} \underbrace{h(x)}_{\text{огр.}}}_{\to0}\Big|_0^\pi + \underbrace{\frac{1}{n+\frac{1}{2}}\underbrace{\int_0^\pi \underbrace{\cos\left(n+\frac{1}{2}\right) \underbrace{h'(x)}_{\text{огр., т.к.}\in C^1}}_{\text{огр., непр.}}dx}_{\text{огр. как ф-ция от } n}}_{\to0}$$

    $$\underbrace{\int_0^{\left(n+\frac{1}{2}\right)\pi} \frac{\sin x}{x}dx}_{\to\text{инт. Дирихле}} = \underbrace{\int_0^\pi \frac{\sin\left(n+\frac{1}{2}\right)x}{x}dx - \int_0^\pi \frac{\sin\left(n+\frac{1}{2}\right)x}{2\sin\frac{x}{2}}dx}_{\to0} + \int_0^\pi \frac{\sin\left(n+\frac{1}{2}\right)x}{2\sin\frac{x}{2}}dx\to\frac{\pi}{2}$$
\end{proof}
%</интегралдирихлеproof>

\subsection*{Верхний предел и нижний предел последовательности}

\begin{definition}
    %<*частичныйпредел>
    \textbf{Частичный предел} вещественной последовательности $x_n$ --- предел вдоль подпоследовательности $n_k$:
    $$n_k\to+\infty, n_1<n_2<\ldots \quad \lim x_{n_k}\in\overline\R$$
    %</частичныйпредел>
\end{definition}

\begin{example}
    $x_n=(-1)^n, n_i=2i$
\end{example}

\begin{definition}
    Дана последовательность $x_n$.
    \begin{itemize}
        \item $y_n:=\sup(x_n, x_{n+1}, x_{n+2},\ldots)$
        \item $z_n:=\inf(x_n, x_{n+1}, x_{n+2},\ldots)$
    \end{itemize}
\end{definition}

\begin{remark}
    \begin{enumerate}
        \item $y_n\downarrow, z_n\uparrow$
        \item $z_n\le x_n\le y_n$
        \item Если изменить конечное число элементов $x_n$, то изменится конечное число элементов $y_n, z_n$
    \end{enumerate}
\end{remark}

\begin{example}
    \begin{enumerate}
        \item $x_n=(-1)^n, y_n\equiv1, z_n\equiv-1$
        \item $x_n=(1+(-1)^n)n, y_n\equiv+\infty, z_n\equiv0$
    \end{enumerate}
\end{example}

% \begin{definition}
    %<*верхнийнижнийпредел>
    \begin{itemize}
        \item \textbf{Верхний предел} $x_n$: $\overline{\lim\limits_{n\to+\infty}} x_n := \lim\limits_{n\to+\infty}y_n$
        \item \textbf{Нижний предел} $x_n$: $\underline{\lim\limits_{n\to+\infty}} x_n := \lim\limits_{n\to+\infty}z_n$
    \end{itemize}
    %</верхнийнижнийпредел>
    Верхний и нижний пределы всегда существуют.
% \end{definition}

\begin{theorem}
    Свойства верхнего и нижнего пределов
    %<*свойстваверхнегоинижнегопределов>
    \begin{enumerate}
        \item $\underline{\lim} x_n \le \overline{\lim} x_n$
        \item $\forall n \ \ x_n \le \tilde x_n \Rightarrow$:
        \begin{enumerate} 
            \item $\overline \lim x_n\le \overline \lim \tilde x_n$
            \item $\underline \lim x_n\le \underline \lim \tilde x_n$
        \end{enumerate}
        \item $\lambda \ge 0 \Rightarrow \overline \lim (\lambda x_n) = \lambda \overline \lim x_n; \underline \lim \lambda x_n = \lambda \underline \lim x_n$, считаем что $0\cdot(\pm\infty)=0$
        \item $\overline \lim -x_n = - \underline \lim x_n ; \underline \lim -x_n = - \overline \lim x_n$
        \item $\overline \lim (x_n + y_n) \le \overline \lim x_n + \overline \lim y_n$, если правая часть имеет смысл, т.е. нет ситуации вида $+\infty - \infty$
        
        $\underline \lim (x_n + y_n) \le \underline \lim x_n + \underline \lim y_n$

        \item $t_n\to l\in\R \Rightarrow \overline \lim (x_n + t_n) = \overline \lim x_n + l$
        \item $t_n\to l\in(0,+\infty) \Rightarrow \overline \lim(t_nx_n) = l \overline \lim x_n$
    \end{enumerate}
    %</свойстваверхнегоинижнегопределов>
\end{theorem}
%<*свойстваверхнегоинижнегопределовproof>
\begin{proof}
    \begin{enumerate}
        \item $y_n \le x_n \le z_n$, по предельному переходу тривиально.
        \item $z_n=\sup(x_n, x_{n+1}, \ldots), \tilde z_n=\sup(\tilde x_n, \tilde x_{n+1}\ldots) \Rightarrow z_n\le \tilde z_n$
        \item $\sup \lambda E = \lambda\sup E$
        \item $\sup - E = -\inf E$
        \item $\sup(x_n+y_n, x_{n+1}+y_{n+1}, \ldots) \le \sup(x_n, x_{n+1}\ldots) + \sup(y_n, y_{n+1}\ldots)$
        \item $\forall \varepsilon > 0 \ \ \exists N_0 \ \ \forall k > N_0 \quad x_k+l-\varepsilon < x_k + t_k < x_k + l + \varepsilon$
        
        $\sphericalangle N>N_0$, перейдем к $\sup$ по $k\ge N$:
        $$y_N + l - \varepsilon < \sup(x_N + t_N, x_{N+1} + t_{N+1}, \ldots)\le y_N + l + \varepsilon$$
        Предельный переход:
        $$\overline \lim x_N + l - \varepsilon \le \overline \sup (x_N + t_N) \le \overline \lim x_N + l + \varepsilon$$
        $$\lim (x_n + t_n) = \overline \lim x_n + l$$
        \item То же самое.
    \end{enumerate}
\end{proof}
%</свойстваверхнегоинижнегопределовproof>

\end{document}