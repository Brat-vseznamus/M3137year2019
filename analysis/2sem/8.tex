\documentclass[12pt, a4paper]{article}

\usepackage{lastpage}
\usepackage{mathtools}
\usepackage{xltxtra}
\usepackage{libertine}
\usepackage{amsmath}
\usepackage{amsthm}
\usepackage{amsfonts}
\usepackage{amssymb}
\usepackage{enumitem}
\usepackage{xcolor}
\usepackage[left=1.5cm, right=1.5cm, top=2cm, bottom=2cm, bindingoffset=0cm, headheight=15pt]{geometry}
\usepackage{fancyhdr}
\usepackage[russian]{babel}
% \usepackage[utf8]{inputenc}
\usepackage{catchfilebetweentags}
\usepackage{accents}
\usepackage{calc}
\usepackage{etoolbox}
\usepackage{mathrsfs}
\usepackage{wrapfig}

\providetoggle{useproofs}
\settoggle{useproofs}{false}

\pagestyle{fancy}
\lfoot{M3137y2019}
\rhead{\thepage\ из \pageref{LastPage}}

\newcommand{\R}{\mathbb{R}}
\newcommand{\Q}{\mathbb{Q}}
\newcommand{\C}{\mathbb{C}}
\newcommand{\Z}{\mathbb{Z}}
\newcommand{\B}{\mathbb{B}}
\newcommand{\N}{\mathbb{N}}

\newcommand{\const}{\text{const}}

\newcommand{\teormin}{\textcolor{red}{!}\ }

\DeclareMathOperator*{\xor}{\oplus}
\DeclareMathOperator*{\equ}{\sim}
\DeclareMathOperator{\Ln}{\text{Ln}}
\DeclareMathOperator{\sign}{\text{sign}}
\DeclareMathOperator{\Sym}{\text{Sym}}
\DeclareMathOperator{\Asym}{\text{Asym}}
% \DeclareMathOperator{\sh}{\text{sh}}
% \DeclareMathOperator{\tg}{\text{tg}}
% \DeclareMathOperator{\arctg}{\text{arctg}}
% \DeclareMathOperator{\ch}{\text{ch}}

\DeclarePairedDelimiter{\ceil}{\lceil}{\rceil}
\DeclarePairedDelimiter{\abs}{\left\lvert}{\right\rvert}

\setmainfont{Linux Libertine}

\theoremstyle{plain}
\newtheorem{axiom}{Аксиома}
\newtheorem{lemma}{Лемма}

\theoremstyle{remark}
\newtheorem*{remark}{Примечание}
\newtheorem*{exercise}{Упражнение}
\newtheorem*{consequence}{Следствие}
\newtheorem*{example}{Пример}
\newtheorem*{observation}{Наблюдение}

\theoremstyle{definition}
\newtheorem{theorem}{Теорема}
\newtheorem*{definition}{Определение}
\newtheorem*{obozn}{Обозначение}

\setlength{\parindent}{0pt}

\newcommand{\dbltilde}[1]{\accentset{\approx}{#1}}
\newcommand{\intt}{\int\!}

% magical thing that fixes paragraphs
\makeatletter
\patchcmd{\CatchFBT@Fin@l}{\endlinechar\m@ne}{}
  {}{\typeout{Unsuccessful patch!}}
\makeatother

\newcommand{\get}[2]{
    \ExecuteMetaData[#1]{#2}
}

\newcommand{\getproof}[2]{
    \iftoggle{useproofs}{\ExecuteMetaData[#1]{#2proof}}{}
}

\newcommand{\getwithproof}[2]{
    \get{#1}{#2}
    \getproof{#1}{#2}
}

\newcommand{\import}[3]{
    \subsection{#1}
    \getwithproof{#2}{#3}
}

\newcommand{\given}[1]{
    Дано выше. (\ref{#1}, стр. \pageref{#1})
}

\renewcommand{\ker}{\text{Ker }}
\newcommand{\im}{\text{Im }}
\newcommand{\grad}{\text{grad}}

\lhead{Конспект по матанализу}
\cfoot{}
\rfoot{Лекция 8}

\begin{document}

\begin{definition}
    $f$ --- допустимая функция на $[a, b)$

    $\int_a^b f$ --- \textbf{абсолютно сходится}, если:
    \begin{enumerate}
        \item $\int_a^b f$ сходится
        \item $\int_a^b |f|$ --- сходится
    \end{enumerate}
\end{definition}

\begin{theorem}
    $f$ --- доп. на $[a, b)$. Тогда эквивалентны следующие утверждения:
    \begin{enumerate}
        \item $\int_a^b f$ сходится
        \item $\int_a^b |f|$ сходится
        \item $\int_a^b f^+, \int_a^b f^-$ оба сходятся
    \end{enumerate}
\end{theorem}
\begin{proof}
    1$\Rightarrow$2 --- тривиально

    2$\Rightarrow$3 : $0\leq f^{\pm}\leq |f|$

    3$\Rightarrow$1 : $f=f^+ - f^- \Rightarrow \int_a^b f = \int_a^b f^+ - \int_a^b f^-$
\end{proof}

\begin{example}
    $$\int_{10}^{+\infty} \frac{\sin x}{x}dx \stackrel{\text{по частям}}{=} \left[\begin{array}{lr}
        u = \frac{1}{x} & du = -\frac{1}{x^2} dx \\
        dv = \sin x dx & v = -\cos x
    \end{array}\right]=-\cos \frac{1}{x} \bigg|_{10}^{+\infty} - \int_{10}^{+\infty} \frac{\cos x}{x^2} dx$$
    Также можно было оставить нижнюю границу $0$, но использовать $v=1-\cos x$

    Первое слагаемое очевидно конечно, а второе конечно по абсолютной сходимости: $\left|\frac{\cos x}{x^2}\right|\leq \frac{1}{x}$. Тогда искомый интеграл сходится.
\end{example}

\begin{example}
    $$\int_1^{+\infty} \frac{\sin x}{x^p}$$
    \begin{itemize}
        \item При каких $p$ сходится?
        \item При каких $p$ абсолютно сходится?
    \end{itemize}
    \begin{enumerate}
        \item $p > 1 \Rightarrow$ абсолютно сходится, т.к. $\left|\frac{\sin x}{x^p}\right|<\frac{1}{x^{p-1}}$
        \item $p > 0 \Rightarrow$ сходится, т.к. (по частям):
        $$\int_1^{+\infty} \frac{\sin x}{x^p} = -\frac{\cos x}{x^p}\bigg|_1^{+\infty} - p \int_1^{+\infty} \frac{\cos x}{x^{p+1}}$$
        Первое конечно, второе абсолютно сходится.
        \item $p \leq 0$, по критерию Коши:
        $$\exists A_n, B_n \to b \quad \int_{A_n}^{B_n} f\not\to0 \Rightarrow \int_a^b f \text{ расходится}$$
        $$A_n:=2\pi n, B_n:=2\pi n+\pi \quad \int_{A_n}^{B_n} \frac{\sin x}{x^p} dx \geq (2\pi n)^{-p}\int_{A_n}^{B_n} \sin x \text{ расходится}$$
        Итого для $p\leq 0$ расходится.
        \item $0<p\leq 1$
    \end{enumerate}
\end{example}

\section*{\textcolor{red}{Скипнуто до конца лекции}}

\begin{example}
    Интеграл Дирихле.

    $$\int_0^{+\infty} \frac{\sin x}{x}dx = \frac{\pi}{2}$$
\end{example}

\end{document}