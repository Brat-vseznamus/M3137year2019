\documentclass[12pt, a4paper]{article}

\usepackage{lastpage}
\usepackage{mathtools}
\usepackage{xltxtra}
\usepackage{libertine}
\usepackage{amsmath}
\usepackage{amsthm}
\usepackage{amsfonts}
\usepackage{amssymb}
\usepackage{enumitem}
\usepackage{xcolor}
\usepackage[left=1.5cm, right=1.5cm, top=2cm, bottom=2cm, bindingoffset=0cm, headheight=15pt]{geometry}
\usepackage{fancyhdr}
\usepackage[russian]{babel}
% \usepackage[utf8]{inputenc}
\usepackage{catchfilebetweentags}
\usepackage{accents}
\usepackage{calc}
\usepackage{etoolbox}
\usepackage{mathrsfs}
\usepackage{wrapfig}

\providetoggle{useproofs}
\settoggle{useproofs}{false}

\pagestyle{fancy}
\lfoot{M3137y2019}
\rhead{\thepage\ из \pageref{LastPage}}

\newcommand{\R}{\mathbb{R}}
\newcommand{\Q}{\mathbb{Q}}
\newcommand{\C}{\mathbb{C}}
\newcommand{\Z}{\mathbb{Z}}
\newcommand{\B}{\mathbb{B}}
\newcommand{\N}{\mathbb{N}}

\newcommand{\const}{\text{const}}

\newcommand{\teormin}{\textcolor{red}{!}\ }

\DeclareMathOperator*{\xor}{\oplus}
\DeclareMathOperator*{\equ}{\sim}
\DeclareMathOperator{\Ln}{\text{Ln}}
\DeclareMathOperator{\sign}{\text{sign}}
\DeclareMathOperator{\Sym}{\text{Sym}}
\DeclareMathOperator{\Asym}{\text{Asym}}
% \DeclareMathOperator{\sh}{\text{sh}}
% \DeclareMathOperator{\tg}{\text{tg}}
% \DeclareMathOperator{\arctg}{\text{arctg}}
% \DeclareMathOperator{\ch}{\text{ch}}

\DeclarePairedDelimiter{\ceil}{\lceil}{\rceil}
\DeclarePairedDelimiter{\abs}{\left\lvert}{\right\rvert}

\setmainfont{Linux Libertine}

\theoremstyle{plain}
\newtheorem{axiom}{Аксиома}
\newtheorem{lemma}{Лемма}

\theoremstyle{remark}
\newtheorem*{remark}{Примечание}
\newtheorem*{exercise}{Упражнение}
\newtheorem*{consequence}{Следствие}
\newtheorem*{example}{Пример}
\newtheorem*{observation}{Наблюдение}

\theoremstyle{definition}
\newtheorem{theorem}{Теорема}
\newtheorem*{definition}{Определение}
\newtheorem*{obozn}{Обозначение}

\setlength{\parindent}{0pt}

\newcommand{\dbltilde}[1]{\accentset{\approx}{#1}}
\newcommand{\intt}{\int\!}

% magical thing that fixes paragraphs
\makeatletter
\patchcmd{\CatchFBT@Fin@l}{\endlinechar\m@ne}{}
  {}{\typeout{Unsuccessful patch!}}
\makeatother

\newcommand{\get}[2]{
    \ExecuteMetaData[#1]{#2}
}

\newcommand{\getproof}[2]{
    \iftoggle{useproofs}{\ExecuteMetaData[#1]{#2proof}}{}
}

\newcommand{\getwithproof}[2]{
    \get{#1}{#2}
    \getproof{#1}{#2}
}

\newcommand{\import}[3]{
    \subsection{#1}
    \getwithproof{#2}{#3}
}

\newcommand{\given}[1]{
    Дано выше. (\ref{#1}, стр. \pageref{#1})
}

\renewcommand{\ker}{\text{Ker }}
\newcommand{\im}{\text{Im }}
\newcommand{\grad}{\text{grad}}

\cfoot{}

\begin{document}

$$I_n(x)=\intt \sin^n x dx = \intt \sin^{n-1}\sin x dx = [u := \sin^{n-1}x, dv:= \sin x dx, v=-\cos x]=-\cos x \sin^{n-1} x+$$
$$+ (n-1)\intt \cos^2x \sin^{n-2}x dx=-\cos x \sin ^{n-1}x + (n-1)\left(\intt \sin^{n-2}x dx - \intt \sin^n xdx\right)=$$
$$=-\cos x \sin ^{n-1}x + (n-1)\left(I_{n-2}(x) - I_n(x)\right)$$
$$I_n=-\cos x \sin ^{n-1}x + (n-1)\left(I_{n-2}(x) - I_n(x)\right)$$
$$I_n=\frac{-\cos x \sin ^{n-1}x + (n-1)I_{n-2}(x)}{n}$$
$$I_0=x$$
$$I_1=\cos x$$
$$I_2=\frac{-\cos x \sin x + I_0}{2}=\frac{-\cos x \sin x + x}{2}$$
$$\left(\frac{-\cos x \sin x + x}{2}\right)'=\frac{1}{2}((\sin2x)' + 1)=\frac{1}{2}(-\cos 2x \cdot 2 + 1)=-\cos 2x + 1=\sin^2 x$$

$$\intt \frac{dx}{(x^2+1)^n}=[u=\frac{1}{(x^2+1)^{n}}, dv=dx, du=-n\frac{1}{(x^2+1)^{n+1}}2xdx]=$$
$$=\frac{1}{(x^2+1)^{n}} + \intt xn\frac{1}{(x^2+1)^{n+1}}2xdx=\frac{1}{(x^2+1)^{n}} + \intt n\frac{x^2+1-1}{(x^2+1)^{n+1}}2dx=$$
$$=\frac{1}{(x^2+1)^{n}} + 2n\left(I_n - I_{n+1}\right)$$
$$I_{n+1}=\frac{1}{2n}\left(\frac{x}{(x^2+1)^n}+(2n-1)I_n\right)$$
$$I_0=x \quad I_1=\arctg x \quad I_2=\frac{1}{2}\left(\frac{x}{x^2+1}+\arctg\right)$$
Проверка
$$\frac{1}{2}\left(\frac{x}{x^2+1}+\arctg\right)'=\frac{1}{2}\left(\frac{x^2+1-2x^2}{(x^2+1)^2}+\frac{1}{x^2+1}\right)=\frac{1}{(x^2+1)^2}$$
\section{Интегралы простейших дробей}
\begin{enumerate}
    \item $$\intt \frac{1}{x-a}dx=\ln|x-a|+C$$
    \item $n\not=1$
    $$\frac{dx}{(x-a)^n}=\frac{1}{(1-n)(x-a)^n}+C$$
    \item $$\frac{dx}{x^2+1}=\arctg x$$
    \item $$\frac{x}{x^2+1}dx=\frac{1}{2}\intt \frac{d(x^2+1)}{x^2+1}=\frac{1}{2}\ln|x^2+1|$$
    \item $$\intt \frac{dx}{(x^2+1)^n}$$
    \item $$\intt \frac{xdx}{(x^2+1)^n}=\frac{1}{2}\intt \frac{d(x^2+1)}{(x^2+1)^n}=\frac{1}{2}\frac{1}{(1-n)(x^2+1)^{n-1}}$$
\end{enumerate}

$$\intt \frac{Ex+F}{x^2+bx+c}dx; b^2-4c<0$$
$$[\text{замена}] = \intt\frac{At+B}{t^2+1}$$
$$\left(x+\frac{b}{2}\right)^2+c-\frac{b^2}{4}$$
$$\left(\frac{Ex+F}{\left(x+\frac{b}{2}\right)^2+c-\frac{b^2}{4}}\right)=\frac{1}{c-\frac{b^2}{4}}\intt \frac{Ex+F}{\frac{(x+\frac{b}{2})^2}{c-\frac{b^2}{4}}+1}dx=\left[t=\frac{x+\frac{b}{2}}{\sqrt{c-\frac{b^2}{4}}}\right]$$
$$\intt \frac{\tilde Et + \tilde F}{t^2+1}dt$$

$$\intt \frac{3x+4}{x^2+2x+2}dx=\intt\frac{3x+4}{(x+1)^2+1}dx=[x+1=t, dx=dt]=\intt\frac{3(t-1)+4}{t^2+1}dt=$$
$$=\frac{3}{2}\ln|t^2+1|+\arctg t + C$$
$$\intt \frac{dt}{t^2+a^2}=\frac{1}{a}\arctg\frac{t}{a}$$

\section{Интегрирование рациональных дробей}

$\frac{P_n(x)}{Q_m(x)}$ --- рациональная дробь.
$$Q_m:=\prod\limits_{k=1}^N(x-a_k)^{r_k}\cdot \prod\limits_{k=1}^M(x^2+b_kx+c_k)^{s_k}$$
$$\frac{P_n(x)}{Q_m(x)}=\text{ целая часть, если } n\geq m + \sum\limits_{k=1}^{N}\sum\limits_{j=1}^{r_k} \frac{A_{kj}}{(x-a_k)^j} + \sum\limits_{k=1}^{M}\sum\limits_{j=1}^{s_k} \frac{B_{kj}x+C_{kj}}{(x^2+b_kx+c_k)^j}$$

$$\frac{x^3+1}{x^3-5x^2+6x}$$
$$\text{Выделим целую часть: } \frac{x^3+1}{x^3-5x^2+6x}=1+\frac{5x^2-6x+1}{x^3-5x^2+6x}=1+\frac{5x^2-6x+1}{x(x-2)(x-3)}=\frac{A}{x}+\frac{B}{x-2}+\frac{C}{x-3}$$

Это метод неопределенных коэффициентов.

$$=\frac{A(x^2-5x+6)+B(x^2-3x)+C(x^2-2x)}{x(x-2)(x-3)}$$

$$x^2: A+B+C=5$$
$$x: -5A-3B-2C=6$$
$$1: 6A=1$$
$$A=\frac{1}{6}$$
$$B+C=\frac{29}{6}$$
$$-\frac{5}{6}-\frac{58}{6}-B=6$$
$$-\frac{63}{6}+\frac{36}{6}=B$$
$$B=\frac{-27}{6}$$
$$C=\frac{56}{6}$$
$$\intt \frac{x^3+1}{x^3-5x^2+6x} dx = x + \frac{1}{6} \ln|x|-\frac{27}{6}\ln|x-2|+\frac{56}{6}\ln|x-3|+C$$
$$\intt \frac{dx}{x^3+1}=\intt \frac{dx}{(x+1)(x^2-x+1)}$$
$$\frac{1}{(x+1)(x^2-x+1)}=\frac{A}{x+1}+\frac{Bx+C}{x^2-x+1}=\frac{A(x^2-x+1)+(Bx+C)(x+1)}{(x+1)(x^2-x+1)}$$
$$x^2=0: A+B$$
$$x=0: -A+B+C$$
$$1=1: A+C$$
$$A=-B \quad C = \frac{2}{3} \quad A = \frac{1}{3} \quad B=-\frac{1}{3}$$

$$\intt \frac{dx}{x^3+1}=\frac{1}{3}\intt\frac{dx}{x+1}+\frac{1}{3}\intt\frac{-x+2}{x^2-x+1}dx$$
1 способ
$$\intt\frac{-x+2}{x^2-x+1}dx=\intt\frac{-x+2}{\left(x-\frac{1}{2}\right)^2+\frac{3}{4}}dx=[t:=x-\frac{1}{2}]=$$
$$\intt\frac{-t+\frac{3}{2}}{t^2+\frac{3}{4}}dx=-\frac{1}{2}\ln|t^2+\frac{3}{4}|+\frac{3}{2}\frac{2}{\sqrt 3}\arctg \frac{2t}{sqrt 3}$$
2 способ
$$\intt\frac{-x+2}{x^2-x+1}dx=\intt\frac{-\frac{1}{2}(2x-1)+\frac{3}{2}}{x^2-x+1}dx=-\frac{1}{2}\ln|x^2-x+1|+\frac{3}{2}\intt\frac{dx}{\left(x-\frac{1}{2}\right)^2+\frac{3}{4}}=$$
$$=-\frac{1}{2}\ln|x^2-x+1|+\frac{3}{2}\cdot\frac{2}{\sqrt 3}\arctg\frac{x-\frac{1}{2}}{\frac{\sqrt 3}{2}}$$

$$\intt\frac{dx}{x^4+1}=\intt\frac{dx}{(x^2+1)^2-2x^2}=\intt\frac{dx}{(x^2-\sqrt 2x+1)(x^2+\sqrt 2x+1)}$$
$$\frac{1}{x^4+1}=\frac{Ax+B}{x^2-\sqrt 2 x + 1}+\frac{Cx+D}{x^2+\sqrt 2 x + 1}=\frac{(Ax+B)(x^2+\sqrt 2 x + 1)+(Cx+D)(x^2-\sqrt 2 x + 1)}{x^4+1}$$
$$x^3:0=A+C$$
$$x^2:0=B+D+\sqrt 2A-\sqrt 2C$$
$$x:0=A+C+\sqrt2B-\sqrt2D$$
$$1:1=B+D$$
$$B=D \quad B=0.5=D \quad A=C=\frac{1}{2\sqrt 2}$$
$$\intt\frac{\frac{1}{2\sqrt 2}x+0.5}{x^2-\sqrt 2x+1}dx=\intt\frac{\frac{1}{4\sqrt 2}(2x-\sqrt 2)+\frac{3}{4}}{x^2-\sqrt 2 x + 1}dx=$$
$$=\frac{1}{4\sqrt 3}\ln(x^2+\sqrt 2 x + 1)+\intt\frac{\frac{3}{4}}{x^2-\sqrt 2 x + 1}dx=\intt\frac{d(x-\frac{\sqrt 2}{2})}{\left(x-\frac{\sqrt2}{2}\right)^2+\frac{1}{2}}=\sqrt 2 \arctg (\sqrt (x-\frac{\sqrt 2}{2}))$$

$$\intt\frac{6x^5+6x+1}{x^6+3x^2+x+8}dx=\ln|x^6+3x^2+x+8|+C$$

\section{Универсальная тригонометрическая подстановка}

$$\intt\frac{dx}{2\sin x - \cos x + 5}$$
$$\tg\frac{x}{2}=:t$$
$$x=2\arctg t$$
$$dx = \frac{2}{1+t^2}dt$$
$$\frac{1}{\cos^2\frac{x}{2}}=\frac{\cos^2\frac{x}{2}+\sin^2\frac{x}{2}}{\cos^2\frac{x}{2}}=1+t^2$$
$$\sin x = 2\frac{\sin\frac{x}{2}}{\cos\frac{x}{2}}\cdot \cos^2\frac{x}{2}=\frac{2t}{1+t^2}$$

\end{document}