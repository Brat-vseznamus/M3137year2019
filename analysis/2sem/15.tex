\documentclass[12pt, a4paper]{article}

\usepackage{lastpage}
\usepackage{mathtools}
\usepackage{xltxtra}
\usepackage{libertine}
\usepackage{amsmath}
\usepackage{amsthm}
\usepackage{amsfonts}
\usepackage{amssymb}
\usepackage{enumitem}
\usepackage{xcolor}
\usepackage[left=1.5cm, right=1.5cm, top=2cm, bottom=2cm, bindingoffset=0cm, headheight=15pt]{geometry}
\usepackage{fancyhdr}
\usepackage[russian]{babel}
% \usepackage[utf8]{inputenc}
\usepackage{catchfilebetweentags}
\usepackage{accents}
\usepackage{calc}
\usepackage{etoolbox}
\usepackage{mathrsfs}
\usepackage{wrapfig}

\providetoggle{useproofs}
\settoggle{useproofs}{false}

\pagestyle{fancy}
\lfoot{M3137y2019}
\rhead{\thepage\ из \pageref{LastPage}}

\newcommand{\R}{\mathbb{R}}
\newcommand{\Q}{\mathbb{Q}}
\newcommand{\C}{\mathbb{C}}
\newcommand{\Z}{\mathbb{Z}}
\newcommand{\B}{\mathbb{B}}
\newcommand{\N}{\mathbb{N}}

\newcommand{\const}{\text{const}}

\newcommand{\teormin}{\textcolor{red}{!}\ }

\DeclareMathOperator*{\xor}{\oplus}
\DeclareMathOperator*{\equ}{\sim}
\DeclareMathOperator{\Ln}{\text{Ln}}
\DeclareMathOperator{\sign}{\text{sign}}
\DeclareMathOperator{\Sym}{\text{Sym}}
\DeclareMathOperator{\Asym}{\text{Asym}}
% \DeclareMathOperator{\sh}{\text{sh}}
% \DeclareMathOperator{\tg}{\text{tg}}
% \DeclareMathOperator{\arctg}{\text{arctg}}
% \DeclareMathOperator{\ch}{\text{ch}}

\DeclarePairedDelimiter{\ceil}{\lceil}{\rceil}
\DeclarePairedDelimiter{\abs}{\left\lvert}{\right\rvert}

\setmainfont{Linux Libertine}

\theoremstyle{plain}
\newtheorem{axiom}{Аксиома}
\newtheorem{lemma}{Лемма}

\theoremstyle{remark}
\newtheorem*{remark}{Примечание}
\newtheorem*{exercise}{Упражнение}
\newtheorem*{consequence}{Следствие}
\newtheorem*{example}{Пример}
\newtheorem*{observation}{Наблюдение}

\theoremstyle{definition}
\newtheorem{theorem}{Теорема}
\newtheorem*{definition}{Определение}
\newtheorem*{obozn}{Обозначение}

\setlength{\parindent}{0pt}

\newcommand{\dbltilde}[1]{\accentset{\approx}{#1}}
\newcommand{\intt}{\int\!}

% magical thing that fixes paragraphs
\makeatletter
\patchcmd{\CatchFBT@Fin@l}{\endlinechar\m@ne}{}
  {}{\typeout{Unsuccessful patch!}}
\makeatother

\newcommand{\get}[2]{
    \ExecuteMetaData[#1]{#2}
}

\newcommand{\getproof}[2]{
    \iftoggle{useproofs}{\ExecuteMetaData[#1]{#2proof}}{}
}

\newcommand{\getwithproof}[2]{
    \get{#1}{#2}
    \getproof{#1}{#2}
}

\newcommand{\import}[3]{
    \subsection{#1}
    \getwithproof{#2}{#3}
}

\newcommand{\given}[1]{
    Дано выше. (\ref{#1}, стр. \pageref{#1})
}

\renewcommand{\ker}{\text{Ker }}
\newcommand{\im}{\text{Im }}
\newcommand{\grad}{\text{grad}}

\lhead{Конспект по матанализу}
\cfoot{}
\rfoot{Лекция 15}

\begin{document}

Альтернативный вывод формулы Вейерштрасса \textit{(из формулы Эйлера)}:
\begin{proof}
    $$\frac{1}{\Gamma(x)}=\lim n^{-x}\frac{x(x+1)\ldots (x+n)}{n!}=x\lim n^{-x}\prod_{k=1}^n \left(1+\frac{x}{k}\right)=$$
    $$=x\lim \underbrace{e^{x\left(1+\frac{1}{2}+\frac{1}{n}\right)-x\ln n}}_{e^{\gamma+O(1)}}\prod\limits_{k=1}^{n}\left(1+\frac{x}{k}\right)e^{-\frac{x}{k}}$$
\end{proof}

\begin{remark}
    $$\Gamma(x)\Gamma(1-x)=\Gamma(x)(-x)\Gamma(-x)=\frac{(-x)}{xe^{\gamma x}\prod \left(1+\frac{x}{k}\right)e^{-\frac{x}{k}}(-x)e^{-\gamma x}\prod \left(1-\frac{x}{k}\right)e^{\frac{x}{k}}}=$$
    $$=\frac{1}{x \prod \left(1-\frac{x^2}{k^2}\right)}=\frac{\pi}{\sin \pi x}$$
\end{remark}

\begin{example}
    $a_n=\frac{P(n)}{Q(n)}$, $P$ и $Q$ --- многочлены.

    $\prod a_n=?$

    Пусть $P$ и $Q$ разложены на множители, т.е:
    $$P(n)=\alpha(n+a_1)(n+a_2)\ldots(n+a_k)$$
    $$Q(n)=\beta(n+b_1)(n+b_2)\ldots(n+b_l)$$
    $$a_n=\frac{\alpha}{\beta}\frac{(n+a_1)\ldots(n+a_k)}{(n+b_1)\ldots(n+b_l)}$$
    Если $k\not=l$, то $a_n\to0$ или $a_n\to+\infty \Rightarrow \prod a_n$ расходится.
    $\sphericalangle k=l$
    $$a_n\xrightarrow[n\to+\infty]{}\frac{\alpha}{\beta}$$
    Если $\frac{\alpha}{\beta}$, то $a_n\not\to1 \Rightarrow \prod a_n$ расходится.
    $\sphericalangle \frac{\alpha}{\beta}=1$
    $$a_n=\frac{\left(1+\frac{a_1}{n}\right)\ldots \left(1+\frac{a_k}{n}\right)}{\left(1+\frac{b_1}{n}\right)\ldots \left(1+\frac{b_l}{n}\right)}=1+\frac{1}{n}(a_1+\ldots+a_k-b_1-\ldots-b_k)+O(\frac{1}{n^2})$$
    Если $\sum\limits_{i=1}^k a_i\not=\sum\limits_{i=1}^l b_i$, то $\prod a_n$ расходится. $\sphericalangle \sum\limits_{i=1}^k a_i=\sum\limits_{i=1}^l b_i$
    $$\prod_{n=1}^N \frac{\left(1+\frac{a_1}{n}\right)\ldots \left(1+\frac{a_k}{n}\right)}{\left(1+\frac{b_1}{n}\right)\ldots \left(1+\frac{b_l}{n}\right)}=\prod_{n=1}^N \frac{\left(1+\frac{a_1}{n}\right)e^{-\frac{a_1}{n}}\ldots \left(1+\frac{a_k}{n}\right)e^{-\frac{a_k}{n}}}{\left(1+\frac{b_1}{n}\right)e^{-\frac{b_1}{n}}\ldots \left(1+\frac{b_l}{n}\right)e^{-\frac{b_l}{n}}}$$
    Равенство состоялось, т.к. $\sum a_i=\sum b_i$.

    По формуле Вейерштрасса:
    $$\prod_{n=1}^N \left(1+\frac{a}{n}\right)e^{-\frac{a}{n}}\xrightarrow[N\to+\infty]{}\frac{1}{ae^{\gamma a}\Gamma(a)}$$
    $$\prod_{n=1}^N \frac{\left(1+\frac{a_1}{n}\right)e^{-\frac{a_1}{n}}\ldots \left(1+\frac{a_k}{n}\right)e^{-\frac{a_k}{n}}}{\left(1+\frac{b_1}{n}\right)e^{-\frac{b_1}{n}}\ldots \left(1+\frac{b_l}{n}\right)e^{-\frac{b_l}{n}}}\xrightarrow[]{}\frac{b_1e^{\gamma b_1}\Gamma(b_1)\ldots b_le^{\gamma b_l}\Gamma(b_l)}{a_1e^{\gamma a_1}\Gamma(a_1)\ldots a_ke^{\gamma a_k}\Gamma(a_k)}=$$
    $$=\frac{e^{\gamma b_1}\Gamma(b_1+1)\ldots e^{\gamma b_l}\Gamma(b_l+1)}{e^{\gamma a_1}\Gamma(a_1+1)\ldots e^{\gamma a_k}\Gamma(a_k+1)}=\frac{\Gamma(b_1+1)\ldots \Gamma(b_l+1)}{\Gamma(a_1+1)\ldots \Gamma(a_k+1)}$$
\end{example}

$$\prod_{n=1}^{+\infty}\frac{4n^2}{4n^2-1}=\prod_{n=1}^{+\infty}\frac{(n-0)(n-0)}{\left(n-\frac{1}{2}\right)\left(n+\frac{1}{2}\right)}=\frac{\Gamma\left(\frac{1}{2}\right)\Gamma\left(\frac{3}{2}\right)}{\Gamma(1)\Gamma(1)}=\Gamma\left(\frac{1}{2}\right)\frac{1}{2}\,\Gamma\left(\frac{1}{2}\right)=\frac{\pi}{2}$$

\subsection*{\centering Градиент}

\begin{definition}
    $f:\R^m\to\R$, дифф. $a$, т.е. $\exists L\in\R^m$
    $$f(a+h)=f(a)+\langle L, h\rangle + o(h)$$
    $L$ --- \textbf{градиент} функции $f$ в точке $a$, обозначается $\grad f(a), \grad_a f, \grad(f, a)$.
    
    Физики \textit{(и млщики)} обозначают $\nabla f$
\end{definition}

\subsection*{\centering Производная по направлению}

``направление'' = ``единичный вектор''

\begin{definition}
    \textbf{Производная по вектору} $h\in\R^m, h\not=0$:
    $$\frac{\partial f}{\partial h}(x) = \lim_{t\to0} \frac{f(x+th)-f(x)}{t}$$
\end{definition}

\begin{enumerate}
    \item $f$ --- дифф. $\Rightarrow f$ дифф. по любому вектору
    \item Частная производная $\frac{\partial f}{\partial x_k}$ --- производная по направлению $e_k=(0 \ldots 0, \underbrace1_{k}, 0 \ldots 0)$
    \item $$\frac{\partial f}{\partial h} = \lim\limits_{t\to0}\frac{f(x_1+th_1, \ldots x_m+th_m)-f(x_1\ldots x_m)}{t}=$$
    $$=\lim\limits_{t\to0}\frac{\frac{\partial f}{\partial x_1}(x) th_1+\ldots+\frac{\partial f}{\partial x_m}(x) th_m+o(t)}{t}=$$
    $$=\frac{\partial f}{\partial x_1}h_1+\ldots+\frac{\partial f}{\partial x_m}h_m=\langle \nabla f, h \rangle$$
\end{enumerate}

\begin{theorem}
    Экстремальное свойство градиента.

    $f:E\subset \R^m\to\R$, $f$ дифф. $a\in Int E$, $\nabla f(a)\not= 0$.
    
    Тогда $l = \frac{\nabla f(a)}{|\nabla f(a)|}$ --- направление наискорейшего возрастания функции, т.е. $$\forall h\in\R^m : |h|=1, -|\nabla f(a)|\le \frac{\partial f}{\partial h}(a) \le |\nabla f(a)|$$
    , причем ``$=$'' достигаеся только при $h=\pm l$, где при ``$+$'' достигается ``$=$'' 
\end{theorem}

\begin{example}
    Градиентный спуск.

    $\sphericalangle z=1-x^2-y^2$

    $\nabla z=(-2x, -2y)$

    $\nabla (z, (0, 1)) = (0, -2) \Rightarrow$ с параболоида из точки $(0, 1)$ быстрее всего съезжать по направлению $\vec v=(0, 1)$

    Это простейший метод нахождения локального минимума.
\end{example}

\subsection*{\centering Частные производные высших порядков}

$f : E\subset\R^m \to\R, a\in Int E$

$k\in\{1\ldots m\}$. Если $\exists g(x)=\frac{\partial f}{\partial x_k}$ в окрестности точки $a$:

$i\in\{1\ldots m\}$, $\frac{\partial g}{\partial x_i}$ называется второй частной производной (производной второго порядка) по переменным $i$ и $k$, обозначается $\frac{\partial^2 f}{\partial x_i \partial x_k}$

В общем случае:
$$\frac{\partial^k f}{\partial x_{i_1}\partial x_{i_2}\ldots \partial x_{i_k}}\stackrel{def}{=}\frac{\partial}{\partial x_i}\left(\frac{\partial^{k-1} f}{\partial x_{i_2}\ldots \partial x_{i_k}}\right)$$

\begin{theorem}
    О независмости частных производных от порядка дифференциирования.
    
    $f: E\subset\R^2\to\R, (x_0, y_0)\in E$

    $\exists r>0 \ \ B((x_0, y_0), r)\subset E$

    Пусть в этом шаре $\exists f''_{xy}, f''_{yx}$ и они непрерывны. Тогда $f''_{xy}(x_0, y_0)=f''_{yx}(x_0, y_0)$
\end{theorem}
\begin{proof}
    $\Delta^2(h, k)=f(x_0 + h, y_0 + k) - f(x_0 + h, y_0) - f(x_0, y_0 + k) + f(x_0, y_0)$

    $\alpha(h):=\Delta^2(h, k)$ при фиксированном $k$

    $\alpha(h)=\alpha(h)-\alpha(0)\stackrel{\text{т. Лагранжа}}=\alpha'(\overline h)h=(f'_x (x_0 + \overline h, y_0 + k) - f'_x(x_0+\overline h, y))h\stackrel{\text{т. Лагранжа}}= f''_{xy}(x_0+\overline h, y_0+\overline k)hk$

    $\beta(k):=\Delta^2(h, k)$ при фиксированном $h$
    
    $\beta(k)=f''_{yx}(x_0+\overline{\overline h}, y_0+\overline{\overline k})hk$
    $$f''_{xy}(x_0+\overline h, y_0+\overline k)hk=f''_{yx}(x_0+\overline{\overline h}, y_0+\overline{\overline k})hk$$
    $$(h, k)\to (0, 0) \Rightarrow (\overline h, \overline k)\to(0, 0), (\overline{\overline h}, \overline{\overline k})\to(0, 0)$$
    $$f''_{xy}(x_0, y_0)=f''_{yx}(x_0, y_0)$$
\end{proof}

\begin{remark}
    %<*классce>
    $E\subset \R^m$, откр. Класс $C^r(E), r\in \N$:

    $f\in C^r(E)$, если у $f$ существуют все частные производные порядка $\le r$ на всём $E$ и они непрерывны.

    $C(E)$ --- непр. функции $=C^0(E)$

    $C(E)\stackrel{\not=}{\supset} C^1(E)\stackrel{\not=}{\supset} C^2(E)\ldots$
    %</классce>
\end{remark}

Общий вид теоремы: $f\in C^r(E) \ \ \forall k\le r$

$\forall x\in E \ \ \forall i_1\ldots i_k :$ Если $(j_1\ldots j_k)$ --- перестановка $(i_1\ldots i_k)$, то:
$$\frac{\partial^k f}{\partial x_{i_1}\ldots \partial x_{i_k}}(x)=\frac{\partial^k f}{\partial x_{j_1}\ldots \partial x_{j_k}}(x)$$

\begin{definition}
    %<*мультииндекс>
    \textbf{Мультииндекс} (для $\R^m$) --- вектор $(k_1, k_2\ldots k_m)$, $k_i\in \N\cup\{0\}$
    \begin{itemize}
        \item $|k|:=\sum_{i=1}^m k_i$ --- \textbf{высота} мультииндекса
        \item $k! = k_1!k_2!\ldots k_m!$
        \item $x\in\R^m \ \ x^k = x_1^{k_1}x_2^{k_2}\ldots x_m^{k_m}$
        \item $f^{(k)}=\frac{\partial^{|k|}}{\partial x^k} f = \frac{\partial^{|k|}f}{x_1^{k_1}x_2^{k_2}\ldots x_m^{k_m}}$
    \end{itemize}
    %</мультииндекс>
\end{definition}

\begin{lemma}
    Полиномиальная формула.

    $a_i\in\R$ \textit{(верно для любого кольца)}. Тогда $\forall r\in\N$:
    $$(a_1+\ldots a_m)^r\stackrel{\text{очев}}=\sum_{n_1=1}^{m}\ldots \sum_{n_r=1}^{m}a_{n_1}a_{n_2}\ldots a_{n_r}=\sum_{j : |j|=r} \frac{r!}{j!}a^j$$
\end{lemma}

\begin{proof}
    По индукции.

    База: \textcolor{red}{скипнуто}

    Переход: $$(a_1+\ldots+a_m)^{r+1}=(a_1+\ldots+a_m)\sum_{\substack{j_1\ldots j_m\ge 0 \\ j_1+\ldots+j_m=r}}\frac{r!}{j_1!\ldots j_m!}a_1^{j_1}\ldots a_m^{j_m}=$$
    $$=\sum \frac{r!}{j_1!\ldots j_m!}a_1^{j_1+1}\ldots a_m^{j_m}+\ldots+ \sum \frac{r!}{j_1!\ldots j_m!}a_1^{j_1}\ldots a_m^{j_m+1}=$$
    $$=\sum_{\substack{k_1 \ge 1 \\ k_2\ldots k_m \ge 0 \\ k_1+\ldots k_m=r+1}} \frac{r!k_1}{k_1!\ldots k_m!}a_1^{k_1}\ldots a_m^{k_m}+\ldots+ \sum_{\substack{k_2 \ge 1 \\ k_1,k_3,k_4\ldots k_m \ge 0 \\ k_1+\ldots k_m=r+1}} \frac{r!k_2}{j_1!\ldots j_m!}a_1^{j_1}\ldots a_m^{j_m+1}=$$
    $$=\sum_{\substack{k_1 \ge 0 \\ k_2\ldots k_m \ge 0 \\ k_1+\ldots k_m=r+1}} \frac{r!k_1}{k_1!\ldots k_m!}a_1^{k_1}\ldots a_m^{k_m}+\ldots+ \sum_{\substack{k_2 \ge 0 \\ k_1,k_3,k_4\ldots k_m \ge 0 \\ k_1+\ldots k_m=r+1}} \frac{r!k_2}{j_1!\ldots j_m!}a_1^{j_1}\ldots a_m^{j_m+1}=$$
    $$=\sum_{\substack{k_1\ldots k_m \ge 0 \\ k_1+\ldots k_m=r+1}}\frac{r!(k_1+\ldots+k_m)}{k_1!\ldots k_m!}a_1^{k_1}\ldots a_m^{k_m}=$$
    $$=\sum_{\substack{k_1\ldots k_m \ge 0 \\ k_1+\ldots k_m=r+1}}\frac{(r+1)!}{k_1!\ldots k_m!}a_1^{k_1}\ldots a_m^{k_m}$$
\end{proof}

\end{document}