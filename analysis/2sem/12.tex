\documentclass[12pt, a4paper]{article}

\usepackage{lastpage}
\usepackage{mathtools}
\usepackage{xltxtra}
\usepackage{libertine}
\usepackage{amsmath}
\usepackage{amsthm}
\usepackage{amsfonts}
\usepackage{amssymb}
\usepackage{enumitem}
\usepackage{xcolor}
\usepackage[left=1.5cm, right=1.5cm, top=2cm, bottom=2cm, bindingoffset=0cm, headheight=15pt]{geometry}
\usepackage{fancyhdr}
\usepackage[russian]{babel}
% \usepackage[utf8]{inputenc}
\usepackage{catchfilebetweentags}
\usepackage{accents}
\usepackage{calc}
\usepackage{etoolbox}
\usepackage{mathrsfs}
\usepackage{wrapfig}

\providetoggle{useproofs}
\settoggle{useproofs}{false}

\pagestyle{fancy}
\lfoot{M3137y2019}
\rhead{\thepage\ из \pageref{LastPage}}

\newcommand{\R}{\mathbb{R}}
\newcommand{\Q}{\mathbb{Q}}
\newcommand{\C}{\mathbb{C}}
\newcommand{\Z}{\mathbb{Z}}
\newcommand{\B}{\mathbb{B}}
\newcommand{\N}{\mathbb{N}}

\newcommand{\const}{\text{const}}

\newcommand{\teormin}{\textcolor{red}{!}\ }

\DeclareMathOperator*{\xor}{\oplus}
\DeclareMathOperator*{\equ}{\sim}
\DeclareMathOperator{\Ln}{\text{Ln}}
\DeclareMathOperator{\sign}{\text{sign}}
\DeclareMathOperator{\Sym}{\text{Sym}}
\DeclareMathOperator{\Asym}{\text{Asym}}
% \DeclareMathOperator{\sh}{\text{sh}}
% \DeclareMathOperator{\tg}{\text{tg}}
% \DeclareMathOperator{\arctg}{\text{arctg}}
% \DeclareMathOperator{\ch}{\text{ch}}

\DeclarePairedDelimiter{\ceil}{\lceil}{\rceil}
\DeclarePairedDelimiter{\abs}{\left\lvert}{\right\rvert}

\setmainfont{Linux Libertine}

\theoremstyle{plain}
\newtheorem{axiom}{Аксиома}
\newtheorem{lemma}{Лемма}

\theoremstyle{remark}
\newtheorem*{remark}{Примечание}
\newtheorem*{exercise}{Упражнение}
\newtheorem*{consequence}{Следствие}
\newtheorem*{example}{Пример}
\newtheorem*{observation}{Наблюдение}

\theoremstyle{definition}
\newtheorem{theorem}{Теорема}
\newtheorem*{definition}{Определение}
\newtheorem*{obozn}{Обозначение}

\setlength{\parindent}{0pt}

\newcommand{\dbltilde}[1]{\accentset{\approx}{#1}}
\newcommand{\intt}{\int\!}

% magical thing that fixes paragraphs
\makeatletter
\patchcmd{\CatchFBT@Fin@l}{\endlinechar\m@ne}{}
  {}{\typeout{Unsuccessful patch!}}
\makeatother

\newcommand{\get}[2]{
    \ExecuteMetaData[#1]{#2}
}

\newcommand{\getproof}[2]{
    \iftoggle{useproofs}{\ExecuteMetaData[#1]{#2proof}}{}
}

\newcommand{\getwithproof}[2]{
    \get{#1}{#2}
    \getproof{#1}{#2}
}

\newcommand{\import}[3]{
    \subsection{#1}
    \getwithproof{#2}{#3}
}

\newcommand{\given}[1]{
    Дано выше. (\ref{#1}, стр. \pageref{#1})
}

\renewcommand{\ker}{\text{Ker }}
\newcommand{\im}{\text{Im }}
\newcommand{\grad}{\text{grad}}

\lhead{Конспект по матанализу}
\cfoot{}
\rfoot{Лекция 12}

\renewcommand{\thesubsection}{\arabic{subsection}.}

\begin{document}

$$\int_a^b fg' = fg\Big|_a^b - \int_a^b f'g$$
$$\int \leftrightarrow \sum \quad a\leftrightarrow g \quad b\leftrightarrow f \quad \sum_{i=1}^k a_i=A_k \quad f'\leftrightarrow b_{k+1}-b_k$$
$$\sum_{k=1}^n a_kb_k = A_nb_n + \sum_{k=1}^{n-1} A_k(b_k-b_{k+1})$$
Это \textbf{суммирование по частям}, оно же \textbf{правило Абеля cуммы}.
\begin{proof}
    $\sphericalangle b_7$, посмотрим с какими коэффициентами оно входит в выражения по обе части равенства:
    \begin{itemize}
        \item Левая часть: $a_7$
        \item Правая часть: $A_7-A_6=a_7$
    \end{itemize}
    Для других случаев тоже верно. %TODO
\end{proof}

\begin{theorem}
    Признак Абеля-Дирихле

    %<*признакабелядирихле>
    \textbf{Дирихле}:
    \begin{enumerate}
        \item Последовательность $A_k=\sum\limits_{i=1}^k a_i$ ограничена: $\exists C_A > 0 \ \ \forall k \ \ |A_k|<C_A$
        \item $b_k$ монотонна и $\to0$
    \end{enumerate}

    \textbf{Абеля}:
    \begin{enumerate}
        \item Ряд $\sum a_k$ сходится
        \item $b_k$ монотонна, ограничена: $\exists C_B > 0 \ \ \forall k \ \ |A_k|<C_B$
    \end{enumerate}

    Если хотя бы один из этих признаков состоялся, $\sum\limits_{k=1}^n a_kb_k$ сходится.
    %</признакабелядирихле>
\end{theorem}
%<*признакабелядирихлеproof>
\begin{proof}
    $$\sum_{k=1}^n a_kb_k = \underbrace{A_nb_n}_{\to0} + \underbrace{\sum_{k=1}^{n-1} A_k(b_k-b_{k+1})}_{\substack{\exists \text{ конечный предел,} \\ \text{т.к. ряд абсолютно сходится}}}$$
    Докажем Дирихле.
    $$\sum_{k=1}^{n-1}|A_k||b_k-b_{k+1}| \le C_A\sum_{k=1}^{n-1}|b_k-b_{k+1}|=\pm C_A\sum_{k=1}^{n-1}b_k-b_{k+1}=\pm \underbrace{C_a(b_1-b_n)}_{\text{огр.}}\le C_AC_B$$
    Докажем Абеля.

    $\exists$ конечный $\beta=\lim\limits_{k\to+\infty} b_k$
    $$\sum_{k=1}^n a_kb_k=\beta\sum_{k=1}^n a_k + \sum_{k=1}^n a_k(b_k-\beta)$$
    Второй ряд сходится по признаку Дирихле, первый сходится по условию.
\end{proof}
%</признакабелядирихлеproof>

\begin{example}
    $$\sum \frac{\sin n}{\sqrt n}$$
    Докажем сходимость по признаку Дирихле:
    $a_n = \sin n, b_n=\frac{1}{\sqrt n}$

    $b_n$ монот., $\to0$

    $A_k=\left(\sum\limits_{k=1}^n \sin k\right)$ - огр.?
    $$|\sin 1 + \sin 2 + \ldots + \sin n| = |\Im(e^i+e^{2i}+e^{3i}+\ldots+e^{ni})| \le |e^i+e^{2i}+\ldots+e^{ni}|=$$
    $$=\left|e^i\frac{e^{ni}-1}{e^i-1}\right|=|e^i|\frac{|e^{ni}-1|}{|e^i-1|}\le 1\frac{2}{|e^i-1|} =: C_A$$
    Итого $A_k$ ограничено $\Rightarrow$ искомая последовательность сходится по признаку Дирихле.
\end{example}

\section*{Свойства сходящихся рядов}

\subsection{Группировка}

% $$\sum_{l=1}^{\infty}b_l$$
\begin{theorem}
    %<*группировкарядов>
    $$\sum_{k=1}^{+\infty} a_k = \underbrace{(a_1 + a_2 + \ldots + a_{n_1})}_{b_1} + \underbrace{(a_{n_1+1} + \ldots + a_{n_2})}_{b_2} + \ldots$$
    $n_1<n_2<\ldots$
    $$A=\sum a_k \quad b_l = \sum_{i=n_l+1}^{n_{l+1}}a_i \quad B=\sum b_l$$
    \begin{enumerate}
        \item Если $A$ сходится, $B$ сходится и имеет ту же сумму
        \item Если $\forall k \ \ a_k \ge 0$, $A$ и $B$ имеют одинаковую сумму.
    \end{enumerate}
    %</группировкарядов>
\end{theorem}
%<*группировкарядовproof>
\begin{proof}
    $$\sum_{i=1}^m b_i = \sum_{i=1}^{n_m} a_i$$
    \begin{enumerate}
        \item $A$ сходится $\Rightarrow \exists$ кон. $\lim \sum\limits_{i=1}^n a_i$. Тогда $$\lim_{m\to+\infty} \sum_{i=1}^m b_i=\lim_{n\to+\infty} \sum_{i=1}^{n_m} a_i=\lim_{n\to+\infty}  \sum_{i=1}^{n} a_i$$
        \item \textcolor{red}{Cкипнуто}
    \end{enumerate}
\end{proof}
%</группировкарядовproof>
%<*группировкарядовremark>
\begin{remark}
    \begin{enumerate}
        \item $B$ сходится $\not\Rightarrow A$ сходится. Контрпример: $\sum(-1)^n$
        \item $a_n\to0$, скобки ``ограниченного размера'': $\exists M \ \ \forall k \ \ |n_k-n_{k-1}|<M$
        
        Тогда $B$ сходится $\Rightarrow A$ сходится:
        $$|a_n|\to0 \ \ |a_n|+|a_{n+1}|\to0 $$
        \textcolor{red}{Скипнуто}
    \end{enumerate}
\end{remark}
%</группировкарядовremark>

\begin{example}
    $$\frac{1}{k(k+1)}=\frac{k}{k+1}-\frac{k-1}{k}$$
    $$\sum_{k=1}^{+\infty} \frac{1}{k(k+1)} = \frac{1}{2}-\frac{0}{1}+\frac{2}{3}-\frac{1}{2}+\frac{3}{4}-\frac{2}{3}+\ldots=0$$
\end{example}

Односторонний предел в $\R$:
$$\lim_{x\to a+0} f(x) = \lim_{x\to a} f\Big|_{[a, +\infty)\cap D}$$
$f:D\subset \R^m \to \R, D_1\subset D, a$ --- предельная точка $D, D_1$

$\lim\limits_{x\to a} f\Big|_{D_1}$ --- предел по подмножеству, аналог одностороннего предела

\begin{definition}
    %<*пределпонаправлению>
    \textbf{Предел по направлению} $l, |l|=1$:
    $$\lim_{t\to0+0} f(a+t \vec l)$$
    %</пределпонаправлению>
\end{definition}

\begin{enumerate}
    \item Если $\exists \lim\limits_{x\to a} f(x) = L$, то $\exists$ и пределы всем направлениям и они равны $L$.
    \item Если пределы по направлениям $\exists$ и не равны, то $\not\exists \lim\limits_{x\to a}f(x)$
\end{enumerate}

\begin{example}
    $$f=\frac{xy}{x^2+y^2}$$
    $$l:=(\cos \varphi, \sin \varphi) \quad \lim_{t\to 0} \frac{t\cos \varphi t\sin \varphi}{t^2(\cos^2 \varphi + \sin^2 \varphi)}=\cos\varphi\sin\varphi\Rightarrow \not\exists\lim_{(x,y)\to(0,0)}f(x,y)$$
\end{example}

\begin{definition}
    Предел вдоль кривой

    \textcolor{red}{Скипнуто}
\end{definition}

%<*линейныйоператор>
Линейное отображение = линейный оператор

$$f:\R^m \to\R^n \text{ --- лин.} \quad \forall \alpha,\beta\in\R \ \ \forall x,y\in\R^m \ \ f(\alpha x + \beta y)=\alpha f(x) + \beta f(x)$$
%</линейныйоператор>

\begin{definition}
    Линейное отображение $f:\R^m \to \R$ --- \textbf{линейный функционал}
\end{definition}

\textcolor{red}{Скипнуто}

Линейные отображения образуют линейное пространство, т.е. это множестве замкнуто по сложению и умножению на скаляры.

Линейное отображение задается матрицей:
$$f:\R^m\to \R^n \quad f\leftrightarrow A = \begin{bmatrix}
    a_{11} & \ldots & a_{1m} \\
    \vdots & \ddots & \vdots \\
    a_{n1} & \ldots & a_{nm} 
\end{bmatrix}$$
$$f(x) = Ax$$

\begin{theorem}
    $\mathcal A:\R^m\to\R^m$ --- лин. оператор

    Эквивалентны следующие утверждения:
    \begin{enumerate}
        \item $\mathcal A$ --- обратим
        \item $\mathcal A(\R^m)=\R^m$
    \end{enumerate}
\end{theorem}

Линейные отображения ``общего вида'':
\begin{itemize}
    \item $\R\to\R \quad f(x)=\alpha x \quad f\leftrightarrow \alpha\in\R$
    \item $\R^m\to\R \quad f(x)=\langle x,a \rangle \quad f\leftrightarrow a\in\R^m$
    \item $\R\to\R^m \quad f(x)=xv \quad f\leftrightarrow v\in\R^m$
\end{itemize}

\subsection{Дифференциирование отображений}

\begin{definition}
    \begin{enumerate}
        \item 
        %<*бесконечномалоеотображение>
        \textbf{Бесконечно малое отображение} $\varphi : E\subset\R^m \to\R^l$

        $x_0$ --- предельная точка $E$
    
        $\varphi$ --- бесконечно малое отображение при $x\to x_0$ $\varphi(x)\xrightarrow{x\to x_0}0$
        %</бесконечномалоеотображение>

        \item %<*омалоеотображение>
        $o(h)$ (оно же $o(|h|)$)
        
        $\varphi : E\subset \R^m \to \R^l$, $0$ --- предельная точка $E$

        $\varphi(h) = o(h)$ при $h\to0$, если $\frac{\varphi(h)}{|h|}\xrightarrow{h\to0}0$

        По-другому: $\exists \alpha : E\to\R^l$ --- бесконечно малое при $h\to0$:
        $$\varphi(h)=|h|\alpha(h)$$
        %</омалоеотображение>
    \end{enumerate}
\end{definition}

\begin{definition}
    %<*дифференциируемоеотображение>
    $F:E\subset\R^m\to\R^l, a\in Int E$

    $F$ --- дифф. в точке $a$, если:
    $$\exists \text{ лин. оп. } L : \R^m\to\R^l \ \ \exists \text{ бесконечно малое } \alpha : E\to\R^l:$$
    $$F(a+h)=F(a)+Lh+|h|\alpha(h), h\to0$$
    $$F(a+h)=F(a)+Lh+o(h)$$
    $$x:=a+h$$
    $$F(x)=F(a)+L(x-a)+|x-a|\alpha(x-a)$$
    %</дифференциируемоеотображение>
\end{definition}

\begin{definition}
    %<*производныйоператорматрицаякоби>
    Оператор $L$ из определения --- \textbf{производный оператор} отображениия $F$ в точке $a$ (``производная''), обозначается $F'(a)$.

    Матрица $F'(a)$ --- \textbf{матрица Якоби} $F$ в точке $a$
    %</производныйоператорматрицаякоби>
\end{definition}

\begin{definition}
    %<*дифференциалотображения>
    
    Выражение $F'(a)h$ называется \textbf{дифференциалом} отображения $F$ в точке $a$.

    Это понимают как:
    \begin{enumerate}
        \item Производный оператор $h\mapsto F'(a)h$
        \item Отображение $E\times \R^m\to\R^l \quad (x, h)\mapsto F'(x)\cdot h$
    \end{enumerate}
    %</дифференциалотображения>
\end{definition}

$F:\C\to\C \quad a\in\C$
\begin{definition}
    $F$ \textbf{комплексно дифференциируема} в точке $A$, если $\exists \lambda\in\C$:
    $$F(a+h)=F(a)+\lambda h + o(h), h\to0$$
    $$h=x+iy \leftrightarrow \begin{pmatrix}
        x \\
        y
    \end{pmatrix}\in\R^2 \quad \lambda=a+bi$$
    $$h\mapsto \lambda h$$
    $$(x+iy)\mapsto (a+bi)(x+iy)=ax-by+i(ay+bx)$$
    $$\begin{pmatrix}
        x \\
        y
    \end{pmatrix} \mapsto \begin{pmatrix}
        ax-by \\
        ay+bx
    \end{pmatrix} = \begin{pmatrix}
        a & -b \\
        b & a
    \end{pmatrix}\begin{pmatrix}
        x \\
        y
    \end{pmatrix}$$
    $$L=\begin{pmatrix}
        a & -b \\
        b & a
    \end{pmatrix}$$
    Этому соответствует вещественное отображение $F:\R^2\to\R^2$
    $$F(a+h)=F(a)+Lh+o(h)$$
\end{definition}

\begin{lemma}
    %<*единственностьпроизводной>
    Производный оператор единственный.
    %</единственностьпроизводной>
\end{lemma}
%<*единственностьпроизводнойproof>
\begin{proof}
    $$\exists \delta>0 \ \ \forall h : |h|<\delta \ \ a+h\in E$$
    Возьмём $v\in\R^m \quad h:=tv, t<\frac{\delta}{|v|}$
    
    По определению дифференциала:
    $$F(a+tv)=F(a)+F'(a)tv + |tv| \alpha(tv)=F(a)+tF'(a)v+|t||v|\alpha(tv)$$
    $$F'(a)v=\frac{F(a+tv)-F(a)}{t} - \overbrace{\underbrace{\frac{|t|}{t}}_{\pm1} |v|\alpha(tv)}^{\xrightarrow[t\to0]{}\pm |v|0}$$
    $$F'(a)v=\lim_{t\to0}\frac{F(a+tv)-F(a)}{t}$$
    Т.к. по всем направлениям производная равна, оператор единственный.
\end{proof}
%</единственностьпроизводнойproof>

\begin{remark}
    О дифференциировании функций нескольких переменных

    $f:E\subset\R^m \to \R , a\in Int E$

    $f$ --- дифф. $\exists \lambda_1 \ldots \lambda_m \in\R, \exists \varphi$ бесконечно малая при $x\to a$

    $$f(x_1\ldots x_m)=f(a_1\ldots a_m)+\lambda_1(x_1-a_1)+\lambda_2(x_2-a_2)+\ldots+\lambda_m(x_m-a_m)+|x-a|\varphi(x)$$
\end{remark}

\begin{remark}
    $F$ дифф. в $a \Rightarrow F$ непр. в $a$
\end{remark}

\end{document}