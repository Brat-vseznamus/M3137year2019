\documentclass[12pt, a4paper]{article}

\usepackage{lastpage}
\usepackage{mathtools}
\usepackage{xltxtra}
\usepackage{libertine}
\usepackage{amsmath}
\usepackage{amsthm}
\usepackage{amsfonts}
\usepackage{amssymb}
\usepackage{enumitem}
\usepackage{xcolor}
\usepackage[left=1.5cm, right=1.5cm, top=2cm, bottom=2cm, bindingoffset=0cm, headheight=15pt]{geometry}
\usepackage{fancyhdr}
\usepackage[russian]{babel}
% \usepackage[utf8]{inputenc}
\usepackage{catchfilebetweentags}
\usepackage{accents}
\usepackage{calc}
\usepackage{etoolbox}
\usepackage{mathrsfs}
\usepackage{wrapfig}

\providetoggle{useproofs}
\settoggle{useproofs}{false}

\pagestyle{fancy}
\lfoot{M3137y2019}
\rhead{\thepage\ из \pageref{LastPage}}

\newcommand{\R}{\mathbb{R}}
\newcommand{\Q}{\mathbb{Q}}
\newcommand{\C}{\mathbb{C}}
\newcommand{\Z}{\mathbb{Z}}
\newcommand{\B}{\mathbb{B}}
\newcommand{\N}{\mathbb{N}}

\newcommand{\const}{\text{const}}

\newcommand{\teormin}{\textcolor{red}{!}\ }

\DeclareMathOperator*{\xor}{\oplus}
\DeclareMathOperator*{\equ}{\sim}
\DeclareMathOperator{\Ln}{\text{Ln}}
\DeclareMathOperator{\sign}{\text{sign}}
\DeclareMathOperator{\Sym}{\text{Sym}}
\DeclareMathOperator{\Asym}{\text{Asym}}
% \DeclareMathOperator{\sh}{\text{sh}}
% \DeclareMathOperator{\tg}{\text{tg}}
% \DeclareMathOperator{\arctg}{\text{arctg}}
% \DeclareMathOperator{\ch}{\text{ch}}

\DeclarePairedDelimiter{\ceil}{\lceil}{\rceil}
\DeclarePairedDelimiter{\abs}{\left\lvert}{\right\rvert}

\setmainfont{Linux Libertine}

\theoremstyle{plain}
\newtheorem{axiom}{Аксиома}
\newtheorem{lemma}{Лемма}

\theoremstyle{remark}
\newtheorem*{remark}{Примечание}
\newtheorem*{exercise}{Упражнение}
\newtheorem*{consequence}{Следствие}
\newtheorem*{example}{Пример}
\newtheorem*{observation}{Наблюдение}

\theoremstyle{definition}
\newtheorem{theorem}{Теорема}
\newtheorem*{definition}{Определение}
\newtheorem*{obozn}{Обозначение}

\setlength{\parindent}{0pt}

\newcommand{\dbltilde}[1]{\accentset{\approx}{#1}}
\newcommand{\intt}{\int\!}

% magical thing that fixes paragraphs
\makeatletter
\patchcmd{\CatchFBT@Fin@l}{\endlinechar\m@ne}{}
  {}{\typeout{Unsuccessful patch!}}
\makeatother

\newcommand{\get}[2]{
    \ExecuteMetaData[#1]{#2}
}

\newcommand{\getproof}[2]{
    \iftoggle{useproofs}{\ExecuteMetaData[#1]{#2proof}}{}
}

\newcommand{\getwithproof}[2]{
    \get{#1}{#2}
    \getproof{#1}{#2}
}

\newcommand{\import}[3]{
    \subsection{#1}
    \getwithproof{#2}{#3}
}

\newcommand{\given}[1]{
    Дано выше. (\ref{#1}, стр. \pageref{#1})
}

\renewcommand{\ker}{\text{Ker }}
\newcommand{\im}{\text{Im }}
\newcommand{\grad}{\text{grad}}

\lhead{Конспект по матанализу}
\cfoot{}
\rfoot{Лекция 3}

\begin{document}

\begin{example}
    %<*неаналитическаяфункция>
    $f(x)=\begin{cases}
            e^{-\frac{1}{x^2}} & , x\not=0 \\
            0                  & , x=0
        \end{cases}$

    $$f'(x)=\frac{2}{x^3}\ e^{-\frac{1}{x^2}}, x\not=0$$
    $f'(0)=?$

    Следствие из теоремы Лагранжа:
    $$\lim\limits_{x\to x_0}f'(x)=A \text{ тогда } f'(x_0)=A$$
    $$f'(0)=\lim\limits_{x\to0}\cfrac{2}{x^3}\ e^{-\frac{1}{x^2}}=\left[\cfrac{0}{0}\right]=\lim 2\cfrac{\cfrac{2}{x^3}\ e^{-\frac{1}{x^2}}}{3x^2}=\lim\cfrac{4}{3}\cfrac{e^{-\frac{1}{x^2}}}{x^5}=\text{ больно, не надо так}$$
    $$\lim\limits_{x\to0}\cfrac{2}{x^3}\ e^{-\frac{1}{x^2}}=\lim\cfrac{\cfrac{2}{x^3}}{e^{\frac{1}{x^2}}}=\left[\cfrac{\infty}{\infty}\right]=\lim\cfrac{\cfrac{-6}{x^4}}{\cfrac{-2}{x^3}\ e^{\frac{1}{x^2}}}=\lim\cfrac{\cfrac{3}{x}}{e^{\frac{1}{x^2}}}=\left[\cfrac{\infty}{\infty}\right]=\lim\cfrac{\cfrac{-3}{x^2}}{\cfrac{-2}{x^3}\ e^{\frac{1}{x^2}}}=\lim\cfrac{3}{2}\cfrac{x}{e^{\frac{1}{x^2}}}=\cfrac{0}{\infty}=0$$

    $$f'(x)=\begin{cases}
            \cfrac{2}{x^3}\ e^{-\frac{1}{x^2}} & , x\not=0 \\
            0                                  & , x=0
        \end{cases}$$
    $$f^{(n)}(x)= \begin{cases}
            P_n\left(\cfrac{1}{x}\right)\cdot e^{-\frac{1}{x^2}} & , x\not=0 \\
            0                                                    & , x=0
        \end{cases}$$

    Заметим, что многочлен Тейлора этой функции при $x\to0$ не становится точнее при увеличении числа слагаемых, т.к. они все $=0$. Таким образом, эта функция по определению неаналитическая.
    %</неаналитическаяфункция>
\end{example}

Будем складывать дроби неправильно: $$\cfrac{a}{b}+\cfrac{c}{d}=\cfrac{a+c}{b+d}$$
Это работает в неравенствах, если $a,b,c,d>0$

\textcolor{red}{Это неверно, т.к.} $\frac{1}{2}+\frac{2}{3}\not=\frac{3}{5}$ \textcolor{red}{, но верно следующее:}

\begin{lemma}
    %<*леммадляштольца>
    ``Неправильное'' сложение дробей:

    $a,b,c,d>0$
    $$\frac{a}{b} < \frac{c}{d} \Rightarrow \frac{a}{b} < \frac{a+c}{b+d} < \frac{c}{d}$$
    %</леммадляштольца>
\end{lemma}
%<*леммадляштольцаproof>
\begin{proof}
    $$\frac{a+c}{b+d} - \frac{a}{b}=\frac{ba-ba+bc-ad}{b(b+d)}=\frac{d}{b+d}\left(\frac{c}{d}-\frac{a}{b}\right)>0$$
    $$\frac{c}{d} - \frac{a+c}{b+d}=\frac{bc-ad}{d(b+d)}=\frac{b}{b+d}\left(\frac{c}{d}-\frac{a}{b}\right)>0$$
\end{proof}
%</леммадляштольцаproof>

\begin{theorem}
    %<*штольца>
    Теорема Штольца.

    Это дискретная версия правила Лопиталя.

    $y_n\to0, x_n\to0$ --- строго монот.

    $$\lim\cfrac{x_n-x_{n-1}}{y_n-y_{n-1}}=a\in\overline \R$$

    Тогда $\exists\lim\cfrac{x_n}{y_n}=a$
    \begin{remark}
        Аналогичное верно, если $x_n\to+\infty, y_n\to+\infty$
    \end{remark}
    %</штольца>
\end{theorem}
%<*штольцаproof>
\begin{proof}
    \begin{enumerate}
        \item $a>0 \quad (a\not=+\infty)$

              $$\forall \varepsilon > 0 \ \ [\varepsilon<a] \ \ \exists N_1 \ \ \forall n>N_1 \ \ a-\varepsilon<\cfrac{x_n-x_{n-1}}{y_n-y_{n-1}}<a+\varepsilon$$
              Берем $N>N_1$
              $$a-\varepsilon<\cfrac{x_{N+1}-x_{N}}{y_{N+1}-y_{N}}<a+\varepsilon$$
              $$\vdots$$
              $$a-\varepsilon<\cfrac{x_{n}-x_{n-1}}{y_{n}-y_{n-1}}<a+\varepsilon$$
              По неправильному сложению:\textit{(оно применимо, т.к. все дроби положительные)}
              $$a-\varepsilon<\cfrac{x_n-x_N}{y_n-y_N}<a+\varepsilon$$
              $n\to+\infty$
              $$a-\varepsilon<\cfrac{x_N}{y_N}<a+\varepsilon$$

        \item $a=+\infty$ доказывается так же
        \item $a<0$ поменяем знак и докажем так же
        \item $a=0$ т.к. знаки $x_n-x_{n-1}$ и $y_n-y_{n-1}$ фикс., $a=+0$ или $a=-0$

              Для $a=+0 \quad \lim\cfrac{y_n-y_{n-1}}{x_n-x_{n-1}}=+\infty$
    \end{enumerate}
\end{proof}
%</штольцаproof>

$x_n=1+2+3+\ldots+n\stackrel{?}{\equ\limits_{n\to+\infty}}y_n$

$$\lim\cfrac{x_n}{y_n}=\lim\cfrac{x_n-x_{n-1}}{y_n-y_{n-1}}=\lim\limits_{n\to+\infty}\cfrac{n}{y_n-y_{n-1}}=\left[y_n:=\cfrac{n^2}{2}\right]=\lim\cfrac{n}{n-\cfrac{1}{2}}=1$$
$$x_n\equ \cfrac{n^2}{2}$$
$$x_n=1^\alpha+2\alpha+3\alpha+\ldots+n\alpha-\cfrac{n^{\alpha+1}}{\alpha+1}\equ z_n$$
$$\lim\cfrac{x_n}{z_n}=\lim\cfrac{n^\alpha-\left(\cfrac{n^{\alpha+1}}{\alpha+1}-\cfrac{(n-1)^{\alpha+1}}{\alpha+1}\right)}{z_n-z_{n-1}}$$
$$n^\alpha-\cfrac{1}{\alpha+1}n^{\alpha+1}\left(1-\left(1-\cfrac{1}{n}\right)^{\alpha+1}\right)=$$
$$=n^\alpha-\cfrac{1}{\alpha+1}n^{\alpha+1}\left((\alpha+1)\cfrac{1}{n}-\cfrac{(\alpha+1)\alpha}{2}\cfrac{1}{n^2}+o\left(\cfrac{1}{n^2}\right)\right)=\cfrac{\alpha}{2}n^{\alpha-1}+o(n^{\alpha-1})=$$
$$=\frac{\alpha}{2}n^{\alpha-1}+o(n^{\alpha-1})$$
$$\lim\cfrac{x_n}{z_n}=\lim\cfrac{\cfrac{\alpha}{2}n^{\alpha-1}}{z_n-z_{n-1}}$$

Функциональные свойства определенного интеграла:
\begin{enumerate}
    \item $\forall \alpha, \beta\in\R,\ \ f,g\in C[a,b]$
          $$\int\limits_{a}^b \alpha f+\beta g dx = \alpha\int\limits_a^b f + \beta\int\limits_a^b g$$
          \begin{proof}
              По формуле Ньютона-Лейбница $f\leftrightarrow F \quad g\leftrightarrow G \quad \alpha f + \beta g \leftrightarrow \alpha F + \beta G$
          \end{proof}
    \item Замена переменных: $f\in C[a,b]$
          $\varphi : \langle \alpha,\beta\rangle \to [a,b], \varphi\in C^1$

          $[p, q]\subset \langle \alpha,\beta\rangle$

          Тогда $$\int\limits_p^q f(\varphi(t))\varphi'(t)dt=\int\limits_{\varphi(p)}^{\varphi(q)} f(x)dx$$
          \begin{proof}
              $f\leftrightarrow F \quad f(\varphi(t))\varphi'(t)\leftrightarrow F(\varphi(t))$
          \end{proof}
          \begin{remark}
              \begin{enumerate}
                  \item $\varphi([p,q])$ может быть шире, чем ``$[\varphi(p), \varphi(q)]$''
                  \item $\varphi(p)$ может быть $>\varphi(q)$
              \end{enumerate}
          \end{remark}

    \item Интегрирование по частям

          $f|_a^b\defeqf(b)-f(a)$

          $f,g\in C^1[a,b]$

          $$\int\limits_a^b fg'=fg|_a^b-\int\limits_a^b f'g$$

          \begin{proof}
              $$(fg)'=f'g+fg'$$
              $$fg'=(fg)'-f'g$$
              Проинтегрируем по $[a,b]$
              $$\int\limits_a^b fg'=fg|_a^b-\int\limits_a^b f'g$$
          \end{proof}
\end{enumerate}

\begin{example}
    %<*неравенствочебышева>
    Неравенство Чебышева

    $f,g\in C[a,b]$ монот. возр.

    $$I_f := \frac{\int_a^b f}{b-a}$$
    Тогда
    $$I_f\cdot I_g\leq I_{fg}$$
    $$\int\limits_a^b f \int\limits_a^b g\leq (b-a)\int\limits_a^b fg$$
    %</неравенствочебышева>
\end{example}
%<*неравенствочебышеваproof>
\begin{proof}
    $x,y\in[a,b] : x\ge y \Rightarrow f(x)\ge f(y), g(x)\ge g(y)$
    $$(f(x)-f(y))(g(x)-g(y))\geq 0$$
    $$f(x)g(x)-f(y)g(x)-f(x)g(y)+f(y)g(y)\geq 0$$
    Интегрируем по $x$ по $[a,b]$ и делим на $b-a$:
    $$I_{fg}-f(y)I_g-g(y)I_f+f(y)g(y)\geq 0$$
    Интегрируем по $y$ по $[a,b]$ и делим на $b-a$:
    $$I_{fg}-I_fI_g-I_gI_f+I_{fg}\geq 0$$
\end{proof}
%</неравенствочебышеваproof>

\begin{example}
    %<*hдляпи>
    $$H_n:=\frac{1}{n!}\int\limits_{-\frac{\pi}{2}}^{\frac{\pi}{2}}\left(\frac{\pi^2}{4}-t^2\right)^n\cos t dt = \left[\begin{array}{ll}
                f = \left(\frac{\pi^2}{4}-t^2\right)^n           & g = \sin t     \\
                df =-2n\left(\frac{\pi^2}{4}-t^2\right)^{n-1}tdt & dg = \cos t dt
            \end{array}\right]=$$
    $$=\frac{1}{n!}\left(\frac{\pi^2}{4}-t^2\right)^n\sin t\Bigg|_{-\frac{\pi}{2}}^{\frac{\pi}{2}}+\frac{2}{(n-1)!}\int\limits_{-\frac{\pi}{2}}^{\frac{\pi}{2}} \left(\frac{\pi^2}{4}-t^2\right)^{n-1}t\sin t=$$
    $$=\left[\begin{array}{ll}
                f = \left(\frac{\pi^2}{4}-t^2\right)^{n-1}t                                                                    & g = -\cos t    \\
                df = \left(-2(n-1)\left(\frac{\pi^2}{4}-t^2\right)^{n-2}t^2 + \left(\frac{\pi^2}{4}-t^2\right)^{n-1} \right)dt & dg = \sin t dt
            \end{array}\right]=$$
    $$=\Bigg[df = \Bigg(-2(n-1)\left(\frac{\pi^2}{4}-t^2\right)^{n-2}t^2 + \left(\frac{\pi^2}{4}-t^2\right)^{n-1} + $$
            $$ + 2(n-1)\frac{\pi^2}{4}\left(\frac{\pi^2}{4}-t^2\right)^{n-2} - 2(n-1)\frac{\pi^2}{4}\left(\frac{\pi^2}{4}-t^2\right)^{n-2}\Bigg)dt\Bigg]=$$
    $$=\left[df = \left( (2n-2) \left(\frac{\pi^2}{4}-t^2\right)^{n-1} + \left(\frac{\pi^2}{4}-t^2\right)^{n-1} - (n-1)\frac{\pi^2}{2}\left(\frac{\pi^2}{4}-t^2\right)^{n-2} \right)dt\right]=$$
    $$=\left[df = \left((2n-1) \left(\frac{\pi^2}{4}-t^2\right)^{n-1} - (n-1)\frac{\pi^2}{2}\left(\frac{\pi^2}{4}-t^2\right)^{n-2} \right)dt\right]=$$
    $$=0+\frac{2}{(n-1)!}\left(\frac{\pi^2}{4}-t^2\right)^{n-1}t(-\cos t)\Bigg|_{-\frac{\pi}{2}}^{\frac{\pi}{2}}+$$
    $$+\frac{2}{(n-1)!}\int\limits_{-\frac{\pi}{2}}^{\frac{\pi}{2}}\left((2n-1)\left(\frac{\pi^2}{4}-t^2\right)^{n-1}-\frac{\pi^2}{2}(n-1)\left(\frac{\pi^2}{4}-t^2\right)^{n-2}\right)\cos t dt=$$
    $$=(4n-2)H_{n-1}-\pi^2H_{n-2}$$
    %</hдляпи>
\end{example}

\begin{theorem}
    %<*иррациональностьпи>

    Число $\pi$ --- иррационально
    %</иррациональностьпи>
\end{theorem}
%<*иррациональностьпиproof>
\begin{proof}
    Пусть $\pi=\frac{p}{q}; H_n$ задано выше

    $$H_n=(4n-2)H_{n-1}-\pi^2 H_{n-2}$$

    $$H_0 = 2, \quad H_1=\int\limits_{-\frac{\pi}{2}}^{\frac{\pi}{2}}(\frac{\pi^2}{4}-t^2)\cos t = 2\int\limits_{-\frac{\pi}{2}}^{\frac{\pi}{2}} t\sin t dt = 2t(-\cos t)\Big|_{\ldots}^{\ldots}+2\int\limits_{-\frac{\pi}{2}}^{\frac{\pi}{2}} \cos t = 4$$

    $$H_n=\ldots H_1+\ldots H_0 = P_n(\pi^2) \text{ --- многочлен с целыми коэффициентами, степень}\leq n$$
    $$q^{2n}P_n\left(\frac{p^2}{q^2}\right)=\text{ целое число }=q^{2n}H_n>0 \Rightarrow q^{2n}H_n\geq 1$$
    $$1\leq\frac{q^{2n}}{n!}\int\limits_{-\frac{\pi}{2}}^{\frac{\pi}{2}}\left(\frac{\pi^2}{4}-t^2\right)^n\cos t dt \leq \frac{q^{2n} 4^n}{n!}\pi\xrightarrow[n\to +\infty]{}0$$
    Противоречие.
\end{proof}
%</иррациональностьпиproof>

$f\leftrightarrow F$
$$\int\limits_{x_0}^x\frac{f^{(n)}(x_0)}{n!}(t-x_0)^ndt = \frac{F^{(n+1)}(x_0)}{(n+1)!}(x-x_0)^{n+1}$$
$$\frac{1}{\sqrt{1-x^2}}=1+\frac{1}{2}x^2+\frac{3}{8}x^4+o(x^4)$$
$$\arcsin t = t + \frac{1}{6}t^3 + \frac{3}{40} t^5 + o(t^5)$$

\begin{definition}
    %<*кусочнонепрерывнаяфункция>

    $f:[a,b] \to \R$, \textbf{кусочно непрерывна}

    $f$ --- непр. на $[a,b]$ за исключением конечного числа точек, в которых разрывы I рода
    \begin{example}
        $f(x) = [x], x\in[0, 2020]$
    \end{example}
    %</кусочнонепрерывнаяфункция>
\end{definition}

\begin{definition}
    %<*почтипервообразная>
    $F:[a,b]\to\R$ --- \textbf{почти первообразная} кусочно непрерывной функции $f$:

    $F$ --- непр. и $\exists F'(x) = f(x)$ всюду, кроме конечного числа точек
    \begin{example}
        $f=\sign x, x\in[-1,1]$

        $F:=|x|$
    \end{example}
    %</почтипервообразная>
\end{definition}

$f$ --- кус. непр.

$x_0=a<x_1\ldots<x_n=b$

$$\int\limits_a^b f := \sum\limits_{k=1}^n \int\limits_{x_{k-1}}^{x_k} f$$

Утверждение: Верна формула Ньютона-Лейбница
%<*ньютоналейбницадлякусочнонепрерывных>
$f$ --- кус. непр. на $[a,b]$, $F$ --- почти первообразная
%</ньютоналейбницадлякусочнонепрерывных>
%<*ньютоналейбницадлякусочнонепрерывныхproof>
\begin{proof}
    $$\int\limits_{a}^{b} f = \sum\int\limits_{x_{k-1}}^{x_k} f = \sum F(t)\Big|_{x_{k-1}}^{x_k}=\sum F(x_k)-F(x_{k-1})=F(b)-F(a)$$
\end{proof}
%</ньютоналейбницадлякусочнонепрерывныхproof>
\begin{example}
    %<*дискретноенеравенствочебышева>

    Дискретное неравенство Чебышева

    $a_1\leq a_2\leq \ldots \leq a_n, b_1\leq b_2\leq\ldots\leq b_n$
    $$\frac{1}{n}\left(\sum\limits_{i=1}^n a_i\right)\cdot \frac{1}{n}\left(\sum b_i\right)\leq \frac{1}{n}\sum a_ib_i$$
    %</дискретноенеравенствочебышева>
\end{example}
%<*дискретноенеравенствочебышеваproof>
\begin{proof}
    $$f(x)=a_i, x\in(i-1, i], i=1\ldots n \text{ --- задана на } (0,n]$$
    $$g(x) = \ldots b_i$$
    $$I_fI_g\leq I_{fg}$$
\end{proof}
%</дискретноенеравенствочебышеваproof>

\section{Приложение определенного интеграла}

%<*segm>
$Segm\langle a,b\rangle = \{[p, q] : [p,q]\subset\langle a,b\rangle\}$ --- множество всевозм. отрезков, лежащих в $\langle a,b\rangle$
%</segm>
\begin{definition}
    %<*функцияпромежутка>

    \textbf{Функция промежутка} $\Phi : Segm\langle a,b\rangle \to \R$
    %</функцияпромежутка>
\end{definition}
\begin{definition}
    %<*аддитивнаяфункцияпромежутка>

    \textbf{Аддитивная функция промежутка}: $\Phi$ --- функция промежутка и
    $$\forall [p,q]\in Segm\langle a,b\rangle \ \ \forall r : p < r < q \quad \Phi([p,q])=\Phi([p,r])+\Phi([r,q])$$
    %</аддитивнаяфункцияпромежутка>
\end{definition}
\begin{example}
    \begin{itemize}
        \item Цена куска колбасы от $p$ до $q$.
        \item Цена билета от станции $p$ до станции $q$.
    \end{itemize}
    Эти две функции обычно не аддитивны.
    \begin{itemize}
        \item $[p,q]\to\int_p^q f$
    \end{itemize}
\end{example}

\begin{definition}
    %<*плотностьаддитивнойфункциипромежутка>
    \textbf{Плотность аддитивной функции промежутка}: $f:\langle a,b\rangle \to\R$ --- плотность $\Phi$, если:
    $$\forall \delta\in Segm\langle a,b\rangle \quad \inf\limits_{x\in\delta} f(x)\cdot len_\delta\leq \Phi(\delta)\leq\sup f \cdot len_\delta$$
    %</плотностьаддитивнойфункциипромежутка>
\end{definition}

\begin{theorem}
    %<*овычисленииаддитивнойфункциипромежуткапоплотности>
    О вычислении аддитивной функции промежутка по плотности

    $f:\langle a,b\rangle \to \R$ --- непр. $\quad \Phi : Segm\langle a,b\rangle\to\R$

    $f$ --- плотность $\Phi$

    Тогда $\Phi([p,q])=\int\limits_p^q f, \quad\forall [p,q]\in Segm\langle a,b\rangle$
    %</овычисленииаддитивнойфункциипромежуткапоплотности>
\end{theorem}
%<*овычисленииаддитивнойфункциипромежуткапоплотностиproof>
\begin{proof}
    $$F(x):=\begin{cases}
            0           & , x=a \\
            \Phi([a,x]) & , x>a
        \end{cases} \text{ --- первообразная } f$$
    Это утверждение ещё не доказано, но если мы его докажем, то:
    $$\Phi([p,q])=\Phi[a,q]-\Phi[a,p]=F(q)-F(p)=\int\limits_p^qf$$
    Докажем утверждение:
    $$\frac{F(x+h)-F(x)}{h}=\frac{\Phi[a,x+h]-\Phi[a,x]}{h}=\frac{\Phi[x,x+h]}{h}=[0\leq\Theta\leq1]=f(x+\Theta h) \xrightarrow[h\to 0]{}f(x)$$
\end{proof}
%</овычисленииаддитивнойфункциипромежуткапоплотностиproof>

Тут последовал пример про нахождение площади круга, но мне лень.

\end{document}