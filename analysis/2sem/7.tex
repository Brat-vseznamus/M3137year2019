\documentclass[12pt, a4paper]{article}

\usepackage{lastpage}
\usepackage{mathtools}
\usepackage{xltxtra}
\usepackage{libertine}
\usepackage{amsmath}
\usepackage{amsthm}
\usepackage{amsfonts}
\usepackage{amssymb}
\usepackage{enumitem}
\usepackage{xcolor}
\usepackage[left=1.5cm, right=1.5cm, top=2cm, bottom=2cm, bindingoffset=0cm, headheight=15pt]{geometry}
\usepackage{fancyhdr}
\usepackage[russian]{babel}
% \usepackage[utf8]{inputenc}
\usepackage{catchfilebetweentags}
\usepackage{accents}
\usepackage{calc}
\usepackage{etoolbox}
\usepackage{mathrsfs}
\usepackage{wrapfig}

\providetoggle{useproofs}
\settoggle{useproofs}{false}

\pagestyle{fancy}
\lfoot{M3137y2019}
\rhead{\thepage\ из \pageref{LastPage}}

\newcommand{\R}{\mathbb{R}}
\newcommand{\Q}{\mathbb{Q}}
\newcommand{\C}{\mathbb{C}}
\newcommand{\Z}{\mathbb{Z}}
\newcommand{\B}{\mathbb{B}}
\newcommand{\N}{\mathbb{N}}

\newcommand{\const}{\text{const}}

\newcommand{\teormin}{\textcolor{red}{!}\ }

\DeclareMathOperator*{\xor}{\oplus}
\DeclareMathOperator*{\equ}{\sim}
\DeclareMathOperator{\Ln}{\text{Ln}}
\DeclareMathOperator{\sign}{\text{sign}}
\DeclareMathOperator{\Sym}{\text{Sym}}
\DeclareMathOperator{\Asym}{\text{Asym}}
% \DeclareMathOperator{\sh}{\text{sh}}
% \DeclareMathOperator{\tg}{\text{tg}}
% \DeclareMathOperator{\arctg}{\text{arctg}}
% \DeclareMathOperator{\ch}{\text{ch}}

\DeclarePairedDelimiter{\ceil}{\lceil}{\rceil}
\DeclarePairedDelimiter{\abs}{\left\lvert}{\right\rvert}

\setmainfont{Linux Libertine}

\theoremstyle{plain}
\newtheorem{axiom}{Аксиома}
\newtheorem{lemma}{Лемма}

\theoremstyle{remark}
\newtheorem*{remark}{Примечание}
\newtheorem*{exercise}{Упражнение}
\newtheorem*{consequence}{Следствие}
\newtheorem*{example}{Пример}
\newtheorem*{observation}{Наблюдение}

\theoremstyle{definition}
\newtheorem{theorem}{Теорема}
\newtheorem*{definition}{Определение}
\newtheorem*{obozn}{Обозначение}

\setlength{\parindent}{0pt}

\newcommand{\dbltilde}[1]{\accentset{\approx}{#1}}
\newcommand{\intt}{\int\!}

% magical thing that fixes paragraphs
\makeatletter
\patchcmd{\CatchFBT@Fin@l}{\endlinechar\m@ne}{}
  {}{\typeout{Unsuccessful patch!}}
\makeatother

\newcommand{\get}[2]{
    \ExecuteMetaData[#1]{#2}
}

\newcommand{\getproof}[2]{
    \iftoggle{useproofs}{\ExecuteMetaData[#1]{#2proof}}{}
}

\newcommand{\getwithproof}[2]{
    \get{#1}{#2}
    \getproof{#1}{#2}
}

\newcommand{\import}[3]{
    \subsection{#1}
    \getwithproof{#2}{#3}
}

\newcommand{\given}[1]{
    Дано выше. (\ref{#1}, стр. \pageref{#1})
}

\renewcommand{\ker}{\text{Ker }}
\newcommand{\im}{\text{Im }}
\newcommand{\grad}{\text{grad}}

\lhead{Конспект по матанализу}
\cfoot{}
\rfoot{Лекция 7}

\begin{document}

\section*{Несобственные интегралы}

\begin{definition}
    %<*допустимаяфункция>
    $f : [a, b) \to \R \quad -\infty<a<b\leq+\infty$

    $f$ \textbf{допустима}, если $f$ --- кусочно-непрерывна на $[a, A] \ \ \forall A\in (a, b)$
    %</допустимаяфункция>
\end{definition}
\begin{definition}
    %<*несобственныйинтеграл,cходимость,расходимость>
    $$\Phi(A):=\int_a^A f$$

    $$?\exists\lim\limits_{A\to b-0} \Phi(A)$$
    
    \begin{itemize}
        \item Если да, то это \textbf{несобственный интеграл} $\int\limits_{a}^{\rightarrow b} fdx$.
        \item Если этот предел конечный, то тот несобственный интеграл \textbf{сходится}.
        \item Если этот предел бесконечный или не существует, то несобственный интеграл \textbf{расходится}.
    \end{itemize}
    %</несобственныйинтеграл,cходимость,расходимость>
\end{definition}

\begin{example}
    $$\int_1^{+\infty} \frac{1}{x^p} dx=\lim_{A\to\infty}\int_1^{A} \frac{1}{x^p} dx$$
    $$\int_1^{A} \frac{1}{x^p} dx=\begin{cases}
        \frac{A^{1-p}-1^{1-p}}{1-p}, & p\not=1 \\
        \ln A - \ln 1, & p=1
    \end{cases}$$
    $$\lim_{A\to\infty}\int_1^{A} \frac{1}{x^p} dx=\begin{cases}
        \frac{1}{p-1}, & p>1 \\
        +\infty, & p<1 \\
        +\infty, & p=1
    \end{cases}$$

    $p>1$ --- интеграл сходится, $p\leq 1$ --- интеграл расходится.
\end{example}

\begin{example}
    $$\int_{\rightarrow 0}^{1} \frac{1}{x^p}dx$$

    $$\lim_{A\to\infty} \int_A^1 \frac{1}{x^p}dx = \begin{cases}
        \text{кон}., & p<1 \\
        +\infty, & p\geq 1
    \end{cases}$$

    Можно разбивать интеграл с $>1$ причиной несобственности на части, где только одна причина:
    $$\int_0^{+\infty} \frac{1}{x^p}dx = \int_0^{10} + \int_{10}^{+\infty}$$
    Если все интегралы в правой части сходятся, то в левой части тоже.
\end{example}

\begin{example}
    $$\int_{-1}^1 \frac{1}{x}dx=\int_{-1}^0 \frac{1}{x}dx + \int_{\rightarrow 0}^{1} \frac{1}{x}dx$$
    $$\int_{-1}^0 \frac{1}{x}dx = -\infty \quad \int_{\rightarrow 0}^{1} \frac{1}{x}dx=+\infty$$
    Итого $\int_{-1}^1 \frac{1}{x}dx$ расходится. Хочется сократить бесконечности, особенно если посмотреть на график $\frac{1}{x}$ --- он симметричен. Кажется, что $\int_{-1}^1 \frac{1}{x}dx=0$. Однако мы все равно считаем этот интеграл расходящимся. Это можно обосновать так: если нагреть одну сторону стула до +200 градусов, а другую охладить до -170, то вы не захотите на нем сидеть, хотя средняя температура адекватная.
\end{example}

\subsection*{Свойства}

\subsubsection*{Критерий Больцано-Коши}

%<*критерийбольцанокошисходимости>
$$\lim_{A\to b-0} \int_a^A \text{ кон.} \Leftrightarrow \forall \varepsilon > 0 \ \ \exists \Delta\in(a,b) \ \ \forall A, B\in(\Delta, b) \quad \left|\int_A^B f\right|<\varepsilon$$
%</критерийбольцанокошисходимости>
%<*критерийбольцанокошисходимостиproof>
\begin{proof}
    Тривиально из определения предела.
\end{proof}
%</критерийбольцанокошисходимостиproof>

\begin{consequence}
    Если $\exists A_n, B_n \to b-0 \ \ \int\limits_{A_n}^{B_n} f \not\xrightarrow{n\to+\infty} 0$, то $\int\limits_a^{\rightarrow b} f$ расходится.
\end{consequence}

\begin{example}
    $$\int_1^{+\infty} \sin\sqrt x dx$$
    Это синусоида с увеличивающимся периодом.

    Чтобы доказать, что интеграл расходится, возьмём $A_n, B_n$ такие что $ \int\limits_{A_n}^{B_n} \sin\sqrt x dx \not\xrightarrow{n\to+\infty} 0$.

    $$A_n:=(2\pi n+\frac{\pi}{6})^2 \quad B_n:=\left(2\pi(n+1)-\frac{\pi}{6}\right)^2$$
    $$\int_{A_n}^{B_n} \sin\sqrt x dx \geq \frac{1}{2}(B_n-A_n)\to\infty$$
\end{example}

%<*свойстванесобственногоинтеграла>
\subsubsection*{Аддитивность по промежутку}

$f$ --- допустима. $[a, b) \quad c\in (a,b)$

Тогда $\int_a^{\rightarrow b} f$ и $\int_c^{\rightarrow b} f$ --- сходятся/расходятся одновременно и, если сходятся, $\int_a^{\rightarrow b} f = \int_a^c f + \int_c^{\rightarrow b} f$

Берем $A>c$ $\int_a^A = \int_a^c + \int_c^A$

\begin{consequence}
    $f$ --- допустима. $[a, +\infty)$, $\int_a^{+\infty} f$ --- сходится. Тогда
    $$\int_A^{+\infty} f \xrightarrow{A\to+\infty} 0$$
    Это называется ``хвост''.
\end{consequence}

\subsubsection*{Линейность}

$f, g$ --- допустима $\int_a^{\rightarrow b} f, \int_a^{\rightarrow b} g$ --- сход.

$\lambda\in\R$

Тогда $\lambda f, f\pm g$ --- допустима b $\int_a^{\rightarrow b} \lambda f, \int_a^{\rightarrow b} f\pm g$ --- сходятся.

$$\int_a^{\rightarrow b} \lambda f = \lambda \int_a^{\rightarrow b} f \quad\quad \int_a^{\rightarrow b} f\pm g = \int_a^{\rightarrow b} f \pm \int_a^{\rightarrow b} g$$

\begin{proof}
    Тривиально.
\end{proof}

\subsubsection*{Интегрирование неравенств}

$f, g$ --- доп., $\int_a^{\rightarrow b} f, \int_a^{\rightarrow b} g$ --- существуют в $\overline \R$

$f\leq g$ на $[a, b)$. Тогда
$$\int_a^{\rightarrow b} f \leq \int_a^{\rightarrow b} g$$

Очевидно: $\int_a^A f\leq \int_a^A g, A\to b-0$

\subsubsection*{Интеграл произведения}

$f, g$ --- дифф. $[a, b); f',g'$ --- допустимы. Это эквивалентно $f,g\in C^1[a, b)$.

Тогда* $$\int_a^{\rightarrow b} fg' = fg\Bigg|_a^{\rightarrow b} - \int_a^{\rightarrow b} f'g$$

* значит, что если два из трех  пределов существуют, то существует третий и выполняется равенство.

\subsubsection*{Интеграл композиции}

$\varphi : [\alpha, \beta) \rightarrow \langle A, B\rangle, \varphi\in C^1$

$f : \langle A, B\rangle \rightarrow \R, f$ --- непр., $\exists \varphi(\beta-0)\in\overline \R$

Тогда*
$$\int_\alpha^{\rightarrow \beta} f(\varphi(t))\varphi'(t)dt = \int_{\varphi(\alpha)}^{\rightarrow \varphi(\beta-0)} f(x)dx$$

\begin{remark}
    $f$ --- кусочно непрерывна на $[a,b]$. $f$ можно также рассматривать на $[a, b)$. Тогда
    $$\int_a^{\rightarrow b} f = \int_a^b f$$
\end{remark}
%</свойстванесобственногоинтеграла>

Упраздняем ``$\rightarrow$''.

\subsection*{Признаки сходимости несобственных интегралов}

$f$ --- допустима на $[a, b), f\geq 0, \Phi(A) = \int_a^A fdx$

$\int_a^b f$ --- сходится $\Leftrightarrow \Phi$ ограничена.

\begin{proof}
    $\int_a^b f$ --- сх. $\Leftrightarrow \lim\limits_{A\to b-0} \Phi(A)$ кон. $\Leftrightarrow \Phi$ --- огр.
\end{proof}

\begin{lemma}
    Признак сравнения:

    %<*признаксравнения>
    $f,g \ge 0$, допустимы на $[a, b)$
    \begin{enumerate}
        \item $f\le g$ на $[a, b)$. Тогда:
        \begin{enumerate}
            \item $\int_a^b g$ --- сходится $\Rightarrow \int_a^b f$ --- сходится
            \item $\int_a^b f$ --- расходится $\Rightarrow \int_a^b g$ --- расходится
        \end{enumerate}
        \item $\exists\lim\limits_{x\to b-0} \frac{f(x)}{g(x)} = l<+\infty:$
        \begin{enumerate}
            \item $\int_a^b g$ --- сходится $\Rightarrow \int_a^b f$ --- сходится
            \item $\int_a^b f$ --- расходится $\Rightarrow \int_a^b g$ --- расходится
        \end{enumerate} 
    \end{enumerate}
    %</признаксравнения>
\end{lemma}
%<*признаксравненияproof>
\begin{proof}
    \begin{enumerate}
        \item $\Phi(A):=\int_a^A f, \Psi(A)=\int_a^A g$
        
        $0\le \Phi(A)\le \Psi(A)$
        
        \begin{enumerate}
            \item $\int_a^b g$ --- сходится $\Rightarrow$ $\Psi$ огр. $\Rightarrow \Phi$ огр. $\Rightarrow \int_a^b f$ --- сходится
            \item $\int_a^b f$ --- расходится $\Rightarrow$ $\Phi$ неогр. $\Rightarrow \Psi$ неогр. $\Rightarrow \int_a^b g$ --- расходится
        \end{enumerate}
        
        \item $l < +\infty \xRightarrow{def} \exists a_1 : \forall x > a_1 \ \ 0\le\frac{f(x)}{g(x)}\le l+1 \Rightarrow f(x) \le g(x)(l+1)$, дальше тривиально \textit{(предположительно по пункту 1.)}
    \end{enumerate}
\end{proof}
%</признаксравненияproof>

%<*признаксравнения2>
\begin{remark}
    $l>0$:
    $$\exists a_2 : \forall x > a_2 \quad \frac{l}{2}<\frac{f(x)}{g(x)}$$
    \begin{enumerate}
        \item $\int_a^b f$ --- сходится $\Rightarrow \int_a^b g$ --- сходится
        \item $\int_a^b g$ --- расходится $\Rightarrow \int_a^b f$ --- расходится
    \end{enumerate}
\end{remark}

\begin{consequence}
    Если $+\infty>l>0$, то:
    \begin{enumerate}
        \item $\int_a^b f$ --- сходится $\Leftrightarrow \int_a^b g$ --- сходится
        \item $\int_a^b f$ --- расходится $\Leftrightarrow \int_a^b g$ --- расходится
    \end{enumerate}
\end{consequence}
%</признаксравнения2>

\begin{example}
    $$\int_0^{+\infty}\frac{\cos^2 x}{1+x^2}dx \text{ сходится?}$$
    С трюком:
    $$\frac{\cos^2 x}{1+x^2}\leq \frac{1}{1+x^2} \quad \int_0^{+\infty} \frac{1}{1+x^2}=\arctan\Bigg|_0^{+\infty}=\frac{\pi}{2}$$
    Более цинично:
    $$\frac{\cos^2 x}{1+x^2}\leq \frac{1}{x^2} \text{ на } [2020, +\infty)$$
    $$\int_{2020}^{+\infty} \frac{1}{x^2} \text{ cходится } \Rightarrow \int_{2020}^{+\infty} f \text{ cходится }$$
\end{example}


\begin{example}
    Этот пример будет на экзамене.

    %<*изучениеинтеграла>
    При каких $\alpha$ и $\beta$ сходится:
    $$\int_{10}^{+\infty} \frac{dx}{x^\alpha (\ln x)^\beta}$$

    Мы знаем, что $\int_1^\infty \frac{dx}{x^p}$ сходится при $p>1$ и расходится при $p\leq 1$.

    При $\alpha>1, \beta>0$
    $$\frac{1}{x^\alpha (\ln x)^\beta} < \frac{1}{x^\alpha}$$
    Таким же образом можно еще что-то выяснить, но мы так делать не будем. Вместо этого воспользуемся методом \textbf{``удавливание логарифма''}

    \begin{enumerate}
        \item $\alpha > 1 \quad \alpha=1+2a, a>0$
        $$0\leq \frac{1}{x^{1+2a}(\ln x)^\beta}=\frac{1}{x^{1+a}}\cdot\frac{1}{x^a (\ln x)^\beta}$$
        $$\beta\geq 0 \quad x^a(\ln x)^\beta \to+\infty$$
        $$b:=-\beta \quad \beta < 0 \quad x^a(\ln x)^\beta = \frac{x^a}{(\ln x)^b}=\left(\frac{x^{\frac{a}{b}}}{\ln x}\right)^b\xrightarrow[x\to\infty]{}\left[\frac{\infty}{\infty}\right]\xrightarrow{\text{лопиталь}}\frac{\frac{a}{b}x^{\frac{a}{b} - 1}}{\frac{1}{x}}\to+\infty$$
        $$x^a (\ln x)^\beta \xrightarrow{x\to+\infty} +\infty \Rightarrow \frac{1}{x^{1+a}}\cdot\frac{1}{x^a (\ln x)^\beta} < \frac{1}{x^{1+a}} \text{ --- сходится}$$

        \item $\alpha < 1 \quad \alpha=1-2a, a>0$
        $$\frac{1}{x^{1-2a}(\ln x)^\beta}=\frac{1}{x^{1-a}}\cdot\frac{x^a}{(\ln x)^\beta}>\frac{1}{x^{1-a}}$$
        \item $\alpha = 1$
        $$\int_{10}^{+\infty} \frac{dx}{x (\ln x)^\beta} \stackrel{y=\ln x}{=}\int_{\ln 10}^{+\infty} \frac{dy}{y^\beta}$$
        Сходится при $\beta>1$, расходится при $\beta\le 1$
    \end{enumerate}
    %</изучениеинтеграла>
\end{example}

\subsection*{Гамма-функция Эйлера}

%<*гаммафункция>
$\Gamma$ --- \textbf{гамма-функция Эйлера}
$$\Gamma(t)=\int_0^{+\infty} x^{t-1}e^{-x}dx$$
%</гаммафункция>

%<*свойствагаммафункции>
\subsubsection*{Область определения}

\begin{enumerate}
    \item $\int_1^{+\infty} x^{t-1}e^{-x}dx$ --- сходится при всех $t\in\R^+$:
    $$\int_1^{+\infty} e^{-x} dx = -e^{-x}\Bigg|_1^{+\infty} = e$$
    $$0\leq x^{t-1}e^{-x}\leq x^{t-1}e^{-\frac{x}{2}}e^{-\frac{x}{2}}$$
    $$x^{t-1}e^{-\frac{x}{2}}\xrightarrow{x\to+\infty} 0\Rightarrow \text{ при больших } x \ \ x^{t-1}e^{-\frac{x}{2}}e^{-\frac{x}{2}}\leq e^{-\frac{x}{2}}$$
    \item $\int_{\rightarrow 0}^1 x^{t-1}e^{-x}dx$
    $$x^{t-1}e^{-x}\equ_{x\to0} x^{t-1} \quad t>0 \text{ сходится, } t\leq 0 \text{ расходится}$$
\end{enumerate}

\subsubsection*{Выпуклость}

Подынтегральное выражение как функция от $t$ является выпуклой функцией (при $x\geq 0$)

$$t\mapsto x^{t-1} e^{-x} = f_x(t)$$
$$f(\alpha t_1 + (1-\alpha)t_2)\leq \alpha f_x(t_1) + (1-\alpha)f_x(t_2)$$
$$\int_0^{+\infty} f_x dx \leq \alpha \int_0^{+\infty} f_x(t_1)dx + (1-\alpha) \int_0^{\infty} f_x(t_2) dx$$
Определение выпуклости: $$x^{(\alpha t_1 + (1-\alpha)t_2)-1}e^{-x}\leq \alpha x^{t_1-1}e^{-x}+(1-\alpha)x^{t_2-1}e^{-x}$$
Зафиксируем $\alpha, t_1, t_2$. Проинтегрируем по $x$ от $0$ до $+\infty$:
$$\Gamma(\alpha t_1 + (1-\alpha)t_2)\leq \alpha\Gamma(t_1) + (1-\alpha)\Gamma(t_2)$$

$\Gamma$ --- выпуклая $\Rightarrow \Gamma$ --- непрерывная

\subsubsection*{Третье свойство}

$\Gamma(t+1)=t\Gamma(t)$
$$\Gamma(t+1)=\int_0^{+\infty} x^t e^{-x}dx = -x^t e^{-x}\Bigg|_0^{+\infty} + t\int_0^{+\infty} x^{t-1}e^{-x} dx=0+t\Gamma(t)$$

\begin{consequence}
    $\Gamma(n+1)=n!$
\end{consequence}
\begin{proof}
    $$\Gamma(n+1)=n\Gamma(n)=n(n-1)\Gamma(n-1)=\ldots = n(n-1)\cdots 1 \Gamma(1)=n!$$
\end{proof}

\subsubsection*{Четвертое свойство}

$$\Gamma(t) = \frac{\Gamma(t+1)}{t}\equ_{t\to+0}\frac{1}{t}$$

\subsubsection*{Пятое свойство}
%</свойствагаммафункции>

%<*интегралэйлерапуассона>
$$\Gamma\left(\frac{1}{2}\right)=\int_0^{+\infty} \frac{1}{\sqrt x} e^{-x} dx \stackrel{x=y^2}{=}2\int_0^{+\infty} e^{-y^2}dy=2\frac{1}{2}\sqrt\pi \text{ --- интеграл Эйлера-Пуассона}$$
%</интегралэйлерапуассона>

%<*интегралэйлерапуассонаproof>
\begin{proof}
    $$1-x^2\leq e^{-x^2}\leq \frac{1}{1+x^2} \quad \forall x\in\R$$
    Оба неравенства следуют из неравенства $e^t \geq 1+t \ \ \forall t$.

    Зафиксируем $n\in\N$
    $$(1-x^2)^n\leq e^{-nx^2}\leq \left(\frac{1}{1+x^2}\right)^{n}$$
    $$\int_0^1 (1-x^2)^n dx \leq \int_0^1 e^{-nx^2} \leq \int_0^{+\infty} e^{-nx^2} \leq \int_0^{+\infty} \frac{1}{(1+x^2)^{n}}$$
    Казалось бы, переход от интеграла $\int_0^1$ к $\int_0^{+\infty}$ очень грубый, но это не так.
    $$\int_0^{+\infty} e^{-nx^2}\stackrel{y=\sqrt n x}{=}\frac{1}{\sqrt n} I$$
    $$\int_0^1 (1-x^2)^n dx\stackrel{x=\cos y}{=} \int_0^{\frac{\pi}{2}} \sin^{2n+1}ydy$$
    $$\int_0^{+\infty} \frac{1}{(1+x^2)^{n}}\stackrel{x=\tg y}{=}\int_0^{\frac{\pi}{2}} (\cos y)^{2n-2} dy = \int_0^{\frac{\pi}{2}} (\sin t)^{2n-2} dt$$
    $$\sqrt n \int_0^{\frac{\pi}{2}} \sin^{2n+1}ydy \leq I \leq \sqrt{n} \int_0^{\frac{\pi}{2}} (\sin t)^{2n-2} dt$$

    $$\int_0^{\frac{\pi}{2}} \sin^n x dx = \frac{(n-1)!!}{n!!}\begin{cases}
        \frac{\pi}{2} , & n \text{ чет.} \\
        1 , & n \text{ нечет.}
    \end{cases}$$
    $$\sqrt n \frac{(2n)!!}{(2n+1)!!} \leq I \leq \frac{(2n-3)!!}{(2n-2)!!}\frac{\pi}{2}\sqrt n$$
    По формуле Валлиса $\frac{(2k)!!}{(2k-1)!!}\cdot\frac{1}{\sqrt k}\to\sqrt \pi$:
    $$\sqrt n \frac{(2n)!!}{(2n+1)!!}=\left(\frac{1}{\sqrt n}\frac{(2n)!!}{(2n-1)!!}\right)\frac{n}{2n+1}\to\frac{\sqrt\pi}{2}$$
    $$\frac{(2n-3)!!}{(2n-2)!!}\frac{\pi}{2}\sqrt n = \frac{\frac{\pi}{2}\sqrt n\frac{1}{\sqrt {n-1}}}{\frac{(2n-2)!!}{(2n-3)!!}\frac{1}{\sqrt {n-1}}}\to \frac{\sqrt \pi}{2}$$
\end{proof}
%</интегралэйлерапуассонаproof>

\end{document}