\documentclass[12pt, a4paper]{article}

\usepackage{lastpage}
\usepackage{mathtools}
\usepackage{xltxtra}
\usepackage{libertine}
\usepackage{amsmath}
\usepackage{amsthm}
\usepackage{amsfonts}
\usepackage{amssymb}
\usepackage{enumitem}
\usepackage{xcolor}
\usepackage[left=1.5cm, right=1.5cm, top=2cm, bottom=2cm, bindingoffset=0cm, headheight=15pt]{geometry}
\usepackage{fancyhdr}
\usepackage[russian]{babel}
% \usepackage[utf8]{inputenc}
\usepackage{catchfilebetweentags}
\usepackage{accents}
\usepackage{calc}
\usepackage{etoolbox}
\usepackage{mathrsfs}
\usepackage{wrapfig}

\providetoggle{useproofs}
\settoggle{useproofs}{false}

\pagestyle{fancy}
\lfoot{M3137y2019}
\rhead{\thepage\ из \pageref{LastPage}}

\newcommand{\R}{\mathbb{R}}
\newcommand{\Q}{\mathbb{Q}}
\newcommand{\C}{\mathbb{C}}
\newcommand{\Z}{\mathbb{Z}}
\newcommand{\B}{\mathbb{B}}
\newcommand{\N}{\mathbb{N}}

\newcommand{\const}{\text{const}}

\newcommand{\teormin}{\textcolor{red}{!}\ }

\DeclareMathOperator*{\xor}{\oplus}
\DeclareMathOperator*{\equ}{\sim}
\DeclareMathOperator{\Ln}{\text{Ln}}
\DeclareMathOperator{\sign}{\text{sign}}
\DeclareMathOperator{\Sym}{\text{Sym}}
\DeclareMathOperator{\Asym}{\text{Asym}}
% \DeclareMathOperator{\sh}{\text{sh}}
% \DeclareMathOperator{\tg}{\text{tg}}
% \DeclareMathOperator{\arctg}{\text{arctg}}
% \DeclareMathOperator{\ch}{\text{ch}}

\DeclarePairedDelimiter{\ceil}{\lceil}{\rceil}
\DeclarePairedDelimiter{\abs}{\left\lvert}{\right\rvert}

\setmainfont{Linux Libertine}

\theoremstyle{plain}
\newtheorem{axiom}{Аксиома}
\newtheorem{lemma}{Лемма}

\theoremstyle{remark}
\newtheorem*{remark}{Примечание}
\newtheorem*{exercise}{Упражнение}
\newtheorem*{consequence}{Следствие}
\newtheorem*{example}{Пример}
\newtheorem*{observation}{Наблюдение}

\theoremstyle{definition}
\newtheorem{theorem}{Теорема}
\newtheorem*{definition}{Определение}
\newtheorem*{obozn}{Обозначение}

\setlength{\parindent}{0pt}

\newcommand{\dbltilde}[1]{\accentset{\approx}{#1}}
\newcommand{\intt}{\int\!}

% magical thing that fixes paragraphs
\makeatletter
\patchcmd{\CatchFBT@Fin@l}{\endlinechar\m@ne}{}
  {}{\typeout{Unsuccessful patch!}}
\makeatother

\newcommand{\get}[2]{
    \ExecuteMetaData[#1]{#2}
}

\newcommand{\getproof}[2]{
    \iftoggle{useproofs}{\ExecuteMetaData[#1]{#2proof}}{}
}

\newcommand{\getwithproof}[2]{
    \get{#1}{#2}
    \getproof{#1}{#2}
}

\newcommand{\import}[3]{
    \subsection{#1}
    \getwithproof{#2}{#3}
}

\newcommand{\given}[1]{
    Дано выше. (\ref{#1}, стр. \pageref{#1})
}

\renewcommand{\ker}{\text{Ker }}
\newcommand{\im}{\text{Im }}
\newcommand{\grad}{\text{grad}}

\lhead{Конспект по матанализу}
\cfoot{}
\rfoot{Лекция 13}

\renewcommand{\thesubsection}{\arabic{subsection}.}

\begin{document}

Рассмотрим такой ряд:

$$1-1+\frac{1}{2}+\frac{1}{2}-\frac{1}{2}+\frac{1}{4}+\frac{1}{4}-\frac{1}{2}+\frac{1}{4}+\frac{1}{4}\underbrace{-\frac{1}{4}+\frac{1}{8}+\frac{1}{8}}_{4 \text{ раза}}+\underbrace{-\frac{1}{8}+\frac{1}{16}+\frac{1}{16}}_{8 \text{ раз}}+\ldots$$

Сходится ли этот ряд? Да, потому что можно разбить на скобки из 3х слагаемых (кроме $1$), каждая из которых $=0$.

Рассмотрим похожий ряд:
$$-1+1-\frac{1}{2}-\frac{1}{2}+\frac{1}{2}-\frac{1}{4}-\frac{1}{4}+\frac{1}{2}-\frac{1}{4}-\frac{1}{4}\underbrace{+\frac{1}{4}-\frac{1}{8}-\frac{1}{8}}_{4 \text{ раза}}+\underbrace{\frac{1}{8}-\frac{1}{16}-\frac{1}{16}}_{8 \text{ раз}}+\ldots=-1$$
Произошла магия --- сумма ряда $=-1$, т.к. $b_n=-a_n$, где $b_n$ --- слагаемое этого ряда, $a_n$ --- прошлого ряда. Но мы просто переставили слагаемые предыдущего ряда $\Rightarrow$ перестановка бесконечного числа слагаемых меняет результат.

\begin{definition}
    $\sum a_k, w:\N\to\N$ --- биекция

    $b_k:=a_{w(k)}$, $\sum b_k$ называется \textbf{перестановкой} ряда $\sum a_k$
\end{definition}
\begin{theorem}
    %<*оперестановкеслагаемых>
    Ряд $A$ абсолютно сходится, тогда его перестановка $B$ тоже абсолютно сходится и имеет ту же сумму.
    %</оперестановкеслагаемых>
\end{theorem}
%<*оперестановкеслагаемыхproof>
\begin{proof}
    \begin{enumerate}
        \item $a_k\ge 0$
        $$S_n^{(b)}=b_1+\ldots +b_n=a_{w(1)}+\ldots+a_{w(n)}\le S^{(a)}_N, N=\max(w(1)\ldots w(n))$$
        Предельный переход: $S^{(b)}\le S^{(a)}$

        Т.к. $A$ --- перестановка $B$, то $S^{(a)}\le S^{(b)} \Rightarrow S^{(a)}=S^{(b)}$

        \item Общий случай
        
        $a_k^+=\max(a_k, 0), a_k^-=\max(-a_k, 0)$

        $\sum b_k^+$ --- перестановка $\sum a_k^+$; $\sum b_k^-$ --- перестановка $\sum a_k^-$

        Срезки сходятся по пункту 1., в силу абсолютной сходимости $\sum a_k^+$ и $\sum a_k^-$ конечны $\Rightarrow S^{(a)}=S^{(b)}$
    \end{enumerate}
\end{proof}
%</оперестановкеслагаемыхproof>

\begin{theorem}
    Римана.

    $\sum a_k$ --- сходится неабсолютно. Тогда:
    \begin{enumerate}
        \item $\exists$ перестановка ряда $A$, которая не имеет предела частичной суммы
        \item $\forall S\in\overline\R$ $\exists$ перестанвка ряда $A$ с суммой $S$
    \end{enumerate}
\end{theorem}
\begin{proof}
    2. Т.к. $\sum a_k$ сходится неабсолютно, существует две кучи - одна из положительных $a_k$, другая из отрицательных. Обе кучи бесконечные и имеют бесконечную сумму. Тогда будем брать элементы из положительной кучи, пока частичная сумма $<S$, потом берем элементы из отрицательной кучи, пока сумма $>S$. Получаем ряд, осциллирующий вокруг $S$. Если есть нулевые элементы, то будем их добавлять в сумму, когда меняем направление.

    1. Будем осциллировать не вокруг $S$, а между $T$ и $S$.
\end{proof}

\begin{example}
    $$\sum_{n=1}^{+\infty}\frac{1}{n(2n-1)}=\sum_{n=1}^{+\infty}\left(\frac{2}{2n-1}-\frac{1}{n}\right)=2\left(1+\frac{1}{3}+\frac{1}{5}+\frac{1}{7}+\ldots\right)-1-\frac{1}{2}-\frac{1}{3}-\frac{1}{4}-\ldots=$$
    $$=2-1-\frac{1}{2}+\frac{2}{3}-\frac{1}{3}-\frac{1}{4}+\frac{2}{5}-\frac{1}{5}+\ldots=$$
    $$=1-\frac{1}{2}+\frac{1}{3}-\frac{1}{4}+\frac{1}{5}-\frac{1}{6}+\ldots$$
    Разложим $f(x)=\ln(1+x)$ по Тейлору:
    $$\ln(1+x)=x-\frac{x^2}{2}+\frac{x^3}{3}+\ldots+\frac{1}{(n+1)!}f^{(n+1)}(c)x^n$$
    $$f^{(n+1)}(c)=\frac{(-1)^n n!}{(1+c)^{n+1}}$$
    $$R_n\le \frac{1}{n+1}\frac{1}{(1+c)^{n+1}}\le \frac{1}{n+1}$$
    $$\ln 2 = 1-\frac{1}{2}+\frac{1}{3}-\ldots$$
    Проблема: сумма этого ряда должна быть $>1$, но мы получили обратное. Это произошло, потому что мы переставили слагаемые неабсолютно сходящегося ряда.
\end{example}

\section*{Произведение рядов}

$(a_1+\ldots+a_k)(b_1+\ldots+b_l)=\sum\sum a_ib_j$

\begin{definition}
    %<*произведениерядов>
    $\sum a_k, \sum b_k$

    $\gamma : \N\to\N\times\N$ --- биекция, $\gamma(k)=(\varphi(k),\psi(k))$

    \textbf{Произведение рядов} $A$ и $B$ --- ряд $\sum_{k=1}^{+\infty} a_{\varphi(k)}b_{\psi(k)}$
    %</произведениерядов>
\end{definition}

\begin{theorem}
    Коши.

    %<*опроизведениирядов>
    Пусть ряды $\sum a_k, \sum b_k$ абсолютно сходятся. Тогда $\forall$ биекции $\gamma:\N\to\N\times\N$ произведение рядов абсолютно сходится и его сумма $=AB$ 
    %</опроизведениирядов>
\end{theorem}
%<*опроизведениирядовproof>
\begin{proof}
    $\sum |a_k|=A^*, \sum |b_k|=B^*, 0\le A^*,B^*<+\infty$

    $$\sum_{k=1}^N |a_{\varphi(x)}b_{\psi(x)}|\le \sum_{i=1}^{M}|a_i|\sum_{j=1}^L |b_j|\le A^*B^*$$
    $$M:=\max(\varphi(1)\ldots \varphi(N)) \quad N:=\max(\psi(1)\ldots \psi(N))$$
    Итого произведение сходится абсолютно.
    
    Произведение для $\overline\gamma\not=\gamma$ есть перестановка произведения для $\gamma \Rightarrow \forall \gamma$ произведение рядов имеет одинаковую сумму.

    Возьмём $\gamma$ такое, что оно обходит точки $\N\times\N$ ``по квадратам'', т.е. не заходит в следующий квадрат, пока не обошло предыдущий. Тогда:
    $$\sum_{k=1}^{n^2}a_{\varphi(k)}b_{\psi(k)}=\sum_{i=1}^n a_i \sum_{j=1}^n b_j\xrightarrow[n\to+\infty]{}AB$$
\end{proof}
%</опроизведениирядовproof>

\begin{example}
    %<*произведениестепенныхрядов>
    $x\in\R, x$ --- фиксированный
    $$\sum_{k=0}^{+\infty} a_kx^k \sum_{j=0}^{+\infty} b_jx^j = \sum_{n=0}^{+\infty} c_n x^n$$
    $$c_n=a_0b_n+a_1b_{n-1}+\ldots+a_nb_0$$
    Это называется произведение степенных рядов.
    %</произведениестепенныхрядов>
\end{example}

\section*{Функции нескольких переменных}

\begin{lemma}
    О дифференциируемости отображения и его координатных функций.

    %<*одифференциируемсотиотображения>
    $F : E\subset \R^m \to \R^n \quad a\in Int E$

    $F(x) = (f_1(x),f_2(x)\ldots f_n(x))$. Тогда:
    \begin{enumerate}
        \item $F$ --- дифф. в $a \Leftrightarrow$ все $f_i$ дифференциируемы в $a$
        \item $\forall i = 1\ldots n \quad i$-я строка матрицы Якоби $F$ есть матрица Якоби $f_i$
    \end{enumerate}
    %</одифференциируемсотиотображения>
\end{lemma}
%<*одифференциируемсотиотображенияproof>
\begin{proof}
    $$F(x) = F(a) + L(x-a) + \varphi(x)|x-a|$$
    $$\forall i \quad f_i(x) = f_i(a) + (L_{1i}, L_{2i}, \ldots L_{mi})\cdot(x-a) + \varphi_i(x)|x-a|$$
    Очевидно оба выражения эквивалентны.
\end{proof}
%</одифференциируемсотиотображенияproof>

\begin{example}
    \begin{enumerate}
        \item $F=\const : \R^m\to\R^l$
        
        $\forall x\in\R^m$ $F$ дифф. в $x$, $F'(x)=\mathbf 0$
        $$F(x)=F(a) + \underbrace{L}_{\mathbf 0}(x-a) + \underbrace{\varphi(x)}_{\mathbf 0}|x-a|$$

        \item $A : \R^m \to \R^l$ --- линейный оператор
        
        $\forall x\in\R^m$ $A$ дифф. в $x$, $A'(x)=A$
        $$A(x)=Aa+A(x-a)+\underbrace{\varphi(x)}_{\mathbf 0}|x-a|$$

        \item $F(x)=v_0+Ax$ --- афинное отображение. $F'(x)=A$
    \end{enumerate}
\end{example}

\subsection{Частные производные}

\begin{definition}
    %<*частнаяпроизводная>
    $f:E\subset \R^m \to \R, a\in Int E$
    
    Фиксируем $k\in\{1\ldots m\}$ $\varphi_k(t):=f(a_1,a_2\ldots t\ldots a_m)$
    
    $\lim\limits_{h\to0}\frac{\varphi_k(a_k+h)-\varphi_k(a_k)}{h} = \varphi_k'(a_k)$ называется \textbf{частной производной} функции $f$ в точке $a$
    %</частнаяпроизводная>
\end{definition}

$$f_k'(a)=\frac{\partial f}{\partial x_k}(a)=D_k f(a)$$

$$\frac{\partial f}{\partial x_k}(a) = \lim_{h\to0}\frac{f(a_1,a_2\ldots a_{k}+h\ldots a_m)-f(a_1\ldots a_m)}{h}$$

\begin{example}
    \begin{enumerate}
        \item $$f(x,y)=x+(y-\alpha)\arctg\frac{x^2+y^2}{\sqrt{xy}+1}$$
        $$\frac{\partial f}{\partial x}(1,1)=1$$
        \item $$f(x,y)=\begin{cases}
            \frac{2xy}{x^2+y^2} &, (x,y)\not=(0,0) \\
            0 &, (x,y)=(0,0)
        \end{cases}$$
        $$\frac{\partial f}{\partial x}(0,0)=\frac{df}{dx}(x,0)=0 \quad \frac{\partial f}{\partial y}(0,0)=0$$
    \end{enumerate}
\end{example}

\begin{theorem}
    Необходимое условие дифференциируемости.

    %<*необходимоеусловиедифференциируемости>
    $f:E\subset \R^m \to\R, a\in IntE, f$ --- дифф. $a$

    Тогда $\exists f_1'(a),\ldots,f_m'(a)$ и матрица Якоби $f$ в точке $a=(f_1'(a),\ldots,f_m'(a))$
    %</необходимоеусловиедифференциируемости>
\end{theorem}
%<*необходимоеусловиедифференциируемостиproof>
\begin{proof}
    $$f(x)=f(a)+(l_1\ldots l_m)(x-a)+\alpha(x)|x-a|$$
    $$\lim_{x\to a} \frac{f(x)-f(a)-L(x-a)}{|x-a|}=0$$
    Посчитаем предел по направлению $x=a+te_k, e_k=(0\ldots 0,1,0\ldots 0)$

    $$f(a+te_k)=f(a)+l_kt+\alpha_k(t)|t| \Rightarrow \exists \frac{\partial f}{\partial x_k}(a)=l_k$$
\end{proof}
%</необходимоеусловиедифференциируемостиproof>

\begin{consequence}
    $F:F\subset \R^m \to\R^l, a\in IntE, F$ --- дифф. в $a$

    Тогда все координатные функции $F_i$ дифференциируемы в $a$ и $F'(a)=\sum_{i=1}^l \sum_{j=1}^m \left(\frac{\partial F_i}{\partial x_j}(a)\right)$
\end{consequence}

\begin{theorem}
    Достаточное условие дифференциирования.

    %<*достаточноеусловиедифференциируемости>
    $f:E\subset \R^m \to\R \ \ \exists r>0 \ \ B(a,r)\subset E$ и в этом шаре $\exists f_1'\ldots f_m$ (конечные) и они непрерывны в точке $a$. Тогда $f$ дифф. в $a$
    %</достаточноеусловиедифференциируемости>
\end{theorem}
%<*достаточноеусловиедифференциируемостиproof>
\begin{proof}
    $\sphericalangle m=2$

    $$f(x_1,x_2)-f(a_1,a_2)=$$
    $$=f(x_1,x_2)-f(x_1,a_2)+f(x_1, a_2)-f(a_1,a_2)=$$
    $$=f_2'(x_1, \overline x_2)(x_2-a_2)+f_1'(\overline x_1, a_2)(x_1-a_1)=$$
    $$=f_2'(a_1,a_2)(x_2-a_2)+f_1'(a_1, a_2)(x_1-a_2)+(f_2'(x_1, \overline x_2)-f_2'(a_1, a_2)) \frac{x_2-a_1}{|x-a|}|x-a|+\text{ аналогично}$$
\end{proof}
%</достаточноеусловиедифференциируемостиproof>

\section*{Правила дифференциирования}

\subsection{Линейность}

$F, G :E\subset \R^m\to\R^l$, дифф. в $a\in IntE$. Тогда

$F+G, \forall \lambda\in\R \ \ \lambda F$ --- дифф. в $a$

$$(F+G)'(a)=F'(a)+G'(a) \quad (\lambda F)'(a)=\lambda F'(a)$$

\begin{proof}
    Сложить определения дифференциирования
    $$F(a+h)=F(a)+F'(a)h + \alpha(h)|h|$$
    $$F(a+h)=G(a)+G'(a)h + \beta(h)|h|$$
    $$(F+G)(a+h)=(F+G)(a)+(F'+G')(a)h + (\alpha+\beta)(h)|h|$$
\end{proof}

\end{document}