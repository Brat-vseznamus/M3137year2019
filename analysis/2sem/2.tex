\documentclass[12pt, a4paper]{article}

\usepackage{lastpage}
\usepackage{mathtools}
\usepackage{xltxtra}
\usepackage{libertine}
\usepackage{amsmath}
\usepackage{amsthm}
\usepackage{amsfonts}
\usepackage{amssymb}
\usepackage{enumitem}
\usepackage{xcolor}
\usepackage[left=1.5cm, right=1.5cm, top=2cm, bottom=2cm, bindingoffset=0cm, headheight=15pt]{geometry}
\usepackage{fancyhdr}
\usepackage[russian]{babel}
% \usepackage[utf8]{inputenc}
\usepackage{catchfilebetweentags}
\usepackage{accents}
\usepackage{calc}
\usepackage{etoolbox}
\usepackage{mathrsfs}
\usepackage{wrapfig}

\providetoggle{useproofs}
\settoggle{useproofs}{false}

\pagestyle{fancy}
\lfoot{M3137y2019}
\rhead{\thepage\ из \pageref{LastPage}}

\newcommand{\R}{\mathbb{R}}
\newcommand{\Q}{\mathbb{Q}}
\newcommand{\C}{\mathbb{C}}
\newcommand{\Z}{\mathbb{Z}}
\newcommand{\B}{\mathbb{B}}
\newcommand{\N}{\mathbb{N}}

\newcommand{\const}{\text{const}}

\newcommand{\teormin}{\textcolor{red}{!}\ }

\DeclareMathOperator*{\xor}{\oplus}
\DeclareMathOperator*{\equ}{\sim}
\DeclareMathOperator{\Ln}{\text{Ln}}
\DeclareMathOperator{\sign}{\text{sign}}
\DeclareMathOperator{\Sym}{\text{Sym}}
\DeclareMathOperator{\Asym}{\text{Asym}}
% \DeclareMathOperator{\sh}{\text{sh}}
% \DeclareMathOperator{\tg}{\text{tg}}
% \DeclareMathOperator{\arctg}{\text{arctg}}
% \DeclareMathOperator{\ch}{\text{ch}}

\DeclarePairedDelimiter{\ceil}{\lceil}{\rceil}
\DeclarePairedDelimiter{\abs}{\left\lvert}{\right\rvert}

\setmainfont{Linux Libertine}

\theoremstyle{plain}
\newtheorem{axiom}{Аксиома}
\newtheorem{lemma}{Лемма}

\theoremstyle{remark}
\newtheorem*{remark}{Примечание}
\newtheorem*{exercise}{Упражнение}
\newtheorem*{consequence}{Следствие}
\newtheorem*{example}{Пример}
\newtheorem*{observation}{Наблюдение}

\theoremstyle{definition}
\newtheorem{theorem}{Теорема}
\newtheorem*{definition}{Определение}
\newtheorem*{obozn}{Обозначение}

\setlength{\parindent}{0pt}

\newcommand{\dbltilde}[1]{\accentset{\approx}{#1}}
\newcommand{\intt}{\int\!}

% magical thing that fixes paragraphs
\makeatletter
\patchcmd{\CatchFBT@Fin@l}{\endlinechar\m@ne}{}
  {}{\typeout{Unsuccessful patch!}}
\makeatother

\newcommand{\get}[2]{
    \ExecuteMetaData[#1]{#2}
}

\newcommand{\getproof}[2]{
    \iftoggle{useproofs}{\ExecuteMetaData[#1]{#2proof}}{}
}

\newcommand{\getwithproof}[2]{
    \get{#1}{#2}
    \getproof{#1}{#2}
}

\newcommand{\import}[3]{
    \subsection{#1}
    \getwithproof{#2}{#3}
}

\newcommand{\given}[1]{
    Дано выше. (\ref{#1}, стр. \pageref{#1})
}

\renewcommand{\ker}{\text{Ker }}
\newcommand{\im}{\text{Im }}
\newcommand{\grad}{\text{grad}}

\lhead{Конспект по матанализу}
\cfoot{}
\rfoot{Лекция 2}

\begin{document}

Продолжение доказательства

\begin{proof}
    По лемме позиция выигрышна хотя бы для одного игрока. Рассмотрим случай, когда она выигрышна для белого игрока.

    В точке $A=(0, k) \rightsquigarrow (0, \frac{k}{n})$
    $$\left|f_1(\frac{A}{n}) - \frac{A_1}{n}\right|\geq\varepsilon$$
    $A_1=0; f_1(\frac{A}{n}) \geq 0 \Rightarrow$ при $v=A$ $$f_1(\frac{v}{n})-\frac{v_1}{n}\geq 0$$

    В точке $B=(n, l) \rightsquigarrow (1, \frac{l}{n})$
    $$\left|f_1\left(\frac{B}{n}\right) - \frac{B_1}{n}\right|\geq \varepsilon$$
    При $v=B$ $$f_1\left(\frac{v}{n}\right)-\frac{v_1}{n}\geq -\varepsilon$$
\end{proof}

\section{Определенный интеграл}

\subsection{Площадь}

\begin{definition}
    $ℇ$ --- множество всех ограниченных фигур в $\R^2$ (``фигура'' = подмножество $\R^2$)
\end{definition}

\begin{definition}
    Площадь это $\sigma : ℇ \to \R_+$, такое что:
    \begin{enumerate}
        \item $A\in ℇ \quad A=A_1\sqcup A_2 \quad \sigma A = \sigma A_1 + \sigma A_2$ (конечная аддитивность)
        \item $\sigma([a,b]\times [c,d])=(d-c)(b-a)$
    \end{enumerate}

    Мы пока что не знаем, существует ли площадь.
\end{definition}

\begin{remark}
    \begin{enumerate}
        \item Монотонность: $A\subset B \quad \sigma A\leq \sigma B$
        
        \item $\sigma$(вертик. отр.) $=0$
    \end{enumerate}
\end{remark}

\begin{definition}
    \textbf{Ослабленная площадь} $\sigma : ℇ \to \R_+$:
    \begin{enumerate}
        \item Монотонна
        \item Нормировка
        \item Ослабленная аддитивность: $E\in ℇ \quad E=E_1\cup E_2 \quad E_1\cap E_2$ --- вертикальный отрезок, $E_1$ и $E_2$ лежат каждый в своей полуплоскости относительно этого отрезка $\sigma E=\sigma E_1 + \sigma E_2$
    \end{enumerate}
\end{definition}

\begin{example}
    \begin{enumerate}
        \item $\sigma E = \inf \left(\sum \sigma P_i : E\subset \bigcup\limits_{\text{конечное}} P_k, P_k \text{ --- прямоугольники} \right)$
        \item $\sigma E = \inf \left(\sum \sigma P_i : E\subset \bigcup\limits_{\text{счётн.}} P_k, P_k \text{ --- прямоугольники}\right)$
    \end{enumerate}
\end{example}

Это разные площади. Покажем это на примере фигуры ``все точки в квадрате с рациональными координатами''. Первая площадь накрывает весь квадрат $\Rightarrow \sigma_1=1$. $\sigma_2=0$. Покажем это, накрыв $n$-тую точку квадратом размера $\frac{\varepsilon}{2^n}\times \frac{\varepsilon}{2^n}$. $\sum \frac{\varepsilon}{4^n}=\varepsilon\frac{\frac{1}{4}}{1-\frac{1}{4}}=\frac{\varepsilon}{3}\to0 \Rightarrow \inf = 0$

\begin{definition}
    $f:\langle a,b\rangle\to\R$

    $f_+:=\max(f, 0)$ --- \textbf{положительная срезка}

    $f_-:=\max(-f, 0)$ --- \textbf{отрицательная срезка}
\end{definition}

\begin{definition}
    $f:[a,b]\to\R; f\geq 0$

    $$\text{Под графиком \textit{(ПГ)}}(f, [a,b]) = \{(x, b) : x\in[a,b]; 0\leq y\leq f(x)\}$$
\end{definition}

\begin{definition}
    $f:[a, b] \to R$, непр.

    $$\int_a^b f = \int_a^b f(x)dx := \sigma\text{ПГ}(f_+, [a,b]) - \sigma\text{ПГ}(f_, [a,b])$$
\end{definition}

\begin{remark}
    \begin{enumerate}
        \item $f\geq 0 \Rightarrow \int^b_a f \geq 0$
        \item $f\equiv c \Rightarrow \int^b_a f = c(b-a)$
        \item $\int^b_a -f = -\int^b_a f$ --- верно, т.к. $(-f)_+=f_-$
        \item $\int^b_a 0 = 0$
    \end{enumerate}
\end{remark}

%<*свойстваинтегралов>
Свойства интегралов:
\begin{enumerate}
    \item Аддитивность по промежутку $c\in(a,b)$
    $$\int^b_a f = \int^c_a f + \int^b_c f$$
    \begin{proof}
        $$\sigma\text{ПГ}(f_+, [a,b])=\sigma\text{ПГ}(f_+, [a,c])+\sigma\text{ПГ}(f_+, [c,b])$$
    \end{proof}
    \item Монотонность: $f, g \in C[a,b] \quad f\leq g$. Тогда
    $$\int^b_a f \leq \int^b_a g$$
    \begin{proof}
        $$\text{ПГ}(f_+)\subset \text{ПГ}(g_+) \Rightarrow \sigma\text{ПГ}(f_+)\leq \sigma\text{ПГ}(g_+)$$
        $$\text{ПГ}(f_-)\supset \text{ПГ}(g_-) \Rightarrow \sigma\text{ПГ}(f_-)\geq \sigma\text{ПГ}(g_-)$$
        $$\sigma\text{ПГ}(f_+)-\sigma\text{ПГ}(f_-)\leq\sigma\text{ПГ}(g_+)-\sigma\text{ПГ}(g_-)$$
    \end{proof}
\end{enumerate}
%</свойстваинтегралов>

\begin{consequence}
    $$\min f\cdot(b - a) \leq \int^b_a f \leq \max f\cdot (b-a)$$
\end{consequence}

\begin{enumerate}[resume]
    \item $$\left|\int^b_a f\right|\leq \int^b_a |f|$$
    $$-|f|\leq f \leq |f|$$
    $$-\int^b_a|f|=\int^b_a-|f|\leq \int^b_a f\leq \int^b_a |f|$$
\end{enumerate}

%<*интегралспеременнымверхнимпределом>
\begin{definition}
    $f\in C[a,b] \quad \Phi : [a,b] \to\R \quad \Phi(x)=\int_a^x f$ --- \textbf{интеграл с переменным верхним пределом}

    $\Phi(a)=0$
\end{definition}
%</интегралспеременнымверхнимпределом>

%<*теоремабарроу>
\begin{theorem}
    $f\in C[a,b] \quad \Phi$ --- интеграл с переменным верхним пределом. Тогда $$\forall x\in[a,b] \quad \Phi'(x)=f(x)$$
\end{theorem}
%</теоремабарроу>
%<*теоремабарроуproof>
\begin{proof}
    Зафиксируем $x\in[a,b] \quad y>x, y\leq b$
    $$\frac{\Phi(y)-\Phi(x)}{y-x}=\frac{\int_x^y f}{y-x} \underset{\underset{\exists c\in[x,y]}{\text{т.о.ср.}}}{=} f(x)\xrightarrow[x\to x+0]{} f(x)$$
    $x>y$
    $$\frac{\Phi(y)-\Phi(x)}{y-x}=\frac{\int_a^y f - (\int_a^y f+ \int_y^x f)}{y-x}=\frac{1}{x-y}\int^x_y f = f(c)\xrightarrow[y\to x-0]{} f(x)$$
\end{proof}
%</теоремабарроуproof>

%<*осуществованиипервообразнойproof>
Теорема о существовании первообразной --- следствие теоремы Барроу.
%</осуществованиипервообразнойproof>

\begin{remark}
    $$\Psi(x)=\int_x^b f$$

    $$\Psi'(x)=-f(x)$$
\end{remark}

$$\left( \int_{x^2}^{10\sqrt x + 1} f(t)dt \right)'=f(10\sqrt x + 1) \frac{5}{\sqrt x} - f(x^2)2x$$

$$\left( \int^{\int_{x^2}^{e^x} \cos y^3 dy}_{\int_x^{\cos x} e^{-n^2} dn} \frac{\sin t}{\sqrt t}dt \right)'$$
Этот интеграл не написать в word. Tex нормас, как видите. Это единственное, зачем Кохась написал этот интеграл.

%<*ньютоналейбница>
\begin{theorem}
    $f\in C[a,b] \quad F \text{ --- первообр. } f$

    Тогда $\int_a^b f = F(b) - F(a)$
\end{theorem}
%</ньютоналейбница>
%<*ньютоналейбницаproof>
\begin{proof}
    $\Phi(x) = \int_0^x f$ --- первообр.

    $\exists C : F = \Phi + C$
    $$\int_a^b f = \Phi(b)=\Phi(b) - \Phi(a) = F(b) - F(a)$$
\end{proof}
%</ньютоналейбницаproof>

\begin{remark}
    Все ослабленные площади совпадают на ПГ$(f, [a,b]), \quad f\in C[a,b]$
\end{remark}

\subsection{Правило Лопиталя}

\begin{lemma}
    Об ускоренной сходимости

    \begin{enumerate}
        \item $f, g : D\subset X \to \R \quad a$ --- предельная точка $D$

        $\exists U(a) : $ при $x\in\dot U(a)\cap D \quad f(x)\not=0, g(x)\not=0$
    
        Пусть $\lim\limits_{x\to a} f(x) = 0 \quad \lim\limits_{x\to a} g(x) = 0$
        
        Тогда $$\forall x_k\to a \quad (x_k\not=a, x_k\in D) \quad \exists y_k\to a(y_k\not=a, y_k\in D)$$
        такое, что $$\lim\limits_{k\to+\infty}\frac{f(y_k)}{g(x_k)}=0 \quad \lim\limits_{k\to+\infty} \frac{g(y_k)}{g(x_k)}=0$$
    
        Таким образом, $g(y_k) \to 0$ быстрее, чем $g(x_k)\to 0$

        \item То же самое, но $\lim f(x)=+\infty, \lim g(x)=+\infty$
    \end{enumerate}
\end{lemma}
\begin{proof}
    \begin{enumerate}
        \item Очевидно.
        $$\forall k \quad \exists N \quad \forall n>N \quad |f(x_n)|<|g(x_k)|\frac{1}{k} \quad |g(x_n)|<|g(x_k)|\frac{1}{k}$$
        $\varepsilon:=|g(x_k)|$

        $$k=1 \quad y_1 := \text{какой-нибудь } x_n : \left|\frac{f(x_n)}{g(x_k)}\right|<1 \quad \left|\frac{g(x_n)}{g(x_k)}\right|<1$$
        $$k=2 \quad y_2 := \text{какой-нибудь } x_n : \left|\frac{f(x_n)}{g(x_k)}\right|<\frac{1}{2} \quad \left|\frac{g(x_n)}{g(x_k)}\right|<\frac{1}{2} $$
        $$\vdots$$

        \item \begin{enumerate}
            \item Частный случай: Пусть $g(x_n)$ возрастает. Берем $k$ : $m:=\min\{n : |f(x_n)|\geq \sqrt{g(x_k)} \text{ или }$
            
            $|g(x_n)|\geq\sqrt{g(x_k)}\}$
            
            $y_k:=x_{m-1}$
            $$\left|\frac{f(y_k)}{g(x_k)}\right|\leq \frac{\sqrt{g(x_k)}}{g(x_k)}=\frac{1}{\sqrt{g(x_k)}}\xrightarrow[k\to+\infty]{}0$$

            \textcolor{red}{Зачем нужно возрастание? Кохась не знает.}

            \item Общий случай: $\tilde g(x_k):=\inf\{g(x_n), n=k, k+1\ldots\} \quad \tilde g(x_k)\to+\infty$
            
            $\tilde g(x_k)\uparrow, \tilde g(x_k)\leq g(x_k)$. Как в пункте (a) построим $y_k$
            $$\frac{f(y_k)}{g(x_k)}\leq \frac{f(y_k)}{\tilde g(x_k)} \leq \frac{1}{\sqrt{\tilde g(x_k)}}\xrightarrow[k\to+\infty]{} 0$$
            $$\frac{g(y_k)}{g(x_k)}\leq \frac{g(x_k)}{\tilde g(x_k)}\leq \frac{1}{\sqrt{\tilde g(x_k)}}\to0$$
        \end{enumerate}
    \end{enumerate}
\end{proof}

\begin{theorem}
    $f,g : (a,b)\to\R \quad a\in\overline\R$

    $f,g$ --- дифф., $g'\not=0$ на $(a,b)$

    Пусть $\frac{f'(x)}{g'(x)}\xrightarrow[x\to a+0]{} A\in\overline\R$

    Пусть $\lim\limits_{x\to a}\frac{f(x)}{g(x)}$ --- неопределенность $\left\{\frac{0}{0}, \frac{+\infty}{+\infty}\right\}$

    Тогда $\exists \lim\limits_{x\to a} \frac{f(x)}{g(x)}=A$
\end{theorem}
\begin{proof}
    $g'\not=0 \Rightarrow g'$ --- сохр. знак $\Rightarrow g$ --- монотонна.

    Для $\frac{0}{0} \quad g(x)\not=0$ в $(a,b)$

    По Гейне $x_k\to a \ \ (x_k\not=a, x_k\in(a,b))$

    Выберем $y_k$ по лемме

    $$\frac{f(x_k)-f(y_k)}{g(x_k)-g(y_k)}=\frac{f'(\xi_k)}{g'(\xi_k)} \text{ --- т. Коши}$$

    $$\frac{f(x_k)}{g(x_k)}-\frac{f(y_k)}{g(x_k)}=\frac{f'(\xi_k)}{g'(\xi_k)}\left(1-\frac{g(y_k)}{g(x_k)}\right)$$

    $x_k\to a \quad y_k\to a \quad \xi_k \to a$
\end{proof}

\begin{example}
    $\frac{\pi}{2} - \arctg x \underset{x\to+\infty}{\sim} g(x)$
    $$\lim\limits_{x\to +\infty} \frac{\frac{\pi}{2} - \arctg x}{g(x)}=\lim\limits_{x\to +\infty}\frac{\frac{1}{x^2+1}}{g'(x)} = 1$$
\end{example}

$$\int_0^x e^{t^2}dt \underset{x\to+\infty}{\sim} g(x)$$
$$\lim\frac{\int_0^x e^{t^2}dt}{g(x)}=\lim\frac{e^{x^2}}{g'(x)}=1$$

\end{document}