\documentclass[12pt, a4paper]{article}

\usepackage{lastpage}
\usepackage{mathtools}
\usepackage{xltxtra}
\usepackage{libertine}
\usepackage{amsmath}
\usepackage{amsthm}
\usepackage{amsfonts}
\usepackage{amssymb}
\usepackage{enumitem}
\usepackage{xcolor}
\usepackage[left=1.5cm, right=1.5cm, top=2cm, bottom=2cm, bindingoffset=0cm, headheight=15pt]{geometry}
\usepackage{fancyhdr}
\usepackage[russian]{babel}
% \usepackage[utf8]{inputenc}
\usepackage{catchfilebetweentags}
\usepackage{accents}
\usepackage{calc}
\usepackage{etoolbox}
\usepackage{mathrsfs}
\usepackage{wrapfig}

\providetoggle{useproofs}
\settoggle{useproofs}{false}

\pagestyle{fancy}
\lfoot{M3137y2019}
\rhead{\thepage\ из \pageref{LastPage}}

\newcommand{\R}{\mathbb{R}}
\newcommand{\Q}{\mathbb{Q}}
\newcommand{\C}{\mathbb{C}}
\newcommand{\Z}{\mathbb{Z}}
\newcommand{\B}{\mathbb{B}}
\newcommand{\N}{\mathbb{N}}

\newcommand{\const}{\text{const}}

\newcommand{\teormin}{\textcolor{red}{!}\ }

\DeclareMathOperator*{\xor}{\oplus}
\DeclareMathOperator*{\equ}{\sim}
\DeclareMathOperator{\Ln}{\text{Ln}}
\DeclareMathOperator{\sign}{\text{sign}}
\DeclareMathOperator{\Sym}{\text{Sym}}
\DeclareMathOperator{\Asym}{\text{Asym}}
% \DeclareMathOperator{\sh}{\text{sh}}
% \DeclareMathOperator{\tg}{\text{tg}}
% \DeclareMathOperator{\arctg}{\text{arctg}}
% \DeclareMathOperator{\ch}{\text{ch}}

\DeclarePairedDelimiter{\ceil}{\lceil}{\rceil}
\DeclarePairedDelimiter{\abs}{\left\lvert}{\right\rvert}

\setmainfont{Linux Libertine}

\theoremstyle{plain}
\newtheorem{axiom}{Аксиома}
\newtheorem{lemma}{Лемма}

\theoremstyle{remark}
\newtheorem*{remark}{Примечание}
\newtheorem*{exercise}{Упражнение}
\newtheorem*{consequence}{Следствие}
\newtheorem*{example}{Пример}
\newtheorem*{observation}{Наблюдение}

\theoremstyle{definition}
\newtheorem{theorem}{Теорема}
\newtheorem*{definition}{Определение}
\newtheorem*{obozn}{Обозначение}

\setlength{\parindent}{0pt}

\newcommand{\dbltilde}[1]{\accentset{\approx}{#1}}
\newcommand{\intt}{\int\!}

% magical thing that fixes paragraphs
\makeatletter
\patchcmd{\CatchFBT@Fin@l}{\endlinechar\m@ne}{}
  {}{\typeout{Unsuccessful patch!}}
\makeatother

\newcommand{\get}[2]{
    \ExecuteMetaData[#1]{#2}
}

\newcommand{\getproof}[2]{
    \iftoggle{useproofs}{\ExecuteMetaData[#1]{#2proof}}{}
}

\newcommand{\getwithproof}[2]{
    \get{#1}{#2}
    \getproof{#1}{#2}
}

\newcommand{\import}[3]{
    \subsection{#1}
    \getwithproof{#2}{#3}
}

\newcommand{\given}[1]{
    Дано выше. (\ref{#1}, стр. \pageref{#1})
}

\renewcommand{\ker}{\text{Ker }}
\newcommand{\im}{\text{Im }}
\newcommand{\grad}{\text{grad}}

\usepackage{sectsty}

\allsectionsfont{\raggedright}
\subsectionfont{\fontsize{14}{15}\selectfont}

\lhead{Итоговый конспект}
\cfoot{}
\rfoot{}

\settoggle{useproofs}{true}

\begin{document}

\section{Определения}

\import{Локальный максимум, минимум, экстремум}{1.tex}{локальныймаксимум}

\import{\teormin Первообразная, неопределенный интеграл}{1.tex}{первообразная}
\get{1.tex}{неопределенныйинтеграл}

\import{Теорема о существовании первообразной}{1.tex}{осуществованиипервообразной}
\get{2.tex}{осуществованиипервообразнойproof}

\import{\teormin Таблица первообразных}{1.tex}{таблицапервообразных}

\import{Равномерная непрерывность}{1.tex}{равномернаянепрерывность}

\import{Площадь, аддитивность площади, ослабленная аддитивность}{2.tex}{фигуры}
\get{2.tex}{площадь}
\get{2.tex}{ослабленнаяплощадь}

\import{\teormin Определенный интеграл}{2.tex}{подграфиком}
\get{2.tex}{определенныйинтеграл}

\import{Положительная и отрицательная срезки}{2.tex}{срезки}

\import{Среднее значение функции на промежутке}{2.tex}{среднеезначениефункциинапромежутке}

\import{Кусочно-непрерывная функция}{3.tex}{кусочнонепрерывнаяфункция}

\import{Почти первообразная}{3.tex}{почтипервообразная}

\import{Функция промежутка, аддитивная функция промежутка}{3.tex}{segm}
\get{3.tex}{функцияпромежутка}
\get{3.tex}{аддитивнаяфункцияпромежутка}

\import{Плотность аддитивной функции промежутка}{3.tex}{плотностьаддитивнойфункциипромежутка}

\import{Выпуклая функция}{4.tex}{выпуклаяфункция}
\get{4.tex}{выпуклаяфункцияremark}
\get{4.tex}{строговыпуклаяфункция}

\import{Выпуклое множество в $\R^m$}{4.tex}{выпуклоемножество}

\import{Надграфик}{4.tex}{надграфик}

\import{Опорная прямая}{4.tex}{опорнаяпрямая}

\import{Гладкий путь, вектор скорости, носитель пути}{5.tex}{гладкийпуть}
\get{5.tex}{векторскорости}
\textcolor{red}{Кто такой носитель пути --- неизвестно, гугл предлагает про СПИД почитать.}

\import{Длина гладкого пути}{5.tex}{длинагладкогопути}

\import{Формулы для длины пути: в $\R^m$, в полярных координатах, длина графика}{5.tex}{}
\subsubsection{В $\R^m$}
\get{5.tex}{длинапутивrm}
\subsubsection{В полярных координатах}
\get{5.tex}{длинавполярных}
\subsubsection{Длина графика}
\get{5.tex}{длинаграфика}

\import{Вариация функции на промежутке}{5.tex}{вариацияфункции}

\import{Дробление отрезка, ранг дробления, оснащение}{6.tex}{дроблениеотрезка}
\get{6.tex}{рангдробления}
\get{6.tex}{оснащение}

\import{Риманова сумма}{6.tex}{римановасумма}

\import{Постоянная Эйлера}{6.tex}{постояннаяэйлера}

\import{Допустимая функция}{7.tex}{допустимаяфункция}

\import{\teormin Несобственный интеграл, сходимость, расходимость}{7.tex}{несобственныйинтеграл,cходимость,расходимость}

\import{Критерий Больцано--Коши сходимости несобственного интеграла}{7.tex}{критерийбольцанокошисходимости}

\import{Гамма функция Эйлера}{7.tex}{гаммафункция}

\import{\teormin Верхний и нижний пределы}{8.tex}{верхнийнижнийпредел}

\import{Частичный предел}{8.tex}{частичныйпредел}

\import{\teormin Абсолютно сходящийся интеграл, ряд}{8.tex}{абсолютносходящийсяинтеграл}
\get{11.tex}{абсолютносходящийсяряд}

% \import{Числовой ряд, сумма ряда, сходимость, расходимость}{}{}

% \import{$n$-й остаток ряда}{}{}

% \import{Критерий Больцано--Коши сходимости числового ряда}{}{}

% \import{Произведение рядов}{}{}

% \import{Произведение степенных рядов}{}{}

% \import{Скалярное произведение, евклидова норма и метрика в R^m}{}{}

% \import{Окрестность точки в R^m, открытое множество}{}{}

% \import{\teormin Сходимость последовательности в R^m, покоординатная сходимость}{}{}

% \import{\teormin Предельная точка, замкнутое множество, замыкание}{}{}

% \import{Компактность, секвенциальная компактность, принцип выбора Больцано-Вейерштрасса}{}{}

% \import{Координатная функция}{}{}

% \import{Двойной предел, повторный предел}{}{}

% \import{Предел по направлению, предел вдоль пути}{}{}

% \import{\teormin Предел отображения (определение по Коши и по Гейне)}{}{}

% \import{Линейный оператор}{}{}

% \import{\teormin Отображение бесконечно малое в точке.}{}{}

% \import{$o(h)$ при $h\to 0$}{}{}

% \import{\teormin Отображение, дифференцируемое в точке}{}{}

% \import{\teormin Производный оператор, матрица Якоби, дифференциал}{}{}

% \import{\teormin Частные производные}{}{}

% \import{\teormin Бесконечное произведение}{}{}

% \import{\teormin Классы $C^r(E)$}{}{}

% \import{Мультииндекс и обозначения с ним}{}{}

\section{Теоремы}

\import{Критерий монотонности функции. Следствия}{1.tex}{критериймонотонности}
\getwithproof{1.tex}{критериймонотонностиследствия1}
\getwithproof{1.tex}{критериймонотонностиследствия2}

\import{Теорема о необходимом и достаточном условиях экстремума}{1.tex}{необходимостьидостаточностьусловиялокальногоэкстремума}

\import{Теорема Кантора о равномерной непрерывности}{1.tex}{теоремакантора}

\import{Теорема Брауэра о неподвижной точке}{1.tex}{теоремаонеподвижнойточке}

\import{Теорема о свойствах неопределенного интеграла}{1.tex}{свойстванеопределенногоинтеграла}
\label{integralproperties}

\import{\teormin Интегрирование неравенств. Теорема о среднем}{2.tex}{интегрированиенеравенств}
Теорема о среднем: $f\in C[a,b], m\le f(x) \le M \Rightarrow \exists c : m\le c \le M \quad \int_a^b f = c(b-a)$
\begin{proof}
    По монотонности интеграла
    $$m(b-a)\le \int_a^b f \le M(b-a)$$
    $$m\le \frac{1}{b-a}\int_a^b f \le M$$
    $$c := \frac{1}{b-a}\int_a^b f$$
\end{proof}
\textcolor{red}{Взято с вики}

\import{Теорема Барроу}{2.tex}{теоремабарроу}

\import{Формула Ньютона-Лейбница, в том числе, для кусочно-непрерывных функций}{2.tex}{ньютоналейбница}
\textcolor{red}{Что с кусочно-непрерывными?}

\import{Лемма об ускоренной сходимости}{2.tex}{обускореннойсходимости}

\import{Правило Лопиталя}{2.tex}{правилолопиталя}

\import{Теорема Штольца}{3.tex}{штольца}

\import{Пример неаналитической функции}{3.tex}{неаналитическаяфункция}
\textcolor{red}{Отсутствует}

\import{Интегральное неравенство Чебышева. Неравенство для сумм}{3.tex}{неравенствочебышева}
\getwithproof{3.tex}{дискретноенеравенствочебышева}

\subsection{Свойства определенного интеграла: линейность, интегрирование по частям, замена переменных}
\given{integralproperties}

\import{Иррациональность числа пи}{3.tex}{hдляпи}
\getwithproof{3.tex}{иррациональностьпи}

\import{\teormin Теорема о вычислении аддитивной функции промежутка по плотности}{3.tex}{овычисленииаддитивнойфункциипромежуткапоплотности}

\import{Площадь криволинейного сектора: в полярных координатах и для параметрической кривой}{4.tex}{площадькриволинейногосектора1}
\get{4.tex}{площадькриволинейногосектора2}

\import{Изопериметрическое неравенство}{4.tex}{изопериметрическоенеравенство}

\import{Лемма о трех хордах}{4.tex}{леммаотреххордах}

\import{Теорема об односторонней дифференцируемости выпуклой функции}{4.tex}{ободностроннейдифференциируемостивыпуклойфункции}

\import{Следствие о точках разрыва производной выпуклой функции}{4.tex}{следствиеоточкахразрывапроизводнойвыпуклойфункции}

\import{Описание выпуклости с помощью касательных}{4.tex}{описаниевыпуклостиспомощьюкасательных}

\import{Дифференциальный критерий выпуклости}{4.tex}{дифференциальныйкритерийвыпуклости}

\import{Обобщенная теорема о плотности}{5.tex}{обобщеннаятеоремаоплотности}

\import{Вычисление длины гладкого пути}{5.tex}{длинапутивrm}

\import{Объем фигур вращения}{5.tex}{объемвращения}

\import{\teormin Интеграл как предел интегральных сумм}{6.tex}{обинтегралекакпределеинтегральныхсумм}

\import{Теорема об интегральных суммах для центральных прямоугольников}{6.tex}{обинтегральныхсуммахдляцентральныхпрямоугольников}

\import{Теорема о формуле трапеций, формула Эйлера-Маклорена}{6.tex}{}
\subsubsection{Теорема о формуле трапеций}
\getwithproof{6.tex}{теоремаоформулетрапеций}
\subsubsection{Формула Эйлера-Маклорена}
\getwithproof{6.tex}{формулаэйлерамаклорена}

\import{Асимптотика степенных сумм}{6.tex}{асимптотикастепенныхсумм}

\import{Асимптотика частичных сумм гармонического ряда}{6.tex}{асимптотикачастичныхсуммгармоническогоряда}

\import{Формула Валлиса}{6.tex}{формулаваллиса}

\import{Формула Стирлинга}{6.tex}{формуластирлинга}
\get{6.tex}{формуластирлингаproof2}

\import{Простейшие свойства несобственного интеграла}{7.tex}{}
\subsubsection*{Критерий Больцано-Коши}
\get{7.tex}{критерийбольцанокошисходимости}
\get{7.tex}{свойстванесобственногоинтеграла}

\import{\teormin Признаки сравнения сходимости несобственного интеграла}{7.tex}{признаксравнения}
\get{7.tex}{признаксравнения2}

\import{Интеграл Эйлера--Пуассона}{7.tex}{интегралэйлерапуассона}
\label{интегралэйлерапуассона}

\import{\teormin Гамма функция Эйлера. Простейшие свойства.}{7.tex}{свойствагаммафункции}
\given{интегралэйлерапуассона}

\import{Изучение сходимости интеграла $\int_{10}^\infty \frac{dx}{x^\alpha (\ln x)^\beta}$}{7.tex}{изучениеинтеграла}

% \import{Теорема об абсолютно сходящихся интегралах и рядах.}{}{}

% \import{Изучение интеграла $\int_1^{\infty} \frac{\sin x\,dx}{x^p}$ на сходимость и абсолютную сходимость}{}{}

% \import{Признак Абеля--Дирихле сходимости несобственного интеграла}{}{}

% \import{Интеграл Дирихле}{}{}

% \import{Неравенство Йенсена для сумм}{}{}

% \import{Неравенство Йенсена для интегралов. }{}{}

% \import{Неравенство Коши (для сумм и для интегралов)}{}{}

% \import{Неравенство Гельдера для сумм}{}{}

% \import{Неравенство Гельдера для интегралов}{}{}

% \import{Неравенство Минковского}{}{}

% \import{Свойства верхнего и нижнего пределов}{}{}

% \import{Техническое описание верхнего предела}{}{}

% \import{Теорема о существовании предела в терминах верхнего и нижнего пределов}{}{}

% \import{Теорема о характеризации верхнего предела как частичного}{}{}

% \import{Частичные пределы последовательности sin(n)}{}{}

% \import{Свойства рядов: линейность, свойства остатка, необх. условие сходимости, критерий Больцано--Коши}{}{}

% \import{Признак сравнения сходимости положительных рядов}{}{}

% \import{Признак Коши сходимости положительных рядов}{}{}

% \import{Признак Коши сходимости положительных рядов (pro)}{}{}

% \import{Признак Даламбера сходимости положительных рядов}{}{}

% \import{Признак Раабе сходимости положительных рядов}{}{}

% \import{Интегральный признак Коши сходимости числовых рядов}{}{}

% \import{Признак Лейбница}{}{}

% \import{Признаки Дирихле и Абеля сходимости числового ряда}{}{}

% \import{Теорема о перестановке слагаемых}{}{}

% \import{Теорема о перестановке слагаемых}{}{}

% \import{Теорема о произведении рядов}{}{}

% \import{Теорема об условиях сходимости бесконечного произведения}{}{}

% \import{Лемма об оценке приближения экспоненты ее замечательным пределом}{}{}

% \import{Формула Эйлера для гамма-функции}{}{}

% \import{Формула Вейерштрасса для Г-функции}{}{}

% \import{Вычисление произведений с рациональными сомножителями}{}{}

% \import{Единственность производной}{}{}

% \import{Лемма о дифференцируемости отображения и его координатных функций}{}{}

% \import{Необходимое условие дифференцируемости. }{}{}

% \import{Достаточное условие дифференцируемости}{}{}

% \import{Лемма об оценке нормы линейного оператора}{}{}

% \import{Дифференцирование композиции}{}{}

% \import{Дифференцирование "произведений"}{}{}

% \import{Теорема Лагранжа для векторнозначных функций}{}{}

% \import{Экстремальное свойство градиента}{}{}

% \import{Независимость частных производных от порядка дифференцирования}{}{}

% \import{Полиномиальная формула}{}{}

\end{document}