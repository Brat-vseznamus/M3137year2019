\documentclass[12pt, a4paper]{article}

\usepackage{lastpage}
\usepackage{mathtools}
\usepackage{xltxtra}
\usepackage{libertine}
\usepackage{amsmath}
\usepackage{amsthm}
\usepackage{amsfonts}
\usepackage{amssymb}
\usepackage{enumitem}
\usepackage{xcolor}
\usepackage[left=1.5cm, right=1.5cm, top=2cm, bottom=2cm, bindingoffset=0cm, headheight=15pt]{geometry}
\usepackage{fancyhdr}
\usepackage[russian]{babel}
% \usepackage[utf8]{inputenc}
\usepackage{catchfilebetweentags}
\usepackage{accents}
\usepackage{calc}
\usepackage{etoolbox}
\usepackage{mathrsfs}
\usepackage{wrapfig}

\providetoggle{useproofs}
\settoggle{useproofs}{false}

\pagestyle{fancy}
\lfoot{M3137y2019}
\rhead{\thepage\ из \pageref{LastPage}}

\newcommand{\R}{\mathbb{R}}
\newcommand{\Q}{\mathbb{Q}}
\newcommand{\C}{\mathbb{C}}
\newcommand{\Z}{\mathbb{Z}}
\newcommand{\B}{\mathbb{B}}
\newcommand{\N}{\mathbb{N}}

\newcommand{\const}{\text{const}}

\newcommand{\teormin}{\textcolor{red}{!}\ }

\DeclareMathOperator*{\xor}{\oplus}
\DeclareMathOperator*{\equ}{\sim}
\DeclareMathOperator{\Ln}{\text{Ln}}
\DeclareMathOperator{\sign}{\text{sign}}
\DeclareMathOperator{\Sym}{\text{Sym}}
\DeclareMathOperator{\Asym}{\text{Asym}}
% \DeclareMathOperator{\sh}{\text{sh}}
% \DeclareMathOperator{\tg}{\text{tg}}
% \DeclareMathOperator{\arctg}{\text{arctg}}
% \DeclareMathOperator{\ch}{\text{ch}}

\DeclarePairedDelimiter{\ceil}{\lceil}{\rceil}
\DeclarePairedDelimiter{\abs}{\left\lvert}{\right\rvert}

\setmainfont{Linux Libertine}

\theoremstyle{plain}
\newtheorem{axiom}{Аксиома}
\newtheorem{lemma}{Лемма}

\theoremstyle{remark}
\newtheorem*{remark}{Примечание}
\newtheorem*{exercise}{Упражнение}
\newtheorem*{consequence}{Следствие}
\newtheorem*{example}{Пример}
\newtheorem*{observation}{Наблюдение}

\theoremstyle{definition}
\newtheorem{theorem}{Теорема}
\newtheorem*{definition}{Определение}
\newtheorem*{obozn}{Обозначение}

\setlength{\parindent}{0pt}

\newcommand{\dbltilde}[1]{\accentset{\approx}{#1}}
\newcommand{\intt}{\int\!}

% magical thing that fixes paragraphs
\makeatletter
\patchcmd{\CatchFBT@Fin@l}{\endlinechar\m@ne}{}
  {}{\typeout{Unsuccessful patch!}}
\makeatother

\newcommand{\get}[2]{
    \ExecuteMetaData[#1]{#2}
}

\newcommand{\getproof}[2]{
    \iftoggle{useproofs}{\ExecuteMetaData[#1]{#2proof}}{}
}

\newcommand{\getwithproof}[2]{
    \get{#1}{#2}
    \getproof{#1}{#2}
}

\newcommand{\import}[3]{
    \subsection{#1}
    \getwithproof{#2}{#3}
}

\newcommand{\given}[1]{
    Дано выше. (\ref{#1}, стр. \pageref{#1})
}

\renewcommand{\ker}{\text{Ker }}
\newcommand{\im}{\text{Im }}
\newcommand{\grad}{\text{grad}}

\lhead{Конспект по матанализу}
\cfoot{}
\rfoot{Лекция 9}

\begin{document}

Для последовательности $x_n$:
$y_n=\sup(x_n, x_{n+1}\ldots), z_n=\inf(x_n, x_{n+1}\ldots)$. Тогда $z_n\leq x_n\leq y_n \ \ y_n\downarrow, z_n\uparrow$
$$\overline{\lim} x_n:=\lim y_n \quad \underline{\lim}x_n:=\lim z_n$$

\begin{theorem}
    Техническое описание верхнего предела.

    %<*техническоеописаниеверхнегопредела>
    \begin{enumerate}
        \item $\overline \lim x_n = +\infty \Leftrightarrow x_n$ --- неогр. сверху
        \item $\overline \lim x_n = -\infty \Leftrightarrow x_n \to -\infty$
        \item $\overline \lim x_n = l\in\R \Leftrightarrow $ a и b:
        \begin{enumerate}
            \item $\forall \varepsilon > 0 \ \ \exists N \ \ \forall n > N \ \ x_n < l + \varepsilon$
            \item $\forall \varepsilon > 0 $ для бесконечного множества номеров $n$ : $l-\varepsilon < x_n$
        \end{enumerate}
    \end{enumerate}
    %</техническоеописаниеверхнегопредела>
\end{theorem}
%<*техническоеописаниеверхнегопределаproof>
\begin{proof}
    \begin{enumerate}
        \item Очевидно, т.к. $y_n=\sup(x_n, x_{n+1}\ldots) = +\infty \Leftrightarrow x_n$ --- неогр. сверху
        \item \begin{itemize}
            \item [``$\Rightarrow$''] $x_n\leq y_n \to -\infty$
            \item [``$\Leftarrow$''] $\forall A \ \ \exists N \ \ \forall n>N \ \ y_n\leq A, x_n < A$
            \end{itemize}
        \item \begin{itemize}
            \item [``$\Rightarrow$''] \begin{enumerate}
                \item $y_n \to l \quad \forall \varepsilon>0 \ \ \exists N \ \ \forall n > N \quad x_n \leq y_n < l + \varepsilon$
                \item Берём $\varepsilon>0$, предположим противное : $\exists$ конечное мн-во $n : l-\varepsilon<x_n$
                
                $] n_0$ --- максимальный номер, такой что $l-\varepsilon<x_{n_0}$, тогда $y_{n_0} \leq l - \varepsilon$, но $y_n\downarrow \Rightarrow \lim y_n\leq l-\varepsilon$
            \end{enumerate}
            \item [``$\Leftarrow$''] $\forall \varepsilon>0 \ \ \exists N \ \ \forall n > N \quad x_n<l+\varepsilon \Rightarrow y_n\leq l+\varepsilon$, но в $x_n, x_{n+1}\ldots $ $\exists x_i : l - \varepsilon < x_i \Rightarrow y_n=\sup(x_n, x_{n+1}\ldots) > l - \varepsilon$. Итого $l + \varepsilon \geq y_n > l - \varepsilon \Rightarrow l=\lim y_n=\overline \lim x_n$
        \end{itemize}
    \end{enumerate}
\end{proof}
%</техническоеописаниеверхнегопределаproof>

\begin{theorem}
    %<*осуществованиипределавтерминахверхнегоинижнегопределов>
    $$\exists \lim x_n \in \overline\R \Leftrightarrow \overline \lim x_n = \underline \lim x_n$$
    %</осуществованиипределавтерминахверхнегоинижнегопределов>
\end{theorem}
%<*осуществованиипределавтерминахверхнегоинижнегопределовproof>
\begin{proof}
    \begin{itemize}
        \item [``$\Rightarrow$''] \begin{enumerate}
            \item $\lim x_n = +\infty \Rightarrow \overline \lim x_n = \lim y_n \geq \lim x_n=+\infty$
            \item $\lim x_n = -\infty$ аналогично
            \item $\lim x_n = l\in\R$ очевидно из технического описания предела, пункт 3.
            \end{enumerate}
        \item [``$\Leftarrow$''] $\underline \lim x_n \leftarrow z_n\leq x_n \leq y_n\to \overline \lim x_n$, по теореме о городовых $\exists \lim x_n=\overline \lim x_n$
    \end{itemize}
\end{proof}
%</осуществованиипределавтерминахверхнегоинижнегопределовproof>

\begin{definition}
    $n_k : n_1 < n_2 < n_3 < \ldots \quad \lim\limits_{k\to +\infty} x_{n_k}$ --- \textbf{частичный предел}
\end{definition}

\begin{theorem}
    О характеризации верхнего предела как частичного.

    %<*охарактеризации>
    \begin{enumerate}
        \item $\forall l$ --- частичный пр. $x_n$ $\underline \lim x_n \leq l \leq \overline \lim x_n$
        \item $\exists (n_k) : x_{n_k} \to \overline \lim x_n \ \ \exists m_k : x_{m_k}\to \underline \lim x_n$
    \end{enumerate}
    %</охарактеризации>
\end{theorem}
%<*охарактеризацииproof>
\begin{proof}
    \begin{enumerate}
        \item $x_{n_k}\to l \quad \underline \lim x_n \leftarrow z_{n_k} \leq x_{n_k} \leq y_{n_k}\to \overline \lim x_n \Rightarrow \underline \lim x_n \leq l \leq \overline \lim x_n$
        \item \begin{enumerate}
            \item $\overline \lim x_n = +\infty \Leftrightarrow x_n$ --- неогр сверху $\Rightarrow$ можно выбрать $x_{n_1}<x_{n_2}<\ldots x_n\to+\infty$
            \item $\overline \lim x_n = -\infty$ тривиально.
            \item $\overline \lim x_n = l\in\R$ $\exists x_{n_k} : l - \frac{1}{k} < x_{n_k} < l + \frac{1}{k}$
        \end{enumerate}
    \end{enumerate}
\end{proof}
%</охарактеризацииproof>

\begin{example}
    %<*частичныепределыsin>
    \begin{enumerate}
        \item $\overline \lim \sin n = 1, \underline \lim \sin n = -1$
        \item $\forall l \in [-1, 1]$ --- частичный передел последовательности $\sin n$
    \end{enumerate}
    %</частичныепределыsin>
\end{example}
%<*частичныепределыsinproof>
\begin{proof}
    \begin{enumerate}
        \item Тривиально
        \item $n_k := \arcsin l + 2\pi k$
    \end{enumerate}
    
    Кроме того, можно составить $n_k\in \N$.

    % Будем ``ходить'' по тригонометрической окружности с шагом 1. Когда мы пересечем 0, этот шаг поделится на две части, меньшая из которых $<\frac{1}{2}$. Теперь можем
\end{proof}
%</частичныепределыsinproof>

\section{Простейшие свойства рядов}

\begin{definition}
    %<*числовойряд>
    $a_1+a_2+\ldots $, $\sum\limits_{i=1}^{+\infty} a_i$ --- \textbf{числовой ряд} ($a_i\in\R$)
    %</числовойряд>
\end{definition}
\begin{definition}
    %<*частичнаясумма>
    $\forall N\in\N \quad S_n:=\sum\limits_{i=1}^n a_i$ --- \textbf{частичная сумма}
    %</частичнаясумма>
\end{definition}

\begin{definition}
    %<*сходимостьрасходимостьряда>
    Если $\exists \lim_{N\to+\infty} S_n=S\in\R$, ряд \textbf{сходится}, иначе ряд \textbf{расходится}.
    %</сходимостьрасходимостьряда>
\end{definition}

\begin{example}
    $x\in\R$
    $$\sum_{n=0}^{+\infty} \frac{x^n}{n!}$$
    $$e^x = \sum_{k=0}^N \frac{x^k}{k!} + \frac{e^{\Theta}}{(N+1)!}x^{N+1}$$
    $$\sum_{n=0}^{+\infty} \frac{2^n}{n!}=e^2$$
\end{example}

\begin{example}
    $$\sum_{n=1}^{+\infty}\frac{1}{n^k}, k\in N, p=-k$$
    $$\sum_{n=1}^{+\infty}n^p = \frac{N^{p+1}}{p+1} + \frac{N^p}{2} + \frac{1}{2} + \frac{1}{2} \int_1^N (x^p)''\{x\}(1-\{x\})=\frac{N^{p+1}}{p+1} + \frac{N^p}{2} + \frac{1}{2} + \mathcal O(\max (1, N^{p-1}))$$
    \begin{itemize}
        \item $p > -1$ расходится
        \item $p = -1$ расходится
        \item $p < -1$ сходится
    \end{itemize}
\end{example}

\begin{definition}
    %<*нйостатокряда>
    $\sum\limits_{k=N}^{+\infty} a_k$ --- $N$-й \textbf{остаток ряда}
    %</нйостатокряда>
\end{definition}

Свойства:
%<*линейностьсвойстваостатка>
\begin{enumerate}
    \item $\sum a_n, \sum b_n$ сходятся, $c_n:=a_n+b_n$. Тогда $\sum c_n$ сходится
    \item $\sum a_n$ --- сходится, $\lambda\in\R$. Тогда $\sum \lambda a_n$ сходится и $\sum \lambda a_n = \lambda \sum a_n$
    \item \begin{enumerate}
        \item $\sum a_n$ --- сходится $\Rightarrow$ любой остаток сходится
        \item остаток сходится $\Rightarrow \sum a_n$ сходится
        \item $r_N = \sum\limits_{n\geq N} a_n$, $\sum a_n$ сходится $\Leftrightarrow r_N\xrightarrow[N\to+\infty]{} 0$
    \end{enumerate}
\end{enumerate}
%</линейностьсвойстваостатка>
%<*линейностьсвойстваостаткаproof>
\begin{proof}
    \begin{itemize}
        \item [(a)] $?m$-й остаток, $N\geq m : \sum\limits_{n=1}^N a_n = \sum\limits_{n=1}^{m-1} a_n + \sum\limits_{n=m}^N a_n$
        \item [(b)] Аналогично.
        \item [(c)] \begin{itemize}
            \item [``$\Leftarrow$''] Тривиально.
            \item [``$\Rightarrow$''] $\sum\limits_{n=1}^{+\infty} a_n = \sum\limits_{n=1}^{m-1}a_n + r_m \xRightarrow{m\to+\infty} \sum\limits_{n=1}^{+\infty} a_n = \sum\limits_{n=1}^{+\infty}a_n + r_{+\infty} \Rightarrow r_N\to0 $
        \end{itemize}
    \end{itemize}
\end{proof}
%</линейностьсвойстваостаткаproof>

\begin{lemma}
    %<*необходимоеусловиесходимости>
    Необходимое условие сходимости:

    $\sum a_n$ сходится $\Rightarrow a_n\to 0$
    %</необходимоеусловиесходимости>
\end{lemma}
%<*необходимоеусловиесходимостиproof>
\begin{proof}
    Тривиально. $a_n = S_n-S_{n-1}\to0$
\end{proof}
%</необходимоеусловиесходимостиproof>

Обратное неверно, например $\sum \frac{1}{n^p}$ расходится, $p\in (0, 1]$

\begin{theorem}
    %<*критерийбольцанокошидляряда>
    
    Критерий сходимости ряда Больцано-Коши:
    $$\sum a_n \text{ сходится } \Leftrightarrow \forall \varepsilon > 0 \ \ \exists N \ \ \forall k > N \ \ \forall m\in\N \quad |a_{k+1}+a_{k+2}+\ldots+a_{k+m}|<\varepsilon$$
    %</критерийбольцанокошидляряда>
\end{theorem}
%<*критерийбольцанокошидлярядаproof>
\begin{proof}
    Тривиально.
\end{proof}
%</критерийбольцанокошидлярядаproof>

Докажем расходимость $\sum\frac{1}{n}$ по критерию Больцано-Коши.
$$m:=k \quad \frac{1}{k+1}+\frac{1}{k+2}+\ldots+\frac{1}{2k}>k\frac{1}{2k}=\frac{1}{2}$$
$$\exists \varepsilon=\frac{1}{2} \ \ \forall N \ \ \exists k>N \ \ \exists m:=k \quad |a_{k+1}+a_{k+2}+\ldots+a_{k+k}|\geq\varepsilon$$

\begin{theorem}
    Признак сравнения.

    %<*признаксравненияряда>
    $a_k, b_k\geq 0$
    \begin{enumerate}
        \item $\forall k \ \ a_k\leq b_k$, или $\exists c>0 \ \ \forall k \ \ a_k\le cb_k$. Тогда $\sum b_k$ cх. $\Rightarrow$ $\sum a_k$ cх., $\sum a_k$ расх. $\Rightarrow$ $\sum b_k$ расх.
        \item $\exists \lim\frac{a_k}{b_k} = l\in [0, +\infty]$. Тогда при
        \begin{enumerate}
            \item [$0<l<+\infty:$] $\sum a_k$ сх. $\Leftrightarrow$ $\sum b_k$ сх.
            \item [$l=0:$] $\sum b_k$ сх. $\Rightarrow$ $\sum a_k$ сх., $\sum a_k$ расх. $\Rightarrow \sum b_k$ расх.
            \item [$l=+\infty:$] $\sum a_k$ сх. $\Rightarrow$ $\sum b_k$ сх., $\sum b_k$ расх. $\Rightarrow$ $\sum a_k$ расх.
        \end{enumerate}
    \end{enumerate}
    %</признаксравненияряда>
\end{theorem}
%<*признаксравнениярядаproof>
\begin{proof}
    \begin{lemma}
        $a_n\geq 0 \quad \sum a_n$ сходится $\Leftrightarrow S_n$ ограничено сверху.
    \end{lemma}
    \begin{proof}
        $\exists$ кон. $\lim S_n \Leftrightarrow S_n$ ограничено сверху.
    \end{proof}
    \begin{enumerate}
        \item $S_n^{(a)}\le S_n^{(b)}; \ \ S_n^{(b)}$ огр. $\Rightarrow S_n^{(a)}$ огр., по леммме $a_n$ сходится. Аналогично расходимость.
        \item \begin{enumerate}
            \item $0<l<+\infty:$ Для $\varepsilon=\frac{l}{2}\ \ \exists N \ \ \forall n>N \ \ \frac{1}{2} l b_n<a_n<\frac{3}{2}l b_n$, дальше по 1 пункту.
            \item $l=0:$ $\forall \varepsilon > 0 \ \ \exists N \ \ \forall n>N \ \ \frac{a_n}{b_n}<\varepsilon \Rightarrow a_n < \varepsilon b_n \Rightarrow$ по 1 пункту.
            \item $l=+\infty:$ $\forall \varepsilon > 0 \ \ \exists N \ \ \forall n>N \ \ \frac{a_n}{b_n}>\varepsilon \Rightarrow a_n>b_n \varepsilon \Rightarrow$ по 1 пункту.
        \end{enumerate}
    \end{enumerate}
\end{proof}
%</признаксравнениярядаproof>

\begin{example}
    \begin{enumerate}
        \item $$\sum_{n=1}^{+\infty} \frac{n^3 + 14n + 1}{n^5 + n^4 + \sqrt n}=\sum_{n=1}^{+\infty} a_n$$
        $a_n\sim \frac{n^3}{n^5}=\frac{1}{n^2}, 2>1 \Rightarrow$ ряд сходится.
    \end{enumerate}
\end{example}

\end{document}