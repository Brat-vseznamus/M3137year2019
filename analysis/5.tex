\documentclass[12pt, a4paper]{article}

\usepackage{lastpage}
\usepackage{mathtools}
\usepackage{xltxtra}
\usepackage{libertine}
\usepackage{amsmath}
\usepackage{amsthm}
\usepackage{amsfonts}
\usepackage{amssymb}
\usepackage{enumitem}
\usepackage{xcolor}
\usepackage[left=1.5cm, right=1.5cm, top=2cm, bottom=2cm, bindingoffset=0cm, headheight=15pt]{geometry}
\usepackage{fancyhdr}
\usepackage[russian]{babel}
% \usepackage[utf8]{inputenc}
\usepackage{catchfilebetweentags}
\usepackage{accents}
\usepackage{calc}
\usepackage{etoolbox}
\usepackage{mathrsfs}
\usepackage{wrapfig}

\providetoggle{useproofs}
\settoggle{useproofs}{false}

\pagestyle{fancy}
\lfoot{M3137y2019}
\rhead{\thepage\ из \pageref{LastPage}}

\newcommand{\R}{\mathbb{R}}
\newcommand{\Q}{\mathbb{Q}}
\newcommand{\C}{\mathbb{C}}
\newcommand{\Z}{\mathbb{Z}}
\newcommand{\B}{\mathbb{B}}
\newcommand{\N}{\mathbb{N}}

\newcommand{\const}{\text{const}}

\newcommand{\teormin}{\textcolor{red}{!}\ }

\DeclareMathOperator*{\xor}{\oplus}
\DeclareMathOperator*{\equ}{\sim}
\DeclareMathOperator{\Ln}{\text{Ln}}
\DeclareMathOperator{\sign}{\text{sign}}
\DeclareMathOperator{\Sym}{\text{Sym}}
\DeclareMathOperator{\Asym}{\text{Asym}}
% \DeclareMathOperator{\sh}{\text{sh}}
% \DeclareMathOperator{\tg}{\text{tg}}
% \DeclareMathOperator{\arctg}{\text{arctg}}
% \DeclareMathOperator{\ch}{\text{ch}}

\DeclarePairedDelimiter{\ceil}{\lceil}{\rceil}
\DeclarePairedDelimiter{\abs}{\left\lvert}{\right\rvert}

\setmainfont{Linux Libertine}

\theoremstyle{plain}
\newtheorem{axiom}{Аксиома}
\newtheorem{lemma}{Лемма}

\theoremstyle{remark}
\newtheorem*{remark}{Примечание}
\newtheorem*{exercise}{Упражнение}
\newtheorem*{consequence}{Следствие}
\newtheorem*{example}{Пример}
\newtheorem*{observation}{Наблюдение}

\theoremstyle{definition}
\newtheorem{theorem}{Теорема}
\newtheorem*{definition}{Определение}
\newtheorem*{obozn}{Обозначение}

\setlength{\parindent}{0pt}

\newcommand{\dbltilde}[1]{\accentset{\approx}{#1}}
\newcommand{\intt}{\int\!}

% magical thing that fixes paragraphs
\makeatletter
\patchcmd{\CatchFBT@Fin@l}{\endlinechar\m@ne}{}
  {}{\typeout{Unsuccessful patch!}}
\makeatother

\newcommand{\get}[2]{
    \ExecuteMetaData[#1]{#2}
}

\newcommand{\getproof}[2]{
    \iftoggle{useproofs}{\ExecuteMetaData[#1]{#2proof}}{}
}

\newcommand{\getwithproof}[2]{
    \get{#1}{#2}
    \getproof{#1}{#2}
}

\newcommand{\import}[3]{
    \subsection{#1}
    \getwithproof{#2}{#3}
}

\newcommand{\given}[1]{
    Дано выше. (\ref{#1}, стр. \pageref{#1})
}

\renewcommand{\ker}{\text{Ker }}
\newcommand{\im}{\text{Im }}
\newcommand{\grad}{\text{grad}}


\lhead{Конспект по математическому анализу}
\rfoot{October 14, 2019}

\begin{document}
 
\noindent
%<*замыкание>
$\overline D=D\cup $ (множество предельных точек $D$) --- замыкание.
%</замыкание>

\begin{remark}
$a\in\overline D$, тогда $\exists (x_n)$ из $D, x_n\to a$
\end{remark}

\begin{remark}
%<*замыканиечерезпересечения>
$\overline D = \bigcap\limits_{\substack{D\subset F \\ F \text{--- замкн.}}} F$ --- мин. (по вкл.) замкн. множество, содержащее $D$.
%</замыканиечерезпересечения>
\end{remark}

\begin{remark}
$D$ --- замкнуто $\Leftrightarrow D=\overline D$
\end{remark}

\begin{definition}
%<*граничнаяточка>
$a$ --- \textbf{граничная точка} $D$, если $\forall U(a) \quad U(a)$ содержит точки как из $D$, так и из $D^c$
%</граничнаяточка>
\end{definition}

\begin{definition}
%<*граница>
\textbf{Граница множества} --- множество его граничных точек. Обозначается $\partial D$
%</граница>
\end{definition}

Упражнение:
\begin{enumerate}
    \item $\partial D=\overline D \setminus IntD$
    \item $\partial D$ --- замкнута
    \item $\forall$ множество предельных точек --- замкнуто.
\end{enumerate}

\begin{definition}
$T$ --- множество, $U$ --- набор неких подмножеств $T$.

При этом: \begin{enumerate}
    \item \O$\in U, T\in U$
    \item $G_1,G_2\ldots G_n\in U \Rightarrow \bigcap\limits_{i=1}^n G_i\in U$
    \item $(G_\alpha)_{\alpha\in A}, \forall \alpha G_\alpha\in U \quad \bigcup\limits_{\alpha\in A}\in U$
\end{enumerate}
Тогда $T$ называется \textbf{топологическим пространством}, $U$ --- ``набор'' открытых множеств в $T$ \textit{(мн-ва $G^c$, где $G\in U$ --- замкн.)}
\end{definition}

$a\in T$, $U(a)$ --- любое открытое множество, содержащее $a$ и $\not =$\O.

\begin{axiom}
    Об отделимости: $\forall x,y\in T \exists U(x), U(y) : U(x)\cap U(y)=$\O  
\end{axiom}

\begin{definition}
%<*последовательностьстремящаясякбесконечности>
В $\mathbb{R}$:
\begin{enumerate}
    \item $x_n\to +\infty \quad \forall E>0 \ \ \exists N \ \ \forall n>N \ \ x_n>E$
    \item $x_n\to -\infty \quad \forall E \ \ \exists N \ \ \forall n>N \ \ x_n<E$
    \item $x_n\to \infty \Leftrightarrow |x_n|\to +\infty$
\end{enumerate}
%</последовательностьстремящаясякбесконечности>
\end{definition}

\begin{remark}
Требование $>0$ не обязательно.
\end{remark}

\begin{remark}
\begin{enumerate}
    \item $x_n\to \infty \Rightarrow x_n$ не огр. \textit{(по модулю)}

    $x_n\to +\infty \Rightarrow x_n$ не огр. сверху

    $x_n\to -\infty \Rightarrow x_n$ не огр. снизу
    \item $x_n\to +\infty$. Тогда $x_n\not\to -\infty$
\end{enumerate}
\end{remark}

Откр. множества:
\begin{enumerate}
    \item Ограниченные открытые множества --- те, что открыт. в $\mathbb{R}$
    \item $U_E (+\infty)=(E,+\infty] \subset\mathbb{\overline R}$
    
    $U_E(-\infty) = [-\infty,E) \subset\mathbb{\overline R}$
    \item Произвольное открытое множество --- либо огр. откр., либо огр.$\cup U_E(+\infty)$, огр.$\cup U_E(-\infty)$, огр.$\cup U_E(+\infty)\cup U_E(-\infty)$
\end{enumerate}

\begin{proof}
Рассмотрим $y=\tan x$

Положим $\tan (\frac{\pi}{2})=+\infty$, $\tan (\frac{\pi}{2})=-\infty$

$\tan$ --- монотонная биекция $[-\frac{\pi}{2}, \frac{\pi}{2}]$ на $\mathbb{R}$

Она обеспечивает биекцию между совокупностью открытых множеств $[-\frac{\pi}{2}, \frac{\pi}{2}]$ и $\ldots$ в $\mathbb{\overline R}$
\end{proof}

В $\mathbb{\overline R}$ рассмотрим функцию $\rho(x,y)=|\arctan x - \arctan y|$ --- метрика.

Покажем, что $x_n\to +\infty$ в смысле исх. опр. $\Leftrightarrow x_n\to+\infty$ в пространстве $(\mathbb{\overline R}, \rho)$

\begin{proof}
$x_n\to+\infty \Leftrightarrow \forall U(+\infty) \ \ \exists N \ \ \forall n>N \ \ x_n\in U(+\infty)$

$x_n\to+\infty$ в пространстве $(\mathbb{\overline R}, \rho) \Leftrightarrow $ высказыванию выше.
\end{proof}

\begin{remark}
$a\in\mathbb{R}$, $(x_n)$ --- вещ. посл. Тогда $x_n\to a$ в смысле обычного опр. $\Leftrightarrow x_n\to a$ в пространстве $(\mathbb{\overline R}, \rho)$
\end{remark}

$\begin{cases}
    x_n\to a, a\in\mathbb{\overline R} \\
    x_n\to b, b\in\mathbb{\overline R}
\end{cases} \Rightarrow a=b$

в $\mathbb{R}^m \quad x_n\to\infty \quad \forall E \ \ \exists N \ \ \forall n>N \ \
||x_n||>E$

$U_E(+\infty)=\{x\in\mathbb{R}^m : ||x||>E\}$

\section{Ревизия}

$(x_n), (y_n) \quad x_n\leq y_n \quad x_n\to x, y_n\to y, \ x,y \in\mathbb{\overline R}$. Тогда $x\leq y$.

\begin{itemize}
\item $y=+\infty$ или $x=-\infty$ --- тривиально.
\item $x=+\infty, y=a\in\mathbb{R}$ --- невозможно
\item остальное --- как в основной теореме.
\end{itemize}

\begin{definition}
Последовательность $(y_n)$ называется \textbf{бесконечно большой}, если $y_n\to +\infty$.
\end{definition}

\begin{remark}
$x_n$ --- бесконечно малая ($\forall n \ \ x_n\not = 0$) $\Leftrightarrow \frac{1}{x_n}$ --- бесконечно большая.
\end{remark}

\begin{proof}
$|x_n|<\varepsilon \Leftrightarrow |\frac{1}{x_n}|>\frac{1}{\varepsilon}$
\end{proof}

\begin{theorem}
Об арифметических свойствах пределов в $\mathbb{\overline R}$.

%<*арифметическиесвойствапределоввrсчертой>
$(x_n),(y_n)$ --- вещ., $x_n\to a, y_n\to b, \quad a,b\in\mathbb{\overline R}$

Тогда:
\begin{enumerate}
    \item $x_n\pm y_n\to a\pm b$
    \item $x_n y_n\to a b$
    \item $\frac{x_n}{y_n}\to\frac{a}{b}$
    \textit{, если $\forall n \ \ y_n\not=0; b\not=0$}
\end{enumerate}
При условии, что выражения в правых частях имеют смысл.
%</арифметическиесвойствапределоввrсчертой>
\end{theorem}

$$\sphericalangle x_n\to +\infty, y_n\to a\in \mathbb{R}$$

$$\forall E \ \ \exists N \ \ \forall n > N \ \ x_n+y_n>E$$

$$\text{Для } E-(a-1) \ \ \exists N_1 \ \ \forall n>N_1 \ \ x_n > E-(a-1)$$

$$\text{Для } E=1 \ \ \exists N_2 \ \ \forall n>N_2 \ \ x_n > a-1$$

Также для $x_n\to+\infty, y_n$ --- огр.снизу $\Rightarrow x_n+y_n\to+\infty$.

$\begin{cases}
    x_n\to+\infty \\
    y_n>\varepsilon, (\varepsilon>0) \text{ при } n>N_0
\end{cases} \Rightarrow x_ny_n\to +\infty$

$y_n$ отделено от нуля при больших $n$.

\begin{remark}
Верны аналогичные теоремы, где вместо $\mathbb{\overline{R}}$ --- $\mathbb{\overline{C}}=\mathbb{C}\cup\{\infty\}$
\end{remark}

Неопределенности:
%<*неопределенности>
\begin{itemize}
    \item $+\infty-\infty$
    \item $0\cdot (\pm \infty)$
    \item $\frac{\pm\infty}{\pm\infty}$
    \item $\frac{0}{0}$
\end{itemize}
%</неопределенности>

\section{Точные границы числовых множеств}

\begin{theorem}
Теорема Кантора о стягивающихся отрезках.

%<*теоремакантора>
Дана последовательность отрезков $[a_1, b_1]\supset[a_2, b_2]\supset\ldots$

Длины отрезков $\to 0$, т.е. $(b_n-a_n)\to_{n\to+\infty}0$

Тогда $\exists!c\in\mathbb{R} \quad \bigcap\limits_{k=1}^{+\infty}[a_k, b_k]=\{c\}$ и при этом $a_n\to_{n\to+\infty} c, b_n\to_{n\to+\infty} c$
%</теоремакантора>
\end{theorem}

\begin{remark}
Вместо ``$b_n-a_n\to 0$'' $\quad \forall \varepsilon>0 \ \ \exists n : b_n-a_n<\varepsilon$
\end{remark}

%<*теоремакантораproof>
\begin{proof}
Берем из аксиомы Кантора $c\in\bigcap\limits_{k=1}^{+\infty}[a_k, b_k]$

$\begin{cases}
    0 \leq b_n-c\leq b_n-a_n \\
    0 \leq c-a_n\leq b_n-a_n
\end{cases} \Rightarrow \begin{cases}
    b_n-c\to 0 \\
    c-a_n\to 0
\end{cases} \Rightarrow \begin{cases}
    b_n\to c \\
    a_n\to c
\end{cases}$

По теореме об единственности предела $c$ однозначно определено.
\end{proof}
%</теоремакантораproof>

\begin{definition}
%<*верхняяграница>
$E\subset \mathbb{R}$. $E$ --- \textbf{огр. сверху}, если $\exists M\in\mathbb{R} \ \ \forall x\in E \ \ x\leq M$. Кроме того, всякие такие $M$ называются {\bf верхними границами} $E$.
%</верхняяграница>
\end{definition}

%<*нижняяграница>
Аналогично ограничение снизу.
%</нижняяграница>

\begin{definition}
%<*супремуминфимум>
$E\subset \mathbb{R}, E\not=$\O.

Для $E$ --- огр. сверху \textbf{супремум} ($\sup E$)--- наименьшая из верхних границ $E$.

Для $E$ --- огр. снизу \textbf{инфимум} ($\inf E$) --- наибольшая из нижних границ $E$.
%</супремуминфимум>
\end{definition}

\begin{remark}
%<*техническоеописаниесупремума>
Техническое описание супремума:
$b=\sup E \Leftrightarrow \begin{cases}
    \forall x\in E \ \ x\leq b \\
    \forall \varepsilon > 0 \ \ \exists x\in E \ \ b-\varepsilon<x
\end{cases}$
%</техническоеописаниесупремума>
\end{remark}

Аналогично для $\inf$

\begin{definition}
$M=\max E : M\in E \ \ \forall x\in E \ \ x\leq M$
\end{definition}

\begin{theorem}
О существовании супремума.

%<*осуществованиисупремума>
$E\subset \mathbb{R}, E\not=$ \O $, E$ --- огр. сверху.

Тогда $\exists\sup E \in\mathbb{R}$
%</осуществованиисупремума>
\end{theorem}

%<*осуществованиисупремумаproof>
\begin{proof}
Строим систему вложенных отрезков $[a_k, b_k]$ со свойствами:

\begin{enumerate}
    \item $b_k$ --- верхняя граница $E$
    \item $[a_k, b_k]$ содержит точки $E$.
\end{enumerate}
$a_1$ --- берём любую точку $E$, $b_1$ --- любая верхняя граница.

Границы следующего отрезка найдём бинпоиском \textit{(математики это называют половинное деление)}.

Если $\frac{a_1+b_1}{2}$ --- верхняя граница $E$, $[a_2,b_2]:=[a_1, \frac{a_1+b_1}{2}]$.

Иначе на $[\frac{a_1+b_1}{2}, b_1]$ есть элементы $E$, $[a_2,b_2]:=[\frac{a_1+b_1}{2}, b_1]$

Длина $[a_k,b_k]=b_k-a_k=\frac{b_1-a_1}{2^{k-1}}\to0$

$\exists!c\in \bigcap [a_k, b_k]$

Проверим: $c=\sup E$ по техническому описанию супремума:

\begin{enumerate}
\item $\forall x\in E \ \ \forall n \ \ x\leq c$
\item $\forall \varepsilon > 0 \ \ c-\varepsilon$ --- не верхн. гран., т.е. $\exists n: c-\varepsilon<a_n$
\end{enumerate}

Доказательство 1: $\forall n\ \ x\leq b_n, x\to x, b_n\to c \Rightarrow x\leq c$ \textit{(предельный переход)}

Доказательство 2: $\forall \varepsilon > 0$ возьмём $n:$ длина отрезка $=b_n-a_n<\varepsilon$.
$$c-a_n<b_n-a_n<\varepsilon$$
$$c-\varepsilon<a_n$$
\end{proof}
%</осуществованиисупремумаproof>

\end{document}