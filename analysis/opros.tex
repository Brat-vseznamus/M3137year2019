\documentclass[12pt, a4paper]{article}

\usepackage{lastpage}
\usepackage{mathtools}
\usepackage{xltxtra}
\usepackage{libertine}
\usepackage{amsmath}
\usepackage{amsthm}
\usepackage{amsfonts}
\usepackage{amssymb}
\usepackage{enumitem}
\usepackage{xcolor}
\usepackage[left=1.5cm, right=1.5cm, top=2cm, bottom=2cm, bindingoffset=0cm, headheight=15pt]{geometry}
\usepackage{fancyhdr}
\usepackage[russian]{babel}
% \usepackage[utf8]{inputenc}
\usepackage{catchfilebetweentags}
\usepackage{accents}
\usepackage{calc}
\usepackage{etoolbox}
\usepackage{mathrsfs}
\usepackage{wrapfig}

\providetoggle{useproofs}
\settoggle{useproofs}{false}

\pagestyle{fancy}
\lfoot{M3137y2019}
\rhead{\thepage\ из \pageref{LastPage}}

\newcommand{\R}{\mathbb{R}}
\newcommand{\Q}{\mathbb{Q}}
\newcommand{\C}{\mathbb{C}}
\newcommand{\Z}{\mathbb{Z}}
\newcommand{\B}{\mathbb{B}}
\newcommand{\N}{\mathbb{N}}

\newcommand{\const}{\text{const}}

\newcommand{\teormin}{\textcolor{red}{!}\ }

\DeclareMathOperator*{\xor}{\oplus}
\DeclareMathOperator*{\equ}{\sim}
\DeclareMathOperator{\Ln}{\text{Ln}}
\DeclareMathOperator{\sign}{\text{sign}}
\DeclareMathOperator{\Sym}{\text{Sym}}
\DeclareMathOperator{\Asym}{\text{Asym}}
% \DeclareMathOperator{\sh}{\text{sh}}
% \DeclareMathOperator{\tg}{\text{tg}}
% \DeclareMathOperator{\arctg}{\text{arctg}}
% \DeclareMathOperator{\ch}{\text{ch}}

\DeclarePairedDelimiter{\ceil}{\lceil}{\rceil}
\DeclarePairedDelimiter{\abs}{\left\lvert}{\right\rvert}

\setmainfont{Linux Libertine}

\theoremstyle{plain}
\newtheorem{axiom}{Аксиома}
\newtheorem{lemma}{Лемма}

\theoremstyle{remark}
\newtheorem*{remark}{Примечание}
\newtheorem*{exercise}{Упражнение}
\newtheorem*{consequence}{Следствие}
\newtheorem*{example}{Пример}
\newtheorem*{observation}{Наблюдение}

\theoremstyle{definition}
\newtheorem{theorem}{Теорема}
\newtheorem*{definition}{Определение}
\newtheorem*{obozn}{Обозначение}

\setlength{\parindent}{0pt}

\newcommand{\dbltilde}[1]{\accentset{\approx}{#1}}
\newcommand{\intt}{\int\!}

% magical thing that fixes paragraphs
\makeatletter
\patchcmd{\CatchFBT@Fin@l}{\endlinechar\m@ne}{}
  {}{\typeout{Unsuccessful patch!}}
\makeatother

\newcommand{\get}[2]{
    \ExecuteMetaData[#1]{#2}
}

\newcommand{\getproof}[2]{
    \iftoggle{useproofs}{\ExecuteMetaData[#1]{#2proof}}{}
}

\newcommand{\getwithproof}[2]{
    \get{#1}{#2}
    \getproof{#1}{#2}
}

\newcommand{\import}[3]{
    \subsection{#1}
    \getwithproof{#2}{#3}
}

\newcommand{\given}[1]{
    Дано выше. (\ref{#1}, стр. \pageref{#1})
}

\renewcommand{\ker}{\text{Ker }}
\newcommand{\im}{\text{Im }}
\newcommand{\grad}{\text{grad}}

\usepackage{bm}
\usepackage{xcolor}
% \usepackage[raggedright]{titlesec}
\usepackage{sectsty}

\allsectionsfont{\raggedright}
\subsectionfont{\fontsize{14}{15}\selectfont}

\lhead{Конспект к опросу}
\cfoot{}
\rfoot{}

\setlength\parindent{0pt}

\begin{document}
\section{Определения}
\subsection{Упорядоченная пара}
Для некоторого множества $X$ и $I$ - множество ``индексов'', тогда $(x_\alpha)_{\alpha\in I}$ - \textbf{семейство элементов} $X$. ($\forall \alpha\in I \ \ x_\alpha \in X$)

\textbf{Упорядоченная пара} --- семейство из двух элементов, построенная при $I=\{1, 2\}$. Обозначается $(a, b)$.
\subsection{Декартово произведение}
\textbf{Декартово произведение} двух множеств --- множество всех упорядоченных пар элементов этих множеств. $A\times B=\{(a,b): a\in A, b\in B\}$

Кроме того, декартово произведение можно обобщить для произвольного числа множеств. $A_1\times A_2\times\ldots\times A_n = \{(a_1,a_2\ldots a_n) : a_1\in A_1, a_2\in A_2\ldots a_n\in A_n\}$

\subsection{Аксиомы вещественных чисел}
\subsubsection{Аксиомы поля}
В множестве $\mathbb{R}$ определены две операции, называемые сложением и умножением, действующие из $\mathbb{R}\times\mathbb{R}$ в $\mathbb{R}$ ($+,\cdot:\mathbb{R}\times\mathbb{R} \rightarrow \mathbb{R}$), удовлетворяющие следующим свойствам:

Аксоимы сложения \textit{(здесь и далее $\forall a\in \R, b\in \R, c\in \R$)}:
\begin{enumerate}
    \itemsep0em
    \item $a+b=b+a$ --- коммутативность
    \item $(a+b)+c=a+(b+c)$ --- ассоциативность
    \item $\exists \bm{0}: \bm0+a=a$
    \item $\exists a': a+a'=\bm 0$
\end{enumerate}

Аксиомы умножения:
\begin{enumerate}
    \item $ab=ba$ --- коммутативность
    \item $(ab)c=a(bc)$ --- ассоциативность
    \item $\exists \bm 1\not =\bm 0: \forall a\in \mathbb{R}: a\cdot\bm 1=a$
    \item $\forall a\not =\bm 0: \exists \tilde{a}: a\cdot \tilde{a}=\bm 1$
\end{enumerate}

Аксоима комбинации сложения и умножения:
\begin{enumerate}
    \item $(a+b)c=ac+bc$ --- дистрибутивность
\end{enumerate}

\textbf{Поле} --- множество, в котором определены операции $+,\cdot$, удовлетворяющие группе аксиом I. Например, $\mathbb{R}, \mathbb{Q}, \mathbb{F}_3$

\subsubsection{Аксиомы порядка}
\begin{enumerate}
    \item $\forall x,y \in \mathbb{R}: x\leq y \text{ или } y\leq x$
    \item $x\leq y; y\leq x \Rightarrow x=y$
    \item $x\leq y; y\leq z \Rightarrow x\leq z$ --- транзитивность
    \item $x\leq y \Rightarrow \forall z\in \mathbb{R}: x+z\leq y+z$
    \item $0\leq x; 0\leq y \Rightarrow 0\leq xy$
\end{enumerate}

\textbf{Упорядоченное поле} --- множество, для которого выполняются аксиомы групп I и II.

$\mathbb{F}_3, \mathbb{C}$ - не упорядоченные поля

$\mathbb{R}, \mathbb{Q}, \mathcal{R}$ - упорядоченные поля

\subsection{Аксиома Кантора, аксиома Архимеда}
\subsubsection{Аксиома Архимеда}
$$\forall x,y>0: \exists n\in \R: nx>y$$

Следствие: существуют сколько угодно большие натуральные числа: $$\forall y\in \mathbb{R}: \exists n\in\mathbb{N}: n>y$$

\textbf{Архимедовы поля} --- упорядоченные поля, в которых выполняется Аксиома Архимеда.

$\mathcal{R}$ - не архимедово поле

$\mathbb{R}, \mathbb{Q}$ - архимедовы поля

\subsubsection{Аксиома Кантора}

Для последовательности вложенных отрезков $\{[a_n,b_n]\}_{n=1}^\infty$ ($\forall n\in\N a_n\leq a_{n+1}\leq b_{n+1}\leq b_n$)

$$\bigcap\limits_{n=1}^\infty[a_n, b_n] \not = \text{\O}$$

$\Q$ не удволетворяет этой аксиоме, в отличие от $\R$.

\subsection{Пополненное множество вещественных чисел, операции и порядок в нем}
\textcolor{red}{Это дополнение?}

\subsection{Максимальный элемент множества}
$M\in A$ называется \textbf{максимальным элементом} множества $A$, если $\forall a\in A \ \ a\leq M$

\subsection{Последовательность}

$x:\N \rightarrow Y$ --- \textbf{последовательность}

\subsection{Образ и прообраз множества при отображении}

Для $A\subset X, f:X\to Y$ \textbf{образ} --- множество $\{f(x), x\in A\} \subset Y$ --- обозначается $f(A)$


Для $B\subset Y$ \textbf{прообраз} --- $\{ x\in X : f(x)\in B \}$ --- обозначается $f^{-1}(B)$

\subsection{Инъекция, сюръекция, биекция}

\textbf{Сюръекция} --- такое отображение $f: X\to Y$, что $f(X)=Y$, т.е. $\forall y\in Y \ \ f(x)=y$ имеет решение относительно $x$.


\textbf{Инъекция} --- такое отображение $f: X\to Y$, что $\forall x_1, x_2 \in X, x_1\not=x_2 \ \ f(x_1)\not=f(x_2)$, т.е. $\forall y\in Y \ \ f(x)=y$ имеет не более одного решения относительно $x$.


\textbf{Биекция} --- отображение, являющееся одновременно сюръекцией и инъекцией, т.е. $\forall y\in Y \ \ f(x)=y$ имеет ровно одно решение относительно $x$.

\subsection{Векторнозначаная функция, ее координатные функции}

Если $F:X\to \R^m; x\mapsto F(x) = (F_1(x), ..., F_m(x))$, то $F$ --- \textbf{векторнозначная функция} \textit{(значения функции - вектора)}


$F_1(x)..F_m(x)$ - координатные функции отображения $F$

\subsection{График отображения}

$$\Gamma_f = \{(x,y)\in X\times Y: y = f(x) \}$$

\subsection{Композиция отображений}

$f: X\to Y, g: Y\to Z$, тогда \textbf{композиция} $f$ и $g$ \textit{(обозначается $g\circ f$)} --- такое отображение, что $g\circ f: X\to Z, x\mapsto g(f(x))$.


Также возможно определение, которое допускает $g: Y_1\to Z, Y_1\supset Y$

\subsection{Сужение и продолжение отображений}

Для $g: X\to Y\ \ $ $f$ --- \textbf{сужение} $g$ на множество $A$, если $f: A\to Y, A\subset X$.


$g$ называется \textbf{продолжением} $f$.

\subsection{Предел последовательности \textit{(эпсилон-дельта определение)}}

Если для $(x_n), a\in\mathbb{R}$ выполняется
$\forall \varepsilon > 0 \ \ \exists N \ \ \forall n>N \ \ |x_n-a|<\varepsilon$,
то $a$ --- \textbf{предел последовательности} $(x_n)$, обозначается $x_n\to a$ или $\lim\limits_{n\to\infty}x_n=a$

\subsection{Окрестность точки, проколотая окрестность}

\textbf{Окрестность точки} $a = \{x\in\R: |x-a|<\varepsilon\}$, обозначается $U_\varepsilon(a)$


\textbf{Проколотая окрестность} точки $a = U_\varepsilon(a)\setminus \{a\}$, обозначается $\dot U_\varepsilon(a)$

\subsection{Предел последовательности (определение на языке окрестностей)}

$$\forall U(a) \ \ \exists N \ \ \forall n>N \ \ x_n\in U(a)$$

\subsection{Метрика, метрическое пространство, подпространство}

На множестве $X$ отображение $\rho: X\times X\to \R$ называется \textbf{метрикой}, если выполняются свойства 1-3:

\begin{enumerate}
    \itemsep0em
    \item $\forall x,y \ \ \rho(x,y)\geq 0$; $\rho(x,y)=0 \Leftrightarrow x=y$
    \item $\forall x,y \ \ \rho(x,y)=\rho(y,x)$
    \item Неравенство треугольника: $\forall x,y,z\in X \ \ \rho(x,y)\leq \rho(x,z)+\rho(z,y)$
\end{enumerate}

\textbf{Метрическое пространство} --- упорядоченная пара $(X, \rho)$, где $X$ --- множество, $\rho$ --- метрика на $X$.

\textbf{Подпространством} метрического пространства $(X,\rho)$ называется $(A, \rho|_{A\times A})$, если $A\subset X$

\subsection{Шар, замкнутый шар, окрестность точки в метрическом пространстве}

\textbf{Шар \textit{(открытый шар)}} $B(a,r)=\{x\in X : \rho(a,x)<r\}$

\textbf{Замкнутый шар} $B(a,r)=\{x\in X : \rho(a,x)\leq r\}$

\textbf{Окрестность точки} $a$ в метрическом пространстве: $B(a, \varepsilon) \Leftrightarrow U(a)$.

\subsection{Линейное пространство}

Если $K$ --- поле \textit{($K=\mathbb{R}$ или $\mathbb{C}$)}, $X$ --- множество, то $X$ называется \textbf{линейным пространством} над полем $K$ (и тогда $K$ называется полем скаляр), если определены следующие две операции:
\begin{enumerate}
\item $+:X\times X \to X$ --- сложение векторов
\item $\cdot:K\times X\to X$ --- умножение векторов на скаляры
\end{enumerate}

Для этих операций выполняются соответствующие аксиомы \textit{(здесь $A,B,C\in X; a,b\in K$)}:

\subsubsection{Аксиомы сложения векторов}
\begin{enumerate}\itemsep0em
    \item $A+B=B+A$
    \item $A+(B+C)=(A+B)+C$
    \item $\exists \bm 0 \in X : A+\bm 0 = A$
    \item $\exists -A\in X : A+(-A)=0$ --- обратный элемент
\end{enumerate}

\subsubsection{Аксиомы умножения векторов на скаляры}
\begin{enumerate}\itemsep0em
    \item $(A+B)\cdot a = A\cdot a + B\cdot a$
    \item $A\cdot(a+b) = A\cdot a + A\cdot b$
    \item $(ab)\cdot A = a(b\cdot A)$
    \item $\exists \bm 1 \in K : \bm 1 \cdot A = A$
\end{enumerate}

\subsection{Норма, нормированное пространство}

\textbf{Норма} - отображение $X\to\mathbb{R}, x\mapsto ||x||$, если $X$ - линейное пространство (над $\mathbb{R}$ или $\mathbb{C}$) и выполняется следующее:
\begin{enumerate}
\item $\forall x \ \ ||x||\geq 0, ||x||=0\Leftrightarrow x=0$
\item $\forall x\in X \ \ \forall \lambda\in\mathbb R(\mathbb{C}) \ \ ||\lambda x||=|\lambda|\cdot||x||$
\item Неравенство треугольника: $\forall x,y\in X \ \ ||x+y||\leq||x||+||y||$
\end{enumerate}

\textbf{Нормированное пространство} --- упорядоченная пара $(X, ||\cdot||)$, где $|| ||$ - норма

\subsection{Ограниченное множество в метрическом пространстве}

$A\subset X$ --- {\bf ограничено}, если $\exists x_0\in X \ \ \exists R>0 \ \ A\subset B(x_0, R)$, т.е. если $A$ содержится в некотором шаре в $X$.

\subsection{Внутренняя точка множества, открытое множество, внутренность}

$a$ --- {\bf внутренняя точка} множества $D$, если $\exists U(a) : U(a)\subset D$, т.е. $\exists r>0 : B(a,r)\subset D$

$D$ --- {\bf открытое множество}, если $\forall a\in D : a$ --- внутренняя точка $D$

{\bf Внутренностью} множества $D$ называется $Int(D)=\{x\in D : x \text{ --- внутр. точка }D\}$

\subsection{Предельная точка множества}

$a$ --- {\bf предельная точка} множества $D$, если $\forall \dot U(a) \ \ \dot U(a)\cap D\not = \text{\O}$

\subsection{Замкнутое множество, замыкание, граница}

$D$ --- {\bf замкнутое множество}, если оно содержит все свои предельные точки.

{\bf Замыканием} множества $D$ называется $\overline D=D\cup$(множество предельных точек $D$)

{\bf Граница} множества --- множество его граничных точек. Обозначается $\partial D=\overline D\ Int D$

\subsection{Изолированная точка, граничная точка}

$a$ --- {\bf изолированная} точка $D$, если $a\in D$ и $a$ --- не предельная.

$a$ --- {\bf граничная} точка $D$, если $\forall U(a) \ \ U(a)$ содержит точки как из $D$, так и из $D^c$

\subsection{Описание внутренности множества}

\begin{enumerate}
    \item $Int D$ --- открыто
    \item $Int D = \bigcup\limits_{\substack{D\supset G \\ G\text{ --- открыт}}}$ --- максимальное открытое множество, содержащееся в $D$
    \item $D$ --- открыто в $X \Leftrightarrow D=Int D$
\end{enumerate}

\subsection{Описание замыкания множества в терминах пересечений}

$\overline D = \bigcap\limits_{\substack{F\supset D \\ F \text{ --- замкн.}}} F$ -- минимальное \textit{(по включению)} замкнутое множество, содержащее $D$. Если $D$ замкнуто, $\overline D=D$.

\subsection{Верхняя, нижняя границы; супремум, инфимум}

$E\subset \mathbb{R}$. $E$ --- огр. сверху, если $\exists M\in\mathbb{R} \ \ \forall x\in E \ \ x\leq M$. Кроме того, всякие такие $M$ называются {\bf верхними границами} $E$.

Аналогично ограничение снизу.

Для $E$ --- огр. сверху {\bf супремум} ($\sup E$)--- наименьшая из верхних границ $E$.

Для $E$ --- огр. снизу {\bf инфинум} ($\sup E$) --- наибольшая из нижних границ $E$.

\subsection{Техническое описание супремума}

$$b=\sup E \Leftrightarrow \begin{cases}
    \forall x\in E \ \ x\leq b \\
    \forall \varepsilon > 0 \ \ \exists x\in E \ \ b-\varepsilon<x
\end{cases}$$

\subsection{Последовательность, стремящаяся к бесконечности}

$$x_n\to +\infty \quad \forall E>0 \ \ \exists N \ \ \forall n>N \ \ x_n>E$$
$$x_n\to -\infty \quad \forall E \ \ \exists N \ \ \forall n>N \ \ x_n<E$$
$$x_n\to \infty \Leftrightarrow |x_n|\to +\infty$$

\subsection{Компактное множество}

$K\subset X$ --- {\bf компактное}, если для любого открытого покрытия $\exists$ конечное подпокрытие $\Leftrightarrow \exists \alpha_1\ldots \alpha_n \quad K\subset\bigcup\limits_{i=1}^n G_{\alpha_i}$

\subsection{Секвенциальная компактность}

\textbf{Секвенциально компактным} называется множество $A\subset X : 
    \forall \text{ посл. } (x_n) \text{ точек } A \\
    \exists \text{ подпосл. } x_{n_k} \text{, которая сходится к точке из } A
$

\subsection{Определения предела отображения (3 шт)}

\subsubsection{По Коши}
    $$\forall \varepsilon>0 \ \ \exists \delta>0 \ \ \forall x\in D \ \ 0<\rho^X(a,x)<\delta \quad \rho^Y(f(x), A) < \varepsilon$$
\subsubsection{На языке окрестностей}
    $$\forall U(A) \ \ \exists V(a) \ \ \forall x\in \dot V(a) \ \ f(x)\in U(A)$$
\subsubsection{По Гейне}
    $\forall (x_n)$ --- посл. в $X$:
    \begin{enumerate}
        \item $x_n\to a$
        \item $x_n\in D$
        \item $x_n\not = a$
    \end{enumerate}
    $f(x_n)\to A$

\subsection{Определения пределов в $\overline\R$}

\section{Теоремы}

\subsection{Законы де Моргана}

Пусть $(X_\alpha)_{\alpha \in A}$ - семейство множеств, $Y$ - множество. Тогда:

\begin{enumerate}
    \item $Y\setminus(\bigcup\limits_{\alpha \in A}X_\alpha) = \bigcap\limits_{\alpha \in A}(Y\setminus X_\alpha)$
    \item $Y\setminus(\bigcap\limits_{\alpha \in A}X_\alpha) = \bigcup\limits_{\alpha \in A}(Y\setminus X_\alpha)$
\end{enumerate}

\subsection{Неравенство Коши-Буняковского, евклидова норма в $\R^m$}

\subsubsection{Неравенство Коши-Буняковского}

$$(\sum a_ib_i)^2 \leq (\sum a_i^2)(\sum b_k^2)$$

\subsubsection{Евклидова норма в $\R^m$}

$$||x||=\sqrt{\sum\limits_{i}^m x_i^2}$$

\subsection{Аксиома Архимеда. Плотность множества $\Q$ в $\R$}

\subsubsection{Аксиома Архимеда}

$$\forall x,y>0: \exists n\in \R: nx>y$$

\subsubsection{Плотность множества $\Q$ в $\R$}

$$\Q \text{ плотно в } \R\xLeftrightarrow{def} \forall a,b \in \R, a<b \ \ (a,b)\cap\mathbb{Q} \not =\text{\O}$$

В любом интервале в $\R$ содержится число $\in \Q$.

\subsection{Неравенство Бернулли}

$$(1+x)^n \geq 1+nx \quad x>-1, n\in\N$$

\subsection{Единственность предела и ограниченность сходящейся последовательности}

\subsubsection{Единственность предела}

$(X,\rho)$ --- метрическое пр-во, $a,b\in X$, $(x_n)$ --- послед. в $X$,
$x_n\xrightarrow[n\to +\infty]{} a$, $x_n\to b$,
тогда $a=b$

\subsubsection{Ограниченность сходящейся последовательности}

Если $(x,\rho)$ --- метрическое пр-во, $(x_n)$ --- послед. в $X$, $x_n$ сходится, тогда $x_n$ --- ограничена.

\subsection{Теорема о предельном переходе в неравенствах для последовательностей и для функций}

Если $(x_n),(y_n)$ --- вещественные последовательности $x_n\to a, y_n\to b$, $\exists N \ \ \forall n > N \ x_n\leq y_n$, \textbf{тогда} $a\leq b$.

Если $f, g:X\to \R$, $a$ --- предельная точка $X$, и $\forall x\in X f(x)\leq g(x)$. Тогда $\lim\limits_{x\to a}f(x)\leq \lim\limits_{x\to a}g(x)$ 

\subsection{Теорема о двух городовых}

Если $(x_n),(y_n),(z_n)$ - вещ. посл., $\forall n \ \ x_n\leq y_n\leq z_n, \lim x_n=\lim z_n =a$, \textbf{тогда} $\exists \lim y_n=a$

\subsection{Бесконечно малая последовательность}

$(x_n)$ --- вещ. посл. называется \textbf{бесконечно малой}, если $x_n\to 0$

\subsection{Теорема об арифметических свойствах предела последовательности в нормированном пространстве и в $\R$}

Если $(X, ||\cdot||)$ --- норм. пр-во, $(x_n),(y_n)$ --- посл. в $X$, $\lambda_n$ ---
посл. скаляров, и $x_n\to x_0, y_n\to y_0, \lambda_n\to \lambda_0$, \textbf{тогда}:
\begin{enumerate}
\item $x_n\pm y_n\to x_0\pm y_0$
\item $\lambda_nx_n\to \lambda x_0$
\item $||x_n||\to||x_0||$
\end{enumerate}

Для $(x_n),(y_n)$ 
--- вещ.посл., $\forall n \ \ y_n\not =0, y_0\not = 0$:
\begin{enumerate}[resume]
\item $\frac{x_n}{y_n}\to\frac{x_0}{y_0}$
\end{enumerate}

\end{document}