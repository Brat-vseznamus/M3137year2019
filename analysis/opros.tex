\documentclass[12pt, a4paper]{article}

\usepackage{lastpage}
\usepackage{mathtools}
\usepackage{xltxtra}
\usepackage{libertine}
\usepackage{amsmath}
\usepackage{amsthm}
\usepackage{amsfonts}
\usepackage{amssymb}
\usepackage{enumitem}
\usepackage{xcolor}
\usepackage[left=1.5cm, right=1.5cm, top=2cm, bottom=2cm, bindingoffset=0cm, headheight=15pt]{geometry}
\usepackage{fancyhdr}
\usepackage[russian]{babel}
% \usepackage[utf8]{inputenc}
\usepackage{catchfilebetweentags}
\usepackage{accents}
\usepackage{calc}
\usepackage{etoolbox}
\usepackage{mathrsfs}
\usepackage{wrapfig}

\providetoggle{useproofs}
\settoggle{useproofs}{false}

\pagestyle{fancy}
\lfoot{M3137y2019}
\rhead{\thepage\ из \pageref{LastPage}}

\newcommand{\R}{\mathbb{R}}
\newcommand{\Q}{\mathbb{Q}}
\newcommand{\C}{\mathbb{C}}
\newcommand{\Z}{\mathbb{Z}}
\newcommand{\B}{\mathbb{B}}
\newcommand{\N}{\mathbb{N}}

\newcommand{\const}{\text{const}}

\newcommand{\teormin}{\textcolor{red}{!}\ }

\DeclareMathOperator*{\xor}{\oplus}
\DeclareMathOperator*{\equ}{\sim}
\DeclareMathOperator{\Ln}{\text{Ln}}
\DeclareMathOperator{\sign}{\text{sign}}
\DeclareMathOperator{\Sym}{\text{Sym}}
\DeclareMathOperator{\Asym}{\text{Asym}}
% \DeclareMathOperator{\sh}{\text{sh}}
% \DeclareMathOperator{\tg}{\text{tg}}
% \DeclareMathOperator{\arctg}{\text{arctg}}
% \DeclareMathOperator{\ch}{\text{ch}}

\DeclarePairedDelimiter{\ceil}{\lceil}{\rceil}
\DeclarePairedDelimiter{\abs}{\left\lvert}{\right\rvert}

\setmainfont{Linux Libertine}

\theoremstyle{plain}
\newtheorem{axiom}{Аксиома}
\newtheorem{lemma}{Лемма}

\theoremstyle{remark}
\newtheorem*{remark}{Примечание}
\newtheorem*{exercise}{Упражнение}
\newtheorem*{consequence}{Следствие}
\newtheorem*{example}{Пример}
\newtheorem*{observation}{Наблюдение}

\theoremstyle{definition}
\newtheorem{theorem}{Теорема}
\newtheorem*{definition}{Определение}
\newtheorem*{obozn}{Обозначение}

\setlength{\parindent}{0pt}

\newcommand{\dbltilde}[1]{\accentset{\approx}{#1}}
\newcommand{\intt}{\int\!}

% magical thing that fixes paragraphs
\makeatletter
\patchcmd{\CatchFBT@Fin@l}{\endlinechar\m@ne}{}
  {}{\typeout{Unsuccessful patch!}}
\makeatother

\newcommand{\get}[2]{
    \ExecuteMetaData[#1]{#2}
}

\newcommand{\getproof}[2]{
    \iftoggle{useproofs}{\ExecuteMetaData[#1]{#2proof}}{}
}

\newcommand{\getwithproof}[2]{
    \get{#1}{#2}
    \getproof{#1}{#2}
}

\newcommand{\import}[3]{
    \subsection{#1}
    \getwithproof{#2}{#3}
}

\newcommand{\given}[1]{
    Дано выше. (\ref{#1}, стр. \pageref{#1})
}

\renewcommand{\ker}{\text{Ker }}
\newcommand{\im}{\text{Im }}
\newcommand{\grad}{\text{grad}}

\usepackage{bm}
\usepackage{xcolor}

\lhead{Конспект к опросу}
\cfoot{}
\rfoot{}

\begin{document}
\section{Определения}
\subsection{Упорядоченная пара}
Для некоторого множества $X$ и $A$ - множество ``индексов'', тогда $(x_\alpha)_{\alpha\in A}$ - \textbf{семейство элементов} $X$. ($\forall \alpha\in A \ \ x_\alpha \in X$)

\noindent
\textbf{Упорядоченная пара} --- семейство из двух элементов, где множеством индексов является $\{1, 2\}$. Обозначается $(a, b)$.
\subsection{Декартово произведение}
\textbf{Декартово произведение} двух множеств --- множество всех упорядоченных пар элементов этих множеств. $A\times B=\{(a,b): a\in A, b\in B\}$
\subsection{Аксиомы вещественных чисел}
\subsubsection{Аксиомы поля}
В множестве $\mathbb{R}$ определены две операции, называемые сложением и умножением, действующие из $\mathbb{R}\times\mathbb{R}$ в $\mathbb{R}$ ($+,\cdot:\mathbb{R}\times\mathbb{R} \rightarrow \mathbb{R}$), удовлетворяющие следующим свойствам:

Аксоимы сложения \textit{(здесь и далее $\forall a\in \R, b\in \R, c\in \R$)}:
\begin{enumerate}
    \itemsep0em
    \item $a+b=b+a$ --- коммутативность
    \item $(a+b)+c=a+(b+c)$ --- ассоциативность
    \item $\exists \bm{0}: \bm0+a=a$
    \item $\exists a': a+a'=\bm 0$
\end{enumerate}

Аксиомы умножения:
\begin{enumerate}
    \item $ab=ba$ --- коммутативность
    \item $(ab)c=a(bc)$ --- ассоциативность
    \item $\exists \bm 1\not =\bm 0: \forall a\in \mathbb{R}: a\cdot\bm 1=a$
    \item $\forall a\not =\bm 0: \exists \tilde{a}: a\cdot \tilde{a}=\bm 1$
\end{enumerate}

Аксоима комбинации сложения и умножения:
\begin{enumerate}
    \item $(a+b)c=ac+bc$ --- дистрибутивность
\end{enumerate}

\textbf{Поле} --- множество, в котором определены операции $+,\cdot$, удовлетворяющие группе аксиом I. Например, $\mathbb{R}, \mathbb{Q}, \mathbb{F}_3$

\subsubsection{Аксиомы порядка}
\begin{enumerate}
    \item $\forall x,y \in \mathbb{R}: x\leq y \text{ или } y\leq x$
    \item $x\leq y; y\leq x \Rightarrow x=y$
    \item $x\leq y; y\leq z \Rightarrow x\leq z$ --- транзитивность
    \item $x\leq y \Rightarrow \forall z\in \mathbb{R}: x+z\leq y+z$
    \item $0\leq x; 0\leq y \Rightarrow 0\leq xy$
\end{enumerate}

\textbf{Упорядоченное поле} --- множество, для которого выполняются аксиомы групп I и II.

$\mathbb{F}_3, \mathbb{C}$ - не упорядоченные поля

$\mathbb{R}, \mathbb{Q}, \mathcal{R}$ - упорядоченные поля

\subsection{Аксиома Кантора, аксиома Архимеда}
\subsubsection{Аксиома Архимеда}
$$\forall x,y>0: \exists n\in \R: nx>y$$

Следствие: существуют сколько угодно большие натуральные числа: $$\forall y\in \mathbb{R}: \exists n\in\mathbb{N}: n>y$$

\textbf{Архимедовы поля} --- упорядоченные поля, в которых выполняется Аксиома Архимеда.

$\mathcal{R}$ - не архимедово поле

$\mathbb{R}, \mathbb{Q}$ - архимедовы поля

\subsubsection{Аксиома Кантора}

Для последовательности вложенных отрезков $\{[a_b,b_n]\}_{n=1}^\infty$ ($\forall n\in\N a_n\leq a_{n+1}\leq b_{n+1}\leq b_n$)

$$\bigcap\limits_{n=1}^\infty[a_n, b_n] \not = \text{\O}$$

$\Q$ не удволетворяет этой аксиоме, в отличие от $\R$.

\subsection{Пополненное множество вещественных чисел, операции и порядок в нем}
\textcolor{red}{Это дополнение?}

\subsection{Максимальный элемент множества}
$M\in A$ называется \textbf{максимальным элементом} множества $A$, если $\forall a\in A \ \ a\leq M$

\subsection{Последовательность}

$x:\N \rightarrow Y$ --- \textbf{последовательность}

\subsection{Образ и прообраз множества при отображении}

Для $A\subset X, f:X\to Y$ \textbf{образ} --- множество $\{f(x), x\in A\} \subset Y$ --- обозначается $f(A)$

\noindent
Для $B\subset Y$ \textbf{прообраз} --- $\{ x\in X : f(x)\in B \}$ --- обозначается $f^{-1}(B)$

\subsection{Инъекция, сюръекция, биекция}

\textbf{Сюръекция} --- такое отображение $f: X\to Y$, что $f(X)=Y$, т.е. $\forall y\in Y \ \ f(x)=y$ имеет решение относительно $x$.

\noindent
\textbf{Инъекция} --- такое отображение $f: X\to Y$, что $\forall x_1, x_2 \in X, x_1\not=x_2 \ \ f(x_1)\not=f(x_2)$, т.е. $\forall y\in Y \ \ f(x)=y$ имеет не более одного решения относительно $x$.

\noindent
\textbf{Биекция} --- отображение, являющееся одновременно сюръекцией и инъекцией, т.е. $\forall y\in Y \ \ f(x)=y$ имеет ровно одно решение относительно $x$.

\subsection{Векторнозначаная функция, ее координатные функции}

Если $F:X\to \R^m; x\mapsto F(x) = (F_1(x), ..., F_m(x))$, то $F$ --- \textbf{векторнозначная функция} \textit{(значения функции - вектора)}

\noindent
$F_1(x)..F_m(x)$ - координатные функции отображения $F$

\subsection{График отображения}

$$\Gamma_f = \{(x,y)\in X\times Y: y = f(x) \}$$

\subsection{Композиция отображений}

$f: X\to Y, g: Y\to Z$, тогда \textbf{композиция} $f$ и $g$ \textit{(обозначается $g\circ f$)} --- такое отображение, что $g\circ f: X\to Z, x\mapsto g(f(x))$.

\noindent
Также возможно определение, которое допускает $g: Y_1\to Z, Y_1\subset Y$

\subsection{Сужение и продолжение отображений}

Для $g: X\to Y\ \ $ $f$ --- \textbf{сужение} $g$ на множество $A$, если $f: A\to Y, A\subset X$.

\noindent
$g$ называется \textbf{продолжением} $f$.

\subsection{Предел последовательности \textit{(эпсилон-дельта определение)}}

Если для $(x_n), a\in\mathbb{R}$ выполняется
$\forall \varepsilon > 0 \ \ \exists N \ \ \forall n>N \ \ |x_n-a|<\varepsilon$,
то $a$ --- \textbf{предел последовательности} $(x_n)$, обозначается $x_n\to a$ или $\lim\limits_{n\to\infty}x_n=a$

\subsection{Окрестность точки, проколотая окрестность}

\textbf{Окрестность точки} $a = \{x\in\R: |x-a|<\varepsilon\}$, обозначается $U_\varepsilon(a)$

\noindent
\textbf{Проколотая окрестность} точки $a = U_\varepsilon(a)\setminus \{a\}$, обозначается $\dot U_\varepsilon(a)$

\subsection{Предел последовательности (определение на языке окрестностей)}

$$\forall U(a) \ \ \exists N \ \ \forall n>N \ \ x_n\in U(a)$$

\end{document}