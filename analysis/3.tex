\documentclass[12pt]{article}
\usepackage{xltxtra}
\usepackage{libertine}
\usepackage{amsmath}
\usepackage{amsthm}
\usepackage{amsfonts}
% \usepackage{bbold}
\usepackage{enumitem}
\usepackage{fullpage}

\setmainfont{Linux Libertine}

\title{Конспект по математическому анализу}
\date{September 23, 2019}

\theoremstyle{plain}
\newtheorem{theorem}{Теорема}
\newtheorem{lemma}{Лемма}

\theoremstyle{remark}
\newtheorem*{remark}{Примечание}
\newtheorem*{consequence}{Следствие}

\theoremstyle{definition}
\newtheorem*{definition}{Определение}

\begin{document}

\maketitle

Определение предела дает функцию $N(\varepsilon)$, хорошо приспособленную
для изучения неравенства $\rho(x_n,a)<\varepsilon$ для $n\in(N;+\infty)$.
Кроме того, для последовательности $r_n=\rho(x_n, a) \quad |r_n|<\varepsilon$.

\begin{theorem}
\textbf{О единственности предела.}
Если $(x,\rho)$ --- метрическое пр-во, $a,b\in X$, $(x_n)$ --- послед.~в~$X$,
$x_n\xrightarrow[n\to +\infty]{} a$, $x_n\to b$,
\textbf{тогда} $a=b$.
\end{theorem}

\begin{proof}
$$\text{Докажем от противного --- пусть }a\not =b. \text{Возьмем } 0<\varepsilon<\frac{1}{2}\rho(a,b)$$
$$\exists N(\varepsilon) \ \ \forall n>N(\varepsilon) \ \ \rho(x_n,a)<\varepsilon$$
$$\exists K(\varepsilon) \ \ \forall n>K(\varepsilon) \ \ \rho(x_n,b)<\varepsilon$$
$$\text{При } n>\max(N(\varepsilon),K(\varepsilon)) \quad \rho(a,b)\leq
\rho(a,x_n)+\rho(b,x_n)<2\varepsilon<\rho(a,b) \text{ --- противоречие}$$
\end{proof}

\begin{definition}
$A\subset X$ --- {\bf ограничено}, если $\exists x_0\in X \ \ \exists R>0 \ \ A\subset B(x_0, R)$
\end{definition}

Пусть $b\in X$. $A$ --- огр. $\Leftrightarrow \exists r>0 \ \ A\subset B(b,r)$

$A\subset B(x_0, R)\Rightarrow A\subset B(b,\rho(x_0,b)\pm R)$

\begin{theorem}
Если $(x,\rho)$ --- метрическое пр-во, $(x_n)$ --- послед. в $X$, $x_n$ сходится, \textbf{тогда} $x_n$ - ограничен.
\end{theorem}

\begin{proof}
$$\text{Пусть } a=\lim\limits_{n\to +\infty} x_n$$
$$\forall U(a) \ \ \exists N \ \ \forall n>N \ \ x_n\in U(a)$$
$$U(a)=\rho(a,\varepsilon) $$
$$r:=max(\varepsilon, \rho(x_1, a), \rho(x_2, a) \ldots \rho(x_N, a))+1 $$
$$\text{тогда } \forall n\in\mathbb{N} \ \ x_n\in B(a,r)$$
\end{proof}

\section*{Порядковые свойства предела}

\begin{theorem}
\textbf{О предельном переходе в неравенствах для $\mathbb{R}$.} Если $(x_n),(y_n)$ ---
вещественные последовательности $x_n\to a, y_n\to b$, $\forall n \ x_n\leq y_n$, \textbf{тогда} $a\leq b$.
\end{theorem}

\begin{proof}
$$\text{Докажем от противного. Пусть } a>b, 0<\varepsilon <\frac{a-b}{2}.$$
$$\exists N(\varepsilon) \ \ \forall n>N \ \ a-\varepsilon<x_n<a+\varepsilon$$
$$\exists K(\varepsilon) \ \ \forall n>K \ \ b-\varepsilon<y_n<b+\varepsilon $$
$$\text{При } n>\max(N,K) \ \ y_n < b + \varepsilon < a-\varepsilon <x_n \text{ --- противоречие}$$
\end{proof}

\begin{remark}
Если вместо "$\forall n \ \ x_n\leq y_n$" потребовать: "$\exists M \ \ \forall n>M \ \ x_n\leq y_n$", то утв. по-прежнему верно
\end{remark}

\begin{remark}
$x_n=-\frac{1}{n} \ \ y_n=\frac{1}{n}$. тогда $x_n\to 0, y_n\to 0$. $x_n<y_n$, но пределы совпадают. То есть даже если $x_n<y_n$ строго, $a\leq b$ --- нестрого.
\end{remark}

\begin{consequence}
$(x_n)$ --- вещественная последовательность, $a,b\in\mathbb{R}$
\begin{enumerate}
\item $\forall n \ \ x_n\leq a \Rightarrow \lim x_n\leq a$
\item $\forall n \ \ x_n\geq b \Rightarrow \lim x_n\geq b$
\item $\forall n \ \ x_n\in [a,b] \Rightarrow \lim x_n\in [a,b]$
\end{enumerate}
\end{consequence}

\begin{theorem}
\textbf{О двух городовых} (о сжатой последовательности). Если $(x_n),(y_n),(z_n)$ - вещ. посл., $\forall n \ \ x_n\leq y_n\leq z_n, \lim x_n=\lim z_n =a$, \textbf{тогда} $\exists \lim y_n=a$
\end{theorem}

\begin{proof}
$$\forall \varepsilon>0 \ \ \exists N \ \ \forall n>N \ \ a-\varepsilon<x_n<a+\varepsilon$$
$$\forall \varepsilon>0 \ \ \exists K \ \ \forall n>K \ \ a-\varepsilon<z_n<a+\varepsilon$$
$$\forall \varepsilon > 0 \ \ \exists N_0=max(N,K) \ \ \forall n>N_0 \ \ a-\varepsilon<x_n\leq y_n\leq z_n<a+\varepsilon$$
$$\text{По определению } \lim y_n=a$$
\end{proof}

\begin{consequence}
$(y_n), (z_n) \ \ \forall n \ \ |y_n|\leq z_n, \ \ \exists \lim z_n=0, \text{ тогда } y_n\to 0$. Доказательство тривиально, т.к. $y_n$ ограничено $z_n$ и $-z_n$.
\end{consequence}

\begin{definition}
$(x_n)$ --- вещ. посл. называется \textbf{бесконечно малой}, если $x_n\to 0$
\end{definition}

\begin{theorem}
Если $(x_n),(y_n)$ --- вещ. посл., $x_n$ --- беск.мал., $y_n$ --- огр., \textbf{тогда} $x_ny_n$ --- беск.мал.
\end{theorem}

\begin{proof}
$$\exists R \ \ \forall n \ \ |y_n|<R, \text{т.к.} y_n \text{--- огр.} $$
$$|x_ny_n|\leq R|x_n|, R|x_n|\to 0 \Rightarrow y_n\to0$$
\end{proof}

\section*{Нормированные пространства}

\begin{definition}
Если $K$ --- поле ($K=\mathbb{R}$ или $\mathbb{C}$), $X$ --- множество, то $X$ называется линейным пространством над полем $K$ (и тогда $K$ называется полем скаляр), если:
\begin{enumerate}
\item $+:X\times X \to X$
\item $\cdot:K\times X\to X$
\end{enumerate}
\end{definition}

\begin{definition}
Норма - отображение $X\to\mathbb{R}, x\mapsto ||x||$, если $X$ - линейное пространство (над $\mathbb{R}$ или $\mathbb{C}$) и выполняется следующее:
\begin{enumerate}
\item $\forall x \ \ ||x||\geq 0, ||x||=0\Leftrightarrow x=0$
\item $\forall x\in X \ \ \forall x\in\mathbb R(\mathbb{C}) \ \ ||\lambda x||=|\lambda|\cdot||x||$
\item Неравенство треугольника: $\forall x,y\in X \ \ ||x+y||\leq||x||+||y||$
\end{enumerate}
\end{definition}

\begin{definition}
Полунорма - норма без свойства $||x||=0\Leftrightarrow x=0$
\end{definition}

\begin{definition}
Нормированное пространство --- $(X, ||\cdot||)$, где $||||$ - норма
\end{definition}

\begin{lemma}
О свойстве полунормы.
\begin{enumerate}
\item $p(\sum\limits_{finite}\lambda_kx_k)\leq\sum\lambda_kp(x_k)$
\item $p(0)=0$ - тут $0\in X$
\item $p(-x)=p(x)$
\item $|p(x)-p(y)|\leq p(x-y)$
\end{enumerate}
\end{lemma}

\begin{proof}
\begin{enumerate}
\item $p(\lambda_1x_1+\lambda_2x_2+...)\leq p(\lambda_1x_1)+p(\lambda_2x_2+...)$
\item тривиально
\item тривиально
\item $-p(x-y)\leq p(x)-p(y)\leq p(x-y)\\$
$p(x)=p(y+(x-y))\leq p(y)+p(x-y)$
\end{enumerate}
\end{proof}

Примеры норм:
\begin{enumerate}
\item $X=\mathbb R^m \ \ ||x||=\sqrt{\sum\limits_{i}^m x_i^2} \\$
$X=\mathbb{C}^m \ \ ||x||=\sqrt{\sum\limits_{i}^m |x_i|^2}$
\item $(\mathbb R^m, ||\cdot||_\infty) \ \ ||x||_\infty=max(|x_1|,|x_2|,...,|x_m|)$
\item $(\mathbb R^m, ||\cdot||_1) \ \ ||x||_1=\sum\limits_i^m |x_i|$
\begin{enumerate}
\item $p(x)=|x_1|$ --- полунорма, но не норма
\end{enumerate}
\end{enumerate}

\begin{remark}
Если $(X, ||\cdot||)$ --- норм. пр-во, тогда $\rho(x,y):=||x-y||$ --- метрика,
порожденная нормой. Не все метрики порождены нормами, например $\rho=\frac{|x-y|}{1+|x-y|}$.
\end{remark}

\section*{Арифметические свойства предела}

\begin{theorem}
\textbf{Об арифметических свойствах предела в нормированном пространстве.}
Если $(X, ||\cdot||)$ --- норм. пр-во, $(x_n),(y_n)$ --- посл. в $X$, $\lambda_n$ ---
посл. скаляров, и $x_n\to x_0, y_n\to y_0, \lambda_n\to \lambda_0$, \textbf{тогда}:
\begin{enumerate}
\item $\forall \varepsilon>0 \ \ \exists N \ \ \forall n>N \ \ ||x_n-x_0||<\varepsilon$
\item $||\lambda_nx_n-\lambda_0x_0||\to x_0$
\item $|||x_n||-||x_0|||\leq||x_n-x_0||$
\end{enumerate}
\end{theorem}

\begin{theorem}
\textbf{Об арифметических свойствах пределов в $\mathbb{R}$}. Для $(x_n),(y_n),(z_n)$ 
--- вещ.посл., $\forall n \ \ y_n\not =0, y_0\not = 0$:
\begin{enumerate}[resume]
\item $\frac{x_n}{y_n}\to\frac{x_0}{y_0}$
\end{enumerate}
\end{theorem}

Доказательство есть, вы там держитесь.

\end{document}