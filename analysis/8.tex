\documentclass[12pt, a4paper]{article}

\usepackage{lastpage}
\usepackage{mathtools}
\usepackage{xltxtra}
\usepackage{libertine}
\usepackage{amsmath}
\usepackage{amsthm}
\usepackage{amsfonts}
\usepackage{amssymb}
\usepackage{enumitem}
\usepackage{xcolor}
\usepackage[left=1.5cm, right=1.5cm, top=2cm, bottom=2cm, bindingoffset=0cm, headheight=15pt]{geometry}
\usepackage{fancyhdr}
\usepackage[russian]{babel}
% \usepackage[utf8]{inputenc}
\usepackage{catchfilebetweentags}
\usepackage{accents}
\usepackage{calc}
\usepackage{etoolbox}
\usepackage{mathrsfs}
\usepackage{wrapfig}

\providetoggle{useproofs}
\settoggle{useproofs}{false}

\pagestyle{fancy}
\lfoot{M3137y2019}
\rhead{\thepage\ из \pageref{LastPage}}

\newcommand{\R}{\mathbb{R}}
\newcommand{\Q}{\mathbb{Q}}
\newcommand{\C}{\mathbb{C}}
\newcommand{\Z}{\mathbb{Z}}
\newcommand{\B}{\mathbb{B}}
\newcommand{\N}{\mathbb{N}}

\newcommand{\const}{\text{const}}

\newcommand{\teormin}{\textcolor{red}{!}\ }

\DeclareMathOperator*{\xor}{\oplus}
\DeclareMathOperator*{\equ}{\sim}
\DeclareMathOperator{\Ln}{\text{Ln}}
\DeclareMathOperator{\sign}{\text{sign}}
\DeclareMathOperator{\Sym}{\text{Sym}}
\DeclareMathOperator{\Asym}{\text{Asym}}
% \DeclareMathOperator{\sh}{\text{sh}}
% \DeclareMathOperator{\tg}{\text{tg}}
% \DeclareMathOperator{\arctg}{\text{arctg}}
% \DeclareMathOperator{\ch}{\text{ch}}

\DeclarePairedDelimiter{\ceil}{\lceil}{\rceil}
\DeclarePairedDelimiter{\abs}{\left\lvert}{\right\rvert}

\setmainfont{Linux Libertine}

\theoremstyle{plain}
\newtheorem{axiom}{Аксиома}
\newtheorem{lemma}{Лемма}

\theoremstyle{remark}
\newtheorem*{remark}{Примечание}
\newtheorem*{exercise}{Упражнение}
\newtheorem*{consequence}{Следствие}
\newtheorem*{example}{Пример}
\newtheorem*{observation}{Наблюдение}

\theoremstyle{definition}
\newtheorem{theorem}{Теорема}
\newtheorem*{definition}{Определение}
\newtheorem*{obozn}{Обозначение}

\setlength{\parindent}{0pt}

\newcommand{\dbltilde}[1]{\accentset{\approx}{#1}}
\newcommand{\intt}{\int\!}

% magical thing that fixes paragraphs
\makeatletter
\patchcmd{\CatchFBT@Fin@l}{\endlinechar\m@ne}{}
  {}{\typeout{Unsuccessful patch!}}
\makeatother

\newcommand{\get}[2]{
    \ExecuteMetaData[#1]{#2}
}

\newcommand{\getproof}[2]{
    \iftoggle{useproofs}{\ExecuteMetaData[#1]{#2proof}}{}
}

\newcommand{\getwithproof}[2]{
    \get{#1}{#2}
    \getproof{#1}{#2}
}

\newcommand{\import}[3]{
    \subsection{#1}
    \getwithproof{#2}{#3}
}

\newcommand{\given}[1]{
    Дано выше. (\ref{#1}, стр. \pageref{#1})
}

\renewcommand{\ker}{\text{Ker }}
\newcommand{\im}{\text{Im }}
\newcommand{\grad}{\text{grad}}

\lhead{Конспект по матанализу}
\cfoot{}
\rfoot{November 11, 2019}

\begin{document}
\begin{proof}
    Продолжим доказательство из прошлой лекции, докажем, $3\Rightarrow1$.

    Рассмотрим секвенциально компактное $K$ и пусть $K$ --- не ограничено. \textit{(случай ограниченного множества тривиален)}

    $\exists x_n : ||x_n||\to+\infty$

    Тогда в этой последовательности нет сходящейся последовательности, т.к. любая $x_{n_k}\to x_0\in\R$ ограничена. Противоречие $\Rightarrow$ $K$ --- не компактно.

    Таким образом, если $K$ --- секвенциально компактно, то $K$ ограничено.

    Докажем замкнутость $K$.

    Пусть $\exists$ предельная точка $x_0\not\in K$

    $\exists x_n\to x_0$

    По секвенциальности $\exists$ подпоследовательность $x_{n_k}\to a\in K$.
\end{proof}

\begin{consequence}
    Принцип выбора Больцано-Вейерштрасса.

    Если в $\R^m$ $(x_n)$ --- ограниченная последовательность, то у неё существует сходящаяся подпоследовательность.
\end{consequence}
\begin{proof}
    $x_n$ --- огр. $\Rightarrow x_n$ содержится в замкнутом кубе. Так как куб секвенциально компактен, $x_{n_k}$ сходится.
\end{proof}
\begin{remark}
    $(x_n)$ --- не огр. $\Rightarrow x_n\to\infty$, т.е. $||x_n||\to+\infty$
\end{remark}
\begin{definition}
    $X$ --- метрическое пространство, $(x_n)$ в $X$

    $x_n$ --- \textbf{фундаментальная, последовательность Коши, сходящаяся в себе}.
    
    $$\forall \varepsilon>0 \ \ \exists N \ \ \forall m,n > N\ \ \rho(x_m, x_n)<\varepsilon$$
\end{definition}
\begin{lemma}
    \begin{enumerate}
        \item $x_n$ --- фунд. $\Rightarrow x_n$ --- ограничена
        \item $x_n$ --- фунд; $\exists x_{n_k}$ --- сходящ. Тогда $x_n$ сходится. 
    \end{enumerate}
\end{lemma}
\begin{proof}
    \begin{enumerate}
        \item $\varepsilon:=1 \ \ \exists N \ \ \forall m, n:=N+1 > N \ \ \rho(x_m, x_{N+1})<1$
        
        $R:=\max(1; \rho(x_1, x_{N+1}),\ldots,\rho(x_N, x_{N+1}))$

        $\forall n \ \ x_n\in B(x_{N+1}, R) \Rightarrow x_n$ сходится.

        \item $\begin{cases}
            \varepsilon > 0 \ \ \exists K \ \ \forall k>K \ \ \rho(x_{n_k}, a)<\varepsilon \\
            \varepsilon > 0 \ \ \exists N \ \ \forall m,n>N \ \ \rho(x_m, x_n)<\varepsilon
        \end{cases} \xRightarrow{?} x_n\to a$

        $\forall \varepsilon > 0 \ \ \exists \tilde N := \max(N, K)$ при $k>\tilde N$ выполняется $k>K$, значит $n_k\geq k > K \Rightarrow \rho(x_{n_k}, a)<\varepsilon$.

        При $n>\tilde N\geq N \ \  m:=n_k>\tilde N\geq N \Rightarrow \rho(x_n, x_{n_k})<\varepsilon$

        Итого $\forall n>\tilde N \ \ \rho(x_n, a)\geq \rho(x_n, x_{n_k})<2\varepsilon$
    \end{enumerate}
\end{proof}
\begin{theorem}
    \begin{enumerate}
        \item В любом метрическом пространстве $x_n$ --- сходящ. $\Rightarrow x_n$ --- фунд.
        \item В $\R^m$ $x_n$ --- фунд. $\Rightarrow x_n$ --- сходящ.
    \end{enumerate}
\end{theorem}
\begin{proof}
    \begin{enumerate}
        \item $x_n\to a \quad \forall \varepsilon>0 \ \ \exists N \ \ \forall n>N \ \ \rho(x_n, a)<\varepsilon$
    
        $x_n\to a \quad \forall \varepsilon>0 \ \ \exists N \ \ \forall n,m > N \rho(x_m, x_n)\geq \rho(x_n, a) + \rho(x_m, a) < 2\varepsilon$

        \item $x_n$ --- фунд. $\Rightarrow x_n$ --- огр. $\xRightarrow{\text{Б.-В.}} \exists x_{n_k}$ --- сходящ. 
        
        $\begin{cases}
            \exists x_{n_k}\text{ --- сходящ.} \\
            x_n\text{ --- фунд.}
        \end{cases} \Rightarrow x_n$ --- сходящ.
    \end{enumerate}
\end{proof}
\begin{definition}
    $X$ --- метрическое пространство называется \textbf{полным}, если в нём любая фундаментальная последовательность --- сходящаяся.
\end{definition}

Верно: $x_n$ --- вещ. посл.

$$\forall \varepsilon>0 \ \ \exists N \ \ \forall n,m>N \ \ |x_n-x_m|<\varepsilon \Leftrightarrow \exists \text{ конечн. }\lim\limits_{n\to+\infty} x_n$$

Это критерий Больцано-Коши.

$f:D\subset X\to Y$, $x_0$ --- предельная точка $D$.

$\lim\limits_{x\to x_0} f(x)=L$

$D_1\subset D, x_0$ --- предельная точка $D_1$.

%<*пределпомножеству>
Предел при $x\to x_0$ по множеству $D_1$ --- это $\lim\limits_{x\to x_0} f|_{D_1}$
%</пределпомножеству>

\begin{definition}
    %<*односторонниепределы>
    В $\R$ одностор. $=\{$ левостор., правостор. $\}$

    Левостор. $\lim\limits_{x\to x_0-0}f(x)=L$ - это $\lim f|_{D\cap(-\infty, x_0)}$

    $$\forall \varepsilon > 0 \ \ \exists \delta > 0 \ \ \forall x\in(x_0-\delta, x_0)\cap D \ \ |f(x) - L|<\varepsilon$$

    Аналогично правостор.
    %</односторонниепределы>
\end{definition}

Если $\lim\limits_{x\to x_0-0} f = \lim\limits_{x\to x_0+0} f=L \Rightarrow \lim\limits_{x\to x_0} f = L$

$\lim\limits_{x\to x_0-0} f \stackrel{\text{обозн.}}{=} f(x_0-0)$

$\lim\limits_{x\to 0-0} f = \lim\limits_{x\to -0} f$

В $\R^2 \lim\limits_{(x_1,x_2)\to(a_1, a_2)} f$

Предел вдоль прямой: $\lim_{r\to 0} f(a_1+r\cos \alpha, a_2+r\sin\alpha)$

\begin{theorem}
    О пределе монотонной функции

    $f:D\subset \R \to \R$, монотонная, $a\in\overline\R$

    $D_1:=D\cap (-\infty, a), a$ --- пред. точка $D_1$

    \begin{enumerate}
        \item $f$ --- возрастает, огр. сверху $D_1$. Тогда $\exists$ конечный предел $\lim\limits_{x\to a-0} f(x)$
        \item $f$ --- убывает, огр. снизу $D_1$. Тогда $\exists$ конечный предел $\lim\limits_{x\to a-0} f(x)$
    \end{enumerate}
\end{theorem}
\begin{proof}
    \begin{enumerate}
        \item $L:= \sup\limits_{D_1} f \quad L\stackrel{?}{=}\lim\limits_{x\to a-0} f(x)$
    
        $\forall \varepsilon > 0 \ \ L-\varepsilon$ --- не верхн. граница для $\{f(x) : x\in D_1\} \ \ \exists x_1: L_\varepsilon<f(x_1)$.

        Тогда при $x\in(x_1, a)\cap D_1 \ \ L-\varepsilon < f(x_1) \leq f(x) \leq L$

        $\exists \delta:=|x_1-a| \ \ \forall x : x\in(x_1, a) \ \ L_\varepsilon\leq f(x) < L+\varepsilon$

        Аналогично доказывается пункт 2.
    \end{enumerate}
\end{proof}

Критерий Больцано-Коши для отображений.

\begin{theorem}
    $f:D\subset X\to Y$, $a$ --- пр. точка $D$, $Y$ --- полное метрическое пространство.

    Тогда $$\exists \lim\limits_{x\to a}f(x)\in Y \Leftrightarrow \forall \varepsilon > 0 \ \ \exists \delta>0 \ \ \forall x_1, x_2\in D \ \ \rho(x_1,a)<\delta; \rho(x_2, a) < \delta \ \ \rho(f(x_1), f(x_2))<\varepsilon$$
\end{theorem}
\begin{proof}
    ``$\Rightarrow$'' как для последовательностей.

    Докажем ``$\Leftarrow$'' по Гейне.

    Заметим, что последовательность $f(x_n)$ --- фундаментальная, т.е.

    $$\forall \varepsilon>0 \ \ \exists N \ \ \forall m,n > N\ \ \rho(f(x_m), f(x_n))<\varepsilon$$

    $$x_n\to a \Rightarrow \exists N \ \ \forall n>N \ \ \rho(x_n, a)<\delta$$

    $$\forall m,n > N \ \ \rho(x_n, a)<\delta; \rho(x_m, a)<\delta \xRightarrow{\text{Фунд.}} \rho(f(x_n), f(x_m))<\varepsilon$$
\end{proof}
\begin{remark}
    В $\R$ критерий Больцано-Коши для функций

    $f:D\subset \R\to\R, a$ --- пред. точка $D$

    $$\forall \varepsilon>0 \ \ \exists \delta>0 \ \ \forall x_1, x_2 \in D\setminus\{a\} \ \ |x_1-a|<\delta; |x_2-a|<\delta$$
\end{remark}

Для $\lim\limits_{x\to x_0} f(x)=+\infty$ критерий Больцано-Коши: $$\forall E \ \ \exists \delta>0 \ \ \forall x_1, x_2 \in D\setminus\{a\} \ \ |x_1-a|<\delta; |x_2-a|<\delta \ \ f(x_1)>E; f(x_2)>E$$ неинтересно.

Для $\lim\limits_{x\to+\infty}f(x)=L$:

$$\forall \varepsilon>0 \ \ \exists \Delta \ \ \forall x_1, x_2\in D \ \ x_1>\Delta; x_2>\Delta \ \ |f(x_1)-f(x_2)|<\varepsilon$$

\section{Непрерывные отображения}

\begin{definition}
    %<*непрерывноеотображение>
    $f:D\subset X\to Y \quad x_0\in D$

    $f$ --- \textbf{непрерывное} в точке $x_0$, если:
    \begin{enumerate}
        \item $\lim\limits_{x\to x_0} f(x)=f(x_0)$, либо $x_0$ --- изолированная точка $D$
        \item $\forall \varepsilon > 0 \ \ \exists \delta>0 \ \ \forall x\in D \ \ \rho(x, x_0)<\delta \ \ \rho(f(x), f(x_0))<\varepsilon$
        \item $\forall U(f(x_0)) \ \ \exists V(x_0) \ \ \forall x\in V(x_0)\cap D \ \ f(x)\in U(f(x_0))$
        \item По Гейне $\forall (x_n):x_n\to x_0; x_n\in D \ \ f(x_n)\xrightarrow[n\to+\infty]{} f(x_0)$
    \end{enumerate}
    %</непрерывноеотображение>
\end{definition}

\begin{definition}
    %<*точкаразрыва>
    Если $\not\exists\lim\limits_{x\to x_0} f(x)$, либо $\exists\lim\limits_{x\to x_0}f(x)\not=f(x_0)$ --- \textbf{точка разрыва}.
    %</точкаразрыва>
\end{definition}

Для $\R \quad \forall \varepsilon>0 \ \ \exists \delta > 0 \ \ \forall x\in D \ \ |x-x_0|<\delta \ \ |f(x) - f(x_0)|<\varepsilon$

\begin{definition}
    Непр. слева и непр. справа
    %<*непрерывностьслева>
    $f$ --- непр. слева в $x_0$, если $f|_{(-D, x_0]\cap D}$ --- непрерывно в $x_0$
    %</непрерывностьслева>
\end{definition}

Если $f$ непрерывно слева и непрерывно справа в $x_0$, то $f$ непрерывно в $x_0$.

\begin{definition}
    %<*родточкиразрыва>
    Пусть $\exists f(x_0-0), f(x_0+0)$ и не все 3 числа равны: $f(x_0-0), f(x_0), f(x_0+0)$. Это \textbf{разрыв I рода \textit{(скачок)}}.

    Остальные точки разрыва --- \textbf{разрыв II рода}.
    %</родточкиразрыва>
\end{definition}

\begin{example}
    \begin{enumerate}
        \item $f(x)=sign(x)=\begin{cases}
            1, x>0 \\
            0, x=0 \\
            -1, x<0
        \end{cases}$ $0$ --- разрыв I рода.

        \item $f(x)=sin(\frac{1}{x})$ $0$ --- разрыв II рода.
    \end{enumerate}
\end{example}

\begin{definition}
    Отображение непрерывно на множестве $D = $ непрерывно в каждой точке множества $D$.
\end{definition}

% \begin{theorem}
    \begin{enumerate}
        \item Арифметические свойства
        \begin{enumerate}
            \item $f,g:D\subset X\to Y \ \ x_0\in D$ ($X$ --- норм. пространство)
            
            $f, g$ --- непр. в $D; \lambda:D\to\R(\C)$ --- непр. $x_0$
            
            Тогда $f\pm g, ||f||, \lambda f$ --- непр. $x_0$

            \item $f,g:D\subset X\to \R \ \ x_0\in D$
            
            $f, g$ --- непр. в $x_0$

            Тогда $f\pm g, |f|, fg$ --- непр. в $x_0$

            $g(x_0)\not=0$, тогда $\frac{f}{g}$ --- непр. $x_0$
        \end{enumerate}
    \end{enumerate}
% \end{theorem}

\end{document}