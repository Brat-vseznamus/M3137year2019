\chapter{19 марта}

\begin{example}
    \begin{prooftree}
        \hypo{\varphi \to \psi}
        \infer1{\exists x.(\varphi \to \psi)}
    \end{prooftree} --- возможно доказуемо, но это не правило вывода для \(\exists \).
\end{example}

\begin{definition}
    \(\alpha_1 \dots \alpha_n\) --- \textbf{доказательство}, если выполняется одно из:
    \begin{enumerate}
        \item \(\alpha_i\) --- аксиома
        \item Существует \(j, k < i\), такие что \(\alpha_k = \alpha_j \to \alpha_i\)
        \item Существует \(j\), такое что \(\alpha_j = \varphi \to \psi\) и \(\alpha_i = (\exists x. \varphi) \to \psi\), причём \(x\) не входит свободно в \(\psi\).
        \item Существует \(j\), такое что \(\alpha_j = \psi \to \varphi\) и \(\alpha_i = \psi \to \forall x.\varphi\), причём \(x\) не входит свободно в \(\psi\).
    \end{enumerate}
\end{definition}

\subsection{Вхождение}

Рассмотрим некоторую формулу и рассмотрим вхождения \(x\) в неё:
\[(P(\underbrace{x}_{1}) \lor Q(\underbrace{x}_2)) \to (R(\underbrace{x}_3) \with (\overbrace{\forall \underbrace{x}_4.P_1(\underbrace{x}_5)}^{\text{Область действия \(\forall \) по \(x\)}}))\]

\begin{itemize}
    \item Вхождение 4 связывающее
    \item Вхождение 5 связано вхождением 4
    \item Вхождения 1-3 свободны.
\end{itemize}

Случай множественного связывания:
\[\underbrace{\forall x. \forall y. \overbrace{\forall x. \forall y. \underbrace{\forall x. P(x)}}^{\text{Область действия \(\forall \) по \(x\)}}}_{\text{Область действия \(\forall \) по \(x\)}}\]

\begin{definition}
    Вхождение \textbf{свободно}, если не связано.
\end{definition}

\begin{remark}
    Свободно входящие переменные нельзя переименовывать, т.к. к формуле могут приписать кванторы, которые используют данные имена переменных. Это ограничение не распространяется на связанные переменные.
\end{remark}

Любая аксиома есть предикат.

\subsection{Свобода для подстановки}

\begin{definition}
    \(\theta\) \textbf{свободен для подстановки} вместо \(x\) в \(\varphi\), если никакая свободная переменная в \(\theta\) не станет связанной в \(\varphi[x: = \theta]\)
\end{definition}

\begin{obozn}
    \(\varphi[x: = \theta]\) --- заменить все свободные вхождения \(x\) в \(\varphi\) на \(\theta\)
\end{obozn}

\begin{example}
    \[(\forall x.\forall y.\forall x.P(x))[x: = y] \equiv \forall x.\forall y.\forall x.P(x)\]
    \[P(x)\lor \forall x.P(x)[x: = y] \equiv P(y)\lor \forall x.P(x)\]
    \[(\forall y.x = y)[x: = y] \equiv \forall y.y = y\]
    В этой формуле новый \(y\) связался.
\end{example}

\begin{remark}
    В определении можно опустить ``\textit{свободная}'' в нашем исчислении, но это не верно в достаточно извращенных исчислениях.
\end{remark}

\begin{lemma}
    Пусть \(\vdash \alpha\). Тогда \(\vdash \forall x.\alpha\)
\end{lemma}
\begin{proof}\itemfix
    Т.к. \(\vdash \alpha\), то существует \(\gamma_1 \dots \gamma_n : \gamma_n \equiv \alpha\)

    Создадим новое доказательство.

    \begin{tabular}{LCr}
        (1)   & \gamma_1                                &                         \\
              & \vdots                                  &                         \\
        (n)   & \gamma_n                                &                         \\
        (n+1) & A\with A \to A                          & (акс.)                  \\
        (n+2) & \alpha \to ((A\with A\to A) \to \alpha) & (акс.)                  \\
        (n+3) & (A\with A\to A) \to \alpha              & (M.P. \(n, n+2\))       \\
        (n+4) & (A\with A\to A) \to \forall x.\alpha    & (введение \(\forall \)) \\
        (n+5) & \forall x.\alpha                        & (M.P. \(n + 1, n + 4\))
    \end{tabular}
\end{proof}

\begin{lemma}
    \label{для дедукции}
    \((\alpha \to \varphi \to \psi) \to \alpha \with \varphi \to \psi\)
\end{lemma}
\begin{lemma}
    \label{для дедукции 2}
    \((\alpha \with \varphi \to \psi) \to (\alpha \to \varphi \to \psi)\)
\end{lemma}
\begin{proof}[Доказательство двух лемм]
    По теореме о полноте исчисления высказываний.
\end{proof}

\begin{theorem}[о дедукции]
    Пусть даны \(\Gamma, \alpha, \beta\).

    \begin{enumerate}
        \item Если \(\Gamma, \alpha \vdash \beta \), то \(\Gamma \vdash \alpha \to \beta\) при условии, если в доказательстве \(\Gamma, \alpha\vdash \beta\) не применялись правила для \(\forall , \exists\) по переменным, входящим свободно в \(\alpha\).

        \item Если \(\Gamma \vdash \alpha \to \beta\), то \(\Gamma, \alpha \vdash \beta \).
    \end{enumerate}
\end{theorem}

\begin{proof}
    По индукции. Пусть доказано \(\alpha \to \delta_i\) для \(i\in [1, n]\), докажем \(\alpha \to \delta_{n + 1}\).

    Рассмотрим случаи:
    \begin{enumerate}
        \item Схемы аксиом 1-10 --- аналогично\footnote{доказательству ИВ}.
        \item M.P. --- аналогично
        \item Аксиомы 11-12 --- аналогично первому пункту.
        \item Пусть \(\delta_{n + 1}\) получено правилом \(\forall \) : \(\delta_{n + 1}\equiv \varphi \to \forall x.\psi\) и существует \(\delta_k \equiv \varphi \to \psi\) и \(k \leq n\), причём \(x\) не входит свободно в \(\varphi\).

              При этом в новом доказательстве уже доказано \(\alpha \to \delta_k\)

              \begin{tabular}{LCr}
                  (1)     & \alpha \to \delta_1                                                                   &                              \\
                          & \vdots                                                                                &                              \\
                  (k)     & \alpha \to (\varphi \to \psi)                                                         &                              \\
                          & \vdots                                                                                &                              \\
                  (n)     & \alpha \to \delta_n                                                                   &                              \\
                          & \vdots                                                                                &                              \\
                  (n+0.1) & (\alpha \to \varphi \to \psi) \to \alpha \with \varphi \to \psi                       & (лемма~\ref{для дедукции})   \\
                  (n+0.2) & \alpha \with \varphi \to \psi                                                         & (M.P.)                       \\
                  (n+0.3) & \alpha \with \varphi \to \forall x.\psi                                               & (введение \(\forall \))      \\
                  (n+0.4) & (\alpha \with \varphi \to \forall x.\psi) \to (\alpha \to \varphi \to \forall x.\psi) & (лемма~\ref{для дедукции 2}) \\
                  (n+1)   & \alpha \to \varphi \to \forall x. \psi                                                & (M.P.)
              \end{tabular}
    \end{enumerate}
\end{proof}

\begin{remark}
    Доказательство пункта 2 аналогично исходному доказательству для исчисления высказываний.
\end{remark}

