\documentclass[12pt, a4paper]{article}

\usepackage{lastpage}
\usepackage{mathtools}
\usepackage{xltxtra}
\usepackage{libertine}
\usepackage{amsmath}
\usepackage{amsthm}
\usepackage{amsfonts}
\usepackage{amssymb}
\usepackage{enumitem}
\usepackage{xcolor}
\usepackage[left=1.5cm, right=1.5cm, top=2cm, bottom=2cm, bindingoffset=0cm, headheight=15pt]{geometry}
\usepackage{fancyhdr}
\usepackage[russian]{babel}
% \usepackage[utf8]{inputenc}
\usepackage{catchfilebetweentags}
\usepackage{accents}
\usepackage{calc}
\usepackage{etoolbox}
\usepackage{mathrsfs}
\usepackage{wrapfig}

\providetoggle{useproofs}
\settoggle{useproofs}{false}

\pagestyle{fancy}
\lfoot{M3137y2019}
\rhead{\thepage\ из \pageref{LastPage}}

\newcommand{\R}{\mathbb{R}}
\newcommand{\Q}{\mathbb{Q}}
\newcommand{\C}{\mathbb{C}}
\newcommand{\Z}{\mathbb{Z}}
\newcommand{\B}{\mathbb{B}}
\newcommand{\N}{\mathbb{N}}

\newcommand{\const}{\text{const}}

\newcommand{\teormin}{\textcolor{red}{!}\ }

\DeclareMathOperator*{\xor}{\oplus}
\DeclareMathOperator*{\equ}{\sim}
\DeclareMathOperator{\Ln}{\text{Ln}}
\DeclareMathOperator{\sign}{\text{sign}}
\DeclareMathOperator{\Sym}{\text{Sym}}
\DeclareMathOperator{\Asym}{\text{Asym}}
% \DeclareMathOperator{\sh}{\text{sh}}
% \DeclareMathOperator{\tg}{\text{tg}}
% \DeclareMathOperator{\arctg}{\text{arctg}}
% \DeclareMathOperator{\ch}{\text{ch}}

\DeclarePairedDelimiter{\ceil}{\lceil}{\rceil}
\DeclarePairedDelimiter{\abs}{\left\lvert}{\right\rvert}

\setmainfont{Linux Libertine}

\theoremstyle{plain}
\newtheorem{axiom}{Аксиома}
\newtheorem{lemma}{Лемма}

\theoremstyle{remark}
\newtheorem*{remark}{Примечание}
\newtheorem*{exercise}{Упражнение}
\newtheorem*{consequence}{Следствие}
\newtheorem*{example}{Пример}
\newtheorem*{observation}{Наблюдение}

\theoremstyle{definition}
\newtheorem{theorem}{Теорема}
\newtheorem*{definition}{Определение}
\newtheorem*{obozn}{Обозначение}

\setlength{\parindent}{0pt}

\newcommand{\dbltilde}[1]{\accentset{\approx}{#1}}
\newcommand{\intt}{\int\!}

% magical thing that fixes paragraphs
\makeatletter
\patchcmd{\CatchFBT@Fin@l}{\endlinechar\m@ne}{}
  {}{\typeout{Unsuccessful patch!}}
\makeatother

\newcommand{\get}[2]{
    \ExecuteMetaData[#1]{#2}
}

\newcommand{\getproof}[2]{
    \iftoggle{useproofs}{\ExecuteMetaData[#1]{#2proof}}{}
}

\newcommand{\getwithproof}[2]{
    \get{#1}{#2}
    \getproof{#1}{#2}
}

\newcommand{\import}[3]{
    \subsection{#1}
    \getwithproof{#2}{#3}
}

\newcommand{\given}[1]{
    Дано выше. (\ref{#1}, стр. \pageref{#1})
}

\renewcommand{\ker}{\text{Ker }}
\newcommand{\im}{\text{Im }}
\newcommand{\grad}{\text{grad}}

\lhead{Математическая логика \textit{(практика)}}
\cfoot{}
\rfoot{18.2.2021}

\mathtoolsset{showonlyrefs=false}

\begin{document}

\begin{enumerate}
    \item Покажите, что если $\Gamma \vdash \alpha$, то $\Gamma \models \alpha$.

          Вспомним доказательство этой теоремы без \(\Gamma\), которое было на лекции. Мы фиксировали оценку и рассматривали доказательство \(\alpha\). По индукции мы доказывали, что каждый шаг доказательства \(\llbracket \delta_n \rrbracket =\) И, и в частности последний шаг, т.е. \(\alpha\) тоже истиннен в данной подстановке. С добавлением \(\Gamma\) у нас в индукционном переходе добавился случай \(\delta_n \in \Gamma\). Но т.к. мы фиксируем оценку такую, что \(\forall \gamma \in \Gamma \ \ \llbracket \gamma \rrbracket =\) И, индукционный переход работает.

    \item Покажите, что если $\Gamma \models \alpha$, то $\Gamma \vdash \alpha$.

          \begin{enumerate}
              \item \(\llbracket \gamma_1 \to \gamma_2 \to \dots \gamma_n \to \alpha \rrbracket =\) И.
              \item \(\models \gamma_1 \to \gamma_2 \to \dots \gamma_n \to \alpha\)
              \item \(\vdash \gamma_1 \to \gamma_2 \to \dots \gamma_n \to \alpha\) по теореме о полноте
              \item \(\Gamma \vdash \alpha\) по теореме о дедукции
          \end{enumerate}

    \item \emph{О законе исключённого третьего.} Покажите, что в интуиционистском исчислении высказываний
          доказуемо следующее:

          \begin{enumerate}
              \item $((A\rightarrow B)\rightarrow A)\rightarrow A \vdash \neg\neg A \rightarrow A$
              \item $A \vee \neg A \vdash \neg\neg A \rightarrow A$

                    Докажем \(A \vee \neg A \neg\neg A \vdash A\):
                    \begin{align*}
                        1.\quad  & (A \to A) \to (\neg A \to A) \to (A\lor \neg A \to A) \tag{акс. 8}                          \\
                        2.\quad  & (A \to A) \tag{было ранее}                                                                  \\
                        3.\quad  & \neg A \to \neg \neg A \to A \tag{акс. 10}                                                  \\
                        4.\quad  & \neg \neg A \to (\neg A \to \neg \neg A) \tag{акс. 1}                                       \\
                        5.\quad  & \neg \neg A \tag{\(\in \Gamma\)}                                                            \\
                        6.\quad  & \neg A \to \neg \neg A \tag{M.P. 4,5}                                                       \\
                        7.\quad  & (\neg A \to \neg \neg A) \to (\neg A \to \neg \neg A \to A) \to (\neg A \to A) \tag{акс. 2} \\
                        8.\quad  & (\neg A \to \neg \neg A \to A) \to (\neg A \to A) \tag{M.P. 6,7}                            \\
                        9.\quad  & \neg A \to A \tag{M.P. 3,8}                                                                 \\
                        10.\quad & (\neg A \to A) \to (A\lor \neg A \to A) \tag{M.P. 1,2}                                      \\
                        11.\quad & A\lor \neg A \to A \tag{M.P. 9, 10}
                    \end{align*}
          \end{enumerate}

    \item Предложите топологические пространства и оценку для пропозициональных переменных,
          опровергающие следующие высказывания:

          \begin{enumerate}
              \item $\neg A \vee \neg\neg A $

                    \[A = (0, +\infty) \quad \neg A = ( -\infty, 0) \quad \neg \neg A = (0, +\infty) \quad \neg A \lor \neg \neg A = \R\setminus \{0\}  \]

              \item $(((A \rightarrow B) \rightarrow A) \rightarrow A)$
                    \[A = (1, 2) \cup (2, 3) \quad B = (3, 4)\]
              \item $\neg\neg A \rightarrow A$
                    \[A = \R\setminus \{0\} \]
              \item $(A \rightarrow (B \vee \neg B)) \vee (\neg A \rightarrow (B \vee \neg B))$
                    \[X = (0, 10) \quad A = (1, 2) \cup (2, 3) \quad B = X\setminus \Z\]
              \item $(A \rightarrow B) \vee (B \rightarrow C) \vee (C \rightarrow A)$
          \end{enumerate}

    \item Можно ли, имея $(A \rightarrow B) \vee (B \rightarrow C) \vee (C \rightarrow A)$, доказать
          закон исключённого третьего в интуиционистской логике?

    \item Известно, что в классической логике любая связка может быть \emph{выражена} как композиция
          конъюнкций и отрицаний: существует схема высказываний, использующая только конъюнкции и отрицания,
          задающая высказывание, логически эквивалентное исходной связке.
          Например, для импликации можно взять $\neg(\alpha\with\neg\beta)$, ведь
          $\alpha\rightarrow\beta\vdash\neg(\alpha\with\neg\beta)$ и $\neg(\alpha\with\neg\beta)\vdash\alpha\rightarrow\beta$.
          Возможно ли в интуиционистской логике выразить через остальные связки:
          \begin{enumerate}
              \item конъюнкцию?
              \item дизъюнкцию?
              \item импликацию?
              \item отрицание?
          \end{enumerate}
          Если да, предложите формулу и два вывода. Если нет --- докажите это.

    \item Назовём теорию \emph{противоречивой}, если в ней найдётся такое $\alpha$, что $\vdash\alpha$ и $\vdash\neg\alpha$.
          Покажите, что исчисления высказываний (классическое и интуиционистское) противоречивы тогда и только тогда,
          когда в них доказуема любая формула.

    \item \emph{Теорема Гливенко.} Обозначим доказуемость высказывания $\alpha$ в классической логике
          как $\vdash_\text{к}\alpha$, а в интуиционистской --- как $\vdash_\text{и}\alpha$.
          Оказывается возможным показать, что какое бы ни было $\alpha$, если $\vdash_\text{к}\alpha$,
          то $\vdash_\text{и}\neg\neg\alpha$. А именно, покажите, что:

          \begin{enumerate}
              \item Если $\alpha$ --- аксиома, полученная из схем 1--9 исчисления высказываний, то $\vdash_\text{и}\neg\neg\alpha$.
              \item $\vdash_\text{и}\neg\neg(\neg\neg\alpha\rightarrow\alpha)$
              \item $\neg\neg\alpha,\neg\neg(\alpha\rightarrow\beta) \vdash_\text{и}\neg\neg\beta$
              \item Докажите утверждение теоремы ($\vdash_\text{к}\alpha$ влечёт $\vdash_\text{и}\neg\neg\alpha$),
                    опираясь на предыдущие пункты, и покажите, что классическое исчисление высказываний противоречиво
                    тогда и только тогда, когда противоречиво интуиционистское.
          \end{enumerate}

\end{enumerate}

\end{document}