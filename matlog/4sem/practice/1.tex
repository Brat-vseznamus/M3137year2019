\documentclass[12pt, a4paper]{article}

\usepackage{lastpage}
\usepackage{mathtools}
\usepackage{xltxtra}
\usepackage{libertine}
\usepackage{amsmath}
\usepackage{amsthm}
\usepackage{amsfonts}
\usepackage{amssymb}
\usepackage{enumitem}
\usepackage{xcolor}
\usepackage[left=1.5cm, right=1.5cm, top=2cm, bottom=2cm, bindingoffset=0cm, headheight=15pt]{geometry}
\usepackage{fancyhdr}
\usepackage[russian]{babel}
% \usepackage[utf8]{inputenc}
\usepackage{catchfilebetweentags}
\usepackage{accents}
\usepackage{calc}
\usepackage{etoolbox}
\usepackage{mathrsfs}
\usepackage{wrapfig}

\providetoggle{useproofs}
\settoggle{useproofs}{false}

\pagestyle{fancy}
\lfoot{M3137y2019}
\rhead{\thepage\ из \pageref{LastPage}}

\newcommand{\R}{\mathbb{R}}
\newcommand{\Q}{\mathbb{Q}}
\newcommand{\C}{\mathbb{C}}
\newcommand{\Z}{\mathbb{Z}}
\newcommand{\B}{\mathbb{B}}
\newcommand{\N}{\mathbb{N}}

\newcommand{\const}{\text{const}}

\newcommand{\teormin}{\textcolor{red}{!}\ }

\DeclareMathOperator*{\xor}{\oplus}
\DeclareMathOperator*{\equ}{\sim}
\DeclareMathOperator{\Ln}{\text{Ln}}
\DeclareMathOperator{\sign}{\text{sign}}
\DeclareMathOperator{\Sym}{\text{Sym}}
\DeclareMathOperator{\Asym}{\text{Asym}}
% \DeclareMathOperator{\sh}{\text{sh}}
% \DeclareMathOperator{\tg}{\text{tg}}
% \DeclareMathOperator{\arctg}{\text{arctg}}
% \DeclareMathOperator{\ch}{\text{ch}}

\DeclarePairedDelimiter{\ceil}{\lceil}{\rceil}
\DeclarePairedDelimiter{\abs}{\left\lvert}{\right\rvert}

\setmainfont{Linux Libertine}

\theoremstyle{plain}
\newtheorem{axiom}{Аксиома}
\newtheorem{lemma}{Лемма}

\theoremstyle{remark}
\newtheorem*{remark}{Примечание}
\newtheorem*{exercise}{Упражнение}
\newtheorem*{consequence}{Следствие}
\newtheorem*{example}{Пример}
\newtheorem*{observation}{Наблюдение}

\theoremstyle{definition}
\newtheorem{theorem}{Теорема}
\newtheorem*{definition}{Определение}
\newtheorem*{obozn}{Обозначение}

\setlength{\parindent}{0pt}

\newcommand{\dbltilde}[1]{\accentset{\approx}{#1}}
\newcommand{\intt}{\int\!}

% magical thing that fixes paragraphs
\makeatletter
\patchcmd{\CatchFBT@Fin@l}{\endlinechar\m@ne}{}
  {}{\typeout{Unsuccessful patch!}}
\makeatother

\newcommand{\get}[2]{
    \ExecuteMetaData[#1]{#2}
}

\newcommand{\getproof}[2]{
    \iftoggle{useproofs}{\ExecuteMetaData[#1]{#2proof}}{}
}

\newcommand{\getwithproof}[2]{
    \get{#1}{#2}
    \getproof{#1}{#2}
}

\newcommand{\import}[3]{
    \subsection{#1}
    \getwithproof{#2}{#3}
}

\newcommand{\given}[1]{
    Дано выше. (\ref{#1}, стр. \pageref{#1})
}

\renewcommand{\ker}{\text{Ker }}
\newcommand{\im}{\text{Im }}
\newcommand{\grad}{\text{grad}}

\lhead{Математическая логика \textit{(практика)}}
\cfoot{}
\rfoot{12.2.2021}

\begin{document}

Т.к. эта практика проходит до первой лекции, то сначала будет немного теории.

Мы занимаемся исчислением высказываний. В высказываниях есть:
\begin{itemize}
    \item Переменные \(A, B, C\) \textit{(пропозициональные)}
    \item Битовые операции \(\lor, \with, \to, \neg\)
\end{itemize}

\begin{definition}
    \textbf{Пропозициональная формула}:
    \begin{enumerate}
        \item \(A\) --- переменная \( \Rightarrow A\) --- формула
        \item \(a, b\) --- переменные \(\Rightarrow a \lor b, a \with b, a \to b, \neg a\)
    \end{enumerate}
\end{definition}

\begin{definition}
    \textbf{Оценка} \?
\end{definition}

\begin{definition}
    \textbf{Тавтология} --- формула, для любой оценки истинна.
\end{definition}

\begin{example}\itemfix
    \begin{itemize}
        \item \(A \lor \neg A\)
        \item \(A \to A\)
        \item \(A \to (B \to A)\)
    \end{itemize}
\end{example}

Схемы аксиом:
\begin{enumerate}
    \item \(a \to (b \to a)\)
    \item \((a \to b) \to (a \to b \to c) \to (a \to c)\)
    \item \(a \to b \to a \with b\)
    \item \(a \with b \to a\)
    \item \(a \with b \to b\)
    \item \(a \to a \lor b\)
    \item \(b \to a \lor b\)
    \item \((a \to b) \to (b \to c) \to (a \lor b \to c)\)
    \item \((a \to b) \to (a \to \neg b) \to \neg a\)
    \item \(\neg \neg a \to a\)
\end{enumerate}

\begin{remark}
    8 аксиома не совпадает с тем, что было на лекции. Возможно, я набагал при переписывании. % TODO #63
\end{remark}

\begin{remark}
    \(a \to b \to c \Leftrightarrow a \to (b \to c)\)
\end{remark}

Можно заметить, что все эти аксиомы --- тавтологии. Это так и должно быть. Вместо \(a, b, c\) можно подставлять произвольные формулы.

\begin{definition}
    \textbf{Вывод} формулы \(\varphi\) есть последовательность формул \(\varphi_1, \dots , \varphi_n = \varphi\), где \(\varphi_i\) --- одна из аксиом или Modus Ponens из \(\varphi_a, \varphi_b\), где \(a, b < i\)
\end{definition}

\begin{definition}
    \textbf{Modus Ponens \textit{(правило вывода)}} из формул \(a\) и \(a \to b\) есть формула \(b\).
\end{definition}

Корректность: можно вывести любую тавтологию.
\begin{proof}
    Очевидно по индукции и определению Modus Ponens.
\end{proof}

Полнота: любую тавтологию можно вывести.
\begin{proof}
    Будет на лекции.
\end{proof}

\begin{example}
    Выведем \(a \to a\)

    \[a: = x, b: = x \to x, c : = x\]
    \begin{align*}
        \varphi_1                   & = (x \to x \to x) \to (x \to (x \to x) \to x) \to (x \to x) \\
        \varphi_2                   & = x \to x \to x                                             \\
        \text{M.p.} \quad \varphi_3 & = (x \to (x \to x) \to x) \?                                \\
    \end{align*}
\end{example}

Пусть \(\Gamma\) --- какой-то набор формул.

\begin{definition}
    Вывод в грамматике \(\Gamma\) формулы \(\varphi\) обозначается \(\Gamma \vdash \varphi\), и его шаги могут иметь вид \(\tau\), где \(\tau\in\Gamma\) \textit{(шаги как в нормальном выводе также разрешены)}.
\end{definition}

\begin{lemma}[о дедукции]
    \(\Gamma \cup \{a\} \vdash \varphi \Rightarrow \Gamma \vdash a \to \varphi\)
\end{lemma}

\begin{exercise}
    Указать про каждое из высказываний, общезначимо, выполнимо, опровержимо или невыполнимо ли оно:
    \begin{enumerate}
        \item \(\neg A \lor \neg\neg A\)
              \begin{align*}
                  \varphi_1 & = \neg A \lor A           \\
                  \varphi_2 & = \neg A \lor \neg \neg A \\
              \end{align*}

              \textbf{Ответ}: общезначимо, выполнимо.
        \item \((A \to \neg B) \lor (B \to \neg C) \lor (C \to \neg A)\)

              При подстановке \((1, 1, 1)\) не выполнено, при \((0, 0, 0)\) выполнено.

              \textbf{Ответ}: выполнимо, опровержимо.

        \item \(((P \to Q) \to P) \to P\)

              При подстановке \((0, 1)\) не выполнено, при \((1, 1)\) выполнено.

              \textbf{Ответ}: выполнимо, опровержимо.

        \item \(\neg A \with \neg \neg A\)

              \(\neg A \with \neg \neg A = \neg A \with A\)

              \textbf{Ответ}: невыполнимо, опровержимо.

        \item \(\neg (A \with \neg A)\)

              Т.к. \(A\with \neg A\) невыполнимо, \(\neg (A\with \neg A)\) общезначимо.

              \textbf{Ответ}: общезначимо, выполнимо.

        \item \(A\)

              \textbf{Ответ}: выполнимо (\(A = 1\)), опровержимо (\(A = 0\)).

        \item \(A \to A\)

              \textbf{Ответ}: общезначимо, выполнимо \textit{(подстановкой)}.

        \item \(A \to \neg A\)

              \textbf{Ответ}: выполнимо (\(A = 0\)), опровержимо (\(A = 1\)).

        \item \((A \to B) \lor (B \to A)\)

              \textbf{Ответ}: общезначимо, выполнимо.
    \end{enumerate}

    Доказать:
    \begin{enumerate}
        \item \(\vdash A \to A\)

              Было до этого.

        \item \(\vdash (A \to A \to B) \to (A \to B)\)

              \begin{align*}
                  \varphi_1 & = a \to a                       \\
                  \varphi_2 & = (a \to a \to b) \to (a \to b)
              \end{align*}

        \item \(\vdash \neg (A \with \neg A)\)

              \begin{align*}
                  \varphi_1 & = (\phi \to \psi) \to (\phi \to \neg \psi) \to \neg \phi                           \\
                  \varphi_2 & = A \with \neg A \to A                                                             \\
                  \varphi_3 & = A \with \neg A \to \neg A                                                        \\
                  \varphi_4 & = (A \with \neg A \to A) \to (A \with \neg A \to \neg A) \to \neg (A \with \neg A) \\
              \end{align*}

              % \item \(\vdash A \with B \to B \with A\)

              %       \begin{align*}
              %           \varphi_1 & = \\
              %       \end{align*}

              % \item \(A \to \neg \neg A\)

              %       \begin{align*}
              %           \varphi_1 & = \\
              %       \end{align*}

              % \item \(A \with \neg A \vdash B\)

              %       \begin{align*}
              %           \varphi_1 & = A \with \neg A \\
              %           \varphi_2 & =
              %       \end{align*}
    \end{enumerate}
\end{exercise}

\end{document}