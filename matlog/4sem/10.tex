\chapter{30 апреля}

\section{Арифметизация математики}

Это идея того, все содержательное в математике может быть выражено как арифметика. Мы к ней подойдём издалека

\subsection{Рекурсивные функции}

Рассмотрим следующие примитивы, чтобы определить рекурсивные функции:
\begin{enumerate}
    \item \(Z : \N \to \N : Z(x) = 0\)
    \item \(N : \N \to \N : N(x) = x + 1\)
    \item \(S_k : \N^m \to \N\) --- подстановка
          \[S_k \langle g, f_1 \dots f_k \rangle(x_1 \dots x_m) = g(f_1(\overline{x}), f_2(\overline{x}) \dots f_k(\overline x))\], где \(\overline x \equiv x_1 \dots x_m\) и если \(f_1 \dots f_k : \N^m \to \N\) и \(g : \N^k \to \N\)
    \item \(P_k^l : \N^k \to \N : P_k^l(x_1 \dots x_k) = x_l\) при \(l \leq k\) --- проекция
    \item \(R\langle f, g \rangle : \N^{m + 1} \to \N\), если \(f : \N^m \to \N, g : \N^{m + 2} \to \N\)
          \[R \langle f, g \rangle (y, x_1 \dots x_m) = \begin{cases}
                  f(x_1 \dots x_m),                                               & y = 0 \\
                  g(y - 1, R \langle f, g \rangle(x_1 \dots x_m), x_1 \dots x_m), & y > 0
              \end{cases} \]

          \(R\) называется \textbf{примитивной рекурсией}.

          \(R\) можно воспринимать как цикл \texttt{for} с переменной цикла \(y\).

          \begin{example}\itemfix
              \begin{enumerate}
                  \item \(R\ev{f,g} x = f(x)\)
                  \item \(R\ev{f,g} x = g(0, f(x), x)\)
                  \item \(R\ev{f,g} x = g(1, g(0, f(x), x), x)\)
              \end{enumerate}
          \end{example}
\end{enumerate}

\begin{definition}
    \(f : \N^m \to \N\) --- \textbf{примитивно-рекурсивная}, если найдётся выражение \(f \) через примитивы \(Z, N, S, P, R\).
\end{definition}

\begin{example}\itemfix
    \begin{enumerate}
        \item \(1(x) = 1\)
              \[1 = S\ev{N, Z}\]
        \item \(( + 2)(x) = x + 2\)
              \[( + 2) = S\ev{N, N}\]
              \[S\ev{N, N}(x) = g(f(x)) = N(N(x)) = x + 2\]
        \item \[( + 3) = S\ev{N, S\ev{N, N}}\]
        \item \(( \times 2)\)

              Промежуточная функция:
              \[( \times 2_a) = R\ev{P_1^1, S\ev{N, P_3^2}}\]
              \[( \times 2) = S\ev{( \times 2_a), P_1^1, P_1^1}\]
    \end{enumerate}
\end{example}

Добавим новый примитив ``минимизация'':
\begin{enumerate}
    \setcounter{enumi}{5}
    \item \(M\ev{f} : \N^m \to N\) при \(f : \N^{m + 1} \to \N\)

          \(M\ev{f}(x_1 \dots x_m) = y\) --- минимальный \(y\) такой, что \(f(y, x_1 \dots x_m) = 0\). Если \(f(y, x_1 \dots x_m) > 0\) при всех \(y\), результат неопределён.
\end{enumerate}

\begin{theorem}
    \(( +), (\cdot), (x^y), (:), (\sqrt{})\), деление с остатком, числа Фибоначчи --- примитивно-рекурсивные функции.
\end{theorem}

Пусть \(p_1, p_2 \dots \) --- простые числа.

\begin{statement}
    \(p(i) : \N \to \N, p(i) = p_i\) --- примитивно-рекурсивная функция.
\end{statement}

\(2^{k_1} \cdot 3^{k_2} \cdot 5^{k_3} \cdot \ldots \cdot p_i^{k_i}\) --- примитивно-рекурсивно

\[\text{plog}_n k = \max t : k \equiv 0 \mod n^t\]
\begin{example}\itemfix
    \begin{enumerate}
        \item \(\text{plog}_5 120 = 1\)
        \item \(\text{plog}_2 120 = 3\)
    \end{enumerate}
\end{example}

\(\text{plog}_k p\) --- примитивно-рекурсивная функция.

Тогда мы можем кодировать \(\ev{k_1 \dots k_n}\) как \(2^{k_1} \cdot 3^{k_2} \cdot 5^{k_3} \cdot \ldots \cdot p_i^{k_i}\) и перевод в любую сторону примитивно-рекурсивен. С помощью такого подхода проще создавать примитивно-рекурсивные функции.

\begin{definition}[Функция Аккермана]
    \[A(m, n) = \begin{cases}
            n + 1,                 & m = 0        \\
            A(m - 1, 1),           & m > 0, n = 0 \\
            A(m - 1, A(m, n - 1)), & n > 0, n > 0
        \end{cases}\]
\end{definition}
\begin{statement}
    \(A(m, n)\) не примитивно-рекурсивно.
\end{statement}
\begin{proof}
    Общая идея: если некоторый текст длины \(n\) задал число \(k\), то добавление одного символа не позволяет получить число больше, чем \(k^k\), т.к \(R\) не может совершить больше \(R\) итераций.
\end{proof}

\begin{theorem}
    \(f\) --- рекурсивная функция. Тогда \(f\) представима в формальной арифметике.
    \label{рекурсивная -> ФА}
\end{theorem}
\begin{theorem}
    \(f\) представима в формальной арифметике. Тогда \(f\) рекурсивна.
\end{theorem}

\begin{proof}
    Пусть \(\vdash \varphi\) и \(\delta_1 \dots \delta_n \equiv \varphi\) --- доказательство \(\varphi\) в формальной арифметике.

    Пусть \(C\) --- рекурсивная функция, проверяющая доказательство в формальной арифметике, т.е. \(C(p, x) = \begin{cases} 0, & \text{доказательство корректно} \\ \neq 0, & \text{доказательство некорректно} \end{cases} \), где \(x\) --- запись доказательства формулы \(p\).

    По теореме \ref{рекурсивная -> ФА} получим формулу \(\sigma\), для которой верно \(\vdash \sigma(p, x, 0)\), если \(p\) --- доказательство формулы \(x\).
\end{proof}

\subsection{Проблема останова}

Пусть есть программа \(P(p, x) = \begin{cases}
    0, & \text{если } p(x) \text{ останавливается}    \\
    1, & \text{если } p(x) \text{ не останавливается}
\end{cases}\)

Рассмотрим программу \texttt{Q}:
\begin{minted}{c}
Q(p)
    if P(p) = 1
        return 0
    else
        while true do;
\end{minted}

Чему равно \(Q(Q)\)? Ни 0, ни 1. Это противоречие.

Мы аналогичным образом сломаем наше доказательство --- создадим формулу ``для меня нет доказательства''.

\setcounter{theorem}{24}
\begin{subtheorem}{theorem}
    \begin{theorem}
        Примитивы \(Z, N, S, P\) представимы в формальной арифметике.
    \end{theorem}
    \begin{proof}\itemfix
        \begin{enumerate}
            \item \(Z : \xi : = x_1 = x_1 \with x_2 = 0\)
            \item \(N : \nu : = x_2 = x_1'\)
            \item \(P_k^l : \pi_k^l : = x_1 = x_1 \with x_2 = x_2 \with \dots \with x_l = x_{k+1} \with \dots \with x_k = x_k\)
            \item \(S\ev{g, f_1 \dots f_k} : g \leftrightarrow \gamma, f_i \leftrightarrow \varphi_i\).
                  \[\exists r_1.\exists r_2.\dots \exists r_k.\varphi_1(x_1 \dots x_m, r_1) \with \varphi_2(x_1\dots x_m, r_2) \with \dots \with \varphi_k(x_1 \dots x_m, r_k) \with \gamma(r_1 \dots r_k, x_{m+1})\]
                  \setcounter{enumi}{5}
            \item \(M\ev{f}\)
                  \[\varphi(x_{m+1}, x_1 \dots x_m, \overline 0) \with \forall y.y < x_{m+1} \to \neg \varphi(y, x_1 \dots x_m, \overline 0)\]
                  \setcounter{enumi}{4}
            \item \(\beta\)-функция Гёделя:
                  \[\beta(b, c, i) = b \% (1 + c \cdot (i + 1))\]
                  \begin{theorem}
                      \(a_0 \dots a_n\) --- некоторые значения \(\in \N\). Тогда найдутся \(b \) \(c\) такие, что \(\beta(b, c, i) = a_i\)
                  \end{theorem}
                  \begin{proof}
                      \begin{statement}
                          Если \(i \neq j\), то \(1 + c(i + 1)\) взаимно просто с \(1 + c(j + 1)\).
                      \end{statement}

                      \begin{proof}
                          \(c : = \max(a_0 \dots a_n, n)!\).

                          Пусть есть некоторый простой \(p\): \(1 + c(i + 1) \equiv 0 \mod p\) и \(1 + c(j + 1) \equiv 0 \mod p\). Тогда \(c(i + 1 - j - 1) \equiv 0 \mod p\) и \(c(i - j) \equiv 0 \mod p\) --- противоречие.
                      \end{proof}

                      \begin{statement}
                          По китайской теореме об остатках найдётся \(b\) с нужными свойствами.
                      \end{statement}
                  \end{proof}

                  \(\beta\) примитивно-рекурсивна и представима в формальной арифметике:
                  \[B(b, c, i, q) = (\exists p.b = p\cdot(1 + c \cdot (1 + i)) + q) \with q < b\]

                  Тогда для \(R\ev{f, g}\), если \(f \leftrightarrow \varphi, g \leftrightarrow \gamma\):
                  \begin{align*}
                      \exists b.\exists c.\exists f.\varphi(x_1 \dots x_n, f) & \with B(b, c, \overline 0, f) \with \forall y.y < x_{n+1} \to \exists r_{y - 1}.B(b, c, y - 1, r_{y - 1}) \\
                                                                              & \with \exists r_{y+1}.B(b,c,y + 1,r_{y+1}) \with \varphi(y, r_y, x_1 \dots x_n, r_{y+1})                  \\
                                                                              & \with B(b, c, x_{n+1}, x_{n + 2})
                  \end{align*}
        \end{enumerate}
    \end{proof}
\end{subtheorem}
\setcounter{theorem}{26}