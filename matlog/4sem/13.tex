\chapter{21 мая}

\subsection{Аксиома выбора}

\begin{axiom}
    Эквивалентны следующие формулировки:
    \begin{itemize}
        \item На любом семействе\footnote{Это синоним слову ``множество''.} непустых множеств \(\{A_S\}_{S \in \mathbb{S}}\) можно определить функцию \(f : \mathbb{S} \to \bigcup_S A_S\), которая мо множеству возвращает его элемент.
        \item Любое множество можно вполне упорядочить.
        \item Для любой сюрьективной функции \(f : A \to B\) найдётся частично обратная \(g : B \to A\), т.е. \(g(f(x)) = x\).
    \end{itemize}
\end{axiom}

\begin{remark}
    Эта аксиома странная, т.к. по третьей формулировке любую хеш-функцию можно сломать. Конечно, они все ломаются перебором, но это не относится к реальному миру.
\end{remark}

\begin{remark}
    Эта аксиома не конструктивна --- сказано, что можно построить функцию/порядок, но не сказано, как.
\end{remark}

\begin{remark}
    Аксиома выбора не даёт парадоксов.
\end{remark}

\begin{remark}
    Можно рассматривать теорию множеств без этой аксиомы, она тоже часто используется и обозначается \textbf{ZF}\footnote{Zermelo–Fraenkel}, а с аксиомой выбора обозначается \textbf{ZFC}\footnote{Zermelo–Fraenkel-Choice}.
\end{remark}

\begin{definition}
    \textbf{Дизъюнктное семейство множеств} --- семейство непересекающихся подмножеств.\footnote{Кажется, в формализации ошибка, т.к. если \(p = q\), то всё ломается. Нужно в конец добавить \({}\lor p = q\).}
    \[D(y) : \forall p.\forall q.p \in y \with q \in y \to p \cap q = \emptyset\]
\end{definition}

\begin{definition}[прямое произведение дизъюнктного множества]
    \[\bigtimes S = \{t\ |\ \forall p.p \in S \leftrightarrow \exists! c.c \in p \with c \in t\}\]
\end{definition}

Формулировка аксиомы выбора, которую мы будем использовать:

\begin{axiom*}[выбора]
    Если \(D(y) \with \forall t.t \in y \to t \neq \emptyset\), то \(\bigtimes y \neq \emptyset\)
\end{axiom*}

\begin{remark}
    В матанализе аксиома выбора используется для эквивалентности предела по Коши и по Гейне.
\end{remark}

\begin{theorem}[Диаконеску]
    Рассмотрим ZF поверх ИИП\footnote{а не КИП}. Если добавить аксиому выбора, то \(\vdash \alpha \lor \neg \alpha\).
\end{theorem}

\begin{axiom}[фундирования]
    \label{фундирования}
    \[\forall x.x = \emptyset \lor \exists y.y \in x \with y \cap x = \emptyset\]
    Иными словами, в каждом непустом множестве есть элемент, не пересекающийся с ним.
\end{axiom}

\begin{remark}
    Эта аксиома запрещает самосодержащие множества.
\end{remark}

\begin{remark}
    Без аксиомы \nameref{фундирования} нельзя определить \(\{a, \{a, b\}\}\) как пару, но можно \(\{\{a\}, \{a, b\}\}\).
\end{remark}

\begin{axiom}[схема подстановки, Френкеля]
    Если \(S\) --- множество, \(f\) --- функция, т.е. существует формула \(\varphi(x, y) : \forall x \in S.\exists!y.\varphi(x, y)\), то \(f(S)\) --- множество.
\end{axiom}

\subsection{Мощность множеств}

\begin{definition}
    Множества \(a\) и \(b\) \textbf{равномощны}, если существует биекция \(a \to b\) и обозначается \(|a| = |b|\).
\end{definition}

\begin{definition}
    \textbf{Кардинальное число \(t\)} --- ординал \(x\), такой что для всех \(y \in x\) \(|y| \neq |x|\)
\end{definition}

\begin{definition}
    \textbf{Мощность} \(|x|\) --- такое кардинальное число \(t\), что \(|t| = |x|\).
\end{definition}

\begin{definition}[строго большая мощность]
    \(|a| < |b|\), если существует \(f : a \to b\) --- инъективно, но нет биекции.
\end{definition}

\begin{statement}
    Если \(a, b\) --- кардиналы и \(|a| = |b|\), то \(a = b\).
\end{statement}

\begin{itemize}
    \item \(\overline 0, \overline 1, \dots \) --- конечные кардиналы.
    \item \(\aleph_0 = |\omega|\)
    \item \(\aleph_1\) --- следующий кардинал за \(\aleph_0\).
    \item \(\vdots\)
\end{itemize}

\begin{example}
    \(|\omega| = |\omega + 1|\), следовательно \(|\omega + 1|\) не кардинал, т.к. \(\omega \in \omega + 1\).
\end{example}

\begin{theorem}[Кантора]
    Рассмотрим множество \(S\) и \(\mathcal{P}(S)\). Тогда \(|\mathcal{P}(S)| > |S|\)
\end{theorem}
\begin{proof}
    Пусть \(f : S \to \mathcal{P}(S)\) --- биекция. Построим \(x \in \mathcal{P}(S)\), не имеющий прообраза. Это можно сделать диагональным методом: \(t = \{s_k \in S\ |\ s_k \notin f(s_k)\}\).
\end{proof}

Напоминание: \(\aleph_1\) --- наименьший кардинал такой, что \(\aleph_1 > \aleph_0\). Существует ли он? Да, т.к. \(|\mathcal{P}(\aleph_0)| > \aleph_0\) по теореме кантора.

Является ли \(\aleph_1\) \(\mathcal{P}(\aleph_0)\)? Это континуум-гипотеза, и её отрицание нельзя доказать.

\begin{theorem}[Кантора-Бернштейна]
    Если \(a, b\) --- множества, \(f : a \to b\) и \(g : b \to a\) инъективны, то существует биекция \(a \to b\).
\end{theorem}
