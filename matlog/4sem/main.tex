\documentclass[12pt, a4paper, oneside]{book}

\usepackage{catchfilebetweentags}
\ExecuteMetaData[../../preamble.sty]{preamble}

\addto\captionsrussian{\renewcommand{\chaptername}{Лекция}}

\let\MakeUppercase\relax

\lfoot{}
\cfoot{}
\rfoot{}
\lhead{\leftmark}

\usepackage{tocloft}
\advance\cftsecnumwidth -1em\relax
\advance\cftsubsecindent -1em\relax
\advance\cftsubsecnumwidth -1em\relax

\usepackage{titletoc}
\titlecontents*{chapter}% <section-type>
  [0pt]% <left>
  {}% <above-code>
  {\bfseries\chaptername\ \thecontentslabel\quad}% <numbered-entry-format>
  {}% <numberless-entry-format>
  {\bfseries\hfill\contentspage}% <filler-page-format>

\let\endtitlepage\relax

\patchcmd{\chapter}{\thispagestyle{plain}}{\thispagestyle{fancy}}{}{}

\usepackage{remreset}
\makeatletter
  \@removefromreset{section}{chapter}
\makeatother

\renewcommand{\thesection}{\arabic{section}.}
\renewcommand{\thesubsection}{\thesection\arabic{subsection}.}

\begin{document}

\title{Математическая логика}
\author{Михайлов Максим}

\maketitle

\tableofcontents

\chapter{12 февраля}

\setcounter{section}{-1}

\section{Мотивация}

\subsection{Математикам}

\begin{axiom}[Архимеда]
    Для любого \(k > 0\) найдётся \(n\), такое что \(kn > 1\).
\end{axiom}

Под эту аксиому не подходят бесконечно малые числа и это является проблемой. Например, \(\lim\limits_{x \to +\infty} \frac{1}{x} = 0 = \lim\limits_{x \to +\infty} \frac{1}{x^2}\), но мы хотим уметь различать эти два числа. Ньютон предложил идею бесконечно малых чисел, откуда пошли последовательности. Возникает вопрос --- что такое последовательность и что такое число?

Общепринятое определение целых чисел \(\N\) происходит из теории множеств. Однако эта теория содержит в себе множество фундаментальных парадоксов, от которых нельзя избавиться.

Возникает вопрос --- а что такое множество? Посмотрим на некоторое множество \(A = \{x\ |\ x\not\in x\}\). Содержит ли оно себя, \(A\in A\)? На этот вопрос нельзя ответить, это называется парадокс Рассела. Есть простой способ его разрешить --- запретить ставить такой вопрос. Нет вопроса --- нет парадокса. Существование такого парадокса ставит под вопрос существование любого множества --- а существует ли \(\N\)? Может быть его существование парадоксально, просто мы не нашли этот парадокс. Пришло чуть более умное решение парадокса --- запретим множества, содержащие себя. Таким образом вывели аксиоматику теории множеств \textit{(Цермело --- Френкеля)}.

\begin{example}
    Рассмотрим множество всех чисел, которые можно задать в \(\leq 1000\) слов русского языка. Фраза ``наименьшее число, которое нельзя задать в \( \leq 1000\) слов'' содержит \( \leq 1000\) слов, т.е. такое число принадлежит искомому множеству --- парадокс.
\end{example}

Возникает идея --- человеческий язык порождает парадоксы, поэтому нужно задать новый язык, который их не порождает. Этот язык и является математической логикой.

\subsection{Программистам}

Математическая логика применяется в двух областях \textit{(для программистов)}:
\begin{enumerate}
    \item Языки программирования
    \item Формальные доказательства
\end{enumerate}

Для языков программирования матлогика применима как теория типов \textit{(переменных)}.

Формальные доказательства нужны например для smart-контрактов, где корректность программы критически важна, т.к. если в нём есть ошибка, у вас злоумышленник заберет все деньги, а вы не сможете этот контракт откатить.

\section{Исчисление высказываний}

\subsection{Язык}

\begin{definition}\itemfix
    \textbf{Язык} содержит в себе:
    \begin{enumerate}
        \item \textbf{Пропозициональные переменные}

              \(A_i'\) --- большая буква начала латинского алфавита, возможно с индексом и/или штрихом.

        \item \textbf{Связки}

              Пусть \(\alpha, \beta\) --- высказывания. Тогда \((\alpha \to \beta), (\alpha \with \beta), (\alpha \lor \beta), (\neg \alpha)\) --- высказывания.

              \(\alpha,\beta\) называются \textbf{метапеременными}.
    \end{enumerate}
\end{definition}

\begin{remark}
    Математическая логика алгеброподобна \textit{(а не анализоподобна)}, т.к. в ней много определений и мало доказательств.
\end{remark}

\subsection{Метаязык и предметный язык}

У нас есть два различных языка --- \textbf{предметный язык} и \textbf{метаязык}. Метаязык --- русский, предметный язык мы определили выше.

\begin{example}
    \(\alpha \to \beta\) --- метавыражение; \(A \to (A \to A)\) --- предметное выражение.
\end{example}

\begin{obozn}
    Метапеременные обозначаются различными способами в зависимости от того, что они обозначают:
    \begin{itemize}
        \item Буквы греческого алфавита (\(\alpha, \beta, \gamma, \dots, \varphi, \psi\)) --- выражения
        \item Заглавные буквы конца латинского алфавита (\(X, Y, Z\)) --- произвольные переменные
    \end{itemize}
\end{obozn}

\begin{example}
    \(X \to Y \Rightarrow A \to B\) --- подстановка переменных. Этот синтаксис не формален, мы будем записывать так:
    \[(X \to Y)[X : = A, Y : = B]\equiv A \to B\]
\end{example}

\textit{Соглашение}. символы логических операций не пишутся в метаязыке.

\begin{example}
    \begin{align*}
        (\alpha \to (A \to X))[\alpha : = A, X : = B]         & \equiv A \to (A \to B)         \\
        (\alpha \to (A \to X))[\alpha : = (A \to P), X : = B] & \equiv (A \to P) \to (A \to B) \\
    \end{align*}
\end{example}

\subsection{Сокращения записи}

\begin{itemize}
    \item \(\lor, \with, \neg\) --- скобки слева направо \textit{(лево-ассоциативные операции)} \textit{(не коммутативные)}
    \item \( \to \) --- правоассоциативная.
\end{itemize}

\begin{remark}
    Здесь операторы записаны в порядке их приоритета
\end{remark}

\begin{example}
    Расставим скобки в следующем выражении:
    \[A \to B \with C \to D\]
    \[A \to ((B \with C) \to D)\]
\end{example}

\subsection{Теория моделей}

\textbf{Модель} состоит из:
\begin{obozn}\itemfix
    \begin{itemize}
        \item \(P\) --- некоторое множество предметных переменных
        \item \(\tau\) --- множество высказываний предметного языка
        \item \(V\) --- множество истинных значений. Классическое --- \(\{\text{П}, \text{Л}\}\)
        \item \(\llbracket\ \rrbracket : \tau \to V\) --- оценка высказывания \textit{(высказывание ставится в скобки)}.
    \end{itemize}
\end{obozn}

\begin{enumerate}
    \item \(\llbracket x \rrbracket : P \to V\) --- задается при оценке. % TODO
    \item \(\llbracket \alpha \star \beta \rrbracket = \llbracket \alpha \rrbracket \textcolor{blue}{\star} \llbracket \beta \rrbracket\), где \(\star\) есть логическая операция ( \(\lor, \with, \neg, \to\)), а \(\textcolor{blue}{\star}\) определено естественным образом как элемент метаязыка.
\end{enumerate}

\subsection{Теория доказательств}

\begin{definition}
    \textbf{Схема высказывания} --- строка, соответствующая определению высказывания + метапеременные.
\end{definition}

\begin{example}
    \[(\alpha \to (\beta \to (A \to \alpha)))\]
\end{example}

10 схем аксиом:
\begin{enumerate}
    \item \(\alpha \to \beta \to \alpha\)
    \item \((\alpha \to \beta) \to (\alpha \to \beta \to \gamma) \to (\alpha \to \gamma)\)
    \item \(\alpha \to \beta \to \alpha \with \beta\)
    \item \(\alpha \with \beta \to \alpha\)
    \item \(\alpha \with \beta \to \beta\)
    \item \(\alpha \to \alpha \lor \beta\)
    \item \(\beta \to \alpha \lor \beta\)
    \item \((\alpha \to \gamma) \to (\beta \to \gamma) \to (\alpha \lor \beta \to \gamma)\)
    \item \((\alpha \to \beta) \to (\alpha \to \neg \beta) \to \neg \alpha\)
    \item \(\neg \neg \alpha \to \alpha\)
\end{enumerate}

\subsection{Правило Modus Ponens и доказательство}

\begin{definition}
    \textbf{Доказательство \textit{(вывод)}} есть конечная последовательность высказываний \(\alpha_1 \dots \alpha_n\), где \(\alpha_i\) --- либо аксиома, либо \(\exists k, l < i : \alpha_k \equiv \alpha_l \to \alpha_i\) \textit{(правило Modus Ponens)}
\end{definition}

\begin{example}
    \(\vdash A \to A\)

    \begin{tabular}{lll}
        1. & \(A \to A \to A\)                                               & сх. акс. 1 \\
        2. & \(A \to (A \to A) \to A\)                                       & сх. акс. 1 \\
        3. & \((A \to (A \to A)) \to (A \to (A \to A) \to A) \to (A \to A)\) & сх. акс. 2 \\
        4. & \((A \to (A \to A) \to A) \to (A \to A)\)                       & M.P. 1, 3  \\
        5. & \(A \to A\)                                                     & M.P. 2, 4
    \end{tabular}
\end{example}

\begin{definition}
    Доказательство \(\alpha_1\dots \alpha_n\) доказывает выражение \(\beta\), если \(\alpha_n \equiv \beta\)
\end{definition}

\chapter{19 февраля}

\begin{obozn}
    Большая греческая буква середины греческого алфавита ( \(\Gamma, \Delta, \Sigma\)) --- список высказываний.
\end{obozn}

\begin{definition}[следование]
    \(\alpha\) следует из \(\Gamma\) \textit{(обозначается \(\Gamma\models\alpha\))}, если \(\Gamma = \gamma_1 \dots \gamma_n\) и всегда, когда все \(\llbracket \gamma_i \rrbracket = \text{ И}\), то \(\llbracket \alpha \rrbracket = \text{ И}\).
\end{definition}

\begin{example}
    \(\models \alpha\) --- \(\alpha\) общезначимо.
\end{example}

\begin{definition}
    \sout{Теория} Исчисление высказываний \textbf{корректно}, если при любом \(\alpha\) из \(\vdash \alpha\) следует \(\models \alpha\).
\end{definition}

\begin{definition}
    Исчисление \textbf{полно}, если при любом \(\alpha\) из \(\models \alpha\) следует \(\vdash \alpha\).
\end{definition}

\begin{theorem}[о дедукции]
    \[\Gamma, \alpha \vdash \beta \Leftrightarrow \Gamma \vdash \alpha \to \beta\]
\end{theorem}

\begin{proof}\itemfix
    \begin{itemize}
        \item [ \( \Leftarrow \)] Пусть \(\Gamma \vdash \alpha \to \beta\), т.е. существует доказательство \(\delta_1 \dots \delta_n\), где \(\delta_n \equiv \alpha \to \beta\)

              Построим новое доказательство: \(\delta_1 \dots \delta_n, \alpha \text{ (гипотеза) }, \beta \text{ (M.P.)}\). Эта новая последовательность --- доказательство \(\Gamma, \alpha \vdash \beta\)

        \item [ \( \Rightarrow \)] Рассмотрим \(\delta_1 \dots \delta_n, \Gamma, \alpha \vdash \beta\). Рассмотрим последовательность \(\sigma_1 = \alpha \to \delta_1 \dots \sigma_n = \alpha \to \delta_n\). Это не доказательство.

              Но эту последовательность можно дополнить до доказательства, так что каждый \(\sigma_i\) есть аксиома, гипотеза или получается через M.P. Докажем это.

              \begin{proof}
                  \textbf{База}: \(n = 0\) --- очевидно.

                  \textbf{Переход}: пусть \(\sigma_0 \dots \sigma_n\) --- доказательство. Покажем, что между \(\sigma_n\) и \(\sigma_{n + 1}\) можно добавить формулы так, что \(\sigma_{n + 1}\) будет доказуемо.

                  У нас есть 3 варианта обоснования \(\delta_{n + 1}\)
                  \begin{enumerate}
                      \item \(\delta_{n + 1}\) --- аксиома или гипотеза, \(\not\equiv \alpha\)

                            Будем нумеровать дробными числами, потому что нам ничто это не запрещает, т.к. нам нужна только упорядоченность.

                            \(\textcolor{blue}{n + 0.2} \quad \delta_{n + 1}\) --- верно, т.к. это аксиома или гипотеза

                            \(\textcolor{blue}{n + 0.4} \quad \delta_{n + 1} \to \alpha \to \delta_{n + 1}\) \textcolor{blue}{(аксиома 1)}

                            \(\textcolor{blue}{n + 1} \quad \alpha \to \delta_{n + 1}\) \textcolor{blue}{(M.P. \(n + 0.2, n + 0.4\))}

                      \item \(\delta_{n + 1} \equiv \alpha\)

                            \(\textcolor{blue}{n + 0.2, 0.4, 0.6, 0.8, 1}\) --- доказательство \(\alpha \to \alpha\)

                      \item \(\delta_k \equiv \delta_l \to \delta_{n + 1},\ k, l \leq n\)

                            \(\textcolor{blue}{k} \quad \alpha \to (\delta_l \to \delta_{n + 1})\)

                            \(\textcolor{blue}{l} \quad  \alpha \to \sigma_l\)

                            \(\textcolor{blue}{n + 0.2} \quad (\alpha \to \sigma_l) \to (\alpha \to (\sigma_l \to \sigma_{n + 1})) \to (\alpha \to \sigma_{n + 1})\) \textcolor{blue}{(аксиома 2)}

                            \(\textcolor{blue}{n + 0.4} \quad (\alpha \to \sigma_l \to \sigma_{n + 1}) \to (\sigma \to \sigma_{n + 1})\) \textcolor{blue}{(M.P. \(n + 2, l\))}

                            \(\textcolor{blue}{n + 1} \quad \alpha \to \sigma_{n + 1}\) \textcolor{blue}{(M.P. \(n + 0.4, k\))}
                  \end{enumerate}
              \end{proof}
    \end{itemize}
\end{proof}

\begin{theorem}
    Пусть \(\vdash \alpha\). Тогда \(\models \alpha\).
\end{theorem}
\begin{proof}
    Индукция по длине доказательства: каждая \(\llbracket \delta_i \rrbracket = \text{ И}\), если \(\delta_1 \dots \delta_n\) --- доказательство \(\alpha\)

    Рассмотрим \(n\) и пусть \(\llbracket \delta_1 \rrbracket = \text{ И}, \dots \llbracket \delta_n \rrbracket = \text{ И}\).

    Тогда рассмотрим основание \(\delta_{n + 1}\)
    \begin{enumerate}
        \item \(\delta_{n + 1}\) --- аксиома. Это упражнение.

              \begin{example}
                  \(\delta_{n + 1} \equiv \alpha \to \beta \to \alpha\)

                  \[\sphericalangle \llbracket \alpha \to \beta \to \alpha \rrbracket^{\llbracket \alpha \rrbracket : = a, \llbracket \beta \rrbracket : = b} = \text{И}\]

                  \begin{center}
                      \begin{tabular}{c|c|c|c}
                          \(a\) & \(b\) & \(\beta \to \alpha\) & \(\alpha \to \beta \to \alpha\) \\ \hline
                          Л     & Л     & И                    & И                               \\
                          Л     & И     & Л                    & И                               \\
                          И     & Л     & И                    & И                               \\
                          И     & И     & И                    & И                               \\
                      \end{tabular}
                  \end{center}
              \end{example}

              Аналогично можно доказать для остальных аксиом.

        \item \(\delta_{n + 1}\) --- M.P. \(\delta_k = \delta_l \to \delta_{n + 1}\)

              Фиксируем оценку. Тогда \(\llbracket \delta_k \rrbracket = \llbracket \delta_l \rrbracket =\) И. Тогда:

              \begin{center}
                  \begin{tabular}{c|c|c}
                      \(\llbracket \delta_k \rrbracket\) & \(\llbracket \delta_{n + 1} \rrbracket \) & \(\llbracket \delta_k \rrbracket = \llbracket \delta_l \to \delta_{n + 1} \rrbracket\) \\ \hline
                      Л                                  & Л                                         & И                                                                                      \\
                      Л                                  & И                                         & И                                                                                      \\
                      И                                  & Л                                         & Л                                                                                      \\
                      И                                  & И                                         & И                                                                                      \\
                  \end{tabular}
              \end{center}

              Первых трёх вариантов не может быть в силу \(\llbracket \delta_k \rrbracket = \llbracket \delta_l \rrbracket =\) И. Таким образом, \(\llbracket \delta_{n + 1} \rrbracket = \) И.
    \end{enumerate}
\end{proof}

\begin{theorem}[о полноте]
    Пусть \(\models \alpha\). Тогда \(\vdash \alpha\).
\end{theorem}

Фиксируем набор переменных из \(\alpha\): \(P_1 \dots P_n\).

Рассмотрим \(\llbracket \alpha \rrbracket^{P_1 : = x_1 \dots P_n: = x_n} = \text{И}\)

\begin{obozn}
    \({}_{[\beta]} \alpha \equiv \begin{cases}
        \alpha,      & \llbracket \beta \rrbracket = \text{И} \\
        \neg \alpha, & \llbracket \beta \rrbracket = \text{Л}
    \end{cases}\) и \({}_{[x]} \alpha \equiv \begin{cases}
        \alpha,      & x = \text{И} \\
        \neg \alpha, & x = \text{Л}
    \end{cases}\)
\end{obozn}

Докажем, что \(\underbrace{{}_{[x_1]} P_1, \dots {}_{[x_n]} P_n}_{\Pi} \vdash {}_{[\alpha]} \alpha\)

\begin{proof}
    По индукции по длине формулы:

    \textbf{База}: \(\alpha = P_i\) \({}_{[P_i]} P_i \vdash {}_{[P_i]} P_i\), значит \(\Pi \vdash {}_{[P_i]} P_i\)

    \textbf{Переход}: пусть \(\eta, \zeta : \Pi \vdash {}_{[\eta]}\eta, \Pi \vdash {}_{[\zeta]}\zeta\) \textit{(по индукционному предположению)}. Покажем, что \(\Pi \vdash {}_{[\eta \star \zeta]} \eta \star \zeta\), где \(\star\) --- все связки

    Это упражнение.
\end{proof}

\begin{lemma}
    \(\Gamma, \eta \vdash \zeta, \Gamma, \neg \eta \vdash \zeta\). Тогда \(\Gamma \vdash \zeta\).
\end{lemma}
\begin{proof}
    Было в ДЗ.
\end{proof}

\begin{proof}[Доказательство теоремы о полноте]
    \(\models \alpha\), т.е. \({}_{[x_1]} P_1 \dots {}_{[x_n]} P_n \vdash {}_{[\alpha]} \alpha\). Но \(\llbracket \alpha \rrbracket = \text{П}\) при любой оценке. Тогда \({}_{[x_1]} P_1 \dots {}_{[x_n]} P_n \vdash \alpha\) при все \(x_i\).

    \begin{lemma}[об исключении допущения]
        Если \({}_{[x_1]} P_1 \dots {}_{[x_n]} P_n \vdash \alpha\) и \({}_{[x_1]} P_1 \dots {}_{[x_n]} \neg P_n \vdash \alpha\), то \({}_{[x_1]} P_1 \dots {}_{[x_{n - 1}]} P_{n - 1} \vdash \alpha\)
    \end{lemma}

    \[\begin{rcases*}
            {}_{[x_1]} P_1 \dots {}_{[x_{n - 1}]} P_{n - 1}, P_n \vdash \alpha \\
            {}_{[x_1]} P_1 \dots {}_{[x_{n - 1}]} P_{n - 1}, \neg P_n \vdash \alpha
        \end{rcases*} \xRightarrow{\text{по лемме}} {}_{[x_1]} P_1 \dots {}_{[x_{n - 1}]} P_{n - 1} \vdash \alpha\]
\end{proof}

% Докажем, что \(\exists a, b \in\R\setminus Q : a^b\in \Q\).
% \begin{proof}
%     Пусть \(a = b = \sqrt{2}\). Если \(\sqrt{2}^\sqrt{2} \not\in \R\setminus \Q\)
% \end{proof}

\section{Интуиционистская логика}

\subsection{ВНК-интерпретация}

Определим выражения:
\begin{itemize}
    \item \(\alpha \with \beta\) --- есть \(\alpha\) и \(\beta\)
    \item \(\alpha \lor \beta\) --- есть \(\alpha\) либо \(\beta\) и мы знаем, какое
    \item \(\alpha \to \beta\) --- есть способ перестроить \(\alpha\) в \(\beta\)
    \item \(\perp\) --- конструкция без построения \textit{(bottom)}
    \item \(\neg \alpha \equiv \alpha \to \perp\)
\end{itemize}

\textbf{Теория доказательств} есть классическая логика без десятой схемы аксиомы, вместо нее \(\alpha \to \neg \alpha \to \beta\)

\textbf{Теория моделей} --- теория, в которой \(\llbracket \alpha \rrbracket\) --- открытое множество в \(\Omega\) --- топологическом пространстве.

В ней определено следующее:
\begin{align*}
    \llbracket \alpha \with \beta \rrbracket & = \llbracket \alpha \rrbracket \cap \llbracket \beta \rrbracket                       \\
    \llbracket \alpha \lor \beta \rrbracket  & = \llbracket \alpha \rrbracket \cup \llbracket \beta \rrbracket                       \\
    \llbracket \alpha \to \beta \rrbracket   & = ((X \setminus \llbracket \alpha \rrbracket) \cup \llbracket \beta \rrbracket)^\circ \\
    \llbracket \perp \rrbracket              & = \emptyset                                                                           \\
    \llbracket \neg \alpha \rrbracket        & = (X \setminus \llbracket \alpha\rrbracket)^\circ
\end{align*}

\chapter{26 февраля}

\subsection{Естественный \textit{(натуральный)} вывод}

Рассмотрим новый способ записи доказательств --- в виде деревьев, называемый естественным выводом.

Тогда язык будет состоять из переменных \(A\dots Z, \lor, \with, \perp, \vdash, \text{---}\)

У нас используются следующие правила вывода:
\begin{enumerate}
    \item \begin{prooftree}
              \infer0[(аксиома)]{\Gamma \vdash \gamma, \gamma\in \Gamma}
          \end{prooftree}
    \item \begin{prooftree}
              \hypo{\Gamma, \varphi \vdash \psi}
              \infer1[(введение \( \to \))]{\Gamma \vdash \varphi \to \psi}
          \end{prooftree}
    \item \begin{prooftree}
              \hypo{\Gamma \vdash \varphi}
              \hypo{\Gamma \vdash \psi}
              \infer2[(введение \(\with\))]{\Gamma \vdash \varphi \with \psi}
          \end{prooftree}
    \item \begin{prooftree}
              \hypo{\Gamma \vdash \varphi \to \psi}
              \hypo{\Gamma \vdash \varphi}
              \infer2[(удаление \( \to \))]{\Gamma \vdash \psi}
          \end{prooftree}
    \item \begin{prooftree}
              \hypo{\Gamma \vdash \varphi \with \psi}
              \infer1[(удаление \(\with\))]{\Gamma \vdash \varphi}
          \end{prooftree}
    \item \begin{prooftree}
              \hypo{\Gamma \vdash \varphi \with \psi}
              \infer1[(удаление \(\with\))]{\Gamma \vdash \psi}
          \end{prooftree}
    \item \begin{prooftree}
              \hypo{\Gamma \vdash \varphi}
              \infer1[(введение \(\lor\))]{\Gamma \vdash \psi \lor \varphi}
          \end{prooftree}
    \item \begin{prooftree}
              \hypo{\Gamma \vdash \psi}
              \infer1[(введение \(\lor\))]{\Gamma \vdash \psi \lor \varphi}
          \end{prooftree}
    \item \begin{prooftree}
              \hypo{\Gamma \vdash \perp}
              \infer1[(удаление \(\perp\))]{\Gamma \vdash \varphi}
          \end{prooftree}
    \item \begin{prooftree}
              \hypo{\Gamma, \varphi \vdash \rho}
              \hypo{\Gamma, \psi \vdash \rho}
              \hypo{\Gamma \vdash \varphi \lor \psi}
              \infer3{\Gamma \vdash \rho}
          \end{prooftree}
\end{enumerate}

\begin{example}
    \begin{prooftree}
        \infer0[(акс.)]{A\vdash A}
        \infer1[(введение \(\with\))]{\vdash A \to A}
    \end{prooftree}
\end{example}

\begin{example}
    \begin{prooftree}
        \infer0[(акс.)]{A\with B\vdash A\with B}
        \infer1{A\with B \vdash B}
        \infer0[(акс.)]{A\with B\vdash A\with B}
        \infer1{A\with B \vdash A}
        \infer2{A\with B \vdash B\with A}
        \infer1[(введение \( \to \))]{\vdash A\with B \to B\with A}
    \end{prooftree}
\end{example}

\subsection{Теория решеток}

\begin{definition}\itemfix
    \begin{itemize}
        \item \textbf{Частичный порядок} --- рефлексивное, транзитивное, антисимметричное отношение.
        \item \textbf{Линейный порядок} --- сравнимы любые два элемента.
        \item \textbf{Наименьший элемент} \(S\) --- такой \(k\in S\), что если \(x\in S\), то \(k \leq x\)
        \item \textbf{Минимальный элемент} \(S\) --- такой \(k\in S\), что нет \(x\in S\), что \(x \leq k\)
        \item \textbf{Множество верхних граней} \(a\) и \(b\) : \(\{x\ |\ a \leq x \with b \leq x\}\).
        \item \textbf{Множество нижних граней} \(a\) и \(b\) : \(\{x\ |\ x \leq a \with x \leq b\}\).
        \item \(a + b\) --- наименьший элемент множества верхних граней \textit{(может не существовать)}.
        \item \(a \cdot b\) --- наибольший элемент множества нижних граней.
        \item \textbf{Решетка} --- множество + отношение, где для каждых \(a,b\) есть как \(a + b\), так и \(a \cdot b\).
        \item \textbf{Дистрибутивная решетка} --- если всегда \(a\cdot(b + c) = a\cdot b + a\cdot c\)
    \end{itemize}
\end{definition}

\begin{lemma}
    В дистрибутивной решетке \(a + b\cdot c = (a + b)(a + c)\)
\end{lemma}

\begin{definition}\itemfix
    \begin{itemize}
        \item \textbf{Псевдодполнение} \(a\) и \(b\) обозначается \(a \to b\) и равно наибольшему элементу множества \(\{c\ |\ a\cdot c \leq b\}\)
        \item \textbf{Импликативная решетка} --- решетка, где \(\forall a, b \ \ \exists a \to b\)
        \item \(0\) --- наименьший элемент решетки.
        \item \(1\) --- наибольший элемент решетки.
        \item \textbf{Псевдобулева алгебра \textit{(алгебра Гейтинга)}} --- импликативная решетка с нулём.
        \item \textbf{Булева алгебра} --- псевдобулева алгебра, такая что \(a + (a \to 0) = 1\)
    \end{itemize}
\end{definition}

\begin{example}
    $$\begin{tikzcd}[ampersand replacement=\&]
            1 \arrow{r} \arrow[swap]{d}{} \& b \arrow{d}{} \\
            a \arrow{r} \& 0
        \end{tikzcd}$$

    \begin{align*}
        a\cdot 0 & = 0 \\
        1\cdot b & = b \\
        a\cdot b & = 0 \\
        a + b    & = 1 \\
    \end{align*}
\end{example}

\begin{lemma}
    В импликативной решетке всегда есть 1.
\end{lemma}
\begin{proof}
    Возьмём \(a \to a = 1\) для некоторого \(a\).

    \[a \to a = \text{н} \{x\ |\ a\cdot x \leq a\} = \text{н}(A)\]
    Таким образом, \(A\) имеет наибольший элемент и это \(a \to a\)
\end{proof}

\begin{theorem}\itemfix
    \begin{itemize}
        \item Любая алгебра Гейтинга --- модель интуиционистского исчисления высказываний.
        \item Любая булева алгебра --- модель классического исчисления высказываний.
    \end{itemize}
\end{theorem}

\begin{definition}[топология]
    Рассмотрим множество \(X\), называемое ``носитель'' и \(\Omega \subset \mathcal{P}(X)\) --- подмножество подмножеств \(X\), называемое ``топология'', такое что:
    \begin{enumerate}
        \item \(\bigcup_\alpha x_i \in \Omega\), где \(x_i\in\Omega\)
        \item \(\bigcap_{i = 1}^{n} x_i \in \Omega\), где \(x_i\in\Omega\)
        \item \(\emptyset\in\Omega, X\in \Omega\)
    \end{enumerate}
\end{definition}

\begin{example}
    Пусть \(X\) --- узлы дерева, \(\Omega\) --- все множества узлов, которые содержат узлы вместе со всеми потомками.
\end{example}

\begin{theorem}
    Пусть \((X,\Omega)\) --- топологическое пространство, \(a + b = a\cup b, a\cdot b = a \cap b, a \to b = ((X\setminus a)\subset b)^\circ\), \(a \leq b \Leftrightarrow a \leq b\), тогда \((\Omega, \leq)\) есть алгебра Гейтинга.
\end{theorem}

\begin{example}
    Дискретная топология --- \(\Omega = \mathcal{P}(X)\). Тогда \((\Omega, \leq )\) --- булева алгебра.
\end{example}

\begin{enumerate}
    \item \(X^0 = X\)
    \item \(a \to 0 = (X\setminus a \cup \emptyset) = X\setminus a\)
\end{enumerate}

Таким образом, \(a + (a \to 0) = a + X\setminus a = X\)

\begin{definition}
    Пусть \(X\) --- все формулы логики. Определим отношение порядка \(\alpha \leq \beta\) это \(\alpha \vdash \beta\). Будем говорить, что \(\alpha \approx \beta\), если \(\alpha \vdash \beta\) и \(\beta \vdash \alpha\).

    \((X /_\approx, \leq )\) есть алгебра Гейтинга.
\end{definition}

\begin{definition}
    \((X /_\approx, \leq )\) --- \textbf{алгебра Линденбаума}, где \(X, \approx\) из интуиционистской логики.
\end{definition}

\begin{theorem}
    Алгебра Гейтинга --- полная модель интуиционистской логики.
\end{theorem}
\begin{proof}
    \(\models \alpha\) --- истинно в любой алгебре Гейтинга, в частности в \((X /_\approx, \leq )\). \(\llbracket \alpha \rrbracket = 1\), т.е. \(\llbracket \alpha \rrbracket = \llbracket A \to A \rrbracket\), т.е. \(\alpha \in [A \to A]_\approx\), т.е. \(A \to A \vdash \alpha\).
\end{proof}

\chapter{5 марта}

\begin{definition}
    \textbf{Полный порядок} --- линейный, где в каждом подмножестве есть наименьший элемент. Множество с полным порядком называют \textbf{вполне упорядоченным}.
\end{definition}
\begin{example}
    \(\N\) --- вполне упорядоченное множество

    \(\R\) --- не вполне упорядоченное множество, т.к. \((a, b)\) не имеет наименьшего \(\forall a, b\). Кроме того, \(\R\) не имеет наименьшего.
\end{example}
\begin{definition}
    \textbf{Предпорядок} --- транзитивное, рефлексивное отношение.
\end{definition}

Как мы знаем из домашнего задания, по предпорядку можно построить частичный порядок, сжав компоненты связности в классы эквивалентности.

\subsection{Табличные модели}

\begin{definition}
    \textbf{Табличная модель} для интуиционистского исчисления высказываний:
    \begin{itemize}
        \item \(V\) --- множество истинностных значений
        \item \(f_\to, f_\with, f_\lor : V^2 \to V\)
        \item Выделенное истинное значение \(T\in V\)
        \item Оценка переменных \(\llbracket P_i \rrbracket \in V\), \(f_{\mathcal{P}} : P_i \to V\)
    \end{itemize}
    И \(\llbracket P_i \rrbracket = f_{\mathcal{P}}(P_i)\), \(\llbracket \alpha \star \beta \rrbracket = f_\star(\llbracket \alpha \rrbracket, \llbracket \beta \rrbracket)\), \(\llbracket \neg \alpha \rrbracket = f_\neg(\llbracket \alpha \rrbracket)\)

    \(\models \alpha\) означает, что \(\llbracket \alpha \rrbracket = T\) при любой \(f_{\mathcal{P}}\)
\end{definition}
\begin{definition}
    \textbf{Конечная} табличная модель --- табличная модель с конечным \(V\).
\end{definition}

\begin{theorem}
    У интуиционистского исчисления высказываний не существует корректной полной табличной модели.
\end{theorem}

Неформально эта теорема говорит, что нельзя считать, что в интуиционистской логике есть три значения --- истинна, ложь и ``неизвестно''.

\subsection{Модели Крипке}

Идея моделей Крипке следующая: общезначимое утверждение истинно во всех мирах.

\begin{definition}[модели Крипке]\itemfix
    \begin{enumerate}
        \item \(W = \{W_i\}\) --- множество миров
        \item \( \leq \) --- частичный порядок на \(W\)
        \item Отношение вынужденности \(W_j \Vdash P_i\), где \(P_i\) --- переменная, т.е. \((\Vdash) \subset W \times \mathcal{P}\)
    \end{enumerate}

    При этом, если \(W_j \Vdash P_i\) и \(W_j \leq W_k\), то \(W_k \Vdash P_i\)
\end{definition}

\begin{definition}\itemfix
    \begin{itemize}
        \item \(W_i \Vdash \alpha\) и \(W_i \Vdash \beta\), тогда \textit{(и только тогда)} \(W_i \Vdash \alpha \with \beta\)
        \item \(W_i \Vdash \alpha\) или \(W_i \Vdash \beta\), тогда \textit{(и только тогда)} \(W_i \Vdash \alpha \lor \beta\)
        \item Пусть во всех \(W_i \leq W_j\) всегда, когда \(W_j \Vdash \alpha\), имеет место \(W_j \Vdash \beta\). Тогда \(W_i \Vdash \alpha \to \beta\)
        \item \(W_i \Vdash \neg \alpha\) значит, что \(\alpha\) не вынуждено нигде, начиная с \(W_i\): \(W_i \leq W_j \Rightarrow W_j \nVdash \alpha\)
    \end{itemize}
\end{definition}

\begin{theorem}
    Если \(W_i \Vdash \alpha\) и \(W_i \leq W_j\), то \(W_i \Vdash \alpha\)
\end{theorem}

\begin{definition}
    Если \(W_i \Vdash \alpha\) при всех \(W_i\in W\), то \(\models \alpha\)
\end{definition}

\begin{theorem}
    ИИВ корректно в моделях Крипке.
\end{theorem}
\begin{proof}
    Рассмотрим \((W, \Omega)\) --- топологию, где \(\Omega = \{w\subset W \ |\ \text{если } w_i \in w, w_i \leq w_j \text{, то } w_j\in w\} \). Это можно представить как множество подлесов, где любая вершина входит со своими потомками.

    \(\{W_k\ |\ W_k \Vdash P_j\}\) --- открытое множество, что очевидно из определения \(\Omega\) и \(\Vdash\).

    Примем \(\llbracket P_i \rrbracket = \{W_k\ |\ W_k\Vdash P_j\}\) и аналогично \(\llbracket \alpha \rrbracket = \{W_k\ |\ W_k \Vdash \alpha\}\). Корректность этого определения докажем в ДЗ.

    Поскольку любая топология является корректной моделью ИИВ, искомое доказано.
\end{proof}

\begin{proof}[Доказательство теоремы о нетабличности]
    Предположим обратное, т.е. существует конечная табличная модель, \(|V| = n\).

    Рассмотрим следующую формулу:
    \[\varphi_n = \bigvee_{\substack{ 1 \leq i, j \leq n + 1 \\ i \neq j}} (P_i \to P_j \with P_j \to P_i)\]

    \begin{enumerate}
        \item \(\nvdash\varphi_n\). Почему? Рассмотрим последовательность миров, таких что \(W_i \Vdash P_i\), состоящую из \(n + 1\) мира. Тогда \(W_i \nVdash (P_i \to P_j) \with (P_k \to P_j)\), таким образом \(\nVdash (P_i \to P_j) \with (P_k \to P_j)\) и \(\nVdash \bigvee (P_i \to P_j) \with (P_k \to P_j)\), а значит \(\nvdash \varphi_n\)
        \item \(\models \varphi_n\) в \(V\) по принципу Дирихле: \(\exists i \neq j : \llbracket P_i \rrbracket = \llbracket P_j \rrbracket\), а значит \(\llbracket P_i \to P_j \rrbracket = \) И, и соответственно \(\llbracket \varphi_n \rrbracket =\) И.
    \end{enumerate}

    Т.к. \(\models \varphi_n\), то \(\vdash \varphi_n\), но это не так --- противоречие.
\end{proof}

\begin{definition}
    Дизъюнктинвость ИИВ: \(\vdash \alpha \lor \beta\) влечет \(\vdash \alpha\) или \(\vdash \beta\)
\end{definition}

\begin{definition}
    \textbf{Алгебра Гёделя} --- алгебра Гейтинга, в которой из \(a + b = 1\) следует \(a = 1\) или \(b = 1\)
\end{definition}

\begin{definition}
    Пусть \(\mathcal{A}\) --- алгебра Гейтинга. Тогда \(\Gamma(\mathcal{A})\) получается переименовыванием \(1\) в \(\omega\) и добавлением нового элемента \(1_{\Gamma(\mathcal{A})}\), являющегося единицей для новой алгебры.
\end{definition}

\begin{theorem}
    \(\Gamma(\mathcal{A})\) есть алгебра Гейтинга и \(\Gamma(\mathcal{A})\) Гёделева.
\end{theorem}

\begin{proof}
    Очевидно.
\end{proof}

\begin{definition}
    \textbf{Гомоморфизм алгебр Гейтинга} --- отображение \(\varphi : \mathcal{A} \to \mathcal{B}\), где \(\mathcal{A}, \mathcal{B}\) --- алгебры Гейтинга, \(\varphi(a \star b) = \varphi(a) \star \varphi(b)\), \(\varphi(1_{\mathcal{A}}) = 1_{\mathcal{B}}\), \(\varphi(0_{\mathcal{A}}) = 0_{\mathcal{B}}\)
\end{definition}

\begin{theorem}
    Если \(a \leq b\), то \(\varphi(a) \leq \varphi(b)\)
\end{theorem}

\begin{definition}
    Пусть \(\alpha\) --- формула ИИВ, \(f, g\) --- оценки ИИВ, где \(f : \text{ИИВ} \to \mathcal{A}, g : \text{ИИВ} \to \mathcal{B}\). Тогда \(\varphi\) \textbf{согласовано} с \(f, g\), если \(\varphi(f(\alpha)) = g(\alpha)\)
\end{definition}

\begin{theorem}
    Если \(\varphi : \mathcal{A} \to \mathcal{B}\) согласована с \(f, g\) и \(\llbracket \alpha \rrbracket_g \neq 1_{\mathcal{B}}\), то \(\llbracket \alpha \rrbracket_f \neq 1_{\mathcal{A}}\)
\end{theorem}

\begin{proof}
    Рассмотрим алгебру Линденбаума \(\mathcal{L}\), \(\Gamma(\mathcal{L})\) и \(\varphi : \Gamma(\mathcal{L}) \to \mathcal{L}\) --- гомоморфизм.
    \[\varphi(x) = \begin{cases}
            1_\mathcal{L}, x = \omega                  \\
            1_\mathcal{L}, x = 1_{\Gamma(\mathcal{L})} \\
            x, \text{иначе}
        \end{cases}\]
    Пусть \(\vdash \alpha \lor \beta\). Тогда \(\llbracket \alpha \lor \beta \rrbracket_{\Gamma(\mathcal{L})} = 1_{\Gamma(\mathcal{L})}\), но по Гёделевости \(\Gamma(\mathcal{L})\) \(\llbracket \alpha \rrbracket = 1\) или \(\llbracket \beta \rrbracket = 1\).

    Пусть \(\nvdash \alpha\) и \(\nvdash \beta\). Тогда \(\varphi(\llbracket \alpha \rrbracket) \neq 1_{\mathcal{L}}\) и \(\varphi(\llbracket \beta \rrbracket) \neq 1_{\mathcal{L}}\). Тогда \(\llbracket \alpha \rrbracket_{\Gamma(\mathcal{L})} \neq 1_\mathcal{L}, \llbracket \beta \rrbracket \neq 1_\mathcal{L}\) --- противоречие.
\end{proof}

\chapter{12 марта}

\section{Изоморфизм Карри-Ховарда}

\begin{remark}
    Эта тема в нашем курсе рукомахательная.
\end{remark}

Пусть \(p\) --- программа, т.е. функция, принимающая \(\alpha\) и возвращающая \(\beta\), т.е. \(p : \alpha \to \beta\)

Можем посмотреть на это с другой стороны: \(p\) доказательство, что из \(\alpha\) следует \(\beta\), например в \texttt{Haskell} \texttt{f a = a} гласит, что \texttt{f} доказывает, что \texttt{A -> A}, где подразумевается \(\forall A\).

Такое сопоставление программам доказательств и высказываниям типов называется изоморфизмом Карри-Ховарда:

\begin{tabular}{|c|c|}
    \hline
    логическое исчисление & типизированное \(\lambda\)-исчисление   \\ \hline
    логическая формула    & тип                                     \\
    доказательство        & программа                               \\
    доказуемая формула    & обитаемый тип                           \\
    \( \to \)             & функция                                 \\
    \(\with\)             & упорядоченная пара                      \\
    \(\lor\)              & алгебраический тип \textit{(тип-сумма)} \\
    \hline
\end{tabular}

\begin{remark}
    Обитаемый тип --- тип, у которого есть хотя бы один экземпляр.
\end{remark}

Несложно заметить, что логика, соответствующая \(\lambda\)-исчислению, является интуиционистской, поэтому мы её в основном изучаем.

\subsection{Алгебраические типы}

Рассмотрим следующее определение списка в \texttt{Pascal}:

\begin{minted}{pascal}
type list : record
    nul : boolean;
    case nul of
        true: ;
        false: next ^list
    end
end;
\end{minted}

Рассмотрим то же самое в \texttt{C}, опустив \texttt{bool} и скажем, что \texttt{nul = (next == null)} \textit{(это в какой-то степени костыльно)}:

\begin{minted}{c}
struct list {
    next: *list;
}
\end{minted}

Определим таким же способом дерево:

\begin{minted}{c}
struct tree {
    tree* left;
    tree* right;
    int value;
}
\end{minted}

Это ещё более костыльно, т.к. то, является ли вершина листом, закодировано в неявном виде.

\begin{definition}
    \textbf{Отмеченное \textit{(дизъюнктное)} объединение} множеств \(A, B\) обозначается \(A\sqcup B\) или \(A \uplus B\) \footnote{или ещё десятком других символов} и равно \(\{\ev{\text{``}A\text{''}, a}\ |\ a\in A\} \cup \{\ev{\text{``}B\text{''}, b}\ |\ b\in B\}\).
\end{definition}

\begin{remark}
    Это определение интуиционистское по своей сути, т.к. если дано \(s\in A \sqcup B\), то мы знаем, из какого множества \(s\).
\end{remark}

\begin{definition}
    Тип, соответствующий такому объединению множеств, называется \textbf{алгебраическим}
\end{definition}

\begin{example}
    В \texttt{C++} такой тип реализован как \texttt{std::variant<...>}
\end{example}

\begin{example}
    Список в \texttt{Haskell}:

    \begin{minted}{haskell}
data List a = nil | Cons a (List a)
    \end{minted}
\end{example}

\subsection{Применение восьмой аксиомы интуиционистской логики}

Вспомним восьмую аксиому интуиционистской \footnote{и классической} логики и запишем её как правило натурального вывода:
\begin{center}
    \begin{prooftree}
        \hypo{\Gamma \vdash \alpha \to \gamma}
        \hypo{\Gamma \vdash \beta \to \gamma}
        \hypo{\Gamma \vdash \alpha \lor \beta}
        \infer3{\Gamma \vdash \gamma}
    \end{prooftree}
\end{center}

Рассмотрим программу в \texttt{Haskell}, которая преобразует список в строку:

\begin{minted}{haskell}
let rec string_of_list l =
    match l with
        Nil -> "Nil"
        Cons(head, tail) -> head ^ ":" ^ string_of_list tail
\end{minted}

Подставим в рассматриваемую аксиому соответствующие значения:
\begin{center}
    \begin{prooftree}
        \hypo{\Gamma \vdash Nil \to string}
        \hypo{\Gamma \vdash list \to string}
        \hypo{\Gamma \vdash Nil \lor list}
        \infer3{\Gamma \vdash string}
    \end{prooftree}
\end{center}

Несложно заметить, что эта аксиома описывает \texttt{match} в \texttt{Haskell} --- мы даем выражения после ``\texttt{->}'', т.е. правила \texttt{Nil -> string}, \texttt{list -> string} и элемент \texttt{Nil} или \texttt{list}, а \texttt{match} возвращает \texttt{string}.

\section{Исчисление предикатов}

\subsection{Язык исчисления предикатов}

Выражения в этом языке бывают двух видов:
\begin{enumerate}
    \item Логические выражения, называемые ``предикаты'' или ``формулы''
    \item Предметные выражения, называемые ``термы''
\end{enumerate}

\(\theta\) --- метапеременная для термов.

Термы бывают двух видов:
\begin{itemize}
    \item Атомы:
          \begin{itemize}
              \item Предметные переменные обозначаются буквами \(a,b,c \dots \)
              \item Метапеременные обозначаются буквами \(x,y,z\)
          \end{itemize}
    \item Применение функциональных символов:
          \begin{itemize}
              \item Функциональные символы: \(f,g,h\) и записывается \(f(\theta_1 \dots \theta_n)\)
              \item Метапеременная тоже обозначается \(f\)
          \end{itemize}
\end{itemize}

Логические выражения:
\begin{itemize}
    \item Применение предикатных символов \(P(\theta_1, \dots \theta_n)\), где \(P\) --- метапеременная для предикатных символов, а предикатный символ --- \(A, B, C \dots \)
    \item Связки \(\with, \lor, \neg, \to \) с правилами из языка классической логики.
    \item Кванторы \footnote{По записи кванторов нет общепринятого соглашения.} \(\forall x.\varphi\) или \(\exists x.\varphi\), где \(\varphi\) --- любое логическое выражение.
\end{itemize}

Мы используем жадность кванторов. \footnote{В отношении жадности кванторов также нет соглашения; встречается запись, где квантор --- унарная операция, аналогичная \(\neg\)} Это значит, что квантор берет в \(\varphi\) все, пока не встретит конец выражения или скобку, которая оканчивает этот квантор.

\begin{example}
    \(\forall x.P(x) \with \forall y.P(y) \equiv \forall x.(P(x) \with (\forall y.P(y)))\)
\end{example}

\subsection{Теория моделей}

Определим оценку формулы в исчислении предикатов:

\begin{enumerate}
    \item Фиксируем \(D\) --- предметное множество, \(V = \{\text{И}, \text{Л}\} \)
    \item Каждому \(f_i(x_1 \dots x_n)\) сопоставим функцию \(f_{f_n} : D^n \to D\)
    \item Каждому \(P_j(x_1 \dots x_n)\) сопоставим функцию \footnote{,  называемую предикат} \(f_{p_n} : D^n \to V\)
    \item Каждой \(x_i\) сопоставим \(f_{x_i}\in D\)
\end{enumerate}

\begin{itemize}
    \item \(\llbracket x \rrbracket = f_{x_i}\)
    \item \(\llbracket \alpha \star \beta \rrbracket\) --- так же, как в исчислении высказываний.
    \item \(\llbracket P_i(\theta_1 \dots \theta_n) \rrbracket = f_{p_i}(\llbracket \theta_1 \rrbracket \dots \llbracket \theta_n \rrbracket)\)
    \item \(\llbracket f_j(\theta_1 \dots \theta_n) \rrbracket = f_{f_j}(\llbracket \theta_1 \rrbracket \dots \llbracket \theta_m \rrbracket)\)
    \item \(\llbracket \forall x.\varphi \rrbracket = \begin{cases} \text{И}, & \text{если } \llbracket \varphi \rrbracket = \text{И} \text{ при всех } k\in D \\
              \text{Л}, & \text{иначе}
          \end{cases} \)
    \item \(\llbracket \exists x.\varphi \rrbracket = \begin{cases} \text{И}, & \text{если } \llbracket \varphi \rrbracket = \text{И} \text{ при некотором } k\in D \\
              \text{Л}, & \text{иначе}
          \end{cases} \)
\end{itemize}

\begin{example}
    \(\forall x.\forall y.E(x, y)\)

    Пусть \(D = \N\), \(E(x, y) = \begin{cases} \text{И}, & x = y \\ \text{Л}, & x \neq y \end{cases} \)

    \(\llbracket\forall x.\forall y.E(x, y)\rrbracket_{x: = 1, y: = 2} =\) Л, т.к. \(\llbracket E(x, y) \rrbracket =\) Л.
\end{example}

Вспомним определение предела последовательности из матанализа:

\[\forall \varepsilon > 0 \ \ \exists N \ \ \forall n > N \ \ |a_n - a| < \varepsilon\]

Перепишем это определение с богомерзкого языка матанализа на православный язык исчисления предикатов. \footnote{Это термины лектора, все претензии от адептов матанализа и других религий --- к нему.}

Пусть \(( >)(a, b) = G(a, b), |a|= m_|(a), ( -)(a,b) = m_{ -}(a, b), m_a : n \mapsto a_n, 0() = m_0\)

\begin{align*}
     & \forall \varepsilon . \varepsilon \to 0\ \exists N . \forall n. (n > N) \to (|a_n - a| < \varepsilon)      \\
     & \forall \varepsilon . \varepsilon \to 0\ \exists N . \forall n. (n > N) \to (|a_n - a| < \varepsilon)      \\
     & \forall e . G(e, m_0)\ \exists n_0 . \forall n. G(n, n_0) \to G(e, m_|(m_{ -}(m_a(n), a)))) < \varepsilon) \\
\end{align*}

\subsection{Теория доказательств}

Все аксиомы исчисления высказываний + Modus Ponens + две схемы аксиом + два правила:
\begin{enumerate}
    \item \((\forall x.\varphi) \to \varphi [x : = \theta]\)
    \item \(\varphi[x : = \theta] \to \exists x.\varphi\)
\end{enumerate}
Обе эти схемы применимы только если \(\theta\) свободен для подстановки вместо \(x\) в \(\varphi\), т.е. никакое свободное вхождение \(x\) в \(\theta\) не станет связным.

\begin{example}\itemfix
    \begin{minted}{c}
int f(int x) {
    x = y;
}
    \end{minted}

    После замены \(y : = x\) код станет следующим:
    \begin{minted}{c}
int f(int x) {
    x = x;
}
    \end{minted}

    И код потеряет свой смысл.
\end{example}

Правила следующие:
\begin{enumerate}
    \item \begin{prooftree}
              \hypo{\varphi \to \psi}
              \infer1[(правило \(\forall \))]{\varphi \to (\forall x.\psi)}
          \end{prooftree}
    \item \begin{prooftree}
              \hypo{\psi \to \varphi}
              \infer1[(правило \(\exists \))]{(\exists x.\psi) \to \varphi}
          \end{prooftree}
\end{enumerate}

\end{document}
