\documentclass[12pt, a4paper]{article}

\usepackage{lastpage}
\usepackage{mathtools}
\usepackage{xltxtra}
\usepackage{libertine}
\usepackage{amsmath}
\usepackage{amsthm}
\usepackage{amsfonts}
\usepackage{amssymb}
\usepackage{enumitem}
\usepackage{xcolor}
\usepackage[left=1.5cm, right=1.5cm, top=2cm, bottom=2cm, bindingoffset=0cm, headheight=15pt]{geometry}
\usepackage{fancyhdr}
\usepackage[russian]{babel}
% \usepackage[utf8]{inputenc}
\usepackage{catchfilebetweentags}
\usepackage{accents}
\usepackage{calc}
\usepackage{etoolbox}
\usepackage{mathrsfs}
\usepackage{wrapfig}

\providetoggle{useproofs}
\settoggle{useproofs}{false}

\pagestyle{fancy}
\lfoot{M3137y2019}
\rhead{\thepage\ из \pageref{LastPage}}

\newcommand{\R}{\mathbb{R}}
\newcommand{\Q}{\mathbb{Q}}
\newcommand{\C}{\mathbb{C}}
\newcommand{\Z}{\mathbb{Z}}
\newcommand{\B}{\mathbb{B}}
\newcommand{\N}{\mathbb{N}}

\newcommand{\const}{\text{const}}

\newcommand{\teormin}{\textcolor{red}{!}\ }

\DeclareMathOperator*{\xor}{\oplus}
\DeclareMathOperator*{\equ}{\sim}
\DeclareMathOperator{\Ln}{\text{Ln}}
\DeclareMathOperator{\sign}{\text{sign}}
\DeclareMathOperator{\Sym}{\text{Sym}}
\DeclareMathOperator{\Asym}{\text{Asym}}
% \DeclareMathOperator{\sh}{\text{sh}}
% \DeclareMathOperator{\tg}{\text{tg}}
% \DeclareMathOperator{\arctg}{\text{arctg}}
% \DeclareMathOperator{\ch}{\text{ch}}

\DeclarePairedDelimiter{\ceil}{\lceil}{\rceil}
\DeclarePairedDelimiter{\abs}{\left\lvert}{\right\rvert}

\setmainfont{Linux Libertine}

\theoremstyle{plain}
\newtheorem{axiom}{Аксиома}
\newtheorem{lemma}{Лемма}

\theoremstyle{remark}
\newtheorem*{remark}{Примечание}
\newtheorem*{exercise}{Упражнение}
\newtheorem*{consequence}{Следствие}
\newtheorem*{example}{Пример}
\newtheorem*{observation}{Наблюдение}

\theoremstyle{definition}
\newtheorem{theorem}{Теорема}
\newtheorem*{definition}{Определение}
\newtheorem*{obozn}{Обозначение}

\setlength{\parindent}{0pt}

\newcommand{\dbltilde}[1]{\accentset{\approx}{#1}}
\newcommand{\intt}{\int\!}

% magical thing that fixes paragraphs
\makeatletter
\patchcmd{\CatchFBT@Fin@l}{\endlinechar\m@ne}{}
  {}{\typeout{Unsuccessful patch!}}
\makeatother

\newcommand{\get}[2]{
    \ExecuteMetaData[#1]{#2}
}

\newcommand{\getproof}[2]{
    \iftoggle{useproofs}{\ExecuteMetaData[#1]{#2proof}}{}
}

\newcommand{\getwithproof}[2]{
    \get{#1}{#2}
    \getproof{#1}{#2}
}

\newcommand{\import}[3]{
    \subsection{#1}
    \getwithproof{#2}{#3}
}

\newcommand{\given}[1]{
    Дано выше. (\ref{#1}, стр. \pageref{#1})
}

\renewcommand{\ker}{\text{Ker }}
\newcommand{\im}{\text{Im }}
\newcommand{\grad}{\text{grad}}

\lhead{Дифференциальные уравнения}
\cfoot{}
\rfoot{17.9.2020}

\setcounter{section}{2}
\setcounter{subsection}{4}

\begin{document}

\begin{theorem}[Общее решение уравнения 1-го пордяка]
    $\sphericalangle p, q \in C(a, b)$
    $$y = \left( C + \int q(x) e^{-\int p(x)dx} \right)e^{\int p(x)dx}, \ \ x\in(a, b), C\in\R$$
\end{theorem}
\begin{proof}
    $$q(x) e^{-\int p(x)dx} e^{\int p(x)dx} + \underbrace{\left(C + \int q(x) e^{-\int p(x)dx}\right) e^{\int p(x)dx}}_{y} p(x) \equiv p(x) y + q \quad \forall x\in(a, b)$$
    Докажем, что других решений нет.

    $\sphericalangle \varphi$ --- решение на $(\alpha, \beta)\subset(a, b)$. Пусть это решение проходит через $(x_0, y_0)$. Через эту точку также проходит решение $f$, описываемой теоремой.

    $\cfrac{\partial f}{\partial y} = p(x) \in C((a, b)\times \R) \Rightarrow$ по теореме Пикара $y \equiv \varphi$ на $(\alpha, \beta)$.
\end{proof}

\textbf{Метод Лагранжа}
\begin{enumerate}
    \item Записываем решение однородного: $y = Ce^{\int p(x) dx}$
    \item Подставим $y = C(x) e^{\int p(x) dx}$ в линейное уравнение первого порядка:
          $$C^1 e^{\int p(x)dx} + Ce^{\int p(x)dx} p(x) = pCe^{\int p(x)dx} + q$$
    \item Подставим $C(x)$ вместо постоянной $C$.
\end{enumerate}

\subsection{Уравнение Бернулли и Риккати}

\begin{definition}[уравнение Бернулли]
    $y' = p(x) y + q(x) y^\alpha, \alpha\not\in\{0, 1\}$
\end{definition}
Замена $z=y^{1-\alpha}$ сводит это уравнение к линейному:
\begin{align*}
    z'                  & = (1-\alpha)y^{-\alpha}    \\
    \frac{y'}{y^\alpha} & = p(x) y^{1-\alpha} + q(x) \\
    \frac{z'}{1-\alpha} & = p(x) z + q(x)
\end{align*}

\begin{definition}[уравнение Риккати]
    $y' = \underbrace{p(x)y^2 + q(x)y + r(x)}_{\text{квадратичное по } y}$
\end{definition}

Частный случай уравнения Риккати $y' = y^2 + x^\alpha$ интегрируется в квадратурах только при определенных $\alpha$.

Если $\varphi$ --- решение уравнения Риккати, то замена $y = \varphi + z$ сводит искомое к уравнению Бернулли. $\varphi$ необходимо угадывать.

\begin{proof}
    \begin{align*}
        (\varphi + z)' & = p(\varphi+z)^2 + q(\varphi+z) + r(x)                \\
        \varphi' + z'  & = p\varphi^2 + q\varphi + 2p\varphi z + qz + r + pz^2 \\
        z'             & = 2p\varphi z + qz + pz^2                             \\
    \end{align*}
\end{proof}

\subsection{Уравнение в полных дифференциалах}

\begin{definition}
    $P(x, y)dx + Q(x, y)dy = 0$ --- \textbf{уравнение в полных дифференциалах}, если $\exists u(x, y): du = Pdx + Qdy$
\end{definition}

\begin{theorem}[Общее решение УПД]\itemfix
    $\sphericalangle P, Q\in C(G)$
    \begin{enumerate}
        \item $y$ --- решение на $(a, b)$

              Тогда $\exists C\in\R : u(x, y)=C$ неявно задает $y$.
        \item $u(x, y)=C$ определяет $y\in C^1(a, b)$

              Тогда $y$ --- решение.
    \end{enumerate}
\end{theorem}
\begin{proof}
    \begin{enumerate}
        \item $\sphericalangle y$ --- решение на $(a, b) \Rightarrow$
              $$P(x, y(x)) + Q(x, y(x)), y'(x) \equiv 0$$
              Левая часть $=\cfrac{d}{dx}u(x, y(x))$
              $$\frac{du(x, y)}{dt} = \frac{\partial u}{\partial x} \frac{\partial x}{\partial t} + \frac{\partial u}{\partial y} + \frac{\partial y}{\partial t}$$
              $$\Rightarrow u(x, y(x)) \equiv C$$
        \item $\sphericalangle y' : u(x, y(x)) \equiv C$

              Продифференцируем.
              \begin{align*}
                  \frac{\partial u}{\partial x} + \frac{\partial u}{\partial y}y' & \equiv 0 \\
                  P + Q y'                                                        & \equiv 0
              \end{align*}
    \end{enumerate}
\end{proof}

$u(x, y)$ находится следующим образом:
$$u(x, y) = \int_{(x_0, y_0)}^{(x, y)} P(x, y)dx + Q(x, y)dy$$

Найдем необходимое условие для $\exists du$:
\begin{align*}
    \sphericalangle du                        & = Pdx + Qdy,\ P, Q\in C^1(G)                                                \\
    \Rightarrow P                             & = \frac{\partial u}{\partial x} + \frac{\partial u}{\partial y}             \\
    \Rightarrow \frac{\partial P}{\partial y} & = \frac{\partial^2 u}{\partial x\partial y} = \frac{\partial Q}{\partial x}
\end{align*}

Это условие и достаточно для $\exists du$, если $G$ --- односвязная область.

\begin{example}
    $\underbrace{e^{-y}}_{P}dx \underbrace{- (2y + xe^{-y})}_{Q} dy = 0$

    Решение.
    \begin{align*}
        \frac{\partial P}{\partial y} = -e^{-y} \\
        \frac{\partial Q}{\partial x} = -e^{-y} \\
    \end{align*}
    Частные производные совпали, область односвязна $\Rightarrow$ УПД.
    $$\frac{\partial u}{\partial x} = P = e^{-y} \Rightarrow U(x, y) = \int e^{-y} dy + C(y) = xe{-y} + C(y)$$
    $$\frac{\partial u}{\partial y} = xe^{-y} (-1) + C'(y) = -(2y + xe^{-y})$$
    $$C'(y) = -2y$$
    $$C = -y^2 + A$$
    $$u(x, y) = xe^{-y} - y^2 + A$$

    Ответ: $xe^{-y}-y^2=C$
\end{example}

\textbf{Геометрический смысл}

$P(x, y)dx + Q(x, y)dy = 0$

$\sphericalangle \begin{cases}
        x = \varphi(t) \\
        y = \psi(t)
    \end{cases}$ --- интегральная кривая.

$$(x_0, y_0) = (\varphi(t_0), \psi(t_0))$$
$$P(\varphi, \psi) \varphi' + Q(\varphi, \psi)\psi' = 0$$
$$P(x_0, y_0) \varphi'(t_0) + Q(x_0, y_0)\psi'(t_0) = 0$$

Таким образом, $(\varphi', \psi')$ перпендикулярно $(P, Q)$ при $t=t_0$, т.е. мы ищем перпендикуляры к данному векторному полю.

\begin{definition}
    $\mu(x, y)$ --- \textbf{интегрирующий множитель} для уравнения $Pdx+Qdy = 0$, если $\mu\not=0$ и $\mu Pdx + \mu Qdy = 0$
\end{definition}

Домножая на $\mu$, можно свести диффур к УПД.

$$\frac{\partial (\mu P)}{\partial y} = \frac{\partial (\mu Q)}{\partial x}$$
$$y' = p(x)y + q(x)$$
$$dy - (p(x)y + q(x))dx = 0$$
$$\frac{\partial \mu}{\partial y}(-p(x)y - q(x)) + \mu(-\varphi) = \frac{\partial \mu}{\partial x}$$

Предположим, что $\mu$ зависит только от $x$.

$$\mu(-\varphi) = \mu' \Rightarrow \mu = Ce^{-\int p(x)dx}$$
\begin{align*}
    \mu(dy - (p(x)y + q(x))dx)                              & = 0                       \\
    e^{-\int p(x)dx} dy - (p(x)y + q(x))e^{-\int p(x)dx} dx & = 0                       \\
    e^{-\int p(x)dx} dy - p(x)ye^{-\int p(x)dx} dx          & = q(x)e^{-\int p(x)dx} dx
\end{align*}
Дорешать --- упражнение

\end{document}