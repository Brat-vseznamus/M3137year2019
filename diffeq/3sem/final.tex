\documentclass[12pt, a4paper]{article}

\usepackage{lastpage}
\usepackage{mathtools}
\usepackage{xltxtra}
\usepackage{libertine}
\usepackage{amsmath}
\usepackage{amsthm}
\usepackage{amsfonts}
\usepackage{amssymb}
\usepackage{enumitem}
\usepackage{xcolor}
\usepackage[left=1.5cm, right=1.5cm, top=2cm, bottom=2cm, bindingoffset=0cm, headheight=15pt]{geometry}
\usepackage{fancyhdr}
\usepackage[russian]{babel}
% \usepackage[utf8]{inputenc}
\usepackage{catchfilebetweentags}
\usepackage{accents}
\usepackage{calc}
\usepackage{etoolbox}
\usepackage{mathrsfs}
\usepackage{wrapfig}

\providetoggle{useproofs}
\settoggle{useproofs}{false}

\pagestyle{fancy}
\lfoot{M3137y2019}
\rhead{\thepage\ из \pageref{LastPage}}

\newcommand{\R}{\mathbb{R}}
\newcommand{\Q}{\mathbb{Q}}
\newcommand{\C}{\mathbb{C}}
\newcommand{\Z}{\mathbb{Z}}
\newcommand{\B}{\mathbb{B}}
\newcommand{\N}{\mathbb{N}}

\newcommand{\const}{\text{const}}

\newcommand{\teormin}{\textcolor{red}{!}\ }

\DeclareMathOperator*{\xor}{\oplus}
\DeclareMathOperator*{\equ}{\sim}
\DeclareMathOperator{\Ln}{\text{Ln}}
\DeclareMathOperator{\sign}{\text{sign}}
\DeclareMathOperator{\Sym}{\text{Sym}}
\DeclareMathOperator{\Asym}{\text{Asym}}
% \DeclareMathOperator{\sh}{\text{sh}}
% \DeclareMathOperator{\tg}{\text{tg}}
% \DeclareMathOperator{\arctg}{\text{arctg}}
% \DeclareMathOperator{\ch}{\text{ch}}

\DeclarePairedDelimiter{\ceil}{\lceil}{\rceil}
\DeclarePairedDelimiter{\abs}{\left\lvert}{\right\rvert}

\setmainfont{Linux Libertine}

\theoremstyle{plain}
\newtheorem{axiom}{Аксиома}
\newtheorem{lemma}{Лемма}

\theoremstyle{remark}
\newtheorem*{remark}{Примечание}
\newtheorem*{exercise}{Упражнение}
\newtheorem*{consequence}{Следствие}
\newtheorem*{example}{Пример}
\newtheorem*{observation}{Наблюдение}

\theoremstyle{definition}
\newtheorem{theorem}{Теорема}
\newtheorem*{definition}{Определение}
\newtheorem*{obozn}{Обозначение}

\setlength{\parindent}{0pt}

\newcommand{\dbltilde}[1]{\accentset{\approx}{#1}}
\newcommand{\intt}{\int\!}

% magical thing that fixes paragraphs
\makeatletter
\patchcmd{\CatchFBT@Fin@l}{\endlinechar\m@ne}{}
  {}{\typeout{Unsuccessful patch!}}
\makeatother

\newcommand{\get}[2]{
    \ExecuteMetaData[#1]{#2}
}

\newcommand{\getproof}[2]{
    \iftoggle{useproofs}{\ExecuteMetaData[#1]{#2proof}}{}
}

\newcommand{\getwithproof}[2]{
    \get{#1}{#2}
    \getproof{#1}{#2}
}

\newcommand{\import}[3]{
    \subsection{#1}
    \getwithproof{#2}{#3}
}

\newcommand{\given}[1]{
    Дано выше. (\ref{#1}, стр. \pageref{#1})
}

\renewcommand{\ker}{\text{Ker }}
\newcommand{\im}{\text{Im }}
\newcommand{\grad}{\text{grad}}

\lhead{Дифференциальные уравнения}
\cfoot{}
\rfoot{Конспект к экзамену}

\usepackage{graphicx}

\begin{document}

\section*{Основные вопросы}

\subsection*{1. Уравнение с разделяющимися переменными: общее решение, общая схема исследования.}

Уравнение с \textbf{разделенными} переменными имеет вид:
\[X(x)dx + Y(y)dy = 0\]

У него решение имеет вид:
\[\int X(x) dx + \int Y(y)dy = C\]

\begin{proof}
    \[\int X(x) dx + \int Y(y)dy = \int X(x) dx + \int Y(y)y' dx = \int (X(x) + Y(y)y')dx = \int 0dx = C\]
\end{proof}

При этом мы получаем общее решение, когда находим такие \(C\), что ответ \(\in C^1\).

Уравнение с \textbf{разделяющимися} переменными имеет вид:
\[p_1(x)q_1(y)dx + p_2(x)q_2(y)dy = 0\]

Если поделить на \(p_2(x)q_1(y)\), то получим уравнение с разделенными переменными. При этом необходимо убедиться, что мы не делим на ноль.

Если \(\exists y_0 : q_1(y_0) = 0\), то \(y\equiv y_0\) --- решение исходного уравнения. Исключив \(y_0\), мы разбиваем область возможных решений на две подобласти.

Аналогично для \(x\).

После разбиения нужно на каждой области найти решение.

\subsection*{2. Линейное уравнение 1-го порядка: общее решение ЛОУ, общее решение ЛНУ. Метод Лагранжа и метод интегрирующего множителя.}

Линейное уравнение первого порядка это
\[y' = p(x)y + q(x)\]

Если \(q\equiv 0\), то это уравнение \textbf{однородно}, иначе \textbf{неоднородно}.

Общее решение ЛОУ это \(y = Ce^{\int p}, C\in\R\)

\begin{proof}
    Заметим, что \(y\equiv 0\) --- решение. По теореме о единственности оно не является особым. т.к. мы рассматриваем \(p\in C(a, b)\).

    \(\sphericalangle y > 0\).
    \[\frac{dy}{y} = p(x)dx\]
    \[\ln y = \int p(x)dx + C\]
    \[y = e^C e^{\int p(x)dx}\]
    По теореме об общем решении уравнения с разделенными переменными это семейство всех решений исходного уравнения при \(y > 0\).

    Аналогично при \(y < 0\)
\end{proof}

Общее решение ЛНУ это
\[y = \left( C + \int qe^{ -\int p} \right)e^{\int p}\]
\begin{proof}
    Подстановкой легко показать, что это решение. Покажем, что нет других решений.

    Пусть есть решение \(\varphi\) на \((\alpha, \beta)\), не подходящее под искомую формулу.

    Пусть \(x_0 \in (\alpha,\beta)\) и \(\varphi(x_0) = y_0\).

    Функция
    \[C = \left( y_0 e^{ -\int p} - \int q e^{ -\int p} dx\right)\Bigg|_{x = x_0}\]

    подходит под искомую формулу, но при этом является решением задачи Коши \(y(x_0) = y_0\), поэтому \(y \equiv \varphi\) --- противоречие.
\end{proof}

\textbf{Метод Лагранжа \textit{(вариации произвольной постоянной)}} --- постоянную \(C\) считают функцией от \(x\) и получают дифур относительно \(C\).

\subsection*{3. Равностепенно непрерывные функции. Лемма Арцела–Асколи.}

Множество функций \(F \), определенных на \(D\), \textbf{равностепенно непрерывно}, если:
\[\forall \varepsilon > 0 \ \ \exists \delta > 0 \ \ \forall f\in F \ \ \forall x_1, x_2\in D \ \ |x_2 - x_1|< \delta \Rightarrow |f(x_2) - f(x_1)|< \varepsilon\]

\begin{lemma}
    Пусть функции последовательности \(\{f_n\}_{n = 1}^{\infty}\) равномерно ограничены \textit{(\(\exists C : \forall n, x |f_n(x)| < C\))} и равностепенно непрерывны на \([a, b]\). Тогда из нее можно выделить подпоследовательность, равномерно сходящуюся на \([a, b]\).
\end{lemma}
\begin{proof}
    Пусть \(M\) ограничивает \textit{(равномерно)} \(f_n\):
    \[M : = \sup_{n, x} |f_n(x)|\]
    \[\sphericalangle \varepsilon_k = \frac{M}{2^{k + 1}}\]
    \[\forall \varepsilon_k > 0 \ \ \exists \delta_k > 0 \ \ \forall f\in F \ \ \forall x_1, x_2\in D \ \ |x_2 - x_1|< \delta_k \Rightarrow |f(x_2) - f(x_1)|< \varepsilon_k\]

    Поделим всю область \([a, b] \times ( - M, M)\) на прямоугольники со стороной \(\varepsilon_1\) и \(\delta_1\).

    %TODO
\end{proof}

\subsection*{4. ЗК для нормальной системы. Лемма о равносильном интегральном уравнении. Лемма: свойства ломаной Эйлера, определённой на отрезке Пеано.}

\subsection*{5. Теорема Пеано о существовании решения ЗК.}

\subsection*{6. Достаточное условие того, что функция удовлетворяет локальному условию Липшица по заданной переменной.}

\subsection*{7. Достаточное условие того, что функция удовлетворяет глобальному условию Липшица по заданной переменной.}

% \subsection*{8. Лемма Гронуолла. Теорема Пикара (доказательство единственности решения).}

% \subsection*{9. Теорема Пикара (доказательство существования решения).}

% \subsection*{10. Теорема существования и единственности решения ЗК для уравнения $n$-го порядка. Следствие с более простыми условиями.}

% \subsection*{11. Критерий продолжимости.}

% \subsection*{12. Теорема существования и единственности максимального решения.}

% \subsection*{13. Теорема о выходе интегральной кривой за пределы любого компакта.}

% \subsection*{14. Признак продолжимости решения системы, сравнимой с линейной. Теорема о существовании и единственности максимального решения ЛС.}

% \subsection*{15. Формула Остроградского–Лиувилля для решений ЛОС.}

% \subsection*{16. Общее решение ЛОС. Лемма о множестве фундаментальных матриц. Лемма об овеществлении.}

% \subsection*{17. Теорема о фундаментальной системе решений ЛОС с постоянными коэффициентами (случай жорданова базиса общего вида). Определение и свойства матричной экспоненты (без доказательств). Решение задачи Коши при помощи матричной экспоненты.}

% \subsection*{18. Общее решение ЛНС и метод вариации постоянных.}

% \subsection*{19. Теорема об изоморфизме решений ЛОС и ЛОУ, формула Остроградского–Лиувилля для решений ЛОУ. Метод вариации постоянных для ЛНУ.}

% \subsection*{20. Общее решение ЛОУ с постоянными коэффициентами.}

% \subsection*{21. Теорема об устойчивости ЛОС с постоянными коэффициентами.}

% \subsection*{22. Классификация точек покоя ЛОС 2-го порядка (случай вещественных корней).}

% \subsection*{23. Классификация точек покоя ЛОС 2-го порядка (случай комплексных корней).}

% \subsection*{24. Теорема Ляпунова об устойчивости.}

\section*{Дополнительные вопросы}

\subsection*{Уравнение 1-го порядка и его решение.}

% \begin{definition}
Это уравнение вида \(F(x, y, y') = 0\).
% \end{definition}
% \begin{definition}
Функция \(\varphi\) --- решение такого дифференциального уравнения, если:
\begin{enumerate}
    \item \(\varphi\in C^1(a, b)\)
    \item \(F(x, \varphi(x), \varphi'(x)) \equiv 0\) на \((a, b)\)
\end{enumerate}
% \end{definition}
\begin{example}
    \(y' - x = 0\), решение \(y = \frac{x^2}{2} + C\).
\end{example}

Методов решения много, все относятся к частным случаям.

\subsection*{Интегральная кривая уравнения.}

Это график решения уравнения.

\subsection*{Общее решение уравнения.}

Это множество всех его решений.

\subsection*{Уравнение 1-го порядка, разрешённое относительно производной. Геометрический смысл.}

Это уравнение вида \(y' = f(x, y)\).

Пусть \(\varphi\) решение этого уравнения. Тогда \(\varphi'(x) = f(x, \varphi(x))\), то есть тангенс угла наклона касательной к интегральной кривой в точке \((x_0, y_0)\) это \(f(x_0, y_0)\)

\subsection*{Ломаная Эйлера.}

% Пусть у нас дифур, разрешённый относительно производной и есть поле направлений, где приложенный вектор направлен по \(\arctg f(x, y)\).

% Будем строить приближение решения дифура следующим образом:
% \begin{enumerate}
%     \item Возьмём точку \(x_0, y_0\) в качестве начальной.
%     \item Сдвинемся вправо по направлению поля так, что \(x\) сдвинется на \(h\) : \(x_1 = x_0 + h\). Тогда мы получим \(y_1\).
%     \item Повторим.
% \end{enumerate}

% Аналогично строится влево.

% Ломаная, которая получается таким образом, называется ломаной Эйлера:
% \[\begin{cases}
%         x_{k+1} = x_k + h \\
%         y_{k+1} = y_k + f(x_k,y_k)h
%     \end{cases}\]

% \subsection*{Отрезок Пеано.}

\subsection*{Уравнение в дифференциалах, его решение и параметрическое решение.}

Уравнение в дифференциалах получается, если в уравнении, разрешенном относительно производной, записать \(y' = \frac{dy}{dx} \):
\[P(x, y) dx + Q(x, y) dy = 0\]

Функция \(\varphi\) --- решение такого дифференциального уравнения, если:
\begin{enumerate}
    \item \(\varphi\in C^1(a, b)\)
    \item \(P(x, \varphi(x)) + Q(x, \varphi(x)) \varphi'(x) \equiv 0\) на \((a, b)\)
\end{enumerate}

Аналогично можно определить решение вида \(x = \psi(y)\).

Функция \(r = (\varphi(t), \psi(t))\) --- \textbf{параметрическое} решение такого уравнения на \(\alpha, \beta\), если:
\begin{enumerate}
    \item \(\varphi, \psi\in C^{1}(\alpha, \beta)\) и \(r'(t)\neq 0\) на \(t\in (\alpha, \beta)\)
    \item \(P(\varphi(t), \psi(t)) + Q(\varphi(t), \psi(t))\psi'(t) \equiv 0\) на \(t\in (\alpha, \beta)\)
\end{enumerate}

\begin{example}
    \[xdx + ydy = 0\]

    Подстановкой тривиально можно убедиться, что \(y = \sqrt{C^2 - x^2}\) --- решение этого уравнения.

    Параметрическое решение \((C\cos t, C\sin t)\)
\end{example}

\subsection*{Особые точки уравнения в дифференциалах.}

\((x_0, y_0)\) --- особая, если \(P(x_0, y_0) + Q(x_0, y_0) = 0\)

\begin{example}
    \[xdx + ydy = 0\]

    Особая точка \((0, 0)\), через нее ничто не проходит.
\end{example}

\subsection*{Геометрический смысл уравнения в дифференциалах и его решения.}

Пусть \(r = (x(t), y(t))\) есть параметрическое решение уравнения на \((\alpha, \beta)\). Тогда при \(t\in(\alpha, \beta)\):
\[P(x(t), y(t))x'(t) + Q(x(t), y(t))y'(t) = 0\]
\[F(r(t))r'(t) = 0\]
Таким образом, любая интегральная кривая в каждой своей точке перпендикулярная вектору \(F(x, y)\)

\subsection*{Задача Коши (ЗК) для уравнения 1-го порядка, разрешённого относительно производной.}

Задача Коши --- задача поиска решения уравнения, удовлетворяющему \(y(x_0) = y_0\).

\begin{theorem}
    \(G\subset\R^2\) --- область, \(f\in C(G), (x_0, y_0)\in G\). Тогда в некоторой окрестности \(x_0\) существует решение задачи Коши.
\end{theorem}
\begin{theorem}
    Как в предыдущей теореме, но \(f'_y\in C(G)\). Тогда решение задачи Коши единственно.
\end{theorem}

Таким образом, может быть такое, что в некоторых \textit{(или всех)} точках решение не единственно.

\subsection*{Особое решение уравнения.}

Это решение уравнения, в каждой точке которого нарушается локальная единственность решения задачи Коши.

\begin{example}
    \[y' = \sqrt[3]{y^2}\]
    Тогда особое решение \(y' \equiv 0\), его в любой точке \((x_0, 0)\) пересекает решение вида \(y = (x - x_0)^3 / 3\)
\end{example}

\subsection*{Однородное уравнение.}

Функция однородна степени \(\alpha\), если \(\forall t,x,y \ \ F(tx, ty) = t^\alpha F(x, y)\)

Однородное уравнение --- уравнение вида
\[P(x, y)dx + Q(x, y)dy = 0\]
, где \(P\) и \(Q\) однородные функции одной степени.

Замена \(z = \frac{y}{x}\) сводит это уравнение к уравнению с разделяющимися переменными.

\subsection*{Геометрическое свойство решений однородного уравнения.}

Пусть \(x = \varphi(t), y = \psi(t)\) --- параметрическое решение однородного дифура. Растянем пространство в \(\lambda\) раз, получим \(x = \lambda \varphi(t), y = \lambda \psi(t)\). При подстановке получим:
\[P(\lambda\varphi, \lambda\psi) \lambda\varphi' + Q(\lambda\varphi, \lambda\psi)\lambda\psi' = 0\]
По однородности:
\[P(\varphi, psi) \varphi' + Q(\varphi, \psi)\psi' = 0\]

Таким образом, любое растяжение \textit{(или сжатие)} решения однородного уравнения приводит к другому решению однородного уравнения.

\subsection*{Уравнение Бернулли.}

Это уравнение вида
\[y' = p(x) y + q(x) y^\alpha, \alpha\in\R\setminus \{0,1\}\]

Поделив на \(y^\alpha\) и заменив \(z = y^{1 - \alpha}\), получаем линейное.

\subsection*{Уравнение Риккати.}

\[y' = p(x) y^2 + q(x)y + r(x)\]

Оно решается только в особых случаях \textit{(например, \(\alpha = 2\))}, но если нашел какое-то решение \(\varphi\), то замена \(y = z + \varphi\) сводит к Бернулли.

\subsection*{Уравнение в полных дифференциалах.}

Это уравнение вида
\[P(x, y) dx + Q(x, y)dy = 0\]
, при этом
\[\exists u : du = P(x, y) dx + Q(x, y) dy\]

Решение имеет вид \(u(x, y) = C\)

Обязательное условие на существование \(u\) это \(P'_y = Q'_x\). Если при этом \(P, Q\in C^1(G)\) и \(G\) односвязна, то это условие еще и достаточно.

Если область прямоугольная, то можно решить систему \(\begin{cases}
    u_x' = P \\
    u_y' = Q
\end{cases}\) следующим образом: Решаем первое уравнение при фиксированном \(y\), после чего заменяем \(C = C(y)\) и находим \(C\) как функцию.

В таком случае \(u\) есть потенциал векторного поля \((P, Q)\).

\subsection*{Интегрирующий множитель.}

Это то, на что мы домножаем уравнение, чтобы получить уравнение в полных дифференциалах.

Если \(\mu\) --- инт. множитель, то
\[(\mu P)'_y = (\mu Q)'_x\]
, то есть
\[\mu'_y P - \mu'_x Q = (Q'_x - P'_y) \mu\]

Это сложно решить, но иногда решается при \(\mu'_x \equiv 0\) или \(\mu'_y \equiv 0\).

\subsection*{Уравнение n-го порядка и его решение.}

Это уравнение вида:
\[F(x, y, y', \dots , y^{(n)}) = 0\]

Его решение на \(a, b\) --- \(\varphi\), такое что:
\begin{enumerate}
    \item \(\varphi\in C^n(a, b)\)
    \item \(F(x, \varphi(x), \varphi'(x), \dots , \varphi^{(n)}(x)) \equiv 0\) на \((a, b)\)
\end{enumerate}

\subsection*{ЗК для уравнения, разрешённого относительно старшей производной.}

Это уравнение вида \(y^{(n)} = f(x, y, y', \dots , y^{(n - 1)})\).

Задача Коши для него имеет вид \(y(x_0) = y_0, y'(x_0) = y_1, \dots , y^{(n - 1)}(x_0) = y_{n - 1}\)

\subsection*{Методы понижения порядка уравнения.}

\begin{itemize}
    \item \(y^{(n)} = f(x) \implies y^{(n - 1)} = \int f(x) dx\)
    \item \(F(x, y^{(k)}, y^{(k + 1)}, \dots , y^{(n)}) \xRightarrow{z = y^{(k)}} F(x, z, \dots z^{(n - k)}) = 0\)
    \item \(F(y, y', \dots , y^{(n)}) = 0\). Тогда пусть \(z = y'\), \(y''_{xx} = z'_y z, y'''_{x x x} = z''_{yy} z^2 + z'_y{^2} z\) и т.д.
    \item Пусть \(F\) линейна по \(y\). Тога можно заменить \(z = y'/y\)
    \item \(F(x, y, y', \dots , y^{(n)}) = \cfrac{d}{dx} \Phi(x, y, y', \dots , y^{(n - 1)}) \Rightarrow \Phi(x, y, y', \dots , y^{(n - 1)}) = C\)
\end{itemize}

\subsection*{Нормальная система уравнений, её решение.}

Нормальная система порядка \(n\) это система вида:
\[\begin{cases}
        \dot{x}_1 = f_1(t, x_1, \dots x_n) \\
        \vdots                             \\
        \dot{x}_n = f_n(t, x_1, \dots x_n) \\
    \end{cases}\]

Можно ввести пару обозначений для краткости:
\[r = \begin{pmatrix} x_1 \\ \vdots \\ x_n \end{pmatrix} \quad f(t, r) = \begin{pmatrix} f_1(t, r) \\ \vdots \\ f_n(t, r) \end{pmatrix} \quad \dot{r} = f(t, r)\]

\(\varphi\) --- решение такой системы, если:
\begin{enumerate}
    \item \(\varphi \in C^1((a, b) \to \R^n)\)
    \item \(\dot \varphi(t) \equiv f(t, \varphi(t))\) на \((a, b)\)
\end{enumerate}

\subsection*{Интегральная кривая нормальной системы.}

Это график решения, но теперь он в \((n + 1)\)-мерном пространстве.

% \subsection*{Глобальное и локальное условие Липшица.}

% \subsection*{Приближения Пикара.}

% \subsection*{Сведение уравнения n-го порядка к равносильной системе.}

% \subsection*{Максимальное решение.}

% \subsection*{Определитель Вронского (решений ЛОС и ЛОУ) и его свойства.}

% \subsection*{Фундаментальная система решений.}

% \subsection*{Фундаментальная матрица.}

% \subsection*{Метод неопределённых коэффициентов для ЛС.}

% \subsection*{Характеристический многочлен ЛУ.}

% \subsection*{Метод неопределённых коэффициентов для ЛУ.}

% \subsection*{Автономная система.}

% \subsection*{Фазовое пространство автономной системы. Фазовая траектория, фазовый портрет, фазовая скорость, точка покоя.}

% \subsection*{Устойчивость по Ляпунову и асимптотическая устойчивость.}

% \subsection*{Функция Ляпунова.}

% \subsection*{Теорема об устойчивости по первому приближению.}


\end{document}
