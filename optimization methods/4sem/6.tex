\chapter{17 марта}

\subsubsection{Критерии Сильвестра проверки достаточных условий экстремума}

\begin{enumerate}
    \item Для того, чтобы \(\mathbf H(x^*) > 0\) и \(x^*\) являлась точкой локального минимума, необходимо и достаточно, чтобы \textbf{угловые} миноры были строго положительными, т.е. \(\Delta_1 > 0, \Delta_2 > 0 \dots \Delta_n > 0\).
    \item Для того, чтобы \(\mathbf H(x^*) < 0\) и \(x^*\) являлась точкой локального максимума, необходимо и достаточно, чтобы знаки \textbf{угловых} миноров чередовались, т.е. \(\Delta_1 < 0, \Delta_2 > 0 \dots ( - 1)^n\Delta_n > 0\)
\end{enumerate}

\subsubsection{Критерии Сильвестра проверки необходимых условий экстремума}

\begin{enumerate}
    \item Для того, чтобы \(\mathbf H(x^*) \geq 0\) и \(x^*\) мог быть точкой локального минимума, необходимо и достаточно, чтобы \textbf{главные} миноры были положительными, т.е. \(\Delta_1 \geq 0, \Delta_2 \geq 0, \dots \Delta^n \geq 0\)
    \item Для того, чтобы \(\mathbf H(x^*) \leq 0\) и \(x^*\) мог быть точкой локального максимума, необходимо и достаточно, чтобы знаки \textbf{главных} миноров чередовались, т.е. \(\Delta_1 \leq 0, \Delta_2 \geq 0, \dots ( - 1)^n \Delta^n \geq 0\)
\end{enumerate}

\begin{definition}
    \textbf{Собственные значения} \(\lambda_i\) матрицы \(\mathbb H(x^*)\) находятся как корни характеристического уравнения \(|H(x^*) - \lambda E| = 0\)
\end{definition}

Если \(H(x)\) --- вещественная, симметричная матрица, то \(\lambda_i\) тоже вещественные.

\subsection{Квадратичные функции}

\begin{definition}
    Функция вида
    \begin{equation}
        f(x) = \sum_{i = 1}^n \sum_{j = 1}^n a_{ij} x_i x_j + \sum_{j = 1}^n b_j x_j + c \label{квадратичная функция}
    \end{equation}
    называется \textbf{квадратичной функцией} \(n\) переменных.
\end{definition}

Положим \(a_{ij} = a_{ji}\)\footnote{На лекции было дано \(a_{ij} = a_{ij} + a_{ji}\), но это не похоже на правду, т.к. тогда \(a_{ji} = 0 \ \ \forall i, j\). Нулевая матрица действительно симметрична, но вряд ли это подразумевалось.}, тогда \(a_{ij}\) задаёт симметричную матрицу \(A\).
\begin{equation}
    f(x) = \frac{1}{2} \ev{Ax, x} + \ev{b, x} + c \label{квадратичная функция с симметричной матрицей}
\end{equation}
, где \(b = \begin{pmatrix} b_1 & \dots & b_n \end{pmatrix}^T \in E_n\) --- вектор коэффициентов, \(x = \begin{pmatrix} x_1 & \dots & x_n \end{pmatrix}^T\)

\begin{prop}[квадратичных функций]\itemfix
    \begin{enumerate}
        \item \(\nabla f(x) = Ax + b\)
              \begin{align*}
                  \frac{\partial f}{\partial x_k} & = \frac{\partial}{\partial x_k} \left( \frac{1}{2} \sum_{j = 1}^n a_{ij} x_i x_j + \sum_{j = 1}^n b_j x_j + c \right) \\
                                                  & = \frac{1}{2} \sum_{i = 1}^n (a_{ik} + a_{ki}) x_i + b_k                                                              \\
                                                  & = \sum_{i = 1}^n a_{ki} x_i + b_k
              \end{align*}

        \item \(\mathbf H(x) = A\)

              \[\frac{\partial^2 f}{\partial x_l \partial x_k} = \frac{\partial}{\partial x_l} \left( \frac{\partial f}{\partial x_k} \right) = \frac{\partial}{\partial x_l} \left( \sum_{i = 1}^n a_{ki} x_i + b_k \right) = a_{kl}\]

        \item Квадратичная функция \(f(x)\), для которой выполнено~\eqref{квадратичная функция с симметричной матрицей}, с положительно определенной матрицей \(A\) сильно выпуклая, т.к. \(\mathbf H(x) = A\) --- симметричная и положительная определенная, а следовательно \(\lambda_i > 0\) и \(\exists \) ортонормированный базис из собственных векторов этой матрицы. В этом базисе:

              \[A = \begin{pmatrix}
                      \lambda_1 & 0         & \dots  & 0         \\
                      0         & \lambda_2 & \dots  & 0         \\
                      \vdots    & \vdots    & \ddots & \vdots    \\
                      0         & 0         & \dots  & \lambda_n
                  \end{pmatrix} \quad A - lE = \begin{pmatrix}
                      \lambda_1 - l & 0             & \dots  & 0             \\
                      0             & \lambda_2 - l & \dots  & 0             \\
                      \vdots        & \vdots        & \ddots & \vdots        \\
                      0             & 0             & \dots  & \lambda_n - l
                  \end{pmatrix}\]

              В этом базисе все угловые миноры матрицы \(A\) и матрицы \(A - lE\) положительны при достаточно малом \(l : 0 < l < \lambda_{\min} \Rightarrow f\) сильно выпуклая.
    \end{enumerate}
\end{prop}

\subsection{Общие принципы многомерной оптимизации}

Алгоритмы многомерной оптимизации обычно используют итерационную процедуру, описываемую следующим образом: \(x^{k + 1} = \Phi(x^k, x^{k - 1} \dots x^0), x^0 \in E_n\). Эти алгоритмы строят последовательность промежуточных результатов \(\{x_k\}\), которая обладает следующими свойствами:
\begin{equation}
    \begin{cases}
        \lim\limits_{k \to +\infty} f(x^k) = f^* = \min\limits_{E_n} f(x), & \text{если \(U^* \neq \emptyset\)} \\
        \lim\limits_{k \to +\infty} f(x^k) = f^* = \inf\limits_{E_n} f(x), & \text{если \(U^* = \emptyset\)}
    \end{cases}
    \label{последовательность результатов}
\end{equation}
, где \(U^*\) --- множество точек глобального минимума функции \(f(x)\).

\begin{definition}
    Если для \(\{x^k\} \) выполняется условие~\eqref{последовательность результатов}, то эта последовательность называется \textbf{минимизирующей}.
\end{definition}

\begin{definition}
    Если для \(U^* \neq \emptyset\) выполняется условие \(\lim_{k \to +\infty} \rho(x^k, U^*) = 0\), то \(\{x^k\}\) \textbf{сходится} к множеству \(U^*\)
\end{definition}

\begin{definition}[расстояние от точки до множества]
    \(\rho(x, U) = \inf_{y\in U} \rho(x, y)\)
\end{definition}

Если \(U^*\) состоит из одной точки \(x^*\), то для \(\{x^k\}\), сходящейся к \(U^*\), \(\lim_{k \to +\infty} x^k = x^*\). Минимизирующая последовательность может и не сходиться к точке минимума.

\begin{theorem}[Вейерштрасса]
    Если \(f*x\) непрерывна в \(E_n\) и множество \(U^\alpha = \{x : f(x) \leq \alpha\} \) для некоторого \(\alpha\) непусто и ограничено, то \(f(x)\) достигает глобального минимума в \(E_n\).
\end{theorem}

\subsection{Скорость сходимости минимизирующей последовательности}

\begin{definition}
    \(\{x^k\} \) сходится к точке \(x^*\) \textbf{линейно} \textit{(со скоростью геометрической прогрессии)}, если
    \[\exists q \in (0, 1) : \rho(x^k, x^*) = q \rho(x^{k - 1}, x^*)\]
    , т.е. \(\rho(x^k, x^*) \leq q^k \rho(x^0, x^*)\)
\end{definition}

\begin{definition}
    Сходимость называется \textbf{сверхлинейной}, если
    \[\rho(x^k, x^*) \leq q_k \rho(x^{k - 1}, x^*)\]
    и \(q_k \to + 0\) при \(k \to +\infty\)
\end{definition}

\begin{definition}
    Сходимость называется \textbf{квадратичной}, если
    \[\rho(x^k, x^*) \leq (c \rho(x^{k - 1}, x^*))^2, c > 0\]
\end{definition}

Критерий окончания итерационного процесса:
\begin{enumerate}
    \item \(\rho(x^{k + 1}, x^k) < \varepsilon_1\)
    \item \(|f(x^{k + 1}) - f(x^k)| < \varepsilon_2\)
    \item \(||\nabla f(x^k)|| < \varepsilon_3\)
\end{enumerate}

\begin{equation}
    x^{k + 1} = x^k + \alpha_k p^k
    \label{итерационный процесс 2}
\end{equation}
, где \(p^k\) --- \textbf{направление поиска} из \(x^k\) в \(x^{k + 1}\), а \(\alpha_k\) --- \textbf{величина шага}. Алгоритмы, которые мы будем рассматривать, различаются этими двумя величинами.

\begin{definition}
    В итерационном процессе~\eqref{итерационный процесс 2} производится \textbf{исчерпывающий спуск}, если величина шага \(\alpha_k\) находится из решения одномерной задачи минимизации
    \begin{equation}
        \Phi_k(\alpha) \to \min_\alpha, \Phi_k(\alpha) = f(x^k + \alpha p^k)
        \label{исчерпывающий спуск}
    \end{equation}
\end{definition}

\begin{theorem}
    Если функция \(f(x)\) дифференцируема в \(E_n\), то в итерационном процессе~\eqref{итерационный процесс 2} с выбором шага с исчерпывающим спуском для любого \(k \geq 1\) выполняется следующее условие:
    \begin{equation}
        \ev{\nabla f(x^{k + 1}), p^k} = 0
        \label{условие при процессе}
    \end{equation}
\end{theorem}
\begin{proof} % первое за курс, ура!
    Для \(\Phi_k(\alpha)\) необходимое условие минимума функции:
    \[\frac{d \Phi_k(\alpha)}{d \alpha} = \sum_{j = 1}^n \frac{\partial f(x^{k + 1})}{\partial x_j} \frac{d x_j^{k + 1}}{d \alpha} = 0\]
    Учитывая, что \(x^{k + 1}_j = x_j^k + \alpha p_j^k\), получаем, что \(\frac{d x_j^{k + 1}}{d \alpha} = p_j^k\)
\end{proof}

\begin{theorem}
    Для квадратичной функции \(f(x) = \frac{1}{2} \ev{Ax, x} + \ev{b, x} + c\) величина \(\alpha_k\) исчерпывающего спуска в итерационном процессе~\eqref{итерационный процесс 2} равна:
    \begin{equation}
        \alpha_k = - \frac{\ev{\nabla f(x^k), p^k}}{\ev{A p^k, p^k}} = - \frac{\ev{Ax^k + b, p^k}}{\ev{Ap^k, p^k}}
        \label{alpha k в исчерпывающем спуске}
    \end{equation}
\end{theorem}
\begin{proof}
    \begin{align*}
        x^{k + 1}           & = x^k + \alpha_k p^k             \\
        Ax^{k + 1} + b      & = Ax^k + b + \alpha_k A p^k      \\
        \nabla f(x^{k + 1}) & = \nabla f(x^k) + \alpha_k A p^k
    \end{align*}
    \begin{align*}
        \ev{\nabla f(x^{k + 1}), p^k}                      & = 0                                                 \\
        \ev{\nabla f(x^k) + \alpha_k A p^k, p^k}           & = 0                                                 \\
        \ev{\nabla f(x^k), p^k} + \ev{\alpha_k A p^k, p^k} & = 0                                                 \\
        \ev{\nabla f(x^k), p^k} + \alpha_k \ev{A p^k, p^k} & = 0                                                 \\
        \alpha_k                                           & = - \frac{\ev{\nabla f(x^k), p^k}}{\ev{A p^k, p^k}}
    \end{align*}
\end{proof}
