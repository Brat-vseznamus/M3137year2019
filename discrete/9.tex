\documentclass[12pt, a4paper]{article}

\usepackage{lastpage}
\usepackage{mathtools}
\usepackage{xltxtra}
\usepackage{libertine}
\usepackage{amsmath}
\usepackage{amsthm}
\usepackage{amsfonts}
\usepackage{amssymb}
\usepackage{enumitem}
\usepackage{xcolor}
\usepackage[left=1.5cm, right=1.5cm, top=2cm, bottom=2cm, bindingoffset=0cm, headheight=15pt]{geometry}
\usepackage{fancyhdr}
\usepackage[russian]{babel}
% \usepackage[utf8]{inputenc}
\usepackage{catchfilebetweentags}
\usepackage{accents}
\usepackage{calc}
\usepackage{etoolbox}
\usepackage{mathrsfs}
\usepackage{wrapfig}

\providetoggle{useproofs}
\settoggle{useproofs}{false}

\pagestyle{fancy}
\lfoot{M3137y2019}
\rhead{\thepage\ из \pageref{LastPage}}

\newcommand{\R}{\mathbb{R}}
\newcommand{\Q}{\mathbb{Q}}
\newcommand{\C}{\mathbb{C}}
\newcommand{\Z}{\mathbb{Z}}
\newcommand{\B}{\mathbb{B}}
\newcommand{\N}{\mathbb{N}}

\newcommand{\const}{\text{const}}

\newcommand{\teormin}{\textcolor{red}{!}\ }

\DeclareMathOperator*{\xor}{\oplus}
\DeclareMathOperator*{\equ}{\sim}
\DeclareMathOperator{\Ln}{\text{Ln}}
\DeclareMathOperator{\sign}{\text{sign}}
\DeclareMathOperator{\Sym}{\text{Sym}}
\DeclareMathOperator{\Asym}{\text{Asym}}
% \DeclareMathOperator{\sh}{\text{sh}}
% \DeclareMathOperator{\tg}{\text{tg}}
% \DeclareMathOperator{\arctg}{\text{arctg}}
% \DeclareMathOperator{\ch}{\text{ch}}

\DeclarePairedDelimiter{\ceil}{\lceil}{\rceil}
\DeclarePairedDelimiter{\abs}{\left\lvert}{\right\rvert}

\setmainfont{Linux Libertine}

\theoremstyle{plain}
\newtheorem{axiom}{Аксиома}
\newtheorem{lemma}{Лемма}

\theoremstyle{remark}
\newtheorem*{remark}{Примечание}
\newtheorem*{exercise}{Упражнение}
\newtheorem*{consequence}{Следствие}
\newtheorem*{example}{Пример}
\newtheorem*{observation}{Наблюдение}

\theoremstyle{definition}
\newtheorem{theorem}{Теорема}
\newtheorem*{definition}{Определение}
\newtheorem*{obozn}{Обозначение}

\setlength{\parindent}{0pt}

\newcommand{\dbltilde}[1]{\accentset{\approx}{#1}}
\newcommand{\intt}{\int\!}

% magical thing that fixes paragraphs
\makeatletter
\patchcmd{\CatchFBT@Fin@l}{\endlinechar\m@ne}{}
  {}{\typeout{Unsuccessful patch!}}
\makeatother

\newcommand{\get}[2]{
    \ExecuteMetaData[#1]{#2}
}

\newcommand{\getproof}[2]{
    \iftoggle{useproofs}{\ExecuteMetaData[#1]{#2proof}}{}
}

\newcommand{\getwithproof}[2]{
    \get{#1}{#2}
    \getproof{#1}{#2}
}

\newcommand{\import}[3]{
    \subsection{#1}
    \getwithproof{#2}{#3}
}

\newcommand{\given}[1]{
    Дано выше. (\ref{#1}, стр. \pageref{#1})
}

\renewcommand{\ker}{\text{Ker }}
\newcommand{\im}{\text{Im }}
\newcommand{\grad}{\text{grad}}

\lhead{Конспект по дискретной математике}
\cfoot{}
\rfoot{October 29, 2019}

\begin{document}
\section{Избыточное кодирование}

Избыточное кодирование позволяет восстановить информацию, даже если часть кода была потеряна.

В избыточном кодировании обычно используют код фиксированной длины, так как код переменной длины сделать избыточным крайне сложно.

$c: \Sigma \to \B^k$

\begin{definition}
    Расстояние Хемминга

    $x, y$ --- строки одинаковой длины.

    $H(x,y)=|\{i\ |\ x[i]\not = y[i]\}|$

    $H(001001, 111000)=3$
\end{definition}

\begin{definition}
    $c$ {\bf обнаруживает} $d$ ошибок, если $\forall a,b\in\Sigma, a\not=b \ \ H(c(a),c(b)) > d$
\end{definition}

\begin{definition}
    $c$ {\bf исправляет} $d$ ошибок, если $\forall a,b\in\Sigma, a\not=b \ \ H(c(a),c(b)) > 2d$
\end{definition}

\begin{definition}
    $\rho : X\times X\to \R^+$ --- {\bf расстояние}, если $\rho$ удовлетворяет следующим аксиомам:
    \begin{enumerate}
        \item $\rho(x,y)\Leftrightarrow x=y$
        \item $\rho(x,y) = \rho(y,x)$
        \item $\rho(x,y) + \rho(y,z) \geq \rho(x,z)$
    \end{enumerate}
\end{definition}

\begin{lemma}
    $H$ --- расстояние.
\end{lemma}

$c$ --- исправляет $d$ ошибок, тогда $x=c(a), x\mapsto y$, изменив $\leq d$ битов.

\begin{lemma}
    $\forall d, \forall \Sigma \ \ \exists$ код, исправляющий $d$ ошибок.
\end{lemma}
\begin{proof}
    $|\Sigma| = n$
    
    $k : 2^k\geq n$

    $\sphericalangle c_1(a) = $ строка длины $k$, соответствующая номеру $a$ в алфавите $\Sigma$

    $\sphericalangle c(a) = c_1(a)c_1(a)\ldots c_1(a)$, всего $2\alpha + 1$ раз.

    Этот код исправляет $d$ ошибок, почему --- хз. Не откажусь от доказательства.
\end{proof}

\begin{definition}
    {\bf Шаром} радиуса $r$ с центром $x$ называетcя $B_r(x)=\{y|\rho(x,y)\leq r\}$
\end{definition}

\begin{definition}
    {\bf Булевым шаром} называется шар, в котором $x,y\in \B^n$
\end{definition}

\begin{definition}
    Объем булева шара --- число векторов, которые в нем содержатся.
\end{definition}

$|B_r(x)|=1+n+C_n^2+\ldots + C_n^r=\sum\limits_{i=0}^{r}C_n^i$

\begin{lemma}
    Если код $c$ обнаруживает $d$ ошибок, то шары радиуса $d$ с центрами в кодовых словах не содержат других кодовых слов.
\end{lemma}
\begin{lemma}
    Если код $c$ исправляет $d$ ошибок, то шары радиуса $d$ с центрами в кодовых словах не пересекаются.
\end{lemma}

\begin{theorem}
    {\bf Граница Хемминга}

    Для $\Sigma$, содержащего $s$ символов, построен код $c:\Sigma\to\B^n$, исправляющий $d$ ошибок.

    Тогда $$2^n\geq s\sum\limits_{i=0}^d C_n^i$$, в частности для $d=1 \quad 2^n\geq s(n+1)$
\end{theorem}

\subsection{Код Хэмминга {\it (оптимальное кодирование для $d=1$)}}

$s=2^k$

Занумеруем все биты от $1$ до $n$.

Все биты либо контрольные, либо информационные. Возьмём $2^i$-тые биты как контрольные, остальные --- информационные. Всего берём столько битов, чтобы набралось $k$ информационных битов.

Например для $k=7:$ $cci_1ci_2i_3i_4ci_5i_6i_7$. Ассимптотически контрольных битов $\log$.

Контрольный бит с номером $2^i$ задается так, чтобы $\xor\limits_{\substack{j=1\ldots n \\ j\&2^i \not= 0}} x[j]=0$

Проверка смотрит, что нужный $\oplus=0$. Все $i$, для которых это не выполняется, суммируются. Бит на полученной позиции был потерян, его нужно поменять.

\begin{theorem}
    Код Хэмминга исправляет одну ошибку.
\end{theorem}
\begin{proof}
    Докажем, что  $\forall a,b\in\Sigma, a\not=b \ \ H(c(a),c(b)) > 2$

    Рассмотрим строку с одним различающимся разрядом $j$. Тогда различаются хотя бы два контрольных бита, т.к. в двоичном представлении $j$ есть хотя бы две единицы.

    Рассмотрим строку с двумя различающимся разрядами $j$ и $k$. Тогда различается хотя бы один контрольный бит, хз почему.
\end{proof}

Найдём ассимптотику.

Пусть всего есть $n$ бит, взяли $\log n$ контрольных и $n-\log n$ информационных.

$$S\~{} 2^{n-\log n} \~{} \frac{2^n}{n}$$

\begin{definition}
    {\bf Линейный код} $c(a)\oplus c(b)=c(p)$
\end{definition}

\begin{lemma}
    $H(x\xor z, y\xor z)=H(x,y)$
\end{lemma}

$H(c(a), c(b)) = H(c(a)\xor c(a), c(a)\xor c(b))=H(0, c(p))=\omega(c(p))$

\begin{lemma}
    Код Хемминга --- линейный
\end{lemma}

\end{document}