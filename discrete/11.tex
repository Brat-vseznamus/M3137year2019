\documentclass[12pt, a4paper]{article}

\usepackage{lastpage}
\usepackage{mathtools}
\usepackage{xltxtra}
\usepackage{libertine}
\usepackage{amsmath}
\usepackage{amsthm}
\usepackage{amsfonts}
\usepackage{amssymb}
\usepackage{enumitem}
\usepackage{xcolor}
\usepackage[left=1.5cm, right=1.5cm, top=2cm, bottom=2cm, bindingoffset=0cm, headheight=15pt]{geometry}
\usepackage{fancyhdr}
\usepackage[russian]{babel}
% \usepackage[utf8]{inputenc}
\usepackage{catchfilebetweentags}
\usepackage{accents}
\usepackage{calc}
\usepackage{etoolbox}
\usepackage{mathrsfs}
\usepackage{wrapfig}

\providetoggle{useproofs}
\settoggle{useproofs}{false}

\pagestyle{fancy}
\lfoot{M3137y2019}
\rhead{\thepage\ из \pageref{LastPage}}

\newcommand{\R}{\mathbb{R}}
\newcommand{\Q}{\mathbb{Q}}
\newcommand{\C}{\mathbb{C}}
\newcommand{\Z}{\mathbb{Z}}
\newcommand{\B}{\mathbb{B}}
\newcommand{\N}{\mathbb{N}}

\newcommand{\const}{\text{const}}

\newcommand{\teormin}{\textcolor{red}{!}\ }

\DeclareMathOperator*{\xor}{\oplus}
\DeclareMathOperator*{\equ}{\sim}
\DeclareMathOperator{\Ln}{\text{Ln}}
\DeclareMathOperator{\sign}{\text{sign}}
\DeclareMathOperator{\Sym}{\text{Sym}}
\DeclareMathOperator{\Asym}{\text{Asym}}
% \DeclareMathOperator{\sh}{\text{sh}}
% \DeclareMathOperator{\tg}{\text{tg}}
% \DeclareMathOperator{\arctg}{\text{arctg}}
% \DeclareMathOperator{\ch}{\text{ch}}

\DeclarePairedDelimiter{\ceil}{\lceil}{\rceil}
\DeclarePairedDelimiter{\abs}{\left\lvert}{\right\rvert}

\setmainfont{Linux Libertine}

\theoremstyle{plain}
\newtheorem{axiom}{Аксиома}
\newtheorem{lemma}{Лемма}

\theoremstyle{remark}
\newtheorem*{remark}{Примечание}
\newtheorem*{exercise}{Упражнение}
\newtheorem*{consequence}{Следствие}
\newtheorem*{example}{Пример}
\newtheorem*{observation}{Наблюдение}

\theoremstyle{definition}
\newtheorem{theorem}{Теорема}
\newtheorem*{definition}{Определение}
\newtheorem*{obozn}{Обозначение}

\setlength{\parindent}{0pt}

\newcommand{\dbltilde}[1]{\accentset{\approx}{#1}}
\newcommand{\intt}{\int\!}

% magical thing that fixes paragraphs
\makeatletter
\patchcmd{\CatchFBT@Fin@l}{\endlinechar\m@ne}{}
  {}{\typeout{Unsuccessful patch!}}
\makeatother

\newcommand{\get}[2]{
    \ExecuteMetaData[#1]{#2}
}

\newcommand{\getproof}[2]{
    \iftoggle{useproofs}{\ExecuteMetaData[#1]{#2proof}}{}
}

\newcommand{\getwithproof}[2]{
    \get{#1}{#2}
    \getproof{#1}{#2}
}

\newcommand{\import}[3]{
    \subsection{#1}
    \getwithproof{#2}{#3}
}

\newcommand{\given}[1]{
    Дано выше. (\ref{#1}, стр. \pageref{#1})
}

\renewcommand{\ker}{\text{Ker }}
\newcommand{\im}{\text{Im }}
\newcommand{\grad}{\text{grad}}

\lhead{Конспект по дискретной математике}
\cfoot{}
\rfoot{\today}

\usepackage{listings}
% \usepackage{courier}

\lstset{basicstyle=\ttfamily\footnotesize, breaklines=true}

\begin{document}
\section{Генерация}
\subsection{Все комбинаторные объекты размера $n$}

Рекурсивная процедура генерации ``$gen(prefix)$''.

\begin{lstlisting}[frame=single, language=C, columns=fullflexible]
gen(prefix):
    if (prefix - к.о.)
        print(prefix)
    for (c - возможный атом)
        newp = prefix + [c]
        if (newp - корр. префикс к.о.)
            gen(newp)
\end{lstlisting}

Для $\B^n$

\begin{lstlisting}[frame=single, language=Python, columns=fullflexible]
gen(prefix)
    if len(prefix) == n:
        print(*prefix)
    else:
        for c in range(2):
            newp = prefix+[c]
            gen(newp)
\end{lstlisting}

\begin{lstlisting}[frame=single, language=Python, columns=fullflexible]
gen(p):
    if p == n:
        print(*a)
        return
    for c=0..1:
        a[p]=c
        gen(p+1)
\end{lstlisting}

\subsection{Перестановки размера $n$}
\begin{lstlisting}[frame=single, language=Python, columns=fullflexible]
gen(p):
    if p == n:
        print(a)
        return
    for c=1..n
        a[p]=c
        if !used[c]:
            a[p] = c
            used[c] = true
            gen(p+1)
            used[c] = false
\end{lstlisting}
\begin{lstlisting}[frame=single, language=Python, columns=fullflexible]
gen(p):
    if p == k:
        print(a)
        return
    for c=1..n
        if (p == 0 or c > a[p - 1]) and (n - c) >= k - (p + 1)
            a[p] = c
            gen(p + 1)
\end{lstlisting}
\subsection{Разбиения на слагаемые}
\begin{lstlisting}[frame=single, language=Python, columns=fullflexible]
gen(p, sum):
    if sum == 0:
        print(a[0:p])
    else:
        for c=min(p==0?n:a[p-1], sum)..1:
            a[p]=c
            gen(p+1, sum-c)
\end{lstlisting}
\subsection{Правильная скобочная последовательность}
``'' --- правильная скобочная последовательность

$A$ --- ПСП $\Rightarrow (A)$ --- ПСП

$A,B$ --- ПСП $\Rightarrow AB$ --- ПСП
\begin{lstlisting}[frame=single, language=Python, columns=fullflexible]
gen(p, bal):
    if p == 2*n:
        print(a)
        return
    if 2*n-p-1 >= bal+1:
        a[p] = (
        gen(p + 1, bal + 1)
    if bal > 0
        a[p] = )
        gen(p + 1, bal - 1)
\end{lstlisting}

\section{Нумерация}

Номер объекта в нумерации с $0$ равен количеству меньших объектов.

\begin{lstlisting}[frame=single, language=Python, columns=fullflexible]
\end{lstlisting}
\end{document}