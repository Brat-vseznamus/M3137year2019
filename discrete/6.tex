\documentclass[12pt, a4paper]{article}

\usepackage{lastpage}
\usepackage{mathtools}
\usepackage{xltxtra}
\usepackage{libertine}
\usepackage{amsmath}
\usepackage{amsthm}
\usepackage{amsfonts}
\usepackage{amssymb}
\usepackage{enumitem}
\usepackage{xcolor}
\usepackage[left=1.5cm, right=1.5cm, top=2cm, bottom=2cm, bindingoffset=0cm, headheight=15pt]{geometry}
\usepackage{fancyhdr}
\usepackage[russian]{babel}
% \usepackage[utf8]{inputenc}
\usepackage{catchfilebetweentags}
\usepackage{accents}
\usepackage{calc}
\usepackage{etoolbox}
\usepackage{mathrsfs}
\usepackage{wrapfig}

\providetoggle{useproofs}
\settoggle{useproofs}{false}

\pagestyle{fancy}
\lfoot{M3137y2019}
\rhead{\thepage\ из \pageref{LastPage}}

\newcommand{\R}{\mathbb{R}}
\newcommand{\Q}{\mathbb{Q}}
\newcommand{\C}{\mathbb{C}}
\newcommand{\Z}{\mathbb{Z}}
\newcommand{\B}{\mathbb{B}}
\newcommand{\N}{\mathbb{N}}

\newcommand{\const}{\text{const}}

\newcommand{\teormin}{\textcolor{red}{!}\ }

\DeclareMathOperator*{\xor}{\oplus}
\DeclareMathOperator*{\equ}{\sim}
\DeclareMathOperator{\Ln}{\text{Ln}}
\DeclareMathOperator{\sign}{\text{sign}}
\DeclareMathOperator{\Sym}{\text{Sym}}
\DeclareMathOperator{\Asym}{\text{Asym}}
% \DeclareMathOperator{\sh}{\text{sh}}
% \DeclareMathOperator{\tg}{\text{tg}}
% \DeclareMathOperator{\arctg}{\text{arctg}}
% \DeclareMathOperator{\ch}{\text{ch}}

\DeclarePairedDelimiter{\ceil}{\lceil}{\rceil}
\DeclarePairedDelimiter{\abs}{\left\lvert}{\right\rvert}

\setmainfont{Linux Libertine}

\theoremstyle{plain}
\newtheorem{axiom}{Аксиома}
\newtheorem{lemma}{Лемма}

\theoremstyle{remark}
\newtheorem*{remark}{Примечание}
\newtheorem*{exercise}{Упражнение}
\newtheorem*{consequence}{Следствие}
\newtheorem*{example}{Пример}
\newtheorem*{observation}{Наблюдение}

\theoremstyle{definition}
\newtheorem{theorem}{Теорема}
\newtheorem*{definition}{Определение}
\newtheorem*{obozn}{Обозначение}

\setlength{\parindent}{0pt}

\newcommand{\dbltilde}[1]{\accentset{\approx}{#1}}
\newcommand{\intt}{\int\!}

% magical thing that fixes paragraphs
\makeatletter
\patchcmd{\CatchFBT@Fin@l}{\endlinechar\m@ne}{}
  {}{\typeout{Unsuccessful patch!}}
\makeatother

\newcommand{\get}[2]{
    \ExecuteMetaData[#1]{#2}
}

\newcommand{\getproof}[2]{
    \iftoggle{useproofs}{\ExecuteMetaData[#1]{#2proof}}{}
}

\newcommand{\getwithproof}[2]{
    \get{#1}{#2}
    \getproof{#1}{#2}
}

\newcommand{\import}[3]{
    \subsection{#1}
    \getwithproof{#2}{#3}
}

\newcommand{\given}[1]{
    Дано выше. (\ref{#1}, стр. \pageref{#1})
}

\renewcommand{\ker}{\text{Ker }}
\newcommand{\im}{\text{Im }}
\newcommand{\grad}{\text{grad}}

\title{Конспект по дискретной математике}

\date{October 15, 2019}

\begin{document}

\maketitle

\section{Оценка числа элементов в схеме}

\begin{theorem}
$B_1$ и $B_2 \ \ \exists c \ \ \forall f \ \ size_{B_1}(f)\leq c\cdot size_{B_2}$
\end{theorem} Доказывалась ранее.

\begin{theorem}
О нижней оценке. Почти все функции требуют $\Omega(\frac{2^n}{n})$ элементов в своей записи. Альтернативная формулировка: $$f(n)=\frac{2^n}{n} \quad g(n): \frac{g}{f}\to 0 \quad F_g=\{\text{булевы функции, } size\leq g(n)\}$$ Тогда $$\frac{|F_g|}{2^{2^n}}\to 0$$
\end{theorem}

\begin{theorem}
О верхней оценке. $$\forall f \text{ --- бул. ф. } \exists \text{схема из ф.э., содержащая } O\left(\frac{2^n}{n}\right) \text{ элементов}$$
\end{theorem}

\section{Линейная программа}

Пример для $x_1\xor x_2$:

\noindent
$y_1=\neg x_1\\$
$y_2=\neg x_2\\$
$y_3=x_1\wedge y_2\\$
$y_4=x_2\wedge y_1\\$
$y_5=y_3\vee y_4\\$

Линейная программа --- нумерованное множество строк вида $$(a, [i_1\ldots i_k])\text{, где } a\in B \textit{(базис)}, i_j \text{ --- индексы переменных}, a:\mathbb{B}^k\to\mathbb{B}$$

\begin{theorem}
Для $f \exists $ линейная программа длины $r \Leftrightarrow \exists$ схема из $r$ функциональных элементов.
\end{theorem}

Оценка: сколько линейных программ над $\{\downarrow\}$ длины $r$? \\
Первая строка: $n^2$ вариантов \textit{(выбор 2 объектов из $n$)} \\
Вторая строка: $(n+1)^2$ варинатов \textit{(выбор 2 объектов из $n$ и $y_1$)} \\
$\vdots$ \\
$(n + r - 1)^{2}$

$$K_{n,r}=\prod\limits_{i=0}^{r-1}(n+i)^2\leq(n+r)^{2r}$$
$$\log_2 K_{n,r}\leq \log_2 (n+r)^{2r}=2r\log_2 (n+r)$$

\begin{lemma}
$$\exists \text{ функция: } size_B(f)\geq\frac{2^n}{2n}$$
\end{lemma}
\begin{proof}
Предположим противное:
$$r<\frac{2^n}{2n}$$
$$\log_2 K_{n,r}\leq 2r\log_2(n+r)<\frac{2\cdot 2^n}{2n}\log_2(n+\frac{2^n}{2n})\leq \frac{2^n}{n}\log_2 2^n=2^n$$

$$\Rightarrow K_{n,r}<2^{2^n} !!!$$

Обощим для произвольного $c$:

$$r<\frac{2^n}{2cn}$$
$$\log_2 K_{n,r}\leq 2r\log_2(n+r)<\frac{2\cdot 2^n}{2cn}\log_2(n+\frac{2^n}{2cn})\leq \frac{2^n}{cn}\log_2 2^n=\frac{2^n}{c}$$

$$\Rightarrow K_{n,r}<2^{\frac{2^n}{c}} !!!$$
\end{proof}

Вернемся к доказательству теоремы 2.

\begin{proof}
% $$g: \frac{g}{\frac{2^n}{n}}\to 0 \Leftrightarrow \forall \varepsilon>0 \ \ \exists n_0: n>n_0 \ \ \frac{g(n)}{\frac{2^n}{n}}<\varepsilon$$

$$\frac{|F_g|}{2^{2^n}}\to 0$$
$$\frac{2^{\frac{2^n}{c}}}{2^{2^n}}=2^{2^n\cdot (\frac{1}{n}-1)}\to 0$$
\end{proof}

\begin{theorem}
$\forall f \ \ \exists \text{схема из функ.эл.} O(\frac{2^n}{n})$
\end{theorem}

\end{document}