\documentclass[12pt, a4paper]{article}

\usepackage{lastpage}
\usepackage{mathtools}
\usepackage{xltxtra}
\usepackage{libertine}
\usepackage{amsmath}
\usepackage{amsthm}
\usepackage{amsfonts}
\usepackage{amssymb}
\usepackage{enumitem}
\usepackage{xcolor}
\usepackage[left=1.5cm, right=1.5cm, top=2cm, bottom=2cm, bindingoffset=0cm, headheight=15pt]{geometry}
\usepackage{fancyhdr}
\usepackage[russian]{babel}
% \usepackage[utf8]{inputenc}
\usepackage{catchfilebetweentags}
\usepackage{accents}
\usepackage{calc}
\usepackage{etoolbox}
\usepackage{mathrsfs}
\usepackage{wrapfig}

\providetoggle{useproofs}
\settoggle{useproofs}{false}

\pagestyle{fancy}
\lfoot{M3137y2019}
\rhead{\thepage\ из \pageref{LastPage}}

\newcommand{\R}{\mathbb{R}}
\newcommand{\Q}{\mathbb{Q}}
\newcommand{\C}{\mathbb{C}}
\newcommand{\Z}{\mathbb{Z}}
\newcommand{\B}{\mathbb{B}}
\newcommand{\N}{\mathbb{N}}

\newcommand{\const}{\text{const}}

\newcommand{\teormin}{\textcolor{red}{!}\ }

\DeclareMathOperator*{\xor}{\oplus}
\DeclareMathOperator*{\equ}{\sim}
\DeclareMathOperator{\Ln}{\text{Ln}}
\DeclareMathOperator{\sign}{\text{sign}}
\DeclareMathOperator{\Sym}{\text{Sym}}
\DeclareMathOperator{\Asym}{\text{Asym}}
% \DeclareMathOperator{\sh}{\text{sh}}
% \DeclareMathOperator{\tg}{\text{tg}}
% \DeclareMathOperator{\arctg}{\text{arctg}}
% \DeclareMathOperator{\ch}{\text{ch}}

\DeclarePairedDelimiter{\ceil}{\lceil}{\rceil}
\DeclarePairedDelimiter{\abs}{\left\lvert}{\right\rvert}

\setmainfont{Linux Libertine}

\theoremstyle{plain}
\newtheorem{axiom}{Аксиома}
\newtheorem{lemma}{Лемма}

\theoremstyle{remark}
\newtheorem*{remark}{Примечание}
\newtheorem*{exercise}{Упражнение}
\newtheorem*{consequence}{Следствие}
\newtheorem*{example}{Пример}
\newtheorem*{observation}{Наблюдение}

\theoremstyle{definition}
\newtheorem{theorem}{Теорема}
\newtheorem*{definition}{Определение}
\newtheorem*{obozn}{Обозначение}

\setlength{\parindent}{0pt}

\newcommand{\dbltilde}[1]{\accentset{\approx}{#1}}
\newcommand{\intt}{\int\!}

% magical thing that fixes paragraphs
\makeatletter
\patchcmd{\CatchFBT@Fin@l}{\endlinechar\m@ne}{}
  {}{\typeout{Unsuccessful patch!}}
\makeatother

\newcommand{\get}[2]{
    \ExecuteMetaData[#1]{#2}
}

\newcommand{\getproof}[2]{
    \iftoggle{useproofs}{\ExecuteMetaData[#1]{#2proof}}{}
}

\newcommand{\getwithproof}[2]{
    \get{#1}{#2}
    \getproof{#1}{#2}
}

\newcommand{\import}[3]{
    \subsection{#1}
    \getwithproof{#2}{#3}
}

\newcommand{\given}[1]{
    Дано выше. (\ref{#1}, стр. \pageref{#1})
}

\renewcommand{\ker}{\text{Ker }}
\newcommand{\im}{\text{Im }}
\newcommand{\grad}{\text{grad}}

\title{Конспект по дискретной математике}
\date{September 24, 2019}

\begin{document}
\maketitle

\begin{definition}
Арифметический базис --- $\oplus, \wedge, 1$
\end{definition}

Полином Жегалкина

Пример: $f(x,y,z)=(x\oplus y)\wedge(x\oplus x\wedge y\oplus 1)\leftrightarrow (x+y)(x+xy+1)
= x^2+x^2y+x+xy+xy^2+y$. Можно заметить, что $x^2=x (mod\ 2)$.
$x^2+x^2y+x+xy+xy^2+y=x+xy+x+xy+xy=xy+y\leftrightarrow x\wedge y\oplus y$

\begin{definition}
Приведенный полином Жегалкина: $\oplus_{\alpha_1,\alpha_2\ldots\alpha_n} x_1^{\alpha_1},
x_2^{\alpha_2}\ldots x_n^{\alpha_n}$
\end{definition}

\begin{theorem}
Любая булева функция, кроме $0$, имеет представление в виде приведенного полинома Жегалкина,
и только одно.
\end{theorem}

\begin{proof}
Существование тривиально - любую функцию записываем в арифметическом базисе и создаем
приведенный полином Жегалкина. Докажем единственность. Всего существует $2^{2^n}-1$ функций
от $n$ аргументов, кроме $0$. Кроме того, столько же существует полиномов Жегалкина от
$n$ аргументов. Если некоторой функции соответствует больше чем один полином Жегалкина, то
некоторой функции не соответствует такой полином --- противоречие.
\end{proof}

\begin{definition}
Булева функция называется линейной, если в её полиноме Жегалкина не используется $\wedge$.
\end{definition}

\begin{remark}
От $n$ переменных существует $2^{n+1}$ линейных функций.
\end{remark}

Для $3$ аргументов:

%! надо скопипаститб с фотки
% \begin{tabular}
%     &0 & 1 & P_1 \\
% x & 0 & 0 & 1\\
% y & 0 & 0 & 0\\
% z & 0 & 1 & 0
% \end{tabular}

%! надо скопипаститб с фотки
$$F_0 \quad f(0,0\ldots,0)=0 \text{ --- всего } 2^{2^n-1}$$
$$F_1 \quad f(1,1\ldots,1)=1 \text{ --- всего } 2^{2^n-1}$$
$$F_l \quad f(x_1,x_2\ldots,x_n)=\oplus_{i\in\{1\ldots n\}} x_i \text{ --- всего } 2^{n+1}$$
$$F_s \quad f(\neg x_1\ldots \neg x_n)=\neg f(x_1\ldots x_n) \text{ --- всего } 2^{2^n-1}$$
$$F_m \quad x_1\ldots x_n, y_1\ldots y_n, x_i\leq y_i \Rightarrow f(x_1,x_2\ldots,x_n)
\leq f(y_1,y_2\ldots,y_n)$$

\begin{definition}
Эти пять множеств функций называются классы Пирса
\end{definition}

\begin{lemma}
Классы Поста замкнуты относительно композиции
\end{lemma}

\begin{lemma}
Для любого класса Поста существует функция, не принадлежащая этому классу
\end{lemma}

\begin{proof}
$\Uparrow$ - стрелка Пирса не принадлежит ни одному классу Поста.
\end{proof}

\begin{definition}
Замыкание $\overline{F}=\{g\}$, где $g$ можно записать формулой в системе связок $F$.
\end{definition}

\begin{definition}
F --- базис, если замыкание на нем - все булевы функции.
\end{definition}

\begin{theorem}
Множество функций $F$ является базисом базис тогда и только тогда, когда в этом классе
содержатся функции всех пяти классов Поста. Другой способ записи: $F$ - базис $\Leftrightarrow \forall
i\in\{0,1,s,m,l\} \ F\not\subset F_i$
\end{theorem}

\begin{proof}
Докажем ``$\Rightarrow$''. $F\subset F_i \Rightarrow^{L1} \overline{F}\subset F_i\Rightarrow^{L2}
\downarrow\not =\overline{F} \rightarrow F$ --- не базис.

Докажем в другую сторону.

Здесь $f_0, f_1, f_l, f_m, f_s \quad f_i\not =F_i$, то есть рассматриваются функции,
не лежащие в соответствующих классах Поста.

Рассмотрим $f_0(0,0,\ldots 0)$.
\begin{enumerate}
\item $f_0(1,1,\ldots,1)=0$

$f_0(x,x,\ldots,x)=\neg x$
\item $f_0(1,1,\ldots,1)=1$


$f_0(x,x,\ldots,x)=1$
\end{enumerate}

Если сделать то же самое для $f_2$, то получим $\neg$ и $0$.

Выпишем все возможные аргументы для $f_m$:

$$
\begin{cases}
    x_1\ x_2\ x_3\ \ldots x_n \quad f_m(X)=1 \\
    y_1\ x_2\ x_3\ \ldots x_n \quad f_m(X)=1 \\
    y_1\ y_2\ x_3\ \ldots x_n \quad f_m(X)=1 \\
    \vdots \\
    y_1\ y_2\ y_3\ \ldots y_n \quad f_m(X)=0 \\
\end{cases}
$$

Заметим, что для некоторого $i$-того набора переменных $f_m(X_i)=0$, а $f_m(X_{i-1})=1$.
$f_m(x_1,x_2,\ldots,x_{i-1},0,x_{i+1}, \ldots x_n)=1 \\$
$f_m(x_1,x_2,\ldots,x_{i-1},1,x_{i+1}, \ldots x_n)=0$. Зафиксируем такие $x$. Тогда
$f_m(x_1,x_2,\ldots,x_{i-1},x,x_{i+1}, \ldots x_n)=\neg x$

Итого, мы получили $\neg x$ из $f_m$

%! Дописать доказательство
Рассмотрим $f_s$. $\exists x_1,x_,2\ldots x_n f_s(x_1\ldots x_n)=f_s(\neg x_1, \ldots x_n)$.
С помощью этого каким-то образом получается одна из констант. Другая константа получается отрицанием.

Рассмотрим $f_l=x\wedge y\wedge z_3\wedge z_4\wedge \ldots \wedge z_k \oplus\ldots\oplus
x\oplus\ldots\oplus z_i\oplus\ldots\oplus u_j$. Мы выбрали нелинейный член с наименьшим
числом элементов, его элементы обозначили за $x,y,z_3,z_4,\ldots,z_k$. Не встречающиеся
в этом члене переменные обозначили за $u_1,u_2,\ldots,u_j$. Если подставить вместо $z_i$ 1,
вместо $u_j$ 0, то $f(x,y,1,\ldots,1,0\ldots,0)=x\wedge y [\oplus x][\oplus y][\oplus 1]=g(x,y)$.
Члены с $u_j$ обратились в ноль.

Если в $g$ есть $\oplus 1$, то от него можно избавиться, взяв $\neg g$. Если есть 
$\oplus x$ (или $y$), то берем $g(\not x, y)$ или наоборот. $x\wedge y\oplus x\oplus y=x\vee y$

\end{proof}

\begin{tabular}{lccccc}
    &сохр. 0 & сохр. 1 & мон. & сам. & лин. \\
    $\wedge$&.&.&.&нет&нет\\
    $\vee$&.&.&.&нет&нет\\
    $\neg$&нет&нет&нет&.&.\\
    \hline \\
    $\wedge$&.&.&.&нет&нет \\
    $\oplus$&.&нет&нет&нет&. \\
    $1$&нет&.&.&нет&. \\
    \hline \\
    $\rightarrow$&нет&.&нет&нет&нет \\
    $0$&.&нет&.&нет&.
\end{tabular}
        
\end{document}