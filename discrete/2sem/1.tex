\documentclass[12pt, a4paper]{article}

\usepackage{lastpage}
\usepackage{mathtools}
\usepackage{xltxtra}
\usepackage{libertine}
\usepackage{amsmath}
\usepackage{amsthm}
\usepackage{amsfonts}
\usepackage{amssymb}
\usepackage{enumitem}
\usepackage{xcolor}
\usepackage[left=1.5cm, right=1.5cm, top=2cm, bottom=2cm, bindingoffset=0cm, headheight=15pt]{geometry}
\usepackage{fancyhdr}
\usepackage[russian]{babel}
% \usepackage[utf8]{inputenc}
\usepackage{catchfilebetweentags}
\usepackage{accents}
\usepackage{calc}
\usepackage{etoolbox}
\usepackage{mathrsfs}
\usepackage{wrapfig}

\providetoggle{useproofs}
\settoggle{useproofs}{false}

\pagestyle{fancy}
\lfoot{M3137y2019}
\rhead{\thepage\ из \pageref{LastPage}}

\newcommand{\R}{\mathbb{R}}
\newcommand{\Q}{\mathbb{Q}}
\newcommand{\C}{\mathbb{C}}
\newcommand{\Z}{\mathbb{Z}}
\newcommand{\B}{\mathbb{B}}
\newcommand{\N}{\mathbb{N}}

\newcommand{\const}{\text{const}}

\newcommand{\teormin}{\textcolor{red}{!}\ }

\DeclareMathOperator*{\xor}{\oplus}
\DeclareMathOperator*{\equ}{\sim}
\DeclareMathOperator{\Ln}{\text{Ln}}
\DeclareMathOperator{\sign}{\text{sign}}
\DeclareMathOperator{\Sym}{\text{Sym}}
\DeclareMathOperator{\Asym}{\text{Asym}}
% \DeclareMathOperator{\sh}{\text{sh}}
% \DeclareMathOperator{\tg}{\text{tg}}
% \DeclareMathOperator{\arctg}{\text{arctg}}
% \DeclareMathOperator{\ch}{\text{ch}}

\DeclarePairedDelimiter{\ceil}{\lceil}{\rceil}
\DeclarePairedDelimiter{\abs}{\left\lvert}{\right\rvert}

\setmainfont{Linux Libertine}

\theoremstyle{plain}
\newtheorem{axiom}{Аксиома}
\newtheorem{lemma}{Лемма}

\theoremstyle{remark}
\newtheorem*{remark}{Примечание}
\newtheorem*{exercise}{Упражнение}
\newtheorem*{consequence}{Следствие}
\newtheorem*{example}{Пример}
\newtheorem*{observation}{Наблюдение}

\theoremstyle{definition}
\newtheorem{theorem}{Теорема}
\newtheorem*{definition}{Определение}
\newtheorem*{obozn}{Обозначение}

\setlength{\parindent}{0pt}

\newcommand{\dbltilde}[1]{\accentset{\approx}{#1}}
\newcommand{\intt}{\int\!}

% magical thing that fixes paragraphs
\makeatletter
\patchcmd{\CatchFBT@Fin@l}{\endlinechar\m@ne}{}
  {}{\typeout{Unsuccessful patch!}}
\makeatother

\newcommand{\get}[2]{
    \ExecuteMetaData[#1]{#2}
}

\newcommand{\getproof}[2]{
    \iftoggle{useproofs}{\ExecuteMetaData[#1]{#2proof}}{}
}

\newcommand{\getwithproof}[2]{
    \get{#1}{#2}
    \getproof{#1}{#2}
}

\newcommand{\import}[3]{
    \subsection{#1}
    \getwithproof{#2}{#3}
}

\newcommand{\given}[1]{
    Дано выше. (\ref{#1}, стр. \pageref{#1})
}

\renewcommand{\ker}{\text{Ker }}
\newcommand{\im}{\text{Im }}
\newcommand{\grad}{\text{grad}}

\lhead{Дискретная математика}
\cfoot{}

\begin{document}

\section{Дискретная теория вероятности}

\begin{definition}
    \textbf{Множетсво элементарных исходов} обозначается $\Omega$. Это множество не более чем счётное в дискретной теории вероятности
\end{definition}
\begin{definition}
    \textbf{Дискретная вероятностная мера \textit{(дискретная плотность вероятности)}} - отображение $p: \Omega \to \R^+; \omega\mapsto$ вероятность исхода $\omega$. $$\sum\limits_{\omega\in\Omega} p(\omega) = 1$$
\end{definition}
\begin{example}
    Честная монета
    $$\Omega = \{0, 1\} \quad p(0) = p(1) = \frac{1}{2} \quad p(\text{\O})=0 \quad p(\{0, 1\})=1 \text{ --- достоверное событие}$$
\end{example}
\begin{example}
    Нечестная монета \textit{(распределение Бернулли)}
    $$\Omega = \{0, 1\} \quad p(0) = p \quad p(1) = q \quad p + q = 1$$
\end{example}
\begin{example}
    Честная игральная кость
    $$\Omega = \{1, 2, 3, 4, 5, 6\} \quad p(1) = p(2) = \ldots = p(6) = \frac{1}{6}$$
    $$Even := \{2, 4, 6\} \quad Big := \{4, 5, 6\}$$
    $$P(Even) = \frac{1}{2} \quad P(Big) = \frac{1}{2}$$
    $$VeryBig := \{5, 6\} \quad P(VeryBig) = \frac{1}{3}$$
\end{example}
\begin{example}
    Честная колода карт
    $$\Omega = \{(r, s)\ |\ r=1\ldots13, s=1\ldots4\} \quad p((r, s))=\frac{1}{52}$$
\end{example}
\begin{example}
    Честная монета с ребром
    $$\Omega = \{0, 1, \perp\} \quad p(0) = p(1) = \frac{1}{2} \quad p(\perp) = 0$$
\end{example}
\begin{example}
    Очень нечестная монета
    $$p = 1 \quad q = 0$$
\end{example}
\begin{definition}
    \textbf{Событие} --- множество элементарных исходов
    $$A\subset\Omega$$
\end{definition}
$$P:2^\Omega \to \R^+ \quad P(A) = \sum\limits_{\omega \in A}: p(\omega)$$
\begin{definition}
    $A$ и $B$ --- \textbf{независимые}, если вероятность их пересечения равна произведению их вероятностей:
    $$P(A\cap B)=P(A)\cdot P(B)$$
\end{definition}
$$P(Even \cap Big) = P(\{4, 6\}) = \frac{1}{3}$$
$$P(Even) \cdot P(Big) = \frac{1}{2} \cdot \frac{1}{2} = \frac{1}{4} \Rightarrow Even \text{ и } Big \text{ не независимы}$$
$$P(Even \cap VeryByg) = P(\{6\}) = \frac{1}{6}$$
$$P(Even) \cdot P(VeryBig) = \frac{1}{2} \cdot \frac{1}{3} = \frac{1}{6} \Rightarrow Even \text{ и } VeryBig \text{ независимы}$$
$$\sphericalangle \Omega_1, p_1, \Omega_2, p_2 \quad \Omega := \Omega_1 \times \Omega_2 \quad P(\omega)=p_1(\omega_1)p_2(\omega_2)$$
\begin{definition}
    События $A_1\ldots A_n$ --- \textbf{независимые в совокупности}, если
    $$\forall I\subset\{1\ldots n\} \quad P\left(\bigcap\limits_{i\in I} A_i\right)=\prod\limits_{i\in I}P(A_i)$$
\end{definition}
\begin{definition}
    События $A_1\ldots A_n$ --- \textbf{независимые попарно}, если
    $$\forall i, j\in\{1\ldots n\} \quad P(A_i\cap A_j)=P(A_i)P(A_j)$$
\end{definition}
\begin{definition}
    \textbf{Условная вероятность} --- вероятность того, что произойдет $A$, если произошло $B$:
    $$P(A|B)=\frac{P(A\cap B)}{P(B)}$$
\end{definition}
$$P(Big|Even) = \frac{P(Big \cap Even)}{P(Even)} = \frac{\frac{1}{3}}{\frac{1}{2}} = \frac{2}{3}$$
$] A$ и $B$ --- независимые
$$P(A|B) = \frac{P(A \cap B)}{P(B)} = \frac{P(A)\cdot P(B)}{P(B)} = P(A)$$
$\sphericalangle A_1, A_2, \ldots A_k$ --- разбиение : $\bigcap\limits_{i=1}^k A_i=\Omega \quad A_i\cap A_j=\text{\O}$
\begin{theorem}
    Формула полной вероятности:
    $$P(B)=\sum\limits_{i=1}^k P(B|A_i)\cdot P(A_i)$$
\end{theorem}
\begin{proof}
    $$P(B)=\sum\limits_{i=1}^k P(B\cap A_i)=\sum\limits_{i=1}^k P(B|A_i)\cdot P(A_i)$$
\end{proof}
\begin{theorem}
    Формула Байеса:
    $$P(A_j|B)=\frac{P(A_j\cap B)}{P(B)}=\frac{P(B|A_j)P(A_j)}{\sum\limits_{i=1}^k P(B|A_i)P(A_i)}$$
\end{theorem}
% [============================================================================]
% |Формула Байеыса:                                                            |
% |P(Aj|B) = P(Cross(Aj,B))/P(B) = P(B|Aj)*P(Aj)/Sum(i = 1 to K): P(B|Aj)*P(Ai)|
% [============================================================================]

\end{document}