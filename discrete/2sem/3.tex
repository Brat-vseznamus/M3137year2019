\documentclass[12pt, a4paper]{article}

\usepackage{lastpage}
\usepackage{mathtools}
\usepackage{xltxtra}
\usepackage{libertine}
\usepackage{amsmath}
\usepackage{amsthm}
\usepackage{amsfonts}
\usepackage{amssymb}
\usepackage{enumitem}
\usepackage{xcolor}
\usepackage[left=1.5cm, right=1.5cm, top=2cm, bottom=2cm, bindingoffset=0cm, headheight=15pt]{geometry}
\usepackage{fancyhdr}
\usepackage[russian]{babel}
% \usepackage[utf8]{inputenc}
\usepackage{catchfilebetweentags}
\usepackage{accents}
\usepackage{calc}
\usepackage{etoolbox}
\usepackage{mathrsfs}
\usepackage{wrapfig}

\providetoggle{useproofs}
\settoggle{useproofs}{false}

\pagestyle{fancy}
\lfoot{M3137y2019}
\rhead{\thepage\ из \pageref{LastPage}}

\newcommand{\R}{\mathbb{R}}
\newcommand{\Q}{\mathbb{Q}}
\newcommand{\C}{\mathbb{C}}
\newcommand{\Z}{\mathbb{Z}}
\newcommand{\B}{\mathbb{B}}
\newcommand{\N}{\mathbb{N}}

\newcommand{\const}{\text{const}}

\newcommand{\teormin}{\textcolor{red}{!}\ }

\DeclareMathOperator*{\xor}{\oplus}
\DeclareMathOperator*{\equ}{\sim}
\DeclareMathOperator{\Ln}{\text{Ln}}
\DeclareMathOperator{\sign}{\text{sign}}
\DeclareMathOperator{\Sym}{\text{Sym}}
\DeclareMathOperator{\Asym}{\text{Asym}}
% \DeclareMathOperator{\sh}{\text{sh}}
% \DeclareMathOperator{\tg}{\text{tg}}
% \DeclareMathOperator{\arctg}{\text{arctg}}
% \DeclareMathOperator{\ch}{\text{ch}}

\DeclarePairedDelimiter{\ceil}{\lceil}{\rceil}
\DeclarePairedDelimiter{\abs}{\left\lvert}{\right\rvert}

\setmainfont{Linux Libertine}

\theoremstyle{plain}
\newtheorem{axiom}{Аксиома}
\newtheorem{lemma}{Лемма}

\theoremstyle{remark}
\newtheorem*{remark}{Примечание}
\newtheorem*{exercise}{Упражнение}
\newtheorem*{consequence}{Следствие}
\newtheorem*{example}{Пример}
\newtheorem*{observation}{Наблюдение}

\theoremstyle{definition}
\newtheorem{theorem}{Теорема}
\newtheorem*{definition}{Определение}
\newtheorem*{obozn}{Обозначение}

\setlength{\parindent}{0pt}

\newcommand{\dbltilde}[1]{\accentset{\approx}{#1}}
\newcommand{\intt}{\int\!}

% magical thing that fixes paragraphs
\makeatletter
\patchcmd{\CatchFBT@Fin@l}{\endlinechar\m@ne}{}
  {}{\typeout{Unsuccessful patch!}}
\makeatother

\newcommand{\get}[2]{
    \ExecuteMetaData[#1]{#2}
}

\newcommand{\getproof}[2]{
    \iftoggle{useproofs}{\ExecuteMetaData[#1]{#2proof}}{}
}

\newcommand{\getwithproof}[2]{
    \get{#1}{#2}
    \getproof{#1}{#2}
}

\newcommand{\import}[3]{
    \subsection{#1}
    \getwithproof{#2}{#3}
}

\newcommand{\given}[1]{
    Дано выше. (\ref{#1}, стр. \pageref{#1})
}

\renewcommand{\ker}{\text{Ker }}
\newcommand{\im}{\text{Im }}
\newcommand{\grad}{\text{grad}}

\lhead{Дискретная математика}
\cfoot{}
\rfoot{Лекция 3}

\begin{document}

Рассмотрим случай, когда $\xi$ принимает значения $1,2\ldots$
$$\sum\limits_{a} a P(\xi=a) = \sum\limits_{i=1}^\infty i P(\xi=i)=\sum\limits_{i=1}^\infty \sum\limits_{j=1}^i P(\xi=i)=\sum\limits_{j=1}^\infty \sum\limits_{i=j}^\infty P(\xi=i)= \sum\limits_{j=1}^\infty P(\xi\geq j)$$

Кидаем честную монету, пока не выпадет 1, $\xi$ --- число бросков.

$$P(\xi\geq i) = \frac{1}{2^{i-1}} \quad \sum\limits_{j=1}^\infty \frac{1}{2^{j-1}}=2$$

$$Corr(\xi, \eta) = \frac{Cov(\xi, \eta)}{\sqrt{D\xi D\eta}}$$
$\sphericalangle \eta=c\xi$
$$Corr(c\xi, \xi) = \frac{Ec\xi\xi - cE\xi E\xi}{\sqrt{c^2D\xi D\xi}}=\begin{cases}
    1, & c > 0 \\
    -1, & c < 0
\end{cases}$$

\section{Хвостовые неравенства}

\begin{definition}
    \textbf{Хвост распределения} --- левая или правая маловероятная часть распределения
\end{definition}

\subsection{Неравенство Маркова}

$\sphericalangle \xi\geq 0$

Если возможных значений бесконечно, то их вероятности должны стремиться к 0, т.к. сумма всех вероятностей 1.

$\sphericalangle kE\xi$. Отрежем ``хвост'' справа от $ka$.
$$P(\xi\geq ka)\leq\frac{1}{k}$$
$$E\xi=\sum\limits_a a\cdot P(\xi=a) = \sum\limits_{a<kE\xi} a\cdot P(\xi=a) + \sum\limits_{a\geq kE\xi}a\cdot P(\xi=a) \geq \sum\limits_{a\geq kE\xi} kE\xi\cdot P(\xi=a)=kE\xi P(\xi\geq kE\xi)$$
$$1\geq kP(\xi\geq kE\xi)$$

\subsection{Неравенство Чебышева}

$\sphericalangle \xi, \eta=(\xi-E\xi)^2$

$$E\eta=D\xi$$
$$P(\eta\geq kE\eta) \leq \frac{1}{k}$$
$k'=\frac{k}{\sqrt{D\xi}}$
$$P((\xi-E\xi)^2\geq k^2D\xi)\leq\frac{1}{k^2}$$
$$P(|\xi-E\xi|\geq k\sqrt{D\xi})\leq\frac{1}{k^2}$$
$\sqrt{D\xi}=:\sigma$

Лучше хвостовые неравенства не придумать, т.к. есть случайные величины, при которых эти неравенства обращаются в равенста. Но можно накладывать дополнительные ограничения на величины и получать более ``быстрые'' неравенства.

Допустим, у нас есть нечестная монета и надо найти $p$. Кинем монету $n$ раз и оценим нашу уверенность в ответе.

$C_n^ip^iq^{n-i} \quad E=pn \quad D = pqn$

$$P\left(|\xi - E\xi| \geq k\sqrt{D\xi}\right)\leq \frac{1}{k^2}$$
$$k\sqrt{pqn}=\left(p-\frac{1}{2}\right)n$$
$$k=\frac{p-\frac{1}{2}}{\sqrt{pq}}\sqrt{n}$$

Рассмотрим следующую случайную величину: $\xi=\sum\xi_i \quad \xi_i$ --- незав., од. распр.
$\eta:=e^{t\xi}, t$ --- параметр
$$P(\eta\geq k)\leq\frac{E\eta}{k} \quad k=e^{ta}$$
$$P(\eta\geq e^{ta})\leq\frac{E\eta}{e^{ta}}$$
$$P(e^{t\xi}\geq e^{ta})\leq\frac{E\eta}{e^{ta}}$$
$]t>0$
$$P(\xi\geq a)\leq \frac{Ee^{t\sum \xi_i}}{e^{ta}}=\frac{E\prod\limits_i e^{t\xi_i}}{e^{ta}}\stackrel{\text{независ.}}{=}\frac{\prod\limits_i Ee^{t\xi_i}}{e^{ta}}=\frac{\left(Ee^{t\xi_i}\right)^n}{e^{ta}}$$
$$P(\xi\geq a)\leq \min\limits_{t>0}\frac{\left(Ee^{t\xi_i}\right)^n}{e^{ta}}$$

$\sphericalangle \xi_i=\begin{cases}
    1, & p \\
    0, & q = 1 - p
\end{cases}$

$\xi_i=1\Rightarrow e^{t\xi_i}=e^t \quad \xi_i=0 \Rightarrow e^{t\xi_i}=1$
$$E(e^{t\xi_i})=e^tp+q=e^p+1-p=p(e^t-1)+1$$
$$P(\xi\geq a)\leq \min\limits_{t>0} \frac{(p(e^t-1)+1)^n}{e^{ta}}$$
$a:=(1+\delta)pn$
$$P(\xi\geq (1+\delta)pn) \leq \frac{(p(e^t-1)+1)^n}{e^{(1+\delta)pnt}}\leq$$
$1+x\leq e^x$
$$\leq\frac{e^{p(e^t-1)n}}{e^{(1+\delta)pnt}}\leq e^{pn(e^t-1-(1+\delta)t)}=$$
$t:=\ln(1+\delta)$
$$=e^{pn(1+\delta-1-(1+\delta)\ln(1+\delta))}\leq$$
$\ln(1+x)\leq x$
$$\leq e^{pn(\delta-(1+\delta)\delta)}\leq e^{-pn\frac{\delta^2}{2}}$$
\end{document}